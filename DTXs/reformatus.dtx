;Református Énekeskönyv (régi fajta)
;http://reformatus-enekeskonyv.foruma.hu/
;kiegészítve és javítva Győrödi Dániel Gábortól kapott szöveg alapján
;
;2013/12/24 átmásolás RJ
;2019/05/26 ellenőrzés és kiegészítés Rieth Kati
NReformátus Énekeskönyv
RRef
CEgyházak

;Kétféle életút
;Bourgeois L., Strasbourg, 1539
;
>1
/1
#E46D75E7
 Aki nem jár hitlenek tanácsán,
 És meg nem áll a bűnösök útán,
 A csúfolóknak nem ül ő székében,
 De gyönyörködik az Úr törvényében,
 És arra gondja mind éjjel, nappal:
 Ez ily ember nagy boldog bizonnyal.
/2
#59EA3962
 Mert ő olyan, mint a jó termőfa,
 Mely a víz mellett vagyon plántálva,
 Ő idejében meghozza gyümölcsét,
 És el nem szokta hullatni levelét;
 Ekképpen amit ez ember végez,
 Minden dolgában megyen jó véghez.
/3
#1566C13D
 De nem ígyen vannak a gonoszak;
 Hanem mint az apró por és polyvák,
 Melyek a széltől széjjelragadtatnak:
 Így az ítéletben meg nem állhatnak
 A gonoszok és kik bűnben élnek,
 Az igazak közt helyet nem lelnek.
/4
#020E2EF6
 Mert az Isten ismeri útukat,
 Az igazaknak érti dolgukat;
 Azért mindörökké ők megmaradnak;
 De akik csak a gonoszságban járnak,
 Azoknak nyilván mind elvész útuk:
 Mert Istennek nem kell az ő dolguk.

;Isten Felkentjének diadalmas uralkodása
;Bourgeois L., Genf, 1542
;
>2
/1
#199F4415
 Miért zúgolódnak a pogányok?
 Mit forgatnak ő bolond elméjükben?
 A földi népeknek mi szándékok?
 Csak hiába valót űznek szívükben.
 E világi királyok egybegyűlnek,
 A fejedelmek tanácsot tartnak,
 Az Isten ellen erős kötést tesznek,
 És az ő Krisztusára támadnak.
/2
#008F8196
 Nagy fennen mondják: mit késünk ezzel?
 Jer, szaggasuk el ezeknek kötelét,
 És minden igájukat rontsuk el,
 Ne viseljünk többé rajtunk effélét!
 De az Úr Isten a magas mennyekben
 Csak neveti dolgukat azoknak,
 Csúfolja őket, ülvén szent székében,
 Kinek ezek semmit nem árthatnak.
/3
#9802F978
 És végre hogy megindúl haragja,
 Kemény beszéddel kezd szólni azoknak,
 Rettenetesképen megriasztja,
 Mely miatt teljességgel elbágyadnak:
 Hogy mertek, ugymond, ez ellen zúgódni,
 Akit én kentem a királyságra?
 És királyi pálcát én adtam néki,
 Az én szent hegyemre, a Sionra.
/4
#D4134E96
 Mond e királyság: im kihirdetem
 Az ő tanácsát és szent dekrétomát:
 A felséges Úr mondá énnekem:
 Én fiam vagy, ma szültelek, fiamat.
 Kérjed éntőlem a sok pogányokat,
 Kiket örökségül neked adok;
 Föld kerekségén sok tartományokat
 Mindenütt én teneked ajánlok.
/5
#3D557D07
 Rontsd meg őket te nagy hatalmaddal,
 Mint a földből égetett cserépedényt,
 És a te királyi vaspálcáddal
 Verd össze és törd apró pozdorjánként!
 Azért fejedelmek, bírák, királyok
 Ezekből tanulságot vegyetek,
 És magatokat jól meggondoljátok,
 Kik e földön regnáltok, ítéltek!
/6
#5392B94C
 Szolgáljatok e hatalmas Úrnak
 Jámbor élettel, igaz félelemmel,
 Örvendezzetek ő nagy voltának,
 De ezek is légyenek rettegéssel:
 Csókoljátok e néktek küldött fiat,
 Hogy erősen meg ne haragudjék;
 El ne mulasszátok parancsolatját,
 Mert szörnyűképpen el kell vesznetek!

;Reggeli ének nehéz időkben
;Bourgeois L., Genf, 1551
>3
/1
#A645E095
 Ó, mely sokan vannak,
 Akik háborgatnak
 Engemet, én Istenem!
 Nagy sok ellenségim
 És sok gyűlölőim
 Tusakodnak ellenem.
 Sokan azt állítják,
 Lelkemről azt mondják:
 Elveszett ennek dolga,
 Segítsége nincsen,
 Mert elhagyta Isten,
 Így szólnak bolond módra.
/2
#8D0F6809
 Mert te, én Istenem,
 Paizsom vagy nekem,
 Ki életem megmented,
 És nagy tisztességre,
 Fejem dicsőségre
 Idővel fölemeled.
 Tehozzád, Úr Isten,
 Kiáltok szüntelen,
 És te megvidámítasz;
 Meghallgatsz kedvedből,
 Sion szent hegyéről
 Nagy segedelmet nyújtasz.
/3
#059B05B6
 Ha ágyamban nyugszom,
 Csendesen aluszom,
 Nincsen semmi félelmem.
 Midőn felserkenek,
 Semmit sem kesergek,
 Mert Isten őriz engem.
 Ha százezer népek
 Mind körülvennének
 Jobb és bal kezem felől,
 Ha rám ütnének is,
 Nem rettegnék mégis
 Semmi veszedelemtől.
/4
#A0208B9A
 Kelj föl, Uram, tarts meg,
 Ellenségim vond meg,
 Megtörvén ő fogokat!
 Mind összepaskolod
 És arcul csapdosod
 Az Isten-utálókat.
 Csak te vagy az Isten,
 Ki minden szükségben
 Meg tudsz szabadítani,
 Ki a te sereged
 Megtartod, szereted,
 És meg szoktad áldani.

;Esti ének nehéz időkben
;Bourgeois L., Genf, 1542
>4
/1
#8CD79F89
 Én igazságomnak Istene,
 Hallgasd meg én kiáltásom!
 Szánj meg és tekints ínségemre,
 Te vigy engemet tágas helyre,
 Midőn itt szorongattatom!
 Ti nagy urak, míglen gyaláztok
 Engemet tisztességemben?
 És ily hívságban míg maradtok?
 Hazudozásra mit vágyódtok?
 Mit gyönyörködtök ezekben?
/2
#61600459
 De vegyétek jól eszetekbe,
 Hogy az Úr Isten engemet
 Bévett kedvébe, kegyelmébe,
 Csuda mód megmentett engemet,
 És meghallja kérésemet!
 Rettegjetek hát és lássátok,
 Hogy ellene ne vétsetek;
 Magatokat meggondoljátok,
 Az ágyasházban ha nyugosztok,
 Hogy lehessen csendességtek!
/3
#9CEFA22E
 Az igazak áldozatjával
 Áldozzatok az Istennek;
 Jó élettel és igazsággal,
 És az Istenben bátorsággal
 Bízzatok és örvendjetek!
 De sokan mondják azt minékünk:
 Ki vezérl minket a jóra?
 Azért téged, Úr Isten, kérünk,
 Mutasd kegyes orcádat nékünk,
 Jöjjön el az áldott óra!
/4
#758BE624
 Mellyel inkább vigasztalsz engem
 És örvendeztetsz szívemben,
 Hogynem kiknek sok mustjok terem,
 Búzájokkal rakva sok verem,
 Mikkel élnek bőségesen.
 Azért élek jó békességben,
 Fekszem, aluszom kedvemre,
 Uralkodván lakom földemben
 Bátorságos örvendezésben,
 Mert az Úr vigyáz éltemre.

;Reggeli könyörgés az esztelenek ellen
;Bourgeois L., Genf, 1542
>5
/1
#9A0D8DFA
 Úr Isten, az én imádságom,
 Kérlek, vegyed füleidbe
 És hallgass meg kérésembe'!
 Én Istenem és én királyom,
 Értsd meg mondásom.
/2
#CFB6A56D
 Tekintsed meg esedezésem,
 És halld meg kiáltásomat,
 Midőn hívlak, királyomat!
 Meghallgatod én könyörgésem,
 Bizonnyal hiszem.
/3
#AF2AE82B
 Jó reggel meghallgatsz engemet,
 Uram, még virradta előtt,
 Idején a nap hogy feljött;
 Elődbe számlálom ügyemet,
 Várván kegyelmet.
/4
#905188AC
 Mert egyedül te vagy oly Isten,
 Kinek a gonoszság nem kell;
 És aki gonosz bűnben él,
 Nem mehet hozzád semmiképpen,
 Míg él vétekben.
/5
#9648439A
 A kábák és az esztelenek
 Nem állhatnak színed előtt,
 Gyűlölsz minden gonosztevőt,
 És tőled távol űzettetnek Az ilyetének.
/6
#5171A4F0
 Akik csak hazugságot szólnak,
 A gyilkosokat, orvokat
 Szörnyen elveszted azokat;
 Kik embert hamisan megcsalnak,
 Megútáltatnak.
/7
#2FA16324
 Én pedig nagy jó reménységgel
 Bémegyek szent templomodba,
 És imádlak szent házadba;
 Nagy jóvoltodért félelemmel
 Szolgállak szívvel.
/8
#FDF47FAB
 Uram, vezérelj igazságban
 Ellenségimnek láttokra,
 Kik igyekeznek káromra;
 Oktass, hogy a te útaidban
 Járhassak jobban.
/9
#89A0F045
 Mert szájukban nincs egy igaz szó,
 Szívük teljes nyavalyával,
 Ő nyelvük szól álnoksággal;
 Torkuk oly, mint a nyílt koporsó,
 Hol nincs semmi jó.
/10
#392CDA45
 Büntesd meg, Uram Isten, őket,
 Tedd semmivé tanácsukat,
 Rontsd el ő találmányukat;
 Kik ellenkeznek, verd meg őket,
 A pártütőket!
/11
#EC465603
 És hogy azok mind örüljenek,
 Akik bíznak csak tebenned,
 Szívből szeretik szent neved:
 Engedd, hogy vígan
 Felségednek Énekeljenek.
/12
#9F5FD1C1
 Az igazat mert te megáldod,
 Te nagy irgalmasságoddal
 Körülveszed, mint paizzsal;
 A gonosz ellen őt megtartod
 És oltalmazod.

;Lelki-testi nyomorúságban (Első bűnbánati zsoltár)
;Bourgeois L., Genf, 1542
>6
/1
#460EB1DF
 Uram, te nagy haragodban,
 Mely miatt vagyok búban,
 Engemet ne feddj meg!
 És haragodnak tüze,
 Szűnjék meg sebessége,
 Melyben ne büntess meg!
/2
#6D93DC3F
 Nékem, Uram, légy irgalmas,
 Mert vagyok nagy fájdalmas.
 Ne hagyj, Uram, kérlek!
 Gyógyítsd meg sérelmimet,
 Elrettent tetemimet
 Újítsd meg, hogy éljek!
/4
#6B423975
 Térj, Uram, kegyesen hozzám,
 Mert, jaj, elfogyatkozám!
 Te nagy irgalmadból
 Szánj meg nagy nyavalyámban
 És keserves kínomban:
 Ments meg a haláltól!
/6
#507A098A
 Elfáradtam bánatimban,
 Egész éjjel sírtomban;
 Könnyhullatásaimmal
 Nedvesítem ágyamat,
 Áztatom nyoszolyámat
 Sűrű siralmimmal.
/7
#6A89F366
 Én szomorúságim miatt
 Én két szemem elsorvadt
 És elhomályosult;
 Ezt szerzik ellenségim,
 Vigadnak gyűlölőim,
 Min szívem elbúsult.
/8
#D0BCEAE2
 Azért minden ellenségim
 És én háborgatóim,
 Pironkodjatok el!
 Már mind hátra térjetek,
 És megszégyenüljetek
 Nagy hirtelenséggel!
/9
#2A49A8C1
 Könyörgésem meghallgatja az
 Úr,és elfogadja
 Én imádságomat,
 És amit tőle kérek,
 Minden jókat megnyerek,
 Úgy kívánja jómat.
/10
#4505FAA9
 Azért minden ellenségim
 És én háborgatóim,
 Pironkodjatok el!
 Már mind hátra térjetek,
 És megszégyenüljetek
 Nagy hirtelenséggel!

;Könyörgés segedelemért
;Bourgeois L., Genf, 1551
>7
/1
#6D585D0F
 Ó, én Uram és én Istenem,
 Tebenned vagyon reménységem,
 Én oltalmamra légy jelen,
 Tarts meg ellenségem ellen,
 Hogy engemet el ne ragadjon,
 Mint éh oroszlán, meg ne rágjon,
 Amidőn nincs segítségem,
 Aki megmentene engem.
/2
#0EABCD8F
 Hogyha én ezt tettem, Úr Isten,
 Avagy hamisság van kezemben;
 Hogyha gonoszt tettem ennek,
 Ki örült békességemnek;
 Hogyha valaki abban megért,
 Hogy gonoszt fizettem a jóért;
 Sőt, ha jól nem tettem azzal,
 Ki nekem volt bosszúsággal:
/3
#5E2F35BE
 Ámbár kergessen ellenségem,
 És bátor megragadjon engem,
 Életemet földhöz verje,
 Dicsőségem porrá tegye!
 Kelj föl azért nagy haragodban,
 Ellenségem ellen támadván;
 Add meg az előbbi tisztem,
 Kit rendeltél, Uram, nekem!
/4
#DB8EB1D1
 Nagy sereggel aztán a népek
 Hozzád gyűlnek és körülvésznek,
 Azért, Úr Isten, támadj fel.
 Magas helyre köztök állj fel!
 És ottan ítéljed a népet,
 Ki jól rendelhetsz mindeneket,
 És engem hűségem szerint,
 Ítélj igazságod szerint!
/5
#B703B737
 Verd meg az istenteleneket,
 És védelmezd meg híveidet,
 Mert mindeneknek titkait,
 Látod, Uram, szívit-lelkit.
 Te vagy paizsom, igaz Isten,
 És nem hagysz el veszedelmemben,
 Ki a híveket megtartod,
 A gonosztól takargatod.
/9
#F65B750A
 Azért Istennek adok hálát,
 Hirdetem ő szent igazságát,
 És felséges szent nevének
 Dícséreteket éneklek.

;Estvéli ének a mennyboltozatról és az emberről
;Bourgeois L., Genf, 1542
>8
/1
#5B72D975
 Ó, felséges Úr, mi kegyes Istenünk,
 Mely csudálatos a te neved nékünk!
 Nagy dicsőséged ez egész földre
 Kiterjed és felhat az egekre.
/2
#A07A7813
 Dicsérnek téged még a csecsszopók is,
 Szájukban viselik nevedet ők is,
 Kik által ellenséget megejtesz,
 És bosszúállót megszégyenítesz.
/3
#9250860F
 Nagy voltát ha megnézem dolgaidnak,
 Melyeket a te kezeid formáltak,
 Az eget, holdat, a fényes napot,
 És szép renddel a sok csillagokot:
/4
#C068D327
 Csudálván mondom: micsoda az ember,
 Ki tőled ennyi sok dicsőséget nyer?
 De micsoda az embernek fia,
 Kiről Felségednek van ily gondja?
/5
#E0A9F152
 Az angyaloknál noha egy kevéssé
 Kisebbé tőd, de nagy dicsőségessé
 Teremtéd őtet és magasztalád,
 Nagy dicsőségre felkoronázád.
/6
#2D6482BB
 Kezed munkáin őtet úrrá tevéd,
 Hogy azokkal bírna, néki engedéd,
 Valamit e világra teremtél,
 Mindeneket lába alá vetél.
/7
#B0B32105
 Juhot, ökröt és egyéb állatokat,
 A hegyen és völgyön élő vadakat,
 Kik az erdőkön széjjel legelnek,
 Avagy sík réten, mezőn tengenek.
/8
#AE823B3D
 És a repeső égi madarakat,
 Kik hangicsálnak szép melódiákat,
 És sok halait a nagy tengernek
 Adád birtokába az embernek.
/9
#D18CE26B
 Ó, felséges Úr, mi kegyelmes Urunk,
 Mely csudálatos a te neved nálunk!
 Felségednek mely nagy dicsősége,
 Mellyel teljes e föld kereksége.

;Hálaének Isten ítéletéért
;Bourgeois L., Genf, 1542
>9
/1
#7318DBCD
 Dicsérlek téged, Úr Isten,
 És áldlak teljes szívemben,
 És a te csudatételidet,
 Hirdetem jótéteményidet.
/2
#61ED3AFA
 Tebenned, Uram, vigadok,
 Nagy örömömben tombolok,
 És a te felséges nevednek
 Szép dicséreteket éneklek.
/3
#DBD416AC
 Mert az én ellenségimet
 Veréd, megtérítéd őket,
 Kik rettegvén, hátra esének,
 Szent színed elől elveszének.
/4
#5F30681A
 Én ügyemet megtekintéd,
 És kegyelmesen felvevéd;
 Ülvén a törvénytevő székben,
 Megmentél igaz ítéletben.
/8
#AFCB9737
 Törvényed te szent mértéked:
 Előhívod a te néped,
 És igazsággal mindeneket,
 Megítélsz minden nemzeteket.
/9
#C4ECE08D
 Te a szegénynek oltalma,
 Őrzője vagy és jutalma,
 Aki szükségnek idejében
 Hozzád óhajt könyörgésében.
/10
#A82280B9
 Azért bíznak csak tebenned,
 Akik ismerik szent neved;
 Kik tőled várnak segítséget,
 Úr Isten, nem hagyod el őket.
/11
#4E95C2DA
 Énekeljetek az Úrnak,
 A Sion hegyén lakónak!
 Sokságát cselekedetinek
 Hirdessétek el minden népnek.
/12
#9B7C471C
 Aki nyilván megkeresi,
 Az igaz vért nem felejti,
 A szegény népet ő nem hagyja,
 Akiknek kiáltását hallja.
/13
#0E557835
 Én Uram és én Istenem,
 Tekintsed meg nagy ínségem:
 Életemet a gonosz gyötri;
 A halál kapuiból végy ki!
/14
#C16EF3F3
 Dícsérhesselek nagy vígan
 A Sionnak kapuiban;
 Hadd örvendezzek én szívemben,
 Hogy engem megtartál kegyesen.
/19
#1057CD5D
 Kelj fel, Uram, és légy jelen,
 Hogy ember erőt ne vegyen;
 A pogányokat hívd elődbe,
 Ítéld meg erős törvényedbe'!
/20
#56CE6240
 Szívökben, Uram, rettentsd meg,
 Hogy magukat gondolják meg,
 És ismerjék a pogány népek,
 Hogy ők is halandó emberek.

;Nyomorultak kiáltása segedelemért
;Bourgeois L., Genf, 1551
>10
/1
#5677CE58
 Mire távozol tőlünk, Úr Isten,
 Ily meszsze, és magad mit rejted el?
 E sok ideig való ínségben,
 Miért hogy minket így felejtesz el?
 Ím, az istentelen kevélységgel
 Kergeti a szegényt és nyomorgatja:
 Hálója essék az önnön nyakába!
/2
#2448C155
 Mert az istentelen dicsekedik,
 És gyönyörködik kívánságában.
 Dicséri a fösvényt, hízelkedik,
 Istent káromol felfuvalkodván,
 Akit megutál nagy hívságában,
 Sőt kevélységében így gondolkodik:
 Hogy nincsen Isten, azzal csúfolódik.
/3
#84B0E347
 Ő bolondságában úgy elmégyen,
 Erős ítéletedet nem féli.
 És kevélységében szól nagy fennen,
 Ellenségit is semminek véli.
 Könnyen elfúhatja, azt ítéli,
 Végre mond: bátran lakom és csendesen,
 Soha nem esem szerencsétlenségben.
/4
#92DD13F0
 Átokkal, szitokkal rakva szája,
 Szól az ő nyelve csak álnokságot,
 E nyelvet szoktatta csalárdságra,
 Mellyel szerez sok bút és bánatot.
 Tolvajok módján megáll barlangot,
 Leshelyből a szegényre ólálkodik,
 Hogy megfoghassa, csak azon forgódik.
/6
#A3C6AE4D
 Még azt is mondja az ő szívében,
 Hogy Isten ezzel semmit sem gondol;
 Orcáját elrejti ezek ellen,
 Szemét behúnyja, máshová fordul.
 Kérlek Uram, e dolgon megindulj,
 Nyújtsd ki kezed, a szegényt ne felejtsd el,
 A nyomorultnak légy jó segítséggel!
/7
#8ACEB0B6
 Miért mívelné a gonosz ember,
 Hogy káromolja s csúfolja Istent?
 Szívében mond: számot Isten nem kér.
 De te, Uram, értesz és látsz mindent.
 Azért a szegény csak reád tekint,
 A nyomorult, az árva benned bízik,
 És kegyesen tőled megsegíttetik.
/8
#06A5E1AE
 Törd meg az istentelennek karját,
 Hívd elődbe és idézd törvénybe,
 Vizsgáld és ítéld meg gonoszságát,
 Ekképen nem mer jönni elődbe.
 És így örökké regnál az Isten:
 A pogányokat földről mind eltörli,
 És igazsággal szent népét vezérli.
/9
#5C7424D7
 Uram, tekintsd meg a nyomorultat,
 A szelídeket kegyesen tartsd meg;
 Vedd füledbe ő kiáltásukat,
 Erősítsd szívüket, vigasztald meg.
 A szegény árvákat védelmezd meg,
 Zabolázd meg az erőszaktevőket:
 E földön óvd a gonoszoktól néped.

;Kishitűség ellen
;Bourgeois L., Genf, 1551
>11
/1
#A0B66407
 Az Istenben bízom jó reménységgel,
 Miért szóltok hát így én lelkemnek,
 Hogy én a ti hegyetekre fussak el,
 És mint egy madár messze repüljek?
 Mert a kegyetlenek ívüket szegzik,
 Hogy a tiszta szívűket meglőjék,
 Már a nyilakat az idegbe tették.
/2
#69CE734F
 Mert igaz ő, és igazságot szeret,
 És kedves orcát mutat azoknak,
 Kik igazságban rendelik éltüket.
 Az Istennek temploma mennyekben,
 Ott fenn vagyon szent széke, melyből széjjel
 Szemei mindent néznek élesen,
 Földi népet is megnézell és szemlél.

;Imádság szorongattatásban
;Bourgeois L., Genf, 1551
>12
/1
#26BDF7B9
 Szabadíts meg és tarts meg, Uram Isten,
 Mert szentid elfogytak, nincs jól tévő,
 És a földön már sok a tökéletlen,
 Nincs emberek közt igaz beszédű.
/2
#C17292D6
 Ezek egymásnak szólnak csak hívságot,
 Dolgukat festik szép beszédekkel;
 Hízelkedvén, mutatnak nyájasságot,
 De nyelvük nem egyez ő szívükkel.
/3
#20750816
 Csalárd ajkukat elveszti az Isten,
 Eltörli hízelkedő nyelvüket,
 Kik kevélységükben szólnak nagy fennen,
  És csúfolják az együgyűeket.
/4
#3D6A6D74
 És ezt mondják: jer, tegyük azt nyelvünkkel,
 Hogy minket minden nagyra becsüljön;
 Ajkunkkal, szánkkal elbírunk ezekkel,
 És senki sincs, ki velünk pöröljön.
/5
#714F4A5D
 Azért mond Isten: ímé, a szegények
 Elhagyatnak, öletnek, pusztulnak;
 Azért őket megszánom és felkelek,
 És kezükből kimentem azoknak.
/6
#750C2880
 Az Istennek mondási oly igazak,
 Mint a drága ezüst, kit a tűzben
 Az ötvösök kohókban tisztítottak,
 És hétszer megeresztettek szépen.
/7
#BA53A6E7
 Tartsd meg azért népedet kegyelmesen,
 Kérünk, jóvoltodból reánk tekints!
 Őrizz meg örökké e nemzet ellen,
 Hozzánk mindenkor jókedvet jelents!
/8
#6D6D22A5
 Mert a gonoszok nagy fennen forgódnak,
 Udvaroltatnak nagy kevélységben,
 Midőn a rosszak felmagasztaltatnak,
 És a jó ember nincs becsületben.

;Meddig még?
;Bourgeois L., Genf, 1542
>13
/1
#F166E448
 Míglen felejtesz el, Uram,
 Míg nem emlékezel rólam?
 A te orcádat énelőlem,
 Örökké elrejted e tőlem?
 Mire nem könyörülsz rajtam?
/2
#BBB688F2
 Míg tanácskozzam szívemben?
 Míg keseregjek elmémben,
 Még nappal is csak gondolkodván,
 Ellenségem reám míg rohan?
 Míg uralkodik fejemen?
/3
#8EBDE59F
 Tekints reám, kegyes Uram!
 Szánjad meg az én nyavalyám!
 Szemeimet világosítsd meg,
 Hogy életemben viduljak meg,
 Halálban el ne aludjam.
/4
#DA7397F9
 Hogy ellenségem ne mondja:
 Íme, erőt vettem rajta!
 És azon ők örvendezzenek,
 Ha engemet elejthetnének,
 És nem állhatnék lábamra.
/5
#80E6DA1A
 Mert én bízom jókedvedben,
 Ki megvigasztalsz szívemben.
 Örvendez azért az én lelkem,
 Hogy Isten az én segedelmem,
 Melyért dicsérem énekben.

;Isten erősebb, mint ellenségei
;Bourgeois L., Genf, 1542
>14
/1
#19BD7CE0
 A bolond így szól az ő szívében:
 'Nincs Isten', azért nagy gonoszságban él;
 Utálatos bűnt teszen, semmit nem fél.
 Ez egész földön aki jót tegyen,
 Senki nincsen.
/2
#D1060FE7
 Az Úr az égből alátekinte
 E földön az emberek fiaira,
 Hogy meglássa, ha kinek esze volna,
 Ha valaki az Istent keresné
 És tisztelné.
/3
#89A5D6F3
 De azt jól látja dicsőségében,
 Hogy a jó útról eltértek mindenek,
 Mindnyájan fertelmes bűnben hevernek;
 Aki az Istent tisztelné híven,
 Csak egy sincsen.
/4
#E3176CBD
 De a hamisak meg nem gondolják,
 Hogy népemet, mint kenyeret, megészik;
 Meg nem térnek, sőt magukat elhiszik.
 Segítségül az Istent nem hívják,
 Sem uralják.
/5
#DF42EFCF
 Azért erősen majd megrémülnek,
 Midőn eszükbe veszik, hogy az Isten
 Az ő híveit megtartja kegyesen;
 Mert ő mellé áll a jó nemzetnek,
 Mint híveknek.
/6
#925C6B8D
 De néktek semmi gondotok erre,
 Hanem ti csak csúfoljátok a szegényt,
 Aki tiszta szívből féli az Istent,
 És csak őtet hívja segítségre
 És mentségre.
/7
#38CFADAE
 A Sionról vajon ki jövend el,
 Ki a szent Izráelt megszabadítja?
 Ha Isten fogságból népét kihozza,
 Örvend a Jákób és az Izráel
 Teljes szívvel.

;Ki mehet az Úr elé?
;(1539) Bourgeois L., Genf, 1542
>15
/1
#6FBB6F9D
 Uram, ki lészen lakója
 A te felséged sátorának?
 Jelentsd meg és add tudtomra,
 Ki lészen örök lakosa
 Szent hegyednek és hajlékodnak?
/2
#C4A3E68A
 Aki jár igaz életben,
 Szól és szolgáltat igazságot,
 Forgódik híven mindenben,
 És hűséges ő szívében,
 És szereti a jámborságot.
/3
#0B542E12
 Ki az ő felebarátját
 Ő nyelvével nem rágalmazza,
 Kárral nem bántja szomszédját,
 Nem rútolja atyjafiát,
 És tisztességét nem gyalázza.
/4
#C234F6CB
 Az istentelent utálja,
 Az istenfélőt megbecsüli,
 Őt nagy tisztességben tartja,
 Esküvését meggondolja,
 Ha kárt is vall, azt meg nem szegi.
/5
#CB65AF33
 Aki pénzével él híven,
 Kölcsön ád, de nem kér uzsorát;
 Az ártatlan ember ellen
 Ajándékot ő nem veszen:
 Aki így tészen, az megállhat.

;A hívek öröksége: Isten
;Bourgeois L., Genf, 1551
>16
/1
#00261735
 Tarts meg engemet, ó, én Istenem,
 Mert reménységem vetem csak tebenned!
 Azért az Úrnak mondjad ezt, lelkem:
 Én Uram vagy te, örvendek tenéked!
 E kívül nem kérkedhetem semmivel,
 Hogy néked használhatnék jótétemmel.
/2
#7C74A4E1
 A szentekkel e földön jól teszek,
 És segítek jámbor istenfélőket,
 De azoknak lészen nagy sérelmek,
 Kik követnek idegen isteneket.
 Ő véres áldozatjukat nem nézem,
 És nevüket ajakimra sem vészem.
/3
#E42C24A7
 Az Úr Isten az én örökségem,
 Mely nekem kiváltképpen részeltetett,
 És ő megtartja híven azt nékem.
 Az én sorsom a legjobb részre esett;
 Hogy az örökség zsinórral osztatott,
 Azzal a legszebbik rész nekem jutott.
/4
#6E460970
 Dicséreteket mondok az Úrnak,
 Ki tanácsával engem jól vezérel;
 Még veséim is erre tanítnak,
 És az ágyban is megintenek éjjel.
 Az Urat szüntelen előttem tartom:
 Mert jobbom felől van, nem ingadozom.
/5
#5E05CDF1
 Öröme vagyon szívemnek ezen,
 Örvendez lelkem és megnyugszik testem:
 Noha a sírban fekvése leszen,
 De azt szívemben nyilván én elhiszem:
 A koporsóban nem hagyod szentedet,
 De rothadás ellen megmented őtet.
/6
#C2C85DBB
 Az életre nekem utat mutatsz,
 Mely életben van az örök boldogság,
 Hol szent színednek gyönyörű voltát
 Láttatod, mely nagy öröm és vigasság.
 Nagy dicsőséged, jobbod erőssége
 Örökké megmarad, soha nincs vége.

;Könyörgés oltalomért
;Bourgeois L., Genf, 1551
>17
/1
#F2592093
 Hallgasd meg igazságomat,
 Én kiáltásom, Uram, értsd meg!
 Hozzám figyelmezz és tekintsd meg
 Szívbéli imádságomat!
 Ítéletet tetőled várok,
 Nézd meg ügyem, láttass törvényt:
 Ítélj meg igazság szerint,
 Mert én álnoksággal nem járok.
/2
#A59C3A31
 Szívem éjjel megpróbáltad,
 És megvizsgáltad teljességgel,
 Láttad, hogy egyez én nyelvemmel:
 Csalárdság nélkül találtad.
 Amit ember szól, avagy művel,
 Én ajakid beszédire,
 Gondot tartok szent Igédre,
 Nem járok a gonosztevővel.
/3
#68B22933
 Vezérljed én járásomat,
 És tarts meg a te ösvényiden,
 Máshova senki ne térítsen;
 Erősítsed lábaimat!
 Téged hívlak én segítségre,
 Uram, láss meg szükségembe,
 Kérésem vedd füleidbe,
 És légy figyelmes beszédemre.
/4
#2336834E
 Te vagy bizonyos oltalmok,
 Akik tebenned reménylenek;
 Jóvoltod mutasd meg ezeknek,
 Hogy lássák a rád támadók:
 Mint a te szemeidnek fényét,
 Úgy őrizz meg, Uram, engem,
 Szárnyad árnyékában fejem
 Takargasd, őrizvén ösvényét.
/8
#04D65154
 Én pedig az igazságban
 Színét meglátom Felségednek,
 És ha álmomból felserkenek,
 Megelégszem jóvoltodban.

;Hálaadás a csodálatos szabadításért
;Bourgeois L., Lyon, 1547
>18
/1
#1B3E459C
 Ó, én Uram, ki erőt adsz énnékem,
 Szeretlek téged, míg leszen életem.
 Én magas kőszálam, ó, én Uram,
 Erős bástyám és erős kőváram!
 Én erős Istenem és bizodalmam,
 Idvességemnek bizonysága, pajzsom!
 Midőn az Urat dicsérvén kérem,
 Ellenségimtől ő megtart engem.
 Halál fájdalmi hogy körülvennének,
 Béliál folyói rettegtetnének,
 Pokol kötele vala körülem,
 Csaknem a halál tőribe esém.
/2
#A22A658F
 Ily ínségemben az Úrhoz kiálték,
 Nagy szükségemben Istennek könyörgék,
 Szózatom templomába felhata,
 És kiáltásom fülibe juta.
 Legottan a föld szörnyen megindula,
 A hegyek fundámentoma mozdula,
 Rengedeznek, reszketnek szertelen,
 Mert haragra indult az Úr Isten.
 Nagy sűrű füst mégyen fel az ő orrán,
 Rettenetes tűz szájából kirohan,
 Mely miatt az ég széjjel villámlik,
 Mert az ő sebes haragja látszik.
/3
#3577183A
 Lehajlatá az eget és lejöve,
 Lábai alatt homály setétsége;
 Alászálla Kérubimon ülvén,
 És a szélnek szárnyain repülvén.
 Nagy setét ködben magát befedezte,
 Melyben magát, mint sátorban, rejtette,
 De a fényesség, mely nála vala,
 A sűrű ködöket eloszlatá.
 Nagy kőesők hullnak, tűzláng villámlék,
 Az Úr mennydörgése égen hallaték;
 Rettenetes szó adaték tőle,
 Kőesőt, villámlást földre löve.
/4
#94D79F0C
 Elszéleszté azokat nyilaival,
 És elrettenté ő villámlásival;
 Meglátszának a vizek mélységi,
 Felnyilának a földnek fenéki,
 A te feddésednek súlyosságától,
 És haragos orrodnak fúvásától.
 Onnan fölül kezeit lenyújtá,
 És engem a vizekből kivona.
 Megszabadíta én ellenségimtől,
 Megmente hatalmas gyűlölőimtől,
 Kik rám rohannak én ínségemben;
 De gyámolom lőn nekem az Isten.
/5
#07B0D025
 Térhelyre hoza s kimente engemet,
 És mutata énhozzám nagy szerelmet,
 Megfizete az igazság szerint,
 Kezeimnek tisztasága szerint.
 Mert az Úrnak útától nem tértem el,
 Az én Istenemtől nem szakadtam el,
 Ítéletire szüntelen néztem,
 És szent törvényét meg nem vetettem.
 De mindenkor híven előtte jártam,
 Gonosztételtől magamat megóvtam:
 Megfizete az igazság szerint,
 Kezeimnek tisztasága szerint.
/6
#AE241246
 Szent vagy és jóltevő a jóltevőkkel,
 És igazat téssz igazán élőkkel,
 Tiszta vagy azokhoz, akik tiszták,
 Elfordulsz tőlük, akik gonoszak.
 Nyomorult szegényeket megsegíted,
 A kevély szeműket megszégyeníted;
 Nékem, Uram, szövétneket gyújtasz,
 És a setétben világot nyújtasz.
 Teáltalad ellenségim seregin
 Általfutok, átszököm kerítésin.
 Mely tökéletes az Isten úta,
 Tiszta és próbált az ő mondása.
/7
#6DE3C7A1
 A benne bízóknak ő erős pajzsa,
 Ez Istennek de ki lehetne mása?
 De holott volna oly erős kőszál,
 Mint az Isten, aki minket táplál?
 Az Isten erejébe felöltöztet,
 És útaimban jól vezérl engemet,
 Lábam gyorsítja, mint a szarvasnak,
 Magas hegyimre hogy felhághassak;
 Ő tanítja kezemet hadakozni
 És az acél kézívet elrontani;
 Idvességednek paizsát nekem
 Adod, és jobb kezed megtart engem
/10
#F7073368
 Él az Isten, kinek legyen dícséret,
 Áldom én idvezítő Istenemet,
 Ki énnékem e hatalmat adta,
 Ellenségimet meghódoltatta.
 Megőriz engemet ellenségimtül,
 Kiken engem fölemelt fejedelmül;
 Az erőszaktévőtől oltalmaz,
 Ki mindenkor halálomra vigyáz.
 Azért dícséretet mondok tenéked,
 A pogányok közt magasztallak téged,
 Szent nevedet éneklem szüntelen,
 Hogy te királyodat felségesen.
/11
#5093F147
 Megsegítéd, vele től kegyességet,
 És néki jelentél ily idvességet.
 E megkenett Dávidnak főképen,
 És maradékinak mind örökkön.

;Isten dicsősége a mennyen és az Igében
;Bourgeois L., Genf, 1542
>19
/1
#B18F5D21
 Az egek beszélik
 És nyilván hirdetik
 Az Úrnak erejit.
 Az ég menynyezeti
 Szépen kijelenti
 Kezének munkáit.
 A napok egymásnak
 Tudományt mutatnak
 Az ő bölcseségéről,
 Egy éj a más éjnek
 Beszél az Istennek
 Ő nagy dicsőségéről.
/2
#D74C1211
 Nincs szó, sem tartomány,
 Holott e tudomány
 Nem prédikáltatnék;
 Mindenfelé mégyen
 E földkerekségen
 Beszédük ezeknek.
 Írásuk kimégyen
 Mind e világ végén,
 Holott a fényes napnak
 Hajlékot az Isten
 Helyheztetett szépen
 Ő lakó szállásának.
/3
#0C0304A6
 Melyben mint vőlegény
 Reggel felkél szépen
 Ő ágyasházából,
 És ugyan örvendez,
 Mint az erős vitéz,
 Ha futásra indul.
 Az égnek egy végén
 Felkél és elmégyen
 Gyorsan a más végére;
 Sehol semmi nincsen
 Ő hévsége ellen,
 Ki magát elrejthesse.
/4
#DE9AF4A9
 Az Isten törvénye
 Tiszta, ő beszéde
 Lelkeket megtérít,
 Hűség bizonysága,
 Kisdedeket abba’
 Bölcsességre tanít.
 Ő parancsolati,
 Igazak mondási,
 Mik szívet vigasztalnak.
 Minden ő törvényi,
 Tiszták szent beszédi,
 Szemet világosítnak.
/5
#07A4926A
 Az Isten félelme
 Tiszta, mindörökké
 Megmarad és megáll.
 Az ő ítélete
 Igaz mindenekbe’,
 Teljes nagy jósággal.
 Aranynál, ezüstnél
 És drágaköveknél
 Sokkal becsületesebb;
 Ő szerelmessége
 És gyönyörűsége
 A méznél is édesebb.
/6
#73DA4D01
 Aki szolgál néked,
 Tanul, Uram, tőled
 Nagy jó tanulságot.
 És ha azt megtartja,
 Jól lészen ő dolga,
 Mert veszen jutalmat.
 Ki tudná bűninek,
 Számát esetinek,
 És ki gondolhatná meg?
 Én sok bűneimet,
 Titkos vétkeimet,
 Uram, nékem bocsásd meg!
/7
#AE2CFF0B
 Szolgádat őrizd meg,
 Kevélységtől tartsd meg:
 Ne essék e bűnbe,
 És én tiszta lészek,
 Semmi bűnt nem tészek,
 Járván te kedvedbe.
 Szájamnak szólása,
 Szívem gondolatja
 Kedves legyen tenéked!
 Adjad, ó, én Uram,
 Kősziklám, megváltóm,
 Hogy ne vétsek ellened.

;Imádság a Messiás-király szabadításáért
;Bourgeois L., Genf, 1551
>20
/1
#114162C9
 Az Úr tégedet meghallgasson
 Te nagy ínségedben,
 A Jákób Istene megtartson
 Te veszedelmedben!
 Küldjön tenéked segedelmet
 Az Úr ő szent házából,
 Tereád fordítsa kegyelmét,
 Tartson meg a Sionból!
/2
#E77CC555
 Áldozatidat megtekintse,
 Mikkel őt tiszteled,
 Égő áldozatodat tüze
 Égesse meg neked.
 Amit a te szíved kívánhat,
 Adja meg ő tenéked,
 Hogy mindennemű szándékodat
 Te jó véghez vihessed.
/3
#2E49E73F
 Adjad, Uram, hogy te nevedben
 A mi zászlóinkat
 Felemeljük nagy örömünkben
 És adjunk hálákat,
 Mondván: Az Úr Isten megőrzi
 Fölkentjét kegyelmével,
 A mennyből őtet erősítí
 Jobb keze erejével.
/4
#B3A0630C
 Némelyek az ő szekerükben,
 És bíznak lovukban;
 De mi a nagy Isten nevében
 Bízunk mint Urunkban.
 Azért ők keményen megesnek,
 Mi pediglen megállunk;
 Ők mind a földhöz verettetnek,
 De mi épen maradunk.
/5
#D1B6FD51
 Tartsd meg, Uram, és engedd nékünk,
 Hogy e király híven,
 Mikoron hozzá esedezünk,
 Segítségül legyen.

;Imádság szabadulás után (Messiási)
;Bourgeois L., Genf, 1551
>21
/1
#6D2D51C5
 Örvendez, Uram, a király
 A te nagy hatalmadban
 És szabadításodban;
 És vigad nagy buzgósággal,
 Hogy őt megsegítéd,
 Ínségből kimentéd.
/2
#A97C2000
 Úgy viseled néki gondját,
 Hogy amit tőled kérend,
 Mindeneket megnyerend.
 Mihelyt felnyitja ő száját,
 Szól alig egy igét:
 Már hallod kérését.
/3
#7E3220D8
 Elébb, hogynem könyörgene,
 Meghallgatod, meglátod,
 Irgalmaddal megáldod,
 És feltéssz az ő fejére
 Sárarany koronát
 Mint királyi pompát.
/5
#AFD42B26
 Őt felvőd nagy dicsőségre
 A te segedelmeddel,
 Örök üdvösségeddel.
 E nagy királyi felségre
 Tőled emelteték
 És ékesítteték.
/6
#D1955B9D
 Szereted őt minden jókkal,
 És a te szent áldásod
 Néki örökké nyújtod.
 Örvendetes vígságokkal
 Őt gyönyörködteted,
 Színedre nézeted.
/7
#D29A6DCE
 E király mindenkor bízik
 Csak az ő Istenében,
 És nem fél veszélyében.
 A Magasságosnak nyugszik
 Irgalmasságában
 És megmarad abban.
/8
#96057FC8
 Megtalálja kezed őket,
 Akik reád támadnak,
 Bosszúságodra járnak.
 És a te gyűlölőidet
 Kezeidből senki
 Soha ki nem menti.
/9
#BFF0618E
 Mint a hév tüzes kemence,
 Haragod körülveszi
 És őket mind ellepi.
 Haragos orcádnak színe
 Őket megemészti,
 Mint a láng, elnyeli.
/11
#EF522F0A
 Mert gonoszra igyekeztek,
 Szándékoztak ellened,
 Hogy bosszantsanak téged.
 Sok csalárdságot terveztek,
 De hogy véghez menjen,
 Erejükben nincsen.
/13
#96C69BBF
 Azért, Uram, már támadj föl:
 Mutasd meg hatalmadat,
 Lássuk erős voltodat,
 Hogy dicsőséges erődről
 Vígan énekeljünk,
 És benned örvendjünk.

;Miért hagyál el engemet? – Krisztus szenvedéséről
;Bourgeois L., Genf, 1542
>22
/1
#132120C6
 Én Istenem, én erős Istenem!
 Miért hagyál el enynyire engem?
 Kiáltásomtól a segedelem
 Nagy távol vagyon.
 Én kiáltok tehozzád egész napon,
 De mégsem felelsz meg, nincs ki megtartson;
 Még éjjel sem hallgatok semmi módon
 Ez ínségben.
/2
#ADE05DB2
 De te szent vagy és az Izráelben
 Te szentséged megmarad mindenben:
 Dicsértetel e gyülekezetben
 Szívvel, lélekkel.
 A mi régi atyáink teljességgel
 Tebenned bíztanak jó reménységgel,
 Szükségükben őket segedelmeddel
 Megtartottad.
/3
#3A864BA2
 Ha ők szívből kiáltottak hozzád,
 Mindjárt őket megszabadítottad;
 Benned bíztak és őket nem hagytad
 Esni szégyenben.
 Én nem vagyok ember, de féreg lévén,
 Minden népeknél vagyok nevetségben:
 Csúfolnak, utálnak és megvet minden
 És keserget.
/4
#1E48337A
 Aki lát, minden csúfol engemet,
 Száját elvonssza, szól merő mérget,
 Fejét rázza, hunyorgatja szemét,
 Reám néz szörnyen,
 Mondván: ez ember bízott az Istenben,
 Szabadítsa meg azért őtet innen;
 Ha szereti az Isten, néki légyen
 Segítségül.
/5
#149B50FD
 Hogy te engemet anyám méhéből
 Kihozál, ottan segedelmem lől,
 Csecsemő koromban is egyedül
 Csak benned bíztam.
 Sőt, mihelyt anyám méhéből származtam,
 Istenem voltál, reád támaszkodtam,
 Bátorsággal tehozzád ragaszkodtam,
 Én Istenem!
/6
#D12BF835
 Ne távozzál azért messze tőlem,
 Ne hagyj el, mert nagy az én gyötrelmem!
 Nincs segítőm, és az én sérelmem
 Nem fáj senkinek.
 Sok erős bikák engem körülvettek,
 A básáni nagy ökrök reám törnek,
 Megölni, megtaposni igyekeznek
 Nagy méltatlan.
/7
#E38A602E
 Az ő szájukat énreám tátván,
 Mint ragadozó, sívó oroszlán,
 Agyarkodnak, hogy engem torkukban
 Béfalhassanak.
 Könnyhullatásim, mint a vizek, folynak,
 Én csontjaim helyükből kimozdulnak.
 Szívem, mint viasz, olvad, bélim fájnak
 Sebek miatt.
/8
#5CF3A61C
 Mint cserép, minden erőm elszáradt,
 Száraz nyelvem az ínyemhez ragadt;
 Porba vetél engem, érzem kínját
 Halál mérgének.
 Mert engemet sok ebek körülvettek,
 Gonosz népek ellenem összegyűltek,
 Kezeimet és lábaimat ezek
 Általszúrták.
/9
#39AC2B61
 Én csontjaimat megolvashatnák,
 Szörnyű szemeket reám fordítnak.
 Kínomban nem szánnak, de gúnyolnak,
 Űznek csúfságot.
 Elosztották egymás közt én ruhámat,
 És öltözetemre vetettek sorsot,
 Hogy abból ne metélnének foltokat,
 Osztván részre.
/10
#4F1F57DF
 Azért tőlem, Uram, ne légy messze,
 Ne késsél, életemnek ereje!
 Kérlek, siess, tekints ínségemre:
 Légy segedelmül!
 Mentsd meg életem az éles fegyvertül,
 Védelmezz meg e sok dühös ebektül,
 Egyedül-voltomat mentsd meg ezektül,
 Jóvoltodbul!
/11
#CB3B58E5
 Tarts meg az éh oroszlán torkátul,
 És az egyszarvú fenevadaktul,
 Mik mostan körülvettek nagy mordul,
 Ó, tarts meg engem!
 Melyért nevedet híven hadd dicsérem,
 Az én atyámfiainak hirdetem,
 És a szent gyülekezetben tisztelem
 Felségedet.
/12
#40D49C66
 Istenfélők, dicsérjétek őtet,
 Jákób fiai, áldjátok nevét;
 Izráel népe, féld e Felséget
 Mint Istenedet!
 Mert nem utálja a szegénynek ügyét,
 És tőle el nem fordítja szent színét,
 De ha kiált, meghallgatja kérését
 Nagy kegyesen.
/13
#727CD18A
 Azért dicséretem rólad leszen
 Minden előtt a gyülekezetben,
 És én fogadásim semmiképpen
 Meg nem töretnek.
 A szegények esznek, megelégesznek,
 Téged az Istent keresők dicsérnek,
 Él az ő lelkük, és benned örvendnek
 Mindörökké.
/14
#E921CB1F
 E földi népnek minden serege
 Az Úrhoz gyűl ez emlékezetre.
 És a pogányoknak nemzetsége
 Néki hajt fejet.
 Mert egyedül az Úr bír mindeneket,
 Övé az ország, és a pogány népet
 Bírja, rajtuk megmutatván erejét
 Kezeinek.
/15
#0EAC21F7
 A kövérek, kik megelégedtek,
 És kik már porrá lenni készültek,
 Dicsérnek téged minden szegények
 És nyomorultak,
 És maradéki mindnyájan azoknak
 Néked szolgálván, térdet-fejet hajtnak
 És firól fira téged ők uralnak,
 Ó, nagy Isten!
/16
#D28AE858
 Minden nemzetek a jövendőben
 Te igazságodat dícséretben
 Tisztelik, magasztalják szívükben
 Az egész földön.

;A jó Pásztor
;Bourgeois L., Strasbourg, 1545
>23
/1
#B1D6CB2B
 Az Úr énnékem őriző pásztorom,
 Azért semmiben meg nem fogyatkozom.
 Gyönyörű szép mezőn engemet éltet,
 És szép kies folyóvízre legeltet;
 Lelkemet megnyugtatja szent nevében,
 És vezérl engem igaz ösvényében.
/2
#F8B1B04A
 Ha a halál árnyékában járnék is,
 De nem félnék még ő sötét völgyén is,
 Mert mindenütt te jelen vagy énvelem,
 Vessződ és botod megvigasztal engem,
 És nekem az én ellenségim ellen
 Asztalt készítesz, eledelt adsz bőven.
/3
#C9DEC6E2
 Az én fejemet megkened olajjal,
 És engemet itatsz teljes pohárral;
 Jóvoltod, kegyességed körülvészen
 És követ engem egész életemben.
 Az Úr énnékem megengedi nyilván,
 Hogy mind éltiglen lakjam ő házában.

;Isten tiszteletére hív a mindenség
;Bourgeois L., Lyon, 1547
>24
/1
#288DE7FE
 Az Úr bír ez egész földdel
 És minden benne élőkkel;
 Övé a földnek kereksége,
 Mit a tengeren épített,
 Folyóvizekkel körülvett,
 Melyben meglátszik bölcsesége.
/2
#509502A2
 Ki mégyen fel a szent helyre,
 Az Úrnak ő szent hegyére?
 És vajon kitől tiszteltetik?
 Akinek tiszta ő szíve,
 És ártatlan az ő keze,
 Aki hamisan nem esküszik.
/3
#9FFD8822
 Ezt az Úr megáldja szépen,
 És az idvezítő Isten
 Adja igazságát őnéki.
 Az pedig a boldog nemzet,
 Mely, nézve az ő szent színét,
 Jákóbnak Istenét keresi.
/4
#DC94F7D0
 Ti szent kapuk, kinyíljatok,
 Fejeteket feltartsátok:
 E dicső király hadd térjen be!
 Micsoda dicső király ez?
 A seregek Istene ez,
 Nagy ennek hadi erőssége.
/5
#0FADFB06
 Ti kapuk, emelkedjetek,
 Fejeteket felvessétek,
 Hogy e király belétek térjen!
 Kicsoda e nagy királyság?
 Ez a Zebaoth uraság!
 Mely nagy az ő dicsőségében!

;Könyörgés oltalomért, vezetésért és bűnbocsánatért
;Bourgeois L., Genf, 1551
>25
/1
#6D46C445
 Szívemet hozzád emelem,
 És benned bízom, Uram;
 És meg nem szégyeníttetem,
 Nem nevet senki rajtam,
 Mert szégyent nem vallanak,
 Akik hozzád esedeznek,
 Azok pironkodjanak,
 Akik hitetlenül élnek.
/2
#2BDF7DE5
 Útaid, Uram, mutasd meg,
 Hogy el ne tévelyedjem;
 Te ösvényidre taníts meg,
 Miken intézd menésem.
 És vezérelj engemet
 A te szent igaz Igédben;
 Oltalmazd életemet,
 Mert benned bízom, Úr Isten.
/3
#E8FB432D
 Emlékezzél jóvoltodból
 Nagy kegyelmességedre,
 Emlékezzél irgalmadról,
 Mely megmarad örökre.
 Ifjúságomnak vétkét,
 Kérlek, hogy meg ne említsed.
 Sőt nagy kegyességedet,
 Én Istenem, megtekintsed.
/4
#34A42705
 Jó és igaz az Úr Isten
 Mind örökkön-örökké,
 A bűnösöket térítvén
 Ő igaz ösvényire;
 És a nyomorultakat
 Életükben igazgatja,
 Nagy kegyesen azokat
 Az ő útában megtartja.
/5
#88E96403
 Az Istennek minden úta
 Kegyesség és nagy hűség
 Azoknak, kik mondására
 Gondot tartnak mindvégig.
 Énnékem kegyelmezz meg,
 Uram, a te szent nevedért.
 És bűnömet bocsásd meg,
 Ne ostorozz nagy voltáért!
/6
#1A52B2FE
 Aki az Úr Istent féli
 És tiszteli szívében,
 Azt ő nagy híven vezérli
 Igaz ösvényeiben.
 Nagy békességben annak
 Minden jó bőven adatik,
 És ő maradékinak
 Gazdag örökség hagyatik.
/7
#6E7E8F36
 Az igaz istenfélőknek
 Megjelenti titkait,
 És az őbenne hívőknek
 Megmutatja kötésit.
 Istenhez szemeimet
 Felemelem szüntelenül,
 Ő megőriz engemet,
 Lábam kivonssza a tőrbül.
/8
#5A58B218
 Térj azért hozzám, Istenem,
 Tekints reám kegyesen,
 És kegyelmezz meg énnékem,
 Mert élek szegénységben.
 Nyavalyája szívemnek
 Napról napra mind öregbül;
 Uram, add végét ennek,
 Végy ki engem ez ínségbül!

;Hűséges szív reménysége
;Bourgeois L., Genf, 1551
>26
/1
#57BE197C
 Légy ítélőm, Uram,
 Mert hűséggel jártam
 És éltem nagy ártatlanul!
 Azért hiszem az Istent,
 Hogy ő engemet megment
 Minden háborúságomtól.
/2
#7E5BE8E0
 Próbálj és kísérts meg,
 Ügyemet jól nézd meg:
 Meglátod tisztaságomat!
 Vizsgáld meg veséimet,
 És próbáld meg szívemet,
 Hogy értsed indulatomat.
/3
#E6048D1B
 Látom szemem előtt
 Kegyelmességedet,
 És azon vagyok szüntelen,
 Hogy minden dolgaimban
 Járjak igazságodban,
 Ne vétsek Felséged ellen.
/4
#7DCA5E71
 A hazug emberek
 Nálam nem kedvesek;
 A tettető csalárdokat
 Szívem szerint gyűlölöm,
 És nagy távol kerülöm
 Az álnoksággal járókat.
/5
#063F00AD
 A gonosztevőknek
 És hamis népeknek
 Társaságukat gyűlölöm.
 Ő gyülekezetükben
 Nem ülök semmiképpen,
 Sőt előttem sem szenvedem.
/6
#3F3CF109
 Belső tisztaságban
 És ártatlanságban
 Én kezeimet megmosom;
 Tisztán téged dicsérlek,
 Áldozván Felségednek,
 Oltárod körül forgódom.
/7
#B966B510
 Éneki felszóval
 És víg hangossággal
 Magasztalom Felségedet;
 Néked adván hálákat,
 Hirdetvén csodáidat,
 Mindenütt áldlak tégedet.
/8
#120DC182
 Uram, hajlékodat,
 Szeretem házadat,
 Holott lakol dicsőséggel,
 Szent helyedet kedvelem,
 És azt feljebb becsülöm
 Minden e világi kincsnél.
/9
#CFBF26B6
 Ostorodat, Uram,
 Fordítsd el énrólam,
 Ne büntess a bűnösökkel!
 Vélük ne verd lelkemet,
 Ne vedd el életemet
 A vérontó emberekkel!
/10
#498D12D8
 Az én lábam megáll,
 Tágas helyet talál,
 És megmarad ösvényiben:
 Azért, Uram, dicsérlek,
 És örömmel tisztellek
 A hívek szent seregiben.
/12
#6FCC2999
 Az én lábam megáll,
 Tágas helyet talál,
 És megmarad ösvényiben:
 Azért, Uram, dicsérlek,
 És örömmel tisztellek
 A hívek szent seregiben.

;Elég nékem az Isten kegyelme (II. Kor. 12:9)
;Bourgeois L., Genf, 1551
>27
/1
#28370D5B
 Az Úr Isten az én világosságom,
 És idvességem, hát kitől félnék?
 Ő életemnek ereje, jól tudom:
 Ki volna hát, akitől rettegnék?
 Midőn a kegyetlen gonosztévők,
 Mint ellenségim, énreám törnek,
 Hogy engemet ugyan megégyenek:
 Megbotolnak és mind elesnek ők.
/2
#F01FCC46
 Hogyha táborral körülvennének is,
 De mégsem félne semmit én szívem;
 Ha szintén az ellenség közt volnék is:
 Őbenne vetném mégis reményem.
 Egy dolgot kívántam én az Úrtól,
 Melyet még most is kérek nagy bízván:
 Hogy lakhassam az Úrnak házában,
 Míg e földön élek jóvoltából.
/3
#190629DC
 Melyet én azért kérek, hogy meglássam
 Az Úrnak felséges dicsőségét,
 És ő szent templomát látogathassam,
 Mely tisztességére építtetett.
 Mert engem hajlékába takarít
 Én háborúságimnak idein,
 És elrejt engem rejtekhelyein,
 Magas kősziklára felemelít.
/4
#2BCB0BDE
 És az én fejemet ő felemeli
 Ellenségimen, kik körülvettek;
 Azért az ő hajlékában őnéki
 Felszóval áldozom és éneklek:
 Uram, mikor hozzád felkiáltok,
 Figyelmetesen hallgass meg engem;
 Kegyelmezz meg, légy gyámolom nékem,
 Midőn tehozzád szívből óhajtok!
/5
#94B5F6BD
 Te felőled ezt mondja az én szívem:
 Keressétek az én szent színemet!
 Azért nékem is ez igyekezetem:
 Nézhessem, Uram, te szent színedet.
 Ne fordítsd el hát orcádat tőlem,
 Szolgádat ne vesd el haragodban;
 Vígasztalóm vagy nyavalyáimban:
 Segíts meg, Uram, ne hagyj el engem!
/6
#94FCCFF2
 Mert mind én atyám s anyám elhagy engem,
 De az Úr kegyesen hozzá vészen.
 Mutasd meg, Uram, te utadat nékem,
 Ellenség ellen tarts ösvényedben!
 Kívánságukra ellenségimnek
 Ne adj engemet, mert sokan vannak,
 Kik ellenem hamisságot szólnak,
 Hazudnak és erőszakot tesznek.
/7
#E1E35434
 Ha nem hittem volna, hogy még éltemben
 Jóvoltát az Úrnak meglátandom
 Az élőknek földén: hát immár régen
 Odalett volna minden én dolgom.
 Várjad azért bizonnyal az Urat,
 Légy víg és bátorságos szívedben,
 Mert téged megtart a nagy Úr Isten,
 Csak tőle várjad hát oltalmadat.

;Ellenség ellen
;Bourgeois L., Genf, 1551
>28
/1
#E38F9628
 Hozzád kiáltok, kegyes Uram,
 Én segítségem és kőváram!
 Hallgass meg kegyelmesen engem!
 Ne hallgass el, mert el kell vesznem!
 Azokhoz hasonló lészek,
 Kiknek a koporsó helyek!
/2
#F16406F7
 Midőn tehozzád esedezem,
 És kezeimet felemelem
 A te szentséges templomodban:
 Hallgass meg én imádságomban!
 Ne büntess a hitlenekkel,
 Ne verj a gonosztevőkkel!
/5
#1112A68C
 Áldott légyen a nagy Úr Isten,
 Ki meghallgata kérésemben!
 Az Úr énnékem erősségem,
 Én paizsom és segedelmem;
 Örvend szívem és énekben
 Dicsérem őtet szüntelen.
/6
#6A9F646B
 Az Úr én népemnek ereje,
 A Krisztusnak nagy erőssége.
 Tartsd meg azért a te népedet,
 És áldjad meg örökségedet:
 Legeltessed és vigasztald,
 És örökké felmagasztald!

;A hét mennydörgés zsoltára (Jel. 10:3 skk.)
;Bourgeois L., Genf, 1551
>29
/1
#4A42222F
 Mostan, ti hatalmasak,
 Tekintetes nagy urak,
 Adjatok az Istennek,
 Dicsőséget nevének!
 Mint hatalmas Istenteket;
 Féljétek, tisztelvén őtet!
 Szent templomában áldjátok,
 És térdet, fejet hajtsatok!
/2
#7A8586FE
 Az Úr szava megzendül,
 A vizeken megdördül;
 Mennydörgő dicsősége
 Elhat a nagy tengerre.
 Az Úrnak rettentő szava
 Nagy hatalmát megmutatja.
 Az Úrnak dördülő szaván
 Nagy volta meglátszik nyilván.
/5
#216D485D
 De az ő templomában
 Ő hívei mindnyájan
 Hirdetik nagy erejét,
 Beszélik dicsőségét.
 Az Úr ült az özönvizen,
 Mint bíró ítélőszéken;
 Az Úrnak ő királysága,
 Örökké megáll országa.

;Hálaének a haláltól való szabadulásért
;Bourgeois L., Genf, 1551
>30
/1
#96E9F416
 Dicsérlek, Uram, tégedet,
 Mert te megtartál engemet,
 És kegyesen felemelél,
 Ellenségimtől megmentél,
 És meg nem engedéd azoknak,
 Hogy nyavalyámon vigadjanak.
/2
#D16971AC
 Hogy felkiálték tehozzád,
 Nyavalyámat meggyógyítád,
 És hogy én csaknem a sírba,
 Esném a halál torkába:
 Ismét feltámasztál engemet,
 Pokoltól megmentéd lelkemet.
/3
#ACC7195D
 Istenes hívek és szentek,
 Az Úrnak énekeljetek,
 Áldjátok őtet mindvégig,
 Mert nem haragszik sokáig,
 Az ember alig gondolhatja,
 Mily hamar elmúlik haragja.
/4
#7AB6CE93
 De az ő kegyelmessége
 Rajtunk megmarad örökre.
 Néha oly dolgom érkezik,
 Min este szívem bánkódik,
 De reggel, mihelyen felkelek,
 Azonnal víg örömet lelek.

;Bizodalmas könyörgés nagy nyomorúságban
;Bourgeois L., Genf, 1551
>31
/1
#0B3B2F80
 Uram, én csak tebenned bíztam,
 Őrizz meg kegyesen,
 Ne essem szégyenben!
 Te igazságod fordítsd hozzám,
 És tarts meg jóvoltodból,
 Ments ki nagy nyavalyámból!
/2
#41D53B2C
 Hajtsd énhozzám, Uram, füledet,
 Ó, én idvességem!
 Siess, tarts meg engem!
 Mutasd meg nagy erősségedet,
 Légy én erős kőváram,
 Melyben bátran lakhassam!
/3
#09AF05A4
 Te vagy kősziklám, erősségem:
 Szent nevedért kérlek,
 Vezérelj, hogy éljek!
 Szabadíts ki a tőrből engem,
 Melyet énnékem vetnek,
 Mert megtartómnak hiszlek!
/4
#69DFF3CF
 Lelkemet kezedbe ajánlom,
 Mert nagy ínségemben
 Megtartál, Úr Isten.
 Szívemet azoktól megvonszom,
 Akik élnek hívságban;
 És csak bízom Uramban.
/5
#7DE14196
 Örvendezek nagy vigasságban,
 Vigadok szívembe',
 Irgalmadra nézve.
 Ha megtekintesz nyavalyámban,
 Szívemet megismered,
 Sérelmit megtekinted.
/6
#C8E92319
 Nem ejtesz engemet kezükben
 Én ellenségimnek,
 Kik engem gyűlölnek,
 Sőt minden ellenségim ellen
 Útat adsz lábaimnak,
 Hogy meg ne akadjanak.
/13
#4B09C87C
 Orcádat világosítsd rajtam,
 És szolgáddal tégy jól
 Te nagy irgalmadból,
 Hogy soha meg ne csúfoltassam,
 Mert tehozzád kiáltok,
 Oltalmat tőled várok!
/15
#64563BB6
 Jóvoltod felette igen nagy,
 Kit tartasz híveknek,
 Akik téged félnek,
 És csudaképen irgalmas vagy,
 Melyet azok jól látnak,
 Akik tebenned bíznak.
/16
#CA893332
 Tenálad azokat elrejted,
 Megőrzöd kegyesen
 A kevélyek ellen.
 Orcádnak rejtekébe vészed,
 Hogy a kegyetlen nyelvek
 Őket meg ne sérthessék.
/17
#2415F111
 Dicsőség adassék az Úrnak,
 Ki tart oltalmában,
 Mint egy szép városban,
 Melynek erős bástyái vannak,
 Hogy aki abban lakik,
 Senkitől nem bántatik.
/18
#176FB774
 Félelmes futásomban mondék:
 Tőled elvetettél,
 Rám nem nézsz szemeddel;
 De tenálad kegyelmet lelék,
 Meghallád könyörgésem,
 Megadád, amit kértem.
/19
#70FC7553
 Az Urat szeressétek, szentek,
 Ki megtart híveket, Büntet kevélyeket!
 Ő legyen néktek reménységtek!
 Higgyétek szívetekben:
 Megvigasztal az Isten.

;A bűnbocsánat útja (Második bűnbánati zsoltár)
;Bourgeois L., Lyon, 1547
>32
/1
#1F639CA9
 Ó, mely boldog az oly ember éltébe',
 Akit az Isten bevett kegyelmébe,
 És megbocsátá az ő vétkeit,
 És befedezte minden bűneit.
 Boldog, akinek ő nagy hamissága
 Istentől néki nincs tulajdonítva,
 És csalárdság nincsen ő szivében,
 Tettetés nélkül jár életében.
/2
#00415B9C
 Hogy bűnömet el akarám hallgatni,
 És neked meg nem akarám vallani,
 Csontjaim ottan elszáradának
 Soksága miatt én siralmimnak,
 Mert éjjel-nappal kezed nehéz volta
 Nagy bűneimért rajtam fekszik vala,
 Elfogya bennem minden erősség,
 Mint nyári hévségben a nedvesség.
/3
#56EB0B44
 De hogy bűnömet előtted megvallám,
 Nagy vétkeimet el nem hallgathatám,
 De híven előbeszélém neked:
 Ott bocsánatot nyerék tetőled.
 Azért az Úr Istennek minden hívek
 Könyörögjenek, míg vagyon idejek,
 Mert ha nagy árvizek jönnének is,
 De nem árthatna ezeknek mégis!
/4
#B7932200
 Te vagy oltalmam, őrizz meg engemet,
 Minden gonosz ellen tartsd meg lelkemet,
 Vigasztalj meg, hogy örvendezhessek,
 És vígan néked énekelhessek!
 Tanítlak téged, úgy mond az Úr Isten,
 És vezérellek az igaz ösvényen,
 Szememmel mindig reád vigyázok,
 És igazgatlak, rád gondot tartok.
/6
#BF02B501
 Örvendezzetek a nagy Úr Istenben,
 És igaz hívek, legyetek örömben,
 Vígadjatok és énekeljetek,
 Akiknek vagyon tiszta szívetek!

;Isten a teremtő és gondviselő
;Bourgeois L., Strasbourg, 1545
>33
/1
#2CDD495A
 Nosza, istenfélő szent hívek,
 Örvendezzetek az Úrnak,
 Mert illik, hogy őtet dicsérjék,
 Kik örülnek igazságnak!
 Áldjátok azértan
 Hangos citerákban!
 Az Úr áldassék!
 Lantban, hegedűben,
 Cimbalmi zengésben
 Magasztaltassék.
/2
#99C51586
 Énekeljetek néki vígan
 Gyönyörű szép új éneket!
 Szép hangicsáló szerszámokban
 Mondjatok ékes verseket!
 Igaz ő mondása,
 Állhatatos dolga:
 Amit az Úr szól,
 Megáll igazságban,
 Minden dolgaiban
 Cselekeszik jól.
/3
#5ADF2ED7
 Ő szereti az igazságot,
 Az ítélet nála kedves,
 Dolgában tart irgalmasságot,
 Mellyel mind e világ teljes.
 Az Úrnak Igéje
 Egeket teremte,
 Melyeket ott fenn
 Szájának lelkével
 Nagy szép seregekkel
 Szerze ékesen.
/4
#E4CEDFDA
 Az Úr Isten mint egy tömlőbe
 Szorítja a tenger vizét,
 Elrejti, mint egy kincses helybe,
 Ő mondhatatlan mélységét.
 Minden földi népek,
 Az Urat féljétek!
 Mind e világon Valakik hol laktok,
 Őtőle tartsatok Szorgalmatoson!
/5
#1493DCCC
 Mert mihelyt ő csak egy igét szól,
 Azonnal megleszen minden,
 És valamit ő megparancsol,
 Nagy hamarsággal meglészen.
 Pogányok tanácsát
 És minden szándékát
 Az Úr megtöri,
 Magukban a népek
 Amit elvégeznek,
 Semmivé tészi.
/6
#E089185A
 De az Úr Istennek tanácsa
 Megmaradánd mindörökké,
 És megáll minden gondolatja,
 Kiterjed minden időkre.
 Boldog az a nemzet,
 Ki ily Urat szeret,
 Mint ő Istenét;
 Boldog a nemzetség,
 Kit kedvel e Felség
 Mint örökségét.
/7
#6FB0FCD1
 Az Úr Isten a magas mennybül
 Alátekint szemeivel,
 E világra néz onnan felül
 Minden nemzetekre széjjel.
 Szép palotájábul
 Szeme aláfordul E széles földre;
 Nagy dicsőségesen,
 Vigyáz nagy fényesen
 Az emberekre.
/9
#3A5CB636
 Aki ő gyors lovában bízik,
 Megcsalatkozik dolgában,
 Aki karjával dicsekedik,
 Meg nem tartatik a hadban;
 De a nagy Úr Isten
 Népére szünetlen Néz szemeivel;
 Akik benne bíznak,
 Hozzá folyamodnak,
 Nem felejti el.
/10
#08DC3C68
 Gondot tart rájuk s a haláltól
 Megtartja őket éltükben,
 Szükségtől és éhenhalástól
 Őrzi a drága időben.
 Lelkünk azért várja,
 Szüntelen óhajtja
 Az Úr oltalmát,
 Ki paizsul végre
 Eljő segítségre,
 Ád diadalmat.
/11
#6268FA47
 Őbenne azért a mi szívünk
 Igen örvendez szüntelen,
 Mert ő minékünk reménységünk,
 És bízunk ő szent nevében.
 Nagy kegyelmességed
 Mirajtunk bővítsed,
 Légy mi gyámolunk!
 Ne hagyj szükségünkben,
 Segíts meg éltünkben,
 Mert téged várunk!

;Isten megvédi az övéit
;Bourgeois L., Genf, 1551
>34
/1
#40F6EF72
 Mindenkoron áldom
 Az Urat, míg engem éltet,
 És az ő szent dicséretét
 Szájamban hordozom.
 Dicsekedvén áldja
 Lelkem jó voltát az Úrnak,
 Mit a szegények hallanak,
 És örülnek rajta.
/2
#5D915CE8
 Magasztaljuk vígan
 Az Úrnak áldott szent nevét,
 És véghetetlen kegyelmét
 Dicsérjük mindnyájan!
 Mert midőn keresém
 És kérém az én Uramat,
 Meghallgatá nagy felszómat,
 És megtarta engem.
/3
#A419E376
 Az Úrra kik néznek,
 Tőle megvigasztaltatnak,
 Soha orcáik azoknak
 Meg nem szégyenülnek.
 A szegény kiálta,
 És meghallgatá az Isten,
 El nem hagyá ínségében,
 De megszabadítá.
/4
#147EE215
 Az Isten angyali
 Hívek körül tábort járnak,
 Istenfélőket megtartnak
 Mint Isten követi.
 Jó voltát az Úrnak
 Kóstoljátok és lássátok,
 Mert bizony azok boldogok,
 Őbenne kik bíznak.
/5
#5B6BF454
 Féljétek az Urat
 Mindnyájan, hívek és szentek,
 Mert nem lesz soha szükségek,
 Kik félik hatalmát.
 Az oroszlánoknak
 Gyakorta szükségük esik,
 De kik az Istent tisztelik,
 Meg nem fogyatkoznak.
/6
#7A2A8132
 Én fiaim, jertek,
 Hallgassatok beszédemre:
 Megtanítlak szent életre,
 Hogy Istent féljétek!
 Kicsoda az ember,
 Ki hosszú életet kíván,
 És minden ő dolgaiban
 Jó állapotot két?
/7
#DB928F22
 Őrizd meg nyelvedet
 A gonosz rágalmazásától,
 Óvd ajakid álnokságától,
 Meg ne sértsd más hírét!
 Tégy jól, gonoszt gyűlölj,
 Békességet kérj és kövess,
 És, hogy Istennél légy kedves,
 Tisztességnek örülj!
/8
#351F4D01
 Az Isten szemei
 Látják a gonosztévőket,
 És ő emlékezetöket
 A földről eltörli.
 A jókat nem hagyja;
 Kik hozzá tiszta szívükből
 Kiáltnak nagy ínségükből:
 Mind megszabadítja.
/9
#28791364
 Közel az Úr Isten
 A töredelmes szívekhez,
 És a sérelmes lelkekhez
 Lészen segítséggel.
 Az igaznak itten
 Ő nyavalyája sokasul,
 De nyomorúságaibul
 Kimenti az Isten.

;„Perelj, Uram, perlőimmel!”
;Bourgeois L., Genf, 1551
>35
/1
#E7AC8A3A
 Perelj, Uram, perlőimmel,
 Harcolj én ellenségimmel,
 Te paizsodat ragadd elő!
 Én segedelmemre állj elő!
 Dárdádat nyújtsd ki kezeddel,
 Ellenségimet kergesd el!
 Mondjad ezt az én lelkemnek:
 Tégedet én megsegítlek!
/4
#9E8A9E52
 Dicsérlek téged szüntelen
 Nagy sűrű gyülekezetben,
 És nagy roppant sereg nép előtt
 Téged dicsérlek minden fölött.
 Ne engedd, hogy örüljenek,
 Akik ok nélkül gyűlölnek;
 Ellenségimet fordítsd el,
 Ne gúnyoljanak szemükkel.
/5
#88ACD56C
 Már azok énekeljenek,
 Kik igazamnak örülnek,
 Mondván: hála legyen az Úrnak,
 Ki nyugalmat ád szolgájának!
 Én nyelvem igazságodat,
 Hirdeti nagy jóvoltodat;
 Dicséretedet nagy híven
 Éneklem minden időben.
/7
#35F26007
 Gyászban jártam lehorgadva,
 Mint ki az anyját siratja,
 De ők szomorú esetemen
 Örülnek, és gyűlnek seregben.
 Hátmögül a gonosz népek
 Engemet szörnyen nevetnek;
 Ártatlan lévén, nem szánnak,
 Sőt csúfolnak és szaggatnak.
/8
#0FFFB330
 A képmutató galibák
 Fogukat rám csikorgatják;
 És rajtam nagy csúfságot űznek,
 Kik csak zabálódást keresnek.
 Uram, míg nézed ezeket?
 Jövel, tartsd meg én lelkemet!
 Egyedül voltom tekintsd meg,
 Ez oroszlánoktól ments meg!
/9
#698C710F
 Dicsérlek téged szüntelen
 Nagy sűrű gyülekezetben,
 És nagy roppant sereg nép előtt
 Téged dicsérlek minden fölött.
 Ne engedd, hogy örüljenek,
 Akik ok nélkül gyűlölnek;
 Ellenségimet fordítsd el,
 Ne gúnyoljanak szemükkel.
/13
#55547146
 Már azok énekeljenek,
 Kik igazamnak örülnek,
 Mondván: hála legyen az Úrnak,
 Ki nyugalmat ád szolgájának!
 Én nyelvem igazságodat,
 Hirdeti nagy jóvoltodat;
 Dicséretedet nagy híven
 Éneklem minden időben.

;Ember gonoszsága – Isten jósága
;Greiter M., Strasbourg, 1525 (1539) után
>36
/1
#13C7736D
 A gonosztévőknek dolgán
 Eszembe veszem azt nyilván,
 Hogy Istenre nincs gondja.
 Magában felfuvalkodik,
 Bűneitől meg nem szűnik,
 A híveket utálja.
 Hamis és hazug beszéde,
 Jó tanúsághoz nincs kedve,
 És nem jár igazsággal;
 Hívságot gondol ágyában,
 Foglalatos gonosz útban,
 Semmi bűnt ő nem utál.
/2
#A36853C0
 Uram, a te nagy hűséged
 Égig ér, kegyelmességed
 Mind a felhőkig felhat.
 Mint a hegy, te igazságod,
 Törvényed mélység, megtartod
 Az embert és a barmot.
 Te kegyességed mily drága!
 Azért a te szárnyad alá
 Emberek folyamodnak,
 Kik jól megelégíttetnek,
 Mint bő vízzel, legeltetnek
 Javaival házadnak.
/3
#5FFFA2C1
 Nálad az élet kútfeje,
 Világodnak nagy ő fénye,
 Mely nekünk szépen fénylik.
 Bővítsd rajtuk kegyességed,
 Akik jól ismernek téged,
 Szívvel neved tisztelik!
 Ne hagyd, hogy a kevély lába
 Rám rohanjon, és hatalma
 Letapodjon a földre;
 Adjad, hogy a hitetlenek
 Megessenek, süllyedjenek,
 Fel se keljenek többé!

;A gonosz szerencséje hiábavalóság
;Bourgeois L., Lyon, 1547
>37
/1
#F6CD0649
 Ne boszszankodjál a gonosztévőkre,
 Midőn őnékik jól vagyon dolguk!
 Ne nézz búskodva ő szerencséjökre,
 Ha látod nékik jó állapotjuk!
 Mert mint a szénafű, levágattatnak,
 És mint a zöld fű, hamar elhullnak.
/2
#D8748EC5
 Tégy jól és bízzál erősen Istenben:
 Békével élhetsz itt ez országban:
 Hűséggel járj el egész életedben,
 Örvendj az Úrnak nagy jóvoltában,
 És valamit kérsz tőle, mind megnyered,
 Mindent megád, amit kíván szíved.
/3
#B5C0E752
 Csak az Istenre támaszd minden dolgod,
 És kétség nélkül bízzál őhozzá,
 Mert megcselekszi, bizonnyal meglátod:
 Ártatlanságod világra hozza,
 Hogy igazságod úgyan lássa minden,
 Mint a fényes nap fénylik délszínben.
/4
#9D43441F
 Bízzál az Úrban, csendes légy szívedben,
 És reménységed vessed őbenne!
 Ne haragudjál jó szerencséjeken
 Azoknak, akik élnek kedvükre!
 Ne gondolj semmit az ő életükkel,
 Hogy velük együtt bűnbe ne ess el!
/5
#CE354560
 Mert a gonoszok mind eltöröltetnek,
 De akik a nagy Istenben bíznak,
 E földnek azok örökösi lesznek:
 A gonosztévők szörnyen elhullnak.
 Majdan, ha ő helyüket megtekinted,
 Aholott laktak, üresen leled.
/19
#307932A8
 Élj igazán, légy hű és tökéletes,
 És nagy jól lészen tenéked dolgod,
 Békességed lészen nagy örvendetes.
 A gonoszok mind vallnak csúfságot,
 Mert ők szertelen nagy ínségbe esnek,
 És teljességgel végre elvesznek.
/20
#63D99E8F
 Mert az Úr oltalma az igazaknak,
 Megmenti őket sok ínségükből,
 Vélük vagyon és tőle megtartatnak.
 És hogy őhozzá fordulnak szívből,
 A gonosztévőktől megszabadítja,
 És jelenvoltával vigasztalja.

;Lelki-testi nyomorúságban (Harmadik bűnbánati zsoltár)
;Bourgeois L., Genf, 1542
>38
/1
#56AF7B5A
 Haragodnak nagy voltában
 Megindulván,
 Ne feddj meg, Uram, engem!
 Búsult gerjedezésedben
 Rám tekintvén,
 Ne büntess meg Istenem!
/2
#B24C9E79
 Nyilaid belém lövettek,
 Mik szereznek
 Énnekem nagy sérelmet;
 Kezeidnek súlyossága
 Hátam nyomja,
 És sanyargat engemet.
/3
#E4D26486
 Testemnek semmi részében
 Épség nincsen
 Haragodnak miatta;
 Nincs békesség tagjaimban,
 Csontjaimban
 Bűneimnek miatta.
/4
#9B641A1C
 Mert az én nagy gyarlóságim
 És bűneim
 Fejem felülhaladták,
 Miknek nehéz, terhes voltát,
 Súlyosságát
 Tagjaim nem bírhatják.
/9
#3FDD81DE
 Minden mostani kérésem,
 Én Istenem,
 Vagyon szemeid előtt,
 És minden fohászkodásom,
 Óhajtásom
 Tőled el nem rejtetett.
/10
#95A4CC67
 Szívem nyugalmat nem lelhet,
 Igen reszket,
 Minden erőm elfogyott;
 Szemeim világossága,
 Vidámsága
 Éntőlem eltávozott.
/15
#22CB9B4E
 De én Istenemben bízom,
 És elvárom,
 Hogy kérésem meghallja,
 Mert szívem hozzá emelem
 És elhiszem,
 Hogy szükségem meglátja.
/18
#D6E641A0
 Hamisságomat megvallom,
 Nem tagadom
 Gonosztéteményimet;
 Bűneim miatt lett sebek
 Kesergetnek
 És gyötrenek engemet.
/21
#467D189A
 Uram, ne hagyj el engemet!
 Nézd ügyemet,
 Egyedül mint hagyattam!
 Kérlek, légy irgalmas nekem,
 Én Istenem,
 Mert csak tebenned bíztam!
/22
#CB5F7BDD
 Azért tőlem ne állj messze,
 Szánj meg végre,
 Én kegyelmes Istenem!
 Segedelmeddel ne késsél,
 Siess, jöjj el,
 Én édes idvességem!

;Imádság a lélek nyugalmáért
;Bourgeois L., Genf, 1551
>39
/1
#9C39661B
 Magamban elvégezém, és mondám:
 Hogy dolgom megtartóztatnám,
 Hogy nyelvem oly igét nem ejtene,
 Mely énnekem bút szerzene,
 Én szájamra zabolát vetettem,
 Míg a hitlen áll előttem.
/2
#E4164AB9
 Én mint a néma, veszteg hallgaték,
 Még a jóról sem beszélék,
 Sőt fájdalmam is titkolnom kelle,
 Min sérelmem öregbüle;
 Ég vala szívem, hogy meggondolám,
 Eltüzesülvén ezt mondám:
/3
#C11B167A
 Mutasd meg, Uram, éltemnek végét,
 És meddig éltetsz engemet?
 A napok számát jelentsd meg nekem!
 Míg e világon kell élnem,
 Mert időm nálad csak egy arasznyi,
 Előtted életem semmi.
/4
#A398F13D
 Bizony mulandó semmi az ember,
 Ki magának sokat ígér!
 Mint az árnyék, az ember elmúlik,
 Mégis szorgalmatoskodik,
 Sokat gyűjt és sok kincset rak össze,
 Nem tudja, kié lesz végre.
/5
#33C04A60
 Uram, hát nékem kiben kell bíznom?
 Nincs kívüled vigasságom!
 Ments ki engemet minden vétkemből,
 És a bolondok nyelvétől
 Őrizz meg, hogy ők ne csúfoljanak,
 Midőn ez ínségben látnak.
/6
#01C24591
 Mint a néma, hallgatok erősen,
 Szájam fel sem nyitom, mert én
 Tudom, hogy ezt mind te cselekedted.
 Ostorod rólam elvegyed!
 Mert kemény kezed nagy volta miatt
 Minden életem ellankadt.
/7
#14CC12FE
 Mert midőn te megfedded az embert
 Az ő nagy gyarlóságáért,
 Azonnal elvész szép ábrázatja,
 Mint a molytól a szép ruha.
 Lám, az ember mely igen mulandó,
 Semmi dolga nem állandó.
/8
#DEE3DAE9
 Hallgasd meg, Uram, könyörgésemet,
 Kérésemre ne légy siket!
 Mert előtted vendég és zarándok,
 Mint atyáim, olyan vagyok.
 Szűnjél meg tőlem, hadd vegyek erőt
 Az én kimúlásom előtt!

;A hit nemes harca
;Bourgeois L., Genf, 1551
>40
/1
#861E3D28
 Várván vártam a felséges Urat,
 És íme, hozzám fordula,
 Kegyelmesen meghallgata,
 És rajtam megmutatá jó voltát.
 Kivőn a mély veremből,
 És a sáros fertőből,
 És én lábaimat
 Szép egyenes kőre
 Elfelhelyeztette,
 Vezérlvén utamat.
/2
#D1039475
 Ada én számba új énekeket Istenünk dicséretire,
 Hogy akik hallgatnak erre,
 Higgyék és féljék ő Istenüket.
 Boldog, aki az Úrban
 Bízik, szemét elhajtván
 A kevély népektül,
 Kiknek minden dolgok
 Hazugságra hajlók:
 Tőlük távol kerül.
/3
#3FAB8440
 Csudatételidnek sok ő száma,
 És nagy bölcs gondolatidnak,
 Hozzánk való jóvoltodnak
 Sokságát senki meg nem mondhatja.
 Ha elkezdem számlálni,
 Nem tudom kimondani,
 Mert te nem kívántad
 A sok áldozatot,
 De hogy fogadjak szót,
 Fülemet alkottad.
/4
#F1787A68
 Égő áldozat nincsen kedvedben,
 A bűnért valók sem kellők.
 Akkor mondom: ím, eljövök,
 Rólam írás van a törvénykönyvben:
 Hogy akaratod tegyem,
 Én kegyelmes Istenem!
 A te törvényedben
 Gyönyörködik lelkem
 És örvendez szívem
 A te szent igédben.
/5
#1036D2D8
 Nagy sok népeknek seregeiben
 Te igazságod hirdettem,
 Meg nem tartóztattam nyelvem;
 Jól tudom, hogy én minden időben
 Jóvoltod magasztaltam,
 Soha el nem hallgattam
 Te idvességedet;
 Hűséggel hirdettem,
 Mindennek beszéltem
 Nagy kegyességedet.
/6
#60C55453
 Uram, én tőlem irgalmasságod
 Ne vond meg, tarts meg kegyesen,
 És igazságod őrizzen:
 Számtalan gonosz reám áradott!
 Sok nyavalya körülvett,
 Nagy sok ínség reám jött,
 Melynek nincsen száma,
 Mint hajam szálinak,
 Mik fejemen vannak;
 Szívem is elhagyja.
/8
#FA925CD7
 Már tebenned mind örvendezzenek,
 Akik keresnek tégedet,
 Kívánják idvességedet,
 Mondván: dicsőség légyen Istennek!
 Noha én szegény vagyok,
 És én szükségim nagyok,
 De rám gondot visel
 Az Úr, én megtartóm,
 Jövel, szabadítóm,
 Úr Isten, ne késsél!

;Hűtelen barátok ellen
;Bourgeois L., Genf, 1551
>41
/1
#851C2786
 Boldog, aki a nyavalyást híven
 Szánja ínségében,
 Mert szükségében őtet ismétlen
 Megmenti az Isten,
 Megtartja éltét, és ez országba'
 Lesz jó állapotja,
 Ellenséginek kívánságába,
 Nem adja markába.
/2
#D2FEEEAA
 Fájdalmában az Isten megtartja,
 Szépen felgyógyítja;
 Betegágyát fordítja örömre,
 És jó egészségre.
 Azért így szólok neked, Istenem:
 Kegyelmezz meg nekem!
 Gyógyítsd meg, Uram, én betegségem,
 Mert igen vétkeztem.
/6
#A9AABED9
 Tisztaságomban engem megtartasz,
 És megszabadítasz,
 És szemeid eleibe állatsz,
 Örökké el nem hagysz.
 Áldott légy, Izráelnek Istene,
 Most és mindörökké!
 Szent neved dicsértessék mindenben,
 Ámen és úgy légyen!

;Óhajtozás Isten után
;Bourgeois L., Genf, 1551
>42
/1
#A051EAB3
 Mint a szép, híves patakra
 A szarvas kívánkozik,
 Lelkem úgy óhajt Uramra,
 És hozzá fohászkodik,
 Tehozzád, én Istenem,
 Szomjúhozik én lelkem,
 Vajon színed eleiben
 Mikor jutok, élő Isten?
/2
#D4F8F93D
 Könnyhullatásim énnékem
 Kenyerem éjjel-nappal,
 Midőn azt kérdik éntőlem:
 Hol Istened, kit vártál?
 Ezen lelkem kiontom,
 És házadat óhajtom,
 Hol a hívek seregében
 Örvendek szép éneklésben.
/3
#13B71410
 Én lelkem, mire csüggedsz el?
 Mit kesergesz ennyire?
 Bízzál Istenben, nem hágy el,
 Kiben örvendek végre,
 Midőn hozzám orcáját,
 Nyújtja szabadítását;
 Ó, én kegyelmes Istenem,
 Mely igen kesereg lelkem!
/4
#1EA998C7
 Mert terólad emlékezem
 E Jordánnak földéről,
 Szent helyedre igyekezem
 Hermon s Micár hegy mellől.
 Mélység kiált mélységet,
 Midőn én fejem felett
 A sok sebes víz megindul,
 Mint egy erős hab, megzúdul.
/5
#34B99EAF
 Sebessége árvizednek,
 És a nagy zúgó habok
 Énrajtam összeütköznek,
 Mégis hozzád óhajtok;
 Mert úgy megtartasz nappal,
 Hogy éjjel vigassággal
 Dicséreteket éneklek
 Néked, erős őrizőmnek.
/6
#26BFFA99
 Mondván: Isten, én kőszálam,
 Mire felejtesz így el?
 Ellenségim vannak rajtam,
 Gyászban járok veszéllyel.
 Mert az ő hamis nyelvek
 Csontjaimban megsértnek,
 Mert így bosszantnak ellened:
 Lássuk, hol vagyon Istened?
/7
#EBFCC30E
 Én lelkem, mire csüggedsz el:
 Mit kesergesz ennyire?
 Bízzál Istenben s nem hágy el,
 Kiben örvendek végre.
 Ki nekem szemlátomást
 Nyújt kedves szabadulást,
 Nyilván megmutatja nekem,
 Hogy csak ő az én Istenem.

;(Folytatás)
;Bourgeois L., Lyon, 1547
>43
/1
#09B78A48
 Ítélj meg engemet, Úr Isten,
 És fogadd fel én ügyemet
 E kegyetlen nemzetség ellen!
 A hamis embernek kezében
 Ne bocsáss, Uram, engemet,
 Tartsd meg én fejemet!
/2
#810486AA
 Uram, engem miért hagyál el?
 Lám, te vagy én erősségem!
 Miért járok keserűséggel?
 Minden örömöm távozék el,
 Mert nyomorgat ellenségem
 És sanyargat engem.
/3
#DEACEECB
 Igazságodat add értenem,
 Világosságod küldd alá,
 Mely megvilágosítson engem!
 Szent hegyedre légyen vezérem;
 Bémenésem igazgassa
 A te hajlékodba!
/4
#F0AEEFC8
 Isten oltárához bemégyek
 Az én Uram eleiben,
 Aki öröme én szívemnek.
 Hegedűvel neked éneklek,
 És hálát adok szüntelen
 Tenéked, Úr Isten.
/5
#2A4339C2
 Miért vagy szomorú, én lelkem?
 Mit kesergesz ily szertelen?
 Bízzál az Istenben, mert hiszem,
 Hogy őtet én még dicsőítem,
 Midőn híven megment engem
 Megváltó Istenem.

;Ne taszíts el örökre!
;Bourgeois L., Genf, 1551
>44
/1
#099C9996
 Hallottuk, Isten, füleinkkel,
 Amit régenten cselekedtél,
 Nékünk atyáink mondották,
 Kik nagy dolgaidat látták;
 A pogány népet kezeddel
 Elvesztéd, földét elpusztítád;
 Néped más helyre vitted el,
 Holott ismét megszaporítád.
/2
#65BDC62E
 Mert nem ő fegyverük által lett,
 Hogy ők megülték e jó földet;
 Nem az ő kezük, sem karjuk
 Volt nekik szabadítójuk,
 De te orcád tekintése
 És a te karod és jobb kezed őket így megsegítette,
 Mert őhozzájuk volt jó kedved.
/3
#30D9BAF4
 Úr Isten, te vagy én királyom
 És az én teljes vigasságom!
 Jákóbnak küldd segedelmed,
 Amint régenten mívelted.
 Általad ellenséginket
 Megökleldezzük és megrontjuk,
 És a mi gyűlölőinket
 A te nevedben letapodjuk.
/4
#B90BB2C7
 Mert én nem bízom kézívemben,
 Sem az én éles fegyveremben;
 Az engem meg nem szabadít,
 Ha az ellenség megszorít,
 De te tartasz meg bennünket
 Minden mi ellenségünk ellen,
 És a mi kergetőinket
 Elveszted, és ejted szégyenben.
/5
#E30D8F88
 Azért az Istent magasztaljuk,
 És szent nevét örökké áldjuk;
 Mindennap dicsérvén őtet,
 Hirdessük nagy kegyességét!
 De minket te megvetettél,
 És juttattál nagy szégyenségben,
 A hadba velünk nem jövél,
 Hogy megtartottál volna épen.
/6
#69CE2EBE
 Az ellenségtől elfuttattál,
 És velünk nagy szégyent vallattál.
 Minden marhánkat eldúlják,
 Kik éltünket háborgatják.
 Hogy minket ők megegyenek,
 Mint a juhokat, úgy adál el,
 Kik minket messze kergetnek,
 A pogányok közibe széjjel.
/7
#D68597F3
 Népedet te semminek tartád,
 És őket csak olcsón eladád,
 Igen kevésre becsüléd,
 Csaknem ingyen odavetéd.
 Azt tőd, hogy mi ellenségink
 És akik mi környülünk laknak,
 Mindenfelől mi szomszédink
 Minket nevetnek és csúfolnak.
/8
#899F13B1
 A pogányok példabeszédet,
 Szólnak mirólunk nevetséget,
 És mindenféle nemzetek
 Csúfolván, fejükkel intenek;
 Gyalázat, szitok szüntelen,
 És szégyen forog én előttem,
 Úgy, hogy nagy szégyenletemben
 Az én orcámat be kell fednem.
/9
#18C837C0
 Nagy sok szidalmat kell hallanom,
 Mely miatt csak elszomorodom,
 Midőn szemem előtt nézem
 Bosszúálló ellenségem.
 Mindezek esnek mirajtunk,
 Mégis nem felejtünk el téged,
 Kötésed ellen nem járunk,
 Mindenben engedünk tenéked.
/10
#69C5A4A7
 A mi szívünk nem fordul vissza,
 De gondolatit rajtad tartja.
 A mi lábunk ki nem mozdul,
 Ösvényedről el nem hajol.
 Mégis így büntetsz bennünket,
 Bevetél a sakálok közé,
 És már mindenfelől minket
 Halál árnyéka környékeze.
/11
#C3526286
 Ha elfeledtünk volna téged,
 Meg sem gondoltuk volna neved;
 Ha más isten-szolgálatra
 Kezünket emeltük volna:
 Nyilván nem tűrted volna el,
 Sőt igen megbüntettél volna,
 Mert mindent látsz te szemeddel,
 Minden szívnek előtted titka.
/12
#43D2F7D0
 De mi teéretted naponkint,
 Üldöztetünk e világ szerint;
 Miként az ártatlan juhok,
 Kik a mészárszékre valók.
 Kelj fel azért, mit aluszol?
 És álmodból már serkenjél föl,
 Támadj föl, és hatalmadból
 Ments ki minket e nagy ínségből!

;Az Isten Felkentjének menyegzője (dicsősége)
;Bourgeois L., Genf, 1551
>45
/1
#EF5B6C1D
 Egy szép dolgot hoz elő az én szívem,
 A dicső királyról lesz éneklésem,
 Kit nyelvem dicsér nagy szép felszóval,
 Mint egy deák az írópennával.
 Sokkal szebb vagy te minden embereknél,
 Mindent felülmúlsz te szép termeteddel,
 Ajakidnak nagy ő kedvessége,
 Mert megáldott az Isten örökké.
/2
#78E5874E
 Te kegyes, erős vitéz, készüljél fel,
 Vedd fegyvered és oldaladra kösd fel,
 Úgy, mint királyi ékességedet,
 Ez ékességben végy győzedelmet!
 A jó igazság vezérlje utadat,
 Jóság, kegyesség bírja járásodat,
 És a te karodnak erejével
 Nagy csudákat téssz, megládd szemeddel!
/3
#23C4D9F8
 Mert hegyesek a te sebes nyilaid,
 Mit megéreznek minden ellenségid,
 Szívüket midőn általszegezed,
 És őket hatalmad alá veted.
 Ó, Úr Isten, a te királyi széked,
 Megmarad mindörökké dicsőséged;
 Pálcája a te királyságodnak:
 Pálcája bizony igazságodnak.
/4
#C9EA9C2B
 Az igazságot te igen szereted,
 A hamisságot viszontag gyűlölöd,
 Azért az Isten mindenek felett
 Víg olajjal megkenett tégedet.
 A te nevedet mindenkor hirdetem,
 Nemzetségről nemzetségre beszélem:
 És minden maradéki ezeknek
 Mindörökké tégedet dicsérnek.

;Erős vár a mi Istenünk!
;Bourgeois L., Genf, 1551
>46
/1
#8EE5EC53
 Az Isten a mi reménységünk,
 Midőn reánk tör ellenségünk,
 Minden háborúságinkban
 Megtart erős hatalmában.
 Azért a mi szívünk nem félne,
 Ha az egész föld megrendülne,
 És a hegyek a tengernek
 Közepibe bedűlnének.
/2
#6D020AF1
 Ha a tenger szörnyen zúgna is,
 Minden víz felzavarodnék is,
 És ha a sebes szélvésszel
 A hegyek hányatnak széjjel:
 A szép folyóvíz mindazáltal
 Az ő szép tiszta folyásával
 Az Istennek szent városát,
 Megvigasztalná hajlékát:
/3
#B5D5FB11
 Mert közepén lakik az Isten,
 Azértan romlása nem lészen;
 Semmi ínségbe nem ejti,
 Az Isten jókor megmenti.
 A pogány népek dúlnak-fúlnak,
 Nagy sok országok feltámadnak,
 De az ő haragos szava
 Mind e földet elolvasztja.
/4
#D4FBFAB0
 De az Isten minden időben
 Mivelünk vagyon ínségünkben;
 Jákób Istene oltalmunk,
 A Zebaóth erős várunk!
 Jertek, lássátok e nagy Úrnak,
 Csuda dolgait hatalmának,
 Ki mind e föld kerekségét,
 Elpusztítja ékességét!
/5
#36F2E12B
 E földön széjjel nagy hadakat,
 Ő megcsendesít háborúkat,
 Ívet, kopjákat megrontat,
 Társzekereket felgyújtat,
 Így szólván: mindnyájan halljátok,
 Hogy erős Istenetek vagyok,
 És hogy birodalmam vagyon
 Minden népen e világon!
/6
#1822207E
 Summa szerint: az erős Isten
 Velünk van minden ínségünkben;
 Jákób Istene oltalmunk,
 A Zebaóth erős várunk!

;Isten a világ királya
;Bourgeois L., Genf, 1551
>47
/1
#0A93251C
 No, minden népek,
 Örvendezzetek
 És tapsoljatok,
 Istent áldjátok
 Szép hangossággal
 És nagy felszóval,
 Mert az Úr Isten
 Nagy felségében
 Ő királysága
 És nagy országa
 Kiterjed meszsze,
 Ez egész földre.
/2
#5F011947
 Hatalmunkba vet
 Nagy sok népeket,
 És meghódoltat sok pogányokat
 Minékünk végre
 Ő nagy ereje.
 Minket ő felvett,
 Örökévé tett,
 Nékünk engedte
 Ő kegyessége
 Jákóbnak tisztét,
 Kit igen szeret.
/3
#3D6DE3E3
 Íme, az Isten
 Szépen felmegyen
 Nagy vigasságban,
 Trombitaszókban;
 Urunk felmegyen
 Nagy dicsőségben.
 Énekeljetek
 Az Úr Istennek
 Zengő verseket,
 Szép énekeket,
 E nagy királynak,
 Mint mi Urunknak!
/4
#C78C1739
 Őnéki minden
 Jól énekeljen!
 Mert minden népek
 Néki engednek.
 Királyi székben
 Ül nagy kegyesen.
 A fejedelmek
 Őhozzá gyűlnek,
 Mint Ábrahámnak
 Istenét, áldják,
 Alázatosan
 Néki szolgálván.
/5
#EF2466D3
 Ez erős Isten Úr mindeneken,
 Népét megőrzi
 És jól vezérli.
 Nagy dicsősége,
 Melynek nincs vége.

;Sion zarándokainak éneke
;Genf, 1562
>48
/1
#A1A99440
 Nagy az Úr méltóságában,
 Az Isten szent városában,
 Holott lakozik dicsősége,
 És dicsértetik ő szent neve.
 A szent Sionnak hegyén,
 Annak északi szélén,
 E gyönyörűséges helyen,
 Holott helyheztetett szépen
 A nagy királynak városa,
 Melynek nincs e földön mása.
/2
#A9FAC884
 Itt jól esmérik házanként,
 Mint menedéket, az Istent,
 Holott királyok egybegyűltek,
 És ők sereggel előjöttek;
 Erősen megszállották,
 Hogy megvegyék, elszánták.
 De ottan megrettenének,
 Megfutamodni kezdének,
 Város mellől elállának,
 Nagy rettegve szaladának.
/3
#679EE222
 Nagy félelme lőn szívüknek,
 Mint gyermekszűlő személynek,
 És mint a hajó a szélvészben,
 Ha napkeleti szél fú szörnyen.
 Ezt mi szemünkkel látjuk,
 Mint azelőtt hallottuk,
 Városában az Istennek,
 Melyet őneki szenteltek,
 Mit magának foglalt Isten,
 Hogy néki szolgáljon minden.
/4
#ED75086E
 Mit megerősített Isten,
 Hogy romlása ne lehessen.
 Nézünk itt te kegyességedre
 Szent templomodnak közepette.
 Dicsőséges szent neved
 Mind e földre kiterjed;
 Ekképp a te dicséreted
 Nagy messzire kiterjeszted;
 Jobbod teljes igazsággal,
 Mindenben irgalmassággal.
/5
#4701C64D
 Örül a Sionnak hegyi
 És a Judának leányi,
 Te igaz ítéleteiden
 Szíve szerint örvendez minden.
 A Siont járjátok meg,
 Tornyait lássátok meg,
 Nézzétek meg ő bástyáit,
 Szépen épített házait,
 Hogy ezt megbeszélhessétek
 A jövendő nemzedéknek.
/6
#6DFFE24A
 Mert igaz és hű az Isten,
 Ki minket minden időben
 Megtart, és vezérli éltünket,
 Mindhalálig őriz bennünket.

;A gazdagságban bízni balgaság
;Genf, 1562
>49
/1
#FF00AA3D
 Hallgassátok meg, minden nemzetek,
 E föld lakói, jól figyeljetek,
 Köznépek és a főrenden valók,
 Minden szegények és a gazdagok!
 Az én szájam szól nagy bölcses-séget,
 És elmém gondol jó értelmeket;
 E példára magam is figyelmezek,
 Hegedűszóban szép mesét jelentek.
/2
#392C614F
 Mit félnék én a gonosz időben,
 Hogy nyomorgatóm engem elejtsen,
 Ha ellenségem mind azon vagyon,
 Hogy engemet láb alá tapodjon?
 Némelyek igen bíznak pénzükben,
 És dicsekednek az ő kincsükben;
 De senki meg nem váltja atyjafiát,
 Nem adhatja meg Istennél váltságát.
/3
#876D6901
 Mert drága a léleknek váltsága,
 Életét senki meg nem válthatja,
 Hogy a halált ő elkerülhesse,
 És a sírba menni ne kellene.
 Mert látja, hogy sem bolond, sem eszes,
 A halál ellen nincsen mentséges,
 És holtuk után minden gazdagságuk,
 Más emberre száll az ő sok jószáguk.
/4
#4E370F84
 Ez őnékiek fő gondolatjuk,
 Hogy mindörökké tartson szép házuk,
 Hogy el ne fogyjon az ő nemzetük,
 És megmaradjon örökké nevük.
 Mert noha vagyon pénzük és tisztük,
 De nem sokáig tartnak mindezek,
 Mert végre minden jóktól elszakadnak,
 Mint oktalan állatok, ők meghalnak.
/6
#32AC14A5
 Mert oda lészen minden ő dolguk,
 A koporsó lészen ő hajlékuk,
 De a haláltól engem az Isten
 Megment, és hozzá vészen kegyesen.
 Annak okáért azzal ne gondolj,
 Hogy némely ember igen gazdagul,
 Mert minden kincset másoknak kell hagyni,
 A dicsőségtől meg kell néki válni.

;Az igazi hálaáldozat
;Bourgeois L., Lyon, 1547
>50
/1
#D0A90AA7
 Az erős Isten, uraknak Ura,
 Szól és e földet mind elő hívja,
 Támadatról és napenyészetről,
 Nagy szépséggel a Sion hegyéről
 Eljő az Isten ő fényességében,
 Semmit el nem hallgat ítéletében.
/2
#9FC8BD4E
 Emésztő tűz mégyen őelőtte,
 Nagy forgó szélvész lészen körüle;
 Szólítja az eget és a földet,
 Hogy megítélje minden ő népét,
 Mondván: gyűjtsétek ide a híveket,
 Kik áldozattal vették kötésemet!
/3
#6E9D95DD
 Az egek hirdetik igazságát
 Az Istennek, mert ítél igazat.
 Én népem, hallgasd meg, szólok neked,
 Mert bizonyságot teszek ellened,
 Én tenéked erős Istened vagyok,
 Áldozatiddal keveset gondolok.
/6
#17131810
 Szükségben tőlem segítséget kérj,
 Én megsegítlek, hogy engem dicsérj!
 A gonosznak mond Isten: de mire
 Vészed törvényem a te nyelvedre?
 Kötésem szájjal vallod, de gyűlölöd
 Intésemet, és igémet megveted.
/7
#09464029
 És mikoron a lopót te látod,
 Együtt futsz véle, dolgát javallod;
 A paráznákkal örömest mulatsz,
 Rossz társaságnak gyakran helyet adsz;
 Szájaddal szólasz nagy sok gonoszságot,
 Nyelveddel szerzesz sok háborúságot.
/8
#F77D13B7
 Leülsz és atyádfiát megszólod,
 Az anyád fiát is rágalmazod;
 Ezt míveled, de én csak hallgatok.
 Azt véled, én is csak olyan vagyok,
 Mint szintén te, de téged előveszlek,
 És szemlátomást bűnödről megfeddlek.
/9
#E45B36FD
 Ezt mostan eszetekbe vegyétek,
 Kik az Istenről elfelejtkeztek,
 Hogy mentség nélkül el ne rántsalak!
 Az becsül engem, aki hálát ad;
 Mond Isten: az ily ember jár jó úton,
 És segedelmem néki jelen vagyon!

;Dávid bűnbánati imádsága (Negyedik bűnbánati zsoltár)
;Bourgeois L., Genf, 155
>51
/1
#2F2524A8
 Úr Isten, kérlek, kegyelmezz nékem,
 És kegyelmedből könyörülj énrajtam,
 Bűneimből tőled megtisztíttassam,
 Nagy irgalmaddal végy körül engem!
 Törüld el az én nagy bűneimet,
 Mosogass jól meg fertelmességimből,
 Amikkel fertéztetem éltemet:
 Tisztítsad el nagy kegyelmességedből!
/2
#D77F2C90
 Mert ismerem jól gyarló voltomat,
 És bűnöm mindenkor forog előttem,
 Melyet csak teellened cselekedtem,
 Mely miatt tészek én nagy siralmat.
 Vétkeztem a te szemeid előtt,
 Melyért engemet méltán megbüntethetsz,
 Rám vethedd kemény ítéletedet,
 Mindazonáltal igazán ítélhetsz.
/3
#80AECB8B
 Mert íme, látom, nyilván jól értem,
 Hogy én a gyarlóságban fogantattam,
 Vétekben az én anyámtól származtam,
 Szüleim bűnös véréből lettem.
 De te az igazságot szereted,
 És igen kedveled a tiszta szívet,
 És a te titkos bölcsességedet
 Nagy kegyelmesen nékem megjelented.
/4
#EB2A8406
 Hints meg, Uram, engemet izsóppal,
 És én azontúl tisztán megújulok;
 Moss meg engem, és szépen megtisztulok,
 És fejérb lészek a tiszta hónál.
 Hogy örvendezhessen az én szívem,
 Add, hogy megértsem te nagy irgalmadat,
 És megvidul minden én tetemem,
 A te haragod melyet összerontott.
/5
#06930EFD
 Színedet rejtsd el vétkeim elől,
 Fertelmes bűneimet ne tekintsed,
 Haragos orcád énrólam elvégyed,
 Tisztíts meg engem minden bűnömből!
 És teremts tiszta szívet énbennem,
 Dolgomat mindig jóra vezéreljed,
 Az erős lelket újítsd meg bennem,
 Hogy kedves légyen életem előtted!
/6
#1495DD12
 Ne vess el engem szent színed elől,
 És ne vedd el szent Lelkedet éntőlem,
 Sőt szerezz teljes örömet énbennem,
 Tégy bizonyossá engedelmedről!
 Indíts szívemben nagy vigasságot,
 És engem vidám lélekkel erősíts,
 Értsem örömmel nagy irgalmadat,
 Kegyelmességgel engemet bátoríts!
/7
#E56D6E83
 És példa lészek erről mindennek,
 A bűnösöket utadra tanítom,
 Bűnükből hozzád térésre indítom,
 Hogy csak tebenned reménykedjenek.
 Ó, én Istenem, én idvességem,
 Szabadíts meg e vérrel buzgó bűntől,
 Hadd énekeljen örökké nyelvem
 Te igazságos szent ítéletedről!
/8
#E90A43F0
 Nyisd meg azért az én ajakimat,
 Hogy az én szájam dicsérhessen téged!
 Ha az áldozat kedves volna néked,
 Nem kíméleném áldozatomat,
 De nem kell néked égő áldozat:
 Az alázatos lelket te szereted,
 Az tenálad a kedves áldozat;
 A töredelmes szívet meg nem veted.

;Jaj a gonosz nyelvnek!
;Genf, 1562
>52
/1
#5C91FD6C
 Mit dicsekedel gonoszságban,
 Te hatalmaskodó?
 Mit fuvalkodol fel magadban,
 Nagyravágyakozó?
 Mert az Istennek jóvolta
 A jókat megtartja.
/7
#B400E403
 Én téged örökké dicsérlek,
 Mert megtartál engem,
 És a te nevedben remélek,
 Míg leszen életem,
 Mert jó a te híveiddel
 Lakoznom örömmel.

;A balgák megbűnhödnek
;Bourgeois L., Genf, 1542
>53
/1
#A4226944
 A bolond így szól az ő szívében:
 'Nincs Isten', azért nagy gonoszságban él;
 Utálatos bűnt teszen, semmit nem fél.
 Ez egész földön aki jót tegyen,
 Senki nincsen.
/2
#B5604213
 Az Úr az égből alátekinte
 E földön az emberek fiaira,
 Hogy meglássa, ha kinek esze volna,
 Ha valaki az Istent keresné
 És tisztelné.
/3
#25A23000
 De azt jól látja dicsőségében,
 Hogy a jó útról eltértek mindenek,
 Mindnyájan fertelmes bűnben hevernek;
 Aki az Istent tisztelné híven,
 Csak egy sincsen.
/6
#E29337E2
 A Sionról vajon ki jövend el,
 Ki a szent Izráelt megszabadítja?
 Ha Isten fogságból népét kihozza,
 Örvend a Jákób és az Izráel
 Teljes szívvel.

;A hit diadalma
;Genf, 1562
>54
/1
#0519C421
 Tarts meg, Uram, én Istenem,
 És szent nevedért védelmezz meg,
 Ártatlan ügyemet tekintsd meg,
 Hatalmaddal támadj mellém!
 Kérem te szent Felségedet,
 Hallgass meg én könyörgésembe'
 És kegyesen vedd füleidbe
 Az én szájamnak beszédét!
/2
#C72FD49D
 Mert ellenségim kevélyen
 Reám támadnak és kergetnek,
 És engem halálra keresnek,
 Eszükbe sem jut az Isten.
 De az Isten megtart engem,
 Kegyelmességét megmutatja,
 És segedelmét hozzám nyújtja,
 Jóra vezérli életem.
/3
#B1B67518
 Ez nékem szánt nagy nyavalyát
 Én ellenségimre téríti,
 És ezt nékiek megfizeti,
 Megmutatván igazságát.
 Akkoron neked víg szívvel
 Hálaadásokat áldozom,
 És szent nevedet magasztalom,
 Mert teljes vagy kegyességgel.
/4
#D2FD0F1D
 Mert minden én ínségemből
 Te megszabadítál engemet,
 Megbüntetéd ellenségimet,
 Mit én megláték szememmel.

;Könyörgés hamis atyafiak ellen
;Genf, 1562
>55
/1
#9025397B
 Hallgasd meg az én könyörgésem,
 Úr Isten, ne fordulj el tőlem,
 Imádságom vedd füleidbe,
 Mert nagy kínokat szenvedek,
 Szívemben igen kesergek,
 Előtted panaszlok reszketve!
/3
#D61097E2
 Kesereg szívem nagy ínségben,
 Élek halálos félelemben,
 Teljességgel elszomorodtam;
 Rettegek, félek, gyötrődöm,
 Reszketvén szörnyen vesződöm,
 Úgy, hogy immár sokszor kívántam:
/4
#8FB9BEE5
 Szárnyaim, ó, ha lehetnének,
 Mint a galamb, ha repülhetnék:
 Én elrepülnék messze földre,
 Elmennék e népek közül.
 Pusztát keresnék, ezektül
 Ahol nyugodalmam lehetne.
/10
#F4D791BF
 Mind este, reggel őt óhajtom,
 Délkorban is őtet kiáltom,
 És meghallgatja könyörgésem,
 Megtart engem békességben
 Minden ellenségim ellen,
 Kik seregbe gyűltek ellenem.

;Könyörgés üldöztetésben
;Genf, 1562
>56
/1
#778C22FB
 Kegyelmezz meg nekem, én Istenem,
 Mert az ember igen kerget engem,
 Nagy hatalommal támad ellenem,
 Hogy engem elejthessen.
 Sok ellenségim háborgatnak szörnyen,
 Elszánták, hogy bényeljenek hirtelen,
 Úr Isten, én ilyen nagy félelmemben
 Benned reménységem!
/2
#A7C7868C
 Én az Úr Istenben dicsekedem,
 Szent Igéjében nem kételkedem.
 Mit tehetne az ember énnekem,
 Aki kerget engemet?
 Visszafordítják az én beszédemet,
 És naponkint abban hányják eszüket,
 Hogy nékem szerezzenek veszedelmet,
 Csak gonoszt gondolnak.
/3
#29F46253
 Énellenem ők összejárulnak,
 Hogy megkapjanak, azon forgódnak,
 És életemtől megfoszthassanak:
 Ez minden ő szándékuk.
 A gonoszságban nagy ő bizodalmuk,
 Azt vélik, hogy jól leszen minden dolguk,
 De haragod ha esik, Uram, rajtuk:
 Őket mind levered.
/4
#685BCC68
 Minden futásimat megemlíted,
 Könnyhullatásim tömlődbe szeded:
 A te könyveidbe fel is jegyzed
 Minden én ínségemet.
 Könyörgésemet midőn hozzád nyújtom,
 Ellenségimet ottan futni látom,
 Mert velem vagy, és te vagy megtartásom,
 Kegyelmes Istenem!
/5
#6C211AB2
 Úr Isten, Fölségedet dicsérem,
 És szent igédet nagynak becsülöm;
 Áldom az Urat, míg lesz életem,
 Bízván ő beszédében.
 Reménységemet vetem az Istenben,
 Irgalmasságára nézek szüntelen,
 Azért félelmem senkitől nem leszen:
 Ki árthatna nékem?
/6
#F4E6C2A0
 Szent fogadásom tartja azt nékem,
 Hogy jóvoltodért neved dicsérjem,
 Mert kegyelmesen megmentéd lelkem
 A halál köteléből.
 És lábaimat megtartád eséstől,
 Hogy én élhessek tenéked kedvesül;
 Az élők fényes világában szentül
 Járjak teelőtted.

;Bizodalom veszedelemben
;Genf, 1562
>57
/1
#0EE4ABD1
 Irgalmazz, Uram, irgalmazz nekem,
 Mert tebenned bízik az én lelkem!
 Több segedelmem sehol nincsen nékem!
 Szárnyadnak árnyékában védelmem,
 Míg veszedelmem eltávozik tőlem.
/2
#DBBAA699
 Én a felséges Istenben bízom,
 Ki jóra vezérli minden dolgom;
 Segedelmet küld alá ínségemben.
 Az ellen lészen nékem oltalmam,
 Aki engem kerget, hogy benyelhessen.
/6
#33406A92
 Én nyelvem és lantom már serkenj föl,
 Véletek ím a hajnalt keltem föl!
 Jó reggel én is ágyamból fölkelek,
 És éneklek nagy dicsőségedről.
 Téged, Uram, minden népnek hirdetlek!
/7
#5CE338D1
 Mert nagy jóvoltod, örök irgalmad
 És igazságod az égig fölhat!
 Nagy dicsőséged láttasd meg az égen,
 És jelentsd meg nagy hatalmasságod
 Az embereknek mind az egész földön!

;Igaztalan ítélkezés ellen
;Genf, 1562
>58
/1
#59711B75
 Ti tanácsban ülő személyek,
 Kik zúgolódtok ellenem:
 Mondjátok csak meg énnekem,
 Ha igaz e, amit beszéltek,
 És ha igazat ítéltek,
 Ádámtól származott népek?
/2
#6EEBAF10
 Sőt ha ember jól megtekinti:
 Álnoksággal jár szívetek,
 Mér hamis fonttal kezetek;
 Nincs dolgotoknak tökéleti.
 A gonoszok eltévedtek,
 Mihelyt anyjuktól születtek.
/8
#CBEA115E
 Végre is azt mondhatja minden,
 Hogy jó az igaznak dolga,
 Sok és nagy az ő jutalma.
 Azt vegye mindenki eszében,
 Hogy Isten mindent megítél,
 Aki gonoszul vagy jól él.

;Könyörgés szorongattatásban
;Genf, 1562
>59
/1
#F1952ACC
 Szabadíts meg engem, Úr Isten,
 És tarts meg ellenségim ellen!
 Mentsd meg azoktól életem,
 Kik feltámadtak ellenem!
 Oltalmazz meg a hamis néptül,
 Ki én veszedelmemnek örül
 És szomjúzik ártatlan vért!
 Ments meg attól szent nevedért!
/2
#24D64F17
 Én ellenségimnek ereje,
 Kezedben minden tehetsége.
 Benned bízom, én Istenem,
 Mert te vagy én segedelmem.
 Az Isten jóvoltát jelenti,
 Nyavalyámnak elejét veszi,
 És megláttatja énvélem,
 Hogy elvész én ellenségem.
/5
#DA66260F
 Én ellenségimnek ereje,
 Kezedben minden tehetsége.
 Benned bízom, én Istenem,
 Mert te vagy én segedelmem.
 Az Isten jóvoltát jelenti,
 Nyavalyámnak elejét veszi,
 És megláttatja énvélem,
 Hogy elvész én ellenségem.
/10
#C7782203
 Mert te vagy, Uram, én oltalmam,
 Reménységem és bizodalmam:
 Azért, ó, én erősségem,
 Mindenütt neved dicsérem,
 Hogy nekem az én szükségemben,
 Segedelmem vagy ínségemben;
 Te vagy én erős kőváram,
 Kegyességed nagy énhozzám.

;Imádság nehéz időkben
;Bourgeois L., Genf, 1562
>60
/1
#627E9FAE
 Minket, Úr Isten, elhagyál,
 És reánk megharagudtál,
 Tőled széjjel eloszlatánk,
 Térj kegyesen ismét hozzánk!
 Mind e földet megindítád,
 Nagy hatalmaddal megrontád,
 Építsd meg az ő nagy romlását,
 Mert romlás miatt alig állhat.
/2
#DD75CDF9
 Népedet keményen tartád,
 És igen megsanyargatád;
 Minket mintegy csípős borral,
 Itatál keserű búval,
 De akik tégedet félnek,
 A zászlót adád nékiek,
 Hogy fölemelnék igazságban,
 Bízván a te szent oltalmadban.
/7
#3BE575D1
 Légy minékünk segítségül,
 Őrizz meg ellenséginktül,
 Mert az emberi segítség
 Hiábavaló epedség.
 Az Isten által minekünk
 Lészen erős győzedelmünk,
 És megszabadít ő bennünket,
 Megtapodja ellenségünket.

;Kiáltás Istenhez az idegenben
;Genf, 1562
>61
/1
#626E321C
 Kiáltásom halld meg, Isten!
 Vedd füledbe
 Az én könyörgésemet,
 Mert én szívem nagy ínségből,
 Meszsze földről
 Kiáltja Felségedet!
/2
#58D876FF
 Végy fel engemet kőszálra,
 Magasságra,
 Hol bátorságom legyen,
 Mert te vagy én erős tornyom,
 Vigasságom
 Én ellenségim ellen!
/3
#F5251B3B
 Te hajlékodban lakásom
 Én kívánom
 És óhajtom szüntelen;
 Szárnyaidnak árnyékába
 Kívánkozva
 Vagyok jó reménységben.
/4
#31690FA4
 Meghallgatsz fogadásomban
 Engem, Uram,
 Nyújtván kegyességedet.
 Örökségüket megadod,
 És megáldod,
 Akik félik nevedet.
/7
#C92261C7
 És aztán vígan éneklek
 Szent nevednek
 Mostan és mindörökké,
 És amely fogadást tettem,
 Megfizetem
 Naponként őnékie.

;Csendes bizodalom Istenben
;Bourgeois L., Lyon, 1547
>62
/1
#0C4172BE
 Az én lelkem szép csendesen
 Nyugszik csak az Úr Istenben,
 Mert csak ő az én idvességem.
 Ő nékem erős kőváram,
 Megtartóm és én oltalmam,
 Minden gonosztól megment engem.
/4
#F7E4C231
 Azért, szívem, reménységed
 Csak az Úr Istenben vessed,
 Élj csak az ő segedelmével!
 Ő nekem magas kőszálam,
 Oltalmam és erős bástyám,
 Hogy soha ne tántorodjam el.
/5
#A83107CA
 Az Isten én idvességem,
 Erősségem, dicsőségem;
 Bízzatok azért csak őbenne!
 Előtte ti szíveteket, töltsétek ki lelketeket,
 Ő legyen lelkünk hiedelme!
/7
#7B24FE63
 Ne bízzatok erőszakban,
 Hamis ragadozástokban,
 Mulandó dolgon ne kapjatok!
 Ha sokasul ti kincsetek,
 Ahhoz ne bízzék szívetek,
 Mert nem állandó gazdagságtok.
/8
#BB654908
 Az Isten egy szót szólt egyszer,
 Melyet én hallottam kétszer:
 Hogy nagy ereje vagyon néki.
 A kegyesség, Uram, tied,
 És te nyilván megfizeted,
 Akinek dolga mint érdemli.

;Vágyódás Isten és a szent hajlék után
;Bourgeois L., Genf, 1551
>63
/1
#52539826
 Isten, te vagy én Istenem,
 Jó reggel kereslek tégedet,
 Hozzád óhajtván, elepedett
 Szomjúság miatt én lelkem.
 Én testem hozzád áhítozik,
 Szomjúságban elhalt szintén
 E puszta és száraz földön,
 Hol semmi víz nem találtatik.
/2
#6CF81E1B
 Mert látni igen kívánja
 A te nagy erős dicsőséged,
 És isteni tiszteletedet
 A te dicső templomodba'.
 Mert nekem kedvesb életemnél
 A te nagy kegyelmességed,
 Melyért én ajakim téged
 Dicsérjenek szép énekléssel.
/3
#3A91B0AC
 Magasztallak én tégedet
 Életemnek minden rendiben,
 Én kezeimet fölemelvén
 Áldom te dicső nevedet.
 Örül, mintha drága étkekkel
 Jóllakott volna én szívem;
 Szent Fölségedet dicsérem,
 Éneklek rólad nagy örömmel.
/4
#FC93EEAF
 Rólad el nem felejtkezem,
 Még ágyamban is emlegetlek,
 És midőn reggel én fölkelek,
 Csak terólad emlékezem.
 Mert te énvelem sokszor jól től,
 És megszabadítál engem,
 Azért most is én életem
 Szárnyaid árnyékában örül.

;A gonoszok meglakolnak
;Bourgeois L., Genf, 1542
>64
/1
#4E0864FB
 Hallgasd meg, Uram, könyörgésem,
 Tarts meg ellenségem ellen,
 Aki reám dühödt szörnyen;
 Félelmétől őrizz meg engem,
 Mentsd meg életem!
/2
#59E068C1
 Rejts engem el a gonoszoktól,
 Akik reám fenekednek,
 Csak gonoszra igyekeznek;
 Tarts meg ő hamis tanácsuktól,
 Háborgásuktól!
/3
#A62574B0
 Akik nyelvüket élesítik,
 Mint öldöklő fegyvereket;
 Mint a nyilat, beszédüket
 Az ártatlan emberre lövik,
 És azt megsértik.
/8
#91854873
 Őket az önnön gonosz nyelvek
 Ejti a veszedelembe,
 És kik ezt veszik eszükbe,
 E dolgon igen megijednek
 És elrémülnek.
/9
#D1C59318
 És nagy félelemmel mindenek
 Hirdetik az Isten dolgát,
 Beszélik annak nagy voltát,
 Melyet innen eszükbe vesznek
 És megértenek.
/10
#C8F9A649
 De szíve igaz embereknek
 Örvend az erős Istenben
 És az ő nagy kegyelmében.
 Az igaz hívek örvendeznek
 És dicsekednek.

;Hálaadás lelki-testi jókért (Aratási ének)
;Bourgeois L., Strasbourg, 1545
>65
/1
#CFB3878D
 A Sionnak hegyén, Úr Isten,
 Tied a dicséret,
 Fogadást tesznek néked itten,
 Tisztelvén tégedet,
 Mert kérésüket a híveknek
 Meghallod kegyesen,
 Azért tehozzád az emberek
 Jönnek mindenünnen.
/2
#D5E190FF
 Rajtam a bűn elhatalmazék,
 Terhelvén engemet,
 De nagy volta kegyességednek
 Eltörli vétkünket.
 Boldog, akit te elválasztál,
 Fogadván házadba,
 Hogy előtted nagy buzgósággal
 Járjon tornácodba.
/3
#52341F24
 Javaival a te házadnak
 Megelégíttetünk,
 Szép dolgain te templomodnak
 Gyönyörködik szívünk.
 A te csuda igazságodból
 Megfelelsz minekünk,
 Hallgass meg, Isten, velünk tégy jól,
 Ó, mi segedelmünk!
/4
#ABB1AB28
 Mindenek csak tebenned bíznak
 E föld kerekségén,
 Akik széjjel messze lakoznak
 A tengernek szélén.
 Te mondhatatlan hatalmaddal
 A magas hegyeket
 Körülfogod, mintegy abronccsal,
 Erősítvén őket.
/5
#5A6E0346
 Tengeri habok nagy zúgását
 Te megcsendesíted,
 A pogány nép zúgolódását
 Ottan elenyészted.
 Nagy félelmükben elbágyadnak
 Mindenek e földön,
 Nagy voltán a te csudáidnak,
 Miknek számuk nincsen.
/6
#6A2B7EB8
 Te megvígasztalsz mindeneket
 Reggel a napfénnyel,
 Biztatsz minden élő rendeket
 Csillagokkal éjjel.
 Áldásiddal meglátogatod
 Az elszáradt földet,
 Hasznos esőkkel meglágyítod,
 Gazdagítván őtet.
/7
#C05B8463
 A te kútaidból a vizek
 Soha el nem fogynak,
 Hogy a szép földi vetemények
 Szaporodhassanak.
 A barázdákat megitatod
 A szántóföldeken,
 A vetést szép esővel áldod,
 Hogy bőven teremjen.
/8
#73750021
 Megkoronázod az esztendőt
 Nagy sok javaiddal,
 Lábaid nyoma kövérséget
 Csepegtet nagy zsírral.
 Lakóhelyei a pusztáknak
 Folynak kövérséggel,
 Hegyek és halmok vigadoznak
 Nagy bő termésekkel.
/9
#19E84242
 A szép sík mezők ékeskednek
 Sok baromcsordákkal;
 Villognak a szép szántóföldek
 Sűrű gabonákkal.
 A hegyoldalak, mezőföldek
 Szép búzanövéssel,
 Örvendeznek és énekelnek
 Nagy gyönyörűséggel.

;Hálaének Isten csodálatos szabadításáért
;Bourgeois L., Strasbourg, 154
>66
/1
#92118CE1
 Örvendj, egész föld, az Istennek,
 És énekelj szép zengéssel!
 Nagy dicsőségét szent nevének
 Mindenek dicsérjék széjjel.
 Mondjátok ezt az Úr Istennek:
 Csudálatosak dolgaid,
 Erősséged nagy, hozzád esnek
 Hízelkedvén ellenségid.
/2
#B59AC3E1
 A te isteni felségedet
 E földön mindenek áldják,
 És dicsőséges szent nevedet
 Énekléssel magasztalják.
 Jertek, és ezt jól meglássátok,
 Minden jól ide figyelmezz:
 Istennek mily csudálatosak
 Dolgai az emberekhez.
/3
#171DC253
 A tengert és a folyóvizet
 Megszárasztja, hogy a népnek
 Lábuk szárazon által mehet,
 Min szíveink örvendeznek.
 Országa megmarad örökké,
 Szeme lát minden népeket,
 Akik feltámadnak ellene,
 Előmenetük nem lehet.
/4
#330BE3D4
 Áldjátok a mi Istenünket,
 E földön minden emberek,
 Dicsérjétek az ő szent nevét
 Nagy zengéssel, minden népek.
 Mert életünket ő megadá
 Az ő nagy kegyességéből,
 Lábainkat meggyámolítá,
 Oltalmazván eleséstől.
/5
#DB33210D
 De minket igen megpróbálál
 És megkesergetél, Isten.
 Megolvasztál és megtisztítál,
 Mint ezüstöt, a nagy tűzben.
 Te minket, szegény szolgáidat
 Adál ellenség tőriben,
 Kik hevederrel tagjainkat
 Megkötözék nagy erősen.
/6
#7CA05F40
 Mi fejünkre népet ültetél,
 Mint a barmok, terhelteténk;
 Nagy árvizeket ránk eresztél,
 Sebes tűzön általmenénk.
 De te kihoztál, Uram, minket,
 Megenyhítél, megnyúgotál,
 Templomodban azért tégedet
 Dicsérlek szép áldozattal.
/8
#894698DD
 Jertek, halljátok, hadd beszéljem
 Tinéktek, istenfélőknek,
 Amiket Isten tett énvélem,
 Mily kegyesen tett lelkemnek!
 Hogy szájammal hozzá kiálték,
 Ottan meghallgata engem,
 Azért beszédével nyelvemnek
 Mindenkor őtet dicsérem.
/10
#670DF1B0
 Légyen áldott a nagy Úr Isten
 Az ő nagy kegyességéért,
 Ki meghallgatott kérésemben,
 És kegyelmesen hozzám tért!

;Isten a népek Ura
;Bourgeois L., Strasbourg, 1545
>67
/1
#9F77167E
 Úr Isten, áldj meg jóvoltodból
 És kegyesen fordulj hozzánk,
 Oltalmazz meg minden gonosztól,
 Szent színedet fordítsd reánk,
 Hogy e földön minden
 Megismerje szépen
 A te utadat,
 És a pogány népek
 Téged tiszteljenek,
 Megtartójukat.
/2
#EDCE5831
 És akkoron dicsérnek téged,
 Dicsérnek téged a népek,
 Nagy tisztességet tesznek néked.
 A pogányok is örülnek,
 Midőn mindeneket
 És a pogány népet
 Szent igazsággal
 Bírod és ítéled
 És jóra vezérled
 Nagy hatalmaddal.
/3
#4B44704E
 Dicsérjen téged minden nemzet,
 Úr Isten, téged dicsérjen!
 A föld teremjen bő gyümölcsöt,
 Áldjon meg minket az Isten!
 Adja szent malasztját,
 Nyújtsa áldomását,
 És ő felségét
 Félje és rettegje
 E föld kereksége,
 Mint ő Istenét.

;Diadalének (A hugenották harci zsoltára)
;Greiter M., Strasbourg, 1525 (1539) után
>68
/1
#73DBBDA5
 Hogyha felindul az Isten,
 Elkergettetnek szertelen
 Minden ő ellenségi.
 És elfutamodnak széjjel
 Ő haragos színe elől
 Minden ő gyűlölői.
 Úgy elűzetnek hirtelen,
 Mint a füst semmivé leszen,
 Elvész minden ő dolgok;
 Mint viasz olvad a tűztől,
 Úgy az ő kemény színétől
 Elvesznek a gonoszok.
/2
#5B136DB2
 De az igazak mindnyájan
 Örvendeznek nagy hívságban
 A kegyes Isten előtt;
 És víg szívvel énekelnek,
 Hogy ő kevély ellenségek
 Megszégyenült, elveszett.
 Nagy örömmel az Istennek,
 Énekeljetek nevének,
 Dicsérjétek, áldjátok;
 Ki puszta földeken megyen,
 Kinek neve erős Isten:
 Őtet magasztaljátok!
/3
#5AF4C21A
 Ő előtte vígadjatok,
 Mert ő az árváknak atyjok,
 Kiket táplál kegyesen.
 Az özvegyeknek oltalma.
 Az ő isteni hatalma,
 Lakván ő szentségében.
 Megáld egyes embereket,
 Ád nékik szép cselédeket
 Háznép tenyészésére;
 A foglyokat ő kimenti,
 A pártütőket rekeszti
 Puszta és parlag földre.
/10
#27E8D9BB
 Ellenségidet megbírod,
 Köztük helyezvén hajlékod
 Lakozol dicsőséggel.
 Áldott legyen az Úr Isten,
 Ki velünk ily nagy jól teszen,
 Látogat kegyességgel.
 Isten a mi segedelmünk,
 Kegyesen lakozik velünk,
 Őriz minket, népeit;
 Halál és élet kezében,
 Ő az erős, örök Isten,
 Megtartja ő híveit.
/16
#A7684471
 Énekeljetek Istennek,
 Ki lakója az egeknek,
 Kiket teremtett régen.
 Holott nagy hatalmával ül,
 Honnan szava úgy megdördül,
 Hogy zeng és harsog minden.
 Dicsérjétek nagy hatalmát,
 Ki felséges méltóságát
 Izráelen láttatja!
 Kinek nagy erejét ott fönn,
 Az égen és a felhőkön
 Senki sem tagadhatja.
/17
#6525A4A4
 Nagy félelmetes az Isten
 Az ő dicső szentségében!
 Ki Izráel népének
 Erőt ád nagy kegyelmesen;
 Azértan minden időben
 Mitőlünk dicsértessék!

;Kiáltás a mélységből
;Bourgeois L., Genf, 1551
>69
/1
#7A888035
 Úr Isten, segíts és tarts meg engem,
 Mert a vizek szinte lelkemig érnek,
 Közepén vagyok a sáros mélységnek,
 Amelyben csaknem elsülylyed fejem!
 Az árvizek öszszeütnek rajtam,
 A kiáltás miatt torkom elrekedt,
 Én szemeimben megfogyatkoztam,
 Midőn várom a te segedelmedet.
/2
#7AB7E690
 Én hajam szálánál többen vagynak,
 Akik engemet ok nélkül gyűlölnek;
 Én ellenségim szertelen erősek,
 És engem eltörleni akarnak.
 Noha semmit nem vettem senkitől,
 De mégis énnékem kell megfizetnem;
 Nincs, Uram, elrejtve színed elől
 Én bolondságom és minden én vétkem.
/3
#C0FE3ED9
 Seregek Ura, hatalmas Isten,
 Aki vezérled a te seregedet!
 Kik benned vetik ő reménységüket,
 Ne hagyd miattam esni szégyenben!
 Izrael Ura, akik tebenned
 Szívből bíznak, ne szégyenítsd meg őket,
 Mert nagy szidalmat szenvedek érted,
 A gyalázat elborítja színemet!
/4
#B154C893
 Mert idegennek rokonim tartnak,
 Atyámfiai sem ismernek engem;
 Szent templomodban gerjedez én szívem,
 Megemészt nagy szerelme házadnak.
 És gyalázóid ő sok szidalmak
 Énrám esének, én pediglen sírtam,
 És böjtöltem, de ők csak csúfoltak
 Engemet, noha minden jót kívántam.
/5
#F1951338
 Bánat miatt én zsákba öltöztem,
 De ők énrólam mesét költöttenek;
 A kapun űlők és a részegesek
 Csúf énekléssel nevetnek engem.
 De én tehozzád nagy buzgó szívből,
 Ó kegyes Isten, könyörgök naponkint;
 Hallgass meg, Uram, kegyességedből,
 És kegyelmezz meg te hűséged szerint!
/6
#4B21D117
 Végy ki engemet e fertős sárból,
 Hogy el ne süllyedjek, gyűlölőimtől
 Ments meg e nagy vizeknek örvényiből,
 Tarts meg ellenségim haragjától,
 Hogy a mély víz engem el ne nyeljen,
 Se pedig felül reám ne boruljon;
 Ne merüljek tengeri mélységben,
 A kútnak szája engem be ne zárjon!
/7
#30D43349
 Úr Isten, nagy a te kegyességed,
 Hallgasd meg azért, amit tőled kérek:
 Irgalmas szemeid reám nézzenek,
 Hadd láthassam nagy kegyelmességed!
 Ne rejtsd el, Uram, kegyes orcádat
 Szegény szolgádtól, mert szorongattattam;
 Ne késsél, halld meg kiáltásomat,
 Add meg kérésem, vigasztalj meg, Uram!

;Uram, ne késlekedjél!
;Bourgeois L., Genf, 1551
>70
/1
#2A30587F
 Siess, ments meg, Uram Isten,
 Mert benned bízom teljes szívvel,
 Azért hamar légy segítséggel
 Minden ellenségim ellen!
 Akik törnek én életemre,
 Mind megszégyeníttessenek;
 Kik nyavalyámon örülnek,
 Térjenek meg nagy szégyenükre!
/2
#7C6636F9
 Nyomorult és szegény vagyok,
 Tarts meg azért, ó, én Istenem,
 Mert csak te vagy én segedelmem:
 Ne késsél, mert majd elfogyok!
 Hogy azok benned örvendjenek,
 Kik segedelmedet várják
 És magukat reád bízzák,
 Ezt mondván: dicsőség Istennek!

;Ne hagyj el engem agg koromban!
;Bourgeois L., Genf, 1551
>71
/1
#C4498633
 Tebenned bízom, én Istenem,
 Kérlek, oltalmazz meg,
 Gyalázattól ments meg,
 Hogy örök szégyenbe ne essem!
 A te nagy jóvoltodból
 Tarts meg minden gonosztól!
/2
#4C8C2162
 Hajtsd hozzám füled, tarts meg engem,
 Nyújtsad segítséged,
 Amint megígérted,
 Hogy segítségül léssz énnékem;
 Légy azért én kősziklám
 És én erős kőváram!
/3
#0DB0501D
 Ments ki a hamisnak kezéből,
 És annak markábul,
 Aki él csalárdul;
 Oltalmazz meg a kegyetlentől!
 Benned bíztam, Uramba',
 Gyermekségemtől fogva.
/4
#A27F43A0
 Hogy származám anyám méhéből,
 Legottan énnékem
 Valál reménységem.
 Anyám méhéből engem kivől,
 Azért neked éneklek
 És szüntelen dicsérlek.
/5
#CD84312D
 Engem csudának tartnak sokan,
 De te vagy énnékem
 Biztos menedékem.
 Adj éneklést az én szájamban,
 Hogy hirdessem nevedet,
 Áldjam dicsőségedet!
/6
#3A973084
 Ez én nyavalyás vén koromban,
 Erőtlenségemben
 Ne vess el, Úr Isten!
 Ne hagyj el én sok nyavalyámban,
 Midőn szegény testemben
 Semmi erősség nincsen!
/10
#CCF5679B
 Szüntelen hirdeti én nyelvem
 Szent igazságodat
 És nagy jóvoltodat.
 Jótétid mindennap beszélem,
 Sokságát nem titkolom,
 Noha számát nem tudom.
/11
#800EE359
 Az Úrnak erejére térek,
 És ő nagy hatalmát,
 Szemlélem jóvoltát.
 Nagy igazságát az Istennek
 Mindenkor emlegetem,
 Soha el nem felejtem.
/12
#2682257A
 Gyermekségemtől fogva engem
 Híven tanítottál
 Csuda dolgaiddal.
 És hogy immáron megvénhedtem,
 És a hajam megőszül:
 Légy most is segítségül!
/13
#7368A9BC
 Míglen a te karod erejét
 Hirdetem mindennek,
 Jövendő nemzetnek.
 És igazságodnak hűségét,
 Melyet lát nyilván minden,
 Magasztalom szüntelen.
/14
#E641DAD5
 Hozzád hasonló ki lehetne?
 Engemet, Úr Isten, Ejtél nagy ínségben.
 De a föld mélységéből végre
 Ismét kivonál engem,
 És megadád életem.
/15
#8CB51CF6
 Erősségemet öregbíted,
 És felmagasztalál,
 Mélységből felhozál.
 Hozzám tére, Uram, felséged,
 Megvigasztala engem,
 Midőn vala ínségem.
/16
#F2E81472
 A te hűségedet, Úr Isten,
 Hirdetem, éneklem,
 Mindennek beszélem.
 Dicsérlek lantnak zengésében,
 Ó, szentek dicsősége,
 Izráelnek szentsége!
/17
#4C74D251
 Ajkaimmal vígan dicsérem
 A te hatalmadat
 És nagy irgalmadat.
 Lelkemet hozzád fölemelem:
 Megtartád életemet,
 Hogy dicsérjelek téged!

;Áldáskívánás a béke fejedelmére
;Bourgeois L., Strasbourg, 1545
>72
/1
#0490F148
 Uram, a te ítéletedet
 Adjad a királynak,
 És igazságodnak értelmét
 A király fiának,
 Hogy ő a te nagy seregedet
 Igazán ítélje,
 És a te sok szegény népedet
 Törvénynyel vezérlje.
/6
#27C3B5A3
 Nagy alázatosan imádják
 Őt minden királyok;
 Minden népek őtet szolgálják,
 És ő lészen urok,
 Mert ő a szegényt megsegíti,
 Aki őtet hívja,
 És a nyomorultat megmenti,
 Kinek nincs gyámola:
/7
#DADD3161
 A szűkölködőket megszánja
 Nagy kegyességében,
 És a szegényt hozzá fogadja,
 Megőrizvén híven.
 Megtartja erőszaktól őket,
 És a csalárdságtól,
 Nagyra becsüli ő véröket,
 Megmenti gonosztól.
/10
#CB7F7125
 E földön minden nemzetségek
 Őtet áldják vígan,
 És szívük szerint dicsekednek
 E kegyes királyban.
 A pogányok is így dicsérik:
 Dicsőség Istennek,
 Aki nagy csudát cselekeszik,
 Ura Izráelnek.

;Jó nékem az Isten közelsége
;Bourgeois L., Genf, 1551
>73
/1
#C306D3EF
 Bizonyára jó az Isten,
 Híveihez kivált képpen,
 Akik szívüket tisztán tartják,
 Az ő jóvoltát azok látják.
 De én már csaknem elhajlék,
 Úgy megtántorodék lábam,
 Járásomban úgy megbotlám,
 Hogy csaknem szörnyen ledűlék.
/2
#F6A4BF31
 Mert bosszankodom e népre,
 Kinek nagy esztelensége,
 Midőn látom, hogy a hitlenek
 Jó szerencsében, vígan élnek;
 A halállal nem bajlódnak,
 Semmi fájdalmat nem látnak,
 És kövérek ő testükben,
 Élnek nagy jó egészségben.
/6
#E5474D97
 Szólnak végre: Úr Isten,
 Látod-e ezt a mennyekben?
 Mi nyavalyánkat megérted-e?
 Keserűségünket szánod-e?
 Ímé az istentelenek,
 Látjátok, élnek örömben,
 Amint kívánják szívükben,
 Gazdagok, jó szerencsések.
/7
#06F25C2D
 Mit használ tehát énnékem,
 Hogy tisztán tartom én szívem?
 És micsoda hasznom van abban,
 Hogy kezem mosom tisztaságban?
 Ímé, mind hiába marad,
 Hogy én így kesergettetem,
 Ottan marad büntetésem
 Reggel, mihelyen megvirrad.
/8
#AD659E4E
 Csaknem én is ígyen szólék,
 De látám, hogy nem jól esnék,
 És ártalmas lenne azoknak,
 Te fiaidnak kik mondatnak.
 Azért jobban meggondolám,
 Magamban meghányám-vetém,
 De sokkal nehézbnek lelém,
 Hogynem végét találhatnám.
/9
#6D2553A9
 Mígnem el-bémenék végre
 Az Istennek szent helyére,
 És végezetre vőm eszemben,
 A gonoszok vége mi légyen.
 Azt is jól megértém aztán,
 Hogy őket te megbünteted,
 Mert síkod földre helyezed,
 És veted a pusztaságban.
/10
#4EC7BA26
 Úgy, hogy csudálják mindenek,
 Hogy ily hamar oda lesznek;
 Vesznek és esnek nagy ínségbe,
 Szégyenségbe és rettegésbe.
 Minden ő gazdagságukat
 Múlandó álommá teszed,
 Melyből ember ha fölébred,
 A lelt jókban semmit nem lát.
/12
#0E876F81
 De én mindig nálad vagyok,
 És közeledben maradok,
 Mert te megtartál jobb kezeddel,
 Nagy ínségemben nem hagyál el.
 Tanácsoddal vezérlj engem,
 És igazgass ösvényedre,
 Aztán végy fel dicsőségre,
 Tégy ily kegyesen énvelem!
/13
#01E135E4
 A mennyekben te vagy nekem
 Csak egyedül én Istenem;
 Ez egész földön senki nincsen,
 Kit kívüled lelkem kedveljen.
 Ha elfogy testem és lelkem,
 Mégis vigasztalsz szívedben;
 Egyéb részem nekem nincsen
 Tenáladnál, én Istenem!
/14
#5B6DF3C3
 Mert kik tőled eltávoznak,
 Azok elvesznek, romlanak;
 Elveszted azokat hirtelen,
 Kik bíznak bálvány-istenekben.
 Azért én néked ezt mondom:
 Közelséged oly jó nékem,
 Mert te vagy én menedékem,
 És jótéted magasztalom!

;Az ellenség a templomban
;Genf, 1562
>74
/1
#C2F466AE
 Miért vetsz minket így el, Úr Isten?
 Mire haragszol mireánk enynyire?
 Míglen gerjed föl haragodnak tüze
 Juhaidnak nyájára ily igen?
/2
#02A33286
 Emlékezzél meg te seregedről,
 Melyet régente magadnak szerzettél;
 Felvett örökségedről emlékezzél,
 A Sion hegyén lakóhelyedről!
/3
#549A9D4B
 Kelj föl, és végre jövel, Úr Isten:
 Szent templomodat rútul összedönték,
 Mindent elrontott benne az ellenség,
 Pusztaságot tőn te szent helyeden!
/4
#BB77973C
 Holott dícséreted hirdettetett,
 Most ellenségid ott ordítnak szörnyen,
 És a tenéked szenteltetett helyen
 Az ő zászlójuk fölemeltetett.
/7
#FCEB2B66
 Te templomodat ők fölgyújtották,
 Szent helyedet megszentségteleníték;
 Drága lakóhelyét te szent nevednek
 Tövestől fogva mind elrontották.
/8
#7D6BC0B0
 Nosza hozzá, mondják ők magukban:
 Pusztítsuk el őket nagy szörnyűséggel!
 Az Istennek szentegyházait széjjel
 Mind megégették e tartományban.
/9
#D6F87E8F
 Már jeleit nem látjuk erődnek,
 Immáron nincsen közöttünk próféta;
 Nincs jövendőmondó, aki tudhatná,
 Mint leszen vége e nagy veszélynek.
/10
#DC187EFB
 Ó, Úr Isten, ezt meddig engeded,
 Hogy ellenségink minket nevessenek?
 Örökké hagyod-e gyűlölőidnek,
 Hogy gyalázattal illessék neved?
/11
#8F31FAB9
 Miért fordítod el kezeidet?
 Jobb kezedet mit dugod kebeledben?
 Vajha egyszer kinyújtanád ismétlen,
 Mutatván nékünk segedelmedet!
/12
#35A53CDA
 De az Isten én királyom régen,
 Aki engemet bírt és jól vezérlett.
 Hatalmát jelentvén e világ előtt,
 Hogy segedelmem ő mindenekben.
/13
#828BC11C
 Te választád kétfelé a tengert,
 A sárkánynak te rontottad meg fejét;
 A cethalaknak fejüket megtörted,
 Nagy tagjuk széjjel a parton hevert
/14
#F564F235
 Miket aztán eledelül küldél
 A népeknek a nagy puszta helyekben;
 Patakvizet és forrást csudaképen
 Az erős kősziklából eresztél.
/15
#9BD83456
 Megszárasztád a folyóvizeket,
 Tied a nap és tied az éjszaka;
 Hogy a setétséget világosítsa,
 Arra teremtéd a napnak fényét.
/16
#5B97E9B1
 Te isteni nagy bölcsességeddel
 Bizonyos határt vetettél a földnek,
 És különbségét a nyártól a télnek
 Elosztád hévséggel és hideggel.
/17
#020B2BD2
 Emlékezzél meg, lásd meg e dolgot:
 Ellenségid mint rútolnak, nevetnek;
 A bolond népek veled csúfot űznek,
 Szent neveden tesznek gyalázatot!
/18
#EAC10DC0
 A te gerlice galambocskádat
 Ne hagyd a gonosz vadaknak megenni;
 A szegényeket eszedből ne vesd ki,
 Inkább kegyesen viseld gondjukat!
/19
#444DCB9F
 Emlékezzél meg szent kötésedről,
 Mert az egész föld merő setétséggel
 Eltölt, és rakva csalárd gonosz néppel,
 Kik nyomorgatnak nagy kegyetlenül!
/20
#16B05FB4
 Ne engedd hátra térni szégyennel
 A te nyomorult szegény szolgáidat;
 Fordítsd hozzájuk te nagy jóvoltodat,
 Hogy nevedet dicsérjék víg szívvel!

;Isten megítéli a kevélyeket
;Genf, 1562
>75
/1
#78586D5B
 Dicsérünk téged, Isten,
 Dicséret légyen neked,
 Mert a te dicső neved
 Hozzánk közel jött híven,
 Mi azért csudáidat,
 Hirdetjük jó voltodat.
/2
#8F6868C8
 Ha időm lesz valaha,
 Én igazán ítélek;
 Ha tüstént elomlanék
 A föld fundámentoma,
 Én mégis ő oszlopát,
 Jól állatnám gyámolát.
/3
#7BF2E99E
 A bolondokat intem:
 Ne bolondoskodjatok!
 Akik pedig gonoszok,
 Térjetek meg, úgy kértem!
 Szarvat ne emeljetek,
 Ily fenn ne beszéljetek!
/4
#D1202166
 Mert a tiszt és az erő
 Nem a napkelet felől,
 Sem a napenyészetről,
 Sem nem a pusztából jő:
 Minden az Istenen áll,
 Ő aláz, ő magasztal.
/5
#51A0B6B0
 A pohár ő kezében,
 Melyben vagyon vörös bor,
 Mit ő teli tölt sokszor
 Minden népnek e földön.
 De végre a gonoszok
 A söprejét megisszák.
/6
#AB2D748F
 Én örökké dicsérem
 A Jákóbnak Istenét,
 Hirdetem dicsőségét,
 És szarvukat megszegem
 Az istenteleneknek,
 Hogy a jók felkeljenek.

;A mennyei Bíró dicsérete
;Bourgeois L., Genf, 1551
>76
/1
#123EA3E5
 Ismeretes az Úr Isten
 A Júdában és szent neve
 Az Izráelnek földében;
 Messze terjed dicsősége,
 A Sálemben szép ő sátora,
 A Sionon lakó hajléka.
/2
#A8269729
 Ott meglátja minden ember,
 Hogy isteni hatalmával
 Megtörik íj, paizs, fegyver,
 Hadat megállít azonnal.
 Nagyobb felségednek ereje,
 Hogynem az ellenség fegyvere.
/3
#0414A9F9
 A bátor is levereték,
 Reájuk mély álom esék;
 Kiket vitézeknek véltek,
 Lőn tudatlan az ő kezek;
 Mihelyen feddésed hallatik,
 Mind szekér, mind ló elaluszik.
/4
#83EDAD13
 Ki állhatna meg előtted,
 Aki ily rettenetes vagy,
 Midőn haragod felgerjed?
 Mennybéli szentenciád nagy,
 Melyet hallatsz az emberekkel,
 A föld megrémül csendességgel.
/5
#CF591B51
 Midőn felkél az Úr Isten,
 Tartván kemény ítéletit,
 Hogy a szegényeket itten,
 E földön megtartsa népit:
 Dicséretedre fordul néked,
 Ha a nép haragszik ellened.
/6
#438B73CA
 Még egyszer nyilván elveszted
 E maradék dúlót-fúlót.
 Jer, dicsérjük Istenünket,
 Megállván fogadástokat,
 Kik mindenkor vagytok ővéle,
 És el nem távoztok őtőle!
/7
#3D3D829A
 Adjatok szép ajándékot
 E rettenetes Istennek,
 Ki megtöri hatalmukat
 A gonosz fejedelmeknek!
 A nagy királyok itt e földön
 Előtte lesznek rettegésben.

;Jelen, múlt és jövendő
;Bourgeois L., Strasbourg, 1545
>77
/1
#484B1814
 Az Istenhez az én szómat,
 Emelém kiáltásomat;
 Hogy felkiálték hozzá,
 Beszédem meghallgatá.
 Mindennémű szükségemben
 Reménységem csak az Isten;
 Éjjel kezem feltartom,
 Az égre hozzá nyújtom.
/2
#34A3A126
 Lelkem nagy bánatba esett,
 Minden vigasztalást megvet,
 Az Isten rettent engem,
 Ha róla emlékezem.
 Noha Istennek szívemben
 Panaszlok nagy ínségemben,
 Lelkem mégsem találhat
 Semmiben nyugodalmat.
/3
#61489C79
 Szemeimet nyitva tartod,
 Aludni éjjel sem hagyod;
 Erőmben úgy elfogyék,
 Hogy egy szót sem szólhaték.
 Gondolám a régi időt,
 És forgatám szemem előtt
 Az elmúlt esztendőket,
 Hányám és vetém őket.
/5
#0E56DC1F
 E harag mindenkor tart-é?
 Az Úr örökké elvet-é?
 Nincsen-e őnekie
 Én hozzám semmi kedve?
 Elfogyott-é nagy kegyelme,
 És ő atyai szerelme?
 És immáron hátra tért,
 Amit nékünk megígért?
/6
#C17057DD
 Teljességgel elfogyott-e
 Hozzánk való nagy szerelme?
 Elaludt-e irgalma
 Az ő nagy haragjába’?
 Végre mondék: oda vagyon
 Az én életem immáron,
 Isten megvonta kezét,
 Nem nyújtja segedelmét.
/7
#67E399FE
 De meggondolám ismétlen,
 Miket míveltél régenten:
 A te nagy csudáidat,
 Miket sok ember látott.
 Csudáidról gondolkodva,
 Miket láték dolgaidba’,
 Elmélkedém erősen,
 Végre szólék ekképen:
/8
#CE16917A
 Ó, erős és kegyes Isten,
 Szent vagy cselekedetidben,
 És sehol senki nincsen
 Hozzád hasonló Isten.
 Csuda, Isten, a te dolgod,
 Amint gyakran megmutatod,
 Minden népek jól látják,
 Nagy voltát hatalmadnak.
/9
#27A34F59
 Népedet te kimentetted,
 Nagy ínségéből kivetted,
 Jákóbnak és Józsefnek
 Nemzetét mindkettőnek.
 Hogy a víz téged megláta,
 Ottan nagy félelme juta,
 És a tenger mélysége
 Legottan megrendüle.
/10
#C991C807
 A felhők esőt ejtének,
 És nagy mennydörgések lőnek,
 A magas ég harsoga,
 Széjjel minden csattoga.
 Sűrű nyilak repülének,
 Sebes kőesők esének,
 Oly villámlások lőnek,
 Hogy az egész föld fénylék.
/11
#7CAB43C5
 A föld megrendüle szörnyen,
 Útad lőn a nagy tengeren;
 Általmenél a vizen,
 Lábad nyoma még sincsen.
 A te szerelmes népedet
 Mint egy nyájat vezérletted,
 Kihozád hatalmaddal
 Mózes és Áron által.

;Istenről beszél a történelem
;Bourgeois L., Genf, 1551
>78
/1
#900E725A
 Hallgass, én népem, az én törvényemre,
 Füledet hajtsad az én beszédemre,
 Amelyek az én szájamból származnak.
 Hogy jól megérthesd mivoltát azoknak,
 Mert én neked oly dolgot beszélek,
 Mit titoknak tarthatnak mindenek.
/2
#D839BBDB
 Oly dolgot, amit a mi atyáinktól
 Hallottunk, és megértettünk azoktól.
 Nem azért, hogy csak mi megemlékeznénk,
 De fiainknak is jól elbeszélnénk;
 Hirdessük azért nagy dicsőségét
 És az ő sok csuda téteményét!
/8
#6A2DAF24
 A pusztán a kősziklát meghasítá,
 És a vízzel, amely abból kifolya,
 Népét megitatá, és azon helybe'
 A kősziklából kútfejet ereszte,
 Melyből bőséges forrás buzdula,
 Mely, mint a patak, sebesen folya.
/27
#F2D95F0F
 Bátorsággal ő seregét kivivé,
 Az ellenséget a tengerbe veszté;
 Mindenütt nyilván szabadon menének,
 Míg a szent föld határába érének,
 Mind a nagy hegyig a dicsért földön,
 Melyet erős jobb kezével megvőn.

;Panaszos ének Jeruzsálem elpusztításáról
;Bourgeois L., Strasbourg, 1545
>79
/1
#ECDB3DC1
 Öröködbe, Uram, pogányok jöttek,
 És szent templomodat megfertőztették,
 Jeruzsálem városát elrontották,
 És széjjel nagy kőrakásokra hányták.
 Szolgáidnak testek,
 Akik megölettek,
 Adattak a hollóknak;
 Húsok te szentidnek
 Ételül vettetnek
 A mezei vadaknak.
/2
#7712B27E
 A városon nagy sok vért kiontának,
 A sok vér, mint vizek, széjjel folyának;
 Tőnek oly szörnyű nagy öldökléseket:
 Nem lőn, ki eltemetné a testeket.
 A mi szomszédságunk
 Csúfságot űz rajtunk;
 Akik körűlünk laknak,
 Minket nem becsülnek,
 Sőt csak megnevetnek,
 Csúfolnak és rútolnak.
/3
#2AE2021C
 Míg haragszol, Uram, reánk ekképpen?
 Haragod míglen gerjedez ily igen?
 Meddig terjeszted bosszúállásodat,
 Mely minket, mint a sebes tűz, elfogyat?
 Bosszúdat azokra,
 Ontsd a pogányokra,
 Kik téged nem tisztelnek!
 Dűtsd az országokra,
 Hol nevedet soha
 Nem tisztelik a népek!
/4
#B5EE50FA
 Mert Jákóbot ők megették, elnyelték,
 Az ő nemzetségét megemésztették;
 Minden házait földre lerontották,
 Kietlen pusztaságra fordították.
 Az atyák vétkéért,
 Ne büntess mindezért,
 Nagy haragod szünjék meg!
 Jővel, mi Istenünk,
 Mert igen gyötretünk,
 És kegyelmesen tarts meg!
/5
#0F7B0ED9
 Tekintsd meg, Uram, kegyelmességedet,
 A te szent nevedért segíts meg minket!
 Szabadíts és tarts meg minket kegyesen,
 Bűneinket bocsásd meg szent nevedben!
 Hogy ne nevettessünk,
 Kérdvén: hol Istenünk?
 Verd meg a pogány népet!
 Vérét szolgáidnak,
 Mit ők kiontának,
 Nékik el ne engedjed!
/6
#02D4D53C
 Jusson elődbe siralmuk azoknak,
 Kik a rabságban óhajtnak és sírnak;
 Siess, mentsd meg, hogy ők el ne vesszenek,
 Kik immár halálra ítéltettenek!
 Szomszédinknak épen
 Fizesd meg hétképen,
 Amit rajtunk míveltek!
 Hogy téged, Úr Isten,
 Szidalmaztak szörnyen:
 Fizesd meg őnékiek!
/7
#5C61931B
 És mi úgy, mint te nyájad és sereged,
 Hálákat adván, magasztaljuk neved;
 Dícséretedet mindenkor hirdetjük,
 És nemzetről-nemzetre kiterjesztjük.

;Isten, állíts helyre minket!
;Genf, 1562
>80
/1
#838124D5
 Hallgasd meg, Izráel pásztora
 És Józsefnek vezérlő Ura,
 Kit őrizsz, mint egy juhnyájat,
 Fordítsd hozzánk szent orcádat!
 Aki ülsz a Kérubimon,
 Jelenjél meg világoson!
/3
#848598D8
 Úr Isten, térj hozzánk ismétlen,
 Őrizz meg minden gonosz ellen!
 Nyújtsad világosságodat,
 Fordítsad ránk irgalmadat,
 Térítsd hozzánk szent orcádat,
 És semmi nekünk nem árthat!
/4
#62806FEB
 Ó, Úr Isten, vajon míg tartod
 Mirajtunk a te nagy haragod?
 Míg veted meg kérésünket?
 Könnyhullatást kenyér helyett
 Adál nékünk, és itattál
 Minket nagy könnyhullatással.
/5
#C575F063
 Te minket feddődésbe ejtél,
 Szomszédinkkal nem becsültetél;
 Az ellenségnek lőnk csúfja!
 Tarts meg, seregeknek Ura!
 Világosítsd színed rajtunk,
 És nem lesz semmi bántódásunk!
/6
#9169928A
 A szőlőtövet Egyiptomból,
 Elhozád szolgálat házából,
 És a jó földbe plántálád,
 Honnan a népet kihajtád;
 E tágas téren nagy messze
 Ő gyökere kiterjede.
/7
#4FED4147
 Ő széles árnyéka kiterjedt,
 Béfödte a magas hegyeket,
 Az ő vesszei magasak,
 Felnőttek, mint a cédrusfák,
 Ágait mind a nagy vízig,
 Kiterjeszté a tengerig.
/8
#29FA9470
 Miért tördeléd el gyepűjét,
 Hogy eresztél belé más népet,
 Ki szőleit leszaggatá?
 Még gyökerét is kitúrá!
 A vadak kinyűvék szörnyen!
 Hogy engedéd ezt Úr Isten?
/9
#8387524F
 Uram, ismétlen térj mihozzánk,
 Mennyekből szemed fordítsd reánk,
 És látogasd meg a szőlőt,
 Melyet jobb kezed ültetett,
 Nézd meg a csemetét végre,
 Kit szerzél dicsőségedre!
/10
#586FB905
 Tűzzel megint megégettetik,
 Teljességgel elpusztíttatik
 Haragodnak nagy tüzében.
 Nyújtsd kezedet, ó, Úr Isten,
 Az emberre, kit kezeddel
 Magadnak erősítettél.
/11
#0B8934C3
 És mi el nem térünk tetőled,
 Csak életünket erősítsed,
 És dicsérjük szent nevedet,
 Úr Isten, vigasztalj minket!
 Világosítsd színed rajtunk,
 És mi nyilván megtartatunk.

;Ünnepi ének
;Genf, 1562
>81
/1
#1245001A
 Örvendezzetek
 Az erős Istennek!
 Énekeljetek
 Dicséreteket,
 Szép énekeket
 Jákób Istenének!
/2
#69D62AD5
 Szép dicséretet
 Néki mondjon minden!
 Lantokban őtet
 És citerákban
 Dicsérjük vígan
 Néki zengedezvén!
/3
#00EF23D4
 Most ez új hóban (évben)
 Néki örvendezzünk
 Trombitaszóban!
 Rendelt időnkben,
 Víg ünnepünkben
 Illik énekelnünk!
/9
#9F591E16
 Hallgasd meg, népem,
 És közlöm tevéled
 Vádló beszédem!
 Halld meg, Izráel,
 Amit szám beszél,
 És azt jól megértsed:
/10
#A2D2ED4F
 Néked ne legyen
 Idegen Istened,
 De egyedül én!
 Csak engem tisztelj,
 Senki mást ne félj.
 Nevemet becsüljed!
/11
#B8EF6996
 Én vagyok neked
 Istened egyedül;
 Ínségben téged
 Én megtartálak
 És kihozálak
 Egyiptom földébül.
/12
#06BA02B8
 Tátsd föl csak szádat,
 És megtöltöm bőven;
 Menten meglátod,
 Hogy nagy bőséggel
 Lesz az eledel
 Csudálatosképpen.
/13
#7C609F48
 De az én népem
 Engem nem hallgata,
 Noha intettem
 Sűrű intéssel,
 De az Izráel
 Füleit bedugta.
/14
#33A761DE
 Min megbúsulék,
 És őket elhagyám,
 Hogy bár menjenek
 Önnön kedvükre,
 És ösvényükre
 Szabadon bocsátám.
/15
#404D9B7B
 Ha népem szívvel
 Szót fogadott volna,
 És ha Izráel
 Én útaimban
 És tanácsomban
 Járni akart volna,
/16
#F3671739
 Én is legottan
 Az ő ellenségét
 Nagy hatalmamban
 Megvertem volna,
 Vetettem volna
 Rájok én kezemet.
/17
#99DA595A
 Ő ellenségét
 Néki adtam volna,
 Jó szerencséjét
 Én őnékie
 Nagy sok időkre
 Terjesztettem volna.
/18
#E6C0DD9C
 Búzát nékie
 Szépet adtam volna
 Eledelére.
 És nagy bőséggel
 A kősziklából
 Mézet adtam volna.

;Isten az igaz Bíró
;Bourgeois L., Genf, 1551
>82
/1
#191FCAEE
 Az Isten áll ő seregében,
 A bírák gyülekezetében,
 És köztük igazat ítél.
 És nékiek ígyen felel:
 Míglen ítéltek ti ekképpen
 A törvény igazsága ellen,
 Hogy személyét a hamisnak
 Nézitek, kedvezvén annak?
/2
#906E8829
 Ítéljetek szegényt igazán,
 És könyörüljetek az árván!
 Ügyét a szegény embernek
 Erőszaktól megmentsétek!
 A nyomorultat a hamistól,
 Kiszabadítsátok markából,
 Hogy a szegény erőt vegyen
 A hatalmaskodó ellen!
/3
#40A80AF3
 De intésemet nem fogadják,
 Sőt értetlenül megutálják.
 Nagy setétségben mennek el,
 Min a föld csaknem süllyed el.
 Ímé, én mondom, hogy ti vagytok,
 Kik isteneknek hívattattok,
 És a magasságos Úrnak
 Ti mondattok fiainak.
/4
#7A5DE6CA
 De mindnyájan meg kell halnotok,
 Mint egyebek, ti is kimúltok,
 Végre ti mind oda lesztek,
 Úgy mint egyéb fejedelmek.
 Támadj fel azért, ó, Úr Isten,
 És ítélj meg mindent e földön,
 Mert neked hatalmad vagyon
 Minden népen e világon!

;Körös-körül veszedelem
;Genf, 1562
>83
/1
#D8CBA93C
 Uram, ne hallgass ily igen,
 Ne maradj e veszteglésben,
 Ne nyugodjál, ó, erős Isten!
 Mert dühösködő ellenségid
 És minden gonosz gyűlölőid
 Felemelték fejüket fennen.
/2
#37A6E7E6
 Kiváltképpen néped ellen
 Csalárd árultatásképpen
 Sok álnok tanácsokat lelnek,
 És akikre te gondot tartasz,
 És rejtekhelyedben oltalmazsz,
 Azok ellen összeesküsznek.
/3
#8A02CDC1
 Mondván: jertek, ím ezeket,
 Veszessük el e nemzetet,
 Töröljük őket el e földről;
 Fogyasszuk el e népet szintén,
 Hogy emlékezet se lehessen
 Az Izráel népe nevéről!
/9
#42A6437C
 Kergesd el forgó szeleddel,
 Nagy rettegéssel ijeszd el,
 Szélvésszel háborítsd meg őket!
 Ő orcájukat szégyenítsd meg,
 Hadd ismerjenek tégedet meg,
 És keressék te szent nevedet.
/10
#CBE25627
 Taszítsd őket nagy szégyenbe,
 És ejtsd be nagy félelembe!
 És megismerjék minden népek,
 Hogy te az élő egy Isten vagy,
 Akinek hatalma igen nagy,
 Akit felségesnek neveznek.

;Sóvárgás a szent hajlék után
;Genf, 1562
>84
/1
#5E0E8758
 Ó, seregeknek Istene,
 Mely kedves gyönyörűsége
 A te szerelmes hajlékidnak!
 Az én lelkem fohászkodik,
 Tornácodba kívánkozik.
 Ó, Istene a magasságnak!
 Áhítozik testem, lelkem
 Tehozzád, élő Istenem!
/2
#34ABBAD6
 A verébnek is van fészke,
 És honjában költ a fecske.
 Én királyom, Zebaoth Isten:
 Hol vannak a te oltárid
 És te szentséges hajlékid,
 Hol dicsértetel felségesen?
 Bizony boldog az oly ember,
 Ki téged házadban dicsér.
/3
#58BB080E
 Ó, boldog az ember nyilván,
 Aki a te útaidban
 Kíván járni szívvel-lélekkel!
 Menvén a siralom völgyén,
 Ahol merő száraz minden,
 Ott is ő nagy hiedelemmel
 Kutakat ás és csatornát,
 Melybe esővizet bocsát.
/4
#28DB75D1
 Mennek erőről erőre,
 Segítségről segítségre,
 Míg hozzád jutnak a Sionra.
 Ó, erős Zebaoth Isten,
 Hajtsd hozzám füled kegyesen,
 És figyelmezzél az én szómra!
 Jákób Istene, nagy Isten,
 Hallgass meg én szükségemben!
/5
#DD7B14A8
 Mi paizsunk, ó, Úr Isten,
 Fölkentedre nézz kegyesen,
 Mert jobb egy nap a te házadban,
 Hogynem ezer nap egyebütt!
 Az Isten tornáca előtt
 Kapunálló lennék inkábblan,
 Hogynem mint sok időt éljek
 Házukban a hitleneknek.
/6
#9B8939A1
 Mert minékünk fényes napunk
 Az Isten, és mi paizsunk,
 Nagy dicsőséggel szeret minket.
 Azokkal kegyelmet tészen,
 Kik járnak a jó ösvényen:
 Sok javaival áldja őket,
 Boldog az ember éltében,
 Ki bízik az Úr Istenben.

;Vigasztalás szenvedésben
;Genf, 1562
>85
/1
#4696B7F9
 Nagy kegyesen től, Uram, földeddel,
 Jákób nemzetivel a fogságban,
 Kiket rabságból haza engedtél,
 Megengedél nékik irgalmasan.
 Megbocsátál, bűnüket elfedéd,
 Ellenük haragodat enyhítéd.
 Uram, kegyesen végy hozzád minket,
 Vedd el rólunk nagy haragod tüzét!
/2
#E684C90F
 Nemde mind örökké haragszol-é?
 Mind éltig nyújtod-é haragodat?
 Népedet már meg nem enyhíted-é?
 Hogy benned lelhesen vigasságot,
 És noha nagyok a mi bűneink,
 Mégis kegyelmed mutasd minékünk,
 És noha tettünk sok gonoszságot,
 De tőled kérünk irgalmasságot.
/3
#F43E9C8E
 Én meghallgatom, mit szól az Isten
 Az ő népének és ő szentinek:
 Békességet beszél kegyelmesen,
 Hogy bolondságból ne vétkezzenek.
 Kik őtet félik, minden meghiggye,
 Közel azokhoz ő segedelme;
 Hogy dicsősége lakjék földünkben,
 Minden ínséget rólunk elveszen.
/4
#D0A97A9D
 Jóság hűséggel összebékélik,
 Békesség, igazság egymást szépen
 Csókolják, e földön nevekedik
 A hit, látszik igazság az égen.
 Mindennemű jót ád az Úr Isten,
 És sok jó gyümölcs terem e földön.
 Az igazság megyen őelőtte,
 És ő járása tart mindörökké.

;Szorongatott ember imádsága
;Bourgeois L., Strasbourg, 1545
>86
/1
#32553BCE
 Hajtsd hozzám, Uram, füledet,
 És hallgasd meg kérésemet,
 Mert igen szegény vagyok,
 Az én szükségim nagyok.
 Tartsd meg testemet, lelkemet,
 Tekintsd kegyes életemet,
 Szolgádhoz térjen kedved,
 Ki bízik csak tebenned!
/2
#8A8B57E8
 Hozzád mindennap óhajtok,
 Nagy szükségemben kiáltok;
 Te nagy irgalmad szerint
 Kegyelmezz meg óránkint!
 Szolgád lelkét vigasztald meg,
 Uram, kiáltásom halld meg,
 Mert szívemet e végre
 Emeltem fel az égre!
/3
#4EF060F0
 Uram, jókedvű s édes vagy,
 A te irgalmasságod nagy,
 Minden emberhez pedig,
 Ki hozzád esedezik.
 Hallgasd meg azért kérésem,
 És nézd meg esedezésem:
 Tekintvén kegyelmedre,
 Figyelmezz beszédemre!
/4
#1FDA5878
 Nagy szükségemben óhajtok,
 Tehozzád szívből kiáltok,
 És te engem meghallgatsz,
 Nyavalyámban el nem hagysz.
 Ugyan nincs sehol több Isten,
 Ki hozzád hasonló légyen,
 Nincsen több erős Isten,
 Ki ily dolgot tehessen.
/5
#E4535376
 E világon minden népek,
 Kiket teremtél, eljőnek,
 És imádnak tégedet,
 Dicsőítik nevedet.
 Mert te nagy és hatalmas vagy,
 Csudatételed sok és nagy;
 Te vagy egyedül Isten,
 Sehol Isten több nincsen.
/6
#F23293BC
 Vezess, Uram, útaidban,
 Hogy járjak igazságodban,
 És csak arra hajtsd szívem,
 Hogy szent nevedet féljem.
 Néked, Uram, hálát adok,
 És teljes szívből vigadok,
 Mindörökké nevednek
 Dicséretet éneklek.
/7
#D57492FC
 Mert megkegyelmezél nékem,
 A pokolból kivőd lelkem;
 Feltámasztál újonnan,
 A mélységből kihozván.
 Imé, az istentelenek,
 Uram, reám dühödtenek,
 Kik csak hatalmaskodnak,
 Semmit reád nem adnak.
/8
#DCD4FDF3
 De te, irgalmas Úr Isten,
 Igaz vagy természetedben,
 Jó, hív és engedelmes,
 Haragra késedelmes.
 Tekints reám, kegyelmezz meg,
 Uram, szolgádat segítsd meg!
 Szolgáló leányodnak
 Fiát foglald magadnak!

;A szent város dicsérete
;Genf, 1562
>87
/1
#C6754816
 Az Úr Isten az ő lakó hajlékát
 Az ő szentséges hegyén helyezi,
 És a Siont ő inkább szereti,
 Hogynem Jákóbnak akármely sátorát.
/5
#6930567C
 Azért ott énekelnek szép éneket,
 Mond az Isten, és itt örvendeznek,
 Mert én dicsőségére e helynek
 Támasztok nagy-szép élő kútfejeket.

;Könyörgés halálos veszedelemben
;Genf, 1562
>88
/1
#EF65FD85
 Úr Isten, én idvességem,
 Éjjel-nappal kiáltok hozzád!
 Könyörgésemet meghallgassad,
 És tekintsd meg nagy ínségem!
 Kegyesen hajtsd hozzám füledet,
 Értsd meg én esedezésemet!
/2
#BD3A0745
 Az én lelkem nyavalyákkal
 Teljességgel eláradt, eltölt:
 Mint aki már a sírba készült,
 És a pokolra alászáll,
 Vagyok ahhoz szinte hasonló,
 Akinek már kész a koporsó.
/3
#9A00BDB8
 Megfosztattam életemtől,
 Mint akiket agyonvertenek,
 Kik a halottak közt hevernek,
 Kikről már nem emlékezel.
 Kik eltemettetvén feküsznek,
 És te kezedből kiestenek.
/4
#BA3DF44C
 A koporsóba től engem,
 Bévetél a setét mélységben,
 Holott haragod nyom keményen;
 Elborítád szegény fejem
 Nagy árvizednek habjaival,
 Mik rám rohannak nagy zúgással.
/5
#A226D333
 Engem te megutáltatál,
 És tőlem én ismerőimet
 Elvitted, elhagytak engemet,
 És a tömlöcbe taszítál,
 Holott kemény fogságban vagyok,
 Melyből ki nem szabadulhatok.
/9
#FB0E607A
 Uram, miért vetsz el engem,
 Miért rejted el szemeidet?
 Szegény vagyok, erőm elveszett,
 Jaj, mely igen gyötrettetem!
 Ez én régi nagy ínségemben
 Előtted vagyok rettegésben.
/10
#582A50CF
 Nagy haragod reám borult.
 Nagy rettegés engem körülvett,
 És teljességgel elmerített,
 Mint az árvíz, reám tódult.
 Sanyargat engem minden dolog,
 Valamely énkörültem forog.

;Isten ígéretei beteljesednek (A Messiás-királyról)
;Genf, 1562
>89
/1
#E2DED85C
 Az Úrnak irgalmát örökké éneklem,
 És hűséges voltát mindenkor hirdetem,
 Mert mondom, hogy megáll mindörökké irgalma,
 Melyet úgy megépít, hogy megálljon mindenha,
 És hogy mind az égig erősíted, megtartod
 Te szent igazságod és a te fogadásod.
/3
#15D33B10
 Az egek hirdetik sok csuda dolgodat,
 A szent gyülekezet te igazságodat,
 Mert vajon kicsoda volna ott fenn az égben,
 Ki nagy hatalommal hozzád hasonló légyen?
 Az erős angyalok seregében vagyon-e,
 Ki e dicső Úrhoz hasonlatos lehetne?
/4
#9087B265
 Igen rettenetes e fölséges Isten,
 Őtet féli minden a szent gyűlésekben,
 Ó, seregek Ura, minden enged tenéked,
 Te nagy erős Isten, vajon ki érne véled?
 Tenálad lakozik a te nagy igazságod,
 Soha el nem múlik a te igaz mondásod!
/7
#9A3BEB97
 Boldog a nép, amely tenéked örvendez,
 Minden dolgát, Uram, ez viszi jó véghez.
 Fényes orcád előtt ezek járnak merészen,
 És a te nevedben örvendeznek szüntelen,
 Mert nagy dicsőségre őket felmagasztalod,
 És jótéteményed rajtuk megszaporítod.
/8
#B54C65FD
 Te vagy ékessége az ő erejüknek,
 Minden hatalmukat te adtad nékiek.
 A te kegyelmedből orcánkat fölemeljük,
 Tetőled, Úr Isten, mi paizsunkat vettük,
 És a mi királyunk a te fegyvered nélkül,
 Ó, Izráel Ura, nem lehet segítségül!
/20
#CD97F646
 Gondold szolgáidnak nagy gyalázatjukat,
 És hogy sok népeknek iszonyú szidalmát
 Keblemben hordozom, kik tégedet bosszantnak,
 Gyalázván homlokát fölkenett királyodnak.
 Dicsőség tenéked és áldassál, Úr Isten,
 Melyre minden néptől mondassék: Ámen, ámen!

;Isten az örökkévaló hajlék
;Bourgeois L., Genf, 1551
>90
/1
#2B8F5A42
 Tebenned bíztunk eleitől fogva,
 Uram, téged tartottunk hajlékunknak!
 Mikor még semmi hegyek nem voltanak,
 Hogy még sem ég, sem föld nem volt formálva,
 Te voltál és te vagy, erős Isten,
 És te megmaradsz minden időben.
/2
#8F427214
 Az embereket te meg hagyod halni,
 És ezt mondod az emberi nemzetnek:
 Légyetek porrá, kik porból lettetek!
 Mert ezer esztendő előtted annyi,
 Mint a tegnapnak ő elmúlása
 És egy éjnek rövid vigyázása.
/3
#8B29FD8E
 Kimúlni hagyod őket oly hirtelen,
 Mint az álom, mely elmúlik azontól,
 Mihelyt az ember felserken álmából,
 És mint a zöld füvecske a mezőben,
 Amely nagy hamarsággal elhervad,
 Reggel virágzik, estve megszárad.
/4
#669A67BB
 Midőn, Uram, haragodban versz minket,
 Ottan meghalunk és földre leesünk,
 A te kemény haragodtól rettegünk,
 Ha megtekinted mi nagy bűneinket,
 Titkos vétkünket ha előhozod
 És színed eleibe állítod.
/5
#38FD92E0
 Haragod miatt napja életünknek
 Menten elmúlik nagy hirtelenséggel,
 Mint a mondott szót elragadja a szél.
 A mi napink, miket nekünk engedtek,
 Mintegy hetven esztendei idő,
 Hogyha több, tehát nyolcvan esztendő.
/6
#AFFDED5F
 És ha kedves volt is valamennyire,
 De többnyire volt munka és fájdalom;
 Elkél éltünknek minden ékessége,
 Elmúlik, mint az árnyék és az álom.
 De ki érti a te haragodat?
 Csak az, aki féli hatalmadat.
/7
#71594EED
 Taníts meg minket azért kegyelmesen,
 Hogy rövid voltát életünknek értsük,
 És eszességgel magunkat viseljük!
 Ó, Úr Isten, fordulj hozzánk ismétlen!
 Míg hagyod, hogy éltünk nyomorogjon?
 Könyörülj már a te szolgáidon!
/8
#41117EB5
 Tölts bé minket reggel nagy irgalmaddal,
 Hogy jókedvvel vigyük véghez éltünket,
 Ne terheltessünk nyomorúságokkal!
 Vigasztalj minket és adj könnyebbséget,
 És haragodat fordítsd el rólunk,
 Mellyel régóta ostoroztatunk!
/9
#699695F7
 Szolgáidon láttassad dolgaidat,
 Dicsőségedet ezeknek fiain!
 Add értenünk felséges hatalmadat,
 Mi kegyes Urunk, ó, irgalmas Isten!
 Minden dolgunkat bírjad, forgassad,
 Kezeink munkáit igazgassad!

;Isten szárnyainak árnyékában
;(1539) Bourgeois L., Genf, 1542
>91
/1
#E7578C49
 Aki a felséges Úrnak
 Lakozik oltalmában,
 És e nagy hatalmasságnak
 Nyúgoszik árnyékában,
 Az ilyen nyilván mondhatja:
 Isten az én kővárom,
 Ő életemnek oltalma,
 És csak őbenne bízom.
/2
#319C2A88
 A vadásznak ő tőritől
 Téged megment féltedben;
 A pusztító betegségtől
 Megoltalmaz kegyesen;
 Téged ő kedves szárnyával
 Takargat és béfedez,
 És az ő igazságával
 Mint paizzsal védelmez.
/4
#344D4C1F
 Ha te mellőled egy felől
 Ezren is elesének,
 És tízezren jobbkéz felől:
 Mégis nem árthat néked,
 Sőt megnézelled kedvedént
 Őket szemeid előtt,
 Mondván: ő érdemünk szerént
 Ez veszély rajtunk esett.
/5
#7DAE36FE
 Azért egyedül Istenben
 Vetem hiedelmemet,
 Aki ül a magas mennyben,
 Abban vesd reménységed,
 És semmi kár téged nem ér,
 Sem nem esel veszélyben,
 És minden gonosz hátra tér,
 Házad felé sem mégyen.
/6
#46968CB7
 Mert az ő szent angyalinak
 Megparancsolta nyilván,
 Hogy téged oltalmazzanak
 Minden te utaidban.
 Tégedet ezek nagy szépen
 Ő kezükben hordoznak,
 Hogy lábad meg ne üsd kőben,
 Oly híven igazgatnak.
/7
#A8B3B06B
 Sárkányon és oroszlánon
 Minden kár nélkül járhatsz,
 Oroszlánkölykön és kígyón
 Lábaiddal tapodhatsz.
 Mond Isten: őtet megtartom,
 Mert engem szívből szeret;
 Én őtet megoltalmazom,
 Mert ismeri nevemet.
/8
#7BA6956D
 Mihelyt hív könyörgésében,
 Őtet menten segítem;
 Véle leszek ínségében,
 Melyből hamar kivészem;
 És nagy dicsőségre őtet
 Emelem, magasztalom,
 És az én segedelmemet
 Őneki megmutatom.

;Az igazságos Isten dicsérete
;Genf, 1562
>92
/1
#BFF4E938
 Ékes dolog dicsérni,
 Uram, Felségedet,
 És a te nevedet
 Énekkel magasztalni.
 Hogy ember áldja reggel
 Te nagy jó voltodat,
 És igazságodat
 Dicsérje minden éjjel.
/2
#F9E5C8AD
 Lantban és hegedűben
 És szép cimbalmokban,
 Hangos citerákban,
 Dicsértessél zengésben!
 Dolgaidon örvendek
 Hatalmadat látván;
 Kezednek csudáján
 Örömömben éneklek.
/3
#2DE01A9C
 Sok és nagy csudálatos
 Te cselekedeted,
 Mélység bölcsességed,
 Beszéded drágalátos,
 E dolgot az esztelen
 Egy szálnyit sem érti,
 És meg sem tekinti,
 Hogy ez miképpen légyen.
/4
#544D403A
 Hogy a gonoszok nőnek,
 Mint a fű a mezőn;
 Virágoznak szépen
 A sok istentelenek,
 Hogy örökké essenek
 A veszedelemben.
 Te vagy örök Isten
 Fölötte mindeneknek.
/7
#78A085C5
 Virágoznak a hívek,
 Mint a szép pálmafák,
 És mint a cédrusfák,
 Mik Libánonon nőnek.
 És az Úr hajlékában
 Ezek plántáltatnak,
 Szépen virágoznak
 Az Isten tornácában.
/8
#EE41A09F
 Ámbátor megőszülnek,
 De mindazonáltal
 Nagy szaporasággal
 Szép gyümölcsöt teremnek,
 Hogy igazságát híven
 Mindenütt hirdessék
 Az én Istenemnek,
 Kiben hamisság nincsen.

;Isten a világ dicsőséges királya
;Genf, 1562
>93
/1
#1537FEF0
 Nagy hatalmával regnál az Isten,
 Öltözvén felséges erejében,
 E föld kerekségét úgy helyezte,
 Hogy mozdulása nem lesz nékie.
/2
#14AC2217
 Királyi széked kezdettől megvolt,
 Istenséged örökké országolt.
 A folyóvizek erősen zúgnak,
 A magas habok igen harsognak.
/3
#AFFBF1BB
 És bár a tenger zúgjon erősen,
 És a habok hánykódjanak fennen,
 De ők mind semmik az Isten ellen,
 Mert ő hatalmasb a magas mennyben.
/4
#D5FC5775
 Uram, megmarad a te mondásod,
 Merő hűség a te tudományod,
 Házadnak szentség ő ékessége,
 Melynek örökké nem leszen vége.

;Az igazság diadala
;Genf, 1562
>94
/1
#02DD4008
 Ó, erős bosszúálló Isten,
 Ki bűnünkért büntetsz erősen:
 Jelentsd meg már hatalmadat!
 Mind e világnak bírája,
 Támadj fel, add meg valóba'
 A kevélyeknek jutalmát!
/2
#C37B2053
 Míglen marad ez büntetetlen,
 Míg fuvalkodnak föl kevélyen
 A gonosz istentelenek?
 Míg élnek ily vigassággal,
 És az ő gonoszságukkal
 Vajon meddig dicsekednek?
/3
#A3A66B85
 Uram, a te népedet rontják,
 És örökséged sanyargatják
 Minden irgalmasság nélkül.
 Özvegyet, árvát megölnek,
 Szegényt, jövevényt elvesztnek;
 Ezt mondják szentségtelenül:
/4
#F86A09E0
 Az Isten e dolgot nem látja,
 A Jákób Istene nem tudja,
 Amiket mi szerzünk mostan.
 Én csudálkozom rajtatok,
 Hogy ily oktalanok vagytok!
 Fontoljátok meg valóban:
/5
#EDE6B250
 Aki a fület teremtette,
 És a látó szemet szerzette,
 Hogy ne látna, sem hallana?
 A pogányok büntetője
 Titeket hogy ne büntetne:
 Ki mást tanít, hogy ne tudna?
/6
#FF6E5A3A
 Az Isten minden szívnek titkát,
 Jól tudja minden gondolatját:
 Semmirekellők, jól látja.
 Boldog, akit te pirongatsz,
 És törvényedre tanítasz,
 És kinek vagy oktatója.
/7
#6D2BB4BE
 Hogyha gonoszul lészen dolga,
 És érkezik háborúsága,
 Mindent elszenved csendesen,
 Míglen az istentelennek
 Vermet, koporsót készítnek,
 Hol örömében vég leszen.
/8
#A109F126
 Mert nem hagyja Isten ő népét,
 El nem taszítja örökségét,
 Sőt vagyon rájok nagy gondja.
 És midőn ideje eljő,
 Mindent igazán ítél ő,
 És a híveket megtartja.
/9
#65C13886
 Ki ment meg a gonosztól engem?
 És ki támad fel énmellettem
 E gonosztévő nép ellen?
 Ha Isten nem őrzött volna,
 Már régen meghaltam volna,
 És most feküdném a sírban.
/10
#1A460CF7
 Midőn mondom: ím, el kell esnem,
 Legottan megsegítesz engem
 Te nagy irgalmasságoddal.
 Mikor nagy bánatban volnék,
 És szívemben kesergenék,
 Megvigasztalál azonnal.
/11
#5E35AFC4
 Ítéletedhez hogy férhetne
 Az istentelenek törvénye,
 Kik jót hamisra fordítnak?
 Nagy sereggel összegyűlnek,
 Hogy az igazat megöljék
 És ártatlan vért ontsanak.
/12
#E820684E
 De én csak az Istenben bízom,
 Ő énnékem erős kővárom,
 És ezeket megbünteti,
 Az ő számtalan bűnükért
 És gonosztéteményükért
 Az Isten őket elveszti.

;Hívogatás Isten imádására
;Bourgeois L., Lyon, 1547
>95
/1
#1B7397A0
 Jertek, örvendjünk mindnyájan
 Az Úrban, mi kősziklánkban!
 Vigadozzunk szép énekekkel,
 Menjünk színe eleiben,
 És magasztaljuk kegyesen
 Örvendetes dicséretekkel!
/2
#79795C08
 Mert hatalmas az Úr Isten,
 És nagy király mindeneken,
 Ő a fejedelmeknek ura.
 Ez egész földnek mélységét
 És a hegyeknek tetejét
 Hatalmas kezébe foglalja.
/3
#28C9E786
 Övé a tenger, amelyet
 Erős kezével teremtett,
 A szárazt is ő szerzé szépen.
 Az Úrnak, jer, imádkozzunk,
 Néki térdet, fejet hajtsunk,
 Aki minket teremtett bölcsen!
/4
#3F35D950
 Mert ő Istenünk és Urunk,
 Mi pedig juhai vagyunk;
 Ő legeltet minket mint nyáját.
 Lágyítsátok szíveteket:
 Hogy ha hívand ma titeket,
 Vajha meghallanátok szavát!

;Isten a pogányoknak is Ura
;Genf, 1562
>96
/1
#EB977C0F
 Énekeljetek, minden népek,
 Új éneket az Úr Istennek!
 E földön őnéki minden
 Dicséretet énekeljen,
 Jóvoltát hirdesse mindennek.
/2
#1175B96F
 A pogányok közt dicsőségét,
 Hirdessétek jótéteményét!
 Mert nagy és erős az Isten,
 Hát inkább tisztelje minden,
 Hogynem mint egyéb isteneket!
/3
#DB52B780
 A pogányoknak sok istenük
 Mind merő bálvány, ha megnézzük.
 De Isten mennyet teremte,
 Nagy hatalom jár előtte,
 Minden dolgai felségesek.
/4
#694B9E6C
 Nagy dicsőségét látja minden
 Ő tiszteletes szent helyében.
 Jertek tehát, minden népek,
 Adjunk hálát az Istennek,
 Mert nagy az ő dicsőségében!
/5
#D1A13699
 Az Úrnak dicsőség adassék,
 És az ő neve dicsértessék!
 Adjunk őnéki hálákot,
 Minden kedves ajándékot
 Tornácaiban szent helyének!
/6
#41138658
 Jer, menjünk az Úr eleiben
 És imádjuk szentséges díszben!
 Templomában szentségének
 Tőle mindenek féljenek
 Ez egész föld kerekségében.
/7
#8E4E8F7E
 Ne rejtsétek a pogányoktól,
 Hogy e nagy Úr Isten országol!
 E földet megerősíti,
 Megrendülni nem engedi,
 Népét ítéli igazságból.
/8
#7A8AF6BD
 Örvendjen az ég hangossággal,
 A föld örüljön vigassággal,
 A tenger zúgjon, a mező
 Zengedezzen és az erdő
 Az Úr előtt nagy hál'adással!
/9
#90BB9824
 Eljő az Isten törvényt tenni
 És mind e földet megítélni!
 És igazsággal e földet,
 Tisztasággal a népeket
 Ítéli: a jókat megmenti.

;Isten eljön ítéletre
;Genf, 1562
>97
/1
#99BBEB45
 Az Úr Isten regnál,
 Ő az erős király,
 Kin mind e föld örvendjen,
 Minden sziget örüljön!
 Felhő áll előtte,
 És homály körüle,
 Ő törvényszékinek,
 És ítéletinek
 Áll erős törvénye.
/2
#EDF4C33E
 Tűz megyen előtte,
 Mellyel ellensége
 Szörnyen megégettetik
 És hamuvá tétetik.
 Hírük sem lesz soha,
 Az ő villámlása
 Fénylik világoson
 Ez egész világon,
 A föld fél, ha látja.
/3
#B1363BDB
 Az Isten színének
 Előtte a hegyek,
 Mint viasz, elolvadnak,
 Mert Ura e világnak.
 Hirdetik az egek
 Mindenféle népnek
 Az ő igazságát
 És dicső nagy voltát
 Az ő hatalmának.
/4
#D6FB84AB
 Hát pironkodjanak,
 Kik bálványt imádnak
 És tisztelnek képeket,
 Mikre vetik szívüket!
 Ti minden istenek,
 Őtet tiszteljétek,
 Állván széke előtt,
 Kit a Sion hallott,
 És örült e hírnek.
/6
#D6A5A1D8
 Ti, Istent szeretők,
 Gonoszt gyűlöljetek,
 Hogy részetek ne légyen
 Hamis cselekedetben.
 Mert az ő szolgáit,
 Megmenti híveit
 A gonosz kezéből;
 Ő nagy erejéből
 Megtartja népeit.
/7
#13F1B812
 Szentihez világát,
 Nyújtja nagy irgalmát;
 Kik egyenes szívűek,
 Tőle örömet nyernek.
 Szent hívek, ez Úrban
 Örvendjetek vígan,
 És az ő szentségét,
 Dicsőséges nevét
 Áldjátok mindnyájan!

;Népek szabadítója az Úr
;Bourgeois L., Strasbourg, 1545
>98
/1
#12D5ABC8
 Énekeljetek új éneket
 Az Úr Istennek örömmel,
 Mert nagy csudákat cselekedett
 Karjának nagy erejével.
 Mivelünk az ő üdvösségét
 Kétség nélkül megérteté,
 Szent igazságát, kegyességét
 Minden népnek kijelenté.
/2
#F3CACFB6
 Meggondolá irgalmasságát,
 És kegyességét tekinté,
 És az ő nagy hűséges voltát
 Az Izráelhez téríté.
 A nekünk küldött üdvösséget
 Ez egész földkerekségen
 Nyilván meglátta minden nemzet.
 Örvendjen néki hát minden!
/3
#9CB58B5E
 Örvendjetek és vigadjatok,
 Mondjatok szép zsoltárokat!
 Cimbalmokkal hangicsáljatok,
 Zendítsetek citerákat!
 A trombitákat fújjátok meg,
 E király előtt zengjetek!
 Csendüljetek, zúduljatok meg
 Tengeren-földön mindenek!
/4
#14453B79
 Az Úr előtt a folyóvizek
 Örvendezzenek mindnyájan,
 A magas hegyek az Istennek
 Tapsoljanak víg voltukban!
 Mert íme, eljő törvényt tenni,
 Megítél e földön mindent,
 Mind e világot jól rendeli,
 A tiszta igazság szerint!

;A háromszor szent Isten dicsérete
;Genf, 1562
>99
/1
#CD9948CA
 Az Úr országol
 És regnál nagy jól,
 A nép megrémül,
 Hogy ő ott fenn ül
 A Kérubimon,
 Ki előtt nagyon
 Félnek és rettegnek
 E földön mindenek.
/2
#E1C2467B
 Nagy az Úr Isten
 Ő erejében.
 A Sion hegyen
 Minden népeken
 Vagyon hatalma,
 Minden őt áldja,
 Mert nagy az ő neve
 És dicső szentsége.
/3
#5104EC6F
 Ez erős Úrnak,
 Mint jó királynak,
 Nem kell hamisság,
 Csak az igazság.
 Jó ítéletet
 A Jákóbbal tett,
 Kit ő igen szeret,
 Mint választott népét.
/4
#A99B13FF
 Istent áldjátok,
 Magasztaljátok;
 Hajtsatok térdet
 Zsámolya előtt!
 Mert ő szentséges!
 Áron és Mózes,
 Az ő szent papjai
 Könyörögtek neki.
/5
#8461C161
 És a Sámuel
 Könyörgésével
 Istent keresé
 Nagy szükségébe'.
 És kérésükben
 Meghallá Isten,
 Megadá nékiek,
 Amit tőle kértek.
/6
#D7C8A0F9
 Fönn a felhőben
 Oszlop képében
 Őket vezérlé
 A puszta helybe',
 Kik ő törvényét
 És szent igéjét
 Híven megőrizték,
 Frigyét kedvellették.
/7
#EEF683A5
 Meghallád, Isten,
 Ő kérésükben,
 Hozzájuk térél,
 És megengedél
 Kegyességedből.
 De hogy bűnükből
 Ők ki nem térnének:
 Megbüntettetének.
/8
#A8F7B125
 Áldjátok őtet
 Mint Istenünket,
 Térdet hajtsatok
 És imádjátok!
 A Sion hegyén,
 Ő lakóhelyén,
 Dicsértetik itten,
 Mert szent az Úr Isten!

;Hálaének a templomban
;Bourgeois L., Genf, 1551
>100
/1
#84B801B9
 E földön ti minden népek,
 Az Istennek örvendjetek,
 Előtte szép énekekkel
 Szolgáljátok őt víg szívvel!
/2
#54092148
 Tudjátok, hogy ez az Isten,
 Ki minket teremtett bölcsen,
 És mi vagyunk ő népei
 És ő nyájának juhai.
/3
#28F86FCD
 Ő kapuin menjetek be,
 Hálát adván szívetekbe!
 Jer, menjünk be tornácába,
 Néki nagy hálákat adva!
/4
#4CEFD317
 Mert nagy az ő kegyessége,
 És megmarad mindörökre,
 És ő hűsége mindenha
 Megáll és el nem fogy soha.

;Uralkodó tüköre (Messiási zsoltár)
;Bourgeois L., Strasbourg, 1545
>101
/1
#5A9391E4
 Mindennek előtte irgalmasságról,
 Lészen éneklésem az igazságról,
 És dicséretet mondok szüntelen
 Néked, Isten.
/2
#F3333F59
 Okossággal járok minden utamban,
 Vajon mikor jössz már el, Isten, hozzám?
 Hogy házamat híven vezérlhessem,
 Igyekezem.
/3
#58C095F8
 Én semmi gonosz dolgot nem kedvelök,
 De minden csalárdságot én gyűlölök,
 Ezekre semmiképpen nem vetem
 Az én kezem.
/4
#0634EED1
 A hamis szívű távol menjen tőlem,
 A gonosz emberhez nem lészen kedvem,
 Nem jön a csalárd ember előmben
 Semmi helyen.
/5
#A1A3BF23
 Titkon ki ő feleit rágalmazza,
 Nem lészen nálam annak maradása,
 A fuvalkodó kevélyt előttem
 Nem szenvedem.
/6
#9818FCFF
 Szemeim inkább azokra nézzenek,
 Akik e földön igazságban élnek,
 Lakjanak nálam, és mint hű szolgák,
 Szolgáljanak.
/7
#55F082E8
 A csalárd ember soha nékem nem kell,
 Én házamban annak nem adatik hely;
 A hazudozó szemeim elül Távol kerül.
/8
#B3AC606E
 Jó reggel kiirtom a gonoszokat,
 Eltörlöm e földről a hamisakat,
 Az Isten városát megtisztítom,
 Tisztán tartom.

;Imádság Sion helyreállításáért (Ötödik bűnbánati zsoltár)
;Genf, 1562
>102
/1
#00536D4A
 Hallgasd meg, Uram, kérésem,
 Tekintsd meg esedezésem!
 Beszédem jusson hozzád,
 Ne rejtsd el tőlem orcád!
 Hajtsd énhozzám te füledet,
 Enyhítsd meg nagy ínségemet!
 Midőn kiáltok, Úr Isten,
 Siess, hallgass meg kegyesen!
/2
#DF7D89D3
 Mert napjai életemnek
 Oly hirtelen elkelének,
 Mint a füst és a pára,
 És mint a tűzhely pora.
 Minden csontom úgy elszáradt,
 Szívem, mint a fű, elhervadt,
 Úgy, hogy az én ételemet,
 Elfelejtem kenyeremet.
/3
#22F9562D
 Bőröm csontjaimhoz ragadt
 Keserves siralmam miatt
 Ez iszonyú ínségben.
 Olyatén lettem szintén,
 Mint pelikán a pusztában
 Sír és kiáltozik árván;
 Ollyá lettem, mint a bagoly,
 Mely a kietlenben huhol.
/4
#88607720
 Olyan vagyok árvaságban,
 Mint a veréb eszterhában,
 Amely egyedűl maradt.
 Ellenség reám támadt.
 Gyaláznak engem naponként!
 És kik bosszantnak óránként,
 Ha kinek veszélyt kívánnak,
 Példát énrólam formálnak.
/5
#782F7352
 Porhamu kenyerem nékem,
 Melyet étel gyanánt észem;
 Italom könnyeimmel
 Elegyítem, mint vízzel,
 A te nagy haragod miatt,
 Melynek tüze úgy fellobbant,
 Hogy engemet fölemeltél,
 Ismét a földhöz ütöttél.
/6
#B2742E01
 Az én időm úgy elmúlék,
 Mint az árnyék, elenyészék;
 Minden testem elasza,
 Mint a lekaszált széna,
 Amely meg nem éled többé;
 De te megmaradsz örökké,
 És a te emlékezeted,
 Mindörökre megtart híred.
/7
#E1CBAA9D
 Azért kelj fel, ó, Úr Isten,
 Haragod ne gerjedezzen!
 Siont immáron szánd meg.
 És neki kegyelmezz meg!
 Mert már ím az idő eljött,
 És amely már sokat késett,
 Jelen vagyon az az óra,
 Melyben dolga fordul jóra.
/8
#CBDFB474
 Hogy a te szolgáid végre
 Összegyűlvén e szent helyre,
 Szeressék köveit is,
 Kedveljék még porát is.
 És aztán a pogány népek,
 Uram, nevedet rettegjék;
 E földön minden királyok
 Megvallják, hogy te vagy Urok.
/9
#A0C8CE4C
 Mert az Úr jól megépíté,
 A Siont megékesíté,
 Dicsőségét is itten
 Megmutatá ismétlen.
 Nyomorultak sok panaszát,
 Meghallgatá nagy siralmát,
 Kérésüket meg nem veté,
 Sőt kegyesen megtekinté.
/10
#24C9C729
 Ember ezt híven fölírja,
 Hogy megmaradjon mindenha;
 Légyen emékezete
 Nemzetségről nemzetre,
 Hogy a jövendő nemzetek,
 Kik még e földre születnek,
 A mennnybéli Istent áldják,
 E dolgáért magasztalják:
/11
#E8C99B3B
 Hogy alánéz onnan felül
 Az ő magas szent helyébül;
 A mennyből alánézell,
 E földre letekintell,
 Hogy meghallja kérésüket,
 A foglyoknak ínségüket,
 És azokat ő megtartja,
 Kiket ítéltek halálra.
/12
#B065C070
 Hogy szent neve az Istennek
 A Sionban dicsértessék,
 És a Jeruzsálembe’
 Áldassék dicsősége,
 Midőn, a népek seregben
 Összegyűlnek e szent helyben,
 A királyok is itt lesznek,
 És szolgálnak az Istennek.
/13
#7FBC4FB9
 Leveré az én erőmet
 Ez útban, és élletemet
 Megrövidíté, és én
 Így szólék keservesen:
 Ne hagyj elvesznem, Úr Isten,
 Életemnek közepében!
 Mert mindörökké megvannak
 Esztendeid, el nem fogynak.
/14
#91AEE972
 E földet te teremtetted,
 És ekképpen helyeztetted;
 Csak te kezed munkája
 Az égnek alkotmánya.
 De ezek mind elveszendők!
 Uram, te megmaradsz, de ők
 Mint a ruha, mind elkopnak,
 Szépségükben meg nem állnak.
/15
#F949A89C
 És midőn, Uram, akarod,
 Őket úgy elváltoztatod,
 Mint az elviselt ruhát,
 Amely tél-túl megszakadt.
 Te vagy, aki régen valál,
 Voltod változás nélkül áll;
 Te esztendeidnek vége
 Nem lészen soha örökre.
/16
#836DA272
 Azért a te szolgáidnak
 Fiai megszaporodnak,
 És megmaradnak végig,
 Azaz örök időkig.
 A néked szolgáló hívek
 És minden ő nemzetségek
 Megállnak és megmaradnak,
 És örökké el nem fogynak.

;A könyörülő Isten
;(1539) Bourgeois L., Genf, 1542
>103
/1
#6C2131A6
 Áldjad, lelkem, Uradat, Istenedet,
 Minden énbennem dicsérje szent nevét
 És az ő mondhatatlan jóvoltát!
 No, dicsérd, lelkem, és az Urat áldjad,
 Feledékenységben el ne hallgassad
 Ő jótéteményinek sok voltát.
/2
#06B05BF4
 Adj hálát néki, aki bűneidet
 Megbocsátja, gyógyítja sérelmidet,
 Kiment minden nagy bajodból híven,
 És életedet veszélytől megmenti,
 A halál veszedelmitől megőrzi:
 Irgalmával megkoronáz szépen.
/3
#853EE45D
 Aki lelkedet kegyesen táplálja,
 Ami kell szádnak, bőséggel megadja;
 Mint a sast, megifjít és megújít.
 És akik méltatlanságot szenvednek,
 Tőle kegyesen azok megmentetnek,
 Hozzájuk nyújtja az ő jókedvit.
/4
#1026F705
 Ő útait megmutatá Mózesnek,
 Azonképen az Izráel népének
 Megjelenté nagy csudadolgait.
 Mert ő irgalmas, kegyes és kegyelmes,
 Nagytűrhető, jó, hű és engedelmes,
 Haraggal nem terheli szolgáit.
/5
#43EDE34D
 És hogyha bűnünkkel ő megbántatik,
 Néha kegyelme tőlünk megvontatik:
 De nem haragszik reánk örökké,
 És nem mindenkor perel ő mivélünk,
 Nem büntet a mi bűnünk szerint bennünk,
 Oly hajlandó ő a kegyelemre.
/6
#91D9FBA3
 Úgy felépíti azokhoz irgalmát,
 Kik szeretik őt, félik ő hatalmát,
 Mily magas a szép ég e föld felett.
 Minden bűnünket mitőlünk oly messze
 Ő elfordítja, szinte mint mely messze
 A napkelettől a napenyészet.
/7
#063CD6A0
 Mint az atya fiaihoz kegyelmes,
 Ő is azokhoz igen engedelmes,
 Kik őt igazán félik, tisztelik,
 Mert jól tudja, mily gyarló a mi voltunk,
 És hogy mi oly romlott emberek vagyunk,
 Mint a por, mely a széltől hintetik.
/8
#4222C1FF
 Ember élete a fűhöz hasonló,
 Felnő és zöldül, de hamar elmúló,
 Mint a gyenge virág a sík mezőn,
 Melyet mihelyt megfúval a meleg szél,
 Elhull és hervad, ékessége elkél,
 Ember nem tudja, hol volt, hova lőn.
/9
#B82CA30C
 De az Úr kegyelme örökké megáll
 Azokon, kik őt félik igazsággal,
 És firól fira terjed irgalma
 Azokon, kik megtartják ő kötésit,
 Akik gyakran megemlítik törvényit,
 És azok szerint járnak mindenha.
/10
#4166702D
 Az Úr a mennyben helyeztette székit,
 Hol gyakorolja igaz ítéletit;
 Ő országlása kihat mindenre.
 Isten angyali, áldjátok az Urat,
 Ti erős lelkek, kik ő akaratját
 Megteszitek, járván ő kedvére!
/11
#7FBD1182
 Dicsérjétek őt, minden ő seregi,
 Kik a mennyekben szolgáltok őnéki
 És cselekszitek szent akaratját!
 Áldjátok az Urat, minden állatok,
 És birodalmát fenn magasztaljátok,
 Örökké áldjad, lelkem, az Urat!

;A teremtő dicsősége
;Bourgeois L., Genf, 1542
>104
/1
#4B8447DD
 Áldjad, lelkem, az Urat és tisztöld,
 Dicsőségével rakva menny és föld.
 A te felséged, Uram, nagy és erős,
 A te ékességed nagy szép és fényös.
 Te öltözeted ékes és tiszta,
 Szép világosság származik róla.
 Az egeket szélesen kiterjesztéd,
 Mint egy kárpitot, úgy felékesítéd.
/2
#1C36AB80
 A vizet körüled, mint kamarát,
 Jól megépítéd, mint szép palotát.
 A felhőkön úgy jársz, mint egy szekéren,
 A szelek szárnyukon hordoznak szépen.
 Angyalidat sebes széllé teszed,
 Száguldó postáidként kiküldöd;
 Mennydörgés, tűzláng, villámlás előtted,
 Mint kész szolgáid, úgy függnek tetőled.
/3
#C0E00F93
 A föld fundámentomát megvetéd,
 Amelyre erősen helyeztetéd,
 Hogy azon mindenkoron megállana,
 És helyéből soha ki nem mozdulna.
 Mely azelőtt a nagy mélységekben
 Vízzel mint ruhával volt elfödvén;
 Őrajta a nagy vizek felül folytak,
 Kik a nagy hegyeken is felülmúltak.
/4
#9FD13796
 De mihelyen te megfeddéd őköt,
 Megfutamának feddésed előtt;
 Hogy meghallák mennydörgését te szódnak,
 A földről sietvén elszaladának.
 A nagy hegyek fölemelkedének,
 És a mély völgyek mind kitetszének;
 Minden megtartja ő tulajdon helyét,
 Melyet Felséged nékiek engedett.
/5
#4122FDA4
 Határát a nagy tenger megtartja,
 Azt semmiképen által nem hágja,
 Hogy többé e fölemelkedett földek
 Ő vize miatt el ne merüljenek.
 A kútfejeket a szép völgyeken
 Te Felséged rendelte nagy szépen;
 A patakok, mik innen kicsordulnak,
 A hegyek között zengedezve folynak.
/6
#6BF5847F
 Arra, hogy a mezőn járó barmok
 És az erdőn lakó vadállatok
 E vizekben és a szép kútfejekben
 Igyanak és megújuljanak szépen.
 E helyen ember hall ékes zengést,
 Égi madaraktől szép éneklést,
 Akik a hűvös, zöld ágakon ülnek,
 És gyönyörűségesen énekelnek.
/7
#93718F1E
 A nagy hegyeket te onnan felül
 Megnedvesíted a magas égbül,
 Hogy a te jó, bőséges kezeiddel
 E földet bétöltsed szép gyümölcsökkel.
 És a barmoknak szénát adsz bőven,
 Miket teremtesz széjjel a réten,
 És az ő munkájából az embernek,
 Füvet és búzát adsz ő életének.
/8
#F30D6DFF
 Vígasságra bort adsz az embernek,
 És kenyért adsz, melytől erősödjék,
 És olajt bőven engedsz őnékie,
 Hogy az ő színe szépen fényljék tőle.
 Te megöntözöd az élőfákat,
 És a szép növendék cédrusokat,
 Miket a Libánon hegyen ültettél,
 Megnedvesíted szép hasznos esőkkel.
/10
#824101DD
 A holdat helyeztetted az égre,
 Hogy az esztendőt ossza részekre;
 Tudja a fényes nap, hol kell lemenni,
 És tudja útját oda vezérelni.
 Te szerzetted a nagy setétséget,
 Hogy elválassza naptól az éjet,
 És éjszaka az erdei sok vadak
 Az ő barlangjukból előballagnak.
/12
#6712FC37
 z ember reggel fölkel idején,
 És tiszti szerint munkára mégyen,
 Szántóföldre, rétre, kertbe, szőllőben
 És munkálkodik ott mind estéiglen.
 Uram, mily nagy a te bölcseséged,
 És bölcseséges cselekedeteid!
 Nagy dolgaidnak, Uram, nincsen száma,
 Mikkel mind e föld kereksége rakva.
/14
#021DAF56
 Minden állat, Uram, tereád néz,
 Szemeit emelvén Felségedhez,
 Eledelt adsz nékik ő idejükben.
 Elődbe gyűlnek néked esedezvén.
 És te mind megelégíted őket,
 Ha megnyitod bőséges kezedet:
 Nem leszen senkinek semmi szüksége,
 Mert te jól tudod, kinek mi szüksége.
/15
#42818C68
 De ha elrejted orcádat tőlek,
 Megrettennek és ottan ledőlnek.
 Ha lehelletit megvonszod azoknak,
 Ottan elhullnak és porrá változnak.
 De ha rájok lehellesz ismétlen,
 Legottan mind megélednek szépen,
 Mert tetőled ők megelevenednek,
 És te megújítod színét e földnek.
/16
#9AED47F0
 Az Úrnak légyen örök tisztesség,
 És övé légyen minden dicsőség!
 Örvend az Úr ő csuda dolgaiban,
 Gyönyörködik ő minden munkáiban.
 Tekintésétől a föld megrémül,
 És az ő haragjától megrendül;
 Reszketvén a nagy hegyek füstölögnek,
 Hogyha az Úrtól ők megillettetnek.
/17
#5F14533C
 Dicséretet az Úrnak éneklek,
 Valamíg én e világon élek;
 Az Úr Istent én egész életemben
 Dicsérem és áldom szép éneklésben.
 De viszontag azt kérem őtőle,
 Hogy éneklésem jókedvvel vegye,
 És aztán teljes szívből örvendezek,
 Szép énekeket mondván szent nevének.

;Isten csodatettei velünk
;Genf, 1562
>105
/1
#76A1E028
 Adjatok hálát az Istennek,
 Imádkozzatok szent nevének!
 Hirdessétek dicséretét
 És minden jótéteményét!
 Beszéljétek a nép előtt
 Nagy csudáit, melyeket tött!
/2
#800921D5
 Néki vígan énekeljetek,
 Sok csuda dolgát dicsérjétek!
 Magasztaljátok szent nevét,
 Kik szívből félitek őtet!
 Örvendjen azoknak szívek,
 Kik az Úrról emlékeznek!
/3
#C289F61E
 Keressétek e kegyes Urat
 És az ő színét és hatalmát!
 Meggondoljátok dolgait,
 Ne felejtsétek csudáit!
 Ítéletit hirdessétek,
 Melyek ő szájából jöttek!
/23
#6C30E6CC
 Népét vígsággal ő kihozta,
 Választott népét vigasztalta.
 A pogányok tartományát,
 Ezeknek adta országát,
 Mit kezükkel munkálkodván,
 Szerzettek volt ez országban.
/24
#8AD9CB67
 Ezt nékiek azért mívelte,
 Hogy gondjuk légyen törvényére,
 Hogy fogadják meg ő szavát,
 Megtartsák parancsolatát,
 És örökké megőrizzék,
 Melyért dicséret Istennek!

;Isten kegyelme népéhez
;Genf, 1562
>106
/1
#64C36D53
 Az Urat áldjátok, mert jó,
 Irgalma örökkévaló!
 Vajon kicsoda mondhatná ki
 Az ő nagy erős hatalmát?
 Sok és nagy az ő dicséreti,
 Melynek ki tudhatná számát?
/2
#A2F8DF41
 Boldog, aki az Úr szavát,
 Megőrzi parancsolatát.
 Uram, énrólam emlékezz meg!
 Népedhez való kedvedért,
 Kérlek, engemet látogass meg
 Üdvözítő szerelmedért!
/3
#13C23C10
  Javaival hogy élhessek
 Te választott híveidnek,
 És hogy szívem örvendezhessen
 Örömén a te népednek,
 És örökséged örömében
 A te népeddel örvendjek.
/26
#9BB12E51
 Az Úr felmagasztaltassék,
 Istene az Izráelnek:
 Dicsértessék az ő szent neve!
 És hogy örökké úgy legyen,
 Minden nép ígyen szóljon erre:
 Az Úrnak dicsőség! Ámen.

;A megváltottak hálaéneke
;Bourgeois L., Lyon, 1547
>107
/1
#D8689468
 Dicsérjétek az Urat,
 Mert nagy ő jóvolta,
 És örökké megmarad
 Az ő nagy irgalma.
 Akik megváltattak
 Őáltala kegyesen,
 Kíntól megtartattak,
 Őtet dicsérjék híven.
/2
#64FFD685
 Kiket ő támadatról
 És napenyészetről,
 Dél felől és északról
 Bégyűjte sok földről,
 Kik a szörnyű pusztán
 Idestova bujdostak,
 Semmi várost ottan
 Lakásra nem találtak.
/3
#53FD135C
 Holott nekik nem vala
 Ételük, italuk,
 Min lelkük elbágyada,
 Nagy bú szállott rájuk.
 Ez ő ínségükben
 Istenhez kiáltának,
 Kitől nagy kegyesen
 Megszabadíttatának.
/4
#EEBF6CEA
 És igaz úton őket
 Nagy szépen hordozá,
 Holott lelnének helyet,
 Városra juttatá.
 Ezek hát víg szívvel
 Az Úr Istent dicsérjék,
 Minden népnek széjjel
 Nagy csudáit beszéljék.
/5
#8A590E69
 Hogy a szomjú lelkeket
 Megitatá vízzel,
 És mind az éhezőket
 Bétölté étellel,
 Kik a nagy setétben
 Fogságban hevertenek,
 Halálos ínségben
 Nagy vasakat viseltek.
/6
#D7E8FC2E
 Mert az Isten igéit
 Ők meg nem gondolák,
 És az ő jótanácsit
 Megveték, csúfolák,
 Melyért ők keményen
 Megostoroztatának,
 Segedelmük nem lőn,
 Midőn így kínlódnának.
/7
#15BA0426
 De, hogy ő szükségükben
 Kiálták az Urat,
 Megmenté őket menten,
 Viselvén gondjukat.
 Ő nagy hatalmából
 Fölbontá kötelüket,
 Halál árnyékából
 Elkibocsátá őket.
/8
#4C6C3493
 Ezek áldják az Istent,
 Dicsérjék kegyelmit,
 Minden nép közt óránként
 Hirdessék csudáit.
 Istennek szívükből
 Áldozzanak hűséggel,
 Csudatételiről
 Énekeljenek széjjel.
/9
#28C219FB
 A bolondok kínlódnak
 Az ő bűnük miatt,
 Az Isten haragjának
 Érzik bennük súlyát.
 Sem étel, sem ital
 Nem kell semmire nékik,
 Mert a szörnyű halál
 Szemük előtt forgódik.
/10
#DC2D4025
 Mihelyt ők kiáltának
 A kegyes Istenhez,
 Tőle megtartatának,
 Lőn nékik kegyelmes.
 Mihelyt ő egy szót szól,
 Megjobbul az erőtlen,
 A beteg meggyógyul,
 Mert megmenti az Isten.
/11
#290A6AE3
 Ezek áldják az Istent,
 Dicsérjék kegyelmit,
 Minden nép közt óránként
 Hirdessék csudáit.
 Istennek szívükből
 Áldozzanak hűséggel,
 Csudatételiről
 Énekeljenek széjjel.
/12
#F544FE51
 Hajókon akik járnak
 A széles tengeren,
 Nagy veszélyben forognak,
 Kereskedést űzvén,
 Azok értik dolgát
 És nagy hatalmát látják,
 Számtalan csudáját
 A mélységről gondolják.
/13
#2A08EDA3
 Hogyha ő a szélnek szól,
 Megindul legottan,
 Nagy hirtelen felzúdul,
 S habokat széjjelhány,
 Melyek felütköznek
 És az eget verdesik,
 Ismétlen ledőlnek;
 Rettegnek, akik nézik.
/14
#859F14C6
 Széjjel úgy tántorognak,
 Mint egy részeg ember,
 Semmihez nem kaphatnak,
 Mert széles a tenger.
 De ha e veszélyben
 Az Istenhez kiáltnak,
 Megmenti kegyesen,
 Hogy ne hánykolódjanak.
/15
#4658966E
 Jó időt támaszt ismét,
 A tengert enyhíti,
 Felemelt habok vizét
 Szépen lefekteti.
 Ott igen örvendnek
 A tenger csöndességén,
 Ha a partra értek,
 Amit kívántak szintén.
/16
#A8C795E5
 Ezek áldják az Istent,
 Dícsérjék kegyelmit,
 Minden nép közt naponként
 Hirdessék csudáit.
 És dícsérjék őtet
 A gyülekezetekben,
 Magasztalják nevét
 A vének seregében.
/17
#D0D2CB42
 Ő a folyóvizeket
 Szintén kiszárasztja,
 A csurgó kútfejeket
 Szárazra fordítja.
 A gyümölcsös földet
 Meddő parlaggá teszi,
 Mert itt sok bűnöket
 Tettek, és azt bünteti.
/18
#92450063
 Esőkkel nedvesíti
 A száraz földeket,
 Folyással ékesíti
 A sovány helyeket.
 És az ily földeket
 Adja éhező népnek,
 Hogy itt lakóhelyet
 És várost építsenek.
/19
#009E8133
 E földet ők bevetik,
 És szőlőt ültetnek,
 És szép fű nevekedik,
 Hasznára a népnek.
 És őket megáldja,
 Hogy szaporodhassanak,
 És fogyatkozása
 Ne legyen a csordának.
/20
#38544D28
 De elfogynak ismétlen,
 És sanyargattatnak,
 Elynomatnak szertelen,
 És szorongattatnak.
 A fejedelmeket
 Megvetettekké tészi,
 A pusztában őket
 Út nélkül széjjelviszi.
/21
#A43D3D69
 Kegyesen fölemeli
 Ő a nyomorultat,
 Cselédit kiterjeszti,
 Mint a sereg nyájat.
 A jók, kik ezt látják,
 Örvendez az ő szívek;
 De szájukat dugják,
 Akik gonoszul élnek.
/22
#FECD4C98
 Akinek esze vagyon,
 Megnézze ezeket,
 És meglátja jó módon
 Az Isten kegyelmét.

;Isten jósága és hűsége (Reggeli ének)
;Bourgeois L., Genf, 1562
>108
/1
#404BB607
 Úr Isten, kész az én szívem,
 És azon vagyon én lelkem,
 Hogy tenéked énekeljen,
 Dicséretet zengedezzen.
 Nosza, lantok és citerák,
 Zendüljetek fel muzsikák,
 Mert igyekezem jó reggelen
 Így menni az Úr eleiben.
/2
#BFBCECE6
 Dicsérlek, Uram, tégedet
 Minden nemzetségek előtt,
 Tisztellek szép énekekkel
 Minden nép előtt víg szívvel.
 Mert a te kegyelmességed
 A széles égre kiterjed,
 Felségednek szent igazsága
 A felhőket mind felülmúlja.
/7
#8FE133F6
 Légy minekünk segítségül,
 Őrizz meg ellenséginktül,
 Mert az emberi segítség
 Hiábavaló epedség.
 Az Isten által minekünk
 Lészen erős győzedelmünk,
 És megszabadít ő bennünket,
 Megtapodja ellenségünket.

;Panasz az istentelenek ellen
;Bourgeois L., Genf, 1551
>109
/1
#CD3C4960
 Ó, Úr Isten, én dicsőségem,
 Ne hallgass, ne felejts el engem!
 Mert rágalmaz az istentelen,
 Száját reám tátotta szörnyen,
 Hazugságot szól ellenem,
 Nyelvével sérteget engem.
/2
#FF638397
 Ok nélkül rólam gonoszt szólnak,
 És nagy ellenségüknek tartnak;
 Azért, hogy én őket szerettem,
 Kegyetlenül gyűlölnek engem;
 Én csak Istenhez szüntelen
 Fohászkodtam ez ínségben.
/13
#C4F95E86
 Irgalmaddal biztatom lelkem,
 Szent nevedért őrizz meg engem,
 Mert szegény szűkölködő vagyok,
 Én szívemnek fájdalmi nagyok!
 Ím, el kell múlnom hirtelen,
 Mint az árnyék a setétben.
/15
#2010560F
 Ez én nagy keserűségemben
 Csúfolnak és gyaláznak szörnyen;
 Fejüket rázzák, midőn látnak,
 És engem gúnyolnak, bosszantnak.
 Azért, Úr Isten, segíts meg,
 Nagy kegyességedért tarts meg!
/17
#593DEE20
 Az Úr Istent én az én számmal,
 Dicsérem szép énekmondással,
 Magasztalom őtet szüntelen,
 Mert ő könyörül a szegényen,
 És azok ellen megtartja,
 Akik ítélik halálra.

;Krisztus örök királysága és papsága
;Bourgeois L., Genf, 1551
>110
/1
#16796F14
 Az Úr Isten monda az én Uramnak:
 Ülj az én hatalmamnak jobbjára,
 Míg ellenségidet, kik rádtámadnak,
 Zsámolyul vetem lábadnak alá.
/2
#118D9218
 Az Úr Sionból leküldi pálcádat
 Birodalmadnak erősségére,
 Ellenségid közt mutasd országlásod,
 Uralkodjál a népnek közötte.
/3
#62AD5F35
 Dicsőségére a te szentségednek
 A nép örvendez győzedelmeden.
 Oly sok fiaid tenéked születnek,
 Mint a hajnali harmat a földön.
/4
#C2D60289
 Mert az Úr Isten megesküdt tenéked,
 Melyet meg nem bán soha örökké,
 Melkisédeknek rendi szerint (értsed)
 Te vagy a főpap most és örökké.
/5
#27485734
 Az Úr, aki ül a te jobb kezeden,
 Ha megharagszik egykor valóban,
 A hatalmas királyokat erősen
 Hatalmával megrontja legottan.
/6
#5E172F4F
 A pogányokon ítéletit tartja,
 Megtölti a földet holttestekkel,
 Ellenségidnek fejüket megrontja,
 Országa kihat e földre széjjel.
/7
#2BA668A2
 Az úton iszik a tiszta patakból,
 Melynek vize foly nagy harsogással,
 Ennek okáért ő nagy hatalmából
 Emeli fejét nagy méltósággal.

;Hálaének Isten testi és lelki áldásaiért
;Bourgeois L., Lyon, 1547
>111
/1
#403BA801
 Hálát adok, Uram, néked,
 Teljes szívből áldlak téged
 A hívek gyülekezetében;
 Megvallom nagy dicsőséged
 És hirdetem dicséreted
 Életemnek minden rendében.
/2
#058F4674
 Nagyok az Úrnak csudái,
 És aki azt megtekinti,
 Örvendez annak az ő szíve.
 És az ő szent igazsága
 És ő dicső méltósága
 Megmarad mind örök időkre.
/3
#6900C1FB
 Csuda dolgait az Isten
 Szerzette emlékezetben
 Kegyes és irgalmas kedvéből.
 Azoknak ételt ád bőven,
 Akik őtet félik híven;
 Megemlékezik kötéséről.
/4
#F19A4E26
 Népével nagy csudákat tett,
 Hogy a pogányok örökét
 Kezükbe adá őnékiek.
 Merő hűség és tisztaság
 És állhatatos igazság
 Minden dolga az ő kezének.
/5
#42050E03
 Igaz minden ő hagyása,
 Minden ő parancsolata,
 Melyben semmi változás nincsen.
 Az ő népét ő megmenté,
 És ővélük kötést szerze,
 Mely megmarad minden időben.
/6
#74E1D911
 Szent és dicső az ő neve,
 És az Úrnak ő félelme,
 A jó bölcsességnek kezdete.
 Ki megtartja ő törvényét,
 És megőrzi szent Igéjét,
 Annak megmarad dicsérete.

;Az istenfélők boldogsága
;Genf, 1562
>112
/1
#41D976BA
 Boldog az ember, ki az Istent
 Féli, tiszteli szíve szerént,
 És az ő törvényét szereti.
 Nagy lesz e földön ő nemzete,
 Öregbül a hívek serege,
 Mert az Úr megáldja és őrzi.
/2
#CEADEEDB
 Gazdagsággal őtet meglátja,
 Mellyel bővölködik ő háza,
 Áll igazsága mindörökké.
 A híveknek a sötétségben
 Támaszt világot a jó Isten,
 Hogy láttassék rajtuk kegyelme.

;Az alázatosokat Isten felemeli
;Bourgeois L., Genf, 1551
>113
/1
#2BC53AF4
 Az Urat ti, ő szolgái,
 Dicsérjétek, mert érdemli,
 Áldjátok szent nevét mindnyájan!
 Dicsértessék szent felsége,
 Most és örökkönörökké
 Ő szent neve áldassék tisztán.
/2
#411BECBD
 Napkelettől enyészetig
 Áldassék neve mindvégig,
 Mert az Úr Isten a mennyekben
 Regnál minden pogányokon,
 Nagy dicső hatalma vagyon,
 Mely felülhat a szép egeken.
/3
#33E2FAD5
 De ki volna hasonlatos
 E mi hatalmas Urunkhoz,
 Kinél felségesb sehol nincsen?
 Aki a mennyből alánéz
 Mindenre, ami van és lesz
 Itt e földön és fenn az égben.
/4
#309543AC
 A szegényt porból felveszi,
 És a sárból felemeli,
 Állapotát felmagasztalván.
 Felülteti végezetre
 A nagy fejedelmek közé
 Az ő népe közt igazában.
/5
#34A8A260
 Az asszony szomorúságát
 Ő magtalansága miatt
 Nagy vigasságra megfordítja.
 Gyermekek anyjává tészi,
 Szép fiakkal körülvészi,
 Házát gyümölccsel szaporítja.

;Isten csodálatos szabadítása
;(1539) Bourgeois L., Genf, 1542
>114
/1
#5CD154E5
 Hogy Izráel kijött Egyiptomból,
 Az idegen népnek országából,
 Megtére Jákób háza.
 Júdát Isten magának szentelé,
 Az Izráelt országul fölvevé,
 Ő lőn nékie Ura.
/2
#A17368F5
 A tenger ezt látván hátra álla,
 A Jordán vize visszafordula,
 Mind hátra sietének.
 A hegyek szökdöstek, mint a kosok,
 És a halmok, mint a juhbárányok,
 Magasan szökdösének.
/3
#E12EF16E
 Mi lelt téged, tenger, mit térsz hátra?
 Mi lelt téged, Jordán, ki űz vissza,
 Hogy elszaladsz ily igen?
 Mit szöktetek, hegyek, mint báránykák,
 És ti halmok, mint a kis juhocskák,
 Mért szöktök ilyen fennen?
/4
#A2A32D82
 Az Úrnak haragos színe előtt,
 Jákób Istene haragja előtt
 Mind e föld megrettenjen!
 Ki a kősziklát tóvízzé teszi,
 Forrásnak útját kőben repeszti
 Hatalmas erejében.

;Egyedül Istené a dicsőség!
;Bourgeois L., Genf, 1542
>115
/1
#4C394E09
 Nem nekünk, Uram, nem nekünk engedd,
 Hanem adj nevednek dicsőséget
 A te nagy hűségedért!
 Mit csúfolnának a pogány népek,
 Mondván: hol vagyon az ő Istenek,
 Ki megmentené őket?
/2
#B56DBAEC
 De Istenünk ő nagy erejével,
 Amit csak akar, mindent megmível
 Mind mennyen és e földön.
 De sok bálványuk a pogányoknak
 Aranyból, ezüstből csináltattak
 Embereknek kezében.
/3
#2B51C32A
 Szájuk vagyon nékik, de nem szólnak,
 Szemük vagyon, de semmit nem látnak:
 Nincs elevenség bennük.
 Kezük vagyon, de semmit nem fognak,
 Lábuk is vagyon, mégsem járhatnak,
 Nem szólal az ő nyelvük.
/5
#DA658D90
 De te, Izráel, Istenben bízzál,
 És az Úr Istenhez ragaszkodjál:
 Ő paizsod tenéked!
 Áronnak háza, Istenben bízzál,
 A nagy Úr Istenhez ragaszkodjál:
 Ő megsegíthet téged!
/6
#F41CE934
 Istenfélők, bízzatok Istenben,
 Higgyetek az ő segedelmében,
 Bízzatok e paizsba'!
 Megemlít az Úr, mert szeret minket,
 És Izráelhez nyújtja kegyelmét,
 Áron házát megáldja.
/7
#8C3E3B2E
 Ő megáldja a kicsinyt, a nagyot,
 Kik őtet félik, mint ő Urokot,
 És kik szolgálják őtet.
 Az Úr titeket bőven megáldjon,
 Sok áldásival megszaporítson,
 És minden nemzedéket!
/8
#0EFB0716
 Megáld a nagy Úr Isten titeket,
 Ki teremté a mennyet és földet
 Minden szép ékességgel.
 Ő magának az eget megtartja,
 Földet az emberfiaknak adja,
 Hogy azt meglakják széjjel.

;Hála és fogadástétel halálból való szabadításért
;Genf, 1562
>116
/1
#59307323
 Szeretem és áldom az Úr Istent,
 Mert meghallgatá az én beszédemet,
 Könyörgésemre hajtá kegyes fülét,
 Melyért imádom őtet naponként.
/2
#757E3751
 Midőn a halál körülvőn engem
 És csaknem szörnyű kötelibe ejte,
 A pokol kínja engemet rettente,
 Nagy volt bánatom és nagy sérelmem:
/3
#7D0953AF
 Segítségül hívám az Úr nevét:
 Lelkemet tartsd meg! ottan megsegíte;
 Az Úr igaz, hív, és nagy ő jó kedve,
 Ő megőrzi az együgyűeket.
/4
#7D55B770
 Nagy nyavalyámban mikor én valék,
 Ottan megmenté nyomorult életem.
 Légy csendességben te azért, én lelkem,
 Látván kegyelmét szent felségének!
/5
#EFC35A39
 Te megmentél a haláltól engem,
 Szemem sírástól, lábamat eséstül;
 Az élők földén járok szünetlenül
 A te színed előtt, én Istenem.
/6
#7B98C6E7
 Hittem Istenben, mikor így szólék,
 Én szegény lelkem vala nagy ínségben,
 És ezt mondám én csüggedezésemben,
 Hogy hazugok már minden emberek.
/7
#154EB5A0
 Mit adjak az Úrnak jótettiért?
 A hálaadó pohárt én felveszem,
 És az Úr jótéteményét hirdetem,
 Szent nevét áldom segítségiért.
/8
#F225CD83
 Fogadásom az egész nép előtt
 Hálaadással megadom nékie.
 A híveknek halála, minden higgye,
 Drágalátos az Úr szeme előtt.
/9
#2E8B8B70
 Imádlak téged, Idvezítőmet,
 Ki engem választál szegény szolgádnak;
 Nékem, szolgáló leányod fiának,
 Föloldozád minden kötelimet.
/10
#75540F5C
 Áldozom néked hálaadással,
 Nevedet minden nép előtt hirdetem,
 És fogadásom, mit előtted tettem,
 Megállom minden előtt vígsággal.
/11
#1BF5F294
 A te házadnak ő pitvarában,
 És egész Jeruzsálemben tisztellek.
 Vélem együtt az Urat dicsérjétek,
 Alleluját mondjatok mindnyájan!

;Felhívás Isten dicséretére
;Bourgeois L., Genf, 1551
>117
/1
#92014079
 Az Urat minden nemzetek,
 Dicsérjétek minden népek,
 Mert nagy az ő kegyessége,
 Mit rajtunk megerősíte,
 És igazsága mindenha
 Áll és marad. Alleluja.

;Isten népének öröméneke (Templomszenteléskor)
;Bourgeois L., Strasbourg, 1545
>118
/1
#3C1ED009
  Adjatok hálákat az Úrnak,
  Mert nagy az ő kegyessége,
  És nagy volta szent irgalmának
  Megmarad most és örökre!
  Izráel, bátorsággal mondjad,
  Hogy megáll kegyelmessége;
  Irgalmas voltáról azt valljad,
  Hogy megmarad mindörökre!
/2
#6E31CB95
 Mondjad bízvást, Áronnak háza,
 Hogy az Úrnak ő jókedve
 És megmarad irgalmassága
 Most és örökkön örökre.
 Istenfélők, mondjátok mostan,
 Hogy az Úrnak kegyessége
 Megmarad ő teljes voltában
 Ettől fogva mindörökre.
/3
#84BA5105
 Az Úrhoz én nagy ínségemben
 Könyörögvén felkiálték,
 És meghallgata kérésemben,
 Tőle segedelmet nyerék.
 Az Úr Isten vagyon énvélem,
 És tőlem másuvá nem tér,
 Hát kitől kelljen nékem félnem,
 Mit árthat nékem az ember?
/4
#2D4206DA
 Jelen vagyon velem az Isten
 Köztük, kik engem segítnek;
 Megvetem őket, félre nézvén,
 Akik engemet gyűlölnek.
 Sokkal jobb az Istenben bízni,
 Hogynem valamely emberben;
 Sokkal jobb benne reménykedni,
 Hogynem a fejedelmekben.
/5
#3D79377E
 Körülvettenek a pogányok,
 De az Úr nevében bízom,
 És oly jó reménységben vagyok,
 Hogy őket mind eltaposom.
 Mint a méhek, úgy körülvettek,
 De őket megverem menten,
 Velük együtt odalesz nevek,
 Mint aszú tövis a tűzben.
/6
#1D9201C4
 Köröskörül fogtak engemet
 És kerengnek körülöttem,
 De én reménylem Istenemet,
 Hogy őket én mind elvesztem.
 Te ellenség, énrám támadtál,
 És meg akartál ejteni,
 De csak hiába munkálódtál,
 Mert éltemet Isten őrzi.
/7
#86C9AB84
 Az Úr énnékem erősségem,
 És én őróla éneklek;
 Csak ő énnékem segedelmem,
 És én őbenne reménylek.
 A hívek vígan énekelnek
 Az ő hajlékukban széjjel,
 Mert jobb keze az Úr Istennek
 Nagy erős dolgokat mível.
/8
#E4A0A08C
 Az Úr megmutatja hatalmát,
 Jobb karját ha felemeli,
 És cselekedik nagy dolgokat,
 Miket nyilván kell dicsérni.
 Én ellenségim megértsétek,
 Hogy az én dolgom ilyetén,
 Hogy én meg nem halok, de élek,
 Az Úr dolgait hirdetvén.
/9
#1E6A055C
 Az Úr megbüntete engemet,
 És igen megostoroza,
 De nem akará elvesztemet,
 És a halálnak nem ada.
 Nyissátok meg azért kapuit
 Az igazság templomának,
 Hogy bemenvén, nagy dicsőségit
 Dicsérhessem e nagy Úrnak.
/10
#480CB3D4
 Mert csak ez kapuja az Úrnak,
 Min a hívek el-bémennek,
 Mik felnyittatnak csak azoknak,
 Kik igazságodban élnek.
 Dicsérlek téged énekekben,
 Hogy engemet megalázál,
 És ismét felvevél kegyesen,
 És engem megszabadítál.
/11
#8214B455
 E kő, amit a házépítők
 Ítéltek megvetendőnek,
 Az épületbe helyezteték,
 Lőn feje a szegeletnek.
 Ez pedig az Úr Istentől lett,
 Aki ezt ígyen rendelte,
 E dolognak szemeink előtt
 Csudálatos hossza-vége.
/12
#00DF7864
 E jeles napot őmagának
 Készítette az Úr Isten,
 Azért e nap jelenvoltának
 Vígadva minden örvendjen!
 Kérlek, Uram, hogy megőrizzed
 E te Sionodat mostan,
 Bírjad és szerencséssé tégyed
 Mindennémű dolgaiban.
/13
#F4DD23E7
 Áldott, aki az Úr nevében
 Eljöve nagy dicsőséggel!
 Áldunk, dicsérünk egyetemben
 Az Úrnak háza népével!
 Adjatok hálát e nagy Úrnak,
 Mert nagy az ő kegyessége,
 És nagy volta szent irgalmának
 Megmarad most és örökre!

;Az Úr Igéjének és törvényének dicsősége
;Aranyábécének is szokták nevezni, mivel héber versei a bibliában nyolcanként – itt négyenként – egyforma betűvel kezdődnek ábécérendben
;Bourgeois L., Genf, 1551
>119
/1
#73BB63D3
 Az oly emberek nyilván boldogok,
 Kik igazsággal járnak életökben,
 Isten törvényére vagyon gondjok,
 És aszerint élnek minden időben,
 Szent bizonyságit akik megőrzik,
 És az Istent szívök szerint keresik.
/2
#E7106D07
 Boldogok azok is, mondom nyilván,
 Akik hamisságot nem cselekesznek,
 De mindenkor az Úr útaiban
 Járnak és szent ártatlanságban élnek.
 Meghagytad, hogy a te parancsodat,
 Jól megőrizzük minden mondásodat.
/3
#F40C18E4
 Vajha én oly boldoggá lehetnék,
 Hogy járhatnék a te szent útaidban
 És engedhetnék szent törvényednek!
 Ha parancsodat nézhetném valóban,
 És azt szívemben bizonnyal hinném,
 Hogy soha semmi szégyenbe nem esném!
;Boldog az ifjú, ki az Urat féli
;
/4
#C72FA2E4
 Hálát adok néked teljes szívből,
 Hogy megtanítasz te ítéletidre,
 Melyek tiszták minden hiba nélkül!
 Megtartom és gondom lesz törvényidre,
 De kérlek téged, ó, én Istenem,
 Hogy soha örökké ne hagyj el engem!
/5
#E9160D9D
 Beszéld meg, mit tégyenek az ifjak,
 Hogy élhessenek ők feddhetetlenül?
 Szent Igéd szerint útjukat szabják.
 Én téged kerestelek szüntelenül;
 Kérlek, Úr Isten, teljes szívemből:
 Eltévelyednem ne hagyj törvényedtől.
/6
#6D26EC99
 A te igédet rejtem szívemben,
 Hogy semmi bűnnel ne bántsalak téged,
 De megmaradjak te ösvényedben,
 Minden dolgomban megtartom törvényed. Ó, áldott Isten, taníts engemet, Hogy igazán értsem rendelésedet!
/7
#F473E13F
 Ítéletedet én ajakimmal,
 És a te szádnak ő kegyes beszédét
 Előszámlálom hálaadással,
 Szent kötéseddel biztatom szívemet;
 Bizonyságidon örvendez lelkem,
 Mik gazdagságnál kedvesebbek nékem.
/8
#FE1BCA11
 Szüntelen nékem gyönyörűségem
 Vagyon csak a te parancsolatidban,
 Te útaidat gyakran említem,
 Hogy el ne essem valaha azokban.
 Szent igazságodban minden kedvem,
 És te ösvényed én el nem tévesztem.
;Csak Isten törvényére tekintek
/9
#778AA345
 Cselekedd ezt szolgáddal kegyesen,
 Hogy én élhessek tovább e világban
 És szent igédet megtartsam híven!
 Én szemeimet nyisd meg világosan,
 Hogy a te törvényed megtekintsem
 És annak csudáit eszembe végyem!
/10
#5F118735
 Míglen én e földön járok-kelek,
 Ne rejtsd el tőlem parancsolatidat,
 Kívánság miatt mert elepedek,
 Igen óhajtom szent igazságodat!
 Ítéletedhez az én szívemben
 Nagy kívánságom volt minden időben.
/11
#174AC7D4
 A kevélyeket, Uram, megrontod,
 Átkozottak és büntetésre méltók,
 Akik megvetik parancsolatod.
 Forduljon el rólam ő gyalázatjok,
 Kik csak azért gyalázzák szolgádat,
 Hogy megőrzöm a te bizonyságidat.
/12
#FCCE6E04
 A fejedelmek énreám törnek,
 Ha összegyűlnek, de a te hű szolgád
 Szentségét nézi ítéletednek,
 Amely szívemnek igaz örömet ád.
 Bizonyságid nékem vigasságim
 És minden dolgomban tanácsadóim.
;Drága orvosság az Úrnak beszéde
;
/13
#FB009F16
 De lám, a porban hever életem,
 Mintha vitetném majd a koporsóba;
 Szent igéd szerint élessz meg engem!
 Midőn útaimat előszámlálva
 Felkiálték, te legott feleltél;
 Rendelésidre taníts meg jókedvvel.
/14
#90543280
 Add értenem parancsolatidat,
 Hogy elmélkedjem a te csudáidrul!
 És elmémben foglalom azokat,
 Szívem keserűség miatt kibuzdul!
 Ígéreted szerint segíts engem,
 Hogy tőled ismét megerősíttessem!
/15
#D4572202
 A hamis útról, Uram, téríts el,
 Törvényeidnek vezérelj útára,
 Min ember járhat szép csendességgel!
 Juttass kegyesen szent igazságodra!
 Ítéletedet én elválasztom,
 És igazságod szemem előtt tartom.
/16
#7691C4EA
 Bizonyságidra hajtom szívemet,
 És életemet a szerint rendelem.
 Szégyenvallástól ments meg engemet!
 Mivel most kitárod bennem én szívem,
 Parancsolatidra lesz nagy gondom,
 És víg örömmel azokat megfutom.
;Életnek ösvényén vezet a Úr
;
/17
#F1EDBC5C
 Én Istenem, taníts útaidra,
 Hogy szent törvényedre gondot viseljek,
 És azokat megtartsam mindenha!
 Adj értelmet, Uram, igazgass, kérlek,
 Hogy törvényedet őrizzem híven,
 És mindenkor megtartsam én szívemben.
/18
#6E9EBD02
 Vezess, hogy benned leljem örömöm,
 Mutasd meg parancsolatid ösvényét,
 Mert azokban igen gyönyörködöm!
 Te rendelésedre hajtsad szívemet!
 Szent bizonyságid végyem eszembe,
 És ne hagyj esnem telhetetlenségbe!
/19
#ACB39DF4
 Fordítsad el az én szemeimet,
 Hiábavalókat hogy ne nézzenek;
 A te utadban éltess engemet!
 Szolgáddal láttasd szent ígéretednek
 Bétöltését, aki téged tisztel
 És mindenkor fél alázatos szívvel.
/20
#F71916BB
 Végy el rólam minden gyalázatot,
 Melytől én igen félek és rettegek!
 Ítéleteid jók, és azokat
 Én megtanulni igen örvendezek.
 Gyönyörködik törvényedben szívem,
 És igazságban éltess, Uram, engem!
;Foglaljam szívembe az Úr törvényét
;
/21
#A1012B75
 Forduljon hozzám, Uram, kegyelmed,
 És segedelmed adjad, hogy láthassam!
 Ígéretedből vélem ezt tegyed,
 Hogy szájukat azoknak bedughassam,
 Akik engemet gyaláznak szörnyen,
 Mert bízom a te szent ígéretedben!
/22
#816523B4
 Szent igaz igéd ne vedd el tőlem,
 Hogy az mindenkor legyen az én számban!
 A te beszéded én reménységem,
 Te törvényedet szívemben foglalván.
 És azt megtartom én minden módon,
 A szerint élvén most és mindenkoron.
/23
#B7F9F3E5
 Szüntelen járok én nagy örömmel,
 Mert parancsolataidat követem;
 Szívem mindenkor azokra szemlél.
 A királyok előtt bízvást beszélem
 A te bizonyságidat, melyektől
 Meg nem rettenek, nem félek szégyentől.
/24
#FA17D244
 Gyönyörködöm a te törvényedben,
 Szent parancsolataidat szeretem
 Mindenek felett egész éltemben.
 Én kezeimet készen felemelem
 A te kedves parancsolatidra,
 És én azokról beszélek mindenha.
;Gondolkodjunk Isten dolgairól
;
/25
#6C39C81B
 Gondold meg azt és jusson eszedbe,
 Amit szolgádnak egyszer megígértél,
 Jó reménységet adván szívembe!
 Minden ínségben vagyok bátor szívvel,
 Mert szent beszéded bizony engemet
 Megújít és megtartja életemet.
/26
#18C868BC
 A kevély népek csúfolnak engem
 És nevetnek, de nem gondolok vélek,
 Hogy törvényedet azért megvetném,
 De a te ítéletidre tekintek,
 Melyeknek örök voltát jól tudom,
 És magamat azokkal vigasztalom.
/27
#24032571
 És miként az istentelen népek,
 Kik elszakadtak a te törvényedtől,
 Gondolatimnak nagy bút szereznek:
 Emlékezésem a te szerzésedről
 Ének volt nékem nagy örömömben,
 Bujdosásimnak minden ő helyökben.
/28
#86657432
 Sem éjjel, sem nappal meg nem szűnöm
 A te nevedről gyakran emlékezni;
 Szent parancsolatidat keresem,
 Főképpen erre szoktam vágyakozni.
 Minden előtt magamban elszántam,
 Hogy a te törvényedet én megtartsam.
;Hiszek Isten ígéreteiben
;
/29
#E87FDD6F
 Hiszem, te vagy az én örökségem,
 Teljes erőmmel azért azon lészek,
 Hogy a te igédet megőrizzem.
 A te színed előtt szívből könyörgök:
 Kegyelmezz meg, Úr Isten, énnekem,
 Mert ígéreted megvigasztal engem!
/30
#AF505C00
 Jól meggondolom az én utamat,
 Hogy a jó útról el ne tévelyedjem;
 Arra vezérlem minden gondomat,
 Bizonyságidra lábamat térítem.
 Igen sietek, nem kések semmit,
 Hogy megtarthassam szent parancsolatid.
/31
#F7852F8D
 Megfosztottak az istentelenek,
 Elpusztítának, de mindazonáltal
 Törvényid tőlem nem felejtetnek.
 Még éjfélkor fölkelek vigassággal
 És tégedet dicsérlek és áldlak
 Ítéletiért szent igazságodnak.
/32
#EACD6644
 Az oly népekhez adom magamat,
 Kik téged félnek és tereád néznek
 És megtartják parancsolatidat.
 Bőségével te kegyelmességednek
 Teljes e világ, azért, én Uram,
 Szent törvényedre tőled taníttassam!
;Isten igéjére építek
;
/33
#546E2127
 Íme, szegény szolgáddal sok jót től,
 Szent ígéreted szerint megsegítél,
 Min most is örvendek tiszta szívből.
 Taníts és áldj engem jó értelemmel!
 Engedd meg nékem ismeretedet,
 Mert igaznak ismerem törvényedet.
/34
#DD81F63C
 Minekelőtte megbüntettetném,
 Az igaz utat elvétettem vala,
 Most életem igédhez rendelem,
 Szorgalmatossággal tekintek arra.
 Uram, jókedvű vagy és irgalmas,
 Azért szerzésedre engemet oktass.
/35
#C034A986
 A kevélyek rólam hamisságot
 Költnek, de én a te szent törvényedet,
 Megőrizem parancsolatidat.
 Kövér ő szívük és megkeményedett,
 Én pedig a te szent törvényedben
 Gyönyörködöm mind egész életemben.
/36
#59D21922
 Jómra lett nékem, hogy megalázál,
 Hogy megtanuljam a te törvényedet,
 Min igyekezem nagy óhajtással.
 Aranyt, ezüstöt és egyéb effélét,
 Mit az emberek nagyrabecsülnek,
 Törvényedhez képest tartok semminek.
;Könyörgés megtérésért
;
/37
#0EB8DB7C
 Kezeiddel formáltál engemet,
 Taníts meg azért parancsolatidra,
 Hogy törvényidnek tudjam értelmét.
 És ezen indulnak nagy vigasságra
 Az istenfélők, látván e dolgot,
 Hogy igédbe vetem bizodalmamat.
/38
#32F61D08
 Igaz vagy, Uram, ítéletidben,
 Tudom, hogy senkit nem büntetsz méltatlan,
 Engem is méltán büntetsz ekképpen.
 Kérlek, cselekedjed ezt irgalmadban,
 Hogy én megvigasztaltassam megint
 Szolgádnak mondott ígéreted szerint!
/39
#3BC0E29E
 Nagy irgalmadat mutasd meg nekem,
 Hogy éljek, mert csak a te törvényedbe'
 Vagyon minden én gyönyörűségem.
 A kevélyek essenek szégyenségbe,
 Kik engem hamis okkal terhelnek,
 De én a te törvényedről beszélek.
/40
#36E690E0
 Térjenek hozzám mostan mindenek,
 Kik téged félnek, törvényed tisztelik
 És a te bizonyságidnak hisznek!
 Tiszta én szívem, el sem tévelyedik,
 De megtartja a parancsolatot,
 Hogy ne valljak szégyent, se gyalázatot.
;Lelki vigasztalásért való könyörgés
;
/41
#3F23628E
 Lelkem elfogy nagy kívánságában,
 Midőn várom a te segedelmedet;
 Bízván igédnek fogadásában.
 Ugyan elfárasztom az én szememet
 Nagy várakozás miatt, így szólva:
 Mikor vigasztalsz meg engem valóba'?
/42
#863AF69F
 Noha én csaknem hasonló vagyok
 Füstön elaszott, megszáradt tömlőhöz,
 Szerzésidre mégis gondot tartok.
 Míglen kell várnom, mikor látsz ügyemhöz?
 Míg halasztod el ítéletedet,
 Ha bünteted meg ellenségeimet?
/43
#32837CA1
 A kevélyek, kik szent törvényedet
 Megvetik, titkon nekem vermet ásnak,
 De ha tekintjük szent szerzésidet,
 Parancsolatid mind jók és igazak.
 Nagy méltatlanul kergetnek engem,
 Tarts meg azért és légy én segedelmem!
/44
#D0551288
 Csaknem elvesztének ők engemet,
 És majd ugyan eltörlének e földről,
 Mégsem hagyom el szent törvényedet,
 Tartsd meg éltemet kegyelmességedből,
 Hogy megtartsam Felséged kötésit,
 És megőrizzem minden bizonyságit.
;Megáll az Istennek igéje
;
/45
#01334AFF
 Mindörökké, Uram, a te igéd
 Megáll és megtart a magas mennyekben,
 Azonképpen isteni hűséged
 Megmarad örökre minden időben,
 Mint az álló föld, mit te fundáltál,
 Mely ő helyében mindenkoron megáll.
/46
#0D553F7B
 Mind ma és mindörökké megállnak,
 Amiket te rend szerint teremtettél,
 És teneked mindenek szolgálnak.
 Hogyha magamat a te törvényeddel
 Nem vigasztaltam volna, már régen
 Elvesztem volna én nagy ínségemben.
/47
#463DF108
 Nem felejtem el szent törvényedet,
 És gondot tartok parancsolatidra,
 Mert te azokban éltetsz engemet.
 Tekints, Uram, kegyelmesen szolgádra!
 Légy segítségem, mert tied vagyok,
 És törvényidnek őrzésire vágyok.
/48
#93C3E5E6
 Istentelenek énreám titkon
 Leselkednek és törnek életemre.
 Én elmélkedem bizonyságidon,
 És minden dolgot ha megnézek végre,
 Látom, hogy mindenek elmúlandók,
 De a te törvényid megmaradandók.
;Nincs Istenen kívül bölcsesség
;
/49
#1C8A28BF
 Nagy szerelmem vagyon törvényedhez,
 Melyről naponként örömest beszélek,
 Mert ez nékem víg örömöt szerez;
 Te parancsolatid bölcsebbé tésznek
 Engemet minden ellenségimnél,
 Mert soha tőlem ők nem távoznak el.
/50
#61D25C33
 Tudósabb vagyok tanítóimnál,
 Akiket már nagy bölcseknek tart minden,
 Mert bizonyságod elmémben megáll.
 Még a véneknél is bölcsebb vagyok én,
 Mert te törvényed szem előtt tartom,
 És elmémet attól el nem fordítom.
/51
#7D1F051F
 Minden hamis utat elkerülök,
 Lábam ne járjon a gonosz ösvényen;
 Igéd megtartásának örülök,
 Ítéletidet tekintem szüntelen.
 Azoktól soha el nem távozom,
 Kik által én tetőled taníttatom.
/52
#12E28C97
 A te beszéded ékes és drága,
 Még a méznél is édesebb én számban,
 Kimondhatatlan gyönyörű volta.
 Igédben van bölcsességem fundálván;
 Bölcsességemet abban keresem,
 A hamisságnak ösvényét gyűlölöm.
;Óránként fáklyám az Ige
;
/53
#34ED79EC
 Óránként fáklyám nékem szent igéd,
 Mely világot tart nékem útaimban,
 Hogy egyenesen járjam ösvényed,
 Amelyen én járhatok bátorságban.
 Megesküszöm és néked megállom,
 Hogy igazságodnak jussát megtartom.
/54
#CD16E555
 Felette igen megnyomorodtam,
 Enyhíts meg és végy fel engem ismétlen,
 Amint nekem megígérted, Uram!
 Az áldozat, mit szájam neked tészen,
 Kérlek, hogy légyen kedves tenálad,
 Ítéletedet én tudtomra adjad!
/55
#C26BE7CD
 Oly nagy veszélyben forog életem,
 Hogy tenyeremben hordozom lelkemet,
 Szent törvényedet mégsem felejtem.
 A hitlenek, kik gyalázzák nevedet,
 Tőrt vetnek nékem mindenütt széjjel,
 Szent szerzésidtől mégsem távozom el.
/56
#08612841
 Bizonyságidat örökül bírom,
 És azokat tartom drága kincsemnek,
 Azokban lesz minden vigasságom!
 Szívemet hajtom a te törvényednek
 Megtartására minden időben,
 És azokat megőrzöm mindvégiglen.
;Ösztönösen gyűlölöm a gonoszt
;
/57
#346FC42F
 Ösztönösen gyűlölöm azokat,
 Akik mindenkor kétfelé gondolnak,
 De szeretem én a te utadat,
 És csak tégedet tartlak oltalmamnak.
 Ígéretedben van reménységem,
 A te szent igéd paizsom énnékem.
/58
#1D5E22A0
 Gonosztevők, menjetek el tőlem,
 Mert én azt mind meg akarom tartani,
 Amit az Isten parancsol nekem!
 Igéd szerint siess engem táplálni,
 Hogy élhessek, légy velem, Úr Isten,
 Ne szégyenüljek meg reménységemben!
/59
#1B9B73F4
 Légy gyámolom, adj jó békességet,
 Úgy lészen kedves énnékem törvényed,
 Abban keresem én örömömet!
 Az olyakat te mind a földhöz vered,
 Kik elhajolnak igazságodtul
 És járnak minden dolgukban álnokul.
/60
#26AB20C9
 Gonoszakat te elvetsz a földről,
 Mint a salakot vagy ércnek szemétjét.
 Szent bizonyságid szeretem szívből!
 Félelmében szívem előtted reszket!
 Testem elepedt nagy rettegésbe',
 A te kemény ítéletedre nézve.
;Pörpatvarnál jobb az Úr törvénye
;
/61
#6B1AB32E
 Pörpatvar nincsen nékem kedvemre,
 És igazságát megadom mindennek;
 Ne adj azért azoknak kezekre,
 Akik engemet szüntelen kergetnek!
 Szolgádat minden jóra vezéreld,
 És a kevélyek ellen védelmezzed!
/62
#84C2B28A
 Az én szemeim elfogyatkoztak,
 Úgy nézik, várják te segedelmedet,
 Óhajtják igazságát szavadnak,
 Ne késsél, Uram, segíts meg engemet!
 Szegény szolgáddal tégy kegyelmesen,
 Taníts igédre, oktass törvényedben!
/63
#9DF7081A
 Szolgád vagyok, adj értelmet nekem,
 Hogy érthessem a te bizonyságidat,
 És jó értelmében gyönyörködjem!
 Ideje, Uram, hogy láttasd dolgodat,
 Mert helye nincs már az igazságnak,
 A te törvényid mind eltiportatnak.
/64
#955B4DA3
 Azért a te szent tanításidat
 Tiszta aranynál is inkább szeretem,
 Mindennél feljebb tartom azokat,
 Életemet én a szerint rendelem,
 Mert igaznak tartom mindenképpen,
 A hamis ösvényt gyűlölöm erősen.
;Rakva csudákkal az Isten igazsága
;
/65
#110ADFF8
 Rakvák bizonyságid nagy csudákkal,
 Hogy azért megtarthassam én szívemben,
 Azon igyekszem nagy buzgósággal.
 A te igéd, ha kik veszik eszükbe,
 Setét szíveket megvilágosít,
 Együgyűeket bölcsességre tanít.
/66
#B6BF4380
 Felfohászkodom gyakran én számmal,
 Mert én azt nagy szívem szerint kívánom,
 Hogy törvényedet értsem bizonnyal.
 Tekints reám és könyörülj szolgádon!
 Irgalmazz nékem, lám, nagy jókedved
 Azokhoz, akik szeretik szent neved!
/67
#FC52A056
 Szent igédben vezéreld utamat,
 És őrizz meg engem a hamisságtól,
 Hogy az rajtam ne vegyen hatalmat;
 Ments meg a népek nyomorgatásától,
 Hogy törvényedet örömmel nézzem
 És parancsolatidat megőrizzem.
/68
#30D12FB8
 Világosítsd meg orcád szolgádon,
 És taníts meg, hogy én jól meggondoljam,
 Beszéded engem mire tanítson!
 Könnyhullatásom szememből azértan
 Mint patak ömlik, mert hogy a népek
 Becsületet nem tesznek törvényednek.
;Summa szerint igaz az Isten
;
/69
#8A1BDB34
 Summa szerint, Uram, te igaz vagy
 Mindennémű te cselekedetedbe',
 Ítéletednek igazsága nagy.
 Te igazságod vehetik eszükbe,
 Bizonyságidat akik megnézik,
 És parancsolatidat megőrizik.
/70
#5E4116C5
 Megöl a bú nagy indulatomban,
 Midőn tekintem a te szent igédet,
 Hogy az ellenség csúfolja bátran,
 És elfelejti minden beszédidet.
 A te szent igéd igen szép tiszta,
 Szolgád azért szereti és megtartja.
/71
#910A4B4F
 Én kicsiny és megvettetett vagyok,
 De mégsem felejtem el törvényedet,
 Sőt mindenütt arra gondot tartok.
 Szent igazságidnak nem látják végét,
 Mert mindörökké ők megmaradnak;
 Törvényed törvénye az igazságnak.
/72
#F5C2494C
 Én kergettetem, vagyok ínségben,
 De nem gondolván semmi nyavalyámmal,
 Nagy örömöm van te törvényedben.
 Te igazságod mindörökké megáll,
 Melyet jelents meg nekem kegyesen,
 És bátorságos leszek életemben.
;Tőle várok szabadítást
;
/73
#B12A52F4
 Teljes szívből hozzád esedezem,
 Uram, hallgass meg engem kegyelmesen,
 Hogy rendelésidet megőrizzem!
 Kérlek, szabadíts meg engem, Úr Isten,
 És legottan igyekezem azon,
 Hogy bizonyságid megtartsam jó módon.
/74
#B0F1B5DD
 Gyakorta reggel virradat előtt
 Könyörgésemben tehozzád kiáltok,
 Igédbe vetvén reménységemet.
 Előbb, hogynem elmennek a virrasztók,
 Az én szemeim vigyáznak, néznek,
 És a te szent igédről elmélkednek.
/75
#E0FF1392
 Kegyességedért halld meg beszédem,
 Tartsd meg életem a te jóvoltodból,
 Hadd vidámuljon meg az én szívem,
 Mert sok hitetlen nép énreám tódul,
 És engem szertelen sanyargatnak,
 De a te törvényedtől távol vannak.
/76
#65B99292
 De te énhozzám, Uram, közel vagy,
 És te igédből tudom réges-régen:
 Szent törvényednek igazsága nagy,
 Te bizonyságid fundáltattak szépen.
 Hogy örökké megmaradnak, tudom,
 És jól értem, azért nyilván kimondom.
;Védelmezz meg a te igazságodért!
;
/77
#560E6854
 Vedd eszedbe én nagy ínségemet,
 Én nyavalyámból, Uram, szabadíts meg,
 Mert nem felejtem el törvényedet!
 Fogadd fel ügyemet és védelmezz meg,
 Megtekintvén szent ígéretedet,
 Éltess a te szent igéddel engemet!
/78
#E4A1A874
 A gonoszoktól, minden meghiggye,
 A segítség s üdvösség távol vagyon,
 Mert ők nem néznek szent szerzésidre.
 Szent irgalmasságod nagy mindenkoron!
 Uram, tarts meg engemet, hogy éljek
 Nagy igazságából ítéletednek.
/79
#6382281D
 Bizonyságidat én nem hagyom el,
 Noha énreám nagy sok népek törnek,
 Kik gyűlölnek és kergetnek széjjel;
 Ó, mily sérelmes ez az én szívemnek,
 Hogy ellenségi az igazságnak
 A te igéddel semmit nem gondolnak!
/80
#E7D20E4E
 Törvényidet én igen szeretem,
 És soha el nem távozom azoktól;
 Kegyességedből tartsd meg életem!
 Te igaz beszéded mindent felülmúl;
 Szent ítéleti igazságodnak
 Mostan és mindörökké megmaradnak.
;Üldöztetésben is Rólad gondolkodom
;
/81
#D5A06C60
 Űznek, kergetnek a fejedelmek,
 Noha senkinek semmit nem vétettem;
 Szent igédtől szívemben rettegek,
 Ígéretedben örvendez én lelkem,
 Mint aki talál nagy gazdag prédát,
 Avagy mint aki nyer sok drága marhát.
/82
#23A89863
 A hazugságot igen gyűlölöm,
 Semmit e földön inkább nem utálok,
 De a te törvényedet szeretem,
 És igazságán oly igen vigadok,
 Hogy meggondolván ítéletedet,
 Naponként hétszer dicsérlek tégedet.
/83
#A9406FFE
 Nagy békességük vagyon azoknak,
 Kik szeretik a te szent törvényedet,
 Semmi veszélyben el nem botolnak.
 Várom, Uram, a te üdvösségedet!
 Abban forgatom minden gondomat,
 Hogy cselekedjem parancsolatidat.
/84
#4AAEDA64
 Bizonyságidra gondot tart lelkem,
 Mert én oly igen szeretem azokat,
 És én csak azokban gyönyörködöm,
 Rendelésidben gyakorlom magamat.
 Előtted vagyon minden életem,
 Nincsen elrejtve tőled én ösvényem.
;Zendüljön fel a szabadító Isten dicsérete!
;
/85
#447786FC
 Zengő kiáltásom jusson hozzád,
 És igazságodat adjad értenem,
 Mint szent igédben te felfogadtad!
 Jusson elődbe én esedezésem,
 Szabadíts meg minden ínségemből,
 A te régen tett szent ígéretedből!
/86
#895015CE
 Ha én megtanulom szerzésedet,
 Az én ajakimmal dicsérlek téged,
 Hirdeti nyelvem te szent igédet,
 Mert minden törvényed és ígéreted
 Merő hűség és tiszta igazság,
 Nem találtatik abban semmi hívság.
/87
#C4153BD6
 Ments meg, Uram, engem kezeiddel,
 Légy erősségem, segedelmem nékem,
 Mert törvényedet szeretem szívvel!
 Üdvözítésed várom, én Istenem,
 Melyben vetettem reménységemet,
 Mert igen kedvelem szent törvényedet.
/88
#B5027B47
 Éltét csak azért kívánja lelkem,
 Hogy ő tégedet, Uram, dicsérhessen,
 Ítéleted légyen segedelmem!
 Mint elveszett juh bujdosom e földön.
 Uram, keress meg engem, szolgádat,
 Nem felejtem el parancsolatidat!

;A rágalmazók ellen
;Bourgeois L., Genf, 1551
>120
/1
#1B3C9DE8
 Én az Úr Istenhez kiálték,
 Mikoron nagy ínségben valék,
 És be nem dugá az ő fülét.
 Hallgasd meg, Uram, kérésemet!
 A népeknek hazug szájától,
 Hamis nyelvek gyalázatjától,
 Életemet e veszélytől
 Tartsd meg kegyelmességedből!

;Isten az ő népének hű őrizője
;Bourgeois L., Genf, 1551
>121
/1
#40549306
 Szemem a hegyekre vetem,
 Onnan felül nékem
 Minden segedelmem.
 Isten az én reménységem,
 Ki az eget formálta
 És e földet alkotta.
/2
#C55AE8AF
 Lábad botlani nem hagyja,
 És aki rád vigyáz,
 Nem szunnyadozik az
 Izráelnek vigyázója,
 Mert az nem aluszik el,
 De rájuk gondot visel.
/3
#B566990A
 Az Úr téged megőrizzen,
 Kezet rád terjesztvén,
 Árnyékkal befedjen!
 Hogy a nap ő hévségében,
 Néked ne ártson éjjel
 A hold az ő fényével!
/4
#139A54DF
 Az Úr őrizze örökké
 Lelkedet esettől,
 Mentse meg veszélytől!
 Az Úr híven megőrizze
 A te kimenésedet
 És te bejövésedet!

;Áldáskívánás az anyaszentegyházra
;Bourgeois L., Genf, 1551
>122
/1
#C0292ABB
 Örülök az én szívembe'
 E kívánatos hírt hallván,
 Hogy mi bémegyünk ezután
 Az Isten lakó helyébe,
 Jeruzsálem, kapuidban
 Mi lábainkkal megállván.
 A Jeruzsálem jól megépült
 Sok ékes épületekkel,
 Holott szép polgári renddel
 Mindenféle nemzet egybegyült.
/3
#08E3E4E8
 Légyen te kőfalaidban
 Csendesség és jó békesség,
 A község közt egyenesség;
 Jó szerencse házaidban!
 Az én atyámfiaiért
 És ott lakó feleimért
 Adjon Isten jó békességet!
 Szentségiért én e helynek,
 Mely szerzetett az Istennek,
 Minden jót kívánok tenéked!

;Könyörülj rajtunk!
;Bourgeois L., Genf, 1551
>123
/1
#4421EDD4
 Tehozzád szemeimet, Úr Isten,
 Emelem az égben!
 Mint szolgának szemei nyitva vannak,
 És az Úrra vigyáznak,
 Mint szolgáló leány ő aszszonyára
 Tekint minden órába,
 Úgy néznek szemeink az Istenre
 És ő kegyelmére.
/2
#BD7FA706
 Kegyelmezz, Uram, kegyelmezz nékünk,
 Mert nincs becsületünk!
 Minden oly szertelenül gyaláz minket,
 Melynél feljebb nem lehet.
 Kevély népek minket szörnyen nevetnek,
 Rajtunk csúfságot űznek;
 Gyalázó szókkal úgy illettetünk,
 Mikkel eltölt lelkünk.

;Isten velünk van a szükségben
;Bourgeois L., Genf, 1551
>124
/1
#56BF8083
 Az Izráel ezt nyilván mondhatja:
 Ha az Isten nem lett volna velünk,
 Ha ő nem lesz vala segítségünk,
 Midőn az embereknek soksága
 Nagy kegyetlen támada ellenünk.
/2
#E1F31A48
 Elnyelnek vala ők elevenen,
 Úgy reánk gerjedt vala haragjok.
 Minket teljességgel a nagy habok
 Vízzel elborítnak vala szörnyen,
 Ránk omlanak vala nagy patakok.
/3
#CB22DBF0
 A nagy árvíz reánk rohan vala,
 És elnyeli vala életünket.
 Dicséret Istennek, hogy ő minket
 Az ő foguktól megszabadíta,
 Hogy ők meg ne ennének bennünket.
/4
#890B5DC0
 Miképpen elszalad a madárka
 A madarász tőriből, akképpen
 Megszabadulánk, a tőrt elszegvén.
 Mert életünknek az Úr oltalma,
 Ki mennyet, földet teremtett bölcsen.

;A reménység meg nem szégyenít
;Bourgeois L., Genf, 1551
>125
/1
#3F509847
 Akik bíznak az Úr Istenben
 Nagy hiedelemmel,
 Azok nem vesznek el
 Semminémű veszedelemben.
 Mint a Sion hegye, megállnak,
 Nem ingadoznak.
/2
#B4BF886E
 És mint a nagy Jeruzsálemet
 Hegyek körülvették
 És nagy kerítések,
 Akképpen Isten az ő népét
 És ő híveit körülveszi
 És megőrizi.
/3
#E2CEE298
 Mert az övéit ő nem hagyja
 A hamis kezében
 És semmi ínségben,
 Hogy ki-ki önmagát megóvja,
 Hogy a hitlen népekkel egybe
 Ne essék bűnbe.
/4
#C5B2F9BE
 Jelen van a jószívűekkel,
 De a hitleneket
 És ő ösvényüket
 Elhagyja a gonosztévőkkel.
 Ő híveinek békességet,
 Ád csendességet.

;Sion foglyainak szabadulása
;Bourgeois L., Genf, 1551
>126
/1
#296EC0F2
 Mikor a Siont az Isten
 Fogságból kihozá híven,
 Úgy mentünk, mint egy álomban,
 Víg nevetés volt szájunkban.
 Dicsekedtünk a mi nyelvünkkel,
 És énekeltünk nagy örömmel;
 A pogány közt minden mondta:
 Ez nyilván az Isten dolga.
/2
#C707FA33
 Az Úr mutatá hatalmát,
 Mivelünk tőn nagy csudákat,
 Azért őt dicsérjük itten,
 Örvendezzünk szíveinkben!
 Hozd ki, Úr Isten, a többit is,
 Vesd végét ő fogságuknak is,
 Mint az erős zúgó széllel
 Mind e föld megszárad széjjel.
/3
#3168AB7E
 Akik nagy könnyhullatással
 Magot vetnek nagy bánattal,
 Aratáskor azok széjjel
 Aratnak majd víg örömmel.
 Sírva mennek ki a vetésre,
 A magot hintik keseregve,
 De a jó kévéket aztán
 Béhordják nagy vigasságban.

;Emberé a munka, Istené az áldás
;Bourgeois L., Genf, 1551
>127
/1
#44C32D89
 Hogyha ember házat épít
 Isten segedelme nélkül,
 Ott a munka hiába kél.
 Ha Isten nem őrzi népét
 És a várost meg nem óvja,
 Az őrzőknek semmi haszna.
/2
#F5A18CF3
 Hiába keltek fel reggel,
 Nagy későn feküdni menvén,
 Alig esztek a kenyérben,
 Mit kerestek verejtékkel,
 Holott kit az Isten szeret,
 Annak könnyen ád eleget.
/3
#B8BDE83B
 Akinek gyermeki vannak,
 Szép ajándékkal láttatik,
 Mely az Istentől adatik.
 Kedves áldása az Úrnak:
 Láthatja ő magzatait,
 Önnön méhének gyümölcsit.
/4
#61E85483
 Ó, mely drága adományok,
 Hogy az ily apró gyermekek
 Ékesen felnevelkednek,
 És ők szintén olyaténok,
 Mint nyilak a jó kézívben,
 Az erős vitéz kezében.
/5
#425B828F
 Nyilván boldognak mondatik,
 Ki bővölködik azokkal,
 Rakva ő tegze nyilakkal.
 Szégyen őrajtuk nem esik,
 Ha ellenség a kapuban
 Beidézi szóválásban.

;Az istenfélő házán áldás van
;Bourgeois L., Strasbourg, 1545
>128
/1
#581A676A
 Boldog az ember nyilván,
 Ki az Istent féli,
 Ő útaiban járván,
 Ösvényit kedveli.
 Mert magadat táplálod
 Kezed munkájával,
 Isten megáldja dolgod,
 Lát jó állapottal.
/2
#7FA51BB9
 Házadban feleséged,
 Mint a szőlővessző,
 Szép gyümölcsöt hoz néked,
 Ha eljő az idő.
 Meglátod gyermekidet
 Te asztalod körül,
 Renddel, mint olajvesszőt,
 Kikben szíved örül.
/3
#1A34A3C7
 Ez igen szép ajándék,
 Mit az Isten enged
 Az őbenne hívőknek,
 Akiket ő szeret.
 És végre te meglátod
 Fiadnak fiait,
 A Sionnak csudálod
 Nagy szép békességit.

;A nyomorgatók megszégyenülnek
;Bourgeois L., Genf, 1551
>129
/1
#C18B3384
 Én ifjúságomtól fogva engem
 Sanyargattak, mondhatja az Izráel,
 Régtől fogván sok bút tettek nékem,
 Mégsem fogyathattak el teljességgel.
/2
#7B7D055D
 Széjjel-keresztül az én hátamat
 Megszántották és megszaggatták szörnyen,
 Vontanak rajtam nagy barázdákat,
 Úgy hogy testemben semmi ép tag nincsen.
/3
#5F43D7D4
 De az igazságnak jó Istene
 Elszaggatá az ellenség kötelét;
 Kik irigykednek Sion hegyére,
 Nagy szégyenség térítse hátra őket!
/4
#9E9998BA
 Mint a hitvány fű, olyak legyenek,
 Amely szokott nőni az eszterhákon,
 Elszáradnak, mielőtt kinőnek,
 Amelyben nincsen sehol semmi haszon.
/5
#800337C0
 Min az arató nem talál annyit,
 Hogy egy marokkal valót arathatna,
 Nemhogy köthetne valami kévét,
 Amit az ember ölébe szorítna.
/6
#44DA5DBE
 És aki itt elmégyen, ne mondja:
 A kegyes Isten áldjon meg titeket,
 Ti aratástok légyen szapora,
 Az Úr nevében áldunk benneteket!

;A mélységből kiáltok, Uram! (Hatodik bűnbánati zsoltár)
;(1539) Bourgeois L., Genf, 1542
>130
/1
#4A7AD0B0
 Tehozzád teljes szívből
 Kiáltok szüntelen:
 E siralmas mélységből
 Hallgass meg, Úr Isten!
 Nyisd meg te füleidet,
 Midőn téged hívlak,
 Tekintsd meg én ügyemet,
 Mert régen óhajtlak.
/2
#5304F763
 Ha, Uram, bűnünk szerint
 Minket büntetnél meg:
 Uram, e világ szerint
 Ki állhatna úgy meg?
 De a te irgalmad nagy
 A téged félőkön,
 És te engedelmes vagy,
 Hogy dicsérjen minden.
/3
#67EBF815
 Énnékem reménységem
 Vagyon csak Istenben,
 És bízik az én szívem
 Ő szent igéjében.
 Én lelkem erős hittel
 Az Urat óhajtja,
 Mint a virrasztó éjjel
 A virradtát várja.
/4
#E4085417
 Izráel, az Istenben
 Vesd reménységedet,
 Mert szent irgalma bőven
 Nagy messze kiterjedt.
 Megsegít ő mindenben,
 Hívein könyörül,
 Az Izráelt kegyesen
 Kimenti bűnébül.

;Hívő lelki alázatosság
;Bourgeois L., Genf, 1551
>131
/1
#5FF06D18
 Uram, nem űz nagyra szívem,
 Szemem nem hányom magasan;
 Nem leledzem oly dolgokban,
 Mik felülhaladnak engem.
/2
#6CC51EE0
 De kivetettem szívemből
 Mindennemű kevélységet,
 Mint midőn a kisgyermeket
 Elfogják anyja tejétül.
/3
#4340918C
 Hogyha olyanná nem lettem,
 Mint az együgyű kis gyermek,
 Kit a csecs mellől elvesznek,
 Meg se kegyelmezz énnékem.
/4
#54057A70
 Izráel, az Úr Istenben
 Vessed minden reménységed!
 Megtartódnak őtet higgyed
 Mostan és minden időben!

;Könyörgés a Sionért
;Bourgeois L., Genf, 1551
>132
/1
#F67E65A9
 Emlékezzél meg, Úr Isten,
 Dávidról és ínségéről,
 Ki néked megesküdt szívből,
 És fogadást tett hűségben
 Az Úrnak, kit Jákób tisztöl.
/2
#9E0195FE
 Ezt, úgymond, én fölfogadom,
 Hogy nem megyek be házamba,
 Se le nem fekszem ágyamba,
 Szemeimet be sem hunyom,
 Szempillámat le sem zárva;
/3
#7C6416ED
 Nyugalmam addig nem lészen,
 Míglen helyet nem keresek
 A Jákób nagy Istenének,
 Holott sátort szerzek szépen,
 Hol dicsősége lakozzék.
/7
#3A555B64
 A Dávidnak az Úr Isten
 Megesküdt mint szolgájának,
 És ő bízvást hihet annak:
 Ím, a te nemzetedből én
 Székedre királyt állatok!
/8
#83EAD085
 És aztán a te gyermekid
 Megtartják én kötésemet,
 És bizonyságtételimet,
 És örökké a te székid
 Megülik, mint örökjüket.
/9
#0D344909
 Mert e Siont az Úr Isten
 Választá lakóhelyének,
 Mondván: e hely kell kedvemnek,
 Itt lakom minden időben;
 Ez helye jótetszésemnek.
/10
#D8B9DD92
 Én őket megelégítem,
 Kenyért adok a szegénynek,
 Ruhájába üdvösségnek
 Papjait felöltöztetem;
 Örömük lesz a híveknek.
/11
#81782456
 Fölnevelem ez egy magvát
 Az én szolgámnak, Dávidnak,
 Kit én fölkentem magamnak.
 Fölkészíttettem lámpását,
 Hogy lássa világát annak.
/12
#679EBEC3
 De viszont ő ellenségit
 Öltöztetem nagy szégyenben,
 Midőn a koronát fején
 Meglátják, és dicsőségit,
 Hogy megvirágzik ékesen.

;A testvéri egyetértés áldása
;Bourgeois L., Genf, 1551
>133
/1
#51C3AD8D
 Ímé, mily jó és mily nagy gyönyörűség
 Az atyafiak közt az egyenesség,
 Ha békével együtt laknak.
 Mint a balzsamolaj, ők olyanok!
 Megáldja az Úr az ilyeneket,
 Nékik ád hoszszú életet.

;Éjtszakai dicséret a templomban
;Bourgeois L., Genf, 1551
>134
/1
#D5717C95
 Úrnak szolgái mindnyájan,
 Áldjátok az Urat vígan,
 Kik az ő házában éjjel
 Vigyázván, vagytok hűséggel.
/2
#10CA7B3A
 Felemelvén kezeteket,
 Dicsérjétek Istenteket,
 Szívből néki hálát adván,
 Őt áldjátok minduntalan!
/3
#1488DA79
 Megáldjon téged az Isten
 A Sionról kegyelmesen,
 Ki teremtette az eget,
 A földet és mindeneket!

;Az egy igaz Isten dicsérete
;Genf, 1562
>135
/1
#D8FA92EC
 Áldjátok az Úr nevét,
 Akik néki szolgáltok!
 Magasztaljátok őtet,
 Kik hív szolgái vagytok,
 Kik állotok házában
 És jártok tornáciban.
/2
#B37F3507
 Dicsérjétek, mert jó ő,
 Áldjátok ő szent nevét,
 Mert ő igen jókedvű!
 Jákóbot, mint ő népét,
 Izráelt elválasztá,
 Örökévé foglalá.
/3
#FE2AC5AF
 Mert tudom, hogy ez Isten
 Erősb más isteneknél,
 Hatalma nagy mindenen;
 Isteni beszédével,
 Amit akar az égen,
 Megtészi földön, vízen.
/4
#E98D7745
 Megáll örökké neve
 És szent emlékezete.
 Ítéli hatalmában,
 Népét igazságában;
 Szolgáinak az Isten
 Megkegyelmez kegyesen.
/8
#9E904D8F
 Megáll örökké neve
 És szent emlékezete.
 Ítéli hatalmában,
 Népét igazságában;
 Szolgáinak az Isten
 Megkegyelmez kegyesen.
/9
#2CC43FC2
 A pogányok bálványi
 Ezüstből és aranyból
 Kézzel szoktak öntetni,
 Csak semmik, bár nézd meg jól.
 Szájukkal ők nem szólnak,
 Ő szemükkel nem látnak.
/10
#BAF0EF34
 Nem hallanak fülükkel,
 Az ő szájuk nem lehell.
 Hozzájuk mind hasonlók
 Az ily képet formálók,
 Kiknek ő reménységek
 Efféle bálványképek.
/12
#35C440B1
 Kik az Urat félitek,
 Szent nevét dicsérjétek!
 A Sionról áldjátok
 És őt magasztaljátok
 Az ő lakóhelyében,
 A szent Jeruzsálemben!

;Isten örök jósága és csodái
;Genf, 1562
>136
/1
#5D112A64
 Dicsérjétek az Urat,
 Mert ő jókedvet mutat,
 És az ő kegyessége
 Megmarad mindörökre.
/2
#AF1C0405
 Áldjátok Istenteket,
 Isteneknek Istenét,
 Mert az ő kegyessége
 Megmarad mindörökre.
/3
#2A0169C7
 Dicsérje őtet minden,
 Mert nagy csudákat tészen,
 És az ő kegyessége
 Megmarad mindörökre.
/4
#6C106D5D
 Minden őt magasztalja,
 Mert ő Uraknak Ura,
 És az ő kegyessége
 Megmarad mindörökre.
/5
#4BB1F6E5
 Ki a mennyet teremté
 És bölcsen ékesíté,
 Mert az ő kegyessége
 Megmarad mindörökre.
/6
#6A038772
 Aki a széles földet
 Szerzette a víz felett,
 És az ő kegyessége
 Megmarad mindörökre.
/7
#48E934C9
 Ki nagy szép lámpásokat
 Fenn az égen alkotott,
 És az ő kegyessége
 Megmarad mindörökre.
/8
#EB774DD7
 Nappal vezérlésére
 A napfényt teremtette,
 És az ő kegyessége
 Megmarad mindörökre.
/16
#B81CA38E
 Népét vezérlé híven
 Szörnyű nagy kietlenben,
 És az ő kegyessége
 Megmarad mindörökre.
/17
#B5069842
 A királyokat széjjel
 Megrontá kezeivel,
 És az ő kegyessége
 Megmarad mindörökre.
/18
#76FE0038
 Erős fejedelmeket ő
 Megölt és elvesztett,
 És az ő kegyessége
 Megmarad mindörökre.
/23
#268EDAB7
 Megemlékezék rólunk
 Látván nyomorúságunk,
 És az ő kegyessége
 Megmarad mindörökre.
/24
#4E36FF82
 Ellenségünk kezéből
 Megmente jókedvéből,
 És az ő kegyessége
 Megmarad mindörökre.
/25
#A0BC7A38
 E földön minden testet
 Táplál és bőven éltet,
 És az ő kegyessége
 Megmarad mindörökre.
/26
#4B45C23E
 Áldjátok az Úr Istent
 Minden népek naponként,
 Mert az ő kegyessége
 Megmarad mindörökre.

;Babilon vizei mellett
;(1539) Bourgeois L., Genf, 1542
>137
/1
#D6E7B6E3
 Hogy a babiloni vizeknél ültünk,
 Ott mi nagy siralomban keseregtünk,
 A szent Sionról megemlékezvén,
 Melynél gyönyörűségesb hely nincsen.
 A nagy búnak és bánatnak miatta
 Hegedűnket függesztettük fűzfákra.
/2
#46FAD8F9
 Akik minket fogva tartottak, kértek,
 Hogy valamit hegedülnénk nékiek,
 És mondanánk sioni éneket.
 Felelvén mondtuk: Miképpen lehet?
 Hogy dicsérhetnénk az Úr Istent vígan,
 Énekelvén ez idegen országban?
/3
#06A5977A
 Íme, neked én azt nyilván felelem,
 Hogy hegedülésemet elfelejtem
 Előbb, hogynem a Jeruzsálemet.
 Míg e fogságban tartnak engemet,
 Inkább ínyemhez ragadjon én nyelvem,
 Ha Jeruzsálemre nem óhajt szívem!
/4
#8D46E3E9
 Az Édom fiairól emlékezzél,
 És nékiek azt, Uram, ne engedd el,
 Amit ők akkoron kiáltottak,
 Midőn Jeruzsálemet rontották:
 "Fosszad, fosszad Jeruzsálemnek népét,
 Földig lerontsad minden épületjét!"
/5
#41A21718
 Te, Babilonnak leánya, meghiggyed,
 Hogy végre még por-hamuvá kell lenned!
 Boldog, aki tenéked e dolgot,
 Megfizeti e méltatlanságot,
 Ki öledből gyermekidet kirántja,
 És az erős kősziklához paskolja!

;Isten az én szabadítóm
;(1539) Bourgeois L., Lyon, 1542
>138
/1
#BD8B31E2
 Dicsér téged teljes szívem,
 Én Istenem, Hirdetem neved.
 Dicsérlek istenek felett
 Én tégedet, Mert azt érdemled.
 És a te szentegyházadban
 Imádkozván, Neved tisztelem,
 Áldásodra én kész vagyok,
 Hálát adok Neked, Istenem.
/2
#01D452F9
 Öregbül nagy dicsőséged,
 Mert megtészed, Amit megmondasz.
 Ha könyörgök ínségemben,
 Engem menten Megszabadítasz.
 Téged minden földön lakók,
 Nagy királyok, Uram, dicsérnek,
 Mert szent igéd tiszta voltát,
 Igazságát Eszükbe vették.
/3
#6072D67A
 Ez Urat, ki felségesen
 Csudát teszen, Felmagasztalják.
 Mondván: nagy ő dicsősége
 És ereje, Őt azért áldják.
 Mert noha ül nagy magasan,
 De lát nyilván Alatt valókat;
 Magas dolgokat is könnyen,
 Lát élesen, Mind fent, mind alatt.
/4
#7D28ADCC
 Mindennemű szükségemben,
 Ínségemben Megerősítesz;
 Láttukra ellenségimnek,
 Kik gyűlölnek, Engem megmentesz.
 Amit az Úr egyszer végez,
 Az jó véghez megyen mindenütt.
 Jókedved megáll: ne hagyd el,
 Sőt végezd el Kezed munkáit!

;Isten mindentudó és mindenütt jelenvaló
;Bourgeois L., Genf, 1551
>139
/1
#6C8D191B
 Uram, te megvizsgálsz engem,
 Megismersz mindent énbennem,
 Vagy állok, ülök, vagy megyek,
 Mind tudod, amit művelek;
 Valamit gondolok szívemben,
 Te azt mind jól érted meszszünnen.
/2
#B80F381F
 Vagy járok-kelek, vagy fekszem,
 Te mindenütt vagy körülem.
 Jól látod minden utamat,
 Érted minden dolgaimat;
 Egy szó sem jő az én nyelvemre,
 Mit előbb ne tudtál volna te.
/3
#9A448CDA
 Körös-körül rajtam minden
 Tőled vagyon teremtetvén,
 Reám bocsátád kezedet,
 Felülmúlja értelmemet.
 Bölcseséged meg nem foghatom,
 Dolgaidon csak álmélkodom.
/4
#05BFA52F
 Lelked előtt hová mennék,
 Holott elrejtve lehetnék?
 Előtted hova szaladjak?
 Égbe menjek? Ott talállak!
 Ágyamat ha vetném pokolban,
 Ott látnálak téged legottan.
/5
#8BC2E539
 Ha a hajnal szárnyát venném,
 És az égre emelkedném,
 És elrepülnék nagy messze,
 A külső tenger szélire,
 Ott is meglelnél, Uram, engem,
 Kezedet el nem kerülhetem.
/6
#34EE9043
 Mondék: talán setétséggel
 Bétakarózhatom éjjel.
 Az sem lenne használatos,
 Mert még az éj is világos,
 Mely körülem annyira fénylik,
 Hogy nékem fényes napnak látszik.
/9
#93D87C1F
 Nálad ismeretes voltam
 Előbb, hogynem formáltattam,
 Hogy még nem voltam, ismertél,
 És napjaimra szemléltél,
 Mik könyvedben voltak felírva,
 Midőn még nem voltak formálva.
/10
#932CF3F9
 Mily drágák a te tanácsid,
 Ha megnézem gondolatid,
 Számtalan soknak találom,
 Ha kimondani akarom;
 Többnek lelem én a fövénynél,
 Mely sok a tengerparton széjjel.
/14
#70524618
 Vizsgálj meg és jól próbálj meg,
 Szívemet valóban nézd meg,
 És lásd meg, minémű vagyok!
 Ha gonosz ösvényen járok,
 És ha olyannak találsz engem,
 Vezess a jó útra, Istenem!

;Könyörgés ellenségtől való szabadításért
;Bourgeois L., Strasbourg, 1545
>140
/1
#DB0EF8CC
 Szabadíts meg engem, Úr Isten,
 A gonosz csalárd embertől;
 Őrizz erőszaktevők ellen,
 Ments meg a vakmerő néptől!
/2
#F1DBE7C7
 Kik csak hamisságot gondolnak
 Mindenkor az ő szívükben,
 És hogy hadakat indítsanak,
 Azon vannak mindenképpen.
/3
#6E656EED
 Élesben fenik ő nyelveket
 A kígyónak fúlánkjánál.
 Mint áspis kígyó, egyebeket
 Megsértnek mérges ajkukkal.
/4
#8CBD6BEB
 Ments meg a gonoszok kezébül,
 Akik erőszakot tesznek,
 És igyekeznek szüntelenül,
 Hogy engemet megejtsenek.
/5
#892C983A
 A kevélyek tőrt vetnek nékem,
 És mindenütt hálót hánynak,
 Kötéllel megvonsszák ösvényem,
 Hogy engem megszorítsanak.
/6
#18E824FF
 Én pedig mondék: Ó, Úr Isten,
 Te vagy én erős istenem!
 Beszédem végyed füleidben,
 Hallgasd meg esedezésem!
/7
#03EB97ED
 Uram, segítség vagy te nékem
 Mindennémű ínségemben,
 Azért védelmezd meg én fejem
 A hadakozó időben!
/8
#CF767EEB
 Ne engedd a hitetleneknek,
 Hogy elővigyék dolgukat!
 Hogy inkább ne kevélykedjenek,
 Rontsd meg gonosz szándékukat!
/12
#5D167468
 Tudom, hogy Isten a szegénynek
 Felfogja ügyét kegyesen,
 Megkegyelmez az erőtlennek,
 Ő igazságát jelentvén.
/13
#BBFD646B
 Az igazak szép énekkel
 Dicsérik te szent nevedet,
 És örökké jó reménységgel
 Megmaradnak színed előtt!

;Esti könyörgés megszentelődésért
;Genf, 1562
>141
/1
#F9DA35A6
 Tehozzád kiáltok, Úr Isten,
 Siess énhozzám, ne késsél,
 Mert téged óhajtlak szívvel,
 Azért hallgass meg kegyelmesen!
/2
#12D90E2F
 Könyörgésem elődbe menjen,
 Mint jó illat füstölgése,
 Kezeim felemelése
 Estvéli áldozatul légyen!
/3
#D72C14E9
 Őrizettel tartsd meg én szájam,
 Amely gondot tartson arra;
 Závárt szerezz ajakimra,
 Hogy tőlük ne légyen nyavalyám!
/4
#D6FF8393
 Ne bocsássad szívem gonoszra,
 Hogy a hamisan élőkkel
 Soha ne ereszkedjem el,
 Az ő nyájas kívánságokra.
/5
#003B2EB6
 A hív ember megfeddjen engem,
 És feddése kedves légyen,
 Mint balzsamolaj fejemen;
 Még verése sem árt énnékem!
/6
#55832D75
 Mert én elhiszem s mondom nyilván,
 Hogy még a hitetlenekért,
 Imádkozom mentségekért,
 Nyavalyájukon szánakozván.
/9
#0F55137B
 Szemeim reád néznek, Uram,
 Reménységem vagyon benned,
 Lelkemet hát el ne vessed,
 Mert te vagy minden bizodalmam!

;Segedelemkérés nagy szorongattatásban
;Bourgeois L., Genf, 1551
>142
/1
#5D1DA3B7
 Én az Úrhoz felkiálték,
 Kiáltván néki könyörgék,
 Jelentvén én panaszimat,
 Megbeszélém nyavalyámat.
/2
#E4AB55CC
 Ha szorongattatik lelkem,
 Te utat mutatsz énnekem,
 Módot találsz, amely által
 Megszabadulok jó móddal.
/3
#93B9BB1E
 Utamra tőrt mernek hányni,
 Amelyen szoktam én járni.
 Tekintek széjjel mellőlem,
 De senki nem ismer engem.
/4
#27ABF682
 Bezárattak utak, ajtók,
 Egy felé sem szaladhatok,
 És ez ilyen ínségemben
 Senki nincsen, ki segítsen.
/5
#FE71B7F9
 Tehozzád kiálték, Uram,
 Mondván: te vagy bizodalmam,
 És egyedül reménységem
 E földön csak te vagy nékem.
/6
#6D80CEE3
 Halld meg az én imádságom,
 Mert igen sanyargattatom!
 Ments meg ellenségim ellen,
 Mert erősbek, hogynem mint én!
/7
#B921D875
 Vedd ki fogságból lelkemet,
 Hogy dicsérjem szent nevedet,
 Hogy jól téssz velem, a hívek
 Ottan engem körülvesznek.

;Könyörgés szabadításért és vezetésért (Hetedik bűnbánati zsoltár)
;(1539) Bourgeois L., Genf, 1542
>143
/1
#9A2BC340
 Hallgasd meg, Uram, kérésemet,
 Vedd füledbe könyörgésemet
 A te ígéreted szerint!
 Hallgass meg és tarts meg engemet
 Te szent igazságod szerint!
/2
#2A628B52
 Szolgádat törvénybe ne idézd,
 Haragodat reám ne gerjeszd
 Az én gonosz bűneimre,
 Mert ez egész földet bár elnézd,
 Nem találsz igaz emberre.
/3
#C6F154E0
 Az ellenség kerget engemet,
 A földhöz verte életemet,
 Helyheztetett a setétbe.
 Setétségben engem elrejtett,
 Mint halottat sír mélyébe.
/4
#2CCA225F
 Az én lelkem elkeseredett,
 Sérelem miatt elepedett;
 Azért hogy elhagyál engem,
 Csaknem elszántam életemet,
 Úgy elkeseredett szívem.
/5
#3ACBD633
 Ez ilyen nagy ínségim között
 Említem a régi időket,
 Nézvén sok csuda dolgodat,
 Melyeket kezed cselekedett,
 Mikkel biztatom magamat.
/6
#FE48F7C1
 Tehozzád, én Uram, Istenem,
 Nagy siralommal felemelem
 És kinyújtom kezeimet.
 Tégedet úgy óhajt én lelkem,
 Mint a száraz föld a vizet.
/7
#96B89D08
 Ne késsél, hallgasd meg kérésem,
 Mert elfogy bennem az én lelkem!
 Színed tőlem el ne térjen,
 Mert immár olyanná kell lennem,
 Mint aki száll sír mélyében!
/8
#8777229A
 Kegyelmezz meg, Uram, énnékem,
 Hallgass meg és őrizz meg engem,
 Mert én csak tebenned bízom!
 Útaidat adjad ismernem,
 Mert tehozzád kívánkozom!
/9
#5D7FFF4F
 Szabadíts meg ellenségimtől,
 Kik engem kergetnek ok nélkül!
 Benned vetem reménységem.
 Viselj gondot az én ügyemről!
 Kérlek, Uram, ne hagyj engem!
/10
#D4C17006
 Taníts meg engemet, szolgádat,
 Hogy tehessem akaratodat,
 Mert te vagy az én Istenem!
 Hogy őrizhessem útaidat,
 Szent lelkeddel vezérlj engem!
/11
#7790CEBB
 Uram, engemet erősíts meg,
 Te szent nevedért vigasztalj meg,
 És a te kegyességedből
 Az én életem szabadítsd meg
 Mindennémű ínségekből!

;Háború és béke
;Bourgeois L., Genf, 1562
>144
/1
#2CC6303A
 Áldott az Úr, ki kezemet tanítja,
 És ujjaimat a hadakozásra,
 Áldott legyen az én jó Istenem,
 Aki mindenkor megőriz engem!
 Ő az én kőváram és szabadítóm,
 Ő én paizsom és én oltalmazóm.
 Tebenned vetem reménységemet,
 Mert alám vetéd az én népemet.
/2
#C1F9B5A0
 De micsoda az ember életében,
 Hogy őreá gondot tartasz ekképpen?
 Micsodák az emberek fiai,
 Hogy felséged őket így kedveli?
 Az emberek dolgát ha megtekinted,
 Legottan őket csak semminek leled;
 Az ő napjai hamar elfogynak,
 És mint az árnyék, ottan elmúlnak.
/3
#8B313AEF
 Hajtsd meg az egeket és szállj le menten,
 Illesd a hegyeket, hogy füstjük menjen!
 Küldjél villámlást, őket verd széjjel,
 Mennyütő nyilaiddal széleszd el!
 És segítségül kezed hozzám nyújtsad,
 E nagy árvizet rólam elfordítsad,
 És szabadíts meg veszedelmemből,
 Őrizz a gonosz idegenektől!
/4
#7CE2623D
 Kiknek szájuk merő hazugságot szól,
 Ő hamis kezükkel tesznek gonoszul.
 Uram, néked mondok új éneket,
 Megzendítem néked hegedűmet.
 Mert te vagy, Uram, ki minden ínségben
 Megoltalmazol bennünket kegyesen,
 Dávidot, szolgádat megtekinted,
 Az öldöklő fegyvertől megmented.
/5
#C599D941
 Ments meg kezükből ez idegeneknek,
 Kik én ellenem szörnyen dühösködnek!
 Szájuk beszéde merő hamisság,
 Kezüknek dolga hitvány mulatság.
 Hogy mint a szép plánták, a mi fiaink,
 Felnevekedjenek mi szép leányink,
 Kik szépségükben úgy villogjanak,
 Mint ékes oszlopi a templomnak!
/6
#F3F8C6DE
 Hogy a mi tárházink tele légyenek,
 Csordáink ezeriglen tenyésszenek,
 Barmunk több légyen sok százezernél,
 Mind falun, városon, mezőn széjjel!
 Jól megrakottak légyenek ökreink,
 Városinkra ne üssön ellenségink,
 Földünkből senki el ne vitessék,
 Utcáinkon panasz ne hallassék!
/7
#2676211F
 Ó, boldog nép, akit így megáld Isten,
 És kinek ily jó szerencsét ad itten!
 Boldog a nép és nem lesz szüksége,
 Amelynek az Úr az ő Istene.

;Isten jó és hatalmas
;Genf, 1562
>145
/1
#5A20679C
 Magasztallak téged, én Istenem,
 Ó, én királyom, neved tisztelem!
 Mindennap hirdetem dicséreted,
 És örökké áldom te szent neved!
 Igen dicséretes és nagy az Isten,
 Megfoghatatlan ő dicsőségében.
 Nemzetről nemzetre te dolgaidat,
 Mindenkor hirdetik nagy hatalmadat.
/2
#B6FB0F40
 Éneklem a te dicsőségedet,
 A te felséges dicséretedet!
 Csudatétidet minden népeknél
 Én kibeszélem mindenütt széjjel,
 Hogy hatalmadat mindenek hirdessék,
 Melyet dolgaidban eszükbe vesznek.
 Dicsőségedet én el nem hallgatom,
 De mindeneknek jól előszámlálom.
/3
#733B76E4
 Hogy hirdessék a te jókedvedet,
 Minden nép előtt kegyességedet,
 Igazságodat ők mind dícsérjék,
 És széjjel mindenütt kihirdessék.
 Irgalmas az Úr, fölötte kegyelmes,
 Engedelmes, haragra késedelmes,
 Igen szelíd, a büntetésre késő,
 Minden teremtett rendinek kedvező.
/4
#DB032077
 Azért amit csak teremtél, minden
 Nagy hatalmadról téged dícsérjen,
 De mindennél inkább a szent hívek
 A te dicsőségedet dícsérjék!
 Hirdessék országod fölséges voltát,
 Mindenkor beszéljék nagy hatalmadat,
 Hogy az emberek fiai érthessék,
 Országodnak nagy dicsőségét nézzék.
/5
#30C5468F
 Megáll örökké a te országod,
 És mindenkor tart uralkodásod.
 Az Úr megtartja az eldűlőket,
 És felemeli az elesteket.
 Mert mindeneknek szemei rád néznek,
 Idején eledelt adsz őnekiek,
 Midőn felnyitod áldott kezeidet,
 Bőven megelégítesz mindeneket.
/6
#AB1E2548
 Az Úr igaz minden útaiban,
 És szentséges minden dolgaiban.
 Közel vagyon ő azokhoz nyilván,
 Akik hozzá kiáltnak valóban.
 És amit az istenfélők kívánnak,
 Őnékik nagy bőségben megadatnak;
 Nagy kegyelmesen őket meghallgatja,
 És üdvösségét nékik megmutatja.
/7
#6DC1A82E
 Megtartja, akik szeretik őtet,
 De mind elveszti a hitleneket.
 Az Úr dícséretit szám hirdesse,
 Az ő nevét minden test dícsérje!

;Ne bizakodjunk emberekben!
;Genf, 1562
>146
/1
#75D05224
 Áldjad, én lelkem, az Urat,
 Hirdessed dicséretét!
 Az Istennek adok hálát,
 Valamíg engem éltet.
 Én az Úrnak éneklek,
 Mindaddig, míglen élek.
/2
#74BC08D4
 Ne légyen bizodalmatok
 Földi fejedelmekben.
 Egy emberben se bízzatok,
 Kiben segítség nincsen!
 Mihelyt lelke kimégyen,
 Menten hamuvá lészen.
/3
#625D7A15
 Minden dolga és szándéka
 Elvész azon nap vele.
 Boldog, akinek oltalma
 A Jákób Istene;
 Akinek mindenekben
 Reménysége az Isten.
/4
#0CD37781
 Ki a mennyet és a földet
 És a tengert teremté,
 És ezekben mindeneket
 Nagy hatalmával szerze,
 Igazsága, hűsége megmarad mindörökre.
/5
#8E4EA842
 Akik méltatlan szenvednek,
 Megmenti a jó Isten,
 A nyomorult éhezőknek
 Ő ád eledelt bőven,
 A foglyokat kihozza,
 Rabságból kioldozza.
/6
#9545A185
 És a világtalanoknak
 Megnyitja ő szemöket;
 Akik dűlőfélben vannak,
 Meggyámolítja őket,
 Mert az igazat híven
 Szereti az Úr Isten.
/7
#C0020CB8
 Veszedelmében megmenti
 A nyavalyás jövevényt;
 Az árvákat megsegíti,
 Rájuk kegyesen tekint;
 A sérelmes özvegyek
 Tőle megenyhíttetnek.
/8
#1EC50E80
 A hitleneket megrontja,
 Ösvényüket elvesztve,
 De megáll az ő országa
 Most és minden időbe'.
 Ó, Sion, a te Urad
 Mindörökké megmarad!

;Isten hatalma és gondviselése
;Genf, 1562
>147
/1
#489D4B9F
 Az Urat dicsérjétek, mert jó,
 És az Istent dicsérni méltó;
 Kedves dolog az Úr Istennél,
 Hogy őtet dicsérjék egy szívvel,
 Mert Jeruzsálemet az Isten
 Megépíti nagy kegyelmesen,
 Ismét az Izráel nemzetjét,
 Öszszegyűjti eloszlott népét.
/2
#0D3F56AF
 Töredelmes szívet meggyógyít,
 Sérelmes lelket ő megenyhít,
 Békötözi az ő sebüket,
 Megkönnyebbíti sérelmüket.
 A csillagokat megszámlálja,
 És azoknak számát jól tudja,
 És mindeniket úgy ismeri,
 Hogy tulajdon nevén nevezi.
/3
#09CE0924
 Nagy a mi Urunk, az Úr Isten,
 Akinél nagyobb semmi nincsen;
 Erőssége kimondhatatlan,
 Bölcsessége számlálhatatlan.
 Az együgyűket felemeli,
 A szelídeket erősíti,
 Megalázza a hitleneket,
 A földre lealázza őket.
/6
#870C9A6B
 Inkább azokban gyönyörködik,
 Kik igazán csak őtet félik,
 Kik teljes szívből kegyes voltát,
 Reménylik mindenkor irgalmát.
 Dicsérd, Jeruzsálem, az Urat,
 Felmagasztaljad nagy hatalmát,
 E kegyes Urat, ó, szent Sion,
 Te is dicsérd áhítatoson!
/10
#655BC200
 Jákóbnak adá szent igéjét,
 Hogy ahhoz szabják életüket;
 Törvényit adá Izráelnek,
 Hogy ők azok szerint éljenek.
 Nem tőn így semmi pogány néppel,
 Nem látá őket ily szentséggel,
 Szent szerzésit ők nem ismerték,
 Ezért Alleluja mondassék!

;Egész világ Istent dicsérje!
;Genf, 1562
>148
/1
#8FD1D1FD
 No, dicsérjétek mindnyájan
 Az Urat a mennyországban!
 Őt dicsérjétek az égben,
 Az ő felséges székében!
 Minden angyalok őt dicsérjék,
 És minden menynyei seregek,
 A nap és hold őt dicsérje
 Minden csillagokkal egybe'!
/2
#A6F992B4
 Az Urat ti magas egek,
 Dicsérjétek minden vizek,
 Mik fenn az égben lebegtek,
 Az Úr nevét dicsérjétek!
 Mert mindeneket igéjével,
 Ő teremte nagy erejével,
 Mindent úgy megerősíte,
 Hogy megálljon mindörökre.
/5
#D253B5B1
 Ifjak, leányok és vének,
 Az Úr Istent dicsérjétek,
 Mert nevének dicső volta
 Mennyet-földet felülmúlja!
 Arcát felemeli népének,
 Minden szentek, őt dicsérjétek!
 Izráel választott népe
 Az Úrnak nevét dicsérje!

;Győzedelmi ének
;Genf, 1562
>149
/1
#45DAC1E9
 Az Úrnak, no, énekeljetek
 Új éneket szent felségének!
 A szentek gyülekezetében
 Dicsérje Őtet minden!
 Az Izráel örvendezzen
 Ő teremtő Istenében,
 Királyuknak örüljenek
 A Sionbéliek!
/2
#4F83071A
 Az Úr nevét síppal és dobbal,
 Dicsérjétek lanttal, kobozzal,
 Légyen hegedűknek zengése
 Neve dicséretére!
 Mert az Isten az ő népit,
 Igen szereti híveit,
 A szegényeket megmenti,
 Sok jókkal szereti.
/3
#033A8890
 Szent hívei az Úr Istennek
 Nagy tisztességet tőle nyernek;
 Örülnek az ő hálóházukba’,
 Az Urat magasztalva.
 Dicséretét az Istennek
 Az ő szájukban viseljék,
 És kétélű fegyver lészen
 Nékiek kezükben!

;Minden lélek dicsérje az Urat!
;Genf, 1562
>150
/1
#9167EF69
 Dicsérjétek az Urat!
 Áldjátok ő szent voltát!
 Dicsérjétek menynyekben,
 Hol országol kegyesen
 Az ő nagy dicsőségébe'!
 Dicsérjétek hatalmát,
 Melyből dicső nagy voltát
 Minden veheti eszébe!
/2
#DA84D30C
 Dicsérjétek őt kürtben,
 Ékes éneklésekben;
 Hegedűkben, lantokban
 És hangos citerákban
 Az Úrnak zengedezzetek!
 Sípokban, orgonákban
 És más szép muzsikákban
 Örvendjetek az Istennek!
/3
#3CF3A4E3
 Az Urat cimbalmokban
 És egyéb szerszámokban
 Mindnyájan dicsérjétek,
 Citerát pengessetek,
 Az Úr szent nevét dicsérvén!
 Minden lelkes állatok,
 Istent magasztaljátok:
 Dicsőség Istennek! Ámen.

;Debrecen, 1774 (1778)
>151
/1
#B4CA0E3B
 Uram Isten, siess
 Minket megsegíteni
 Ily nagy szükségünkben,
 Krisztus Jézusért,
 Mi Urunkért
 És Megváltónkért.

>152
/1
#A392A8B3
 Szent Isten, noha néked
 Az egek illő széked.
 Ott dicsérnek tégedet.
 Biztat szent igéreted.
 Hogy azért meg nem veted
 Földön lakó népedet.
 Együtt vagy vélek,
 Kikben alázatos a lélek.
 Mi is hát térdet hajtunk,
 Hogy rajtunk könyölülj s kegyelmeddel
 Fogadd el,
 Mikor tehozzád emel kezet
 Ez a gyülekezet.

>153
/1
#13E24BBC
 Ó mely boldog ember az,
 Ki téged, élő, igaz
 Egy Istent megismerhet,
 És te szent hajlékodban
 Lélekben, igazságban
 Dicsérhet és tisztelhet.
 Ó mi Istenünk,
 Adtad e boldogságot nekünk;
 Adjad, hogy ismerhessünk
 Még jobban,
 Naponként dicsérhessünk
 BuzgóbbanI
 Szent Lelkedet hozzánk e végre
 Küldd el segitségre!

>154
/1
#C7542EB3
 Úr Jézus, mely igen drága
 A te igédnek világa,
 Mely bölccsé tévén az elmét,
 Szüli az Úrnak félelmét.
 Gerjeszd fel most indulatunk,
 Hogy mig igédre hallgatunk,
 Végyen bennünk épületet
 A hit, reménység és szeretet:
 Tégy bölcsekké, tégy szentekké!

>155
/1
#2AD7D877
 Ó Úr Isten, légy közöttünk,
 Kik imádásodra jöttünk,
 Szivünk egyedül rád hallgasson,
 Hogy igédnek áldott szava,
 Melytől függ lelkünknek java,
 Minket minden jóra bírhasson:
 Szent e hely, légyenek szentek,
 Kik e házban megjelentek.

>156
/1
#626E41E5
 Úr Isten, mi sok szükséget érezvén körülöttünk,
 Segedelem-kérés végett
 E szent helyre feljöttünk.
 Uram, a könyörgőnek
 Illesd meg szivét és száját,
 A téged tisztelőnek
 Áldd és szenteld meg munkáját.

>157
/1
#D2127248
 Könyörülj rajtunk Úr Isten!
 És hallgasd meg a mi imádságinkat!

>158
/1
#461129F8
 Ha te meg nem tartasz, Uram Isten,
 Hiába vigyáznak a mi szemeink.

>159
/1
#73D438DD
 Atya, Fiú, Szentlélek, teljes Szentháromság.
 Egy és örök Istenség,
 Mennybéli uraság;
 Igaz, erős, kegyelmes,
 Jó, hív irgalmasság:
 Téged áld ég, föld s tenger:
 Áldott Szentháromság!

>160
/1
#49F0A2B7
 Minden teremtett állatok az
 Úr Istent áldják,
 néki sok ezer angyalok
 'Szent, szent, szent' azt mondják.

;Tinódi Lantos Sebestyén (1549)
>161
/1
#9924CD7A
 Siess, keresztyén, lelki jót hallani,
 Régi törvényből harcolni tanulni,
 Az igaz hit mellett mint kell bajt vívni,
 Krisztusban bízni.
/2
#937CB98A
 Mert nem hiába ezt az ó törvénybe,
 Próféták írták Biblia könyvébe;
 Szép tanulság ez most az új törvénybe',
 Mi eleinkbe'.
/3
#6F020386
 Jól tudja földön ezt minden keresztyén:
 Nem csak fegyverrel oltalmaz az Isten.
 Ezt minden népnek tudására adom:
 Istenünk vagyon!
/4
#08FCF486
 Fejedelemség vagyon csak Istenben,
 Minden hatalom vagyon ő kezében;
 Kiket ő akar, föld kerekségében:
 Emeli égben.
/5
#D2872B40
 Ne ess kétségbe ő nagy jóvoltában,
 Az igaz hitben erős légy magadban,
 Mint Dávid, úgy jársz párviadalodban,
 Hitvallásodban.
/6
#ACBEDBD7
 Dávidot Isten hagyá királyságban,
 Ő ellenségit veté gyalázatban.
 Dicsérjük Istent nagy hálaadásban,
 Énekmondásban.

;Debrecen, 1774
>162
/1
#7ABF30CC
 Ím, béjöttünk nagy örömben,
 Felséges Isten,
 A te szentidnek gyülekezetébe,
 A te templomodba,
 Felséges Atya Isten.
/2
#5F6D2B48
 Itt megállunk teelőtted,
 Felséges Isten!
 És igaz hitből áldozunk előtted,
 Vallást teszünk rólad,
 Felséges Atya Isten!
/3
#57584B60
 Vágyik lelkünk szent Igédhez,
 Felséges Isten!
 Mint a szomjúhozó szarvas a vízhez,
 A hideg kútfőhöz,
 Felséges Atya Isten!
/4
#1972874C
 Örvendezünk mi szívünkben,
 Felséges Isten!
 Mert bejutottunk immár teelődbe,
 A te templomodba,
 Felséges Atya Isten!
/5
#D88A617A
 Csak ez nékünk vigasságunk,
 Felséges Isten!
 Hogy lakozik a te neved közöttünk,
 Dicsértetel tőlünk,
 Felséges Atya Isten!
/6
#3240EBC7
 Tartsd meg azért békességben,
 Felséges Isten,
 E kicsiny seregecskét igaz hitben,
 Te tiszteletedben,
 Felséges Atya Isten!
/7
#678B55B5
 Prédikáltasd szent igédet,
 Felséges Isten!
 Ne hagyd szomjúhozni a mi lelkünket,
 Áldd meg életünket,
 Felséges Atya Isten!
/8
#CE851140
 Zengedeznek mi ajakink,
 Felséges Isten!
 Örvendetes szókban, dicséretekben,
 Ékes énekekben,
 Felséges Atya Isten!
/9
#146E75BF
 Áldott vagy te magas mennyben,
 Felséges Isten!
 Kit illet dicséret a szent templomban,
 Anyaszentegyházban,
 Felséges Atya Isten!

;Debrecen, 1774
>163
/1
#09DBDBA9
 Örvend mi szívünk,
 Mikor ezt halljuk:
 A templomba mégyünk,
 Hol Úr Istennek
 Szent Igéjét halljuk.
/2
#7D85C2ED
 Megállunk hittel,
 Örök Úr Isten,
 A te templomodban,
 És tiszta szívvel
 Dicsérünk mi téged.
/3
#A030DDD2
 Áldd meg, Úr Isten,
 A te népedet,
 Kik téged szeretnek,
 Tartsd meg közöttünk
 A gyülekezetet.
/4
#D918604A
 Légyen békesség,
 Felséges Isten,
 Anyaszentegyházban,
 Oltalmazz minket
 Minden háborúnkban.
/5
#9E94AE27
 Dicséret neked,
 Atya Úr Isten,
 A te szent Fiaddal,
 És Szentlélekkel,
 Mi vigasztalónkkal!

;Ahle J. R., 1664
>164
/1
#47F51996
 Kegyes Jézus, itt vagyunk
 Te szent Igéd hallására,
 Gyúljon fel kívánságunk
 Idvesség tanulására,
 Hogy a földtől elszakadjunk,
 Csak tehozzád ragaszkodjunk.
/2
#0652CCEC
 Elménket, értelmünket
 Lelki sötétség fogta bé,
 De szent Lelked szívünket
 Tiszta fénnyel úgy töltse bé,
 Hogy jót gondoljunk és szóljunk,
 Mert csak tőled kell azt várnunk.
/3
#D248669C
 Dicsőségnek napfénye,
 Istentől jött világosság!
 Indítsd lelkünk készségre,
 Nyisd meg fülünk, szívünk és szánk!
 Hitvallásunk, könyörgésünk,
 Urunk Jézus, halld meg, kérünk!

;Bremen, 1680
>165
/1
#A8281FB2
 Itt van Isten köztünk,
 Jertek őt imádni,
 Hódolattal elé állni,
 Itt van a középen, minden csendre térve
 Őelőtte hulljon térdre.
 Az, aki Hirdeti S hallja itt az Ígét:
 Adja néki szívét!
/2
#72D9ED85
 Itt van Isten köztünk:
 Ő, kit éjjel-nappal
 Angyalsereg áld s magasztal.
 Szent, szent, szent az Isten!
 Néki énekelnek
 A mennyei fényes lelkek.
 Halld, Urunk, Szózatunk,
 Ha mi, semmiségek
 Áldozunk Tenéked!
/3
#6ECC1C23
 Csodálatos Felség,
 Hadd dicsérlek Téged:
 Hadd szolgáljon lelkem Néked!
 Angyaloknak módján
 Színed előtt állván, bárcsak mindig orcád látnám!
 Add nékem Mindenben
 Te kedvedben járnom,
 Istenem, Királyom!
/4
#B120B881
 Általjársz Te mindent;
 Rám ragyogni engedd
 Életadó, áldott Lelked!
 Mint a kis virág is Magától kibomlik,
 Rá ha csöndes fényed omlik:
 Hagyj, Uram, Vidáman
 Fényességed látnom
 S országod munkálnom!
/5
#6FBC2510
 Egyszerűvé formálj
 Belső, lelkiképpen, békességben, csöndességben.
 Tisztogasd meg szívem:
 Tisztaságod lássam
 Lélekben és igazságban.
 Szívemmel Mindig felszállhassak sasszárnyon:
 Csak Te légy világom!
/6
#67182127
 Jöjj és lakozz bennem:
 Hadd legyen már itt lenn
 Templomoddá szívem-lelkem!
 Mindig közellévő: jelentsd Magad nékem,
 Ne lakhasson más e szívben;
 Már itt lenn Mindenben
 Csakis Téged lásson,
 Leborulva áldjon!

;Görlitz, 1648
>166
/1
#54E732EF
 Urunk Jézus, fordulj hozzánk,
 Szent Lelkedet ma töltsd ki ránk,
 Kegyelmeddel minket segélj,
 Az egy Igazságra vezérlj.
/2
#7D4A7213
 Nyisd meg szánkat hál'adásra,
 Készítsd szívünk buzgóságra,
 Hitünk s értelmünk neveljed,
 Neved velünk ismertessed.
/3
#23E87C7E
 Míglen éneklünk mennyégben:
 Szent, szent, szent az erős Isten,
 És színről színre láthatunk,
 A fényességben vigadunk.
/4
#3FD787F8
 Dicsőség Atya Istennek,
 Fiúnak és Szentléleknek:
 A dicső Szentháromságnak
 Mindenek áldást mondjanak!

;Crüger J., 1647
>167
/1
#37202C3C
 Jöjj, mondjunk hálaszót
 Hűszájjal és hű szívvel,
 Mert rajtunk itt az Úr
 Nagy csoda dolgot mível.
 Már anyaölben is
 Volt mindig gondja ránk.
 A sok jót, mellyel áld,
 Ki sem mondhatja szánk.
/2
#FB3B523F
 Dús kincséből az Úr
 Jó békességet adjon,
 Hogy szívünkben a kedv
 Víg és derűs maradjon.
 Ne hagyja híveit
 Bú-bajban sohasem;
 A rossztól óvja meg
 Itt s túl ez életen.
/3
#CD096000
 Az Atyát és Fiút
 És a Szentlelket áldom;
 A menny Urát, kiben
 Szent egybe forrt a három;
 Aki úgy szól ma is,
 Ahogy régente szólt,
 Nem változik: Az Ő,
 És az lesz, aki volt.

;Langran Jakab, 1807-1889
>168
/1
#6F8F5E12
 Ó, Atya Isten, irgalmas nagy Úr,
 Bűnbánó szívvel ím eléd borul
 Hű néped, áldva felséges neved,
 Hogy esdve kérje nagy kegyelmedet.
/2
#EF78AE50
 Gondolsz ránk, híven oltalmaz kezed,
 Rólunk egy percre azt le nem veszed,
 Irgalmasságod mindig oly közel,
 És erős karod minket átölel.
/3
#5076DBC6
 Nagy jóságodra méltók nem vagyunk,
 Rossz útra térve gyakran elhagyunk;
 Áhítjuk mégis szent igéd szavát,
 Megtérő gyermekid fogadd be hát.
/4
#D7CCC3C7
 Kérünk, Úr Isten, Krisztus-Jézusért,
 Vérrel pecsételt szent szerelmedért:
 Irgalmasságod közöld mivelünk,
 És tárd ki szíved, végy be, Istenünk!

>169
/1
#BF298918
 Úr Isten, téged szívünkből dícsérünk;
 Végye Felséged most hát kedvesen,
 mindőn szívesen imádunk és kérünk.

;Debrecen, 1781
>170
/1
#D1E41F51
 Jövel, ó, áldott Szentlélek!
 Gerjeszd fel híveidet,
 És szent munkájokban vélek
 Közöld kegyelmeidet,
 Hogy szívünk egyenesen
 Az Úrra függesztessen.

;Kolozsvár, 1907
>171
/1
#FAA0BB99
 Megáll az Istennek Igéje,
 És nem állhat senki ellene,
 A nagy Isten vagyon mivelünk,
 És Szentlelke lakozik bennünk.

;Kolozsvár, 1785
>172
/1
#D7E75704
 Szűkölködünk Nagymértékben Segedelem nélkül,
 Reménykedünk, Örök Isten, Te légy segítségül.
 Dicsérhessünk és lehessünk
 Jézus szava hallgatói,
 Igaz megtartói.

>173
/1
#83ADA5B4
 Nem vagyunk mi magunkéi,
 De Jézus vére bére;
 Lelkünk, testünk Istenéi
 Az ő tiszteletére. Urunk Jézus,
 jöjj most el, Lelked által segíts fel!
 Dicsőítünk mi testünkben,
 Magasztalunk mi lelkünkben!

;Debrecen, 1778
>174
/1
#5610EEEF
 Atya, Fiú, Szentlélek,
 Készíts szent Igédre,
 Jézus érdemét ruházd
 Áldott híveidre.

>175
/1
#F55FF16B
 Gyűjtsünk oly kincset, melyet a tűz koha,
 Hév forrósága nem emészt meg soha.
 Tűzben megpróbált arany ez a kincs,
 Melyhez semmi veszélynek jussa nincs.
 A koporsónak sincs rajta hatalma,
 Elvesze reá nézve diadalma;
 Elkísér lételünknek céljára,
 A Teremtó színe látására.

>176
/1
#4D84F1C6
 Szent, szent, szent a Seregeknek ura!
 Teljes mind a széles föld az ő dicsőségével!
 Ámen.

>177
/1
#9DB000DD
 Urunk, irgalmazz nékünk!
 Krisztus, irgalmazz nékünk!
 Urunk, irgalmazz nékünk!

>178
/1
#B8E28ACE
 Dicsőség az Atyának,
 A Fiúnak és Szentlélek Istennek,
 Valamint volt kezdetben,
 Azonképpen most és mindörökké,
 Ámen, Ámen.

>179
/1
#6F826F12
 A bűnből hozzád sietek,
 mert szomjazom kegyelmedet,
 te nyugtass meg, én Jézusom,
 ím néked szívem átadom.

;Hassler H. L., 1601
>180
/1
#98DA5151
 A töredelmes szívet,
 Te, Uram, szereted,
 Az engedelmes lelket
 Soha meg nem veted.
 Ezzel a reménységgel
 Tehozzád óhajtunk,
 Légy, kérünk, segítséggel
 És könyörülj rajtunk.

>181
/1
#A336E09F
 Jer, áldott vendég, várunk tégedet,
 tisztítjuk a te útadat, egyengetjük a te ösvényedet:
 mutasd meg nálunk magadat!
 Erős hittel szívünkbe fogadunk,
 állandó szállást ott adunk.

;Debrecen, 1774
>182
/1
#783AFD6B
 Karácsony ünnepében,
 Karácsony ünnepében
 Örvendezünk szívünkben:
 Mert Isten ő szent Fiát,
 Mert Isten ő szent Fiát
 Adta meg nékünk testben,
 Ki az ő népét
 Megszabadítá a bűnöktől,
 És a régi kígyótól Megmenté,
 Nékünk a dicsőséget Megnyeré:
 Légyen dicsőség Királyunknak
 Most és mindörökké!

;Wittenberg, 1523
>183
/1
#D74B39B0
 Istennek Báránya,
 Ki bűnünket elveszed:
 Irgalmazz nékünk!
 Istennek Báránya,
 Ki bűnünket elveszed:
 Irgalmazz nékünk!
 Istennek Báránya,
 Ki bűnünket elveszed:
 Add ránk békességed!
 Ámen, Ámen.

;Erfurt, 1542
>184
/1
#B773F2D1
 Krisztus, ártatlan Bárány,
 Ki miértünk megholtál,
 A keresztfa oltárán
 Nagy engedelmes voltál.
 Hordozván bűneinket,
 Te váltottál meg minket:
 Irgalmazz nékünk, ó, Jézus!

;1200 körül
>185
/1
#D5504C80
 Krisztus feltámadott,
 Kit halál el ragadott;
 Örvendezzünk, vigadjunk,
 Krisztus lett a vigaszunk, Alleluja!
 Ha ő fel nem támad,
 Nincs többé bűnbocsánat,
 De él, ezért szent nevét,
 Zengjük ő dicséretét,
 Alleluja, Alleluja! Alleluja!
 Örvendezzünk, vigadjunk,
 Krisztus lett a vigaszunk. Alleluja!

;Debrecen, 1774
>186
/1
#3E581946
 E húsvét ünnepében,
 E húsvét ünnepében
 Dicsérjük Istent szívvel,
 Ki értünk megholt Fiát,
 Ki értünk megholt Fiát
 Feltámasztotta testben.
 Ennek örül föld,
 Tenger és a menny víg kedvében;
 Minden élő állatok
 Sok rendben,
 Mik égben, földön vannak,
 Sok részben,
 Fák, füvek és minden virágok
 Újulnak örömben.

;Debrecen, 1774
>187
/1
#D40254C5
 E húsvét ünnepében,
 E húsvét ünnepében
 Örvendjünk, keresztyének!
 Szívünk teljességében,
 Szívünk teljességében
 Illik szánkba víg ének.
 A feltámadott Jézus nékünk zálogot adott,
 Hogy bár a föld gyomrába Tétetünk,
 Megújul valójába' Életünk;
 E hittel midőn ünnepelünk,
 Te légy, Jézus, velünk!

;Debrecen, 1774
>188
/1
#9409388E
 E pünkösd ünnepében,
 E pünkösd ünnepében
 Dicsérjük Istent szívvel,
 Ki Szentlelkét szívünkben,
 Ki Szentlelkét szívünkben
 Osztogatja bőséggel.
 Ennek örül föld,
 Tenger és a menny víg kedvében;
 Minden élő állatok
 Sok rendben,
 Mik égben, földön vannak
 Sok részben,
 Fák, füvek és minden virágok
 Újulnak örömben.

;Debrecen, 1774
>189
/1
#A56431A8
 E pünkösd ünnepében,
 E pünkösd ünnepében
 Zeng nyelvünk dicséreti,
 Mert az Úr Szentlelkében,
 Mert Az Úr Szentlelkében
 Híveit részelteti:
 Melynek ereje
 Minden ismeretnek kútfeje;
 Világosságot ő gyújt Szívünkben,
 E rőt, bátorságot nyújt Éltünkben,
 Áldott Lélek, te légy mellettünk,
 S újjá születtetünk.

;Debrecen, 1774
>190
/1
#439ED2F2
 Ez esztendőt megáldjad,
 Ez esztendőt megáldjad
 Kegyelmedből, Úr Isten!
 Bőséggel ékesítsed,
 Bőséggel ékesítsed,
 Te szent Jehova Isten!
 Tégedet áldnak,
 Kik lakoznak e földnek színén;
 És a puszta helyek is Bőséggel,
 Halmok áhítozódnak Víg kedvvel;
 Boldog, kit magadnak választál,
 Sionnak királya!

;Debrecen, 1774
>191
/1
#BC2F4548
 Ez esztendőt áldással,
 Ez esztendőt áldással,
 Koronázd meg, Úr Isten,
 Hogy víg hálaadással,
 Hogy víg hálaadással
 Dicsérje neved minden.
 Tetőled jőnek
 Kedves napjai az időnek;
 Nyújts hát, kérünk, bőséget Kezeddel,
 Tőlünk a békességet Ne vedd el;
 De midőn megtartod testünket,
 Ne hagyd el lelkünket!

;Kolozsvár, 1744
>192
/1
#64DE3BAA
 Adj békességet, Úr Isten,
 A mi időnkben a földön,
 Mert nincsen nékünk több senki
 Bajvívónk és hadakozónk,
 Hanem csak te, Úr Isten.

;Debrecen, 1774
>193
/1
#4D4E6630
 Az Úr Jézus Krisztusnak kegyelme,
 És Atyánknak, Istennek szerelme,
 Szentlélekkel áldott közösségben
 Légyen mindig mindnyájunkkal. Ámen.

>194
/1
#887EF214
 Tartsd meg e gyülekezetnek,
 Melyet törvényid vezetnek,
 Nagyjait, kicsinyjeit;
 Igazgassad tanítóit,
 Hüséges előljáróit,
 Felsőbb és köz rendeit.
 Adjad, hogy szívvel-lélekkel
 Dicsérhessünk a szentekkel,
 S csak téged tisztelhessünk.
 És ha eljő amaz óra,
 Melyben szállunk koporsóba:
 Mennybe hozzád mehessünk!

;Crüger J., Berlin, 1653
>195
/1
#759BFEBC
 Áldjuk Istent végével
 Isteni tiszteletünknek,
 Mondjunk immár örömmel
 Dicséretet szent nevének!
 Mi Istenünk legyen áldott,
 Hogy lelkünkben gazdagított.
/2
#EFCE0551
 Immár Isten áldása
 Kimondatott mindnyájunkra,
 Menjünk örömmel haza
 Tisztünkre és dolgainkra.
 Ő Szentlelke erősítsen,
 Minket tovább is vezessen.
/3
#4851481E
 Áldja meg kimentünket,
 Áldja bemenetelünket,
 Áldja meg kenyerünket,
 Szentelje meg életünket;
 Áldjon meg boldog halállal,
 Végre örök boldogsággal.

;Debrecen, 1774
>196
/1
#4B52E3D6
 Mondjatok dicséretet,
 Keresztyének, az Úr Istennek,
 Szentséges tiszteletet
 Adjatok ő nevének.
/2
#5478B6D1
 Örvendjetek őneki
 Igaz isteni buzgóságban,
 Mert köztetek lakozik
 Az Anyaszentegyházban.
/3
#20683592
 És ő megelégíti
 Lelketeket lelki kenyérrel;
 Biztat ő igéjével
 Édes ígéretivel.
/4
#92317F06
 Szent, Uram, a te neved,
 Szent vagy te a magas mennyekben,
 Szent vagy és rettenetes
 Minden nemzetségekben.
/5
#172FC17F
 Tégedet illet, Uram,
 Mi tiszteletünk, háladásunk,
 A mi gyülekezetünk,
 Ünnepet szentelésünk.
/6
#D6BC9B39
 Hála légyen tenéked,
 Hogy megjelentéd te magadat,
 Hogy megértettük, Uram,
 A te akaratodat.
/7
#3E3E5BEF
 Dicséret és dicsőség
 Légyen tenéked magasságban,
 Dicséret és tisztesség
 Az Anyaszentegyházban.

>197
/1
#71475A11
 Hálaadásunkban rólad emlékezünk,
 Kegyelmes Istenünk, tégedet tisztelünk
 Te nagy jóvoltodért, és felmagasztalunk,
 Haszontalan szolgák, mert ezzel tartozunk.
/2
#88053309
 Hajtsd le füleidet, értsd meg kérésünket,
 Mi édes Istenünk, ne nézd bűneinket;
 Tiszta szívből kérünk, mi könyörgésünket
 Vedd hozzád, Úr Isten, esedezésünket.
/3
#A834EBD4
 Gerjeszd fel igédnek szerelmét lelkünkben,
 Szent­lélek Istennek szentségét szívünkben;
 Adj igaz értelmet bőséggel elménkben,
 Téged szolgálhassunk e földön éltünkben.
/4
#694B6CA5
 Tanúságra erőt csak te tőled kérünk,
 Hogy szent istenséged dícsértessék tőlünk:
 Ha először a te országod keressük,
 Semmiben nem lészen fogyatkozás nékünk.
/5
#8D092106
 Azért gyorsaságot elkezdett dolgunkban,
 Végig megmaradást mi hivatalunkban,
 Szorgalmatosságot s hűséget tisztünkben,
 Egymás-szeretetet plántálj életünkben.
/6
#15B112FB
 Engedd meg ezeket, Szent Atyánk, Krisztusért,
 Mi Üvezítőnkért, kegyes táplálónkért,
 Ki egy hatalomban Szentlélek Istennel
 Élsz és uralkodol mindörökké, Ámen.

;Crüger J., 1562
>198
/1
#37A0B116
 Ti keresztyének, dicsérjétek Istent,
 Kik Úr Jézusnak e földön szolgáltok
 És mindenkoron a szent helyen vagytok:
 Dicsérjétek Istent!
/2
#7580A3BD
 Imádkozzatok az Atya Istennek,
 Jézus nevében hozzá siessetek,
 És tiszta szívet hozzá emeljetek:
 Dicsérjétek Istent!
/3
#7F4BD36A
 Áldjon meg minket az Atya Úr Isten,
 Megigazítson a Fiú Úr Isten,
 És megszenteljen a Szentlélek Isten
 Mindenkoron! Ámen.

;Debrecen, 1774
>199
/1
#932281CF
 Adjunk hálákat az Atya Istennek,
 Mennynek és földnek szent teremtőjének,
 És embereknek gondviselőjének,
 Éltetőjének.
/2
#CB0007AE
 Mert ő mihozzánk atyai szerelmét,
 Kijelentette drága szent igéjét,
 Mellyel táplálja híveinek lelkét,
 Nyújtja kegyelmét.
/3
#A9FB4BB7
 És ő megáldja benne reménylőket,
 Erősít minden erőtelenséget,
 Világosítja homályos szívünket,
 Setét elménket.
/4
#6363DA77
 Atya Istennek mindezekért légyen
 Dicséret és nagy dicsőség mennyégben,
 Fiával s Szentlelkével egyetemben,
 Örökké! Ámen.

;Vulpius Menyhért, 1609
>200
/1
#D024AEA5
 Ó, maradj kegyelmeddel
 Mivelünk, Jézusunk,
 Hogy a bűnös világnak
 Tőribe ne jussunk.
/2
#530F8C6A
 Ó, maradj szent igéddel
 Mivelünk, Megváltónk,
 E földi vándorlásban
 Te légy útmutatónk.
/3
#61D5B448
 Ó, maradj, világosság,
 Mivelünk fényeddel,
 Te vezess a sötétben,
 Hogy ne tévedjünk el.
/4
#A6BBB1AC
 Ó, maradj áldásoddal
 Mivelünk, Úr Isten,
 Szent kegyelmed áraszd ránk
 Minden szükséginkben.
/5
#546B1953
 Ó, maradj oltalmaddal
 Mivelünk, hű pajzsunk,
 Hogy e világ diadalt
 Ne vehessen rajtunk.
/6
#4152E932
 Ó, maradj hűségeddel
 Mivelünk, szent Isten,
 Adj erőt, hogy megálljunk
 Mindvégig a hitben.

>201
/1
#23631446
 Uram, kegyes indulatomban
 Ím, teszem vallásom:
 Ez életemben, halálomban
 Fő vigasztalásom:
 Hogy én az Úr Jézus Krisztusnak
 Tulajdona vagyok,
 Rajtam irgalmas szent voltának
 Kegyelmei nagyok.
/2
#38ED06CD
 Ki, midőn voltam halál révén,
 Teljes bűn mérgével,
 Megváltott, ügyemet felvévén,
 Drága szent vérével.
 Az ördög terhes fogságából
 Híven feloldozott,
 A bűnnek kemény igájából
 Szabadságra hozott.
/3
#E5302E2D
 Ezután sem hágy el végképen
 Kivánt segedelme,
 Sőt, mint fiát, engem akképen
 Takargat védelme.
 Gondot visel én életemre
 Hatalmas szent karja,
 Semmi gonosz nem száll fejemre,
 Ő ha nem akarja.
/4
#4662324B
 Ha midőn atyai vesszejét
 Felvonja ellenem,
 S bölcs dorgálásának erejét
 Kelletik szenvednem:
 Akkor is lankadott szívemet
 Szentlelke bíztatja,
 Hogy rajtam lévő veszélyimet
 Javamra fordítja.
/5
#C550DA67
 Reménységem neveli bennem
 S bizonyossá tészen,
 Hogy végre jutalmam énnékem
 Örök élet lészen,
 Melyre is készíti lelkemet
 És hívja magához.
 Hogy szabjam egész életemet
 Szent akaratjához.
/6
#4D0EA244
 Uram, taníts gyarló éltemnek
 Nagy bűnös voltára,
 Ó vezess én árva fejemnek
 Szabadítójára:
 Magamat néki hadd adhassam
 Lelki ajánlással,
 Drága jóvoltát felválthassam
 Méltó hál'adással.

>202
/1
#B0B905BB
 Fő boldogságom tartom én ebben,
 Hogy az Úr Jézus tudományát vallom,
 Ember nem szólott soha bölcsebben,
 Azért beszédit nagy örömmel hallom.
 Ezek éltemnek vezéri lévén,
 Megvfgasztalnak a halálnak révén.
/2
#513E786E
 Ha bűnöm kárhozattal fenyeget,
 E hitben vetem teljes reménységem,
 Hogy Megváltóm tett értem eleget;
 Nem árthat nékem semmi ellenségem.
 Ő visel gondot úgy életemről,
 Hogy egy hajszál sem esik le fejemről.
/3
#6D898167
 Megfoghatatlan nagy szerelmével
 Jézusom úgy vont kebelére engem,
 Hogy lelkem-testem tehetségével
 Őneki örök tulajdona lettem.
 Szent akaratját örömest, készen
 Teljesfteni leafőbb gondom lészen.
/4
#EF99ACA0
 Add Uram, kérlek, tudomásomra,
 Hogy nyavalyámnak okát hol keressem,
 Hogy óhajtott szabadulásomra
 Segedelmeddel hadd törekedhessem.
 Jótéteményed mily csudálatos!
 Oktass, hogy légyek hozzád háládatos.

>203
/1
#D149E01A
 Mennybéli Úr Isten, hallgass meg engemet az én könyörgésemben,
 Ne hagyj el engemet ily nagy nyavalyámban, tekints meg szükségemben;
 A nagy bánat miatt már adj vígasságot az én nagy bús szívemben,
 A te Szentlelkeddel vigasztalj meg kérlek, én keserűségemben.
/2
#3634AB16
 Jól tudom, micsoda nagy okai vannak én nyomorúságomnak:
 Mert az álnok világ hízelkedésinek, hittem csalárdságának;
 Az ő szép szavának, sok ajándékának, nékem bút hozának,
 Engemet mindenben ez világ kedvébe: jaj, ím mibe hajtának!
/3
#A1719AF8
 Jaj nékem, bűnösnek és gyarló embernek, ha rajtam nem könyörűlsz,
 Azén nagy bűnömért, hatalmas Istenem, nékem meg nem kegyelmezsz;
 Soha nem lehet így nékem vígasságom, hogyha bűnöm szerint versz,
 Már minek örüljek, felséges Úr Isten, ha velem jól nem tészesz?
/4
#D45C4FF0
 Hová legyek Uram, ha elhagysz engemet, kihez hajtsam fejemet?
 Föld kerekségében sehol nem találom, kire bízzam lelkemet;
 Nincsen itt e földön, ki elvenné rólam az én nagy sok bűnömet,
 De nagy félelemmel te hozzád kiáltok: hallgasd meg kérésemet.
/5
#C0CB73B2
 Hová legyek immár, hatalmas Istenem, ily fogyott életemmel?
 Mert lám, Uram, hozzád sokszor felkiálték keserves beszédemmel;
 Miért nyomoritál, én édes Istenem, most ez keserűséggel?
 Mit használok immár itt ez világ szerint bujdosó életemmel?
/6
#E696D355
 Ó miért te, Uram, ennyi sok ideig a te szent személyedet,
 Reám nem fordítád kegyelmessé Uram, a te kegyességedet?
 Én siralmim elől elfordítád, Uram, a te szent füleidet;
 De ha meghallgatnál, megvigasztalhatnád az én nagy bús szívemet.
/7
#2DD2F75F
 Ne útálj meg, Uram, én sok bűneimért, számtalan esetimért,
 Mert méltónak vallom magamat mindenkor az én undokságimért
 A megbüntetésre és felserkentésre az én kényességimért;
 De nem vetsz el, Uram, szent szemeid előI az én érdemem szerént.
/8
#F291FF1E
 Ezen kérlek téged, mennynek-földnek Ura, királyoknak Királya,
 Hogy míg e világban éltemet kívánod, azt csak Szentlelked bírja,
 Minden dolgaimat és én útaimat szüntelen igazgassa,
 Hogy te szent nevedet hozzám jóvoltodért én lelkem magasztalja.
/9
#E86DA361
 Ne bánkódjál lelkem, ne hagyd el magadat nagy keserűségedben, ó én édes szívem, mit keseregsz bennem ily igen életedben?
 Hagyd el a bánatot, végy nagy vígasságot a hatalmas Istenben:
 A nagy bánat helyett ő ád jó örömet a Szentlélek Istenben.
/10
#2240E09D
 Dicsőség tenéked, magasságban, Isten, királyoknak Királya!
 Dicséret mondassék néked, Krisztus Jézus, minden uraknak Ura!
 Velük egyetemben Szentlélek Úr Isten, mi lelkünknek világa: Teljes
 Szentháromság, egy bizony Istenség, lelkemnek boldogsága!

;Kolozsvár, 1744
>204
/1
#F6815A4B
 Boldogok, akik lelki szegények,
 Szívből könyörögnek,
 Mert a mennyország adatik nékiek.
/2
#413B92BF
 Boldogok, akik sírván bánkódnak,
 És károkat vallnak,
 Mert Istentől ők megvigasztaltatnak.
/3
#DD30E3BA
 Boldogok azok, akik szelídek
 És nem háborognak,
 Mert ők e földön örökséget bírnak.
/4
#59B00DA6
 Boldogok, akik az igazságot
 Éhezik, szomjazzák,
 Mert Istentől ők megelégíttetnek.
/5
#FB20A643
 Boldogok azok, kik irgalmasok,
 Máson könyörülnek,
 Irgalmasságot mert Istentől nyernek.
/6
#4290CD8A
 Boldogok, akik tiszta szívűek,
 Hitből megtisztultak,
 Mert megismervén az Istent, meglátják.
/7
#FA73F882
 Boldogok, akik békességszerzők,
 Mert ők hívattatnak
 Az igaz hitben Isten fiainak.
/8
#4E30F8F0
 Boldogok, akik az igazságért
 Üldözést szenvednek,
 Mert a mennyország adatik ezeknek.
/9
#0DC8AE41
 Boldogok lesztek, mikor
 Krisztusért Emberek gyűlölnek,
 Megszidalmaznak s gonoszul üldöznek.
/10
#B98138A4
 Akkor azért hát örvendezzetek és vigadozzatok,
 Mert mennyben lészen ti nagy jutalmatok.

;Kolozsvár, 1744
>205
/1
#4E81D3A1
 Ne szállj perbe énvelem,
 Ó, én édes Istenem!
 Mert meg nem igazul teelőtted lelkem,
 Elkárhoztathatsz engem.
/2
#00A80F9A
 Mert én anyám méhében
 Fogantattam vétekben,
 E világra lettem eredendő bűnben,
 Kárhozatos esetben.
/3
#08CFF7BE
 Néked bűnöm megvallom,
 Mert én azzal tartozom;
 Te bocsáthatod meg, Bizonnyal jól tudom:
 Segítségedet várom.
/4
#E7873DEC
 Bűneimnek tőréből,
 Az ördögnek kezéből,
 Ments ki megérdemlett nagy veszedelmemből:
 Tekints reám az égből!
/5
#4F9A58D5
 Szent Fiad haláláért,
 Keserves nagy kínjáért,
 Áldott szent vérének el-kifolyásáért:
 Kegyelmezz meg mindezért!
/6
#2EAAC440
 Az ő tisztaságáért
 És ártatlanságáért,
 Szent kezein való sebeknek helyéért,
 töviskoronájáért!
/7
#2786BE16
 Én fertelmességemet
 És oly nagy sok vétkemet,
 Mikkel megfertőztem undokul lelkemet,
 Rútítottam testemet:
/8
#3CED7F14
 Bocsásd meg, ó, Úr Isten!
 Ó, áldott Atya Isten!
 Kegyelemmel bőves, irgalmas jó Isten!
 Nagy türelmű szent Isten!
/9
#63C8A425
 A te áldott Szentlelked,
 Kérlek, tőlem el ne vedd,
 Sőt újítsd meg bennem, hogy dicsérjem neved,
 Szolgálhassak tenéked.
/10
#2E95E83A
 Reád bíztam magamat,
 Te viseljed gondomat,
 Igazgasd jó útra az én lábaimat
 És minden szándékimat.
/11
#742F00AD
 Vigasztald meg szívemet,
 Búban epedt lelkemet;
 Ne hányd szemeimre undok vétkeimet
 És cselekedetimet.
/12
#B9C4DEA0
 Csak tégedet dicsérlek,
 Míg e világon élek,
 Mert tudom, nyugalmat csak tenálad lelek,
 Mikor innét kikelek.
/13
#A2959092
 Mindörökké áldassál
 És felmagasztaltassál,
 Áldott Atya Isten, a te szent Fiaddal,
 Szentlélek áldásával!

;Kolozsvár, 1744
>206
/1
#694CFA37
 Én Istenem! Sok nagy bűnöm
 Lelkemet szorongatják;
 Itten nincsen,
 Ki segítsen,
 Hát kihez folyamodjam?
/2
#8E9F5DF4
 E világon
 Minden úton
 Akármerre indulok:
 Súlyos terhem, betegségem
 Ki elvegye, nem látok.
/3
#C2A7C3D7
 Hozzád térek,
 Végy be, kérlek,
 Jó Atyám, s haragodban
 Meg ne büntess,
 Légy kegyelmes
 Hozzám te szent Fiadban.
/4
#77E9D7E7
 Ha terhelnem,
 Kell szenvednem
 Az igaz ítélettől:
 Bár itt büntess,
 Csak ott kedvezz:
 Ne vesszek el örökűl.
/5
#DC0E0D07
 Ha kereszted
 Reám veted,
 Adj engedelmes szívet,
 Hogy tűréssel
 És reménnyel
 Várjam üdvösségemet!
/6
#4C1F58D6
 Nyújtsd kegyelmed
 És Szentlelked,
 Hogy engem vezéreljen;
 Bűn, ellenség
 S hitetlenség
 Kárhozatra ne vigyen!
/7
#D5CCAB0D
 Mint kis madár,
 Ha szélvész jár
 Az égi csattogásban,
 Rejtez fákban
 És odúkban,
 Hol megmaradhat bátran:
/8
#1CCF1172
 Így a bűnök,
 Halál s ördög
 Engemet ha rettentnek:
 Rejtekhelye Vagy, reménye,
 Ó, Krisztus, én lelkemnek.
/9
#0533D85F
 Sebeidben
 Rejts el engem,
 Hol bízvást megmaradok;
 Bár kínt valljak,
 Vagy meghaljak,
 Jól tudom: hozzád jutok.
/10
#B2762924
 Te énnékem
 Ellenségem
 Haláloddal meggyőzted,
 Boldog helyem,
 Idvességem
 A mennyben megszerezted.
/11
#5A6DB78E
 Áldott Isten
 Egy felségben,
 Tebenned szívem örül,
 Mert te szavad
 Áll s megmarad:
 Aki hiszen, idvezül.

;Magyar népi dallam
>207
/1
#A70CFC14
 Seregeknek hatalmas nagy királya,
 Könyörgésem székedet megtalálja,
 Mert szívemet sok bú állja,
 Olyan bűntől, mint megterhelt gálya.
/2
#DB9B08A9
 Ím, előtted megaláztam magamat,
 Földre hajtván szomorodott orcámat;
 Halld meg, Atyám, csendes szómat,
 Add meg szívből kívánt lelki jómat.
/3
#288BB203
 Ne vess el hát az én sok bűneimért,
 Ne is büntess háládatlanságomért,
 De sőt inkább szent Fiadért,
 Kegyelmezz meg érdemes kínjáért.
/4
#DB43C109
 Kihez hajtsam búba merült fejemet?
 Ki tölti bé megsebhedett szívemet?
 Ha te is elhagysz engemet,
 Elveszek, ha nem szánod lelkemet.
/5
#D2556424
 Az én lelkem reszket előtted állván,
 Mint a gyenge levél nyárfának ágán,
 Ellened tett bűnöm látván,
 Gyászban járván sír s kegyelmet kíván.
/6
#EC96932F
 Gyakran szívből folyamodom elődbe,
 Könyörgésem hasson fel az egekbe;
 Uram, hadd jusson elődbe,
 Vétkeimet temesd a tengerbe.
/7
#E498D6D6
 Örök Isten, felette irgalmas vagy,
 Megtérőkhöz kegyelmed is igen nagy;
 Engem, talpig bűnöst ne hagyj,
 Vigasztalást inkább szívemnek adj.
/8
#5FD0948F
 Engem bűnöst hiszen földből formáltál,
 Töredelmes cserép helyre állattál,
 Sokszor sok bűnért sujtoltál,
 Addig vertél, míg hozzád hajtottál.
/9
#37E64A74
 Reám vigyázz a szép felvont egekből,
 Dicsőséges királyi szent székedből;
 Halld meg, mert könyörgök szívből,
 Mosogass meg undok vétkeimből.
/10
#8E33DE1A
 Gyarló szívem magasztalja nevedet,
 Hogy pokolra nem eresztéd lelkemet;
 Megbocsátád vétkeimet,
 Melyért áldlak, édes Istenemet.

;Bourgeois L., Genf, 1542
>208
/1
#43CDDAF4
 Uram, bűneink soksága,
 Undoksága Érdemli haragodat,
 Méltók vagyunk, hogy ellenünk,
 Szent Istenünk,
 Felemeld ostorodat.
/2
#8D50728F
 De tudjuk, hogy ki megvallja
 És megbánja Bűneit s hozzád megtér,
 Azt nem hajtod el előled,
 Sőt tetőled
 Bűnbocsánatot az nyér.
/3
#5756D43C
 Azért hát mi is járulunk
 És borulunk Elődbe fájdalmasan,
 Bűnös lelkünknek kegyelmet, engedelmet
 Kérvén alázatosan.
/4
#49EC3454
 Szánj meg, Uram, ily ügyünkben
 S megtértünkben
 Függeszd fel ostorodat.
 Ó, hajtson hozzánk békére Fiad vére,
 S felejtsd el haragodat.
/5
#018CAD15
 Szólj hozzánk, Uram, csendesen
 És édesen;
 Félelmünk mindjárt széled,
 És lelkünk is e szózattól,
 Mint harmattól
 A hervadt virág, éled.
/6
#F68D6C79
 Uram, végy minket kedvedbe,
 Szerelmedbe,
 S vigasztald meg szívünket.
 Tőlünk soha ne maradj el,
 Se ne hagyj el,
 Hanem fogjad kezünket.

>209
/1
#1CE280A3
 Tökéletes volt minden tekintetben,
 Mind amit, Uram, teremtél kezdetben.
 Az ember testét te alkotád szépen,
 És lelkét épen.
/2
#E51B335F
 Tenmagad képét adtad ő reája,
 Szent ártatlanság volt ékes ruhája,
 Bölccsé is tetted, hogy téged ismerne
 S élvén dícsérne.
/3
#EF115A2F
 Első szüleink de félre térének,
 A kísértőnek mikor engedének,
 Már mi mindnyájan, kik tőlük eredtünk,
 Bűnben születtünk.
/4
#36C47389
 Tagjaink mind, és bennünk minden részek
 Restek a jóra, a gonoszra készek,
 Nincs erőnk s a jót, noha jónak hisszük,
 Véghez nem visszük.
/5
#938409C5
 Fertőbe estünk, Uram végy ki onnan,
 Szülvén minket Szentlelkeddel újonnan;
 Újítsd meg rajtunk, hogy legyünk te néped,
 Isteni képed.

;Debrecen, 1781
>210
/1
#0281BFC2
 Ó, áldandó Szentháromság!
 Nyögésemre figyelmezz.
 Jövel, siess, Főboldogság,
 Én Istenem, kegyelmezz!
 Ó, Uram, Uram, ne hagyj!
 Mert reménységem te vagy.
/2
#ACA073FE
 Ím, körülvett már az ínség,
 Minden bűnöm rám tódult,
 Szorongat a végső szükség,
 Fetrengek, mint egy bódult.
 Ó, szerelmes szent Atyám!
 Légy nékem erős bástyám.
/3
#9BFC4238
 Ölelgess szent szerelmeddel,
 Örök szövetségedből,
 Takargass be védelmeddel,
 Adj részt örökségedből:
 Nézz régi irgalmadra
 S egyetlenegy Fiadra.
/4
#C1E7B698
 Ó, egyetlenegy segítőm,
 Uram Jézus, irgalmazz!
 Légy vélem, én Idvezítőm,
 A gonosztól oltalmazz!
 Hiszen te értem jöttél,
 S értem is vért öntöttél.
/5
#54946DB1
 Gyógyítsd lelkem betegségét
 Véres verejtékeddel,
 A halálnak édességét
 Újítsd drága véreddel,
 Mert csak az lelkem bére,
 Más érdem álljon félre.
/6
#184AED38
 Ó, Isten Báránya, kedvezz,
 Mert most vagyon az óra;
 Mondd lelkemnek: Légy idvez,
 Dolga forduljon jóra.
 Ó, Úr Jézus, könyörülj,
 Könyvedből ki ne törülj!
/7
#8BCAFDD9
 Ó, vigasztaló Szentlélek,
 Jövel segítségemre,
 És míg itt mozgok és élek,
 Készíts idvességemre.
 Tartsd meg bennem a hitet,
 Mit kegyelmed készített.
/8
#F4628A8E
 Bátoríts a halál ellen,
 És ha már elaluszom,
 Hogy a Sátán el ne nyeljen,
 Te légy, Jézus, végső szóm.
 Maradj velem mindvégig:
 Kísértess bé az égig.
/9
#2AD52147
 Ó, Úr Jézus, légy Jézusom:
 Légy irgalmas lelkemnek!
 Ó, vedd hozzád egy orvosom,
 Míg megadod testemnek,
 Ó, Jézus, tőlem ne fuss!
 Ó, jövel, Uram Jézus!

>211
/1
#BF7DD7D5
 Meghagytad nékünk, Úr Isten,
 Hogy járjunk törvényedben
 És téged tiszteljünk
 Teljes szivvel, lélekkel,
 És minden mi erőnkkel
 Tégedet szeressünk.
/2
#BCDC17FA
 A mi felebarátinkat
 Szeressük, mint magunkat,
 Haragot elhagyván;
 Senkinek kárt ne tégyünk,
 Mindenhez hivek légyünk
 és éljünk igazán.
/3
#7A10D11C
 De jaj, mily gyarló életem,
 Mert Ádámtól születtem:
 Bűnre vagyok hajló.
 Elmémnek gondolatja,
 Szivemnek indulatja
 Te ellened járó.
/4
#0C980C12
 Szent igédet nem követem,
 Sőt gyakran elfelejtem,
 Jókra tunya szivem;
 Feleimet gyűlöli,
 Vagy csak szinnel szereti:
 Mily nehéz énnékem.
/5
#87BEFB3A
 Méltán ezért haragodat,
 Érdemlem ostorodat,
 De jó reménységgel
 Nézek fel szent Fiadra,
 Mint kegyes szószólómra,
 Kiért hozzám tértél.
/6
#34FBF425
 Mert én csak őbenne bizom,
 Nem hányom méltóságom,
 Mely semmire kellő.
 Ő engem megigazít,
 Gonosztól megszabadit,
 Oly igen jó kedvü.

;Kolozsvár, 1744
>212
/1
#1C67FC2A
 Én nem perlek,
 És nem merlek,
 Igaz bírám, vádolni,
 Ha elkezded
 Ítéleted
 Énrajtam gyakorolni.
/2
#EC20EB57
 Enyhén bántál,
 Nem kívántál
 Semmit sem erőm felett;
 Az izgató
 Csalogató
 Mégis engem bűnre vett.
/3
#7F64D9A3
 Megvettetést,
 Nagy büntetést
 Ezért előre látok;
 Szörnyű dolog
 Lesz, ha megfog
 A törvényben írt átok.
/4
#693A85C1
 Sokszor hallom:
 Az irgalom
 Nálad éri az eget;
 De ismétlen
 Véghetetlen
 Igazságod fenyeget.
/5
#1EE5FAEB
 E kettő közt
 Olyan eszközt
 Találnom lehetetlen,
 Hogy jó maradj
 És mégse hagyj
 Egy bűnt sem büntetetlen.
/6
#CD880A45
 De midőn én
 Kételkedvén
 E két mélység közt állok:
 Útmutatást,
 Megtartatást
 Szent igédben találok.

;Kolozsvár, 1744
>213
/1
#39794FC7
 Mindenható Úr Isten,
 Mi, bűnös emberek,
 Gyónást és vallást teszünk
 Mint töredelmesek.
/2
#37324010
 Mert mi igen vétkeztünk
 Istenséged ellen,
 Mint teremtő, megváltó
 És szent Atyánk ellen.
/3
#E421DD94
 Életünknek rendiben
 Igen megbántottunk,
 Mi nagy sok bűneinkkel
 Téged bosszantottunk.
/4
#6E705B7E
 Gonosz szóval, szándékkal,
 Látással, hallással,
 Irigységgel, mordsággal,
 Rágalmazásokkal.
/5
#E501760F
 Szitkos, átkos voltunkkal,
 Haragtartásunkkal,
 Hamisan mi hitünket
 Gyakran mondásunkkal;
/6
#D62FC3B4
 Isteni káromlással
 És hitetlenséggel,
 Mi nyomorúságinkban
 Békételenséggel.
/7
#ABDE66C6
 Istennek szent igéjét
 Noha gyakran halljuk,
 De mibennünk majd semmi
 Gyümölcsét nem látjuk.
/8
#DC3F49EF
 Az isteni és atya-
 Fiúi szeretet
 Nincsen bennünk, de vagyon
 Éktelen, rút élet.
/9
#4D26280C
 Bujaság és torkosság,
 Megfojt a kevélység;
 Telhetetlen s átkozott,
 Izgat a fösvénység.
/10
#971E1102
 Mint tengernek fövénye
 Megszámlálhatatlan:
 Azonképpen mi bűnünk
 Nálunk tudhatatlan.
/11
#B110EB4D
 Azért mi is magunkat
 Istenséged előtt,
 Bűnösöknek vádoljuk
 Mind e világ előtt.
/12
#9CD1FC58
 Ne nézd mi bűneinket
 És gonoszságinkat,
 De tekintsd kegyelmesen
 Irgalmasságodat.
/13
#529DB347
 És el ne feledkezzél
 Te ígéretedről,
 A bűnösökhöz való
 Kegyelmességedről.
/14
#98C90F77
 Azért néked könyörgünk,
 Felséges Úr Isten:
 Ne állj bosszút szertelen
 Mi gonosz bűnünkön.
/15
#55D6E231
 De minékünk megbocsáss
 Te szent Fiad által,
 Hogy mi idvezülhessünk
 Ő irgalma által.
/16
#F60F0B3F
 Hisszük, hogy meghallgattál
 Mi könyörgésünkben,
 Azért lelki örömmel
 Mi ezt mondjuk: Ámen.

>214
/1
#363874BC
 Sok nyavalyánkban Atyánkhoz kiáltunk,
 Mert segedelmet sehol sem találunk,
 Csak ő minékünk Atyánk és gyámolunk,
 Minden oltalmunk.
/2
#0C1CE3A3
 Tégedet azért alázatos szivvel
 Kérünk: oltalmazz, Atyánk, jobbkezeddel,
 És táplálj minket a te szent Igéddel:
 Lelki étellel.
/3
#6A8CC915
 Itt evilágon miglen mi nyomorgunk,
 Csak segedelmed alá folyamodtunk,
 Mert igen kegyes Atyánk vagy minékünk:
 Minden életünk.
/4
#7B48487C
 Pokol ajtaja bünért reánk nyíla,
 Hogy az Istennek haragja indúla,
 Minden nyavalya ottan reánk szálla:
 Bűnnek jutalma.
/5
#C03BB91F
 Ha bűneinket számlálod, Úr Isten,
 Senki közülünk nem mehet elődben,
 Mert mi fetrengünk számtalan sok bűnben,
 Undok életben.
/6
#3668F88D
 Azért könyörgünk: ne nézd bűneinket,
 De tekints kegyes szemeiddel minket,
 A te Fiadért szeress ingyen minket,
 Szabadits minketl
/7
#6561ACC0
 Ne tégy mi vélünk a mi bűnünk szerint,
 De tekints minket kegyességed szerint,
 A Krisztus Jézus kedves volta szerint,
 Érdeme szerint.
/8
#C7E1314E
 Végy el mi rólunk ennyi sok ínséget,
 Adj nékünk, kérünk, immár csendességet;
 Ne fogyatkozzunk el sok ínségink közt:
 Adj nékünk erőtl
/9
#2F5E982F
 Sok háborgásink lecsendesedjenek,
 Hogy hatalmasok tégedet féljenek,
 Mindenkoron téged ismerjenek
 Örök Istennek.
/10
#296D1F03
 Dícséret légyen az Atyának mennyben,
 Ő szent Fiának örök dicsőségben,
 Szentléleknek velük egyetemben,
 Örökké, Ámen.

;Pozsony, XVIII. század végén
>215
/1
#52C1DC89
 Eltévedtem, mint juh,
 Eltévedtem, mint juh,
 A bűnösök útjára,
 Ó, segíts, Jézusom,
 Őriző pásztorom,
 Hogy ne jussak romlásra.
 Te ontál drága vért
 Elveszett juhokért,
 Viselj gondot a nyájra.
/2
#DE1F6C27
 Én is juhod vagyok,
 Én is juhod vagyok,
 Nyájadnak legkisebbje,
 Kit te megtéríthetsz,
 Bűnből kivezethetsz
 A szép kies helyekre.
 Kérlek azért hitből,
 Töredelmes szívből,
 Fogadj, végy kegyelmedbe.
/3
#7FEF7C2F
 Ím, előtted állok,
 ím, előtted állok,
 Ajtód előtt zörgetek,
 Bár titkos bűnökkel
 És nyilván lévőkkel
 Vétkeztem teellened:
 Kérlek mindazáltal,
 Nagy irgalmassággal
 Fogadd vissza gyermeked.
/4
#981672F5
 Bár a hit szívemben,
 Bár a hit szívemben
 Oly kicsiny, mint mustármag,
 Mégis bármi gyenge,
 Szentlelked nevelje,
 Nevelje fel nagy fának,
 Hogy terjedjen ága
 És legyen virága
 Kedves néked, Urának.
/5
#02E61E9C
 Míg porsátoromban,
 Míg porsátoromban
 Tartasz, mint egy tömlöcben,
 Nevelj igaz hitben,
 Munkás szeretetben,
 Hogy élhessek itt bölcsen
 És kimúlásomig,
 Utolsó órámig
 Tisztemet hűn betöltsem.
/6
#17A7B005
 Szállj le, Uram, hozzám,
 Szállj le, Uram, hozzám,
 Jöjj, ó, Jézus, sietve,
 Vágyakozó szívvel,
 Kiterjesztett kézzel
 Várlak immár epedve,
 Hogy veled mennyekbe,
 Örömmel menjek be
 Ábrahám kebelébe.

;Kuruc dallam
>216
/1
#814A2848
 Végtelen irgalmú Isten, hozzád kiáltok.
 E nagy mélységekből szabadulást kívánok;
 Értsd meg kérésimet,
 Érjék füleidet ennyi sok zokogások.
/2
#C4819F6E
 Igazságod szerint ha cselekedel vélünk,
 Ha mind számon tartod, amit ellened vétünk:
 Senki nem állhatja Súlyos kezed miatt,
 Tőled ha büntettetünk.
/3
#A73BE682
 Fölötte bőséges nálad a kegyelmesség
 Atyai kegyesség, szelíd engedelmesség,
 Néked hát szentektől
 S földi emberektől
 Adassék nagy tisztesség.
/4
#22AD22C8
 Az Urat óhajtom, várja lelkem az Urat
 Szívem szomjúhozván, hogy mutassa jóvoltát;
 Igéjéből értem,
 Melyben reménységem:
 Nem vonja meg irgalmát.
/5
#2D04582B
 Lelkem az Úr Istent buzgósággal óhajtja.
 Mint éjjeli strázsa napköltét alig várja,
 Reggelről-reggelre, Estéről-estére
 Szívem Istent úgy várja.
/6
#10171B69
 Valakik az Úrnak választott szenti vagytok,
 Egyedül csak benne légyen bizodalmatok,
 Mert ő nála váltság, Vagyon irgalmasság,
 Ő lesz néktek Atyátok.
/7
#AEB5DEC4
 Izráelt megváltja, és a keresztyénséget,
 Hívek seregének megbocsátja vétküket.
 Minden nyavalyából, Az örök halálból
 Kimenti ő lelküket.

;Wittenberg, 1524
>217
/1
#DC78C9D3
 Bűnösök, hozzád kiáltunk:
 Úr Isten, könyörülj rajtunk!
 Nyisd meg a te füleidet,
 Hallgasd meg könyörgésünket!
 Mert ha te mind megítéled,
 Amit vétettünk ellened,
 Mind elkárhozunk előtted.
/2
#43CF60E9
 Nagy a te irgalmasságod,
 A bűnt Te megbocsáthatod;
 Nincs nékünk semmi érdemünk,
 Már semmi ártatlanságunk.
 Nincs ki kérkedjék előtted,
 Félünk mindnyájan Tégedet,
 És könyörgünk mi Tenéked.
/3
#074DAC41
 Bízzunk azért az Istenben,
 És nem a mi érdemünkben,
 Nyugodjék Őbenne lelkünk,
 Ő légyen mi reménységünk;
 Lám, nékünk nyilván ígéré,
 Hogy akar oltalmunk lenni,
 Azért higgyünk csak Őnéki.
/4
#718D076E
 Erősek legyünk hitünkben,
 Bízzunk csak az Úr Istenben;
 Ne essünk Benne kétségben,
 Ne szomorkodjunk lelkünkben.
 Minden keresztyén hív légyen,
 Ki megújult Szentlélekben
 És bízik csak az Istenben.
/5
#15B017A9
 Ha minékünk sok bűnünk van,
 Istennek több kegyelme van,
 Ő irgalmának nincs vége,
 bár sok az emberek vétke.
 Ő nékünk kegyes pásztorunk,
 Ki őriz minket pokoltól
 És megment a kárhozattól.

>218
/1
#DE79EB88
 Életünknek rendiben körülvesz a halál;
 Segedelmet, kegyelmet lelkünk már hol talál?
 Csak nálad Urunk, Isten!
 Mert bánjuk bűnünk, mely miatt
 Haragvásod felgyúladt.
 Szent, igaz Úr Isten,
 Szent erős Úr Isten,
 Szent irgalmas megváltó Urunk, te nagy örök Úr:
 A keserű halálban ne hagyj immár elvesznünk!
 Irgalmazz nekünk!
/2
#7FA20F00
 Pokol dühe támad ránk halálnak képében,
 Szabadítást ki nyujthat ilyen nagy ínségben?
 Csak te, mi Urunk, Isten.
 Felindítja nagy irgalmad ennyi bűn és ily bánat.
 Szent, igaz Úr Isten,
 Szent erős Úr Isten,
 Szent irgalmas megváltó Urunk, te nagy örök Úr:
 A mély pokol lángjától ne hagyj kétségbe esnünk!
 Irgalmazz nekünk!
/3
#8F3C26E0
 Pokol rettentésébe kerget minket bűnünk;
 Kihez máshoz mehetnénk, hol megmenekülünk?
 Csak hozzád Urunk, Krisztus!
 Kiontád drága véredet bűnért tévén eleget.
 Szent, igaz Úr Isten,
 Szent erős Úr Isten,
 Szent irgalmas megváltó Urunk, te nagy örök Úr:
 Vigasztaló hitünktől ne hagyj minket el esnünk!
 Irgalmazz nekünk!

>219
/1
#9A8E07C0
 Atya Úr Isten, könyörülj rajtam te nagy jóvoltodért,
 Sok bűneimet bocsásd meg nékem irgalmasságodért;
 Ne hagyj, Úr Isten, kérlek, elvesznem, a te szent nevedért.
/2
#F7954DF5
 Imé, Úr Isten, mind talpig vagyok a bűnben előtted,
 Mint merő vérben befertőztetett, úgy állok előtted:
 Mint a nyár-levél, úgy reszketek én féltemben tetőled.
/3
#32B1A997
 Anyám mikoron az ő méhében engemet fogada,
 Nagy siralommal és fájdalommal e világra hoza:
 Bűnben fogada, róla reám is a bűn elárada.
/4
#113B5C3E
 Nincs hová lennem, nincs hová fognom elveszett fejemet,
 Féli, rettegi elbúsult lelkem igaz Istenemet:
 Vajjon mikoron mossa el rólam az én bűneimet?
/5
#9C2256CE
 Ím te elődbe támasztom, Uram, az én bűneimet:
 Lássad te magad, irgalmas Atyám, halálos sebemet;
 Kössed be nékem orvosságoddal keserves szívemet l
/6
#B9F958CB
 Mely igen tiszta lennék, Úr Isten, ha megmosogatnál,
 Bizony, a napnál, holdnál, csillagnál: szebb lennék a hónál;
 Senki fehérebb nálam nem lenne, ha megtisztítanál.
/7
#DA239126
 Hallgasd meg azért fohászkodását a te híveid­nek,
 Kik most énvelem sírván-óhajtván, együtt könyörögnek:
 Hatalmat venni ne hagyj mi rajtunk a mi bűneinknek.

;Kájoni-kódex, XVII. század
>220
/1
#136FB22E
 Bocsásd meg, Úr Isten, ifjúságomnak vétkét,
 Sok hitetlenségét, undok fertelmességét,
 Töröld el rútságát, minden álnokságát,
 Könynyebbítsd lelkem terhét.
/2
#7A947C00
 Az én búsult lelkem én nyavalyás testemben
 Tétova bujdosik, mint madár a szélvészben;
 Tőled elijedett, Tudván, hogy vétkezett,
 Akar esni kétségben.
/3
#AC065275
 Akarna gyakorta hozzád ismét megtérni,
 De bűnei miatt nem mer elődbe menni,
 Tőled oly igen fél. Reád nézni sem mér,
 Színed igen rettegi.
/4
#D159A23A
 Semmije sincs pedig, mivel elődbe menjen,
 Mivel jóvoltodért viszont téged tiszteljen,
 Nagy alázatosan, Méltó haragodban
 Tégedet engeszteljen.
/5
#8033049D
 Bátorítsad, Uram, azért biztató szóddal!
 Mit használsz szegénynek örök kárhozatjával?
 Inkább hadd dicsérjen
 E földön éltében
 Szép magasztalásokkal.
/6
#D86F52A5
 Térj azért, én lelkem, kegyelmes Istenedhez,
 Szép könyörgésekkel béküljél szent kezéhez,
 Mert lám, hozzá fogad, Csak reá hagyd magad:
 Igen irgalmas Úr ez.
/7
#86706ABE
 Higgyünk mindörökké egyedül csak őbenne,
 Őrizkedjünk bűntől, ne távozzunk őtőle.
 Áldott az ő neve Örökké mennyekbe',
 Ki már megkegyelmeze.

>221
/1
#9F3059EE
 Légy irgalmas, Úr Isten, minékünk,
 Mert ellened álnokúI vétkeztünk,
 Mégis benned vagyon reménységünk,
 Bár bűnünkkel téged ingerlettünk.
/2
#083E2361
 Vedd el rólunk méltó haragvásod,
 És mutassad nagy irgalmasságod,
 Hogy ismerjük meg ebből jóvoltod:
 Vedd el rólunk ezenképen átkod.
/3
#176A8B67
 Csak bűn a mi gyarló természetünk,
 Álnokságban, vétekben születtünk;
 Veszedelmes a mi földi éltünk,
 Elkárhozunk, hogyha meg nem térünk.
/4
#94A2B2CC
 Voltaképen ellened vétkeztünk,
 Látásodra gonoszt cselekedtünk,
 Igaz vagy te, nem kell kételkednünk,
 Győzedelmes, ha megítéltetünk.
/5
#041AA761
 Derékképen tisztitsál meg minket,
 Mosd el rólunk Fiadért bűnünket,
 Elvesztett szép öltözetünket
 Ránk feladván, fogadd el lelkünket.
/6
#B9B5657D
 Siess belénk új szíveket adni,
 Szentlelkeddel lelkünk megújítni,
 Tiszta hittel megajándékozni:
 Add örökké szent színedet látni.

>222
/1
#A143A21F
 Ó mennyeknek fényessége,
 És szenteknek dicsősége,
 Bűnösöknek reménysége,
 Egyetlen egy segítsége!
/2
#736DF071
 Te előtted vallást teszünk,
 Hogy mi mindnyájan vétkeztünk,
 Méltó haragodba estünk,
 És örök halált érdemlünk.
/3
#957907EA
 Mert ímé már a mi bűnünk
 Hatalmasan regnál bennünk,
 Melyért a mi ellenségünk
 Követeli jussát bennünk.
/4
#26FBCA2C
 Nincs szívünkben reménységünk,
 Nincs lelkünkben csendességünk;
 Magunkban megfogyatkoztunk,
 Halál miatt nyomorodtunk.
/5
#24880789
 Már könyörülj meg mirajtunk,
 Úr Isten, mert magunk vagyunk;
 Szent Fiadért esedezünk:
 Szánj meg és ne hagyj elvesznünk!
/6
#EBFE8E4C
 Vessed reánk szemeidet,
 Vedd el rólunk bűneinket,
 Igazíts meg bűnből minket,
 Vigasztald meg szíveinket.
/7
#C141FAF3
 Dicsőítünk, szent Atyánkat,
 Véled  együtt Megváltónkat,
 A mi Vigasztalónkat:
 A dicső Szentháromságot.

;Neumark György, 1657
>223
/1
#5C29C294
 Istenem, én nagy bűnös ember,
 Szent színed elé járulok,
 Vétkem oly mély már, mint a tenger,
 Mentségért hozzád fordulok.
 Én Istenem, én Istenem,
 Irgalmazz, kérlek, énnekem!
/2
#BCBB247F
 Szívem szerint ím elkesergem
 Gonosz és sok bűneimet:
 Hogy tőled én eltévelyedtem,
 Elhagytalak, Teremtőmet.
 Én Istenem, én Istenem,
 Irgalmazz, kérlek, énnekem!
/3
#7B4E8AA3
 Hallgasd meg én fohászkodásim
 Atyai nagy szerelmedből,
 Bocsásd meg minden rút bűneim,
 Mentsd ki szívem ez ínségből.
 Én Istenem, én Istenem,
 Irgalmazz, kérlek, énnekem!
/4
#D94D2E12
 Ne büntess úgy, mint érdemlettem
 Tőled én undok bűnömmel,
 Mert akkor nyilván el kell vesznem:
 Térj hozzám hát jó kedveddel.
 Én Istenem, én Istenem,
 Irgalmazz, kérlek, énnekem!
/5
#9E221CD8
 Csak egy szót mondj, hogy újjá légyek,
 Mondd ezt énnékem, bűnösnek:
 Megengedtem, menj el békével,
 Meglásd, többé ne vétkezzél.
 Én Istenem, én Istenem,
 Irgalmazz, kérlek, énnekem!
/6
#4650218C
 Nincs kétségem, megvigasztaltál,
 Erősítéd én szívemet,
 Könyörgésemben meghallgattál,
 Érzem már szent kegyelmedet.
 Én Istenem, én Istenem,
 Irgalmazz, kérlek, énnekem!

>224
/1
#838C7399
 Kegyelmezz meg nékünk, nagy Úr Isten,
 Ne hagyj minket elvesznünk bűnünkben,
 Kik mindenkor bízunk szent nevedben:
 Bűnünk szerint ne büntess tömlöcben!
/2
#7F748D07
 A te néped tenger vétkét vedd el,
 És fedezgesd bűnét kegyelmeddel,
 Mert szent Fiad megválta vérével,
 Nem akart ő elvetni bűnünkkel.
/3
#B000CF5E
 Idvezitő Isten, kérünk téged:
 Jelenjék meg rajtunk kegyességed,
 Távozzék el rólunk ítéleted,
 Légyen vélünk te szent istenséged.
/4
#AB2D3409
 Vajjon egyéb náladnál ki volna,
 Hogy fogság­ból kiszabaditana?
 Hanem csak te, Istennek szent Fia,
 Ki magadat adád nagy fogságra.
/5
#01C07998
 Mutasd ki már irgalmasságodat,
 Kegyes Isten, és nagy jóvoltodat:
 Add ismernünk a te szent Fiadat,
 Ő kedvéért jelentsd meg magadat I
/6
#A1B67BB3
 Hallgathassuk a te szent igédet,
 És hihessük te ígéretedet:
 Adjad nékünk a jó békességet,
 Szabadítsd ki a fogságból néped!
/7
#DD90823B
 Közel vagyon hívekhez az Isten,
 Mert nem múlóképen jő közénkben,
 De örökké lakozik szívünkben:
 Nem hágy vesznünk a hitetlenségben.
/8
#C4BECB06
 Higyjünk azért a Jézus Krisztusban,
 Ne maradjunk bálványimádásban:
 Nem hágy minket az Isten fogságban,
 És nem adja népét nagy rabságban.

;Boroszlói kézirat, XVI. század
>225
/1
#15E523E6
 Nagy hálát adjunk az Atya Istennek,
 Mennynek és földnek szent teremtőjének,
 Oltalmazónknak, kegyes éltetőnknek,
 Gondviselőnknek.
/2
#24EC1F89
 Hála tenéked, mennybéli nagy Isten,
 Hogy szent igédet adtad mi elménkben,
 És hogy ezáltal véssz ismeretedben,
 Te kegyelmedben.
/3
#CB3DF04A
 A romlás után nem hagyál bűnünkben,
 Sőt te Fiadat ígéréd igédben,
 Hogy elbocsátod őtet miközénkben,
 Emberi testben.
/4
#3776CBA8
 Felséges Isten, tenéked könyörgünk,
 Hogy mutassad meg szent Fiadat nékünk,
 Hogy őtet látván, benne remélhessünk,
 Idvezülhessünk.
/5
#277D4AA0
 Adjad, hogy lássuk a világosságot,
 Te szent igédet, az egy igazságot,
 A Krisztus Jézust: örök vigasságot,
 És boldogságot.
/6
#28C61E9C
 És ne ismerjünk többet a Krisztusnál,
 Ne szerethessünk egyebet Jézusnál;
 Maradhassunk meg a te szent Fiaddal,
 Krisztus Urunkkal.
/7
#B6ECAA42
 Adj igaz hitet a te szent Fiadban,
 És jó életet minden útainkban;
 És Szentlelkeddel vigy be hajlékodba,
 A boldogságba.
/8
#BEC1DC8F
 Dicsőség néked mennyben, Örök Isten,
 Ki dicsértetel a te szent igédben,
 Krisztus Jézusban, mi Idvezítőnkben,
 És Szentlélekben.

;Kuruc dallam
>226
/1
#3DFAA2B5
 Krisztusom, kívüled nincs kihez járulnom,
 Ily beteg voltomban nincs kitől gyógyulnom.
 Nincs ily fekélyemből ki által tisztulnom,
 Veszélyes vermemből és felszabadulnom.
/2
#EE63DDE4
 Gyújtsd meg szövétnekét áldott szent igédnek
 És bennem virraszd fel napját kegyelmednek;
 Igaz utat mutass nékem, szegényednek,
 Járhassak kedvére te szent Felségednek.
/3
#82F08524
 Várlak, Uram, azért reménykedő szívvel,
 Miként a vigyázó virradást vár éjjel;
 Hozd fel szép napodat nékem is jó reggel,
 Hogy szolgálhassalak serényebb elmével.
/4
#B7BB15FC
 Dicsértessél Atya Isten, magasságban,
 Mi Urunk Krisztussal mind egy méltóságban,
 És a Szentlélekkel mind egy hatalomban:
 Háromság egy Isten, áldj meg dolgainkban.

;Frankfurt, 1662
>227
/1
#6EC5A48B
 A mi szívünk csak tehozzád, Jézus,
 Isten Báránya, Óhajtozik, híveidnek
 Drága fényes aranya,
 Mert halálunk megrontója,
 Örök életnek adója Vagy egyedül,
 Krisztusunk, Idvezítő Jézusunk.
/2
#4F4A11AA
 Csudálkozván nézi elménk
 Atyádnak nagy szerelmét,
 Álmélkodván is szemléli
 Hozzánk való jó kedvét,
 Melyet abban megmutatott,
 Hogy minékünk téged adott;
 Azért, édes Krisztusunk,
 Vagy szerelmes Jézusunk.
/3
#8DCD5786
 Szegény bűnös embereknek
 Idvességes reménye,
 Benned hívőknek öröme,
 Boldogsága, szépsége:
 Jövel, jövel hozzánk, kérünk,
 Mindenkoron légy mivélünk,
 Mert csak te vagy Krisztusunk
 És kegyelmes Jézusunk.
/4
#9D7BB75D
 Reád bízzuk mi magunkat
 Itt ez árnyék világban,
 Idvezítő szerelmedből
 Könyörülj fiaidon.
 Légy vezérünk, oltalmazónk,
 Kegyes mesterünk, tanítónk,
 Édes Atyánk, Krisztusunk,
 Gondviselő Jézusunk.
/5
#7F4209D4
 Erősítsd bennünk hitünket,
 Idvességünk eszközét,
 Vesd tengerbe bűneinket,
 Kerüljük pokol tüzét.
 Így mi halálunk óráján,
 Életünknek végső táján,
 Lelkünket, ó, Krisztusunk,
 Vedd kezedbe Jézusunk.
/6
#40D1A4DB
 Nincsen nékünk itt e földön
 Maradandó városunk,
 Hanem repeső elmével
 Jövendőt óhajtozunk,
 Ott sok ezer angyalokkal,
 Téged áldunk szent Atyáddal:
 Méltó vagy, ó, Krisztusunk,
 Erre, édes Jézusunk.
/7
#E5231261
 Hozzád hajtjuk csak fejünket,
 Ó, mi édes Megváltónk,
 Koporsónkba ha beszállunk,
 Ott is léssz gyámolítónk.
 Benned édesen aluszunk,
 Eljöveteledig nyugszunk,
 És örökké, Krisztusunk,
 Véled élünk, Jézusunk.
/8
#84E3F8B6
 Zörgetőknek megnyittatik Kegyelemnek ajtaja;
 Megígérte az Úr Jézus,
 Hogy kérésünk megadja.
 Jövel, Jézus, és ne késsél,
 Már mellőlünk el ne térjél:
 Idvezíts, ó, Krisztusunk,
 Mi kegyelmes Jézusunk!

>228
/1
#3E756EBA
 Jehova, csak néked éneklek, ki volna más tehozzád fogható?
 Dícséretet csak néked zengek, csak téged áldlak, ó Mindenható!
 Jővel, segíts, én édes Jézusom, hogy könyörgésem mennybe feljusson!
/2
#C4E1F5D5
 Ó vonj, Atyám, Fiadhoz engem,
 Hogy szent Fiad is hozzád elvigyen;
 S míg jóvoltodat áldva zengem,
 Keblem Szentlelked hajléka legyen;
 Add békességed ízét érzeni és dícséreted vigan zengeni!
/3
#80F1A15B
 Ha semmi mást nem kérek tőled,
 Csak amit Lelked által kérhetek,
 Te szükségim beteljesited,
 Mert Jézus hozzád útat készitett,
 Ki gyötrelmével értem áldozott,
 És üdvösséget jussomúl hozott.
/4
#629114DD
 Jó tudnom azt, hogy közbenjáróm
 Hű Jézusom, ki jobbod felől ül;
 Ő általa, ha hitből várom,
 Kérésem nálad bizton teljesül
 Jó nékem, hogy mig tart ez életem,
 Szent neved vigan, áldva zenghetem!

;Régi angol dallam
>229
/1
#6586E267
 Hű pásztorunk, vezesd a te árva nyájadat,
 E földi útvesztőben te mutass jó utat;
 Szent nyomdokodba lépve, a menny felé megyünk,
 Ó, halhatatlan Ige, vezérünk, Mesterünk.
/2
#F59130CD
 Mert boldog az az ember, ki dicsér tégedet,
 És kóstolgatja mindennap szent beszédedet;
 Hát legeltessed igéddel bolygó nyájadat,
 És terelgessed Lelkeddel juhocskáidat.
/3
#513D6388
 Szentlelkedet töltsd ránk ki mint hajnal harmatát,
 És adj fejünkre tőled nyert ékes koronát,
 Hogy áldozatra felgyúlt, megszentelt életünk
 Oltárodon elégjen, Királyunk, Mesterünk!

;Angol dallam
>230
/1
#3C8D1DB3
 Áll a Krisztus szent keresztje
 Elmúlás és rom felett,
 Krisztusban beteljesedve
 Látom üdvösségemet.
/2
#F45ABB89
 Bánt a sok gond, űz a bánat,
 Tört remény vagy félelem:
 Ő nem hágy el, biztatást ad:
 Békesség van énvelem.
/3
#9843B1E3
 Boldogságnak napja süt rám;
 Jóság, fény jár utamon:
 A keresztfa ragyogásán
 Fényesebb lesz szép napom.
/4
#54E46AD5
 Áldássá lesz ott az átok,
 Megbékéltet a kereszt;
 El nem múló boldogságod,
 Békességed ott keresd!
/5
#44B89132
 Áll a Krisztus szent keresztje
 Elmúlás és rom felett,
 Krisztusban beteljesedve
 Látom üdvösségemet.

;Bourgeois L., Genf, 1551
>231
/1
#7ED87BAA
 Uram, a te igéd nekem
 A sötétben szövétnekem;
 Mind igazak és ámenek,
 Amik szádból kijöttenek,
 Azért amit nem látok szemmel,
 Béveszem szavadra hitemmel.
/2
#C1AB3960
 Bízom hozzád erős hittel,
 Hogy te mindent megcselekszel,
 Amit szent igédben ígérsz:
 Hogy kegyelmesen hozzám térsz,
 És megbocsátván bűneimet,
 Megadod örök életemet.
/3
#498AC8F6
 E nagy jót neked köszönöm,
 Mely nekem arra ösztönöm,
 Hogy a Jézust, kiért velem
 Közöltetik a kegyelem,
 Tartsam lelkem megtartójának,
 Szeressem, engedvén szavának.
/4
#9A5CE7B4
 Igazgass, Uram, engemet,
 Hogy megőrizzem hitemet;
 Ha von magához e világ,
 Én mint Krisztusba oltott ág,
 Tőle vegyem tápláltatásom,
 Míg az élők közt lesz lakásom.

>232
/1
#CBB64DE2
 Hiszek a mennybéli egy Istenben,
 Mindenható és kegyelmes Atyában,
 Mennynek és földnek teremtő Urában,
 Ki mindent tart hű gondviselésben.
/2
#ABED3AFA
 És az ő egyszülött szent Fiában,
 Jézus Krisztusban, a mi egy Urunkban,
 Aki a Szentlélektől fogantaték
 S Szüz Máriától tisztán születék.
/3
#009A5D4A
 Szenvede Pilátusnak alatta,
 Felfeszíttetvén keresztre, meghala;
 Eltemettetvén poklokra leszálla,
 Harmadnap halálból feltámada.
/4
#0DB5EFEC
 Felméne mennybe, üle jobbjára
 A mindenható Atya Úr Istennek:
 Itéletükre onnan az élőknek
 És holtaknak eljő bizonyára.
/5
#80B11131
 Hiszek a szentelő Szentlélekben
 És egy keresztyén Anyaszentegyházat,
 Egyességét a szenteknek mindenben,
 És a bűnöknek megbocsánatát.
/6
#CC8D1F25
 Hiszem a testnek feltámadását
 És idvességgel az örök életet,
 Mellyet az Isten híveknek megszerzett:
 Lelkem e hittel bíztatja magát.

;Wittenberg, 1543
>233
/1
#5D291C84
 Úr Isten, te tarts meg minket,
 És szent igédben hitünket;
 Rontsd meg mi ellenségünket
 És minden kegyetleneket.
/2
#7C6800F8
 Kik a Krisztust háborgatni
 És szent székiből levetni
 Akarják, őtet rontani,
 Ő híveiben kergetni.
/3
#EF30D526
 Krisztus, ki vagy urak Ura
 És királyoknak Királya:
 Jelentsed istenségedet
 És törd meg ellenségidet.
/4
#79DAA0FC
 Tartsd meg minden híveidet,
 Te szegény keresztyénidet,
 Hogy dicsérhessünk tégedet,
 Tartsd egyességben népedet.
/5
#1E091338
 Ó, áldott Szentlélek Isten,
 Vigasztalj minket e földön,
 Légy jelen mi szükségünkben,
 Minden keserűséginkben.
/6
#755FFBF8
 Nevelj minket igaz hitben,
 Krisztusnak ismeretiben;
 Végy minket szent szerelmedbe
 És holtunk után örömbe.

;Wittenberg, 1524
>234
/1
#8CE9A958
 Jer, kérjük Isten áldott Szentlelkét
 Legfőképpen az igaz hitért,
 Hogyha jő a végóra, mellénk álljon,
 Hazatérésre készen találjon,
 Könyörüljön.
/2
#4B2BC280
 Jer, Világosság, ragyogj fel nekünk,
 Hogy csak Krisztus légyen mesterünk,
 El ne hagyjuk őt, mi hű Megváltónkat,
 Aki népének örökséget ad.
 Könyörüljél.
/3
#7DCCC88B
 Ó, Szeretet, áraszd ránk meleged,
 Hadd kóstoljuk édességedet;
 Tiszta szívből mindenkit hadd szeressünk,
 Egyességben és békében éljünk.
 Könyörüljél.
/4
#EC3E0259
 Ínségeinkben fő Vigasztalónk,
 Halál ellen megbátorítónk,
 Össze ne hagyj esni, ha ellenségünk
 Reánk jő s romlást készít már nékünk.
 Könyörüljél.

;Kolozsvár, 1744
>235
/1
#51D9B781
 Hallgass meg minket, nagy Úr Isten,
 E mostani nagy szükségünkben,
 És tekints meg minket mi életünkben,
 Hogy ne essünk e földön hitetlenségbe,
 Ördög kezébe.
/2
#51AA6B53
 Mert csak te vagy a világosság,
 Életünkben te vagy igazság,
 Mi setét szívünkben nagy világosság,
 És szomorú lelkünkben te vagy vigasság,
 Örök boldogság.
/3
#DAA53A73
 Távoztass tőlünk hamisságot,
 Add szívünkbe az igazságot,
 és a mi lelkünkbe nagy bátorságot,
 Hogy nyilván elhihessük a boldogságot:
 Örök országot.
/4
#C3D89C34
 Adj Szentlelket a tanítóknak,
 Egyetemben a hallgatóknak,
 Hogy mind engedhessünk akaratodnak,
 Dicséretet mondhassunk te szent Fiadnak,
 Jézus Krisztusnak.
/5
#E1BC4FFC
 Hála néked, mennybéli Isten,
 Ki vigasztalsz minket éltünkben,
 És el nem hagysz minket nagy szükségünkben,
 Megerősítesz inkább az igaz hitben,
 Ígéretedben.

;Kolozsvár, 1744 (1565)
>236
/1
#74AE8CCE
 Mindenek meghallják és jól megtanulják,
 Kik segedelmüket nem Istentől várják:
 Nincsen Isten nélkül segítség és idvesség.
/2
#DF7E8F63
 Ha nem az Úr Isten építi a házat,
 Ahány építője, mind hiába fárad.
 Nincsen Isten nélkül segítség és idvesség.
/3
#DFC4B550
 Csak hiába lészen reggel felkeléstek
 Néktek, kik erős Istenben nem hisztek:
 Nincsen Isten nélkül segítség és idvesség.
/4
#70DC02B8
 Ekképpen történik mindnyájan tinektek,
 Munkával, bánattal kenyeret kik esztek:
 Nincsen Isten nélkül segítség és idvesség.
/5
#AC310036
 Nagy könnyen az Isten mindent ád azoknak,
 Kik csak benne bíznak s hozzá fohászkodnak:
 Nincsen Isten nélkül segítség és idvesség.
/6
#DAA11AD6
 Mint a sebes nyilak az erős kezében,
 Erősek a hívek Isten kegyelmében:
 Nincsen Isten nélkül segítség és idvesség.
/7
#3749BEAB
 Boldog, aki lelkét hittel erősíti,
 Minden ellenségét bizonnyal meggyőzi:
 Nincsen Isten nélkül segítség és idvesség.

;Debrecen, 1774
>237
/1
#EF1ACA52
 Reménységemben hívlak, Uram Isten!
 Reggeli órám rólad áll szívemben;
 Testem is óhajt hozzád e helyben,
 Hol éltető víz nincs e kietlenben.
/2
#0B33EFC6
 Csudámra vagyon szépsége házadnak,
 Vajha ott volnék, hol téged szolgálnak!
 Orvosság én lelkem fájdalmának,
 Az lenne öröm könnyező orcámnak!
/3
#80185C66
 Élesztő nékem nagy-drága beszéded,
 Melyben dicsőségedet kijelented,
 Embernek szívét azzal emeled,
 Szent biztatásra mikor kényszeríted.
/4
#C6AFB584
 Velem viselem ott is ízit annak,
 Valahol nékem nyugodalmat adnak,
 Csak jókat tőlem mindenütt hallnak,
 Édességedben ajakim mozognak.
/5
#E381154B
 Sőt mind egész jövendő életemben
 Áldlak, dicsérlek tégedet örömben.
 Élet vagy: azért tiszteletedben
 Hozzád emelem kezeimet hitben.
/6
#50BE74F2
 Azért ajakim csak téged dicsérnek,
 Szájam és nyelvem rólad énekelnek,
 Ágyamban is veled beszélgetnek,
 Éjjel és nappal téged emlegetnek.
/7
#2E333BD6
 Mint drágalátos illatú kenettel
 És kívánatos, ízes eledellel:
 Úgy vidul lelkem éltetéseddel,
 Mikor vigasztalsz angyali örömmel.
/8
#97654933
 Te vagy, ki nékem váltságot ígérhetsz,
 Ki engemet szent szárnyad alá rejthetsz,
 Vélem egyedül minden jót tehetsz,
 Előttem, Uram, soha el nem mehetsz.
/9
#6017BB15
 Te ígéretedhez mert támaszkodom,
 Az én lelkemben csak tebenned bízom;
 Én árvaságom azzal táplálom,
 Hogy jobb karodnak árnyékában nyugszom.

;Brno, 1798
>238
/1
#BB61BACA
 Teremtő Istenünk, Édes Atyánk nékünk!
 Hatalmas vagy, Irgalmad nagy:
 Ínségemben, Szükségemben,
 Kérlek, engem ne hagyj!
/2
#54127C49
 Szívem benned bízik,
 Hozzád fohászkodik;
 Nézz csendesen, Kegyelmesen,
 Fájdalmimon, Bánatimon
 Könyörülj kegyesen.
/3
#98469183
 Tudom, hogy kegyelmed
 Enyhít mindeneket,
 Bús szíveket, Sebeseket,
 Szegénységben, Betegségben
 ellankadt fejeket.
/4
#F197F4A1
 Jelen van az idő,
 Hogy te, jó segítő,
 Meghallgatod Siralmimat,
 Nékem adod Tanácsodat,
 Mert tudod sorsomat.
/5
#9387FA7E
 Megváltó Istenünk,
 Szabadító Urunk:
 Vagy paizsom,
 Fő orvosom,
 Reménységem, Idvességem
 És gyönyörűségem.
/6
#0FA4570E
 Azért, ó, Jézusom,
 Legbölcsebb orvosom,
 Siess, siess,
 És lelkemnek
 Segítője Légy testemnek,
 Megepedt szívemnek!
/7
#BC426187
 Nálad lehetetlen,
 Tudom, semmi sincsen.
 Nyújtsd kezedet:
 Bús szívemet,
 Erősítsd bágyadt lelkemet,
 Mert áldlak tégedet.
/8
#D73D2736
 Szent Lélek Istenünk,
 Vigasztaló Urunk:
 Bátoríts meg,
 Oltalmazz meg,
 Igaz hitben,
 Szerelmedben
 Minden jókkal áldj meg!
/9
#C52D76F6
 Keresztemben tűrést,
 Adj boldog szenvedést,
 Hogy abban ne Zúgolódjam,
 És búban ne Tántorodjam,
 Hozzád folyamodjam.
/10
#2802AAE6
 Légy erős gyámolom,
 Csak rád támaszkodom.
 E világi Életemben,
 Halálom után mennyégben
 Véled legyek, Ámen.

;Wittenberg, 1535
>239
/1
#A335825E
 Úr Jézus, hozzád kiáltok,
 Ó, halld meg esdeklésem;
 Tetőled kegyelmet várok,
 Ne hagyj kétségbe esnem!
 Az igaz hitet kívánom,
 Ó, adjad azt énnékem:
 Néked élnem,
 Mindennek használnom,
 Szent Igédet követnem.
/2
#9EEC5419
 Ezt is kérem, én Istenem,
 Megadhatod azt könnyen:
 Szégyenbe ne engedj esnem,
 Tarts meg reménységemben.
 Kiváltkép, ha ki kell múlnom,
 Hogy csak tebenned bízzam,
 Ne magamban,
 Sem én jóságomban;
 Ezt örökké megbánnám.
/3
#9799DDA2
 Adjad, hogy teljes szívemből
 Engedjek vétetteknek,
 Te is ingyen kegyelmedből
 Kegyelmezz vétkeimnek,
 Igéd légyen eledelem,
 Mellyel lelkem tápláljam
 És megóvjam,
 Ha ínség ellenem
 Támad, el ne nyomassam.
/4
#1F8402D6
 Gyönyörűség, vagy félelem
 Te tőled el ne űzzön,
 Állhatatos mindvégiglen
 Légyek az igaz hitben.
 Az van kezedben: add ingyen,
 Nem cselekedetimből,
 De kedvedből
 Ments ki a halálból,
 Az örök kárhozatból!
/5
#65098133
 Nagy harcban, ellenkezésben
 Te segélj meg, Istenem!
 Csak a te nagy kegyelmedben
 Van nyugtom és védelmem.
 Ha kísértet jő, védelmezz;
 Szent kezed megsegítsen,
 Meg ne ejtsen,
 Ami kárt szerezhet;
 Tudom: nem hagysz elvesznem.

;Kolozsvár, 1744
>240
/1
#4264C229
 Ó, én két szemeim, ti az Úrra nézzetek,
 Hogy kegyelmes hozzám, mindenkor elhiggyétek;
 Az ő nagy szerelmét, Hozzám nagy jó voltát
 Mindenkor hirdessétek.
/2
#1E9A0644
 Megragadlak, Uram, az én igaz hitemmel,
 Reád támaszkodom erős reménységemmel,
 El sem is bocsátlak, Amíg meg nem hallgatsz,
 Az én lelki kezemmel.
/3
#E3B5E025
 Felindítlak téged nagy nyomorúságimmal,
 Az én bűneimnek számtalan sokságával;
 Mindaddig kiáltok,
 Míg bé nem kötsz engem
 Szent irgalmasságoddal.
/4
#5A19FD3B
 Ó, mennybéli Isten, te vagy én reménységem,
 Kinek hatalmában, kezében ellenségem;
 Ott vagy te, Úr Isten,
 Én nagy segítségem,
 Ahol nincs reménységem.
/5
#4AE9BBCE
 Tanítsál meg engem a te igaz utadra,
 Hadd lássak elmenni a te igazságodra;
 Én ellenségimet
 Gonosz szándékukban
 Ne bocsássad szájukra.
/6
#1897E8E5
 Ha tebenned, Uram, nem reménylettem volna,
 A nagy bánat miatt megemésztettem volna;
 Az élőknek földén
 A te javaidat
 Meg nem láthattam volna.
/7
#E8E1FBA3
 Azért, ó, én lelkem, serkentsd fel te magadat:
 Mit töröd, fárasztod nagy bánatban magadat?
 Majd meg fogod látni Uradnak jóvoltát,
 Csak el ne hagyd magadat.

;Kolozsvár, 1744
>241
/1
#D813B8FE
 Szent vagy örökké, Atya Úr Isten,
 A magas menynyekben,
 Ki teremtettél és megtartottál
 E nyomorult testben.
/2
#D6946D29
 Szent a mi Urunk, Úr Jézus Krisztus,
 Kit értünk bocsátál,
 Kivel megváltál, Hogy testünk miatt
 Lelkünk el ne vesszen.
/3
#9C1C89EA
 Szent a Szentlélek, Ki az Atyával
 Egy és a Fiúval,
 Aki vigasztal És tanít minket
 Az örök életre.
/4
#7647A8D8
 Szentség, dicsőség,
 Légyen tisztesség
 A Szentháromságnak,
 Ki uralkodik egy Istenségben
 Most és mindörökké.

;Kolozsvár, 1744
>242
/1
#56C64924
 Téged, Úr Isten, mi keresztyének
 Dicsérünk és áldunk,
 És Atya, Fiú, Szentléleknek vallunk.
/2
#C861C111
 Néked a tiszta, ártatlan és szent
 Angyalok szolgálnak,
 Szüntelen hangos felszóval kiáltnak:
/3
#08D64130
 Szent, szent, szent Isten, te seregednek
 Vagy Ura, Istene!
 Teljes a menny s föld nagy dicsőségeddel!
/4
#F51BDF0E
 Téged a földön a keresztyének
 Szent gyülekezetje,
 Szent Atya Isten, mindenkoron dicsér.
/5
#D38F3DB0
 És imádandó te szent Fiadat,
 Az Úr Jézus Krisztust,
 És vigasztaló Szentlélek Úr Istent.
/6
#23DCEEBD
 Oltalmazzad meg veszedelemtől
 A te népeidet:
 Oltalmazz bűntől s lelki kártól minket.
/7
#36BAE70F
 Dicsérünk téged, mennyei Atyánk,
 A te szent Fiaddal
 És mindörökké Szentlélek Istennel.

>243
/1
#CF7BB363
 Téged, ó Isten, dícsérünk,
 Téged Urunknak vallunk,
 Fő királyunknak esmérünk,
 Áldunk és magasztalunk:
 Véghetetlen felséged.
 Felséged. dicsőséged.
 Angyal, ember nem foghatja,
 mindenik csak csudálhatja.
/2
#9E280414
 Királyi széked kerítik
 Kérubok és szeráfok,
 Magukat arcra terítik
 Sok százezer angyalok,
 Minden boldog seregek,
 Kikkel az egek zengnek,
 Lerakván koronájukat,
 Téged imádnak, Urukat.
/3
#36D49687
 És ezt kiáltják szüntelen:
 Szent, szent, szent a Jehova!
 Dicsősége földön mennyen
 Elterjed mindenhova!
 Mindenütt jelen való,
 Mindent látó és halló,
 Királya az angyaloknak,
 Ura földi királyoknak.
/4
#394ADFE3
 Téged dicsér, Atya Isten,
 Szent egyházadnak népe,
 Mind ott fenn, mind alant itten,
 Mint szent felséged képe.
 A megszerzett idvesség,
 A hit és a kegyesség,
 Mind nagy kegyelmed munkája,
 Dicsőséged prédikálja.
/5
#30954E56
 Istennek egyszülött Fia,
 Ki bűn nélkül testünket
 Felvőd és lől atyánkfia,
 Hogy megváltsál bennünket.
 Ó, dicsőség királya,
 Minden keresztyén szája
 Szívből származott énekkel
 Dícséretedre énekel.
/6
#8C72451B
 Ó, Szentlélek, egy Istenség
 Az Atyával s Fiúval,
 Téged egész keresztyénség
 Megszentelőjének vall.
 Ó, dicső Szentháromság,
 Legfőbb szentség és jóság,
 Egy idvezítő Istenünk,
 Téged örökké dicsérünk.
/7
#923D36DF
 Ó győzedelmes királya,
 Jézus, megváltottidnak!
 Feje s irgalmas bírája
 Véren vett szolgáidnak!
 Jőjj, vezéreljed néped,
 És áldd meg örökséged;
 Tápláljad lelkünk igéddel,
 Szent testeddel, szent véreddel.
/8
#288CFE9C
 Hitünket, reménységünket
 Szentlelkeddel erősítsd,
 Istenhez szeretetünket
 S feleinkhez öregbítsd.
 Vígy előbb a szentségben,
 Adj részt a dicsőségben
 Testi halálkor lelkünknek,
 Feltámadáskor testünknek.
/9
#081058A9
 E végre még ez életben
 Bocsásd meg bűneinket,
 Hogy a jövő ítéletben
 Ne vádoljanak minket.
 Bízunk is szent véredben,
 Kezesi érdemedben;
 A halálban is remélünk,
 Hisszük, meg nem szégyenülünk.

>244
/1
#8F613D49
 Áldott légy, örök Úr Isten,
 Ki vagy foghatatlan felségben,
 A mennyei nagy dicsőségben,
 Angyaloknak nagy örömökben
 És dicséretökben.
/2
#3C85BF89
 Áldott légy, szentséges Atyánk,
 Ki kegyesen tekintél reánk,
 Bár mi halálra méltók valánk,
 Mert vétkezett mi első atyánk,
 Kitől mind származánk.
/3
#B82F5FCA
 Tőled azért el nem vetél,
 De minket oly igen szeretél,
 Hogy szent Fiadnak nem kedvezél,
 E világra kit értünk Küldél,
 Kivel hozzád szerzél.
/4
#E8C0E327
 Könyörülj mi rajtunk, kérünk,
 Szent igédet adjad értenünk,
 Honnan téged megismerhessünk,
 Hogy örökké Véled élhessünk,
 És örvendezhessünk.
/5
#AEA3A532
 Áldott légy, Úr Jézus Krisztus,
 És mi egyetlenegy Megváltónk,
 Mi életünk és közbenjárónk,
 Szent Atyád előtt egy szószólónk
 És egy bizodalmunk!
/6
#64638E23
 Tekints meg, édes Mesterünk,
 Ne hagyj bűnünk miatt elvesznünk;
 Szent hitedet erősítsd bennünk,
 Mindenben néked engedhessünk
 Benned bizhassunk.
/7
#B8987D3C
 Mert mi csak te tőled függünk,
 Azért csak tenéked könyörgünk,
 Hogy életünkben légy mivelünk,
 Csak te tőled viseltessünk,
 El ne tévelyedjünk:
/8
#F93F1E33
 Adj igaz fejedelmeket
 És keresztyén, bölcs tanitókat,
 Kik hirdessék akaratodat
 És hozzánk való jó kedvedet,
 Irgalmasságodat.
/9
#45CF4B79
 Áldott légy, Szentlélek Isten,
 Ki Atya-Fiúval vagy Isten
 Egy természetben s dicséretben,
 Egy akaratban és szentségben
 És egy dicsőségben.
/10
#AC51B897
 Áldott légy, árváknak atyja,
 Özvegyeknek kegyes bírája,
 Tévelygőknek igazgatója,
 Szomorú szívek biztatója
 És vigasztalója!
/11
#04A44A4C
 Bátoríts minket hitünkben,
 Adj szeretetet mi szívünkben,
 Tarts meg mindvégiglen ezekben,
 Hogy halálunknak idejében
 Ne essünk kétségben.
/12
#FB6C6ED0
 De mehessünk az életre,
 Hol Atya Istennek felsége
 És a Fiúnak dicsősége
 Láttatik véled egyetembe'
 Örökkön-örökre.

>245
/1
#649FC383
 Dícséret a Szentháromságnak,
 Egy örök Istennek,
 Ki Ura mindennek
 És teremtője a világnak:
 Csak övé légyen a dicsőség
 És a tisztesség!
/2
#C4A526AA
 Dicséret légyen az Atyának,
 Ki teremtett épen
 És takargat szépen
 Szárnyai alatt hatalmának,
 Sőt gondot tart bűnös testemre
 S egész éltemre.
/3
#68F6071C
 Dicséret légyen a Fiúnak,
 Ki értem testté lett,
 S minden bűnt eltörlött,
 És hozzám nagy-bő irgalmúnak
 Jelenti magát az életben
 S az ítéletben.
/4
#EB35D924
 Dicséret légyen Szentléleknek,
 Ki engem megszentelt
 S magának pecsételt,
 És vígassága a hiveknek,
 Ki vezérel az Igaz hitre
 S örök életre.
/5
#CD1051FC
 Dicséret, áldás és dicsőség,
 Erő és hatalom,
 Örök birodalom,
 Felmagasztalás és tisztesség
 Légyen lelkemtől néked, Isten,
 Mennyben és itten!

>246
/1
#1EA853B0
 Adj Úr Isten, nékünk Szentlelket,
 Szent ígédben épits meg minket,
 Adjad megismernünk természetünket,
 Tekintsd meg szükségünket, könyörgésünket:
 Adj igaz hitet.
/2
#802DF9BE
 Te vagy, Atyánk, örök Úr Isten,
 Kiben semmi kétségünk nincsen;
 Neveljed hitünket te szent igédben,
 Szent Fiadnak, Krisztusnak ismeretében,
 Idvességünkben.
/3
#FBDCDC13
 Te vagy, Krisztus, mi Idvezitőnk,
 Szent Atyádnál békesség-szerzőnk;
 Mi életünk vagy te: lakozzál bennünk,
 Hogy minden mi dolgunkban téged
 Dicsérjünk dicsőítsünk.
/4
#7E5FF9B0
 Ó Szentlélek, kérünk tégedet:
 Bátorítsad a mi szivünket;
 Vígasztalj meg minket s igazgass minket,
 Hogy hallván megérthessük te szent igédet,
 Idvességünket.

>247
/1
#2646213F
 Imádandó Isten, e széles föld néked
 Zsámolyod és a menny fényes ülö széked,
 A te mindenható kezeid csinálmánya
 És erőd tüköre e világ alkotmánya;
 Csak homályt találunk elöttünk és mélységet,
 Elménkkel megfogni akarván ily felséget.
/2
#4F401521
 Kezdete sohasem volt a te idődnek,
 Esztendeid soha el nem végeződnek.
 Minden dolgok töled származnak, mint kútfőtől;
 Te lételt nem vettél semmi megelőzőtől,
 A változásoknak körülötted helyt nem adsz,
 Aki régen voltál, te ugyanazon maradsz.
/3
#62D9F815
 Láthatatlan lélek, te betöltesz mindent,
 Jelen vagy itt alant, valamint odafent.
 Nincs oly dolog, melyet szemeid ne látnának,
 Bár fedjék ködei a sötét éjszakának.
 Ó, emberek, nincsen az Isten előtt titok,
 Tudva vannak nála minden gondolatitok.
/4
#9DBE40CD
 Csudákkal teljes a természet temploma,
 Melyekből kitetszik bölcseséged nyoma.
 Nagy vagy a nagyokban, bölcs vagy a kicsinyekben,
 Gondod el nem fárad annyi sok ezerekben;
 Véglen jóságod a menny és föld hirdeti,
 Sokaságát nyelvünk mert ki nem beszélheti.
/5
#6B2F5B3E
 Szenteknek szentje vagy, ó Isten s nincs hiba
 Szent akaratodnak határozásiba'.
 Megáll igazságod, ha tartasz törvényszéket,
 Mivel jól ismered a szivet és veséket;
 Mégis a bűnösnek halálán te nem örülsz,
 Inkább nagy irgalmad szerint rajta könyörülsz.
/6
#6F05632D
 Boldog nép az, amely imád ily felséget,
 Boldog, kit ily felség kegyelmébe bévett,
 Akinek hatalma mindenre elégséges,
 Jósága megadja, akinek mi szükséges.
 Tartsd fenn, ó jó Atyánk, köztünk ez ismeretet,
 Hogy te nem egyéb vagy, hanem merő szeretet.

;Genf, 1562
>248
/1
#3F446B1D
 A csillagos égnek seregei ott fent
 Tündöklő fényükkel dicsérik az Istent.
 Az egek boltjai alatt forgó világok
 Mély bölcsességéről beszélő bizonyságok;
 Az ő szavára lett tenger, hegy, füvek és fák
 Jóvoltát erejét szüntelenül kiáltják.
/2
#03D6D4BF
 Hát én, kibe lelket a jó Teremtő tett,
 Némán vesztegeljek s ne dicsérjem őtet?
 Nem, sőt minden lelki erőmet összeszedem,
 Elmémnek szárnyain székihez emelkedem;
 Szólok, s ha tétováz rebegése nyelvemnek,
 Könnyeim tanúi lesznek tiszteletemnek.
/3
#15D89D33
 Nyelvem rebeg, de te, ki szívem vizsgálod,
 Abban oltárodat füstölve találod.
 Én, ha a napfénybe mártanám is ecsetem,
 Felséges voltodnak árnyékát sem festhetem;
 A te tisztelőid bár legyenek angyalok,
 Csak erőtlen hangon dicsértetel általok.
/4
#AC972BBB
 Ki ruházott fényt a sok ezer napokra,
 Színes sugárokat rakván orcájokra?
 Ki indítá útnak a számtalan földeket?
 Ki kötött egymáshoz ennyi ezer testeket?
 Kimért útjaikon ezeket ki mozgatja?
 Uram, a te szádnak mindenható szózatja.
/5
#D57DB659
 Te tartod a zsengés tavasznak kezeit,
 Míg kiteregeti színes szőnyegeit;
 Aranyszín ruhát adsz az ért gabonafőre,
 Fürtökből koronát tűzöl a szőlőtőre,
 A megfagyott földet hópelyhekbe pólyálod,
 Lakosit élesztő örömökkel kínálod.
/6
#F3558B21
 Uram, ezer nyelvek kellenének számba,
 Jóságod csudáit hogy vehessem számba;
 Még ami gonosz is, használ nékünk más részbe,
 Istennel nem tartók, ezt vegyétek jól észbe,
 És ha nem vagytok megilletve jóvoltától,
 Féljetek, mint szökött rabszolgák, haragjától!
/7
#AE2681A1
 Felelj, kételkedő: ki mennydörög ott fent?
 A szélvész harsogó zúgása kit jelent?
 A sík tenger vizét ki emeli hegyekké?
 A szárazföld színét ki süllyeszti völgyekké?
 Mindegyik éltető elem ím, felemelte
 Szavát, hogy van Isten: hát mit kételkedel te?
/8
#67E951E7
 Uram, munkáidról lészen dicséretem,
 Míg tart az e végre adatott életem.
 Kérlek, vedd jó néven, ha egy gyarló féregnek
 Ajkai töredék dicséretet rebegnek!
 Te látod, mily buzgók bennem az indulatok,
 Melyeket jól érzek, ki nem magyarázhatok.
/9
#D8667E8D
 Ó, ha kivetkezvén e testből idővel,
 Állok széked előtt koronázott fővel,
 Több lesz énekemben az erő és méltóság,
 Magasztalván téged, ó, felséges valóság.
 Röpüljetek elő, kívánt idők, sebesen,
 Hogy örömeimben több változás ne essen!

>249
/1
#53CFCAAD
 Nagy az Úr, ki fényes házát építette,
 Ott fenn a számtalan egeknek felette,
 Hol sok ezer napok fényesítik pitvarát,
 A magasság s mélység határozzák udvarát:
 Újul a föld színe vidám tekintetétől,
 Tengelyestől reszket bús gerjedezésétől.
/2
#24FF8E26
 Dicsérjétek őtet, a mennyei boltok
 Alatt, ó, csillagok, mik ott lámpásoltok;
 Ti napok seregi, világító fáklyátok
 Neve dicséreti között lobogtassátok;
 Ti sok rendű holdak, a bújdosó földekkel
 E nagy, erős Urat dícsérjétek énekkel.
/3
#0594DD6B
 Énekeljen neki az egész természet,
 Napkelet, dél, észak és a napenyészet;
 Kettőztessék meg a dicséret, midőn annak
 Hangjai a hegyek felől visszapattannak;
 És te, ember, Kit ő tett e földön elsővé,
 Változz dicsőségét mindíg buzgó kútfővé.
/4
#CAA2A164
 Magasztald fel őtet, hogy használj magadnak,
 Dícséretid neki mert hasznot nem adnak;
 A bűnt vetkezzed le rólad s a testiséget,
 Hozzá emelkedve dicsőítése végett;
 Lehullnak, ha vele társalkodol, mennyei
 Származású lelked földön ragadt szennyei.
/5
#F2C4F805
 Egyszer se fogjon úgy a nap futásához,
 Egyszer se jusson el úgy lenyugvásához,
 Hogy őt ne dicsérnéd szívednek tolmácsával,
 Egyesítvén szavad a természet szavával;
 Ez szólít meg téged, hogy hál'adásra serkenj:
 Szívednek-értelmednek beszél a föld és menny.
/6
#37614C33
 Ha esők omlanak a földre záporral,
 Ha az aszály miatt füstöl a föld porral,
 Ha fagyos fürtjeit látod a zuzmarának,
 Ha a liliomok a réteken pompáznak,
 Ha tisztán süt a nap, vagy fergetegek jőnek:
 Nagyságos dolgait hírdesd a Teremtőnek!
/7
#D659512B
 Ha elboritással az árvíz fenyeget,
 Ha a dögvész pusztít, ha a had rettegtet:
 Bízz benne s énekelj őnéki dícséretet,
 Mert híven szereti az emberi nemzetet;
 Ő a bölcs, ő tudja, néked mi használ, nem te:
 Győződj meg, hogy téged boldogságra teremte.
/8
#E375301E
 Ez a kegyes lsten nem vet engem is el,
 Rólam is hűséges volta gondot visel;
 A kincs és dicsőség ha enyémek nem lettek:
 Igazság-vizsgáló lelket adott helyettek;
 Adott külső-belső ép érzékenységeket,
 Kezeimbe lantot, szájamba új éneket.
/9
#F84E780C
 Uram, amit adtál, tartsd meg azt számomra,
 Több nem kívántatik én boldogságomra;
 Én pedig mit adjak néked gyarló létemre,
 Háládatlanságom hogy ne térjen szememre?
 Imádlak téged, bételvén szent félelemmel,
 Dícséretedet eggyé teszem éltemmel.

>250
/1
#36CA857E
 Atya Úr lsten, rólad vallást tészek,
 Dícséretedre szám és szívem készek:
 Te vagy az erős, ki tudtál isteni
 Szóddal semmiből mindent teremteni;
 Amit fent s alant szemeink szemlélnek,
 Általad vannak, mozognak és élnek.
/2
#8D674CB8
 Világosság lőn parancsolatodra,
 A menny és a föld előállt egy szódra;
 A levegő eget kiterjesztetted,
 Azzal az egész földet körülvetted;
 Szódra vált külön a föld a vízektől,
 Ékesíttetvén fáktói és füvektől.
/3
#1BD46FEE
 A napot, holdat és a csillagokat
 Te függesztetted fel, mint lámpásokat,
 A madaraknak te adtál szárnyakat,
 Te küldéd vízben lakni a halakat;
 A barmok testét élettel élesztéd,
 Az emberrel bölcs munkád bérekesztéd.
/4
#938EC455
 Imádlak téged, aki teremtettél,
 Szeretlek, mivel Atyámmá is lettél;
 Ha terheltetik szükséggel életem,
 Bizodalmamat benned helyheztetem,
 Mert hatalmas vagy engem megtartani,
 Még nyavalyám is jómra fordítani.

;Debrecen, 1774
>251
/1
#AEE7BAEA
 Meghódol lelkem tenéked, nagy Felség,
 Szentháromságban ki vagy egy Istenség.
 Csak téged illet minden tisztesség,
 Mert téged ural az egész föld s ég.
/2
#D2ED5320
 Imád a nagy ég ő teljességével,
 Mondván: szent, szent, szent az Úr felségével!
 Teljes a föld s ég dicsőségével,
 Seregek Ura erősségével!
/3
#18696565
 Imád a földnek kiterjedt nagysága,
 Mind e világnak nagy hatalmassága;
 A sok népeknek minden országa,
 Roppant táborok sűrű sokasága.
/4
#B49DCBBB
 Imád téged a napnak fényessége,
 Az éjszakának titkos setétsége;
 Imád a holdnak ő teljessége:
 A csillagoknak szép ékessége.
/5
#921430D5
 Imád a tenger s a vizek folyása,
 A sok hegyeknek magas fennállása;
 Minden szeleknek széjjeloszlása
 S a madaraknak ékes szólása.
/6
#A9176FB0
 Imádlak én is téged, Teremtőmet,
 Gondviselőmet és Idvezítőmet,
 Megszentelőmet s erősítőmet:
 Én Istenemet, egy Segítőmet.
/7
#6AA558CB
 Imádlak téged, Urát kezdetemnek,
 Urát végemnek s egész életemnek,
 Fő szerző okát a természetnek,
 Halálnak Urát és kegyelemnek.
/8
#3EA24253
 Imádlak téged, egyedül Uramat,
 Nem vetem másban én bizodalmamat;
 Mikor imádlak, halld meg én szómat,
 Írd be könyvedbe hódolásomat.

>252
/1
#D2BB240D
 Mennyei seregek. boldog, tiszta lelkek.
 Az Úrra örökké kik az égen néztek:
 őtet teljes szívből ti mind dicsérjétek!
/2
#0599D47C
 Angyalok, az Úrnak követi kik vagytok,
 Szentek, kik ő székét mind körülálljátok:
 Örökké az Urat felmagasztaljátok!
/3
#2477BB3B
 Fényes nap világa, ez világ fáklyája;
 Szép hold, éj lámpása, csillagok nagy száma
 Az Úrnak szent nevét mindörökké áldja.
/4
#40F8F72C
 Mert csak ő egyedül minden teremtője,
 S mindent bír, valamint magában rendelte;
 Megmarad mindenek ellen ő szerzése.
/5
#38C72C49
 Azért hát, ti hivek, Úrnak szent serege,
 Kik leginkább vagytok néki szerelmébe',
 Örökké szent nevét dícsérjétek mennybe'!

;Genf, 1562
>253
/1
#7F777BBA
 Isten kezét mutatja
 Az égnek boltozatja,
 Mely felettem kiterjed,
 Szívem örömre gerjed
 E remeknek látására.
 Alkotója lelkemet
 És minden érzésemet
 Ragadja csudálására.
/2
#148E0736
 Ha elmémmel felhágok
 E számtalan világok
 Roppant alkotmányába:
 Ott látom valójába'
 Nagy voltát a Teremtőnek.
 Minél feljebb repülök,
 Annál jobban szédülök
 Magasságán e tetőnek.
/3
#BA131E87
 Ki gyújtá meg ezeket
 Az örökös tüzeket?
 E nagy testeket fontba
 Ki vetette, hogy pontba
 Egymásnak megfeleljenek?
 Ki mérte ki útjukat
 És örök pályájukat,
 Hogy erről el ne térjenek?
/4
#E7572515
 Te vagy az, Mindenható,
 Kihez hasonlítható
 Nincs sem földön, sem égen!
 Te vagy, ki voltál régen,
 Kiben nincsen fogyatkozás,
 Kiben nincsen hajdani,
 Jövendő vagy mostani,
 Kihez nem járul változás.
/5
#59075AEE
 Én hát mély tisztelettel,
 Mely teljes szeretettel,
 Előtted megnémulok,
 Zsámolyodhoz borulok,
 És imádom nagy voltodat,
 Hogy bármily kicsiny vagyok,
 Szintúgy, mint ezen nagyok,
 Tapasztalom jóságodat.

;Debrecen, 1778
>254
/1
#7003EA57
 Mindenkoron áldom az én Uramat,
 Kitől várom én minden oltalmamat.
 Benne vetem minden bizodalmamat;
 Mindenkoron dicsérem, mint Uramat.
/2
#1FE78C82
 Igen vigad és örvendez én lelkem,
 Az Istennek segedelmét hogy kérem,
 Nyomorultak meghallják, azt örvendem,
 Vigadjanak Istenben, arra intem,
/3
#361FC9D1
 És mikoron Istenhez kiáltottam,
 Kegyelmesen tőle meghallgattattam,
 Őáltala hamar megszabadultam,
 Háborúságimban is megtartattam.
/4
#6B5FEBFD
 Lám, Istennek angyala mind tábort jár,
 Az istenfélő emberek körül jár.
 Az Istentől azért ki oltalmat vár,
 Útaiban mindenütt az nagy jól jár.
/5
#FEBE5247
 Segítségül azért Istent hívjátok,
 Ő jóvoltát kóstoljátok, lássátok!
 Igen nagy-jó, azt bizonnyal tudjátok:
 Benne bízó emberek mind boldogok.
/6
#D126F4FC
 Valamíglen élsz ez árnyék világban,
 Szántszándékkal ne élj a gonoszságban,
 Sőt életed foglaljad minden jóban,
 Hogy lakozzál Istennek oltalmában.
/7
#CE7260CF
 Sok jók közt a békességet szeressed,
 És éltedben mindenkor azt keressed;
 E világnak békességét ne nézzed,
 Az ördöggel ne légyen közösséged.
/8
#526610AA
 A felséges Isten szemei vannak
 Igazakon, kik csak őbenne bíznak;
 Mindazok, kik tőle oltalmat várnak,
 Kérésükben mindig meghallgattatnak.
/9
#7A9A7EEB
 Igen közel az Úr Isten azoknak,
 Töredelmes szívvel akik óhajtnak;
 Alázatos lélekkel akik járnak,
 Sok ínségből bizton megszabadulnak.

;Régi magyar dallam
>255
/1
#DA264B1D
 Mely igen jó az Úr Istent dicsérni,
 Felségednek, én Uram, énekelni,
 Szent nevedet dicsérvén magasztalni
 És mindenütt e világon hirdetni.
/2
#9AAFDB47
 Igen reggel irgalmadat hirdetni,
 Igazságodról éjjel gondolkodni,
 Hegedűvel, orgonával zengetni,
 Minden éneklő szerszámmal tisztelni.
/3
#D37D5A20
 Csudaképpen én vigasztalást vészek,
 Cselekedetidre hogyha tekintek,
 Kezeidnek munkájában örvendek,
 Teremtőmnek, megváltómnak éneklek.
/4
#247B7BE5
 Az esztelen ember ezt nem esméri,
 A hitetlen bolond ember nem érti;
 Kinek rólad nincs igaz esméreti:
 Szent Fiadban mert nincs hite őnéki.
/5
#A98698CA
 E világon gonoszok gyökereznek,
 Kik mindenkor hamisan cselekesznek;
 Mint a füvek, virágoznak, terjednek,
 Hogy örökül-örökké elvesszenek.
/6
#1EAE5720
 Lám, ezeket, Uram, felséges Isten,
 Kik támadnak a te szent igéd ellen,
 Viaskodnak a te híveid ellen:
 Megbünteted, mert vagy örök Úr Isten.
/7
#B1B60331
 Rólad, Uram, akik megemlékeznek,
 Mint pálmafák úgy szintén ők zöldellnek;
 Mint cédrusfák, ugyan meggyökereznek
 Az igazak, kik igaz hitben élnek.
/8
#7C62AAE6
 Vallást tesznek
 Minden emberek előtt,
 Hogy az Isten igaz mindenek fölött;
 Hamisságot soha nem cselekedett,
 Mint kőszikla, ő ád nagy erősséget.

;Debrecen, 1778
>256
/1
#D594052B
 Irgalmazz, Úr Isten, immáron énnékem!
 Irgalmazz, Úr Isten, immáron énnékem,
 Mert tebenned bízik, Uram, az én lelkem,
 És tebenned nyugszik, Uram, az én szívem.
/2
#F717949D
 Szárnyad alá vetem az én reménységim,
 Míg elmúlnak tőlem az én ellenségim,
 És míg eltávoznak tőlem én bűneim:
 Csak tebenned lésznek, Uram, én örömim.
/3
#9A96845B
 Tehozzád kiáltok, hatalmas Úr Isten,
 Mert nincsen, énvélem ki már jót tehessen,
 Én ellenségimtől engem megmenthessen,
 És én dolgaimban ki jóra vihessen.
/4
#5B56E465
 Velem, én szent Atyám, nagy-sok jókat tettél,
 Mert énnékem mennyből őrizőt küldöttél,
 Én ellenségimtől engem megmentettél
 És az én szívemben örömöt szerzettél.
/5
#9F9539BE
 Azért téged, Uram, én felmagasztallak,
 Te dicsőségedben hatalmasnak mondlak,
 Mind e világ előtt irgalmasnak vallak;
 És jóakarómnak én tégedet hívlak.
/6
#EBEC1404
 Kész már az én szívem néked énekelni,
 Kész most jóvoltodért nagy hálákat adni,
 És mindenek előtt téged megvallani,
 A te szent nevedet örökké dicsérni.
/7
#D0F90F21
 Kelj fel azért mostan, én nagy dicsőségem,
 Légy mindenben nékem kedves segítségem,
 Erőtlenségemben légy én erősségem
 És veszedelmemben légy oltalmam nékem.
/8
#980C23A3
 Én felmagasztalom irgalmasságodat,
 Mindenkoron vallom te igazságodat,
 Mind ez egész földön hatalmasságodat:
 Mindenkor hirdetem a te jó voltodat.

;Kolozsvár, 1744
>257
/1
#0FEF807F
 Ne hagyj elesnem, felséges Isten, keserűségemben!
 Te szent Fiadért légy segítséggel: ne essem kétségben,
 Mert mindenfelől, látod, Úr Isten, vagyok kísértetben.
/2
#1C7B7F76
 Az írás rólad, Felséges Isten, bizonnyal azt mondja,
 Hogy valakinek tebenned vagyon szíve nyugodalma,
 Az olyan ember meg nem szégyenül, mert te vagy oltalma.
/3
#7FFEFF62
 Gyermekségemtől fogva, Úr Isten, mind ez ideiglen
 Téged hívtalak én segítségül minden szükségemben,
 Mostan is nincsen több bizodalmam sem földön, sem mennyen.
/4
#0680CF34
 Nincsen szívemnek több bizodalma, Úr Isten, náladnál,
 Valamíg gyötresz, szabad légy velem, csak ne haragudjál,
 Mint kegyes Atya, fiadat dorgálj, csak hogy meg nem utálj.
/5
#E7C51B96
 Csak te egyedül voltál, Istenem, énnékem gyámolom,
 Nagy fájdalmimban és romlásimban az én vigasztalóm;
 Ne hagyj elesnem s megszégyenülnöm, kegyes oltalmazóm!
/6
#7D139799
 Mely nagy örömem és bizodalmam vagyon nékem ebben:
 Hogy ígéreted, mint drága zálog, itt van én szívemben;
 Krisztus Jézusért engem meghallgaszt, tudom, kérésemben.
/7
#6A5D8AEF
 Jelentsd meg hozzám, felséges Isten, kegyelmességedet
 És véghetetlen, kegyes atyai te nagy szerelmedet,
 Hogy teljesítsd be könyörgésemre szent ígéretedet.
/8
#2F4F0CA5
 Add meg, Úr Isten, te szent nevedért, amit tőled kérek,
 Szent Fiad által, teljes szívemből melyért most könyörgök,
 Mert csak tebenned bízom, Úr Isten, míg e testben élek.

;Kolozsvár, 1744
>258
/1
#DE56CF28
 Fohászkodom hozzád, Uram, Istenem!
 Kérlek, kegyelmesen hallgass meg engem,
 Mert tebenned soha nem volt kétségem,
 Azért most is tehozzád esedezem.
/2
#7B07B222
 Látod, Uram, igen megnyomorodtam.
 Előtted nagy nyavalyára jutottam,
 De míg te szent istenségedben bíztam,
 Soha semmiben el nem hagyattattam.
/3
#AF63BF1D
 Reménységem míg el nem fogyatkozott,
 A te ígéreted nálam nyilván volt,
 Hogy énnékem mind megadod azokat,
 Melyeket én szívem tőled óhajtott.
/4
#F8A46D4A
 Azért téged hívlak csak segítségre,
 És magamat nem is bízom senkire;
 Én lelkemet vigyed hálaadásra,
 És szívemet juttasd nagy vigasságra
/5
#BFD27460
 Irgalmasságodat mikor hallhatom,
 Legott elfelejtem minden bánatom;
 Abban vagyon nékem nagy vigasságom,
 Bűneimnek bocsánatját hogy bírom.
/6
#1338A6AF
 Jelentsd nékem a te akaratodat,
 Fordítsd hozzám szent irgalmasságodat;
 Add meg nékem most, amit tőled várok,
 Melyért dicséretet örökké mondok.

>259
/1
#E14380E6
 Benned bíztam, Uram Isten, soha ne gyaláztassam;
 A te szent igazságodért kérlek, szabadittassam:
 éjjel-nappal könyörgök, halld meg én kérésemet.
/2
#099C8139
 Nincs én nékem segítségem: Uram, te légy oltalmam;
 Bátorságom vagy én nékem és örök bizodalmam:
 Hajtsd hozzám te füledet, siess, tarts meg engemet.
/3
#89CB1F82
 Erősségem ördög ellen, te vagy nagy reménységem,
 Szent nevedért ments meg engem, mert nincs nékem érdemem:
 Éjjel-nappal könyörgök, halld meg én kérésemet.
/4
#7A9850FE
 De sietnek ellenségim titkon tőrbe ejteni;
 Uram, te vagy reménységem, nékik ne hagyj elveszni:
 Hajtsd hozzám te füledet, siess, tarts meg engemet.
/5
#F4994C5C
 Ím ajánlom kezeidbe én szomorú lelkemet:
 Igazmondó örök Isten, oltalmazz meg engemet:
 Éjjel­nappal könyörgök, halld meg én kérésemet.
/6
#37DC2F99
 Csak hívságot követőknek rontója vagy, Úr Isten;
 Benned erősen hívőknek irgalmas vagy szüntelen:
 Hajtsd hozzám te füledet, siess, tarts meg engemet.
/7
#3C0D9BDE
 Értem immár hozzám való kedves akaratodat;
 Benned bíztam: mutasd immár nagy irgalmasságodat:
 Éjjel-nappal könyörgök, halld meg én kérésemet.
/8
#9A3A48E7
 Isten, mily nagy dicsőséget tartasz szeretőidnek,
 Hol örökké gyönyörködnek és véled örvendeznek:
 Hajtsd hozzám te füledet, siess, tarts meg engemet.
/9
#6BD2BEDB
 Soha nékik már nem lészen szükségük, szegénységük,
 Tartod őket titkos házban, hol nincs semmi félelmük:
 Éjjel-nappal könyörgök, halld meg én kérésemet.
/10
#532FE8AA
 Vallást tészek, hálát adván néked, kegyes Istenem:
 Mindennémü szükségemben jelen voltál én nekem;
 Hajtsd hozzám te füledet, siess, tarts meg engemet!

>260
/1
#C454E50F
 Szent Dávid próféta éneklő könyvének huszonharmad részében,
 Bizván az Istennek az őreá való nagy gondviselésében.
 És hálákat adván ő szent Felségének mind egész életében.
 Igyen dicsekedik lelkében minden hív keresztyének képében.
/2
#545BEF2F
 Tudom, hogy pásztorom s vezérem énnékem az én Uram Istenem,
 Kinek gondja vagyon testemre, lelkemre, mert ő teremtett engem,
 Én ő juha vagyok, ő hozzá hallgatok, mert ő megváltott engem;
 Azért benne bízom, tudom, élet nélkül hogy én nem szükölködöm.
/3
#2D730DDB
 Az ő mezejének szép zsíros füvével szépen legeltet engem,
 Szent ígéretinek édes beszédével mikor vígasztal engem;
 Anyaszentegyházban és az ő aklában szépen megnyugtat engem,
 LopótóI, farkastól, hamis tanítótóI ott megoltalmaz engem.
/4
#43ED0C5F
 Reggel, hogy kiviszen, én előttem mégyen igaz tudományával,
 Minden napon kétszer megitat engemet ő lelki italával,
 Lelki folyóvíznek és élő kútfőnek ő gyenge folyásával,
 Evangéliumnak anyaszentegyházban ő prédikálásával.
/5
#B928CF0A
 Azzal én lelkemet ugyan megújítja és újonnan megáldja,
 Mennyországnak útát és az Igaz hitet énnékem megmutatja;
 Az ő szent Fiában szomorú lelkemet csak egyedül biztatja,
 MéltatIanságomban az ő szent Fiáért irgalmasságát nyújtja.
/6
#128F0C2A
 Ily nagy reménységgel és bizodalommal azért te benned bízom,
 Uram, Istenem, testemet, lelke­met én tenéked ajánlom;
 Ha szintén a halál völgye árnyékában az én fejemet látom:
 Mert te vagy én vélem, szabadulásomat azért tetőled várom.
/7
#462C8A3E
 Megvígasztalt engem te pásztori vessződ és te pásztori botod,
 A sok nyomorúság, mellyel híveidet megsújtod, sanyargatod;
 Mellyel beszédedhez oly nagy szépen őket szoktatod és tanítod,
 Hogy el ne vesszeneK, sőt veled legyenek, mindörökké megtartod.
/8
#0421C2E4
 Ily nagy készülettel ellenségim előtt asztalt szerzesz énnékem,
 A te szent Fiadnak egy áldozatjával táplálod az én lelkem;
 Megvígasztalsz engem s mindenkor megáldod bőséggel én életem,
 Én ellenségimnek nagy bosszúságukra megtartod az én lelkem.
/9
#DB2AED33
 Halandó testemnek ő gyarlóságától megszabadítasz engem,
 E világ sem árthat sem álnokságával, sem bosszújával nékem,
 Mert én te általad fó ellenségemet, az ördögöt meggyőztem,
 Az én életemet csak te benned, bízván, mindenkor helyezhetem.
/10
#1DEE412C
 A te jóvoltodból, kegyelmességedből mindezeket én várom,
 Mert ezekre való nagy méltóságomat magamban nem találom,
 Azért minden napon az én életemet én csak tereád bízom,
 És tudom bizonnyal, hogy tebenned soha én meg nem csalatkozom.
/11
#2DB64726
 Lészek mindörökké az én Istenemnek és Uramnak házában,
 Az én Krisztusomnak pásztorsága alatt az anyaszentegyházban,
 Az ő juhainak, igaz híveinek kívánt társaságában,
 Kikkel egyetemben örökké vígadok, az örök boldogságban.

;Kolozsvár, 1744
>261
/1
#A51D98C7
 Kegyelmes Isten, kinek kezében
 Életemet adtam,
 Viseld gondomat, vezéreld utamat,
 Mert csak rád maradtam.
/2
#B401F8FA
 Add meg énnékem én reménységem
 Szerint való jódat!
 Áldd meg fejemet, ki bízik benned,
 S viseljed gondomat!
/3
#0FA433D4
 A szép harmatot miként hullatod
 Tavasszal virágra,
 Sok jódat, Uram, úgy hullasd reám,
 Te régi szolgádra.
/4
#91592F5A
 Hogy mind holtomig szívem legyen víg,
 Téged magasztalván,
 Mindenek előtt s mindenek fölött
 Szent nevedet áldván.

;Pozsony, XVIII. század végén
>262
/1
#D1333A64
 Forog a szerencse,
 Mit bízunk őbenne?
 Semmiben nem állandó.
 Csak ideig kedvez,
 Tündöklő üveghez
 Mindenképpen hasonló.
 Ki minthogy eltörik,
 Így ő is változik,
 Állapotja romlandó.
/2
#56AB4E58
 E rossz, csalárd világ,
 Ki merő hamisság,
 Engem már majd elveszte,
 Mint halat horogra,
 Úgy csala sok búra
 Hízelkedő beszéde.
 Étkével táplála,
 Melyben rejtve nála
 Iszonyú csalárd mérge.
/3
#B9B00571
 Nyughatatlan sok gond
 Íme, annyira ront,
 Hogy nem soká megemészt.
 Ha nem szánja lelkem
 Teremtő Istenem,
 Sok bú, siralom elvészt,
 Sőt én ellenségem
 Csudálja, hogy engem
 A föld is el nem süllyeszt.
/4
#956FA8F9
 Éljek-e? - nem tudom,\.Mert késik halálom
 S kínjaim nevekednek.\.Kétségbe essem-e,
 Halált szerezzek-e\.Nyomorodott fejemnek?
 Azt nem mívelhetem,
 Mert elveszti lelkem,
 Haragja Istenemnek.
/5
#76CB5A1D
 Tűrnöm hát jobb lészen,
 Mert még elővészen
 Istenem, szent Fiáért.
 Bűnöm elfelejti,
 Pokolra sem veti
 Lelkem, ki hozzá megtért.
 Bár súlyos kereszttel
 Próbál és megterhel
 Méltatlan életemért.
/6
#AFAFBD31
 Szenvedek békével,
 Magamat is ezzel
 Biztatom mindenekben:
 Szenvedett Megváltóm
 Többet, én jól tudom,
 Üdvösségemért testben.
 Nem egyedül vagyok
 Földön, ki nyomorgok:
 Szenvedik ezt is többen!
/7
#68D832D5
 Megszán még Istenem,
 Ennyi sok siralmim,
 Tudom, jóra fordulnak.
 Múlnak sóhajtásim,
 Noha most könnyeim
 Szemeimből csordulnak.
 Mint nap eső után,
 Így bánatim után
 Örömim megújulnak.

;Kodály Zoltán: Psalmus Hungaricus, 1923
>263
/1
#EE43F0EE
 Mikoron Dávid nagy búsultában,
 Baráti miatt volna bánatban,
 Panaszolkodván nagy haragjában,
 Ilyen könyörgést kezde ő magában:
/2
#5B22DE2D
 Istenem, Uram, kérlek tégedet:
 Fordítsad reám szent szemeidet;
 Nagy szükségemben ne hagyj engemet,
 Mert megemészti nagy bánat szívemet.
/3
#C2855AB3
 Csak sírok-rívok nagy nyavalyámban,
 Elfogyatkoztam gondolatimban,
 Megkeseredtem nagy búsultomban,
 Ellenségemre való haragomban.
/4
#00E5CAA4
 Hogyha énnékem szárnyam lett volna,
 Mint a galamb, elrepültem volna;
 Hogyha az Isten engedte volna,
 Innét én régen elfutottam volna.
/5
#ED0FCAA6
 Akarok inkább pusztában laknom,
 Vadon erdőben széjjel bujdosnom,
 Hogynem mint azok között lakoznom,
 Kik igazságot nem hagynak szólanom.
/9
#97E40122
 Egész e város rakva haraggal,
 Egymásra való nagy bosszúsággal;
 Elhíresedett a gonoszsággal,
 Hozzá fogható nincsen álnoksággal.
/10
#C4251C1A
 Gyakorta köztünk gyűlések vannak;
 Özvegyek, árvák nagy bosszút vallnak.
 Isten szavával ők nem gondolnak,
 Mert jószágukban felfuvalkodtanak.
/15
#90E6BD56
 Én pedig, Uram, hozzád kiáltok,
 Reggel és délben s estve könyörgök;
 Megszabadulást tetőled várok,
 Az ellenségtől mert én igen tartok.
/18
#BC3F9805
 Te azért, lelkem, gondolatodat,
 Istenbe vessed bizodalmadat;
 Rólad elvészi minden terhedet
 És meghallgatja te könyörgésedet.
/19
#8F60FBC6
 Igaz vagy, Uram, ítéletedben:
 A vérszopókat ő idejükben
 Te meg nem áldod szerencséjükben;
 Hosszú életük nem lészen a földön.
/20
#35035608
 Az igazakat te mind megtartod,
 A kegyeseket megoltalmazod,
 A szegényeket felmagasztalod,
 A kevélyeket aláhajigálod.
/21
#1ED053E3
 Ha egy kevéssé megkeseríted,
 Az égő tűzbe el-bétaszítod:
 Nagy hamarsággal onnét kivonszod,
 Nagy tisztességre ismét felemeled.
/22
#A3BFD08F
 Szent Dávid írta a Zsoltárkönyvben,
 Ötvenötödik dicséretében,
 Melyből a hívek, keserűségben,
 Vigasztalásért szerzék így versekben.

;Stralsund, 1665
>264
/1
#0623B3BC
 Áldjad, én lelkem, a dicsőség erős királyát!
 Őnéki menynyei karokkal együtt zengj hálát!
 Zúgó harang, Ének és orgonahang,
 Mind az ő szent nevét áldják!
/2
#5C469F47
 Áldjad Őt, mert az Úr mindent oly szépen intézett!
 Sasszárnyon hordozott, vezérelt, bajodban védett.
 Nagy irgalmát Naponként tölti ki rád:
 Áldását mindenben érzed.
/3
#BCD7CBD1
 Áldjad Őt, mert csodaképpen megalkotott téged,
 Elkísér utadon, tőle van testi épséged.
 Sok baj között Erőd volt és örömöd:
 Szárnyával takarva védett.
/4
#64B36958
 Áldjad Őt, mert az Úr megáldja minden munkádat,
 Hűsége, mint az ég harmatja, bőven rád árad.
 Lásd: mit tehet Jóságos Lelke veled,
 És hited tőle mit várhat.
/5
#A2B2B94F
 Áldjad az Úr nevét, Őt áldja minden énbennem!
 Őt áldjad, lelkem, és Róla tégy hitvallást, nyelvem!
 El ne feledd: Napfényed Ő teneked!
 Őt áldjad örökké! Ámen.

;Gesius B., 1603
>265
/1
#37DDDDC9
 Hagyjad az Úr Istenre
 Te minden utadat,
 Ha bánt szíved keserve,
 Ő néked nyugtot ad.
 Ki az eget hordozza,
 Oszlat felhőt, szelet,
 Napját rád is felhozza,
 Atyád ő, áld, szeret.
/2
#75493D49
 Az Úrra bízzad dolgod:
 Könnyebbül a teher;
 Ezer baj közt is boldog,
 Aki nem csügged el.
 Minek a gond, a bánat?
 Mit gyötröd lelkedet?
 Az Istent kérjed, várjad,
 S megnyered ügyedet.
/3
#1A56A731
 A Te irgalmasságod
 Van rajtam, Istenem,
 Te jól tudod, jól látod,
 Hogy mi használ nekem.
 Sorsomat úgy intézed,
 Amint te akarod;
 Bölcs a te végzésed,
 Ha áld, ha sújt karod.
/4
#5E5E94BD
 Utad van számtalan sok,
 Uram, és eszközöd;
 Reánk is szent áldásod
 Bőséggel öntözöd.
 Művednek akadálya,
 Szünetje nincs soha;
 Úgy téssz, amint kívánja
 Gyermekeid java.
/5
#7D4B056E
 Bízzál, bánatos lélek!
 Mit bánt a bú, a gond?
 Él még, ki annyi vészek
 Torkából már kivont.
 Bajaidból kiment ő,
 Szűnnek keserveid;
 Rád még a jó Teremtő
 Víg napot is derít.
/6
#C04C6049
 Őbenne vesd halálig
 Jó reménységedet:
 Ő biztos révbe szállít
 A bajból tégedet.
 Bár késik a segítség
 És nem találsz vigaszt:
 Eloszlik gond és kétség
 Előbb, mint véled azt.
/7
#A548402C
 Ő megcselekszi végre
 Velünk azt, ami jó;
 Ösvényünk erőssége
 Te vagy, Mindenható!
 Bár nehéz földi pályánk,
 Könny lepi és tövis,
 De örök pálma vár ránk:
 Utunk a mennybe visz.

;Bremen, 1680
>266
/1
#900F644B
 Egek nagy Királya, Magasztalunk téged,
 Térjen nevedre dicséret! Mulandó és gyarló
 Életünk átjárja Végtelen kegyelmed árja.
 Jöjj, segíts, Jóra ints; Rólad zengjen nyelvünk,
 Vígan énekeljünk!
/2
#E20CE677
 Dicsérjed, világom, Alkotód hatalmát, Mindenek felett uralmát!
 Hirdesse a csillag Ránk sugárzó fénye, Mily csodás az Isten lénye!
 Nagy égbolt, Nap és hold, Kik ott fennen jártok:
 Ti is Őt áldjátok!
/3
#FC99CC2E
 Zengjen minden szívben Boldog hálaének A világ Teremtőjének!
 Nagy kegyelmessége Gondot visel rólunk, számon tartja minden dolgunk.
 Égben fenn, Földön lenn Áldassék jósága,
 Nagy irgalmassága!
/4
#DDC4A862
 Halleluját zengve, Magasztalva áldjad Krisztus által jó Atyádat!
 Ha az Istent élő, Igaz hittel féled És a szíved Jézusé lett:
 Boldogság, Üdv vár rád Fenn az ég honában,
 Hófehér ruhában.

;B. Monoetius Cranachensis, 1565
>267
/1
#126BE8D0
 Mire bánkódol, ó, te, én szívem,
 Mit töröd magad ilyen igen?
 Ím, a testi jókért
 Bízzál csak az Úr Istenben,
 Ki uralkodik menynyekben.
/2
#CA7D7CB5
 A nagy Úr Isten nem hágy tégedet,
 Tudja ő minden szükségedet,
 Mind e világ övé;
 Azért minden szükségedben
 Bízzál csak az Úr Istenben.
/3
#0EB150B7
 Én Uram és én Istenem te vagy:
 Szegény szolgádat, engem ne hagyj,
 Ó, én édes Atyám!
 Én te szegény fiad vagyok,
 Senkiben bízni nem tudok.
/4
#EBDBB22C
 A gazdag bízik az ő kincsében,
 De én bízom csak az Istenben,
 Noha szegény vagyok,
 Mert aki őbenne bízik,
 Soha meg nem csalatkozik.
/5
#E1D9B8B4
 Nem kérek, Uram, én gazdagságot,
 Ami szükségem, te jól tudod,
 Mert mindeneket látsz,
 De főképpen arra kérlek,
 Uram, hogy teveled légyek.
/6
#1F6D8FC1
 Minden, ami e világon vagyon,
 Ezüst vagy arany s egyéb vagyon,
 Akármi légyen az,
 Csak kevés ideig tarthat,
 És idvességet nem adhat.
/7
#79070361
 Nagy hálát adok, Uram, teneked,
 Értenem hogy adtad ezeket
 A te szent igédből.
 Adj mindvégig megmaradást.
 És idvességes kimúlást!

>268
/1
#8FF0EB0C
 Dicsérjük hálaadással
 Az egy fővalóságot,
 Ki hűséges gondtartással
 Vezérli e világot.
 Megtartatik általa,
 Amit teremtett vala,
 Ami kezdetben volt, ma is
 Megvan, és nincs egy híja is.
/2
#224027EE
 Tud, mint tőle készült művet
 Minden dolgot egyenként,
 Szálanként lát minden füvet
 Minden port szemenként.
 Úgy vigyáz az őszeme
 Fö]di teremtésire,
 Hogy mind megelégül épen,
 Magához illendőképen.
/3
#F8E9E054
 E világnak történeti
 Lesznek akaratjából,
 Nem a szerencse vezeti
 Azokat csak vaktából,
 Hanem úgy mennek végbe
 A földön és az égbe',
 Amint szájából kijőnek
 Az egy fő gondviselőnek.
/4
#1F0BF2ED
 Ha hát tőle kívánatos
 Jókat áldásul vészek,
 Jóvoltának háládatos
 Megismerője lészek;
 Ha rám nyavalyát ereszt,
 Rajtam lévén a kereszt,
 Bár, hogy könnyítsem, nincs módom:
 Ellene nem zúgolódom.

;Bourgeois L., Genf, 1551
>269
/1
#94F95354
 Istenre bízom magamat,
 Magamban nem bízhatom;
 Ő formált, tudja dolgomat,
 Lelkem ezzel biztatom.
 E világ szép formája
 Az ő keze munkája.
 Mit félek? - mondom merészen:
 Istenem és Atyám lészen.
/2
#FFD2C531
 Öröktől fogva ismerte,
 Hogy mire lesz szükségem,
 Éltem határát kimérte,
 Szükségem s elégségem.
 Lelkem, hát ne süllyedezz,
 A hitben ne csüggedezz!
 Egy kis bajt nem győznél-e meg?
 Hogy tántorítana ez meg?
/3
#89CE17EA
 Tudja Isten kívánságod,
 Ád is, mert csak ő adhat,
 De bölcs, Uram, te jóságod,
 Tudod, sok elmaradhat.
 Tudom: gondod reám nagy,
 Mivel édes Atyám vagy;
 Mint akarod, hát úgy légyen!
 Másként hinnem volna szégyen.
/4
#717CC6C3
 A valóságos igaz jót
 Az Úr meg nem tagadja;
 Nagy gazdagság és rakott bolt
 Nem fő jó, ritkán adja.
 Ki az Isten tanácsát,
 Megszívleli mondását,
 Azt ő Lelkével serkenti,
 Gondját is megédesíti.
/5
#12B6DA3E
 E világnak dicsősége
 Igen hamar elmarad,
 Kit ma gondok sújtnak, végre
 Holnap diadalt arat.
 Csak Atyámban bízhatom,
 Ő megsegít, jól tudom,
 Mert az igazaknak Atyja
 Hű szolgáját el nem hagyja.

;Hintze J., Berlin, 1670
>270
/1
#EA3363C5
 Légy csendes szívvel és békével
 Életednek Istenében!
 Ő bír örömnek bőségével,
 Véle boldogulsz mindenben,
 Ő kútfőd és ő fényes napod,
 Őtőle jön minden vígságod:
 Légy csendes szívvel!
/2
#999C100F
 Ő bír vígsággal, kegyelemmel,
 Oly szívvel, mely feddhetetlen,
 Ahol ő van, nem sért semmivel,
 Hordozzon bár mélységekben;
 Sok nyavalyádat elfordítja,
 A halált is kezében tartja:
 Légy csendes szívvel!
/3
#F1819C22
 Mint légyen néked s másnak dolga,
 Az nála nincsen elrejtve,
 Arra néz szemének világa,
 Ki búval van megterhelve.
 Ő számlálja fohászkodásid,
 Összefogja könnyhullatásid:
 Légy csendes szívvel!
/4
#2C405499
 Ha egy sem volna e világon,
 Kire magad rábízhatnád,
 Akkor is állna oldaladon,
 És hűségét tapasztalnád.
 Ő tudja titkos bánatidat,
 Elvenni mikor kell azokat:
 Légy csendes szívvel!
/5
#5158DCAE
 Ő hallja lelked óhajtásit,
 Mit nem mernél megmondani,
 Néki szíved titkos panaszit
 Éppen bízvást megvallani.
 Nincs messze ő, sőt itt áll köztünk,
 Meghallja s adja, amit kérünk:
 Légy csendes szívvel!
/6
#BA1C2556
 Ki a mezei madaraknak
 Megadja eledelüket,
 Ki juhoknak s egyéb barmoknak
 Szépen tartja életüket,
 Ő téged, egyet eltartani,
 Éhség ellen meg tud menteni:
 Légy csendes szívvel!
/7
#D6B7640A
 A segítség ha kissé késik,
 De ugyan csak eljő végre,
 A várakozás, rosszulesik,
 De majd válik üdvösségre;
 Ami lassan jő, bizonyosabb,
 Ami késik, kívánatosabb:
 Légy csendes szívvel!
/8
#EAA326F9
 Ne vedd szívedre, az ellenség
 Amit rád költ hazugsággal.
 Hadd jöjjön csúfság, keserűség:
 Ítél az Úr igazsággal.
 Ha Isten a te jó barátod,
 Nem lehet emberektől károd:
 Légy csendes szívvel!
/9
#95B65ACB
 Hogy is lehetne másként dolgunk?
 Szenvedni kell az embernek,
 Amíg e földnek útján járunk,
 Csak nyavalyák környékeznek.
 A kereszt botja bárha terhel,
 Azt is letesszük életünkkel:
 Légy csendes szívvel!
/10
#DF910076
 Van Isten népének szombatja,
 Akkor az Úr megszabadít,
 Ínség kötelét elszaggatja,
 Teljes szabadságra állít,
 S az idvezültek seregében
 Vég nélkül élünk békességben:
 Légy csendes szívvel!

>271
/1
#D491EDC4
 Mint Isten akarja, legyen,
 Szent az ő végezése.
 Kész segítségükre megyen.
 Kik benne bíznak végre.
 Ínségekből, Kegyelméből
 Megsegít, sújt mértékkel:
 Ki benne bízik, nem esik
 Szégyenbe, és nem vész el.
/2
#CF4F6E9C
 Az Úr vígságom s reményem,
 Bizodalmam s életem,
 Mint akarja, mind úgy légyen,
 Megnyugszik abban hitem.
 Igaz szava: hajam szála
 Nála számba vétetett,
 Vigyáz, őriz, sem tűz, sem víz
 Nékem hát kárt nem tehet.
/3
#05321355
 Azért bízvást e világból,
 Amint az Úrnak tetszik,
 Elmégyek akaratjából:
 Mindenkor jómra válik.
 Néki adom és ajánlom
 Végső órán lelkemet,
 Bűnön, poklon és halálon
 Jézus vett győzedelmet.
/4
#5E492CE8
 Uram, még ez egyre kérlek,
 Ne tagadd ezt meg tölem:
 Ha ostromol gonosz lélek,
 Ó, ne hagyj megcsüggednem!
 Segélj engem, én Istenem,
 Neved dicsőségére,
 Ki ezt hiszi, az megnyeri,
 Ámen-t hát mondok erre.

;Gastorius Severus, Weimar, 1681
>272
/1
#FE411E05
 Mind jó, amit Isten tészen,
 Szent az ő akaratja,
 Ő énvélem is úgy tégyen,
 Mint kedve néki tartja.
 Ő az Isten, Ki ínségben
 Az övéit megtartja,
 Hát légyen, mint akarja.
/2
#A12ACB70
 Mind jó, amit Isten tészen,
 Ő engemet meg nem csal,
 De igaz ösvényen viszen,
 Én megelégszem azzal,
 Hogy kedvében, Kegyelmében
 Ő forgatja dolgomat,
 Csak rá hagyom magamat.
/3
#C9D55C5F
 Mind jó, amit Isten tészen,
 Ő engem meg nem utál,
 Mint jó orvosom, úgy tészen,
 És mérget ő nem kínál.
 Orvosságot, Boldogságot
 Énnékem készít, tudom,
 Azért csak benne bízom.
/4
#1FAFAA51
 Mind jó, amit Isten tészen,
 Ő az én idvességem,
 Ő velem rosszul nem tészen,
 Rábízom egész éltem.
 Örömömben, Keresztemben
 Mind nyilván megmutatja,
 Hogy javamat akarja.
/5
#D3BEEFD5
 Mind jó, amit Isten tészen:
 Ha oly pohárt innék is,
 Amelynek íze szívemnek
 Nagy-keserűn esnék is,
 De eltűröm, Mert víg öröm
 Felváltja ezt végtére,
 Sok búm enyhítésére.
/6
#DEFD3967
 Mind jó, amit Isten tészen,
 Mind örökké ezt vallom,
 Ha rajtam bú, bánat lészen
 S kell bosszúságot látnom.
 Mindazáltal Megvigasztal,
 Mint édes Atyám, engem,
 Mert csak ő segítségem.

;Debrecen, 1774
>273
/1
#39E9F1C1
 Az Úr Istent magasztalom,
 Jó voltáról emlékezem,
 Mindig hozzá folyamodom,
 Mert meghallgat, azt jól tudom.
/2
#BC73DAE7
 Számtalan kínokban valék,
 Bűneimért kit érdemlék:
 De ismét megkönnyebbülék,
 Mihelyt vigasztalást hallék.
/3
#BCFCA91F
 Elterjesztvén kezeimet,
 Kiáltottam Istenemet,
 Csak bé sem hunytam szememet;
 Nem leltem sehol helyemet.
/4
#37A9041C
 Magamban én dolgaimról,
 Elébbeni életemről,
 Nagy kedves nyájasságomról:
 Gondolkodtam énekimről.
/5
#83BB02AC
 Nagy erősen, igaz hittel,
 Magam Isten beszédével,
 Kegyelmes ígéretével,
 Bátorítám Szentlelkével.
/6
#7C4C13BB
 Csuda irgalmasságodat,
 Hiszem, Uram, hatalmadat,
 Onnan vészen bizodalmat,
 Én lelkem minden oltalmat.
/7
#3EA292EB
 Kezeidnek karjaiban,
 Élet, halál birtokodban,
 Megmutatád, hogy markodban,
 Vagyon minden oltalmadban.
/8
#FBFBF3BE
 Erős vitéz mint népeit,
 Az ember ő két szemeit,
 Mint jó pásztor ő juhait:
 Úgy oltalmazza híveit.
/9
#E27DCC62
 A híveknek számuk vagyon,
 Nevük nála írva vagyon,
 Hajuk szála számon vagyon:
 Rájuk gondja van oly nagyon.
/10
#62477CFD
 Ne félj azért háborúdban,
 Ó, én lelkem, nyavalyádban;
 Erős légy bizodalmadban,
 Mert vagy Isten oltalmában.

;Neumark György, 1657
>274
/1
#219FA0C1
 Ki Istenének átad mindent,
 Bizalmát csak beléveti,
 Azt csudaképpen őrzi itt lent,
 Ínség, baj közt is élteti.
 Ki mindent szent kezébe tett,
 Az nem fövényre épített.
/2
#816F1324
 A súlyos gondok mit használnak,
 A sóhaj, sok jajszó mit ér,
 Ha sebeink még jobban fájnak
 S mindennap kínunk visszatér?
 Így terhünk egyre súlyosabb,
 Ha lelkünk búnak helyet ad.
/3
#536087A7
 Csak légy egy kissé áldott csendben:
 Magadban békességre lelsz,
 Az Úr rendelte kegyelemben
 Örök, bölcs célnak megfelelsz.
 Ki elválasztá életünk,
 Jól tudja, hogy mi kell nekünk.
/4
#ADD969BC
 Zengj hát az Úrnak s járd az utat,
 Mit éppen néked Ő adott;
 A mennyből gazdag áldást juttat
 S majd Jézus ád szép, új napot.
 Ki Benne bízik és remél,
 Az mindörökké Véle él.

;Prágai graduale, 1567
>275
/1
#4A954B13
 Az Úr Isten az én reménységem,
 Erősségem Mindenféle ínségben;
 Csak tőle várom Igaz boldogságom:
 S meg is találom.
/2
#F817AF89
 Benne élek, haláltól nem félek:
 Jót reménylek, Hogy tőle el nem térek;
 El nem enyészem A sírban egészen:
 Mennyben lesz részem.
/3
#248354A9
 Semmi engem tőle el nem választ,
 Jól tudván azt, Hogy sok jóval eláraszt;
 Erősít engem Erőtlenségemben
 És szükségemben.
/4
#9BDCDA3A
 Azért egész életem fogytáig
 Csodálom itt Szent kezének munkáit,
 S intem lelkemet: Áldjad Istenedet,
 Üdvözítődet.
/5
#8038251D
 Oltalmazzad, Uram, egyházadat,
 Szent nyájadat, mely vallja szent Fiadat,
 Ki bűneinkért Ártatlan bárányként
 Szenvedett sok kínt.
/6
#615D90DA
 Hogy e földön szent gyülekezeted
 Dicséretet Zengvén, áldja nevedet,
 Míg szemtől szemben Magasztalunk mennyben
 Mind egyetemben.

;Kálmán Farkas, 1838-1906
>276
/1
#3998A368
 Egyedüli reményem,
 Ó, Isten, csak te vagy;
 Jövel és nézz meg engem,
 Magamra, ó, ne hagyj!
 Ne légy tőlem oly távol,
 Könyörülj hű szolgádon,
 Úr Isten, el ne hagyj!
/2
#F2663AA1
 Ha a nehéz időkben
 Elcsügged a szívem,
 Vigasztalást igédben,
 Uram, te adj nekem!
 Ha kétség közt hányódom
 És mentségre nincs módom,
 Te tarts meg, Istenem!
/3
#D1C28074
 A földön ha elvesztem
 Szerelmem tárgyait,
 Maradjon meg mellettem
 Szerelmed és a hit;
 Csak azt el ne veszítsem,
 Mi benned, ó, Úr Isten,
 Remélni megtanít!
/4
#CF88F176
 Földi jó és szerencse
 Mulandó, mint magunk,
 De a hit drága kincse
 Örök és fő javunk;
 Hitünk áll rendületlen,
 Hogy Isten véd szüntelen:
 Élünk vagy meghalunk.
/5
#FD13A958
 Uram, a nyomorultat, a gyöngét el ne hagyd,
 Az árvát, elhagyottat
 Gyámolítsd te magad!
 A szegényt, ki remélve
 Csak reád néz az égre:
 Úr Isten, el ne hagyd!

;Crüger J., 1649
>277
/1
#E47B6E20
 Gondviselő jó Atyám vagy,
 Ó, én édes Istenem!
 Hozzád vágyom, benned élek,
 Üdvöt mástól nem remélek.
 Látom én, hogy minden elhagy
 E világon, csak te nem!
/2
#B864BBD4
 Mint az alélt bús virágra
 Megújító harmatot:
 Vérző szívem fájdalmára
 Csak te hintesz balzsamot.
 Könnyű sorsom terhe rajtam,
 Ha imára nyílik ajkam.
/3
#B2FB7C88
 Rám-rám derül ismeretlen
 Utamon egy kis öröm,
 Azt is a te véghetetlen
 Jóságodnak köszönöm;
 Hálakönnyem tündöklése
 A te neved hirdetése.
/4
#F514565B
 Gyenge vagyok, lankadoznak
 Buzgóságom szárnyai,
 Bármily híven vágyakoznak
 Színed elé szállani;
 Ó, adj erőt, hogy míg élek,
 Egyedül csak néked éljek!
/5
#FEB18DDE
 Ó, add, hogy ha majd bevégzem
 E mulandó életet,
 Lelkem tisztán és egészen
 Egyesüljön teveled.
 El ne vonjon semmi többé,
 Tied legyek mindörökké!

;Debrecen, 1774
>278
/1
#9E69DD3D
 Dicsőült helyeken, menynyei paradicsomban,
 Akik vigadoztok véghetetlen boldogságban,
 Szent és ártatlan állapotban;
 Az Úrnak nevét énekszóban,
 Magasztaljátok vigasságban.
/2
#8ADABB0E
 Lássátok, ti dicső angyali lelkek az égben,
 A fénylő mennyei testek roppant seregében:
 Mint uralkodik dicsőségben
 Az Úr, és őtet felségében
 Imádjátok tiszta szentségben!
/3
#CD45EA27
 Nap, hold, csillagos ég!
 Ti az Urat dicséritek,
 Jóságát, hatalmát, bölcsességét hirdetitek;
 Mert amely rendet ő tinéktek
 Kiszabott, az a ti törvénytek
 És ti annak híven engedtek.
/4
#7B25F5A4
 Tenger, föld, szigetek s a hegyek pompás szépsége,
 Erdők és ligetek, folyóvizek kiessége,
 Sík mezők, rétek ékessége,
 Vetések kellemetessége:
 Mind a teremtő dicsősége!
/5
#86FC9565
 Fák, bokrok, csemeték, gyümölcsöket hozó kertek,
 Pompás liliomok s minden virági szépségek,
 Színekkel kihímezett rétek,
 Apróbb és nagyobb növevények:
 A gondviselést hirdessétek.
/6
#F3A0C8C0
 E földön lakozó minden állat boldogsága,
 A szárnyon repdeső madaraknak vidámsága,
 A halak s férgek sokasága,
 Egyszóval az élők világa
 Mondja: mily nagy az Úr jósága!
/7
#8D8FFB75
 Bölcs Isten remeki!
 Kik az ő képei vagytok
 És őtőle lehelt emberi lélekkel bírtok:
 Az örökkévalót lássátok
 Munkáiban és imádjátok;
 Egy szívvel és szájjal mondjátok:
/8
#9FB83633
 Dicséret, dicsőség, tisztesség és hálaadás,
 A szentek Urának légyen örök magasztalás!
 Kiben soha nincs megváltozás,
 Vagy ígérettől elhanyatlás:
 Tőle fejünkre szálljon áldás!

>279
/1
#1A858CBC
 Vígak e föld lakosai,
 Mert megnyíltak ajtajai
 Az Úr bó tárházának,
 Víg a magvető, hogy néki
 Orcájának verejtéki
 Nem hiában hullának,
 Mert ahol hintett marokkal,
 Most kiterjesztett karokkal
 Hordja kévéi sorát;
 Aki segitségül hiván
 Az Urat, dolgozni kiván,
 Nyer tőle ily adományt.
/2
#BFFAA308
 Ezért, ó jóltévő Atyánk,
 A mi szívünk, nyelvünk és szánk
 Csak tégedet magasztal:
 Hogy ékeskedik a mező
 Gyümölccsel és az éhezö
 Számára készül asztal.
 Mert amit más nem adhatott,
 Hintél esőt és harmatot
 A szomjúzó földekre,
 Napodat az igazaknak,
 Valamint a hamisaknak
 Felhoztad mezejükre.
/3
#F756A38E
 Tartsd meg, Uram, a munkásnak
 Erejét az aratásnak
 Örvendetes idején,
 Oltalmazd meg kezünk között,
 Amit a föld gyümölcsözött
 Kinek-kinek mezején,
 És adjad, hogy mi azokkal
 A tőled vett áldásokkal
 Háládatosan éljünk;
 Akár iszunk, akár eszünk,
 Vagy akármit cselekeszünk,
 Jóvoltodról beszéljünk.
/4
#DA805DDF
 Ha a kenyeres kosárok
 Bételnek, mi mint sáfárok,
 Úgy tartsuk a magunkét.
 Marhánkat el ne tagadjuk,
 Sőt abból önként kiadjuk
 A felebarátunkét.
 Add tudtunkra, hogy el nem vész
 Az ajándék, sót az a rész
 Marad örökre nálunk,
 Melyet fordítunk a végre,
 Hogy másokat, ha szükségre
 Jutnak, abból tápláljunk.

;Genf, 1562
>280
/1
#2BBEE0FA
 Ismét egyik esztendeje,
 Istentől kimért ideje
 Telék el a mulandóságnak:
 Az égi testek órája
 Lefolyván, lett új példája
 A közös változandóságnak;
 Elmúlt vége, mint kezdete,
 Már van csak emlékezete.
/2
#D35B7752
 Boldog, ki csendes lélekkel,
 És nem könnyező szemekkel
 Tekinthet vissza folyására,
 Ki, ha magát megkérdezi,
 Belső örömmel érezi,
 Hogy szolgált ez jobbulására,
 Hogy ment mind az ismeretben
 Előbb, mind a szeretetben.
/3
#A7198E4D
 Boldog, kinek nem kell szánni
 Elvesztett idejét s bánni
 Megbecsülhetetlen óráit,
 Kinek hív emlékezeti
 Elébe nem helyezteti
 Helyrehozhatatlan hibáit,
 Ki az időt megbecsülte,
 A hivalkodást kerülte.
/4
#E76CEDC3
 De vajon melyik halandó
 Volna annyira állandó
 A jóban, aki meg ne esne?
 Kinek volna olyan szíve,
 Hogy a kísértésnek íve
 Rajta oly nyílást ne keresne,
 Melyen lopva hozzá férjen
 És az elevenig érjen?
/5
#0C9BDEB6
 Tökéletességnek Atyja,
 Az öröm szívemet hatja,
 Valahányszor azt elgondolom,
 Mely sok jókat vettem tőled!
 De elrejtőzném előled,
 Ha meg másrészről megfontolom:
 Azokkal mily rosszul éltem,
 Elvesztegetni nem féltem.
/6
#3520B771
 Minthogy azért te teheted,
 Véghez egyedül viheted,
 Hogy a jót akarjam és tégyem:
 Engedd, hogy a múlt esztendőt,
 Mint már vissza nem jövendőt,
 Magamnak tükörül felvégyem;
 A jót benne követhessem,
 A rosszat elkerülhessem.

>281
/1
#4372C5B7
 Lelkünk, testünk teremtője, kegyes gondviselője,
 Imádjuk jóságodat
 És irgalmasságodat,
 Hogy minden elfolyt időnkben,
 Közelebb ez esztendőnkben
 A te kezed béfedezett,
 Gonoszoktól védelmezett.
/2
#3635EECE
 Szent igédet hirdettetted,
 Mi lelkünket legeltetted,
 Adtál jó békességet,
 A jóra tehetséget;
 Ételt, italt, ruhát adtál,
 Egészséget szolgáltattál,
 Es az esztendő végére,
 Juttattál várt estvéjére.
/3
#CFF6CCD7
 Uram, a te kegyelmednek
 És hű gondviselésednek
 Ily sok ajándékiért,
 Drága adományiért
 Teljes szívünkből, lelkünkből
 És minden tehetségünkből
 Áldjuk te istenségedet,
 Nyújtsad erre kegyelmedet.
/4
#69D528F7
 Jézus szent áldozatjáért
 És érdemes haláláért
 Töröld el bűneinket
 És minden vétkeinket,
 Melyeket eltölt időkben,
 Közelebb ez esztendőnkben
 Cselekedtünk és gondoltunk,
 Vétkes nyelvünkkel szólottunk.
/5
#953ABA95
 Urunk, újítsd meg szívünket,
 Világosítsd értelmünket,
 Adj a jóra kegyelmet,
 Vezérlő segedelmet,
 Hogy új életet élhessünk,
 Végre te hozzád mehessünk,
 Szent Felségednek elébe,
 A boldogok seregébe.

;Reformáció előtti dallam, Debrecen, 1774 szerint
;
>282
/1
#1A3889D5
 Nékünk születék mennyei király,
 Idvezítőnek kit mond az angyal:
 Új esztendőben mi vigadjunk,
 Született Jézust mi imádjuk!
/2
#6079115E
 Régen megírák ezt a próféták,
 Hogy fiat szül majd egy nemes virág:
 Új esztendőben mi vigadjunk,
 Született Jézust mi imádjuk!
/3
#891DD5E1
 Szűz Máriától gyermek születék,
 De Szentlélektől ő fogantaték:
 Új esztendőben mi vigadjunk,
 Született Jézust mi imádjuk!
/4
#C7992C47
 Nagy hatalmas lőn e kisded gyermek,
 Megtöretének sok ellenségek:
 Új esztendőben mi vigadjunk,
 Született Jézust mi imádjuk!
/5
#F906FCD6
 Győzedelmet vőn a kárhozaton,
 Győzedelmet vőn örök halálon:
 Új esztendőben mi vigadjunk,
 Született Jézust mi imádjuk!
/6
#4A6D0CF7
 Nincsen a bűnnek hatalma rajtunk,
 Kiknek e gyermek lészen oltalmunk:
 Új esztendőben mi vigadjunk,
 Született Jézust mi imádjuk!
/7
#A4D0EF3A
 Adjunk nagy hálát az Úr Istennek
 És idvezítő Jézus Krisztusnak:
 Új esztendőben mi vigadjunk,
 Született Jézust mi imádjuk!
/8
#F2C675A9
 Dicsőíttessék a Szentháromság,
 Adja minékünk szent ajándékát:
 Új esztendőben mi vigadjunk,
 Született Jézust mi imádjuk!

>283
/1
#4D75CA25
 Az Úrnak jóvolta napjainkhoz napokat told,
 Melyeket bizonyos részekre oszt a nap és hold;
 Nyomain e két vezetőnek
 Tél, tavasz, nyár és ősz eljőnek,
 Engedvén a bölcs Teremtőnek.
/2
#ABE569B7
 Jó Atyánk, az elmúlt esztendő minden szakasza
 Jóságod tanúja, szívünknek nincs rád panasza;
 Bizony, az Úr minket kedvellett,
 Mert lelkiadományi mellett
 Megadta, ami nékünk kellett.
/3
#5CAD70A5
 Úr Isten, ki minket sok áldásiddal töltél be,
 Ez új esztendővel jó kedved ne szakaszd félbe
 Áldd meg kezdetét s végét ennek,
 Hogy midőn napjai lemennek,
 Mondhassuk : dicsőség Istennek I
/4
#39547370
 Vigyázz híveidre s hab közt hánykódó hajódra,
 Vigyázz országunkra s minden elöljáróinkra,
 Akik népedet úgy vezessék,
 Hogy igazság és jó békesség
 Egymást csókolva ölelgessék.
/5
#C991C94F
 E gyülekezeten, mely e helyre telepedett,
 Könyörülj, Úr Isten,
 Bővítsd rajta kegyelmedet,
 Áldd meg nagyjait, kicsinyjeit,
 Mind köz, mind tanácsos rendjeit;
 Töröld el a sírók könnyeit!
/6
#597EDB74
 Adj vidám órákat, ha nekünk azt jónak látod,
 Békességes türést, ha vessződet ránk bocsátod;
 Ha több esztendőt nem számlálunk
 És ma vagy holnap el kell válnunk:
 Add, hogy légyen boldog halálunk!

>284
/1
#E12E09FB
 Áldjuk a nagy Isten jóságát,
 Magasztaljuk irgalmasságát;
 Szívünkből szánk énekeljen,
 Dícséretével bételjen
 E mi kegyelmes jó Atyánknakl
/2
#CE9B05C7
 Mert megújítá esztendőnket,
 Meghosszabbította időnket,
 Virrasztván új szakaszára
 Es e nappal új napjára
 Világi rövid életünknek.
/3
#EA579EA5
 Ó, nagy Isten, hát mik vagyunk mi?
 Porok, és nincs érdemünk semmi.
 Nagy bűnösök, méltatlanok,
 Mert vétkeink számtalanok:
 Mégis kedvezél életünknek.
/4
#7CFECB0C
 Mivel fizessünk, Atyánk, néked,
 Kinek a menny felséges széked?
 Semmire nincsen szükséged:
 Elégséged, dicsöséged
 Magadban vagyon határ nélkül.
/5
#91B1CDF1
 Nálad legkedvesebb ajándék
 A szent életre való szándék;
 A jó szív és tiszta lélek,
 Ő, legfelségesebb Lélek,
 Nem útáltatik meg tetőled.
/6
#88A63EFB
 Teremts új szívet hát mibennünk,
 Hogy lehessen elődbe mennünk,
 Es nálad kedvet találjunk,
 Boldog napokat számláljunk
 E felderült új esztendőben.
/7
#391E38DD
 Lelki s testi áldások Atyja,
 Egeidnek gazdag harmatja
 Szálljon mennyből híveidre,
 Tulajdon teremtésidre,
 Kik tőled várnak minden jókat.
/8
#482B98BC
 A Jézusért, közbenjárónkért,
 Mi egyetlenegy szószólónkért
 Légy kegyelmes, kérésünket,
 Halld meg esedezésünket,
 Kik rádbizzuk minden ügyünket.

;Bourgeois L., Strasbourg, 1545
>285
/1
#39792376
 Örök Isten, kinek esztendők
 Nincsenek létedben,
 Jelen vannak múltak, jövendők
 Egy tekintésedben;
 Minket pedig, mihelyt születünk,
 Már a koporsó vár,
 A bűn miatt lefoly életünk a halál velünk jár.
/2
#CBBB916A
 Ó, mely sokan elaluvának
 A múlt esztendőbe!
 Kik nálunknál jobbak valának,
 Szálltak temetőbe,
 Minket pedig, kedvező Atyánk,
 Eddig takargattál,
 S ez új esztendőt derítvén ránk,
 Erre eljuttattál.
/3
#B6E7FF51
 Uram, a te lelked ereje
 Vélünk ily jót tégyen,
 Hogy jókedvednek esztendeje
 Ez az új is légyen;
 Atyai karoddal forgassad
 A mi dolgainkat,
 Dicsőségedre igazgassad
 Minden szándékinkat.
/4
#B4BCEE66
 Adj minékünk megújult szívet
 És új indulatot,
 Tehozzád mindenekben hívet
 És szent akaratot.
 Újítsd meg rajtunk a te képed,
 Mely áll szent életben,
 Hogy lehessünk választott néped,
 Élvén szeretetben.
/5
#CD90D223
 Ez esztendőt testi jókkal is,
 Uram, koronázzad,
 A szükséges mulandókkal is
 Házunk felruházzad;
 Földünket bő gyümölcsözéssel
 Munkálkodásunkra
 Áldd meg, hogy szükséges terméssel
 Szolgáljon javunkra.
/6
#7D6B1E3F
 Kegyelmed s áldásod újítsad
 Te szent egyházadon:
 Közelebbről azt szaporítsad
 Itt lévő nyájadon.
 Gátold hazánkban áradását
 A sok gonoszságnak,
 Fordítsd el eshető romlását
 Nemzetnek, országnak.
/7
#1D6208D8
 Tekintsd meg a szűkölködőket,
 Légy az árvák atyja,
 Vigasztalja a kesergőket
 Szád édes szózatja,
 Akik betegágyukban nyögnek,
 Vidámítsd lelküket;
 Ínségükben kik könyörögnek,
 Add meg kérésüket.
/8
#A396CA18
 Végre midőn mind megavúlunk
 Az esztendeinkkel,
 Egymás után mi is kimúlunk
 Emlékezetinkkel:
 Uram, sorsunkon könyörülvén,
 Vígy a dicsőségbe,
 Minden bűnünket eltörölvén,
 Újíts meg az égbe'!

;Debrecen, 1774
>286
/1
#1F93BED6
 Jer, dicsérjük az Istennek Fiát,
 A szép szűznek áldott szent magzatját,
 E világnak édes Megváltóját,
 Bűnösöknek kegyes szószólóját.
/2
#E58A469C
 Jézus Krisztus, kegyelmes Megváltónk,
 Úr Istentől adott tanítónk,
 Szent Atyáddal felséges megtartónk,
 Szentlélekkel mi megigazítónk!
/3
#A52A5D1D
 Téged vallunk hatalmas Istennek,
 Kezdet nélkül való természetnek,
 Szent Atyáddal egy örök Istennek:
 Szentlelkétől lelkeztél Istennek.
/4
#530896BE
 Felöltözél az emberi testbe,
 E világra jövél miközinkbe,
 Igaz hitet adál mi szívünkbe,
 Új világot gerjesztél lelkünkbe'.
/5
#F7DE8FF6
 Szent véreddel minket te megváltál,
 Bűneinkből szépen kitisztítál,
 Szentlélekkel megvilágosítál,
 És Atyádnak kedvébe juttattál.
/6
#C09D50A9
 A keresztfán érettünk meghaltál,
 Harmadnapra ismét feltámadtál,
 Mennyországban felmagasztaltattál,
 Tisztességgel megkoronáztattál.
/7
#83F353B9
 Azért vallunk téged királyunknak
 És örökké való főpapunknak,
 Isten előtt minden gyámolunknak,
 E világon minden oltalmunknak.
/8
#C40DE093
 Te vagy nékünk mi nagy igazságunk,
 Jámborságunk és ártatlanságunk,
 Te vagy nékünk szentségünk, váltságunk,
 Isten előtt örök boldogságunk.
/9
#E8D19F04
 Te vagy nékünk a mi reménységünk,
 Ez világon mi nagy tisztességünk,
 Isten előtt minden dicsőségünk,
 Mennyországban örök idvességünk.
/10
#F8C7726F
 Ó, Istennek drága kincstartója,
 Szentléleknek rajtunk nyugtatója,
 Nyomorodott árvák megtartója,
 Mi hitünknek megkoronázója.
/11
#E918B22F
 Ó, Istennek kedves áldozatja,
 Kit jó szemmel megtekint az Atya,
 És megmarad az ő jó illatja,
 Mindörökké kedves foganatja.
/12
#A06B8002
 Édes Jézus, Elődbe járulunk,
 Széked előtt arccal leborulunk,
 Keservesen tehozzád óhajtunk,
 És nagy sírván mennybe felkiáltunk.
/13
#10A64BAF
 Adjad nékünk a te Szentlelkedet,
 Terjesztessed a te szent igédet;
 Ismerjünk meg mindnyájan tégedet,
 Dicsérhessük a te szent nevedet.
/14
#5DE54BF7
 Tekints reánk most a magas mennyből,
 Látogass meg királyi székedből,
 Tégy jól vélünk kegyelmességedből,
 Ments meg minket keserűséginkből.
/15
#B01AF43C
 Vigasztald meg a mi siralminkat,
 Teljesítsd bé mi kívánságinkat;
 Mi is néked hálaadásinkat,
 Bémutatjuk mi áldozatunkat.
/16
#0605387D
 Édes Jézus, néked hálát adunk
 Jóvoltodért és felmagasztalunk,
 Ajándékot elődbe mutatunk,
 Dicséretet te nevedre mondunk.
/17
#964F5ED4
 Dicsértessék az Atya Úr Isten,
 Dicsértessék a Fiú Úr Isten,
 Dicsértessék a Szentlélek Isten,
 Szentháromság egy örök Úr Isten!

>287
/1
#BAC23B4D
 Járuljunk mi az Istennek szent Fiához,
 A mi idvezitö Urunk Jézus Krisztushoz,
 Mint kegyes és hatalmas szabaditónkhoz,
 Testi-lelki nyavalyánkban
 Csak egy vigasztalónkhoz.
/2
#E99C8134
 Ó irgalmas Megváltó Fiú Úr Isten,
 Ki országolsz szent Atyáddal földön és mennyen,
 Tehozzád esedezünk könyörgésünkben:
 Segitségül légy minékünk
 Mostani szükséginkbenl
/3
#9C6ED870
 Sok szüKségink jóllehet vannak minékünk,
 De most néked kiváltképpen ezen könyörgünk:
 Szentlelked erössége légyen mivelünk,
 Hogy az igaz tudományban
 ~pületet vehessünk!
/4
#9AC73A8B
 Mi elménknek homályát távoztassad el,
 Aján­dékozz engedelmes szívvel, lélekkel,
 Hogy a te szent igédet vegyük jó kedvvel,
 Tanulhassuk és éltünkben
 Kövessük nagy örömmel.
/5
#5D3C1196
 Öregbitsed, neveljed bennünk a hitet
 És az igaz szeretetet, adj reménységet;
 Adj minden jóra való igyekezetet,
 Tiszta lelki ismeretet és szentséges életet!
/6
#8D685C08
 Szent nevedet engedjed, nyilván vallhassuk,
 Az ördögöt s e világot megútálhassuk,
 És te szent országodba kivánkozhassunk,
 Felségeddel egyetemben
 Örökké vigadhassunk!
/7
#903C5DCA
 Légyen néked dicséret és nagy tisztesség,
 Ki Atyával s Szentlélekkel vagy egy Istenség,
 Felséges hatalmasság, örök fényesség.
 Téged illet mindörökké
 Véghetetlen dicsőség!

>288
/1
#663AD129
 E világnak fényessége,
 És szenteknek idvessége,
 Krisztus Jézus, egy reménye,
 Mennynek, földnek teremtője!
/2
#F66DCDD4
 Kegyetlen halált meggyőzéd,
 Ördög hatalmát el vévéd,
 Pokol torkát bérekesztéd,
 Bűnünket rólunk el vévéd.
/3
#C7BD27B5
 Te általad megváltattunk,
 Te általad szabadultunk,
 Te általad igazultunk,
 Te általad megtartattunk.
/4
#422DDFA4
 Alázatosan könyörgünk
 Buzgó szívből esedezünk:
 Légy minékünk segítségünk
 Mert jól látod, mely sok bűnünk.
/5
#391F4B3A
 Add Szentlelked ajándékát.
 Örök életnek jutalmát,
 Engeszteld Atyád haragját,
 Láthassuk ő szent irgalmát.
/6
#F122985A
 Te vagy mennyország kapuja,
 És idvességnek ajtaja,
 Bünösöknek szószólója,
 Es csak egy közbenjárója.
/7
#E028E3FC
 Dicsőség légyen Atyának,
 Tenéked, ő szent Fiának,
 És a mi Vígasztalónknak,
 A teljes Szentháromságnak.

>289
/1
#6D5BE048
 Zengj nyelvem ékesen,
 Az én édes Jézusomnak,
 Kegyes lelki orvosomnak
 Szólj szerelmetesen,
 Mert csak ő érdemel
 Áldást és minden szerelmet,
 Ki reám önt bő kegyelmet
 S kőszálra felemel.
/2
#832E5687
 Jézus, én reményem,
 Bűnös lélek bizodalma
 S hozzád térőknek oltalma
 Gyönyörü Napfényem!
 Kegyes vagy azokhoz,
 Kik szívből téged keresnek,
 És ismerős híveidnek
 Béköszönsz házukhoz.
/3
#3C336260
 Ó Jézus, élő kút,
 Szívemnek nagy édessége
 S elmémnek gyönyörűsége,
 Mely csupán hozzád fut;
 Tégedet keresni
 Legfőbb öröm és vígasság;
 Nincsen több lelki igazság,
 Mint téged követni.
/4
#F4D98B9A
 Jézus, kegyes király,
 Ki mindennek szivét bírod,
 S arra szent neved felirod,
 Kérlek, hozzám is szállj;
 Mert ha te elfoglalsz,
 Elszáll a világ szerelme,
 Tündöklik bennem az elme,
 Ebben sok jót találsz.
/5
#BB8F0BD2
 Kedves szereteted,
 Mert szent véredet jókedvből,
 Kiontád, hogy, mint kútfőből,
 Folyjék ismereted.
 Megnyerted, hogy lássuk,
 Szent Atyádnak dicsőségét,
 Es hogy lelkünk idvességét
 Tovább ne halasszuk.
/6
#082354EB
 Dicsérd hát minden nép
 Az Úr Jézust és szeressed,
 Buzgó szívedből keressed,
 Mert minden dolga szép.
 Ő a kegyelemnek és minden jónak kútfeje,
 Reménylő szívnek ereje,
 S oka az örömnek.
/7
#AAA44DFE
 Uram, hadd érezzem
 Hű szerelmednek bőségét,
 Hogy biztos jelenlételét
 Kincsül megszerezzem.
 Bár ne érdemeljem,
 De jóvoltod bizodalmát
 Nyújtsd, hogy ezen birodalmát
 Kérni merészeljem.
/8
#FD1B1AE0
 Csak téged kívánlak
 Valahol járok, vagy kelek,
 És végre, hogyha rád lelek,
 Magamnak foglallak.
 Jőjj azért s hűvösíts
 Jelenléted illatjával,
 Szentlélek áldozatjával
 Lelkemben megfrissíts.
/9
#32A9DC3B
 Tégedet dicsérlek,
 E világnak megváltóját,
 Kárhozatnak elrontóját,
 Mert szívből szeretlek.
 Valahol hát lészesz,
 Szívemmel utánad járok,
 Mit tőled vissza nem várok,
 Mert hozzád felvészesz.
/10
#FAD656D9
 Tiéd, Uram, az ég
 Minden ő teljességével,
 Hol országolsz dicsőséggel,
 Melyben nincs semmi vég.
 Ne hagyj Uram, kérlek,
 Vedd el szívemet magadhoz,
 Mert úgy állhatok jobbodhoz
 S örökké dícsérlek.
/11
#BDC2F60A
 Dicsérje és áldja
 Lelkem az Atya Úr Istent,
 Ki benned idvezít mindent,
 Ha szódat fogadja.
 Dicséret, dicsőség
 Légyen Szentlélek Istennek,
 Ki hitet adott lelkemnek,
 Hogy befogja az ég.

>290
/1
#EB653487
 Ó mi kegyelmes Krisztusunk,
 Ki vagy nékünk igazságunk,
 Békességünk és váltságunk:
 Kelljen, kérünk, imádságunk.
/2
#A1EBE1BE
 Mi szívünket, kegyes Isten,
 Igaz hitben, reménységben
 Kiváltképen erősítsed,
 Hogy bízhassunk csak tebenned.
/3
#8306737C
 Te isteni jóvoltodat,
 Ismerhessük hatalmadat,
 Melyet hozzánk megmutattál,
 Mikor érettünk áldoztál.
/4
#1DA7A82E
 Csak egyedül e világnak
 Viseléd terhét bajának,
 Nyerél nékünk idvességet
 És mennyei örökséget.
/5
#64961221
 Azért ily nagy szerelmedért,
 Kérünk, ne vess el bűnünkért,
 Ne vehessen ellenségünk
 Dühösséggel semmit tőlünk.
/6
#03F94B61
 Ó mi irgalmas Királyunk,
 És győzedelmes Hadnagyunk,
 Hozzád hitből folyamodunk,
 Tudjuk, meg nem fogyatkozunk.
/7
#6F5525AD
 Kérünk, ily nehéz ügyünkbe'
 Ne adj ellenség kezébe,
 Ne vesd bűnünket elődbe,
 Végy be inkább kegyelmedbe.
/8
#F01208DE
 Keresztyéni egyességben,
 Nevednek szentelésében,
 Szent igédnek értelmében,
 Tarts meg minket mindvégiglen.
/9
#711E3C13
 Bírjad lelki országodat,
 Véreden vett jószágodat,
 Hogy e siralom völgyében
 Légyünk csendesek szívünkben.
/10
#D2FF32FC
 Adj állandó egyességet,
 Lelki-testi békességet,
 Hogy megfutván itt a pályát,
 Vehessük el a koronát.
/11
#168D18EC
 Dícsértessél áldott Isten,
 Ki minket könyörgésünkben
 Meghallgatsz nagy kegyelmesen
 És részeltetsz mindezekben.

>291
/1
#A91E4A1D
 Ú mi kegyelmes Krisztusunk,
 Ki vagy nékünk igazságunk, .
 Békességünk és váltságunk:
 Kelljen, kérünk, imádságunk.
/2
#D78BF951
 Mi szívünket, kegyes Isten,
 Igaz hitben, reménységben
 Kiváltképen erősítsed,
 Hogy bízhassunk csak tebenned.
/3
#6F8E851C
 Te isteni jóvoltodat,
 Ismerhessük hatalmadat,
 Melyet hozzánk megmutattál,
 Mikor érettünk áldoztál.
/4
#64DF1FED
 Csak egyedül evilágnak
 Viseléd terhét bajának,
 Nyerél nékünk idvességet
 És mennyei örökséget.
/5
#97E56ACA
 Azért ily nagy szerelmedért,
 Kérünk, ne vess el bűnünkért,
 Ne vehessen ellenségünk
 Dühösséggel semmit tőlünk.
/6
#AC117130
 Ú mi irgalmas Királyunk,
 És győzedelmes Hadnagyunk,
 Hozzád hitből folyamodunk,
 Tudjuk, meg nem fogyatkozunk.
/7
#FD0C230E
 Kérünk, ily nehéz ügyünkbe'
 Ne adj ellenség kezébe,
 Ne vesd bűnünket elődbe,
 Végy be inkább kegyelmedbe.
/8
#18D4D23F
 Keresztyéni egyességben,
 Nevednek szentségében,
 Szent igédnek értelmében,
 Tarts meg minket mindvégiglen.
/9
#7E853E30
 Bírjad lelki országodat,
 Véreden vett jószágodat,
 Hogy e siralom völgyében
 Légyünk csendesek szívünkben.
/10
#B1117E74
 Adj állandó egyességet,
 Lelki-testi békességet,
 Hogy megfutván itt e pályát,
 Vehessük el a koronát.
/11
#9ACF5574
 Dícsértessél áldott Isten,
 Ki minket könyörgésünkben
 Meghallgatsz nagy kegyelmesen
 És részeltetsz mindezekben.

>292
/1
#2434B700
 Ó dicsőségnek felkent fejedelme,
 Úr Jézus Krisztus, hívek segedelme,
 Kit az Úr Isten rendelt váltságra
 S felemelt bölcsen nagy méltóságra:
 Felékesített, mint drága ruhával,
 A Szentléleknek bő ajándékával,
 És reád bízván az idvességet,
 Adott válladra hármas tisztséget.
/2
#EEADB0A7
 Te vagy az Úrnak egy Főprófétája,
 Te vagy titkának világos táblája,
 Ki megjelentéd nagy indulatját,
 A váltság felől szent akaratját:
 Hogy akik még nagy sötétben volnának,
 Az igazság napjára találnának
 Es általlátnák, mint napnak fényét
 Az idvességnek igaz ösvényét,
/3
#CF598153
 Te vagy az Úrnak megszentelt Főpapja,
 Általad költ fel jókedvének napja,
 Ki szent testednek áldozatjával
 Váltságot nyertél annak árával.
 Szent érdemednek most is jelentése
 Olyan, mint szádnak drága könyörgése,
 Ez által szived valamit kiván,
 Az Úr megadja néked azt nyilván.
/4
#6EF128C8
 Te vagy az Úrnak felkent nagy Királya,
 Kinek vettettek mindenek alája,
 Aki népedet tiszta igéddel,
 Oktatod s vezérled Szentlelkeddel.
 Minden ügyében felállasz mellette,
 Ellenségivel harcolsz ő helyette,
 Minden veszélyét karod elhajtja
 S az idvességben végig megtartja.
/5
#8E08564A
 E nagy királynak nálam is zálogja,
 Igaz hit által én is vagyok tagja,
 Ötet ismerem, mint fő Uramat
 És keresztyénnek vallom magamat.
 Áldott nevéről bátran vallást tészek,
 Szent törvényének követője lészek,
 Lelkem és testem hálaadással,
 Nékie vallom hív ajánlással.
/6
#A1B01E35
 Mivel hivattam lelki vitézségre,
 Mig lelkem eljut a várt dicsőségre,
 Ellenségimnek dühösségökkel
 Harcolok bátran igaz lélekkel.
 A bűn s test ellen holtig törekedem,
 Szép koronámat másnak nem engedem,
 Hogy harcom után diadalommal
 Együtt örüljek az én Urammal.
/7
#D04C4F66
 Te pedig, hiveidnek szemefénye,
 Úr Jézus Krisztus, életem reménye,
 Vedd fel szolgádat erős karodra,
 És vigyázz énrám, gyenge juhodra.
 Oktass és taníts hathatós igéddel,
 Biztass, erősíts, a te Szentlelkeddel,
 Hogy végre pályafutásom után,
 Hozzád mehessek az élet útján.

;Debrecen, 1774
>293
/1
#5FC144DE
 Jézus, ó, mi Idvezítőnk,
 Kiben vagyon hitünk,
 Ki miértünk születtél,
 És nagy halált szenvedtél;
 Irgalmazz minékünk!
/2
#28462B04
 Rólunk elvetted haragját
 A te szent Atyádnak;
 Vérednek hullásával
 Megengesztelted hozzánk:
 Irgalmazz minékünk!
/3
#6896BBC6
 Az ördögnek tömlöcéből,
 Kivettél a bűnből;
 A halál köteléből,
 Megmentettél rabságból:
 Irgalmazz minékünk!
/4
#4D3467AA
 Te bűn nélkül születtetvén
 A Szűznek méhéből,
 Értünk válladra vetted
 A mi adósságinkat:
 Irgalmazz minékünk!
/5
#27A07591
 Bűnön, poklon és halálon
 Birodalmad vagyon;
 Az élet és kegyelem
 Csak te kezedben vagyon:
 Irgalmazz minékünk!
/6
#84C490CD
 Kérünk téged azért mostan:
 Tekints reánk mennyből,
 Kik hozzád esedezünk
 És segítséget várunk:
 Irgalmazz minékünk!

;Crüger J., 1653
>294
/1
#54DF2528
 Jézus, vigasságom!
 Esdekelve várom
 Áldó szavadat!
 A te jelenléted
 Megvidámít, éltet,
 Bátor szívet ad.
 Légy velem,
 Ó, mindenem!
 Nálad nélkül nem is élek:
 Te vagy örök élet!
/2
#AA509446
 Jézus, menedékem!
 Hű oltalmam nékem
 Te vagy egyedül!
 Lelkem a viharból,
 Bűnből, minden bajból
 Hozzád menekül.
 Bár a föld
 Mind romba dőlt,
 S ha a pokol hada hány tőrt:
 Jézus maga áll őrt!
/3
#22913AF5
 Jézus, üdvösségem!
 Te vagy földön-égen
 Örök örömem!
 Kik szeretjük Istent,
 Zengjünk neki itt lent
 S otthon: odafenn!
 Lelkem esd,
 Hogy Te vezesd!
 S hazahívó szavad várom,
 Jézus, Vigasságom!

;Frankfurt, 1662
>295
/1
#D50BE26F
 Jézusom, ki árva lelkem
 Megváltottad véreddel,
 Kárhozattól óvtál engem,
 Bűnös szívem, ó, vedd el!
 Add, hogy néked megháláljam.
 Hogy nem hagytál a halálban, megmutattad:
 Bármit adj,
 Én oltalmam csak te vagy.
/2
#7A301A73
 Jézus, benned bízva bízom,
 Elpusztulnom, ó, ne hagyj!
 Te, ki bűnön, poklon, síron
 Egyedüli győztes vagy:
 Gyönge hitben biztass engem,
 Készíts arra, hogy én lelkem
 Láthat majd fenn, ó, Uram,
 Mindörökké boldogan.

;Nicolai F., Frankfurt a. M., 1599
>296
/1
#73B55FBF
 Szép tündöklő hajnal csillag,
 Ki kegyelemmel felragyog,
 Jessének szép vesszeje,
 Dávid királynak magzatja,
 Sok királyoknak királya,
 Lelkemnek vőlegénye:
 Kedves, kegyes, kellemetes,
 ékes és gyönyörűséges,
 Gazdagsággal dicsőséges.
/2
#CF0796FA
 Ó, gyöngyös, drága korona,
 Embernek, Istennek fia,
 Menny áldott, szent gyümölcse!
 Szívemnek vagy lilioma,
 Édes evangélioma,
 Kedvességeknek kincse!
 Eggyem, lelkem, Szép violám,
 égi mannám, eledelem:
 Rólad el nem feledkezem.
/3
#34733CD6
 Öntsd mélyen az én szívembe,
 Szerelmed tüzét lelkembe,
 Ó, drága jáspis kövem!
 Fogadj hozzád s vigasztalj meg,
 Hogy eleven tagod legyek,
 Benned legyen örömem.
 Hozzád kiált
 Sebhedt szívem, lelki rózsa: légy orvosa;
 Nélküled nincs vigassága.
/4
#47D17D98
 Mennyből nagy öröm fénylik rám,
 Midőn szemeddel énreám
 Kegyelmesen tekintesz,
 Ó, Uram Jézus, szent Lelked,
 Szent igéd, tested és véred
 Vélem közölvén, éltetsz.
 Ilyen híven
 Tarts öledben, táplálj engem mindvégiglen,
 Mert fogadtad szent igédben.
/5
#3E2A5F34
 Teremtő Atyám, Úr Isten,
 Ki engem örök időkben
 Szent Fiadban szerettél:
 Ő engem magának jegyzett,
 Szentlélekkel elpecsétlett,
 Megtisztított vérével.
 Vigadj és adj szívem hálát,
 Istent imádd, hogy mennyégben
 Idvességet szerzett nékem.
/6
#8DDDC77B
 Szóljon tehát az orgona,
 Gyönyörűséges muzsika,
 Ujjongva énekeljek,
 Hogy megváltó Jézusomnak,
 Drága szép kegyes Uramnak
 Szerelmében örvendjek.
 Zengjen, pengjen Vigasságos,
 buzgóságos hál'adó szó,
 Nagy az Úr, ezekre méltó.
/7
#31F8DA86
 Ily kedvvel vigad én szívem,
 Mert drága kincsem az Isten,
 Ki kezdet s vég mindenben.
 Ő engem dicséretire,
 Mennybe viszen nagy örömre,
 Min tapsolok szívemben.
 Ámen, ámen! Jövel, Uram,
 szép koronám, jöjj sietve:
 Téged várlak reménykedve!

>297
/1
#4D94209C
 Ó seregeknek Hatalmas királya,
 A nagy Istennek,
 Mi teremtőnknek,
 Gondviselőnknek
 Szerelmes szent Fia:
/2
#C24183F6
 Szent indulattal Minden hívő lélek
 Téged magasztal, ó fő bölcseség,
 Áldott Istenség!
 Én is, amíg élek.
/3
#718B644C
 Uram, mert te vagy
 Amaz örök Ige,
 Eredeted nagy,
 Minden dolgoknak,
 Amelyek vannak,
 Vagy kezdete s vége.
/4
#5384E8D8
 Az Úr Istennek
 Te vagy dicsősége,
 Dicsőségének
 Minden vég nélkül
 Te vagy egyedül
 Teljes fényessége.
/5
#890B6A9A
 Te vagy részese
 Isteni voltának;
 Szent jelentése
 Tégedet mond s vall
 Minden bizonnyal
 Tulajdon Fiának.
/6
#C622533D
 Te általad lett
 Nékünk is Atyánkká;
 Kegyelemből tett
 Engedelmedért,
 Szent érdemedért
 Fogadott fiakká.
/7
#ADD8C146
 Te vagy az Ura
 Választott népednek,
 És hű pásztora
 Minden időben
 Most s jövendőben
 Gyenge seregednek,
/8
#1AD7282B
 Kiket megváltál
 Drága szent véreddel,
 Megszabaditál
 Bűntől, haláltól,
 A kárhozattól
 Elégtételeddel.
/9
#C0578148
 Azért már élünk
 Néked, nem magunknak;
 Hát úgy bánj velünk,
 Hogy nagy örömmel
 Szolgáljunk szívvel
 Néked, mi Urunknak I

;Wittenberg, 1523
>298
/1
#623E7C08
 Jer, Krisztus népe, nagy vígan
 Mind egy örömre keljünk,
 És egybeforrva, boldogan
 Csak arról énekeljünk,
 Hogy könyörülő Istenünk
 Mily áldott csodát tett velünk,
 Mit drágán szerzett nékünk.
/2
#FBA62006
 A Sátán tett rám rabigát
 És már halálba vesztem,
 A bűn gyötört éj- s napon át,
 Mert benne gyökereztem.
 Mind jobban elsüllyedtem én,
 Nem volt számomra már remény,
 Megült a bűnnek átka.
/3
#BF93A1C6
 Nem használt, sőt káromra lett,
 Ha bármi jót is tettem,
 Mert az égi ítéletet
 Önkényből megvetettem.
 Mégis kínzott a félelem,
 Hogy csak a halál van velem
 És a poklokra hullok.
/4
#D39C267B
 Ám az örök szent irgalom
 Nagy ínségem megszánta,
 És könyörülvén ily bajon,
 Megenyhítni kívánta.
 Mint jó Atya, szívébe vett,
 Nem játék volt, amit megtett
 Legfőbb kincsével értem.
/5
#58B96FF7
 Így szólott Egyszülöttjéhez:
 Jött irgalomnak éve,
 Én diadalmam, menj, siess,
 Légy népem üdvössége,
 Bűn átkából segítsd ki hát,
 Fojtsd meg a ráleső halált:
 Az embert térítsd hozzád.
/6
#013C63C9
 S ím, az Atyának engedett
 A Fiú, hozzám jöve,
 Egy tiszta szűztől született
 Testvéremül a földre.
 Járt mint nagy titkos hatalom,
 Felölté önnön alakom,
 Hogy a Sátánt lebírja.
/7
#7D8533D8
 Szólt hozzám: tarts ki már velem,
 Most célod el kell érned,
 Immár enyém a küzdelem,
 Kiállok készen érted.
 Te az enyém, én tied,
 S hol én vagyok, ott lesz helyed:
 Szét nem választ az ellen.
/8
#35ACE776
 És bár kioltja életem,
 És bár kiontja vérem:
 Mindezt javadra szenvedem,
 Hű légy e hitben vélem.
 A halált éltem megveszi
 És szentségem jóváteszi
 A bűnt, hogy üdvözülhess.
/9
#2E61D70D
 Én az Atyához felmegyek,
 Ha végeztem a földön,
 Hogy aztán Mestered legyek,
 A Lelket rád kitöltöm,
 Ki félelmedben bátorít,
 S hogy engem ismerj, megtanít
 És igazságban járat.
/10
#BEFDE655
 Mit tettem és hirdettem én,
 Azt kövesd szóban, tettben,
 Az Úr országát építvén
 És dicsőségét egyben.
 Ne hagyd, hogy hívság s emberek
 Megrontsák lelki kincsedet:
 Ez légyen örök részed!

;Stainer J. után
>299
/1
#1AC0D5A8
 Jézus hív, bár zúg, morajlik
 Életünk vad tengere;
 Halk hívása tisztán hallik:
 'Jer, kövess, ó, jöjj ide!'
/2
#3B200AC5
 Vedd a példát Andrástól, ki
 Hallva hívó szózatot,
 hálóját se vonszolá ki:
 Érte mindent elhagyott.
/3
#57313AAD
 Jézus hív, hogy Őt imádjad,
 Megragad, hogy el ne ess,
 Mert kísért öntelt világod:
 'Jöjj, engem jobban szeress!'
/4
#C7758B79
 Ha nehéz az élet terhe,
 Roskadozva hordom azt:
 Bús orcám Hozzá emelve,
 Jézusban lelek vigaszt.
/5
#28A69ACA
 Uram, hozzám légy kegyelmes,
 Tedd Tieddé szívemet,
 Hadd lehessek engedelmes,
 Néked élő gyermeked!

;Dykes J.B., 1823-1876
>300
/1
#F403C092
 Lelkem drága Jézusa,
 Hozzád hajt a félelem,
 Míg üvölt a habtusa,
 S nő a vész a tengeren,
 Rejts el, rejts el, itt ne hagyj,
 Míg eláll a fergeteg;
 Biztonságos révet adj,
 S majd fogadd el lelkemet.
/2
#9A5ADC93
 Nincs nekem más enyhelyem,
 Szívem Téged hív s keres,
 Ó, maradj itt, Mesterem,
 Őrizz, adj erőt, szeress!
 Véled állom a vihart,
 Hit s erő Te vagy, Te Szent,
 Szárnyad árnyával takard
 Fejemet, a védtelent.
/3
#59CA5ED8
 Csak Te kellesz, én Uram,
 Benned mindent meglelek;
 Támogasd, ki elzuhan,
 Gyógyítsd meg, ki vak s beteg.
 Szent szavadra hallgatok,
 Tévedés az én bajom,
 Én hamisság s bűn vagyok,
 Te igazság s irgalom.
/4
#B7D48C9E
 Kegyelem vagy, égi jó,
 Mely minden bűnt eltörül,
 Hagyd, hogy gyógyító folyó
 Tisztogasson meg belül.
 Élet-kút vagy, lüktetés,
 Vízmerítni drága hely,
 Ó, buzogj fel bennem és
 Öröklét felé emelj.

;Kolozsvár, 1744
>301
/1
#3EF6CC83
 Új világosság jelenék,
 Ó, tévelygés csendesedék;
 Isten igéje jelenék,
 Újonnan nékünk adaték.
/2
#3753863F
 Evangéliom erejét,
 Krisztust, áldott szent Igéjét,
 Atya Isten nagy jó kedvét,
 Megmutatá ő kegyelmét.
/3
#42B7BE62
 Kit sok száz esztendeiglen
 Eltitkolt volt Atya Isten,
 Mint megmondá jövendölvén
 Ámos próféta könyvében.
/4
#FA1BAB02
 Ezt a mi hitetlenségünk
 És nagy telhetetlenségünk,
 Érdemlette tévelygésünk,
 Emberbeli reménységünk.
/5
#2E841AFB
 Igaz az Isten Igéje,
 Kivel él ember elméje,
 Kinek megmarad ereje
 És el nem vész ő reménye.
/6
#01CDC959
 Kérünk, Úr Isten, tégedet,
 Erősítsd meg híveidet,
 Hogy vehessük szent Igédet
 És vallhassuk te hitedet.
/7
#7B5BCC6F
 Mert csak te vagy bizodalmunk,
 Ördög ellen nagy gyámolunk,
 Testünk ellen diadalmunk,
 E világ ellen oltalmunk.
/8
#33E4F9ED
 Dicsőség légyen Atyának
 És egyetlenegy Fiának,
 Ezeknek Ajándékának;
 A dicső Szentháromságnak.

>302
/1
#00756619
 Ó népeknek Megváltója.
 Jővel, szűznek szent Magzatja.
 Mert mind ez világ csudálja:
 Istennek szűztől lesz fia.
/2
#0358FE01
 Atyától lőn kijövése,
 És ahhoz lesz megtérése;
 Pokolig lőn leszállása,
 De felmégyen mennyországra.
/3
#849A282E
 Ki Atyával vagy egyenlő,
 Testben légy nagy győzedelmű;
 Te isteni nagy erődből
 Ments meg erőtlenségünktől.
/4
#A568D358
 A te jászlad immár fénylik,
 Az éj megvilágosodik;
 Homály, setétség távozzék,
 A hitnek világa fényljék.
/5
#A7573490
 Dicsőség légyen Atyának
 És egyetlen egy Fiának
 Szentlélekkel egyetemben
 Most és mind örökké, Ámen.

;Gregorián dallamból, Erfurt, 1524
>303
/1
#80EE13A5
 Jöjj, népek Megváltója,
 Szűznek ékes virága,
 Mind e világ csudálja,
 Mint jöttél, Isten Fia.
/2
#F8015DBE
 Nem emberi erőtől,
 De Szent Lélek Istentől
 Ige testbe öltözék,
 Szűz méhe megvirágzék.
/3
#AF74832A
 Jöve ágyas házából,
 Tiszta szűz szent méhéből;
 Isten, ember ő egyben,
 Eljött hozzánk már testben.
/4
#D13EEB46
 Szent Atyjától földre jött,
 Szállt poklokra és győzött,
 Atyjához emelteték,
 Ő székibe ülteték.
/5
#69BD69BB
 Atya Isten szent Fia,
 E világnak istápja:
 Gyarló néped sok baja
 Rád szállt, légy bajvívója!
/6
#D774768A
 Jászolod immár fénylik,
 Új világa tündöklik,
 Melytől éj elenyészik,
 Hitünk megerősödik.
/7
#BAD686B9
 Dicsőség néked, Urunk,
 Mi kegyelmes Megváltónk!
 Dicsőség szent Atyádnak
 És mi Vigasztalónknak!

;Bourgeois L., Genf, 1551. (130. zsoltár)
>304
/1
#1C209B86
 Kapuk, emelkedjetek!
 Kiáltó szó hallik,
 Ím, jő fejedelmetek,
 Az idő hajnallik:
 Harmattal rakott feje,
 Véle sok áldása,
 Bétölt teljes ideje,
 Hogy minden test lássa.
/2
#763AB310
 Ímhol jő a Vőlegény,
 Lelkem, menj elébe,
 Keresd nyugtod, mint szegény,
 Gazdag kebelébe'.
 Kedves vendéget várok,
 Szívem ajtajárul
 Hulljanak a závárok,
 Mert már közel járul.
/3
#A2988170
 Már az ég harmatozzon,
 A föld igazságot,
 Mint bő gyümölcsöt, hozzon:
 Indíts vigasságot,
 Ó, kegyes Immánuel,
 Mert várlak valóba';
 Ó, te kisded Sámuel,
 Jöjj el már Silóba!
/4
#B2F55F32
 Az utat egyengessed,
 Szívemben a mérget
 S ürömgyökért égessed,
 Lelkemről a kérget,
 A keménységet vedd ki,
 Hogy meglágyulhassak,
 A gazt s gerendát szedd ki
 Szememből, hogy lássak.
/5
#01E85222
 Az Illésnek lelkével
 Ruházz fel engem is,
 Szent szerelmed tüzével
 Égjen én lelkem is.
 Áldj meg oly kegyességgel,
 Hogy higgyek és szóljak,
 Hűségedre hűséggel
 Másokat unszoljak.
/6
#031FCDA5
 Isten Báránya, jövel,
 Mutasd szelídséged;
 Uram, felemelt fővel
 Várom idvességed.
 Igazság napja, támadj,
 Adj világosságot,
 Magamnak nincs: reám adj
 Örök igazságot.
/7
#FF5CB4D1
 Dávid gyökere s ága,
 Fényes hajnalcsillag,
 Pogányok kívánsága,
 megígért áldott mag;
 Isai törzsökéből
 Származott vesszőszál:
 Nékem Atyád kedvéből
 Erős torony s kőszál!
/8
#B476F536
 Bennem az Úr temploma
 Általad készüljön,
 Vesszen a bűnnek nyoma,
 Lelked újjászüljön.
 Méltóztass személyedre
 E gyarló világon,
 Dicső jelenlétedre
 Dűljön le a Dágon.
/9
#8A650DB1
 Tégy szívedre pecsétül,
 Bélyegül karodra:
 Így lelked erejétül
 Élek csak számodra.
 Mindvégig velem maradj
 Mennyei erővel,
 Holtom óráján ne hagyj,
 Jövel, Ámen, jövel!

;Debrecen, 1781
>305
/1
#A31C99B1
 Álmélkodással csudáljuk
 Véghetetlen szerelmed,
 Ó, Isten, ha megvizsgáljuk
 Kijelentett kegyelmed;
 Ezt száj ki nem mondhatja,
 Nyelv nem magyarázhatja.
/2
#9D09AD4E
 Mert az emberi nemzetet
 Annyira becsülötted,
 Hogy te egyetlenegyedet
 Érette elküldötted
 Emberi ábrázatban,
 Hogy élne gyalázatban.
/3
#251D087D
 Ó, Isten bölcsességének
 Megfoghatatlan titka!
 Mély tengerét szerelmének
 Ki értené, mily ritka;
 Ki ezt eszébe venné,
 Mélyen szívébe tenné.
/4
#C3B05BD4
 Mi azért vígan dicsérünk,
 Ó, jó Atyánk, tégedet,
 Magasztalunk s arra kérünk,
 Hogy te szeretetedet
 Gerjesszed fel szívünkben,
 Jobban-jobban lelkünkben.

;Bourgeois L., Strasbourg, 1545
>306
/1
#EACD5CCA
 Kegyes lelkek, az Urat dicsérjétek,
 Áldott Idvezítőnket tiszteljétek,
 Ki bételjesíté, amit ígére,
 Magát megalázván Isten létére,
 Felvevé testünket,
 Eltörlé bűnünket
 Szent ártatlan voltával;
 Fogadjuk hát, midőn
 Mostan is hozzánk jön,
 Dicséret mondásával.
/2
#DFBBC25A
 Lelkünk sebe már nem gyógyulhatatlan,
 Mert meggyógyítá azt a halhatatlan;
 Magára vett minden viszontagságot,
 Hogy így készítsen lelkünknek váltságot.
 Felkeresé nyáját,
 Elvégzé munkáját,
 Végyen hát dicsőséget!
 Zengjen annak ének,
 Ki hozott népének
 Ily kívánt idvességet!
/3
#29F0D4E6
 Ne vess meg, Jézusunk, tovább is kérünk,
 Légy életünkben hatalmas vezérünk;
 Szent igéd és Lelked segítsen minket,
 Hogy meggyőzzük lelki ellenséginket.
 Aki úgy szerettél,
 Hogy emberré lettél,
 Isten lévén, érettünk:
 Add viszont szeretnünk,
 Szent példád követnünk,
 Valamíg tart életünk.

>307
/1
#49536921
 Dícséretet mond nyelve mindennek
 Tenéked, könyörülö Istennek,
 Hogy meglátván nyavalyás sorsunkat,
 Elvégezted szabadításunkat,
 E végre Fiad küldéd e világra,
 Hogy útat nyisson nékünk mennyországra.
/2
#3934F403
 Megaláztad, hogy felmagasztaljon,
 ő szomorkodott, hogy vígasztaljon;
 Szegénységben élt, hogy gazdagítson,
 Meghalt, hogy minket megszabadítson
 Kárhozatától az örök halálnak:
 Ily dolgot még angyalok is csudálnak.
/3
#448D5123
 Áldott légy, Atyánk, hogy könyörültél
 És hozzánk ilyen megtartót küldtél!
 Vígy elébb ennek ismeretében,
 Adj idvességet Fiad nevében;
 Igaz hit által köss össze ővele,
 Melynek a szeretet légyen kötele.
/4
#A9547F2B
 Tekints Fiadnak szent érdemére,
 Melyet elődbe tett drága vére;
 Érte bocsásd meg sok bűneinket.
 A kárhozattól szabadíts minket,
 Általa végre hozzád égbe érjünk,
 Ott szent Fiaddal s Lelkeddel dicsérjünk!

>308
/1
#122F525A
 Igaz Isten, ígéretedben
 Változhatatlan valóság!
 Amit te a te beszédedben
 Megmondasz, az mind valóság.
 Könnyebb megavulni,
 Végképpen elmúlni
 A természetnek,
 Mint semmibe menni
 Az igaz isteni
 Szent ígéretnek.
/2
#8FC58F47
 Megmondottad volt még kezdetben
 Az első egy pár embernek,
 Hogy ők a szomorú esetben
 Mindörökké nem hevernek:
 Küldesz vigasztalót,
 Ő magvukból valót,
 Ki eredeti
 Elvesztett jussukba
 És boldogságukba
 Visszahelyezi.
/3
#6A4AA44C
 Meglett ez, és a Szűz méhében
 Fogantatott az, akinek
 Jézusi felséges nevében
 Áldás szól a föld népinek.
 Így az ígértetett
 És kijelentetett
 Teljes időnek
 Hajnala felderült
 A sötétségben ült
 Sok kesergőnek.
/4
#6592CAF1
 Lelki örömmel megújulva
 Imádjuk szent Felségedet,
 Hogy ismét mireánk fordulva
 Szemléljük régi kedvedet.
 Megnyitjuk szívünket,
 Tárjuk kebelünket,
 Hogy Jézusunkat
 Ekképpen fogadjuk,
 Híven általadjuk
 Néki magunkat.
/5
#684D8E49
 Tarts meg bennünket az országban,
 Melyet ő köztünk állított,
 Adj részt abban a boldogságban,
 Melyet a földre szállított!
 Majd éltünk végével
 Bocsáss el békével,
 Hogy oda térjünk,
 Hol az ő hívei
 Száma közt mennyei
 Országlást érjünk.

>309
/1
#C5B49463
 Mennyei Ige, jelenél,
 Örök Atyától kijövél,
 Testet magadra felvevél,
 S abban minket idvezítél.
/2
#66EBAF68
 Te vagy Atyánknak Igéje,
 Kit Ádámnak megígére,
 És Ábrahámnak hirdete,
 Dávidnak is megjelente.
/3
#963EFAA9
 Világosítsd meg elménket,
 Szenteld meg a mi szívünket,
 Hogy ismerhessünk tégedet,
 Útálhassuk bűneinket.
/4
#C8253475
 Erősítsd bennünk hitünket,
 Nyerhessük idvességünket;
 Te légy minékünk épület,
 Út, igazság, örök élet.
/5
#C7B4BFCB
 Hogy mikor eljössz ítélni,
 Minden szív titkát kivenni,
 A jóknak minden jót adni,
 A gonoszokat büntetni:
/6
#5170B4C1
 Akkor tőled bűneinkért
 Ne vess el; lám te azokért
 Magas keresztfán ontál vért:
 Sőt idvezíts érdemedért.
/7
#9E585082
 Dicséret és nagy dicsőség,
 Adassék néked tisztesség,
 Ki Atyával egy Istenség,
 Szentlélekkel vagy egy Felség.

;Debrecen, 1774
>310
/1
#B93C5D67
 Küldé az Úr Isten
 Hűséges szolgáját
 Szűzhöz Názáretben
 Gábriel angyalát
 Hozzánk jókedvében:
/2
#4C05C9FD
 Menj el Máriának
 E jót megmondani,
 A régi Írásnak
 Titkát jelentsed ki
 Angyali erőddel.
/3
#0072EED4
 Mondd ezt: ó, szent, kegyes,
 Üdvözlésem végyed;
 Ajándékkal teljes,
 Az Úr van tevéled;
 Szűnjék hát félelmed.
/4
#C52DA45D
 Szent szűz, méhedbe vedd
 Az Úr Isten Fiát,
 Melyben megőrizzed
 A szüzesség jussát,
 Minden tisztaságát.
/5
#8385D878
 Hallá s elfogadá
 E parancsolatot,
 Hivé és fogada
 Méhében magzatot,
 De nagy Csudálatost,
/6
#0DF29394
 Emberi nemzetnek
 Hű tanácsadóját,
 Jövendő életnek
 Maradandó atyját
 Örök békességben;
/7
#6885DD64
 Kinek erőssége
 Minket úgy őrizzen,
 Hogy a világ vétke
 Minket meg ne sértsen,
 Pokolra ne vessen;
/8
#E1FA7738
 De a Bűnbocsátó
 Végye el vétkünket,
 Légyen igazítónk
 S adjon örökséget
 Mennyek országában.

>311
/1
#5576D9F0
 Szent Ézsaiás így ír
 Krisztusnak szent születéséről.
 Hogy egy vesszőszál felnevekedik
 Jesse gyökeréről.
/2
#02331E0E
 Ennek gyökerén egy virágocska. - úgymond ­ nevekedik,
 Kin az Istennek, ő szent Atyjának lelke megnyugoszik.
/3
#708977E3
 Győzedelmesnek, nagy hatalmasnak, irgalmasnak vallja,
 A kegyelemmel és igazsággal őt teljesnek mondja.
/4
#A6FEC23B
 Ez által lészen az Úr Istennel nékünk békességünk,
 Mert közbenjárónk szent Atyja előtt ő lészen minékünk.
/5
#ED39E2FD
 Drága jószágot és örökséget szerez ő minékünk,
 Adósságinkat nagy gazdagsággal megfizeti értünk.
/6
#073CBD3E
 Igazságával mind e világot megítélni fogja,
 Mert az Úr Isten az ítéletet kezébe bocsátja.
/7
#BCF73A00
 Nagy békességet, lelki örömet hoz ő evilágra,
 Mert birodalmat, nagy királyságot veszen fel vállára.
/8
#EABF1CFC
 Általa leszen, hogy a bárányok a farkassal laknak,
 Semmi bántásuk oroszlánoktól nem leszen juhoknak.
/9
#73C7244D
 Győzedelmének hatalmasságát valakik meglátják,
 Krisztus Jézusnak isteni voltát mindazok imádják.
/10
#2A95B3EE
 Nagy bátorsággal szegény bűnösök akkor könyörögnek;
 Krisztus nevében lelki és testi ajándékot vesznek.
/11
#01283C12
 Az ő zászlóját felemelteti Anyaszentegyházban,
 És szent igéjét prédikáltatja e széles világban.
/12
#A7F10075
 Ezt megcselekszi a seregeknek Ura és Istene,
 Az Úr Istennek reánk áradott buzgó nagy szerelme.
/13
#BCC9AA26
 Nagy vígasságunk, örvendezésünk legyen e gyermekben;
 Mert idvességünk vagyon minékünk szent születésében.

;Erfurt, 1572
>312
/1
#635B62C0
 Várj, ember szíve, készen!
 Mert jő a Hős, az Úr,
 Ki üdvösséged lészen.
 Szent győztes harcosúl,
 Fényt, éltet hozva jő,
 Megtört az ősi átok:
 Kit vágyakozva vártok,
 Betér hozzátok Ő.
/2
#D5826ECE
 Jól készítsétek útát!
 A Vendég már közel!
 Mi néki gyűlölt, útált,
 Azt mind vessétek el!
 A völgyből domb legyen,
 Hegycsúcs a mélybe szálljon,
 Hogy útja készen álljon,
 Ha Krisztus megjelen.
/3
#299EA1FB
 Az Úr elé ha tárod
 A szív alázatát,
 Őt nemhiába várod:
 Betér hozzád, megáld.
 A testi gőg: halál!
 De bűnödet ha bánod,
 Szent Lelke bőven árad,
 S a szív üdvöt talál.
/4
#9D9B030A
 Ó, Jézusom, szegényed
 Kér, vár, epedve hív:
 Te készítsd el: tenéked
 Lesz otthonod e szív.
 Jer hű szívembe hát!
 Habár szegény e szállás,
 De mindörökre hálás,
 Úgy áldja Krisztusát.

;Herman Miklós, Lipcse, 1554
>313
/1
#178F8D4E
 Dicsérd Istent, keresztyénség,
 Ő dicsőségében,
 Kitől árad rád idvesség,
 Fia érdemében,
 Fia érdemében.
/2
#B8F67E71
 Leszállván Atyja kebléből,
 Kicsiny gyermek korban,
 Szegénységben, ruha nélkül
 Fekszik a jászolban,
 Fekszik a jászolban.
/3
#D661A6B0
 Letette minden hatalmát,
 Erőtelen vala,
 Felvette szolga formáját
 Mindeneknek Ura,
 Mindeneknek Ura.
/4
#430AFB90
 Anyjának ölében nyugszik,
 Tejével táplálja;
 Az angyalok ezt örvendik,
 Mert Dávidnak fia,
 Mert Dávidnak fia.
/5
#17C7921F
 Ki az utolsó időben
 Eljövendő vala,
 Hogy építtessék hívekben
 Az Isten országa,
 Az Isten országa.
/6
#CF8D4B6A
 Csuda változással testet
 Ő magára felvőn,
 Adván nékünk idvességet,
 Mennyben részessé tőn,
 Mennyben részessé tőn.
/7
#E3DE35A8
 Én úr, ő pedig lett szolga,
 Ó, csuda változás!
 Jobbat Jézus mit adhatna:
 Nékünk boldogulás,
 Nékünk boldogulás.
/8
#88BF8066
 Ma Paradicsom kapuját
 Ismét megnyitotta,
 Kérub nem állja ajtaját,
 Ezért minden áldja,
 Ezért minden áldja.

;Debrecen, 1774
>314
/1
#A16CE732
 Jézus, születél idvességünkre,
 Amint régenten vala ígérve,
 Atya Istennek nagy szerelme,
 Emberi nemhez tetszik ebbe
 Kegyessége.
/2
#D40D7064
 Első szüleink vétkei miatt
 És az ördögi csalárdság miatt
 Hoztunk magunkra örök halált,
 És kárhozatot, pokol kínját, sok nyavalyát.
/3
#EB6CCCC2
 Nem volt senki sem, ki megmentene,
 Ámde szent Atyád megkönyörüle,
 Téged ígére, s idő telvén
 Szabadítóul nékünk mennyből
 Alákülde.
/4
#9A33FB2E
 Értünk felvevél a Szűztől testet,
 Melyben szenvedél s tettél eleget;
 Elvévéd a mi bűneinket,
 Nyervén minékünk örök éltet,
 Dicsőséget.
/5
#A7B2953A
 Siess már hozzánk, megváltó Urunk!
 Adjad, hogy végig benned bízhassunk,
 És szent Atyáddal megláthassunk,
 Mennyben tevéled lakozhassunk,
 Vigadhassunk.
/6
#705061B2
 Légyen tisztesség te Felségednek
 És szent Atyádnak, mi teremtőnknek,
 És egyetemben Szentléleknek:
 Dicsőség a Szentháromságnak,
 Egy Istennek!

;Debrecen, 1774
>315
/1
#2D1A5AA3
 Krisztus Urunknak áldott születésén,
 Angyali verset mondjunk szent ünnepén,
 Mely Betlehemnek mezejében régen
 Zengett ekképpen:
/2
#CF191BCC
 A magasságban dicsőség Istennek,
 Békesség légyen földön embereknek,
 És jóakarat mindenféle népnek
 És nemzetségnek!
/3
#E139765A
 A nemes Betlehemnek városába'
 Gyermek született szűztől e világra,
 Örömet hozott Ádám árváira,
 Maradékira.
/4
#207EEFF2
 Eljött már, akit a szent atyák vártak,
 A szent királyok akit óhajtottak,
 Kiről jövendőt próféták mondottak,
 Nyilván szólottak.
/5
#203D7EC0
 Ez az Úr Jézus, igaz Messiásunk,
 Általa vagyon bűnünkből váltságunk,
 A mennyországban örökös lakásunk,
 Boldogulásunk.
/6
#D4B5B37E
 Hála legyen mennybéli szent Atyánknak,
 Hála legyen született Jézusunknak,
 És Szentléleknek, mi vigasztalónknak,
 Bölcs oktatónknak!
/7
#0261F597
 Ó, örök Isten, dicső Szentháromság,
 Szálljon mireánk mennyei vigasság,
 Távozzék tőlünk minden szomorúság,
 Légyen vidámság!

;Debrecen, 1774
>316
/1
#36B4D926
 Az Istennek szent angyala
 Mennyekből hogy alászálla,
 És a pásztorokhoz juta,
 Nékiek eképen szóla:
/2
#D64A637C
 Mennyből jövök most hozzátok,
 És íme, nagy jó hírt mondok,
 Nagy örömet majd hirdetek,
 Melyen örvend ti szívetek.
/3
#24AA4720
 E mai nap egy kis gyermek
 Szűztől született tinéktek,
 A gyermek szép és oly ékes,
 Vigasságra kellemetes.
/4
#93BA3F05
 Már lehozta az életet,
 Mely Istennél volt készített,
 Hogy ti is véle éljetek,
 Boldogságban örvendjetek.
/5
#829A70A0
 Ez lesz néktek a jegy róla:
 Ímé, fekszik a jászolba',
 Ott megtaláljátok őtet,
 Kitől menny, föld teremtetett.
/6
#E825C7B4
 Ez Úr Krisztus mi Istenünk,
 Nyavalyáinkból kimentőnk,
 Ő lészen az Idvezítő,
 Minden bűnünkből kivévő.
/7
#D425B208
 Nyílj meg, szívem, lásd meg jobban,
 Ki fekszik itt e jászolban?
 Ez a gyermek bizonyára
 Az Úr Jézus, Isten fia.
/8
#78D43B03
 Jertek hát, mi is örvendjünk,
 A pásztorokkal bémenjünk,
 Lássuk, mit adott az Isten
 Hozzánk való szerelmében.
/9
#D4A2D0BE
 Mindeneknek teremtője,
 Miért vagy ily szegénységbe'?
 Hogy fekszel az aszú szénán,
 Szamár s ökrök közt aludván?
/10
#8AB3177B
 Nincs-é senki e világon,
 Ki tégedet béfogadjon?
 Nincsen-é meleg helyecskéd,
 Sem gyengén rengő bölcsőcskéd?
/11
#E41F673D
 Néked bársonyod s tafotád,
 Aszú széna lágy párnácskád;
 Noha nagy dicső király vagy,
 Mostan ímé, mily szegény vagy!
/12
#8E8B3A04
 Ó, én szerelmes Jézusom,
 Édes megváltó Krisztusom!
 Jövel, csinálj csendes ágyat,
 Szívemben magadnak házat!
/13
#24F8F312
 Ó, kedves vendég, nálam szállj,
 Bűnömtől ne iszonyodjál,
 Jöjj be hozzám, te szolgádhoz,
 Szegény megtérő juhodhoz!
/14
#1AE59071
 Én lelkemnek rejtekébe,
 Zárkózz emlékezetébe,
 Hogy el ne felejthesselek,
 Sőt örökké dicsérjelek!
/15
#112748D2
 A mennyei magas égben
 Istennek dicsőség légyen,
 Ki szent Fiát küldé értünk,
 Hogy Megváltónk lenne nékünk.

>317
/1
#1923C794
 Jer, dícsérjük e szent napon a mi Urunkat.
 Bizony méltó dicséretünkre,
 Nagy tisztességre.
 Mert született ez nap nékünk idvességünkre.
/2
#D5677FD9
 Vigasságnak és örömnek ez a nagy napja,
 Min örülnek mind az angyalok,
 No azért mi is
 Adjunk hálát e gyermeknek, mint mi Urunknak.
/3
#D6F56682
 Atya Isten Ő szent Fiát nem azért küldé,
 E világot hogy megítélje és elveszítse,
 De megtartsa, szabadítsa és idvezitse.
/4
#2B840294
 Bizonyságul e világra azért születék,
 Hogy azok, kik teljes lélekkel Ő benne hisznek,
 Örök életet várjanak, Istent nézhessék.
/5
#1E61AD01
 Azért higyjen minden ember Jézus Krisztusban,
 Mert kik bíznak Ő érdemében, Születésében.
 Nem kárhoznak el, sőt véle örülnek mennyben.
/6
#09AAD960
 Ó te áldott Idvezitő, Úr Jézus Krisztus,
 Hálát adunk ez nap tenéked
 És kérünk téged,
 Tiéidet magas mennyből hogy megtekintsed.
/7
#16B527A0
 Adjad nékünk ajándékul te Szentlelkedet,
 Hogy örökké vallhassunk téged, és dicsérhessünk;
 Holtunk után mennyországban téged láthassunk.
/8
#4E9DFD88
 Dicsértessék Szentháromság e mai napon,
 Atya, Fiú és a Szentlélek: Egy bizony Isten;
 Dícséretet néki mondjunk örökké, Ámen.

;Debrecen, 1774
;
>318
/1
#2BA080E4
 Jer, mindnyájan örüljünk,
 És szívünkben vigadjunk,
 Mert született Úr Jézus nekünk.
/2
#15809085
 Kit az Atya Úr Isten,
 Könyörülvén emberen,
 Elbocsájta teljes időben.
/3
#73102860
 Elhagyá gazdagságát,
 Véghetetlen országát,
 Hogy érettünk adja önmagát.
/4
#024DB34E
 Ő életnek adója,
 Szívek vigasztalója,
 Lelkünk megvilágosítója.
/5
#768F90F2
 Azért jöve, hogy éljünk,
 Isten kedvébe essünk,
 Érdeméből kegyelmet nyerjünk.
/6
#30549B0F
 Nagy szeretet mindenhez,
 Hogy Isten emberekhez
 Jöve, fertelmes bűnösökhöz.
/7
#9FB1482C
 Akik benne nem bíznak,
 Sőt bízni sem akarnak,
 Örök halállal mind meghalnak.
/8
#4BDCD2FD
 Mi azért e felségben,
 Emberré lett Istenben:
 Higgyünk mi egy reménységünkben.

>319
/1
#85257DC6
 Kezdetben volt az Ige,
 A Krisztus ő Felsége,
 Szent Atyjának egyetlenegyje.
/2
#FECE5A26
 Leszálla mennyországból,
 Atyja akaratjából,
 Istenségnek szent tanácsából.
/3
#7B4073D6
 E világra adaték,
 És minékünk születék,
 Emberi nem hogy megváltatnék.
/4
#B86591FA
 Mint Ádámban kesergünk:
 A Krisztusban vígadunk,
 Mert ő nékünk minden örömünk.
/5
#C0171456
 Mert Ádámban meghaltunk,
 De lőn Krisztus mi Urunk,
 Ki által ismét feltámadunk.
/6
#98F21F55
 Örvendjünk és vígadjunk,
 Istennek hálát adjunk,
 Hogy halálból életre jutunk.

;Debrecen, 1774
>320
/1
#61CD8282
 Ez nap nékünk dicséretes nap,
 Bizony vigasságnak napja,
 És idvességnek bizodalma,
 Mert született ez nap nékünk mi váltságunkra
 A Krisztus Jézus, Istennek Fia.
 Áronnak veszszeje megvirágozék,
 Tiszta szűztől gyermek születék,
 Menynyei királyul nékünk adaték,
 Krisztus Jézusnak nevezteték.
/2
#46278696
 Ez Gyermek volt a megígért mag
 A mi első atyáinknak, Ádám atyánknak, Ábrahámnak,
 Kiben minden nemzetségek megáldatnának,
 Örök életre feltámadnának.
 Áronnak vesszeje megvirágozék,
 Tiszta szűztől gyermek születék,
 Mennyei királyul nékünk adaték,
 Krisztus Jézusnak nevezteték.
/3
#87DCE8F3
 Megtöreték e Gyermek miatt
 Az ördögnek nagy hatalma,
 A halál, ördög, bűn országa;
 Megnyittaték mennyországnak erős kapuja:
 Istennek kedve mireánk szálla.
 Áronnak vesszeje megvirágozék,
 Tiszta szűztől gyermek születék,
 Mennyei királyul nékünk adaték,
 Krisztus Jézusnak nevezteték.
/4
#72449B25
 Nincsen immár semmi félelmünk
 A mi nyomorúságinktól,
 Bűntől, haláltól, kárhozattól,
 Sem a Mózes törvényének kemény átkától,
 Ördögnek rajtunk nagy bosszújától.
 Áronnak vesszeje megvirágozék,
 Tiszta szűztől gyermek születék,
 Mennyei királyul nékünk adaték,
 Krisztus Jézusnak nevezteték.
/5
#B63E421B
 Megtöretnek a pogány népek,
 Kik e Gyermekben nem hisznek;
 A nagy Istennek nem kellenek,
 Az ördögnek hatalmába örökké esnek,
 Akik a bűnnek véget nem vetnek.
 Áronnak vesszeje megvirágozék,
 Tiszta szűztől gyermek születék,
 Mennyei királyul nékünk adaték,
 Krisztus Jézusnak nevezteték.
/6
#476E5B24
 Hálát adjunk az Úr Istennek,
 Atya-Fiú-Szentléleknek,
 És örüljünk mind e Gyermeknek;
 Nagy örömet az angyalok nékünk hirdetnek,
 Dicsérvén Istent, így énekelnek:
 Áronnak vesszeje megvirágozék,
 Tiszta szűztől gyermek születék,
 Mennyei királyul nékünk adaték,
 Krisztus Jézusnak nevezteték.

;Bourgeois L., Lyon, 1547
>321
/1
#474493EA
 Hogy eljött az időknek teljessége,
 Bétölt már minden szentek reménysége;
 Kit régtől fogva minden szent vára:
 A Fiú testet öltött magára.
 Nyilván lett hozzánk Isten jó szándéka:
 Ím, emberek közt van az ő hajléka;
 E világ éppen már-már megavult,
 De Jézus eljöttével megújult.
/2
#2D3F4786
 Uraknak Ura értem lett szolgává,
 Tévén én dolgom a maga dolgává.
 Ami a testnek nem volt lehető,
 Elvégzi Jézus, mindent tehet ő.
 Ó, mint szerette Isten e világot!
 Számára nevelt egy szép szál virágot.
 Ma fakadt fel az élet kútfeje,
 És megvirágzott Áron vesszeje.
/3
#C0BF8651
 Ezáltal a kegyelmi frigy felépült,
 Hogy halál árnyékában amely nép ült,
 Láthasson szép nagy világosságot,
 A bűnös is nyerhessen váltságot.
 Eljött, hogy a békességet hirdesse,
 Hogy az elveszett juhot megkeresse,
 Hogy az ördögnek dolgát elrontsa,
 Hogy értem drága vérét kiontsa.
/4
#D12A8DE8
 Ó, Isten, hozzám kötéd így magadat,
 Hogy értem küldéd világra Fiadat,
 Sok gonoszságom nem tekintetted,
 Veszendő sorsom szívedre vetted.
 Nem gondolál szent gyönyörűségeddel,
 Csak azzal, hogy jót tégy ellenségeddel;
 Csuda, hogy annak Istene lettél,
 Kinek teljességgel nem kellettél.
/5
#8364E06D
 Már megítélted szegény lelkem perét,
 Rám árasztottad szerelmed tengerét;
 Végét nem érem én e mélységnek,
 Angyalok is csak rebegnek ennek:
 Ó, nagy szeretet, melyhez hasonló nincs!
 Ha lett volna még Istennél nagyobb kincs,
 Nem tartózkodott volna iránta;
 Ez volt a legtöbb; nékem ezt szánta.
/6
#85BCB708
 E kedves vendéget már mint fogadjam?
 Dávid fiának immár mimet adjam?
 Ha nincs a vendégházakban helyed,
 Ímhol van szívem, itt hajtsd le fejed.
 Hagyd ott a barmot, jászolt és istállót,
 Hadd nyerjelek meg, mennyből hozzám szállót!
 Itt szállj, galambom, karjaim készek,
 E száraz fán vár egy üres fészek.
/7
#39C9705D
 Kereslek, Uram, engem te is keress;
 Szeretlek, tudod, ó, hát te is szeress!
 Tedd egy szívvé szívemet szíveddel,
 Ejts rabul engem hívó szemeddel.
 Vonj, hogy atyádhoz általad mehessek,
 Élj bennem, benned hogy én is élhessek!
 Ó, Uram, tőled hová mehetnék?
 Elveszném, tiéd ha nem lehetnék.

>322
/1
#96B1583B
 E világot, bár ez bűnt tett,
 Az Isten úgy szerette,
 Hogy kibocsátani kész lett
 Önnön Fiát érette.
 Emberek, az Úrnak kedve
 Nincsen halálotokba',
 Ó, angyalok, legörbedve
 Nézzetek e titokba:
/2
#D02A5B75
 Ím, az Atya kebeléből
 Leszáll ama Szerelmes,
 Lesz az ő jó tetszéséből
 Halálig engedelmes.
 Földi szállással váltja fel
 A dicsőségnek helyét,
 De annyi helyet alig lel,
 Ahol lehajtsa fejét.
/3
#120B095B
 Úr Jézus, lelkemtől kérdem:
 Téged mi indíthatott,
 Hogy értünk, kikben nincs érdem,
 Vállalj ily szolgálatot?
 Nem más, hanem az irántunk
 Benned égő szeretet,
 Mely, bár bűnünkkel megbántunk,
 Mégis meg nem hűlhetett.
/4
#FCE70218
 Isten, ha a te kebeled
 Kincsét adtad érettünk,
 Midőn mint pártütők, veled
 Még ellenségeskedtünk:
 Adj meg már Ővele mindent,
 Ami arra szükséges,
 Hogy itt légyen életünk szent,
 S halálunk idvességes.
/5
#C637A0F9
 Jézus, felépítetted már
 A mennyeknek országát,
 Hol, aki törvényidben jár,
 Megleli boldogságát.
 Mi, tenéked térdet hajtó
 Híveid, esedezünk:
 Nyíljon meg nékünk az ajtó,
 Vezess be, fogván kezünk.

;Bourgeois L., Strasbourg, 1545
>323
/1
#70C22F76
 Dicsőség a magas mennyekben Istennek és ide alatt,
 A földi alacsony helyekben
 Békesség és jóakarat!
 Így énekelnek az Istennek
 Az égi karok, buzdítván
 Az élőket, kik örvendeznek,
 Ez éneklést megújítván.
/2
#ECD68B2E
 Dicsőség, dicsőség az égben Istennek, ki úgy szerette
 E világot, hogy szegénységben
 Szent Fiát eleresztette,
 Hogy minket, gyarló halandókat,
 Kiket szomorú fogságba'
 Tart vala a bűn, mint rabokat,
 Helyheztessen szabadságba.
/3
#85DD6452
 Ó, emberi testbe öltözött Jézusunk, lásd, mint gerjedez
 A mi szívünk az öröm között,
 Úgy újul és úgy éledez,
 Mint mikor a nap feljöttével,
 Kilövellvén az életet,
 Elűzi az éjjelt s fényével
 Felkölti a természetet.
/4
#18D67485
 Jövel, fogadd el te magadnak
 E szívet és lakozz ebben;
 Ez az, amit adhatnak s adnak
 Híveid legszívesebben
 Azért a csuda szeretetért,
 Amelyet hozzánk mutattál,
 Midőn a mi már vesztére tért
 Lelkünkért alászállottál.
/5
#07AF5B59
 Hozd el mihozzánk te magaddal
 Az isteni békességet;
 Bűnös lelkünknek irgalmaddal
 Nyújts biztatást, reménységet,
 Hogy bús háborúnk, amely belől
 Régóta szaggat már és tép,
 Ne kezdődjék elől, meg elől,
 Hanem hallgasson el végképp.
/6
#D47C59D1
 Szülj újjá, értünk ma született
 Jézusunk, a te lelkeddel
 Ezen a néked szenteltetett
 Ünnepnapon, s kegyelmeddel
 Úgy igazgass és bírj bennünket,
 Hogy nyomdokid követhessük,
 És e mi földi életünket
 Mennyeivel cserélhessük.

;(1539) Bourgeois L., Lyon, 1542
>324
/1
#2961D813
 Örvendjetek, keresztyének,
 Nyíljatok meg nyelvek és szívek,
 Az idvesség Istenének
 Mondjatok áldást, minden hívek!
 Felváltatott nagy örömmel
 A haláltól való félelem,
 Mit véghetetlen érdemmel
 Meggyőz az isteni kegyelem.
/2
#87DF730D
 A Megtartó ma született,
 Az örök Isten emberré lett,
 És ma visszaszereztetett
 Az igazság s elvesztett élet.
 Kiküldé szerelmes Fiát
 Istenünk a teljes időben,
 Hogy a kezes a bűn díját
 Fizesse s szenvedje testében.
/3
#3E68B3B9
 Ó, imádandó titkai
 Ama békesség tanácsának,
 Ó, nagyhatalmú dolgai
 A menny és föld szabad Urának!
 Aki által teremtetett
 S lett minden serege az égnek,
 Az Ige testté született,
 Személye az egy Istenségnek.
/4
#A16C13AB
 Ennek örülnek az egek,
 A mély titkon elcsudálkoznak,
 Hirdetik angyalseregek
 Jézust, s előtte leborulnak.
 A Magasságost tisztelik,
 A földön békességet zengnek,
 És ünnepnappá szentelik
 megjelenését az Istennek.
/5
#FF9D68CB
 Áldom én is szent nevedet,
 Én királyom, szenteknek szente,
 És vígan ülöm ünneped,
 Ó, Jézus, Istennek felkentje!
 Magasztalom Felségedet
 És imádlak szent félelemmel,
 De egyszersmind szerelmedet
 Megölelem igaz hitemmel.

;Nicolai F., Frankfurt a. M., 1599
>325
/1
#8636B4EB
 Szívünk vígsággal ma bétölt,
 Mert ígéret szerint felkölt
 Istenfélők számára
 Az igazság fényes napja:
 Újtestámentomnak papja
 Eljött, kit sok szent vára.
 Csillag villog, Ragyog már rám,
 melyet Bálám láta régen,
 Fénylik, mint szép nap az égen.
/2
#19BDD57E
 Megszáná az Úr estünket,
 Felöltözé portestünket,
 Szent kegyelme mily drága!
 Győzelmei már megszűnnek
 A kárhozatszerző bűnnek,
 Törvénynek nem sújt átka;
 Mérge, férge a pokolnak
 Megromolnak ma végképpen,
 Nem árthatnak semmiképpen.
/3
#521677FE
 Próféta, kinek nincs mása,
 Jött hozzánk, hogy népét lássa,
 Bölcsességre oktassa;
 A főpap jött, hogy áldozzon,
 Tiszta szívet, lelket adjon,
 Világ bűnét elmossa;
 Eljött s meglett a szenteknek,
 Felkenteknek fejedelme,
 Királya, fősegedelme.
/4
#B6910A38
 Emberré lőn Isten Fia,
 Emberekért atyánkfia
 Hogy így ő nékünk lenne.
 Máriától, tiszta szűztől,
 Nem földi, vagy testi tüztöl,
 Hogy minket szentté tenne.
 Fényes, kényes Eljövése,
 Születése nem volt néki,
 Bár a menny királyi széki.
/5
#0A1D6086
 Légyek benned, te énbennem,
 Jézus, engedj hozzád mennem,
 Ha te is hozzám jöttél;
 Adjad: légyek a te híved,
 Essék meg rajtam hü szíved,
 Ha kegyelmedbe vettél,
 Mert nincs más kincs,
 Mely hívekkel, bús szívekkel jól tehetne,
 Boldogságot szerezhetne.
/6
#32A87B67
 Szelídség volt minden dolga,
 Önként leve értünk szolga,
 Megalázta önmagát;
 Földi fényt, hírt nem kergete,
 Szerény munkás volt élete
 S vérén szerzé birtokát;
 De épp ez szép bizonysága,
 Hogy országa lelki, belső,
 Királysága égi, első.
/7
#5702838D
 Jézus, engedj hozzád mennem,
 Éljek benned, te énbennem,
 Ha te hozzánk eljöttél;
 Adjad, legyek igaz híved,
 Ó, essék meg rajtam szíved,
 Ha kegyedbe bevettél;
 Mert nincs más kincs, mely hívekkel,
 Bús szívekkel jót tehetne,
 Boldogságot szerezhetne.

>326
/1
#80533655
 Dicsőség mennyben az Istennek!
 Dicsőség mennyben az Istennek!
 Az angyali seregek
 Vígan így énekelnek:
 Dicsőség, dicsőség Istennek!
/2
#5F795264
 Békesség földön az embernek!
 Békesség földön az embernek,
 Kit az igaz szeretet
 A Jézushoz elvezet,
 Békesség, Békesség Embernek!
/3
#CC9629AB
 Dicsérjük a szent angyalokkal,
 Imádjuk a hív pásztorokkal
 Az isteni Gyermeket,
 Ki minket így szeretett,
 Dicsérjük, Imádjuk
 És áldjuk!
/4
#AE869BC1
 Ó, Jézus! ne vess meg bennünket,
 Hallgasd meg buzgó kérésünket!
 Jászolodnál fogadjuk,
 Hogy a vétket elhagyjuk,
 Ó, Jézus, Ne vess meg: Hallgass meg!
/5
#A2F35F09
 Dicsőség az örök Atyának
 És értünk született Fiának,
 Mindkettő Szent Lelkének,
 Áldások kútfejének:
 Dicsőség, Dicsőség Istennek!

>327
/1
#024440BD
 Ó, jöjjetek, hívek, ma lelki nagy örömmel,
 A jászolhoz Betlehembe jöjjetek el!
 Megszületett az angyalok királya:
 Ó, jöjjetek, imádjuk,
 Ó, jöjjetek, imádjuk,
 Ó, jöjjetek, imádjuk az Úr Krisztust!
/2
#972BEF46
 Az életnek szent Ura, dicsőség Királya
 Itt fekszik a jászol mélyén nagy szegényen.
 Nagy dicsőséges, szent és örök Isten!
 Ó, jöjjetek, imádjuk,
 Ó, jöjjetek, imádjuk,
 Ó, jöjjetek, imádjuk az Úr Krisztust.
/3
#1AC12965
 Ti angyali lelkek, ma zengjetek az Úrnak
 És vigadva örvendjetek, buzgó hívek!
 A magas mennyben dicsőség Istennek!
 Ó, jöjjetek, imádjuk,
 Ó, jöjjetek, imádjuk,
 Ó, jöjjetek, imádjuk az Úr Krisztust.
/4
#2E4FB4E2
 Úr Jézus, ki ez napon érettünk születtél,
 Csak tégedet illet szívünk tisztelete!
 Isteni Gyermek, testet öltött Ige!
 Ó, jöjjetek, imádjuk,
 Ó, jöjjetek, imádjuk,
 Ó, jöjjetek, imádjuk az Úr Krisztust!

;„Quem pastores” kezdetű, XIV. századi himnusz
>328
/1
#DD1E353B
 Jöjjetek Krisztust dicsérni,
 Bízó szívvel hozzá térni,
 Énekekkel zengve kérni,
 Krisztus népe, jöjjetek.
/2
#B5C8337D
 Bűn, pokol már búban éljen,
 Ördögöt hadd ölje szégyen,
 Üdvösségünk szent ölében
 Már levetjük mind a bút.
/3
#6ECF759A
 Küldte Őt az Úr kegyelme
 Öröklétre, győzelemre,
 Hogy szívünket felemelje
 Boldogságos ég felé.
/4
#D7BB2F1E
 Irgalommal szánva minket,
 Nagy jósága ránk tekintett,
 S ördögcsalta bús szívünket
 Mennymagasból látni jött.
/5
#3469FC10
 Áldott óra, boldog óra,
 Nagy hitünknek meghozója,
 Ajkunk zengő hálaszóra
 Nyílik, édes Jézusunk.
/6
#6A2C7597
 Jászol-ölben drága Gyermek,
 Ég felé vigyen kegyelmed,
 Hol dicsérve énekelnek
 Édes hangú angyalok.

;J. S. Bach (1685–1750), Lipcse, 1736
>329
/1
#4C9A111E
 Itt állok jászolod felett, ó, Jézusom, szerelmem,
 Eljöttem, elhoztam neked, amit kezedből nyertem;
 Vedd elmém, lelkem és szívem,
 Hadd adjam néked mindenem,
 Hogy kedves légyek néked!
/2
#2896AE45
 Nem éltem még e föld színén: te értem megszülettél;
 Még rólad mit sem tudtam én: tulajdonoddá tettél;
 Még meg sem formált szent kezed,
 Már elválasztál engemet,
 Hogy társam légy e földön.
/3
#D9AE0735
 Halálban, éjben vártam én: fölkelt a nap rám véled.
 Terólad ömlik rám a fény: a béke, boldog élet,
 A lélek ékességei;
 Belőlük hitnek mennyei
 Szép tisztasága árad.
/4
#578F533C
 Csak nézlek boldog szívvel, és nem győzlek nézni téged,
 Szóm és erőm mind oly kevés, hogy elmondhassa néked:
 Bár felfoghatna tégedet
 Az emberszív és ismeret,
 Hogy megfejthesse titkod!
/5
#DFF14D71
 Megváltóm, egy kérésemet nem vetheted meg nékem:
 Hogy szívem mélyén tégedet hordozhatlak, remélem,
 És bölcsőd, szállásod leszek;
 Jövel hát, tölts el engemet
 Magaddal: nagy örömmel!

;Kolozsvár, 1744
>330
/1
#B43273F4
 Örvendezzen már e világ,
 Légyen mindenben vigasság,
 Mert Krisztus mindenért váltság,
 Világ bűnéért orvosság.
/2
#DBCB8F3E
 Zakariás szent próféta
 Már ezt régen megmondotta:
 Örvendezz, Sion leánya,
 És örülj nagy vigasságba'!
/3
#3790FE42
 Ne félj, mert íme örömed:
 A te királyod jő neked;
 Szamár vemhén telepedett,
 Mutatván nagy szelídséget.
/4
#F2054C0B
 Véle vagyon istensége,
 mondhatatlan nagy kegyelme,
 Jóvolta, idvezítése,
 Mert ő mindeneknek feje.
/5
#A14DF795
 Elől s utól nagy serege
 Szentlélek teljességében
 Hozsánnát kiált a mennybe,
 Dávid fiának örömben.
/6
#CB0BAD4A
 Némelyek ruházatjokkal,
 Krisztus útját méltósággal,
 Mások hintik zöld ágakkal:
 Királyt tisztelnek azokkal.
/7
#B465E478
 Mi is azért királyunknak,
 Menjünk elébe Urunknak;
 Vigyünk szép pálmaágakat:
 Hitünk győzedelmes voltát.
/8
#E22D5A17
 Dicsőség Atyánknak mennyben,
 Mi királyunknak aképpen
 Szentlélekkel egyetemben
 Mostan és minden időben.

;Bourgeois L., Lyon, 1547
>331
/1
#1D406F12
 A nagy király jön: Hozsánna! Hozsánna!
 Zeng e kiáltás előtte, utána;
 Zöld ágakat szeldelnek útára,
 Békességet hoz népe javára.
 Áldott, aki jött az Úrnak nevébe'
 Általa léptünk az Isten kedvébe;
 Békesség ott fenn a mennyországban,
 Áldott az Isten a magasságban!
/2
#499EECD5
 Ó, szentegyház, hívek boldog országa!
 Mily édes ez a Jézus királysága!
 Szelíd, szegény ez és alázatos,
 De nagy hatalmú és csodálatos.
 Igaz ez és a bűnből szabadító,
 A bűnt, halált és népeket hódító;
 Vasvesszővel bírja ellenségét,
 De szelíden őrzi örökségét.
/3
#D52A76F5
 Jézus király és magát annak vallja,
 De hogy királlyá tegyék, nem javalja;
 Sőt, noha Isten, Sion királya:
 Lett az időben szolgák szolgája.
 Jön szamárháton alázatossággal,
 Aki pedig bír az egész világgal;
 Nem jön királyi fényes bíborban,
 Nem fegyverekkel zörgő táborban.
/4
#5C325628
 Ó, édes Jézus, Atyádnak szent Fia!
 Ó, Isten, néped kegyelmes királya!
 Vezéreld jóra egész éltünket,
 Tégy tulajdon népeddé bennünket.
 Légy segítségül, ki a magasságban
 Ülsz drága véreden szerzett országban;
 Tégy engedelmes, hű polgárokká
 S nyert kincseidben birtokosokká.
/5
#AA586227
 Hogy csak a Jézus és az ő szent Atyja
 Törvénye légyen és szent akaratja
 Cselekedetink zsinórmértéke:
 Áldott királyunk királyi széke
 Hű szíveinkben légyen felemelve,
 És hűségére életünk szentelve;
 Tégyen méltóvá a Jézus vére
 A boldog lelkek lakóhelyére.

>332
/1
#0934B700
 Jézusom, imádjuk szent nevedet,
 Áldjuk erős szerelmedet,
 Aki öröktől fogva szerettél,
 BŰnünkért emberré lettél,
 A büntetést kiálltad helyettünk,
 És meg is haltál érettünk.
/2
#72371878
 A mindennel biró szegénnyé lett;
 Küszködik az örök élet
 A rettentö halál fájdalmival,
 Harcol kemény kínjaival:
 Halála nemétől iszonyodik,
 Vért izzad és szomorkodik.
/3
#AC412FB2
 Ím kiadá Isten édes Fiát,
 Hogy fizesse a bűn dlját,
 És minden bűnt magára vállala
 Az, aki bűnt nem tud vala;
 Ki által ég és föld formáltatik,
 Emberek által kínzatik.
/4
#D40F3DD9
 Aki fog ítélni mindeneket,
 Holtakat, eleveneket,
 Azt most halandó birák büntetik,
 A halálra ítéltetik.
 Felbomol a természetnek sora:
 Meghal az életnek Ura.
/5
#30CC4BBB
 Bújj el és rejtsd el tekintetedet,
 Fényes nap: aki tégedet
 Teremtett, szenved; testbe öltözött
 Isten függ az ég s föld közötti
 Föld, rendülj, és sírt készíts testének,
 Ennek az Isten Szentjének.
/6
#0334CA88
 Ó, isteni végtelen szeretet
 Imé, mivel nem lehetett
 Más mód, hanem ártatlannak kellett
 Szenvedni a bűnös helyett:
 Hogy én lehessek újra ártatlan,
 Meghalt ím a Halhatatlan.
/7
#68019FB0
 Ez tészi végtelen érdeművé,
 Győzhetetlen erejűvé
 Azt az embertől ki nem telhetett,
 De Istentől rendeltetett
 Teljes elégtételt, mely bűnünket,
 Elveszi bűntetésünket.
/8
#8FA77B23
 Nyerj nékünk is, Jézus, érdemeddel,
 Véres elégtételeddel,
 édes Atyádtól, a megbántatott
 Istentől bűnbocsánatot;
 Társaidnak híveidet vedd be
 Mennyei örökségedbe.

>333
/1
#8661CBCC
 Buzdítsd fel Uram, lelkemet,
 Hogy dicső fejedelmemet,
 Kit, míg a mérték bételt,
 Sok szenvedés felszentelt,
 Kísérjem a Golgotára,
 Hol igazságod oltára,
 De a körül kegyelmednek
 Jó illatjai terjednek.
/2
#907F424F
 Te, kit az égből lehallott
 Szó Isten fiának vallott,
 Melyet megerősítél,
 Mikor csudákat tettél:
 Rád nem jöhetett a vádban,
 Hogy álnokság volna szádban;
 Nem vala több célod ennél,
 Hogy széjjel járván, jót tennél.
/3
#12D39F6F
 Mégis, mint egy nyilvánlévő
 Hitető, vagy gonosztévő,
 Halálra kerestetel,
 Törvénybe idézteteI.
 Ellened az ütött pártot,
 Ki veled egy tálba mártott,
 Árát felvette vérednek,
 Már utánad leselkednek.
/4
#DC35CF29
 Sőt kiment a szentencia,
 Hogy vagy a halálnak fia;
 A töviskoronával
 S a megátkozott fával,
 Amelyet válladra tőnek,
 A feszítők előjőnek;
 Vernek és csúfolnak egyre,
 Mig kivezetnek a hegyre.
/5
#230B3BDE
 Itt tested dárdával szúrva,
 Kezed, lábad szeggel fúrva,
 Végig a magas kereszt
 Oldalain vért ereszt,
 Míglen, lehajtván fejedet,
 Atyádnak ajánlt lelkedet
 Kiadod, leírhatatlan
 Kínok között, ó ártatlan.
/6
#B22B396E
 Már ezt a nap sem állhatja,
 Megsötétül ábrázatja;
 Hát én rád hogy nézhetek?
 Ó, Megváltóm, reszketek,
 Mert a világ bűne terhét,
 Melyet most a bíró rád vét,
 Sokkal neveltem éltemben:
 Te szenvedsz az én képemben.
/7
#B089640D
 Uram, e szent vért tekintsd meg.
 Lelkem ez által tisztítsd meg;
 Ne legyek én átkozott,
 Ha kezesem áldozott
 Én ez oltárt megölelem,
 Ha szorongat a félelem,
 Biztatásomat halálom
 Óráján ebben találom.

;Bourgeois L., Genf, 1551
>334
/1
#527F41E8
 Ó, Isten, ki a törődött
 Szívet meg nem utálod,
 Sőt a bánatból ejtődött
 Könnyeket megszámlálod:
 Kedveljed érzésimet
 És elmélkedésimet,
 Melyek szívemben támadnak
 Halálán te szent Fiadnak.
/2
#34843D4D
 Mint sír a kertben magában,
 Mint küszködik a harccal,
 Mely a sötét éjszakában
 A földre ejti arccal,
 Hol kínjait lelkében
 Érezvén és testében
 Vércseppel verejtékezik,
 Elalél és csüggedezik.
/3
#5675C409
 De új erőt vesz magának,
 Lelkét megbátorítja,
 Mint a vitéz, ha harcának
 Kezdete tántorítja.
 Az eláruló csókra
 Önként megy a kínokra
 A fegyverekbe öltözött
 Vérengző kísérők között.
/4
#DC562B44
 A bíró bár elismeri,
 Hogy ő vétkét nem látja,
 Elereszteni nem meri,
 Ostor alá bocsátja.
 Látja a nép véresen,
 Nincs szíve, mely megessen,
 Fakad ily gyilkos lármára:
 Feszítsd fel a keresztfára!
/5
#D2ADDB0A
 Függ már a fán kiterjesztett
 Kezekkel felszegezve,
 A föld testéből eresztett
 Vérrel van béfedezve.
 Elalélván végtére
 A kínok érzésére,
 Meghal, Atyjához sóhajtva,
 Fejét keresztjére hajtva.
/6
#D0ABC7D1
 Ó, kínok közt elenyészett
 Megváltóm a keresztfán!
 Téged az egész természet
 Búslakodva sirat s szán.
 Reng a föld alkotmánya,
 A holtakat kihányja;
 A fényes nap is bújába',
 Borul sötét éjszakába.
/7
#82F9A0BF
 Elhal énbennem is a szív
 Ez iszonyú látásra,
 Bánatba merült lelkem hív
 Bűneimért sírásra,
 Mert tudom, hogy helyettem,
 Ki sokképpen vétettem,
 Szenvedéd a vereséget,
 Mely így vet éltednek véget.
/8
#6EE9114F
 Bízom más részről, mert a hit,
 Hogy értem ontottál vért,
 Rettegésemben megenyhít,
 Tudván, hogy érdemedért
 Részeltetem az égben
 Ama nagy dicsőségben,
 Hol te, ki értem szenvedtél,
 Örök méltóságot vettél.

;Debrecen, 1774
>335
/1
#AA9A220B
 Ó, ártatlanság báránya,
 E világnak ki vagy ára,
 Megtartója, táplálója,
 Áldott légy, egek királya!
/2
#5EDA3BBC
 Hogy érdemlettük ezt tőled,
 Hogy miértünk ezt felvégyed?
 Hogy szent tested vereséget,
 Szenvedjen ennyi sérelmet?
/3
#FA4DF7EE
 Bár tiszta ártatlanság vagy,
 Káromlónak is mondanak,
 Ámde te mégis vesztegelsz,
 Ily hamis vádra nem felelsz.
/4
#F6A9FAD6
 Mivel hogy Isten fiának,
 Mondod magad Messiásnak:
 Káromlónak elneveznek,
 Nem hisznek téged Istennek.
/5
#2F8B31DE
 Áldott légy ezért, Jézusunk,
 Édes megváltó Krisztusunk,
 Fő tanítónk és orvosunk,
 Megszabadító Királyunk!

>336
/1
#8B12FB8D
 Dicsérd lelkem Istenedet,
 Ki úgy szeretett tégedet,
 Hogy szent Fia kínt szenvedett,
 Éretted megfeszíttetett.
/2
#8A341412
 Keresztfán bűnödért hala,
 Ő szent vérével áldoza,
 Halál tőriből oldoza,
 Pokoltól megszabadíta.
/3
#8882A487
 Mert az Isten őtet tevé,
 Ki bűnt nem tud vala, bűnné,
 Hogy mi lennénk ő általa
 Az Istennek igazsága.
/4
#C6939372
 Nem vala néki formája,
 Sem szép ékes ábrázatja;
 Testét vereségnek adta,
 Gyalázattól el nem vonta.
/5
#89AC31EC
 Emberek között nem kedves,
 Megvettetett és beteges,
 Sok kínokkal rettenetes,
 Szörnyű fájdalmakkal teljes.
/6
#AE954104
 Oly lett, mint ki előtt magát,
 Ember elrejti orcáját,
 Mert semminek ítéltetett,
 Csúf-beszéddel illettetett.
/7
#67EF6C88
 Azt állították jóllehet,
 Hogy Istentől megveretett,
 De a mi betegséginket,
 Ő viselte sérelminket.
/8
#727D5C3A
 Midőn megostoroztatott,
 Sebeivel meggyógyított,
 A mi sok áInokságinkért,
 Megrontatott bűneinkért.
/9
#55079F87
 Mi, mint juhok, elszéledtünk,
 A bűn útján eltévedtünk:
 Bűntetését bűneinknek,
 Szenvedte nagy vétkeinknek.
/10
#A541A2BD
 Noha ő nem cselekedett
 Semminémű hitlenséget,
 Álnokság szájában nem volt,
 Bűn benne nem találtatott:
/11
#388C23CF
 Mint a bárány, olyanná lett,
 Mely mészárszékre vitetett,
 S mint juh, őtet nyírők előtt:
 Megnémult, szája veszteglett;
/12
#86C28D3B
 Érettünk káromoltatott,
 És méreggel itattatott,
 Szörnyen megostoroztatott,
 Ruhája sorsra osztatott.
/13
#7F9C2C06
 És verejtékezett vérrel,
 Koronáztatott tövissel,
 Által veretett dárdával,
 így kiált fel nagy felszóval:
/14
#A6D82B1B
 Én Istenem, én Istenem!
 Miért hogy elhagyál engem?
 Kezedbe ajánlom lelkem,
 Immár letészem életem.
/15
#4CD00DCE
 A mennyei magas égben
 Istennek dicsőség légyen,
 Ki szent Fiát küldé értünk,
 Hpgy Megváltónk lenne nékünk.

;A Krisztusnak a keresztfán lett hét szavaiból”
;Kolozsvár, 1744
>337
/1
#B6226943
 Paradicsomnak te szép élő fája,
 Ó, kegyes Jézus, Istennek Báránya,
 Te vagy lelkünknek igaz Megváltója,
 Szabadítója.
/2
#904D5E27
 Értünk egyedül szörnyű kínt szenvedtél,
 Megfeszíttetvén töviset viseltél,
 Mi bűneinkért véreddel fizettél
 És megölettél.
/3
#845B5770
 Csudánkra vannak a te szép gyümölcsid,
 Nagy kínjaid közt való szent beszédid,
 Kiket keresztfán szóltál, szent mondásid,
 Hét szép szavaid.
/4
#9665CC1C
 Első szódban így könyörgél Istennek:
 ATYÁM, BOCSÁSD MEG BŰNÖKET EZEKNEK,
 MERT NEM TUDJÁK ŐK MOST, MIT CSELEKESZNEK,
 Kegyetlenkednek.
/5
#CCF13873
 Lőn második szód a szegény tolvajhoz,
 Bűnén kesergő s törődő latorhoz,
 Mondván: VELEM LÉSSZ MA PARADICSOMBAN,
 SZENT ORSZÁGOMBAN.
/6
#5545E3D4
 Keserves szívű szentséges anyádnak,
 Harmadik szódat nyújtád Máriának,
 És ugyanakkor szóltál szent Jánosnak,
 Mondván azoknak:
/7
#B1272F69
 ÍMHOL AZ ANYÁD, kedves tanítványom,
 ÍMHOL A FIAD, ASSZONY, már ajánlom,
 Gyámolítódul ezt a Jánost hagyom,
 Kegyesen adom.
/8
#F0D69309
 Ártatlan Bárány, keserűségedben,
 Negyed szód ez lőn nagy kísértetedben:
 ÉN ISTENEM, ÉN ISTENEM, ÜGYEMBEN
 MIÉRT HAGYÁL EL?
/9
#C7E89C41
 Jövendölések mind beteljesedvén,
 És minden dolgok már elvégeztetvén,
 Mondád ötödször: SZOMJÚHOZOM igen,
 Szíved epedvén.
/10
#C280011B
 Mikoron, Uram, a mérget elvévéd,
 Ottan hatodszor mondál ilyen igét:
 Bételjesedett és ELVÉGEZTETETT
 Váltságnak dolga.
/11
#5E0AFD64
 Reád érkezvén a szomorú halál,
 Hetedszer ily szót s imádságot mondál:
 ATYÁM, KEZEDBE TÉSZEM LE LELKEMET,
 Én életemet.
/12
#51B6B10E
 Édes Jézusunk, szenteld meg lelkünket,
 Hogy mi is megbocsáthassuk bűnüket
 Mindeneknek, kik ellenünk vétettek
 És elestenek.
/13
#B22D95AF
 Adjad, hogy mi is értük könyörögjünk,
 Téged követvén, szívből esedezzünk,
 Hogy sok szentekkel tehozzád mehessünk,
 Idvezülhessünk.
/14
#F9B8A6B2
 A pályafutást mi is elvégezvén,
 Lelkünket ajánlhassuk szent kezedben;
 Mint megváltottak, mondhassuk nagy szépen
 Éltünk végében:
/15
#56A68B80
 Hála légyen a mennybéli Istennek,
 Ki megváltója lőn bűnös embernek,
 És megszerzője szent békességünknek:
 Idvességünknek.

;Ősi ír dallam
>338
/1
#7078BFA0
 Lelki próbáimban, Jézus, légy velem,
 El ne tántorodjék tőled életem.
 Félelem ha bánt, vagy nyereség kísért,
 Tőled elszakadnom ne hagyj semmiért.
/2
#113B9717
 Ha e világ bája engem hívogat,
 Nagy csalárdul kínál hitványságokat:
 Szemem elé állítsd szenvedésidet,
 Vérrel koronázott, szent keresztedet.
/3
#A0F45271
 Tisztogass bár bajjal olykor engemet:
 Kegyelmeddel szenteld szenvedésemet;
 Bár e test erőtlen: te oltárodon
 Keserű pohárral, hittel áldozom.
/4
#CD00E1A0
 Ha halálra válik testem egykoron:
 Ragyogjon fel lelked e hitvány poron;
 Ama végső harcon rád bízom magam:
 Örök hajlékodba fogadj be, Uram!

;Lengyel dallam, 1559
>339
/1
#3AABFB8F
 Jézus, Istennek Báránya,
 Kínjaidat ég s föld szánja.
 A nap, a nap sötétté változik,
 A föld, a föld reng és ingadozik.
/2
#922B1B4E
 Hegyek, halmok süllyedeznek,
 A kősziklák repedeznek,
 Holtak, holtak
 Sírból feltámadnak,
 A szent, a szent
 Városban jelt adnak.
/3
#22506658
 Íme, a templom kárpitja
 Kettéhasad és megnyitja
 Helyét, helyét
 A szentek szentének,
 Jelét, jelét Jehova frigyének.
/4
#89151781
 Mindezekből, ó, mit értsünk?
 És szívünkbe vajh' mit véssünk?
 Isten, Isten
 Végtelen kegyelmét,
 Hozzánk, hozzánk
 Csuda nagy szerelmét.
/5
#C971F3EE
 Fiát a szeretet Atyja
 Kereszt kínjaira adja,
 Értünk, értünk
 Hogy eleget tégyen,
 Urunk, urunk S üdvözítőnk légyen.
/6
#47AB791A
 Köztünk van a Szentek Szentje:
 A híveknek ezt jelentse
 Az ép, az ép Kárpit hasadása,
 Földig, földig Kettéhasadása.
/7
#B9CF19FB
 Jézus, ki értünk szenvedtél,
 Hogy éljünk, halálra mentél:
 Néked, néked
 Szívből hálát adunk,
 Holtig, holtig
 Híveid maradunk.

;Crüger J., 1640
>340
/1
#0F42C298
 Te drága Jézus, mi történt tevéled,
 Hogy oly keményen sújt a zord ítélet?
 A szörnyű vétket el mivel követted?
 Mi volt a tetted?
/2
#D65F710D
 Megostoroznak, tövissel csúfolnak,
 Arcodba vágnak, gúnyolódva szólnak,
 Epét ecettel kínálgatni mernek,
 Keresztre vernek.
/3
#51DA7BC0
 Mondd, ennyi kínnak mi az eredetje?
 Jaj, vétkeimmel vertelek keresztre!
 Amit Te szenvedsz, Jézus, én okoztam,
 Fejedre hoztam.
/4
#75264F7E
 S mily büntetés, mit a világ Reád mért?
 A jó nyájőrző szenved a juháért;
 A bűnért, melyet szolgák elkövettek,
 Az Úr fizet meg.
/5
#5AEA760D
 Meghal a jó, ki hűség volt s alázat,
 Az él, ki Isten bántására lázadt;
 A vétkes ember sértetlen, s bilincsben
 Ott áll az Isten.
/6
#17220389
 Ó, mérhetetlen szeretet, csodás hit,
 Amely a kínok zord útjára rávitt!
 Én vígadozva élek és örömben,
 Te kín-özönben.
/7
#121A9AE9
 Ó, nagy Királyom, minden kor Királya,
 Hűségedet hogy hirdethesse hála?
 Nincs emberszív, melyben tanács fakadhat:
 Néked mit adhat?
/8
#6969312A
 Ha trónusodnál, Jézusom, Vezérem,
 Fejem ragyogva fürdik majd a fényben:
 Énekelek, hol szentül zeng az ének,
 Dicsérve Téged.

;Hassler H.L., 1601
>341
/1
#E1F75D23
 Ó, Krisztusfő, te zúzott,
 Te véres szenvedő,
 Te töviskoszorúzott
 Kigúnyolt drága fő,
 Ki szépség tükre voltál,
 Ékes, csodás remek,
 De most megcsúfolódtál:
 Szent fő, köszöntelek!
/2
#7303EAC2
 Ékességed, te drága,
 Melytől máskor remeg
 Világ hatalmassága,
 Köpés mocskolta meg.
 Milyen halványra váltál!
 Szemed fényét, amely
 Szebb volt minden sugárnál,
 Ki rútította el?
/3
#F6471CFF
 Mind, ami kín, ütés ért,
 Magam hoztam Reád;
 Uram, e szenvedésért
 Lelkemben ég a vád.
 Feddő szót érdemelve
 Itt állok én, szegény,
 S kérlek, lelked kegyelme
 Sugározzék felém.
/4
#24B671A8
 Itt állok — ó, ne vess meg —
 A gyötrelmek helyén;
 Amíg ki nem hűl tested,
 El nem mozdulok én.
 S ha életed kilobban,
 Alácsuklik fejed,
 Ölemben és karomban
 Lesz nyugtató helyed.
/5
#9D00EB19
 Ó, légy érette áldott,
 Jézus, Egyetlenem,
 Hogy szörnyű kínhalálod
 Nagy jót akar velem.
 Add, hogy hódolva híven
 Tőled ne térjek el,
 S ha hűlni kezd a szívem,
 Benned pihenjek el.
/6
#8B2F3079
 Mellőlem el ne távozz,
 Ha majd én távozom,
 A kínban, mit halál hoz,
 Állj mellém, Jézusom.
 Ha lelkem félve reszket,
 S rettent a meghalás,
 Nagy kínod és kereszted
 Legyen vigasztalás.
/7
#F8E61354
 Légy pajzsom és reményem,
 Ha kétség látogat,
 Véssem szívembe mélyen
 Kereszthalálodat.
 Rád nézzek, Rád szünetlen,
 S ha majd szívem megáll,
 Öleljen át a lelkem -
 Így halni: jó halál.

;Cantus Catholici, 1561
>342
/1
#B1F41E8E
 Jézus, világ megváltója,
 Üdvösségem megadója,
 Megfeszített Isten Fia,
 Bűnömnek fán függő díja:
 Jézus, engedd hozzád térnem,
 Veled halnom, veled élnem.
/2
#CF916F51
 Szent kereszteden kereslek,
 Szomorú szívvel szemléllek,
 Mert így gyógyulást reménylek:
 Moss meg szent véredben s élek.
 Jézus, engedd hozzád térnem,
 Veled halnom, veled élnem.
/3
#9729A6F8
 E keresztről, én Reményem,
 Tekints reám szerelmesen,
 Épen téríts hozzád engem,
 Mondván: „bűnöd elengedtem”.
 Jézus, engedd hozzád térnem,
 Veled halnom, veled élnem.
/4
#566C8E84
 Ne gerjedezz vétkem ellen,
 Sőt véreddel kegyelmesen,
 Ily mocskos beteget, híven
 Moss meg, s bűntől üres lészen.
 Jézus, engedd hozzád térnem,
 Veled halnom, veled élnem.
/5
#279CBD78
 Engem ily nagy szerelmedből,
 Végy hozzád szent kegyelmedből,
 Vegyek erőt keresztedből,
 És végbúcsút bűneimtől.
 Jézus, engedd hozzád térnem,
 Veled halnom, veled élnem.
/6
#5E4E3A27
 Ajánlom magamat néked:
 Sebeidben szívemet vedd;
 Ó, nyíljál fel, piros forrás,
 mert nagy bennem rád a vágyás!
 Jézus, engedd hozzád térnem,
 Veled halnom, veled élnem.
/7
#9AF48647
 Ez keserves halálodért,
 Melyet felvettél éltemért,
 Vedd szívemet mindezekért,
 Megérdemlett jutalmadért.
 Jézus, engedd hozzád térnem,
 Veled halnom, veled élnem.
/8
#6788400A
 Nyílj fel, édes szív rózsája,
 Jó illatú violája,
 Hogy lehessen maradása
 Szívemnek, s benned lakása.
 Jézus, engedd hozzád térnem,
 Veled halnom, veled élnem.
/9
#E18F10FA
 Engem, bűnöst, kérlek, ne hagyj,
 Halál rabját kínra ne adj;
 Sőt, ha eljő majd halálom:
 Szent jobbodra engedj állnom.
 Jézus, engedd hozzád térnem,
 Veled halnom, veled élnem.

;Freylinghausen énekeskönyve, 1704
>343
/1
#4E42A340
 Ó, Krisztus, láttam szenvedésed,
 S borzongásom véget nem ért,
 Jaj, hogy halálkín lett a részed
 Érettem, árva bűnösért.
/2
#C3EB74C5
 A természet velünk zokogja
 Halálodért fájdalmait,
 Elbújt a nap, gyászukba rogyva
 Siratnak választottaid.
/3
#72BD7885
 Szent áldozat, tedd, meg ne szűnjünk
 Kereszten látni tégedet!
 Szent véred mossa csak le bűnünk,
 Lelkünk előtt az tár eget.
/4
#8F41511C
 Jóságod mély és mély a hála,
 Amellyel hozzád fordulunk;
 Bűnünkért mentél kínhalálba,
 Egyetlen Megváltó Urunk.

;Középkori himnuszdallamból
>344
/1
#EE80E07D
 Királyi zászlók lobognak,
 Fénylik titka keresztfának;
 Az élet ottan halni tért,
 Holtával nyervén drága bért.
/2
#FB2D5212
 Átverve szeggel tagjai,
 És kinyújtva szent karjai;
 Idvességünknek ára lett,
 Bűnünkért ő tett eleget.
/3
#3FB02DF2
 Dávidnak bétölt írása,
 Minden népeknek mondása:
 Isten, kit felfeszítetek,
 Úr lesz e fán felettetek.
/4
#E2CE96D6
 Légy áldott, oltár s áldozat,
 Mellyel Krisztusunk áldozott:
 Holton az élet elesett,
 S holtával adott életet.
/5
#D7BEEA4C
 Élő kútfő, Szentháromság!
 Téged áldunk, legfőbb Jóság!
 Kérünk: a kereszt győzelmét,
 Közöljed vélünk érdemét!

;Kálmán Farkas, 1838-1906
>345
/1
#E64FE882
 Ím, nagy Isten, most előtted szívem kitárom,
 Menedékem nincs sehol e földi határon;
 Ha te nem jössz bánatomra biztató szóval,
 Italom könny, a kenyerem keserű sóhaj.
/2
#B4624A1A
 Ha a világ nem tudná is számos bűnömet,
 Teelőled elrejtenem semmit sem lehet;
 Látja Lelked minden bűnöm, melynek átka sújt:
 Vedd le rólam, ó, Úr Isten, vedd le ezt a súlyt!
/3
#57DB9582
 Jézusomra föltekintek a kereszt alatt,
 Nincs szívemnek nyugodalma vétkeim miatt;
 Ó, ne büntesd, Uram, azt, kit megtört a bánat:
 Szálljon reám irgalmadból béke, bocsánat!
/4
#568BE99B
 Szent Fiadért, ki engemet vérén megváltott,
 Hallgass meg, ha bűnbánattal hozzád kiáltok!
 Vigaszoddal térj kegyesen beteg szívemhez,
 Hozzád térő gyermekednek, Atyám, kegyelmezz!

;Magyar dallam
>346
/1
#888D6A50
 Győzhetetlen én kőszálom,
 Védelmezőm és kővárom,
 A keresztfán drága áron
 Oltalmamat tőled várom.
/2
#DDF50536
 Sebeidnek nagy voltáért,
 Engedj kedves áldozatért,
 Drága szép piros véredért,
 Kit kiöntél ez világért.
/3
#125B8A15
 Reád bíztam én ügyemet,
 Én Jézusom, én lelkemet,
 Megepedett bús szívemet,
 Szegény árva bús fejemet.
/4
#4665C0DC
 Irgalmazz meg én lelkemnek,
 Ki vagy ura mennynek, földnek,
 Könyörgök csak Felségednek,
 Én megváltó Istenemnek.
/5
#AF6334FB
 Mutass, Jézus, kies földet,
 Lakásomul adj jó helyet,
 Ez életben csendességet,
 Jövendőben idvességet.

;Bourgeois L., Genf, 1551
>347
/1
#27DE05D3
 Jézus, ki a sírban valál,
 Általad megholt a halál,
 Az élet pedig feltámadott,
 Mert szent tested meg nem rothadott.
 Él a Jézus, a mi fejünk,
 Keresztyének, énekeljünk,
 Ülvén húsvét ünnepeket,
 Új győzedelmi éneket.
/2
#7DB27BD1
 Hol van, koporsó, hatalmad?
 Elveszett nyert diadalmad.
 Hová lett, ó, halál, a fúlánk,
 Melyet fensz már régóta reánk?
 Már nem rettegünk miatta,
 Mert Jézus meghódoltatta
 Ama félelmek királyát,
 megnyitván sírjának száját.
/3
#AA6D5C14
 Nincs már szívem félelmére
 Nézni sírom fenekére,
 Mert látom Jézus példájából,
 Mi lehet a holtak porából.
 Szűnjetek meg, kétségeim,
 Változzatok, félelmeim,
 Reménységgé, örömökké,
 Mert nem alszom el örökké.
/4
#68A595D7
 Sőt hiszem, hogy e tört cserép
 Edény leend még egyszer ép,
 És tetemim megépíttetnek,
 Bár veséim megemésztetnek.
 Gyalázat elvettetésem,
 De pompás lesz kikelésem,
 Új eget látván ezekkel
 Az újra megnyílt szemekkel.
/5
#35A4A6E4
 Jézus, segíts engem ebben,
 Hogy éltem folyjék szentebben,
 És hogy ne menjek ítéletre,
 Támassz fel engem új életre.
 A te lelkednek ereje
 Az új életnek kútfeje;
 Hogy hadd legyek élő személy,
 Lelked által énbennem élj!

;Genf, 1562
>348
/1
#2250DAC8
 Örvendezzetek, egek,
 Ti is, földi seregek!
 Mindnyájan örüljetek,
 Vígan énekeljetek,
 Mert Urunk feltámadott,
 Nékünk életet adott.
/2
#9002D1F8
 Jézus él, mi is élünk,
 A haláltól nem félünk,
 Mert legyőzte a halált,
 Örök váltságot talált
 Isteni erejével,
 Hathatós érdemével.
/3
#D7EF473A
 Nékünk megigazulást
 És a bűnből gyógyulást,
 Istennel békességet
 És boldog reménységet
 Nyert feltámadásával,
 örök igazságával.
/4
#09632401
 Előtted arcra esünk,
 S kérünk, édes kezesünk:
 Részeltess halálodnak
 És feltámadásodnak
 Drága érdemeiben,
 Édes gyümölcseiben.
/5
#2D6789D2
 Cselekedd Szentlelkeddel,
 Végtelen érdemeddel,
 Hogy új életet éljünk,
 Végre porból felkeljünk
 Örök, nagy boldogságra
 És halhatatlanságra.

;Bourgeois L., Genf, 1551
>349
/1
#98D4F5B4
 Jézus meghalt bűneinkért,
 Harmadnap feltámadott;
 Mi megigazulásunkért
 Mindenről számot adott.
 Mennyei szent Atyjának,
 És ő igazságának
 A váltságot megfizette
 A bűnösöknek helyette.
/2
#3959D380
 Életét maga letette
 Önként és jó kedvéből,
 Azt maga ismét felvette
 Isteni erejéből.
 Ó, csudáknak csudája!
 Íme, az Isten fia
 Értem életet áldozott,
 Élővé holtból változott.
/3
#811307FD
 A meghalt Jézus vére szólt
 Nékünk oly drága dolgot,
 Mert e vér Isten vére volt,
 Bűnt, halált ezzel oldott.
 Vér, melynek nincsen mása,
 Ó, Betlehem forrása!
 Ó, Jézus egy áldozatja,
 Isten gyönyörű illatja!
/4
#FA9421AC
 Hogy ez kedves volt Istennek,
 Azt azzal megmutatta,
 Hogy szent testét ő Szentjének
 Sírban soká nem hagyta.
 És a halál kötelét,
 A koporsónak tőrét
 Harmadnap széjjel oldozta,
 Fiát fogságból kihozta.
/5
#46901AD9
 Jézus, én megholt életem,
 Jézus, feltámadásom,
 Benned bűntől mentté lettem,
 Benned igazulásom.
 Ó, adj hát segítséget,
 Lelki elevenséget
 Az első feltámadásra:
 Az új életben járásra.
/6
#AF651776
 Istenséged megmutatád
 Életedben, holtodban;
 Hatalmasan bizonyítád
 Sírból kiszállásodban.
 Ez isteni erővel
 Jövel, ó, Jézus, jövel!
 Adj új életet s meghalást,
 S majd második feltámadást!

;Debrecen, 1774
>350
/1
#29A71C5D
 Feltámadt a mi életünk,
 Vígan méltó énekelnünk,
 Úr Krisztust dicsérnünk,
 E szent napon is áldanunk,
 Angyalokkal őt imádnunk,
 Mint Urunkat, félnünk,
 Mert őt magasztalják
 Nap, hold s égi seregek,
 A mennyei szentek.
/2
#1F0A9E8F
 A földben minden gyökerek,
 Fáknak bimbói terjednek,
 Mezők megzöldülnek,
 Ég madarai zengenek,
 Fákon vígan énekelnek,
 Szárnyukon repdesnek;
 Minden illatozó fűvek
 Gyönyörködtetnek,
 Dicséretre intnek.
/3
#92267CD6
 Mert feltámadt ő igazán,
 Angyala jelenté nyilván
 Koporsónak jobbján;
 Tanítványival vigadván
 Megjelent Galileában,
 Pétert vígasztalván:
 Örvendj te, ki voltál
 Gyakran bűnért sírásban,
 Alázatosságban.
/4
#71A82A7E
 Dicséret a nagy Istennek,
 Életet ki nyert népének,
 A bűnös embernek,
 Őt részeltetvén egeknek,
 Gyönyörűségében minden
 Lakóhelyeinek;
 Lelki javaival
 Népét meglátogatja,
 Megtérését várja.

>351
/1
#A6526DF4
 Emlékezzünk ez napon
 Urunknak haláláról
 És feltámadásáról,
 Néki adván hálákat:
 Irgalmazz nékünk!
/2
#28C008BF
 A Krisztus feltámada
 Nékünk idvességünkre,
 És feltámadásával
 Nékünk hoza életet:
 Irgalmazz nékünk!
/3
#6737A761
 Ezen mi mind örüljünk,
 Az Úr Istent dicsérjük,
 És adjunk nagy hálákat
 Édes Idvezitőnknek:
 Irgalmazz nékünk!
/4
#984073A3
 Krisztus nékünk adaték,
 Igaz húsvéti bárány,
 Ki elmosá vérével
 E világnak bűneit:
 Irgalmazz nékünk!
/5
#EDC3CD31
 Krisztus nagy kint szenvede
 És eltörlé a halált
 Dicső győzedelmével
 Nékünk idvességünkre:
 Irgalmazz nékünk!
/6
#B47438EC
 Hála légyen tenéked,
 Mi édes Idvezitőnk,
 Hogy minket megmentettél
 Minden ellenséginktől:
 Irgalmazz nékünk!
/7
#A468C5B5
 Adjad te Szentlelkedet,
 Hogy ezeket hihessük
 És vallhassuk örökké
 Mindeneknek előtte:
 Irgalmazz nékünk!
/8
#C25E9BC6
 Adjad, hogy feltámadjunk
 Mi is minden bűnünkböl,
 És járhassunk előtted
 Új életben örökké:
 Irgalmazz nékünk!

>352
/1
#91010186
 Krisztus ím feltámada,
 Nékünk örömet ada,
 Vére reánk árada,
 Haszna rajtunk marada:
 Nyerj irgalmat közbenjáró Jézus Krisztus, szent Atyád előtt!
/2
#3CB966A6
 Krisztus meghalt bűnünkért,
 Feltámadott éltünkért,
 Mi vígasztalásunkért,
 Megigazulásunkért;
 Nyerj irgalmat közbenjáró Jézus Krisztus, szent Atyád előtt!
/3
#C948B8BA
 Krisztus kínja érdemünk,
 Szent halála váltságunk,
 Vére ártatlanságunk,
 Felkölte vígasságunk;
 Nyerj irgalmat közbenjáró Jézus Krisztus, szent Atyád előtt!
/4
#6823E230
 Ez az Isten Báránya,
 Váltságunknak aranya,
 Halálunknak halála,
 A kígyónak rontója;
 Nyerj irgalmat közbenjáró Jézus Krisztus, szent Atyád előtt!
/5
#ACD50C9A
 Az ő feltámadása
 Lőn pokolnak romlása;
 Az ő örömmondása
 Lelkünknek vidulása;
 Nyerj irgalmat közbenjáró Jézus Krisztus, szent Atyád előtt!
/6
#6BC2B703
 Ő megmente bűnünktől,
 A törvénynek átkától,
 Istennek haragjától,
 Az ördögtől, pokoltól:
 Nyerj irgalmat közbenjáró Jézus Krisztus, szent Atyád előtt!
/7
#0B9068A6
 Azért mi is bűnünkből,
 Feltámadjunk vétkünkből,
 Evilági kínunkból
 És mi nagy halálunkból;
 Nyerj irgalmat közbenjáró Jézus Krisztus, szent Atyád előtt!
/8
#6D0D4B0D
 Közbenjáró Krisztusunk,
 Egyetlenegy váltságunk,
 Kérünk, légy mi oltalmunk,
 Hogy végre boldoguljunk;
 Nyerj irgalmat, közbenjáró Jézus Krisztus, Szent Atyád előtt!

;Cantus Catholici, 1651
>353
/1
#DC423CAB
 Krisztus feltámada Igazságunkra,
 Utat szerze mennyországra,
 Örök boldogságra.
/2
#A1B510D5
 Mind e világ terhét Vállára vette,
 A hatalmas Atya Istent
 Értünk megkövette.
/3
#3B2C2E3D
 De lám, ezt nem érti
 A hálátlanság,
 Emberekre honnan szállott
 Ennyi nyomorúság.
/4
#48662BCF
 Azért nem fogadják
 Isten beszédét,
 Jóra intő szent Igéjét:
 Krisztust, idvességét.
/5
#0B3E61D6
 Krisztus feltámada — sokan kiáltjuk,
 De a bűnnek undokságát
 Mi meg nem utáljuk.
/6
#0DE94598
 Tudva, bűnben élünk,
 Semmit nem félünk,
 Azért a Krisztus halála
 Nem használ minékünk.
/7
#8D5342A5
 Támadjunk fel testben
 Azért a bűnből,
 Melyért mi kirekesztettünk
 A nagy dicsőségből.
/8
#1B15518D
 Vegyük nagy jó kedvvel
 Krisztus jóvoltát,
 Atya Isten előtt való
 kedves áldozatját.
/9
#36C57396
 Dicsőség mennyégben
 Az Úr Istennek,
 Atya, Fiú, Szentléleknek,
 Mindörökké, Ámen.

>354
/1
#33B1ED71
 A Krisztust megfeszíték
 Kegyetlen gonosz népek.
 Ki miértünk születék.
 Mennybül nékünk adaték:
 Kegyelmes Krisztus,
 Irgalmazz nékünk!
/2
#9768624B
 Halálának utána
 Harmadnap feltámada,
 Hogy miértünk meghala,
 Örök halált megrontá.
 Kegyelmes Krisztus,
 Irgalmazz nékünk!
/3
#F950C5DC
 Vagyon nagy bizodalmunk,
 Mert mi is feltámadunk,
 Krisztussal uralkodunk,
 Vele együtt vígadunk,
 Kegyelmes Krisztus,
 Irgalmazz nékünk!
/4
#825B7237
 Reménységünk nagy legyen,
 Hitünk gyümölcsös legyen,
 Más istenünk ne legyen,
 Mely a pokolra vigyen:
 Kegyelmes Krisztus,
 Irgalmazz nékünk!
/5
#85C82686
 Szálla alá poklokra,
 Sokaknak váltságára,
 Poklot hagyá bánatba',
 Mert nincsen bizodalma,
 Kegyelmes Krisztus,
 Irgalmazz nékünk!
/6
#4AB5C33C
 Igazságot szeressünk,
 Hamisságot ne tegyünk,
 Szegényeket segéljünk,
 Csak a Krisztusban higgyünk:
 Kegyelmes Krisztus,
 Irgalmazz nékünk!

>355
/1
#612068BE
 Mély gyászba borult a természet,
 Nap, hold, minden csillagokkal;
 Dél, kelet, észak és enyészet
 Sírt a sok szent asszonyokkal,
 Mikor a Megváltó szenvedett,
 Feszíttetvén keresztfára,
 Mikor meghalt s eltemettetett,
 Hogy lenne váltságunk ára.
/2
#D8957F2F
 De e mély gyászos siralomnak
 Vége leve nemsokára,
 És az érzékeny közbánatnak
 Harmadnap lőn véghatára;
 Elmúlt a gyász, megszünt a jajszó,
 Az öröm visszaadatott,
 Hogy hallatott ez angyali szó:
 Nincs itt az Úr, feltámadott!
/3
#5DB73CF8
 Ó, hát ezen mi is örüljünk
 Az istenfélő hívekkel;
 E feltámadt Úrhoz készüljünk
 Buzgó és kegyes lélekkel;
 Keljünk fel a bűn halálából,
 Akik még abban heverünk,
 Sőt gyarlóságunk mély álmából,
 Uram, költs fel, arra kérünk I
/4
#883537D9
 Hogy így szent példádat követvén,
 Éljünk új és szent életet;
 Minden bűnünket hátra vetvén,
 Úgy nyerjük meg jó kedvedet;
 Végre rövid életünk után
 Egyességedben lehessünk,
 Az igazság egyenes útján
 Szent országodba mehessünk!

;Herman M., 1560
>356
/1
#D25A8E9B
 Felvirradt áldott szép napunk,
 Ma teljes szívvel vigadunk,
 Ma győz a Krisztus, és ha int,
 Rab lesz sok ellensége mind.
 Halléluja!
/2
#C8311213
 Az ősi kígyót, bűnt, halált,
 Kínt, poklot, szenvedés jaját
 Legyőzte Jézus, Mesterünk,
 Ki most feltámadott nekünk.
 Halléluja!
/3
#65CBDBA1
 Az élet győz, a mord halál
 A prédát visszaadta már,
 Nagy úrságának vége lett,
 Krisztus hozott új életet.
 Halléluja!
/4
#7752F793
 A nap s a föld s minden, mi él,
 Ma bút örömmel felcserél,
 Mert a világnak zsarnoka
 Nem kelhet többé fel soha.
 Halléluja!
/5
#8139C433
 Mi is éljünk vígadva hát,
 Daloljunk szép halléluját,
 Hadd zengje Krisztust énekünk,
 Ki sírból feltámadt nekünk!
 Halléluja!

;Berlin, 1653
>357
/1
#641C8EF5
 Jézus, én bizodalmam
 És Megváltóm életemben,
 Benned van nyugodalmam.
 Nem kell semmitől rettegnem,
 Halál nagy éjszakája,
 Bár rettentsen fúlánkja.
/2
#ADB88BD7
 Jézus, én Megváltóm, él,
 Én is látom az életet:
 Lészek Idvezítőmnél,
 Immár semmi sem rettenthet:
 Ő a Fő, és nem hagyja,
 Hogy elvesszék egy tagja.
/3
#75B99C50
 Reménység kötelével
 Véle már összeköttettem,
 Erős hitem kezével
 Már őbelé helyheztettem;
 El nem szakaszt tőle már
 Sem élet, sem a halál.
/4
#FC178DE7
 Földi porból vett porom
 Végre ismét porba tér meg,
 De föltámaszt egykoron
 Megváltóm, hogy véle éljek
 Végtelen dicsőségben:
 Nála lészek mennyégben.
/5
#7C523597
 Ami fáj itt és sóhajt,
 Az ott lészen dicsőséges;
 Földből szépen ott kihajt,
 Mi itt ínségekkel teljes.
 Erőtelenségemet,
 Ott letészem bűnömet.
/6
#C63EA72E
 Bátorságban légyetek:
 Jézus hordoz, mint övéit,
 Hát ne keseregjetek:
 Krisztus újonnan megépít.
 Angyalának szavára
 Felkeltek nemsokára.
/7
#67E42A25
 Emeld fel hát lelkedet,
 hagyj el minden földi vágyat;
 Bízd rá arra szívedet,
 Kiből idvességed árad.
 Jézusnál tartsd kincsedet,
 Légyen Jézusé szíved!

;Debrecen, 1778
>358
/1
#0D0E710D
 Krisztus mennybe felméne,
 Hogy nékünk helyet szerzene,
 Atyjával megbékéltetne,
 Életre bévinne.
/2
#D2D01215
 Ó, mi kegyelmes Mesterünk,
 Emlékezzél meg mirólunk,
 Ki meghaltál volt érettünk,
 Légy jelen mivelünk!
/3
#8D74B234
 Te látod mennyből éltünket
 És nagy keserűséginket;
 Vigasztald meg mi lelkünket,
 Hogy higgyünk tégedet.
/4
#D6DD450A
 Mert megfogadtad minekünk,
 Hogy léssz örökké mivelünk;
 Adjad Szentlelked minekünk,
 Hogy benned hihessünk.
/5
#12A6ACDD
 És oltalmazz meg mindentől,
 Szent Atyádnak haragjától,
 Ördögtől és kárhozattól,
 Minden dühösségtől.
/6
#06C04D52
 Tekints nagy gyarlóságinkra,
 E világ csalárdságára;
 Vigy bé a nagy boldogságba,
 Te szent országodba.
/7
#E3A43288
 Egyetemben keresztyének,
 Az Úr Krisztust dicsérjétek,
 Őnéki hálát adjatok,
 Felmagasztaljátok.
/8
#4BD6212E
 Dicséret légyen Atyának
 És ő Fiának, Krisztusnak,
 És a mi Vigasztalónknak,
 A Szentháromságnak.

;(1539) Bourgeois L., Lyon, 1542
>359
/1
#188E841D
 Úr Jézus, aki felséggel,
 És dicsőséggel mentél égbe
 És ottan vettél hatalmat,
 Nagy birodalmat, teljességbe:
 Lelkünk áldja istenséged
 És híven téged magasztalunk,
 Bárha nem látunk szemünkkel,
 De hitünkkel megtapasztalunk.
/2
#9270C4A2
 Te is e dicsőségedből,
 Szent székedből fordítsd le szemed,
 Erőtlen teremtésidre,
 Híveidre öntsd ki érdemed.
 Mennyben létednek hasznában,
 Javaiban részesekké tégy;
 Szent Atyád előtt érettünk,
 Kik vétettünk, esedezőnk légy.
/3
#854B7B70
 Mivel te utat nyitottál
 És tanítottál mennybe menni,
 Adjad, hogy téged kövessünk
 És siessünk nyomodba' lenni.
 Segítsd igyekezetünket,
 Vond szívünket te magad után,
 Hogy a te akaratodnak
 S nyomdokodnak járhassunk útján.
/4
#83F1EE06
 Míg bujdosunk e pusztában
 S mint hazánkba, az égbe érünk,
 Légy kezesünk, védj bennünket,
 És hitünket neveljed, kérünk,
 Hogy a veszedelmek között
 Megütközött köztünk ne légyen,
 Hitünknek elevenségét,
 Reménységét ne érje szégyen.
/5
#A8F82C66
 Majd ha megfutjuk pályánkat,
 Várt pálmánkat a kezünkbe add,
 Lelkünket, ó, szerelmesünk,
 Hű kezesünk, magadhoz fogadd!
 Testünket is emeld végre
 Dicsőségre isteni karral,
 Hol tégedet szemlélhessünk,
 Dicsérhessünk az égi karral.

>360
/1
#689856C6
 Szent egek, minden boldogok hajléki,
 A nagy király jön, nyíljatok meg néki!
 Nagy oszlopi a levegő égnek,
 Engedjetek útat e felségnek!
 Aki kezdetben titeket teremtett,
 Aki azt mondá: légyen ég és föld, s lett;
 Aki Igéje a nagy Istennek,
 Aki által lettenek mindenek.
/2
#ADE7BD8D
 Ő a békesség örök tanácsában,
 Mint aki legfőbb a váltság titkában,
 Emberré lett a teljes időben,
 A bűn terhét szenvedte testében.
 Meghalt, áldozván halálos kínjával,
 De fel is támadt mága hatalmával;
 Ím, győzedelmes bajnok módjára
 Felmegy és ül szent Atyja jobbjára.
/3
#9ED19FBB
 Mennybe menése az ember Fiának
 Záloga lelkünk mennyei jussának.
 Melyet szerzett, mint kezes, vérével
 És közöl minden igaz hívével.
 Ő emberré, mi atyánkfiává lett,
 Atyjának minket örökösivé tett;
 Maga előttünk mennybe felmene,
 Nékünk is helyet hogy készítene.
/4
#3C8B40F2
 Ó, kegyes Jézus, idvességünk ára,
 Vállalt kezes, ki szent Atyád jobbjára
 Magasztaltattál, te, ki örök, nagy.
 Atyáddal egy és örök Isten vagy:
 Arra szereztél útat érdemeddel,
 Isten Atyánktól nyert örökségeddel,
 Hogy dicsőséged lakóhelyébe
 Mégy könyörülő főpap képébe'.
/5
#E50F22A9
 Áldunk s imádunk mindörökké téged
 és dicsőítjük örök istenséged;
 Jó vagy, Uram, te, igen irgalmas,
 Pap és király vagy, kegyes, hatalmas.
 A te érdemed a mi nyereséünk,
 mennybe menésed élő reménységünk,
 Hogy minket is, mint örökségedben,
 Részeltetsz megnyert dicsőségedben.
/6
#3C22C8FE
 Adjad, Úr Jézus, hogy még evilágban
 Sátorozván is, ama mennyországban
 Részt vehessünk, s mint tulajdonunkkal,
 Szenttül éljünk mennyei jussunkkal,
 És hogy e földet jó cselekedettel,
 Hálaadással, buzgó tisztelettel,
 Mint lakos társai a szenteknek,
 Pitvarivá tégyük az egeknek.

>361
/1
#B0A2C4C0
 Vedd el jutalmadat Krisztus, ki magadat
 Megüresítéd valal Harcodnak vége lett,
 Minden egek felett Az Úr felmagasztala.
 E példán minden szent Jutalmat vár ott fent,
 Bár itt nem él boldogul, Mert, hogy felemeled,
 Mennybe meneteled Adtad erre zálogul.
/2
#85B98E2E
 Jobbján szent Atyádnak Tégedet imádnak
 A mennyország polgári; Látják dicsőséged,
 Mely körülvesz téged, E föld minden határi.
 Mi is leborulunk, Kérvén: tégy ott rólunk
 Szent Atyád előtt vallást! Félelminket vedd el,
 Esedezéseddel Nyerj nekünk vígasztalást!
/3
#225215FB
 Hiszünk, Uram, benned, Bár el kelle menned
 S nem látunk testi szemmel. Ó, viselj a mennyből
 Gondot híveidről, Hű pásztor, kegyelmeddel.
 Légy ott, hol nevedbe' Gyülekezik egybe
 Két, avagy három híved; Ha hozzád óhajtunk,
 Essék meg mirajtunk Te könyörülő szíved.
/4
#E6D03A2C
 Midön végezetre Eljössz ltéletre
 Az égnek felhőiben, ülvén méltósággal
 Szemben a világgal Dicsőséged székiben:
 Szólj nekünk ily módon, Kik állunk jobbodon,
 Mint nyájadnak juhai: Kész hajlékim rátok
 Várnak, ó, bírjátok, Jó Atyám áldottai!

>362
/1
#9CC4197E
 Ébredjél fel világ bűneidből,
 Vedd eszedbe magad a nagy bűnből,
 Melyet néked utolsó időkről,
 Magyarázott Jézus Krisztus ő Ítéletéről.
/2
#4DB2E520
 Minden írások immár bétöltek,
 A próféták valamit hirdettek,
 A szent atyák amit reménylettek,
 Urunk Jézus Krisztus által véghez vitettenek.
/3
#C3FAF36A
 Nem késik már az ő eljövése,
 Közel vagyon e világnak vége,
 Minden dolgoknak már eljövése;
 Krisztus Jézus e világra eljő ítélnie.
/4
#F06B7965
 E világnak istentelensége,
 Az uraknak nagy kegyetlensége,
 A községnek engedetlensége
 Jele bizony, hogy közel már Krisztus eljövése.
/5
#2D14D42D
 Minden rendek vagynak bátorságban,
 Gyönyörködnek csak az álnokságban;
 Szeretet nincs az atyafiakban,
 És kevesen vannak immár, kik hisznek Krisztusban.
/6
#9FA65B29
 Esznek, isznak, nagy bátran lakoznak,
 Mennyországról de nem gondolkoznak;
 Elmúlandó dolgokban forgódnak;
 Azt sem tudják, mely órában azoktól megválnak.
/7
#C13A7BF8
 Siketségre vesznek minden intést,
 Megvetnek már minden jó rendelést;
 A tanítók nem látnak épülést,
 Szentirásból mikor tesznek idvességes intést.
/8
#854BEA1B
 Adjunk hálát az Atya Istennek,
 Dícséretet mondjunk Szent Fiának,
 Véle egyetemben Szentléleknek:
 Mindörökkön és örökké a Szentháromságnak.

>363
/1
#648C702B
 Jön a harag napja lánggal,
 Végez majd a rossz világgal,
 Mondja Dávid a szent Pállal,
/2
#9AD12B51
 Hogy remeg majd holt, meg élő,
 Hogyha jön a bűnítélő,
 Minden tettet számon kérő,
/3
#A67BD563
 Rémes hangja trombitáknak,
 Minden hanton holtig áthat,
 S felriasztva trónhoz állnak
/4
#973FFE70
 Természet és halál retten,
 hogyha kél a holt ijedten,
 Számot adni feleletben.
/5
#3C62964A
 Írott könyvet előadnak,
 Minden titkok benne vannak:
 Bűn könyve a bírság napnak
/6
#35FD0E52
 Már ha Bíró ül a székben,
 Nem maradhat bűn sötétben,
 Nincs kivétel büntetésben
/7
#6DED50CF
 Mit mondok majd, árva lélek,
 Pártfogónak, jaj, kit kérek,
 Hogyha ott a jók is félnek?
/8
#64AAE95A
 Nagy Felség, ki úgy remegtet,
 S jónak ingyen ad kegyelmet:
 Irgalomnak kútja, ments meg!
/9
#BFF8B958
 Drága Jézus ne feledd el,
 Értem jöttél kegyelmeddel;
 El ne ejts a rossz sereggel!
/10
#8E8973B6
 Értem fáradt drága tested,
 Értem hordtál kínt s keresztet,
 Kárba munkád mégse veszhet!
/11
#FE661147
 Hű bírája bosszulásnak,
 Add, hogy bűnbocsátást lássak,
 Míg nem jő a számadásnap.
/12
#7956CF09
 Elbukott nőt feloldottál,
 Latornak is jó szót adtál,
 Engem is már megbíztattál.
/13
#4B9007D8
 Nyögve hordom annyi bűnöm,
 Szégyenkezni meg nem szűnöm:
 Lelked rajtam könyörüljön!
/14
#2C60FFEB
 Mind méltatlan, amit kérek,
 Mégse hagyd, hogy rosszat érjek,
 Örök tűzben el ne égjek!
/15
#E568B728
 Majd ha minden bűnös ott áll,
 Baktól juhnak elválasszál,
 Kobbkezedhez kérlek osszál!
/16
#8C1FEF96
 S majd ha szód lesz veszte soknak,
 S ítélt rosszak égni fognak:
 Végy rendjébe boldogoknak.
/17
#DB025155
 Esdekellek térden állván,
 Porrá morzsolt szívvel árván,
 Végnapomon jól vigyázz rám!
/18
#1C8C9954
 Az a könnynek lesz a napja,
 Melyen bírószó fogadja majd a sírból felkelőket,
 Istenem, ne vesd el őket!
 Kegyes Jézus el ne hagyj,
 Örök nyugodalmat adj! Ámen.

>364
/1
#C0AE2D8B
 Ments meg, Uram, engem az örök haláltól,
 Ama rettenetes napon minden bajtól!
 Midőn az ég és föld meg fognak indulni
 S eljössz a világot lángokban ítélni.
/2
#ED0A0E3F
 Reszket minden tagom, borzadok és félek,
 Én, e földön küzdő szegény bűnös lélek;
 Ama napon engem ítéletre vonnak,
 Midőn az ég és föld egyben megindulnak.
/3
#0BCF622E
 Haragnak napja az, kínok, ínség napja,
 Nagy nap, mely a bűnöst gyötri, szorongatja,
 Midőn te, ki mindig éltél és fogsz élni,
 Eljössz a világot lángokban ítélni.
/4
#EB412404
 Örök nyugodalmat adj, ó, Uram, nékünk,
 Örök világosság fényeskedjék nékünk,
 Hogy trónusod körül mi udvart állhassunk,
 S téged, boldogítót, örökké áldhassunk.
/5
#741922AB
 Ments meg, Uram, engem az örök haláltól,
 Ama rettenetes napon minden bajtól,
 Midőn az ég és föld meg fognak indulni
 S eljössz a világot lángokban ítélni.

>365
/1
#C2E86048
 Jézus Krisztus, mi királyunk,
 Teremtőnk és igazságunk,
 Megváltónk és vidámságunk,
 Ki testben lakozál nálunk:
/2
#28D89278
 Honnan érdemlettük tőled,
 Hogy bűneinket elvégyed,
 A szörnyű halált is szenvedd,
 E világot kedvedbe vedd?
/3
#8966D636
 Pokolra értünk leszállál,
 Fogságból kiszabaditál,
 A halálból feltámadtál,
 Mennyben nékünk helyt foglaltál.
/4
#80FCD4CC
 Kérünk azért, édes Urunk,
 Légy szószólónk és gyámolunk,
 És mikor innen kimúlunk,
 Adjad szent színedet látnunk.
/5
#9AA9EFA5
 Dicsőség légyen tenéked!
 Atyádnak együtt tevéled!
 Áldassék a te Szentlelked,
 Kit nálunk hagyál helyetted!

;Holéczy Sámuel, Psalmodia, 1818
>366
/1
#A356B33F
 Az én időm, mint a szép nyár,
 Menten lejár,
 Nem meszsze tőlem a vég.
 Ám a lélek el nem enyész,
 Sőt bére lész
 Jó vagy jaj: pokol vagy ég.
/2
#F4E0C2E8
 Nem szükség hát veszteg ülnöm,
 Kell készülnöm:
 Égjen jól a szövétnek,
 Mert majd amaz öt szüzekkel,
 Mint ezekkel,
 Vélem is számot vetnek.
/3
#C7E37DED
 Ó, Uram, hová kell lennem,
 Ha kell mennem
 Veled, bírámmal szemben!
 Már ezt is alig állhatom,
 Ha forgatom
 Ezt előre eszemben.
/4
#A73943D1
 A kürtzengés máris hallik,
 Már hajnallik,
 Útban már az angyalok.
 Életem csak egy tenyérnyi,
 S számon kéri
 Jézusom, ha meghalok.
/5
#5D01F8CE
 Taníts meg, Uram, hogy holtom
 S rövid voltom
 Soha el ne felejtsem,
 És a jövendő életet,
 Ítéletet
 Szívemből ki ne ejtsem.
/6
#86EFFCB8
 Lelkem, mondj le hát e testről,
 Jóra restről;
 Egy úton ezzel futnál?
 Kérlek, vedd ezt jól szívedre:
 E vér vedre
 Majd eltörik egy kútnál.
/7
#A9B01CC5
 Segíts ezt megfeszítenem,
 Ó, Istenem!
 Magam nem bírok véle,
 Mert ha ma megöldöklöttem,
 Ura lettem:
 Holnap felkél új mérge.
/8
#EEF514CA
 Uram, ha arra kell mennem,
 Hogy kell lennem
 Tanúnak a hit mellett:
 Láttass nyílt eget lelkemmel,
 Így testemmel
 Ám ne légyen kímélet.
/9
#A393EFE7
 Mindebből észbe vehetem:
 Harc életem,
 Sok ellenségim vernek,
 Lelkem sok ütközet vérzi,
 Ki nem érzi
 Vágtát ennyi fegyvernek?
/10
#A8E6FDEB
 Időm kevés, de sok a baj,
 Három a jaj,
 Ki ne kívánná végét?
 Boldog, aki pályát futott,
 Célra jutott
 S megtartotta hűségét.
/11
#8A0849CF
 Kezem én is feléd nyújtom,
 Szabadítóm,
 Jézusom, hozzád tartok!
 Bízzál, lelkem, nem süllyedsz itt,
 Kormány a hit,
 Várnak már a révpartok.

;Bernouilli F.
>367
/1
#75E47282
 Emeljük Jézushoz szemünk,
 Jön már királyi győztesünk,
 Mennyből leszáll s együtt leszünk.
 Lelkünk vigyázni meg ne szünjön,
 Felséges várástól feszüljön,
 Az álmot űzd el, készen állj,
 Krisztusnép, jön, jön a Király!
/2
#1512F1AD
 Azt mondta Jézus: Idelenn
 Új próba és új küzdelem
 A hívők sorsa szüntelen.
 Azért ne csüggedjünk, ne féljünk,
 Az út rövid, végére érünk.
 Az álmot űzd el, készen állj,
 Krisztus-nép, jön, jön a Király!
/3
#6F77A810
 Éneklünk és a perc szalad,
 A nap, mely nesztelen halad,
 Az öröklét felé mutat.
 De míg hangunk majd zengve szárnyal
 Hozsánnás angyal-kar szavával,
 Az álmot űzd el, készen állj,
 Krisztus-nép, jön, jön a Király!
/4
#CDCF58FD
 Ó, kérünk, Jézus, jó Urunk,
 Te szabd meg életünk, utunk:
 Hány éjszakánk lesz s hány napunk.
 Bölcs szívedből töltsd meg szívünket.
 Te ismered jól kis hitünket:
 Küldj, küldj szent sóvárgást nekünk,
 Hogy majd ha jössz, készen legyünk.

;Középkori latin dallamból, Debrecen, 1774
>368
/1
#9F43176A
 Jövel, Szentlélek Isten,
 Tarts meg minket Igédben,
 Ne légyünk setétségben:
 Maradjunk igaz hitben.
/2
#48814BC5
 Szenteld meg mi szívünket,
 Világosítsd elménket,
 Hogy érthessük Igédet,
 Mi édes Mesterünket.
/3
#73888984
 Adj isteni félelmet
 És bizonyos értelmet;
 Igéddel taníts minket,
 Gerjeszd fel mi szívünket.
/4
#13A32166
 Vigasztald meg elménket;
 Mindenben segíts minket;
 Öregbítsed hitünket,
 Távoztassad bűnünket.
/5
#9F5CA0A0
 Hogy téged az Atyával
 És az ő szent Fiával
 Dicsérhessünk mindnyájan
 A fényes mennyországban.

;Debrecen, 1774
>369
/1
#D03E6F85
 Jövel, Szentlélek Úr Isten,
 Lelkünknek vigassága,
 Szívünknek bátorsága,
 Adjad minden híveidnek
 Te szent ajándékodat;
 Jövel, vigasztaló Szentlélek Isten!
/2
#729B154F
 Jövel megnyomorultaknak
 Nemes vigasztalója,
 Árváknak édes Atyja!
 Töltsd bé siralmas szívünket
 Mennyei nagy örömmel.
 Jövel, mi lelkünknek Édes vendége!
/3
#3401623F
 Távoztasd el mi lelkünknek
 Hitetlen sötétségét;
 Világosíts meg minket,
 Hogy az Istennek igéjét
 Hallhassuk és érthessük!
 Jövel és taníts meg Az igaz hitre.
/4
#44FACEFF
 Jövel, gerjeszd fel szívünkben
 Szent szerelmednek tüzét,
 Rontsd el a gyűlölséget,
 Hogy mi egyesek lehessünk
 Isteni szerelmedben:
 Jövel, mi lelkünknek Nagy vigassága!
/5
#51AE260C
 Te vagy bizony örök Isten,
 Ki Atyától s Fiútól
 Mindörökké származol.
 Te vagy mi urunk Krisztusnak
 Áldott, szent ígérete:
 Te vagy mi lelkünknek Megszentelője.
/6
#0DDB8C3A
 Te vagy, ki a prófétáknak
 Általa régen szóltál,
 Krisztust nekünk ígéréd;
 Te a szent apostoloknak
 Szívüket bátorítád;
 Te vagy erőssége Minden szenteknek.
/7
#F870F89C
 Te vagy a nagy Úr Istennek
 Mennyei ajándéka,
 Igazságnak mestere.
 Taníts minket a Krisztusnak
 Igaz ismeretire!
 Te vagy mi lelkünknek Bölcs tanítója.
/8
#74E86803
 Világosítsd meg elménket,
 Hogy hihessük a Krisztust
 Egy idvességnek lenni,
 És az áldott Atya Istent
 Kegyes atyánknak lenni;
 Téged ismerhessünk Vigasztalónknak.
/9
#DDD9E80C
 Adjad szent ajándékodat,
 Bátorítsad lelkünket,
 Hogy vallhassuk a Krisztust!
 Adjad, hogy mi meggyőzhessük
 Az ördög csalárdságát!
 Te vagy mi biztatónk, Minden oltalmunk.
/10
#9D6E41AE
 Biztasd félelmes szívünket,
 Hogy kétségbe ne essünk
 Halálunknak idején,
 De nagy bátorsággal, vígan
 E világból kimúljunk!
 Jövel, vigasztaló Szentlélek Isten!

;Debrecen, 1774
>370
/1
#E2BFF5C8
 Jövel, Szentlélek Úr Isten,
 Töltsd bé szíveinket épen,
 Mennyei szent ajándékkal,
 Szívbéli szent buzgósággal,
 Melynek szentséges ereje
 Nyelveket egyező hitre
 Egybegyűjte sok népeket,
 Kik mondván, így énekeljenek;
 Alleluja! Alleluja!
/2
#47048535
 Te, szentségnek új világa,
 Vezérelj Igéd útjára,
 Taníts téged megismernünk,
 Istent atyánknak neveznünk.
 Őrizz hamis tudománytól,
 Hogy mi ne tanuljunk mástól,
 És ne légyen több más senki,
 Hanem Krisztus, kiben kell bízni!
 Alleluja! Alleluja!
/3
#5D402D96
 Ó, mi édes Vigasztalónk,
 Légy kegyes megoltalmazónk,
 Hogy maradjunk útaidban,
 Ne csüggedjünk háborúnkban.
 Erőddel elménket készítsd,
 Gyenge hitünket erősítsd,
 Hogy halál és élet által
 Hozzád siessünk hamarsággal!
 Alleluja! Alleluja!

;Debrecen, 1778
>371
/1
#05022C8E
 Jézus Krisztus, egy Mesterünk,
 Mennyei szent bölcsességünk,
 És nékünk bizonyos idvességünk.
/2
#3ABD1428
 Mostan néked mi könyörgünk,
 Szent nevedért esedezünk:
 A te Szentlelkedet adjad nékünk!
/3
#5625E83B
 Mi szívünket megújítsa,
 És sebeinket gyógyítsa:
 Bűnös életünket megjobbítsa.
/4
#E96AA284
 Munkáinkat megszentelje,
 Bennünk a hitet nevelje,
 Utunkat hazánkba vezérelje.
/5
#86D5322C
 Bölcsességre megtanítson,
 Hogy ördög meg ne csalhasson:
 Kísértések ellen bátorítson.
/6
#0DF762BD
 Plántáljon nagy egyességet,
 Igaz és szent szeretetet,
 Szívünkbe isteni szent félelmet.
/7
#A39072D5
 Hogy téged bátran vallhassunk,
 E földön néked élhessünk,
 Mennyben aztán veled lakozhassunk.

;Debrecen, 1774
>372
/1
#C2FBAB4A
 Könyörögjünk az Istennek Szentlelkének,
 Bocsássa ki magas mennyből fénylő világát,
 Végye el mi szívünknek minden homályát,
 Hogy érthessük Istenünknek Mindenben akaratját.
/2
#5E29745A
 Ó, Szentlélek, árváknak kegyelmes Atyja,
 Szegény gyarló bűnösöknek bátorítója,
 Hitükben tántorgóknak erős gyámola,
 És az ő reménységüknek Csak te vagy táplálója.
/3
#4A3524B9
 Te vagy a mi lelkünknek édes vendége,
 A mi szomorú szívünknek igaz öröme,
 Lelki háborúinknak csendesítője:
 Az örök életnek bennünk Csak te vagy elkezdője.
/4
#18B1F7F5
 Te tanítád régenten a prófétákat,
 Igazgatád ő nyelvüket és írásukat,
 Te tetted bölccsé a szent apostolokat,
 Hogy megtérítsék tehozzád Mind e széles világot.
/5
#DC1A9AC6
 Vedd el a mi szívünknek hitetlenségét,
 Világosítsd meg elménknek nagy setétségét,
 Rontsd el a gyűlölségnek kegyetlenségét,
 Engedd a te szent hitednek Mindenütt egyességét!
/6
#6F809FAA
 Válassz minket magadnak élő templomul,
 Végyed mi könyörgésünket szent áldozatul;
 Vedd ki már e világot a kárhozatból,
 Engedj igaz hitben való Kimúlást e világból.
/7
#FD37E1D1
 Dicsértessél, felséges Atya Úr Isten,
 Légyen áldott a te neved, Fiú Úr Isten,
 Ezekkel egyetemben, Szentlélek Isten:
 Maradjon a te áldásod A te bűnös népeden.

;Középkori dallamból, Wittenberg, 1524
>373
/1
#88349333
 Jövel, teremtő Szentlélek,
 És híveiddel légy vélek,
 Szent ajándékiddal szívek
 Újuljon és teljesedjék.
/2
#1F04C558
 Hathatós Vigasztalónak
 És Isten ajándékának,
 Avagy hét ajándékúnak,
 Isten jobb keze ujjának;
/3
#1BC30EB7
 Mondatol élő kútfőnek,
 Tűznek és lelki kenetnek,
 Atyának ígéretinek
 És az igaz szeretetnek.
/4
#C2CD3673
 Gerjessz világot elménkben,
 Önts szeretetet szívünkben,
 Erősíts minket hitünkben
 És nagy erőtlenségünkben.
/5
#8CE87089
 Adj nékünk teljes örömet,
 Idvességhozó kegyelmet;
 Köztünk minden gyűlölséget
 Ronts el, és adj egyességet.
/6
#E7A8E4E6
 Távoztasd ellenséginket,
 És add meg békességünket;
 Mindenkor vezérelj minket,
 Utálhassuk bűneinket.
/7
#A254E5C7
 Adjad ismernünk az Atyát,
 És az ő egyszülött Fiát,
 És hinnünk, hogy mindkettőtül
 Szentül származol, vég nélkül.
/8
#4136DAC6
 Dicsőség az egy Istennek,
 Atya, Fiú, Szentléleknek;
 Kedves ajándéka ennek
 Lakjék szívében mindennek.

;Debrecen, 1778
>374
/1
#F4A0924D
 A pünkösdnek jeles napján
 Szentlélek Isten küldeték,
 Megerősítni szívüket
 Az apostoloknak.
/2
#971052C5
 Melyet Krisztus ígért vala
 Akkor a tanítványoknak,
 Mikor méne mennyországba
 Mindenek láttára.
/3
#8AD0532C
 Tüzes nyelveknek szólása,
 Úgy mint szeleknek zúgása
 Leszálla az ő fejükre
 Nagy hirtelenséggel.
/4
#67010A68
 Bételvén ők Szentlélekkel,
 Kezdének szólni nyelveken,
 Amint nékik a Szentlélek
 Ad vala szólani.
/5
#39098607
 Örüljünk azért őneki,
 Mondván szép ékes éneket,
 Felmagasztalván ő nevét
 Örökkön-örökké.
/6
#B99038C3
 Dicsértessél Atya Isten,
 És megváltó Fiú Isten,
 Szentlélekkel egyetemben,
 Mindörökké, Ámen.

;Genf, 1562
>375
/1
#09B040E7
 Jézus, az ígéretet
 Ím, bételjesítetted,
 Bátorító Lelkedet
 Mihozzánk elküldötted,
 Aki által híveid
 Elnyerik érdemeid.
/2
#CADCD687
 Ama megfeszíttetett
 Test az égbe vitetett,
 És helyette küldetett
 E reánk kitöltetett
 Lélek: örökkévaló
 Gyámol és vigasztaló.
/3
#96E0FA25
 Isten, aki tűzben jött
 Mózest elbocsátani,
 S szélben ment Illés előtt
 Őtet bátorítani,
 Most kettős erőben: szent
 Tűzben s szélben megjelent.
/4
#0893C26F
 Bátorítja szívüket
 a Jézus híveinek;
 Tudományt és nyelveket
 Oszt kiküldötteinek.
 Tudatlanból tanítót
 Tesz, s betegből gyógyítót.
/5
#8990A2D9
 Egy halász, ha prédikál,
 fog sok ezer lelkeket,
 S míg dühösen űzi Pál
 Az eloszlott híveket,
 Útban éri leverő
 Mennyei tüzes erő.
/6
#430AADF3
 Terjed e tűz az egész
 Föld színére hirtelen,
 Fú e szél s hatalmat vész,
 Ahol akar, szüntelen,
 Lelkesíti sorsosit,
 Szentel és világosít.
/7
#25283D57
 Add nékünk is, Istenünk,
 A te áldott Lelkedet,
 Szent tűz adja érzenünk
 Éltető kegyelmedet;
 Adj hitet, szeretetet:
 Lelki boldog életet.

;Debrecen, 1774
>376
/1
#0459E846
 Ó, áldott Szentlélek, ki az ég dicsőségével,
 Leszállván a földre zúgó szél és tűz jelével
 A tanítványok gyűlésébe,
 Úgy munkálkodtál,
 hogy lelkükbe'
 Csuda erőkkel telének be.
/2
#AADE42FE
 Hatalmas erőddel őket egy szempillantásba'
 Megvilágosítád s hoztad olyan változásba,
 Hogy aznap, melyen előálltak
 És a népeknek prédikáltak:
 Anyaszentegyházat fundáltak.
/3
#93427E70
 Jövel mihozzánk is, részeltess ajándékidban,
 Lakozzál mibennünk, mint élő templomaidban!
 Adj hitet, adj jó reménységet,
 Adj szentid között egyességet,
 Békességet és idvességet.
/4
#6E11FCE5
 Oszlasd el homályos elménknek tudatlanságát,
 Enyhítsd meg elepedt szívünknek szomorúságát;
 Éreztessed még itt létünkben
 Az örömöt a mi lelkünkben,
 Melyet adsz örök életünkben!

;Bourgeois L., Strasbourg, 1545
>377
/1
#776CB017
 Szentlélek, végy körül bennünket,
 Szenteld meg szívünket,
 Készíts neved imádására,
 Magasztalására,
 Hogy téged szívből imádhassunk,
 Hálákat adhassunk:
 Hiszszük, a mi szánknak szózatja
 Egeid meghatja.
/2
#61D607F0
 Szentlélek, mi imádunk téged,
 Valljuk istenséged.
 Hisszük, hogy az új ember szíve
 saját kezed míve.
 Te vagy a hitnek mind szerzője,
 Mind elvégezője,
 Te gyújtasz szívünkben világot,
 Forró buzgóságot.
/3
#79D0EA3D
 Szakaszd el hát most is szívünket,
 Minden érzésünket
 A sok hiábavalóságtól,
 E csalárd világtól,
 Hogy az Igének hallgatói,
 Légyünk megtartói;
 Mely szívünkben gyökeret verjen,
 Gyümölcsöt teremjen.

;Debrecen, 1774
>378
/1
#8EFCD746
 Adjunk hálát mindnyájan
 Az Atya Úr Istennek,
 És mondjunk dicséretet
 Mi teremtő Istenünknek,
 Ki egybegyűjte most minket,
 Hogy ünnepet szenteljünk,
 És szent Igéjével éljünk.
/2
#6D604571
 Ó, kegyes Atya Isten,
 Te vagy Úr mindenekben,
 Ki megjelentéd magad
 Szent igédben itt e földön,
 És sok csudatételidben,
 A te áldott Fiadban,
 Mi kegyes Idvezítőnkben.
/3
#8E3BD9EC
 Légy kegyelmes minékünk
 A te áldott Fiadért,
 Az Úr Jézus Krisztusért,
 Mi szentséges Megváltónkért,
 És ne állj bosszút mirajtunk
 A nagy hitetlenségért,
 Fertelmes sok bűneinkért!
/4
#E13F5640
 De szentelj meg bennünket,
 Bírj és segíts meg minket,
 Gerjeszd fel mi lelkünket
 És a mi gyarló szívünket,
 Hogy téged megismerhessünk,
 Segítségül hívhassunk,
 Néked hálákat adhassunk.
/5
#ED746341
 Te igazgasd elméjét
 És minden tanúságát,
 Vezéreld útát, nyelvét
 A mi lelkipásztorinknak,
 És oktassad elméjüket
 A te szent beszédedet
 Figyelmesen hallgatóknak.
/6
#EB884AF5
 Tarts meg minden időben
 Minket az igaz hitben,
 És igaz értelmében
 A szent evangyéliomnak;
 Légyen foganatos köztünk
 A te kegyes beszéded,
 És minden jóra intésed.
/7
#EA2102EF
 Hogy sok népek tehozzád
 Megtérjenek bűnükből
 És téged szolgáljanak
 Az ő szívükből-lelkükből;
 Hogy mindenek imádjanak
 S csak téged tiszteljenek,
 Hogy örökké élhessenek.

;Debrecen, 1774
>379
/1
#4511F6C1
 Emlékezzél, Úr Isten, híveidről,
 Lelkitesti sokféle szükséginkről,
 Viselj gondot irgalmasságodból
 Különösen te szentegyházadról.
/2
#B53F29EA
 Adjad nékünk a kenyeret éltünkben,
 Amely táplál és erősít hitünkben,
 Nevekedést ád reménységünkben
 És megszentel lelkünkben-testünkben.
/3
#A260367A
 Ne hagyj minket, Úr Isten, szomjúhoznunk,
 Az élő víz adassék most minékünk,
 Mely víz után soha nem szomjúzunk,
 Mert tebenned örökké vigadunk.
/4
#A6082302
 Akik hallják, Úr Isten, beszédedet,
 Ebből értik irántuk jó kedvedet,
 Szent Fiadért kegyelmességedet,
 Hogy közlöd vélünk örökségedet.
/5
#1FC12472
 Adjad nékünk most is te Szentlelkedet,
 Ne hallgassuk hiába szent Igédet,
 Sőt a szerint féljük szent nevedet,
 Magasztaljuk mindég Felségedet.
/6
#6AF51D1F
 Dicsértessék már az Atya Úr Isten;
 Ő szent Igéje, a Fiú Úr Isten;
 Egyetemben Szentlélek Úr Isten:
 Szentháromság egy örök Úr Isten!

;Székel Balázs, 1546
>380
/1
#A77AD617
 Semmit ne bánkódjál, Krisztus szent serege,
 Mert nem árthat néked senki gyűlölsége,
 Noha e világnak rajtad dühössége,
 De nem hágy szégyenben Krisztus ő Felsége.
/2
#315D2A37
 Királyi nemzet vagy, noha te kicsiny vagy,
 Az Atya Istennél bizony te kedves vagy;
 Ő szent Fia által már te is fia vagy,
 Minden dicsőségben, higgyed, hogy részes vagy.
/3
#A5D296C7
 Hogyha te igazán a Krisztusban bízol,
 Higgyed, hogy lélekben Istennél gyarapszol,
 Ha Krisztus véréből igaz hittel iszol:
 Higgyed, hogy örökké meg nem szomjúhozol.
/4
#19FA4C04
 Akármint halásszon az ördög utánad,
 Az ő tagjaiban dühösködjék rajtad,
 Mind tőrrel, fegyverrel siessen utánad,
 Ha Krisztusban bízol, higgyed, az sem árthat.
/5
#8015A65B
 Rajtad semmit sem fog a pogány ellenség,
 Noha nehéz néked a rettenetesség;
 Bátor koncra hányjon a hitlen ellenség:
 Feltámaszt a Krisztus, néked nagy reménység.
/6
#3129E17A
 Oltalmazza Krisztus az ő szent egyházát,
 Miként a jó pásztor saját juhocskáját;
 Valaki hallgatja a Krisztus mondását,
 Viseli mindenkor szorgalmatos gondját.
/7
#6DD48AA1
 Siess most mihozzánk, Krisztus, segélj minket,
 A te szent igéddel neveljed hitünket
 És te Szentlelkeddel bírjad életünket,
 Hogy minden dolgunkban dicsérhessünk téged.
/8
#132BD1D7
 Igaz fogadásod és minden beszéded,
 Nyilván vagyon immár minden dicsőséged;
 Ne hagyd elpusztulni e kicsiny sereget,
 Melyért a keresztfán vallál nagy gyötrelmet!
/9
#0EEB4E80
 Ne nézzed, Úr Isten, e világ vakságát,
 Te nagy jóvoltodról háládatlanságát,
 A te igéd ellen ily nagy káromlását:
 Nézzed a Krisztusnak ártatlan halálát.
/10
#FAD8D35D
 Vedd el már mirólunk a sok ellenséget,
 Vedd ki miközülünk a sok gyűlölséget;
 Essenek szégyenbe minden ellenségink,
 Kik dicsőségedet tagadják s nevetik.
/11
#39791FE0
 Senkiben nem bízik az anyaszentegyház,
 Hanem csak tebenned, ki minket oltalmazsz,
 És igazságoddal mindenkoron táplálsz,
 Te szent sebeiddel minket megvigasztalsz.
/12
#D7D9319D
 Hálát adunk néked, mennybéli nagy Isten,
 Ki vagy egy szentségben és három személyben.
 Ne hagyj elrettennünk keserűségünkben,
 Halálunk óráján ne essünk kétségben!

>381
/1
#A3669EF6
 Jézus Krisztus, mi kegyelmes Hadnagyunk,
 Könyörgéssel te elődbe járulunk,
 Kik e siralomnak völgyén nyomorgunk:
 Te légy nékünk segítségünk, oltalmunk.
/2
#B7B137EB
 Elmerültünk a sok nyomorúságban,
 Éjjel-nappal vagyunk szomorúságban,
 Számtalan sokféle sanyarúságban,
 Az ellenségnek vagyunk ő torkában.
/3
#60DB37F6
 Reád hagyjuk anyaszentegyházunkat,
 Hisszük, hogy jól látod nyavalyáinkat;
 Elmerülni ne hagyd kicsiny hajódat,
 Parancsolj és csendesítsd a habokat.
/4
#92AA44F6
 Ébredj Uram, mert immár elmerülünk,
 Veszedelmes habokban elsüllyedünk;
 Szükségünkben tehozzád esedezünk:
 Jézus Krisztus, hallgass meg, téged kérünk!
/5
#6AB74F83
 Kérünk Uram, maradj velünk szálláson,
 Járj mivelünk, népeiddel egy úton.
 Látod, immár vagyunk napenyészetkor;
 Térj mihozzánk, járj mivelünk egy úton.
/6
#CF1F8C0E
 Védelmezzed a te kicsiny bárkádatl
 Benned bízik a te kűzdő egyházad;
 Te szent véred érette kiontottad:
 Elmerülni ne hagyd, ha megváltottad!
/7
#55A548FB
 Ne hagyj kérünk, egyetlenegy Jézusunk,
 Jusson te elődbe mi imádságunk
 És hallgass meg, egyetlenegy Szószólónk,
 Tekints reánk, hatalmas Közbenjárónk!
/8
#D746CE33
 Sőt légy erős kőfala te népednek,
 Szaporítsad számukat híveidnek,
 Vezéreljed nyelvét hirdetőidnek,
 Kik nevedért gyakran bosszút szenvednek.
/9
#5B795C8D
 Szent az Atya, mindenek megtartója,
 Szent a Fiú, mindenek megváltója;
 Ó Szentlélek, hívek bátorítója,
 Szent, szent, szent a te neved, minden vallja!

>382
/1
#F134F7ED
 Hogy panaszolkodik az anyaszentegyház, a Jézusnak jegyese,
 Melyet megtisztíta az ő bűneiből a Krisztusnak szent vére,
 Aki a szent Páltól az élő Istennek mondatik szent házának,
 Minden igazságnak fundámentomának és erős oszlopának:
/2
#07E93FEC
 Hallgass meg engemet, én Uram, Teremtőm, mindeneknek Istene,
 Ó, te igazságnak, minden jámborságnak oltalma és őrzője!
 Én gyenge koromban viselsz vala engem a te áldott méhedben:
 Ne hagyj el immáron vénségem idején, nagy erőtlenségemben!
/3
#956BBB27
 Mert ítéletedben, én édes Istenem, lám elhagytál engemet,
 Szegény árváimmal, kevés seregemmel megútáltál engemet;
 Gyászos özvegységre és nyomorúságra juttatád én ügyemet:
 Martalékra vetéd a hitetlen népnek én nyomorult fejemet.
/4
#86072935
 Az ennen szolgáim, kiket felneveltem, uralkodnak én rajtam,
 Az én koronámat és öltözetemet lefosztották én rólam;
 Mégis én annyira ezeket nem bánom, mint az én árváimat:
 Szörnyű nagy vakságra, kárhozat útjára tőlem elszakadtakat.
/5
#854E1CED
 Megfelel az Isten szent Ézsaiásnál az anyaszentegyháznak,
 És így vígasztalja szomorú szívüket nyomorult árváinak:
 Noha egy kevéssé, egy szempillantásban tégedet elhagytalak,
 Ismét hozzám veszlek, irgalmasságomból téged megszabadítlak.
/6
#ECC4B4EF
 És ha képes volna, hogy az édes anya elfelejtse magzatját,
 Nem akarok mégsem én elfelejtkezni mind örökké terólad;
 Az én tenyeremre írtalak fel téged: megemlékezem rólad,
 A te oltalmadra erős gondom lészen, mely örökké megmarad.
/7
#2A98A42A
 Krisztus Urunk mondja evangyéliomban, mely nagy gondja van reánk
 A nagy Úr Istennek, hogy még hajaink is tőle megszámláltatvák;
 Azért keresztyének, illik hálát adnunk és az Istent dicsérnünk,
 A mi szíveinknek nagy-szép ajándékit előtte bemutatnunk.
/8
#A7F4D24D
 Dicsértessél azért, mennyei Úr Isten, ily nagy kegyes voltodért ;
 Vigasztalj meg minket a te szent Fiadért, az Úr Jézus Krisztusért;
 Bátoríts meg minket nyomorúságinkban a te áldott igéddel,
 Hogy énekelhessünk a te szent nevednek örökké új énekkel.

>383
/1
#227AF113
 Könyörgünk néked, ó Isten Fia,
 Nyomorultak bizonyos oltalma,
 Ki uralkodol a mennyországba':
 Tekints reánk, Krisztus Jézus, megnyomorodtakra I
/2
#9351CB76
 Szertelen ront minket a pogányság,
 Mert övé már minden hatalmasság;
 Vagyunk immár sokaknak csak csúfság:
 Ugy minékünk, Krisztus Jézus, mindenben bátorságI
/3
#E9A2B507
 Istállókat sok szent templomidba
 Csináltanak te bosszúságodra,
 Melyben készek minden gonoszságra:
 Tekints reánk szent Atyáddal, megnyomorodtakra!
/4
#F11A9A42
 Oltalmazzad keresztyén népedet,
 Rontsad meg már a pogány népeket,
 Kik foglalnak bűnben sok időket:
 Tiltsd meg immár, Krisztus Jézus, szertelenségüketl
/5
#9EFB3165
 Légy minékünk kegyelmes királyunk;
 Ne bírassék pogánytól országunk,
 Se ne nézzed a mi bűnös voltunk:
 Szent Atyáddal, Jézus Krisztus, emlékezzél rólunk I
/6
#35E3CD91
 Te híveid, tehozzád kiáltunk,
 Igaz hittel hozzád folyamodunk;
 Pogányságban élni nem akarunk:
 Mennyországban uralkodást te általad várunk!
/7
#67785D08
 Vígasságos időket adj érni
 Szolgáidnak, kik tudnak tisztelni,
 Szent Atyádnak mondásiban járni:
 Bátorságot adj ő nékik ellenséget győzni!
/8
#699AFBF5
 Segedelem te tőled adassék,
 Hogy dícsérjen tégedet a község;
 Hozzád illik a bizony Istenség,
 És te néked esedezik egész keresztyénség.
/9
#EDCC4240
 Te népednek légy oltalmazója,
 Téged kérünk, Istennek szent Fia!
 Énekeljünk néked vígasságban,
 Szent Atyáddal, Szentlélekkel ki vagy mennyországban!

>384
/1
#AF9271C3
 Sok ínségünkben hozzád kiáltunk,
 Felséges Úr Isten,
 Mert parancsoltad, hogy megkeressünk
 A mi szükségünkben.
/2
#02DD15AA
 Tudjuk, hogy méltán mi bűneinkért
 Te megvertél minket,
 Mert nem tiszteltünk, sőt ingerlettük
 Te szent Felségedet.
/3
#B4DDDC3B
 A mi atyáink mi velünk együtt
 Ellened vétkeztek,
 Azért, hogy téged ők nem ismertek,
 Téged nem tiszteltek.
/4
#0F146CA1
 Reád maradtunk, mert mindenektől
 Mi elhagyattattunk,
 Az ellenségtől ostromoltatunk,
 Naponként rontatunk.
/5
#2846BDE2
 Ígéretedből, te szent Ígédből
 Téged megismertünk;
 Szent Fiad által néked könyörgünk:
 Légy kegyelmes nékünk!
/6
#ECCDFCA8
 Ne hagyd, tekintsd meg ígéretedet,
 És a te népedet,
 Kik tiszta szívböl mostan dicsérik
 A te szent nevedet.
/7
#F8A9F3EC
 Végy ki immáron e nagy inségböl,
 Pokol köteléből,
 Hogy te szent neved magasztaltassék
 Mi nemzetségünktől.
/8
#A1D2E0E9
 Sok hálát adunk, mint Istenünknek
 És mi Teremtőnknek,
 Te Szent Fiadnak és vígasztaló
 Szentlélek Istennek.

>385
/1
#C4036C35
 Keserves szívvel Magyarországban mondhatjuk magunkról
 A nagy siralmat, kit Jeremiás régen írt zsidókról:
/2
#08A1BB54
 Emlékezzél meg, hatalmas Isten; nyomoruságinkról ;
 Tekints mi reánk, állj bosszút immár mi nagy romlásinkról!
/3
#DCA77E4B
 Mert örökségünk tőlünk fordúla pogány nemzetségre,
 Mi lakó helyünk szálla mi rólunk idegen népekre.
/4
#31A9E10B
 Édes atyánktól immár megváltunk, árvaságra juttunk,
 Édes anyánkkal mind egyetemben özveggyé maradtunk.
/5
#63BC0529
 Megszomjúhozván, a mi vizünket drága pénzen isszuk;
 Nagy fáradsággal bégyüjtött fánkat immár áron vesszük.
/6
#BEDF40A9
 Sírván mondhatjuk, felséges Isten: vétkei atyáknak,
 Kik mi közülünk régen kimúltak, mireánk szállának.
/7
#3D7DBC85
 Kegyetlen szolgák nagy dühösséggel rajtunk uralkodnak;
 Nem találhatunk már reménységet szabadulásunknak.
/8
#40E16874
 Nagy félelemmel és rettegéssel pusztában bujdosunk:
 Eledelünket, mi életünket kezünkben hordozzuk.
/9
#369270B2
 Tisztes asszonyok tisztaságukban meggyaláztatának,
 A gyenge szűzek sok városokban szeplőket vallának.
/10
#F2387550
 A fejedelmek akasztófára felfüggesztetének,
 És vén népeknek ó tekintetek nem becsültetének.
/11
#2FA16864
 Ifjú népekkel és gyermekekkel gonoszúl élének;
 Kegyetlenséget, éktelenséget köztünk mívelének.
/12
#6F230122
 A mi szívünknek minden öröme már elfogyatkozék,
 Gyülekezetűnk nagy siralomra fordula s változék.
/13
#B11DB59A
 A mi fejünknek szép ékessége kiesék közülünk:
 Jaj mindörökké immár minékünk, mert igen vétkeztünk!
/14
#FA1F5775
 Nagy bánat miatt mi szívünk, lelkünk igen keseredék,
 S a mi szemünk siralom miatt mind megsetétedék.
/15
#91D151D9
 Keresztyéneknek lakó helyeik úgy elpusztultanak,
 Hogy vadak, rókák, hamis tanítók most immár ott laknak.
/16
#6F76C76F
 Te pedig, kegyes, irgalmas Isten, örökké megmaradsz,
 Te országodban, birodalmadban örökké megállasz.
/17
#BDEA6260
 Miért örökké elfelejtkezel mi rólunk, Úr Isten?
 Miért hagysz minket sok ideiglen e veszedelemben?
/18
#515561AA
 Téríts tehozzád, és mi megtérünk, kegyelmes Úr Isten!
 Újítsd meg immár mi napjainkat, mint régi időben.

>386
/1
#135E6E87
 Emlékezzél,mi történék, Uram mirajtunk,
 És tekintsd meg, mely nagy szidalomban mi vagyunk,
 Mert bűnünkért, Uram,tőled ostoroztatunk,
 Azért szükség éjjel-nappal hozzád kiáltnunk.
/2
#915650CD
 Örökségünk, édes hazánk másra fordúla,
 Mi házunk és jószágunk idegenre szálla;
 Ügyünk juta már minékünk nagy árvaságra:
 Mert Istennek nem akaránk térni útjára.
/3
#E927C3D2
 Hitetlen nép közt keressük mi kenyerünket:
 Mert földünkben nem segéljük a szegényeket.
 Nyomorultakról elfordítjuk szemeinket:
 Ezért Isten vete reánk haragvó szemet.
/4
#4B730060
 Atyáink is vétkeztenek, de már meghaltak,
 Mi is követői voltunk álnokságiknak;
 Ím így vásott meg beléje foguk fiaknak:
 Mert ellene járunk Isten akaratjának.
/5
#EF0841AB
 Közöttünk kik szolgák voltak, most uralkodnak,
 Gazdák lévén, mert Istennek mi sem szolgálánk;
 Minden Isten-tiszteletet már megútálánk:
 Azért nincsen, aki által megszabadulnánk.
/6
#5F3D8FE0
 Ím csak markunkban viseljük a mi lelkünket.
 Nagy keserüséggel esszük mi kenyerünket.
 Félelem és sok rettegés megemészt minket:
 Fegyver elől kietlenbe mentjük fejünket.
/7
#063B017C
 Nagy haragja vagyon rajtunk az Úr Istennek:
 Természeti ellen élünk szent Igéjének
 És ellene járunk minden Ő szerzésének:
 Ezzel adunk okot minden büntetésének.
/8
#BD4ADA7A
 Bizodalmunk vagyon benned, felséges Isten.
 Mert megmaradsz mind örökké ígéretedben;
 Ha gyötrődünk bűnünk szerint a mi testünkben:
 Azért nem hagysz elszakadnunk tőled lelkünkben.
/9
#F9702D8A
 Vedd el rólunk, kérünk téged, nagy haragodat
 És újitsd meg már minékünk a mi napunkat;
 Kegyelmes Úr Isten, tartsd meg mi házainkat!
 Dícsérhessünk mind örökké, mint szent Atyánkat!

>387
/1
#BFAEC45A
 Úr Isten, légy most mivelünk,
 Mert sokan támadtanak ellenünk,
 Nincsen kihez a mi fejünket
 Szükségünkben hajtanunk.
/2
#D7CF2168
 Reménységünk, bizodalmunk,
 Mindenkor te voltál mi oltalmunk:
 Azért most is te Felségednek
 Bátorsággal könyörgünk.
/3
#F33DAC5A
 A te szent Fiadért kérünk:
 Légy nékünk paizsunk és oltalmunk,
 Félelmünkben szárnyaid alatt
 Megmaradni hogy tudjunk.
/4
#7F107A18
 Meggyűlölt az Isten minket,
 Azt mondják: nem szeretsz, Uram, minket,
 Bfineinkért nagy haragodból
 Akarsz verni most minket.
/5
#D2D98A15
 Ezt mi mind megérdemeljük,
 Vétkünket ha mind előszámoljuk;
 A bűn ellen nagy haragodat
 Akarod, hogy ismerjük.
/6
#978EFD17
 Könyörülj immár mirajtunk,
 Lelkünkben kik reád támaszkodunk,
 Szabadlts meg: ellenséginknek
 Kezébe hogy ne jussunk!
/7
#3615B092
 Nagy hatalmadat hirdetjük,
 Ezért jóvoltodat dicsőltjük,
 Szent nevedet mindenek előtt
 Nagy felszóval dicsérjük.
/8
#4FE73296
 Kíméletlen sokszor vertél,
 De hamar ismétlen hozzánk tértél,
 Hogy lelkünkben inkább tisztuljunk:
 Tűzbe azért vetettél.
/9
#E88770DB
 Örökké, lám, el nem hagytál,
 A te Szentlelkeddel vigasztaltál;
 Most se hagyj el, kegyelmes Atyánk,
 Ha fiaddá fogadtál.
/10
#DDE4FCAC
 Mikoron megmenekedünk,
 Melyet már bizonnyal mi reménylünk,
 Dicséretet és nagy hál'adást
 Mi tenéked éneklünk.

;Erfurt, 1521
>388
/1
#13F8B3CB
 Hallgasd meg, Jézus Krisztus,
 Te megszomorodott
 S igen megkeseredett
 Szegény juhaidat,
 Hallgasd meg kegyelmesen
 A te szent egyházadat,
 Mely megnyomorodott.
/2
#6EA476C8
 Ne hagyd, édes Jézusunk,
 Nyomorult népedet,
 Szent véreddel megváltott
 Kicsiny seregedet;
 Ne hagyjad elpusztulni
 A te örökségedet:
 A keresztyénséget.
/3
#2BB468D2
 Kit a Sátán sokképpen
 Most megkörnyékezett,
 Sok fertelmes bűnökkel,
 És igen kísértget;
 Sok sanyarúságokkal
 És méreggel keserget,
 Szidalommal illet.
/4
#D8239292
 Siess, láss azért hozzánk,
 Kegyelmes Istenünk,
 El ne vess színed elől,
 Szerelmes jegyesünk!
 Ne hagyj el minket és ne
 Feledkezzél el rólunk:
 Messze ne menj tőlünk!
/5
#275DFD0E
 Maradj meg, Uram, vélünk,
 Mert beestvéledik
 És immáron a nap is
 Majdan elnyugoszik;
 Nagy homály és setétség
 Ígyen majd következik:
 A bűn sokasodik.
/6
#7079C79F
 Számtalan sok gonoszság
 Közöttünk megbővült,
 A te szent szereteted
 Igen meghidegült,
 És minden rendbéli nép
 Tőled elidegenült:
 A bűnben elmerült.
/7
#129A6877
 Hallgass meg azért minket
 És ments meg ezektől
 A pokolbéli ördög
 Nagy dühösségétől,
 Minden tévelygésektől,
 Oktalan, hamis hittől,
 Kétségbeeséstől.
/8
#1283798C
 El ne távozzék tőlünk
 A Szentlélek Isten,
 Légyen ő gyámolítónk
 A mi életünkben;
 Tartsa meg szent Igédnek
 Szerelmét mi szívünkben:
 Hogy legyünk kedvében.
/9
#94557394
 Úgyszintén vezéreljen
 Minket igazságban,
 Tartson meg szent nevednek
 Erős vallásában,
 És vigasztaljon minket
 Minden háborúságban
 És nyomorúságban.
/10
#A0D0A0BC
 Hogy dicséretet mondjunk
 A te szent Atyádnak,
 Ővéle egyetemben Néked, mi Urunknak;
 Mindörökkön, örökké
 A Szentlélek Istennek,
 Mi megszentelőnknek.

>389
/1
#454291E3
 Jövel, légy vélünk, Úr Isten,
 Segíts meg minket ügyünkben;
 Adj erőt az ellenségen,
 Mely reánk tör mind szüntelen,
 Háborgat minket hitünkben.
/2
#C9DC114D
 Bízik ő sokaságában,
 Fegyverében, jó lovában,
 Elhitte már ő magában,
 Hogy nincsen már e világban,
 Ki megrontsa hatalmában.
/3
#B1E0091D
 Bátran kergeti népedet,
 Ostorozza, híveidet,
 Pusztítja örökségedet,
 Ostromolja seregedet,
 Nem féli büntetésedet.
/4
#0311880B
 Azért tenéked könyörgünk:
 Támadj fel most mi mellettünk,
 Emeld fel zászlódat köztünk,
 Viadalmunkban légy velünk:
 Biztass, hogy meg ne rettenjünk.
/5
#E9E64432
 Bátorítsad mi szívünket,
 Igazgassad fegyverünket,
 Tőlünk el ne vedd eszünket,
 Egyengessed seregünket:
 Rettentsd meg ellenségünket.
/6
#7E0C4567
 Nékünk nincs annyi hatalmunk,
 Hogy ő ellene állhassunk;
 Te légy azért mi oltalmunk,
 Fegyverük ellen paizsunk,
 Erősségünk, diadalmunk.
/7
#4DFDF92B
 Nem sokaság a mi erőnk:
 Egyedül te vagy reményünk!
 Noha sokan nem lehetünk,
 Ha te velünk vagy, nem félünk:
 Sokat kevesen megverünk.
/8
#AAEC0631
 Gyakorta kevés nép által
 Nagy hadakat levágatál,
 Erőseket letapodtál,
 Kevélyeknek torkán jártál,
 Kegyetleneket rontottál.
/9
#D586D6FB
 Ők vívnak uralkodásért,
 Prédáért, ragadományért;
 Mi hitünkért, szent nevedért,
 Özvegyekért és árvákért,
 Megromlott édes hazánkért.
/10
#04ADD8E8
 Ne mondják azt mi felőlünk,
 Hogy nékünk nincsen Istenünk:
 Mutasd meg erődet bennünk,
 Viaskodjál együtt velünk,
 Magasztald fel kezed köztünk!
/11
#236C7C59
 Hogy mindenek, kik ezt hallják,
 Győzedelmedet csudálják,
 Jóvoltodat magasztalják,
 Hatalmasságodat áldják,
 Erődet meg ne útálják.
/12
#A96DFA4D
 Mi is innen visszatérvén,
 Jó szerencsénken örülvén
 Adunk hálát, felemelvén
 Szemeinket, és dícsérvén
 Téged mind örökké, Ámen.

;Erős vár a mi Istenünk!
;Bourgeois L., Genf, 1551
>390
/1
#47626CD2
 Erős vár a mi Istenünk,
 Jó fegyverünk és pajzsunk,
 Ha ő velünk, ki ellenünk?
 Az Úr a mi oltalmunk. Az ős ellenség
 Most is üldöz még,
 Nagy a serege,
 Csalárdság fegyvere;
 Nincs ilyen több a földön.
/2
#A91F8099
 Erőnk magában mit sem ér,
 Mi csakhamar eles nénk;
 De küzd értünk a hős vezér,
 Kit Isten rendelt mellénk.
 Kérdezed: ki az? Jézus Krisztus az,
 Isten szent Fia,
 Az ég és föld Ura,
 Ő a mi diadalmunk.
/3
#57081E3C
 E világ minden ördöge
 Ha elnyelni akarna,
 Minket meg nem rémítene,
 Mirajtunk nincs hatalma.
 E világ ura
 Gyúljon bosszúra:
 Nincs ereje már, Reá ítélet vár:
 Az Ige porba dönti.
/4
#2686C71E
 Az Ige kőszálként megáll,
 Megszégyenül, ki bántja;
 Velünk az Úr táborba száll,
 Szent Lelkét ránk bocsátja.
 Kincset, életet,
 Hitvest, gyermeket
 Mind elvehetik,
 Mit ér ez őnekik?
 Miénk a menny örökre.

>391
/1
#571A0BD7
 Vedd el, Úr Isten, rólunk haragodat,
 És a te kemény, véres ostorodat;
 Nagy bűneinkért mireánk ne ontsad
 Bosszúállásodat.
/2
#8B22320F
 Ha bűnünk szerint kezdünk mi fizetni,
 Ítéletedet nem állhatja senki,
 Mint e világnak első veszedelmi
 Nékünk megjelenti.
/3
#985BCBBF
 Mire mireánk gerjedez haragod,
 Ó, mi szent Atyánk, rajtunk nagy csapásod?
 Kik nem különbek vagyunk, mint árnyékok,
 Rothadandó porok.
/4
#7559EF21
 Megfertőztettünk atyáink vétkével
 És gyarló lelkünk-testünk sok sebével;
 Mi romlásunkat, nagy veszedelmünket
 Nézd kegyelmes szemmel.
/5
#73EA4B7A
 Adjad minékünk a te Szentlelkedet:
 Ismerhessük meg mi Közbenjárónkat,
 Mert megismertük mi gonoszságinkat:
 Vedd el haragodat.
/6
#46390108
 Ne hagyd elveszni ilyen nagy munkáját,
 Ne hagyd hiába az ő szent halálát,
 A mi Urunknak keserves nagy kínját,
 Az ő áldozatját.
/7
#19A421BB
 Végye el rólunk a mi bűneinket,
 És mosogassa vérével vétkünket
 Ő, ki Atyjával és a Szentlélekkel
 Bizony örök Isten.

;Wesley S.S., 1810-1876
>392
/1
#80B9DF27
 Az egyháznak a Jézus a fundámentoma,
 A szent Igére épült fel lelki temploma.
 Leszállt a mennyből hívni és eljegyezni őt,
 Megváltva drága vérén a váltságban hivőt.
/2
#F5556CDC
 Kihívott minden népből egy lelki népet itt,
 Kit egy Úr, egy keresztség és egy hit egyesít.
 Csak egy nevet magasztal, csak egy cél vonja őt,
 És egy terített asztal ád néki új erőt.
/3
#A7001B22
 A világ fejedelme feltámad ellene,
 Vagy hamis tudománytól gyaláztatik neve,
 S míg egykor felderül majd az Úrnak hajnala,
 Csak virrasztói kérdik: „Meddig az éjszaka?”
/4
#EC8008E2
 Sok bajban, küzdelemben meghajszolt, megvetett,
 De szent megújulásért
 És békéért eped,
 Míg látomása egykor dicsőn beteljesül
 S a győzelmes egyház Urával egyesül.
/5
#FC33BF6B
 A három-egy Istennel már itt a földön egy
 S az üdvözült sereggel egy nép és egy sereg.
 Ó, mily áldott reménység: ha itt időnk lejár,
 Te boldog szenteiddel fenn Nálad béke vár!

>393
/1
#DEF00E39
 Ne csüggedj el, kicsiny sereg,
 Ha rád zúdul vad ellened,
 Hogy végképp öszszetörjön;
 Bár elpusztításodra tör,
 Gond, kételkedés mit gyötör?
 Nem lesz ez így örökkön!
 Bízzál: ügyed az Istené,
 Népét ő el nem ejtené:
 Ő áll majd boszszút érted;
 Ő állít Gedeont melléd,
 Általa harcodban megvéd,
 Szent igéjét és téged.
/2
#132B2E36
 Él az Úr, áll ígérete:
 Ördög s világ minden csele
 Megszégyenül mirajtunk!
 Vélünk az Úr, mi ővele,
 Végtelen az ő ereje:
 Győzelmet kell aratnunk.
 Krisztusunk, segélj, el ne hagyj,
 Pártfogónk végig te maradj:
 Oltalmazz neved által,
 Hogy mint hű nyájad, teneked
 Zenghessünk dícséreteket
 Víg, boldog hál'adással!

;Luttenberger Tihanyi Ágost, 1890
>394
/1
#81A9EA76
 Térj magadhoz, drága Sion,
 Van még néked Istened,
 Ki atyádként felkaroljon,
 Szívét oszsza meg veled!
 Azt bünteti, kit szeret,
 Másképp ő nem is tehet:
 Sion, ezt hát jól gondold meg,
 Szabj határt bús gyötrelmednek.
/2
#FA6F6C5C
 Hullámok ha rémítenek
 Mérhetetlen víz felett,
 S a habok közt szíved remeg,
 Hogy sírod is ott leled;
 Ha aludni látod őt,
 Ki reményed és erőd:
 Sion, soha ne feledd el:
 Ő megvívhat tengerekkel!
/3
#A929CBC8
 Bár hegy, halmok rengenének,
 Miket égi kéz emelt,
 S indulása a nagy égnek
 Végromlásra adna jelt:
 Ezt látva is el ne hidd,
 Hogy ez a perc elveszít;
 Sion, addig meg nem dőlhetsz,
 Míg oltalmad Istentől lesz!
/4
#14A4EE56
 Bár könnyűid omlanának
 Gyöngyökül a tengerbe,
 És elhalván hangja szádnak,
 Csak pihegnél, mint gerle,
 Bár vér volna bíborod
 S kő megszánná nyomorod:
 Sion, ne félj a gonosztól,
 Baj nem ér, míg Benne bízol!
/5
#A3406C19
 Bár hordozzad zsarnok láncát,
 Érjen kínos rabhalál,
 Ha hitedet el nem játszád,
 Útad égbe nyitva áll.
 Örvendj mindig és vígadj,
 Emlékezz, ki népe vagy!
 Sion, nincs több Isten egynél,
 Benne hát ne kételkedjél!
/6
#08762EDB
 Ó, ne csüggedj, ím, az estnek
 Már leszállnak árnyai,
 Kihez ajkid oly hőn esdnek,
 Halld: Atyádnak hangja hí.
 Ő gyalázat, kín helyett
 Néked jobbján ad helyet;
 Sion, a menny lesz te részed,
 Föld gyötrelmét hát ne nézzed!
/7
#C50C43D8
 Végső áldást mondj hazádra,
 Mely távolról int feléd,
 Égi honnak a határa
 Van már hozzád közelébb.
 Édes érzés mért fog el,
 Melytől olvad szív, kebel?
 Sion, minden másképp lesz ott,
 El fog tűnni nagy sírásod.
/8
#B8A83984
 Angyalok, ti fényes lelkek,
 Zengjetek víg éneket,
 Mert már biztos révbe tért meg,
 Kit bús szélvész hányt-vetett!
 Már meggyőzte a halált,
 Istenéhez égbe szállt:
 Sion, onnnan számkivetni
 Nem fog téged soha senki!

;Basel, 1745 (Herrnhag, 1735)
>395
/1
#AD2732AA
 Isten szívén megpihenve
 Forrjon szívünk egybe hát,
 Hitünk karja úgy ölelje
 Édes Megváltónkat át!
 Ő fejünk, mi néki tagja,
 Ő a fény, mi színei;
 Mi cselédek, ő a gazda,
 Ő miénk, övéi mi.
/2
#F5AA25E0
 Szeretetben összeforrva,
 Egy közös test tagjai,
 Tudjuk egymásért harcolva,
 Ha kell, vérünk ontani.
 Úgy szerette földi nyáját
 S halt meg értünk jó Urunk;
 Fájna néki, látva minket,
 Hogy szeretni nem tudunk.
/3
#EB39C494
 Nevelj minket egyességre,
 Mint Atyáddal egy te vagy,
 Míg eggyé lesz benned végre
 Minden szív az ég alatt;
 Míg Szentlelked tiszta fénye
 Lesz csak fényünk és napunk,
 S a világ meglátja végre,
 Hogy tanítványid vagyunk.

;Halle, 1704 (1690)
>396
/1
#348D1039
 Ébredj, bizonyságtévő Lélek!
 A várfalakra őrök álljanak,
 Kik bátran szólnak harcra készek,
 Ha éj borul le, vagy ha kél a nap.
 Hívásuk zengjen meszsze szerteszét,
 Az Úrhoz gyűjtve népek seregét!
/2
#FFD2505C
 Ó, bárha lángod fellobogna
 S ébredne föl sok nemzet fényinél
 Ó, bár sok szolga, sarlót fogva,
 Aratna, mígnem leborul az éj!
 Urunk, e roppant, ért vetésre nézz:
 A munka sok, a munkás oly kevés!
/3
#9E3EB016
 Küldd útra hírnökid csapatját,
 És adj erőt onnan felül nekik,
 Hogy veszni a pogányt se hagyják,
 És szerteűzzék Sátán seregit.
 Országod jöjjön el minél elébb,
 Hirdetve szent neved dicséretét!

;Walch J., 1876
>397
/1
#BBDECB62
 Ó, Sion, ébredj, töltsd be küldetésed,
 Mondd a világnak: hajnalod közel!
 Mert nem hagy az, ki népeket teremtett,
 Senkit sem éjben, bűnben veszni el.
 Légy örömmondó békekövet,
 Hirdesd: a Szabadító elközelgetett!
/2
#068127E4
 Lásd: millióknak lelke megkötözve,
 Rabláncként hordoz sötét bűnöket;
 Nincs kitől hallja: Megváltónk keresztje
 Mily gazdag élet kútja lett neked.
 Légy örömmondó békekövet,
 Hirdesd: a szabadító elközelgetett!
/3
#63946CD9
 Mondd minden népnek: elveszett juháért
 Mit tett a Pásztor - csuda szerelem -
 Földig hajolt a kárhozott világért
 S meghalt alant, hogy élhess odafenn.
 Légy örömmondó békekövet,
 Hirdesd: a Szabadító elközelgetett!
/4
#1F20DC9B
 Küldj fiaidból, akik nemhiába
 Élvezik kincsed: Hirdessék szavad;
 Öntsd lelked értük győzelmes imába:
 Mindent, mit adtál, Krisztus visszaad.
 Légy örömmondó békekövet,
 Hirdesd: a Szabadító elközelgetett!
/5
#285D8FEE
 Ő visszajön, Sion, előbb, mint véled,
 Felfedi titkát minden szív előtt.
 Egy lélekért se érjen vádja téged,
 Hogy temiattad nem látta meg Őt.
 Légy örömmondó békekövet,
 Hirdesd: a Szabadító elközelgetett!

;Hatton J., 1793
>398
/1
#FA415A73
 Úr lesz a Jézus mindenütt,
 Hol csak a napnak fénye süt,
 Úr lesz a messze tengerig,
 Hol a hold nem fogy s nem telik.
/2
#FCE9E6E7
 Őneki mondjunk hő imát,
 Díszítsük azzal homlokát,
 Jó illat légyen szent neve,
 Minden napon dicsérete.
/3
#67948D09
 Országok, népek és nyelvek,
 Ő dicsőségét zengjétek,
 Gyermekek hangja hirdesse:
 Áldott a Jézus szent neve!
/4
#3433CE39
 Ő királysága bő áldás,
 Ott van a felszabadulás,
 Fáradtak ott megnyugszanak,
 Ínségesek megáldatnak.
/5
#14A46373
 Minden teremtés dicsérje,
 A Király Krisztust tisztelje;
 Angyali ének zengjen fenn,
 S mind e föld mondja rá: Ámen.

>399
/1
#2E6A720D
 Imhol vagyok, édes Uram, Istenem,
 Kész vagyok mindenben néked engednem,
 Szent nevedért szörnyű halált szenvednem.
 Csak az igaz hitben ne hagyj tőled elesnem!
/2
#08882788
 Oda megyek, ahova parancsolod,
 Akár tűzre, akár vízre akarod,
 Vagy fegyverre, tömlöcre te kivánod:
 Legyen, Uram Isten, valamint te akarod!
/3
#8782A290
 A halálra imhol viszem fejemet,
 Nem szánom én kibocsátni lelkemet,
 A Krisztusért kiontatni véremet:
 Ezzel megmutatnom az én igaz hitemet.
/4
#1766B0BE
 Én elhagyom már e földi házamat,
 Nem szánom az én édes magzatimat,
 Se nem szánom jámbor házas társamat,
 Én atyámfiait, sok jámbor barátomat.
/5
#C1052234
 Semmi nékem itt e földön lakásom;
 Ha senkitől nem volna is bántásom:
 Mennyországra vagyon nagy kívánságom,
 Mert ott vagyon nékem maradandó városom.
/6
#091B9C28
 Egy keveset most ha én itt tűrendek,
 A Krisztusért ha valamit szenvedek:
 Nagy bőséges jutalmat tőle vészek,
 Örök dicsőségben véle mert együtt lészek.
/7
#2EF1F887
 Erősb leszek mind e széles világnál,
 Gyorsabb leszek vadaknál, madaraknál,
 Fényesb leszek én a fényes szép napnál,
 Drágalátosb lészek minden szép sáraranynál.
/8
#A381A96D
 Halálból mert megyek örök életre,
 Kárhozatból az örök idvességre,
 Gyalázatból az örök dicsőségre,
 És keserűségből angyali nagy örömre.
/9
#BA445087
 Halálból mert megyek örök életre,
 Kárhozatból az örök idvességre,
 Gyalázatból az örök dicsőségre,
 És keserűségből angyali nagy örömre.
/10
#2337047A
 Én elmegyek Atyámhoz, Istenemhez,
 Én lelkemnek megváltó Mesteréhez;
 Mennyországban vala örökségemhez,
 És a vígasztaló Szentlélek Úr Istenhez.
/11
#C6DF1A09
 Istennek ez oly nemes ajándéka:
 Hogy a bűnösöket ő nem útálja,
 De nagy szeretettel hozzá fogadja,
 Az ő szent országát Krisztusért nékik adja.
/12
#B38E9D96
 Fordítsd hozzám, szent Atyám, szemeidet,
 Bocsássad el mennyből te Szentlelkedet,
 És tartsad meg mindvégig te hívedet,
 Hogy a világ előtt vallhassalak tégedet.
/13
#0538901B
 Én kivallom a te igazságodat,
 Mindeneknek szent irgalmasságodat;
 Adjad nekem a te szent malasztodat,
 És bátorítsad meg ilyen szegény szolgádat!
/14
#7437CD69
 Semmit nem gondolok az én kínommal,
 Semminémű testem-szakadásával,
 Nem gondolok nagy szörnyű halálommal,
 Mert majd együtt lészek megváltó Krisztusommal.
/15
#A20DA085
 Jöjj el immár, ó én édes orvosoml
 Jöjj el, megváltó szép Jézus Krisztusom
 Jöjj el, megnyugtatóm, édes Megváltóm:
 Testemet-lelkemet te kezedbe ajánlom I

>400
/1
#8AEB01A7
 Hagyjátok el hív keresztyének.
 Fájdalmát a ti szíveteknek.
 Senki ne sirassa halottját,
 Mert megnyerte ő boldogságát.
/2
#1DFE0218
 Anyák szünjenek meg slrástól,
 Elfelejtkezvén magzatukról,
 Mert nem halál az ő haláluk,
 De életben megújulásuk.
/3
#A776079A
 Ekképpen az elvetett magvak
 Rothadásukban gyarapodnak,
 És szépen fejüket vetendők,
 Végezetül jó gyümölcs termők.
/4
#04499DB9
 Fogadd be azért a holt testet,
 Kebeledbe, föld, vegyed őtet,
 Takargassad az ő tagjait,
 Mint édes anya magzatait.
/5
#724F611E
 Szentléleknek edénye volt ez,
 Az élő Isten képe volt ez,
 A Krisztus Jézus lakott ebben,
 Volt Istentől nyert kegyelemben.
/6
#B6921D11
 Föld, híven tartsd e testet benned,
 Mert mikor Krisztus majd eljövend,
 Számon kéri ezt akkor tőled,
 Így lészen bizonnyal, el higyjed!
/7
#F05FBCE9
 Noha ez mostan rothadandó
 És idővel porrá válandó,
 De feltámad az Úr szavára,
 Angyali zengő harsonára.

>401
/1
#329B13F7
 Szívem szerint kivánom Utolsó órámat,
 Mert nyomorúság, bánat Emészti napomat.
 Megválni kész vagyok már Várván én jussornat,
 Melyet Jézus megszerzett. Mint fő jutalmamat.
/2
#575DBEA3
 A bűntől és haláltól Megváltál engemet,
 Ó Jézus, véred árán Megszerzéd üdvömet.
 Nem rettent már halálom, Sem semmi félelem;
 Nyugalmamat tebenned, Megváltóm, meglelem.
/3
#EAF43E2A
 A kedvére élőnek
 Keserű a halál,
 De én várom nyugodtan,
 Ha jő, készen talál.
 Jól tudom, fenn a mennyben
 Lesz dicsőbb életem,
 Így hát még a halál is
 Csak nyereség nekem.
/4
#357725B1
 Ez a világ magához
 Ám vonja szivemet,
 Bár igérjen bőséggel
 Elmúló kincseket,
 Nem ejthet tőrbe engem
 Már semmi földi jó:
 Megáll az Úr igéje,
 Más minden elmúló.
/5
#6FBD9738
 Elválni kedvesinktől
 Fájdalmas és nehéz,
 De szivem szent reménnyel
 A jövendőbe néz.
 Jézus, én bizodalmam,
 Hiszem, hogy egy napon
 Szerettimet a mennyben
 Újra megláthatom.
/6
#C3BDA2F2
 El nem bocsátlak többé,
 Nem hagylak, Jézusom,
 Te vagy üdvöm, reményem,
 Én csak benned bízom.
 Szent karod támogasson
 A végső harcon át,
 Hogy elvegyem kezedből
 Az égi koronát.

>402
/1
#85B10FB2
 Ó, örök hatalmú mennyei szent Isten,
 Minden dolgaimban benned van reményem,
 Te vagy oltalmam, vígaszom nékem.
/2
#41D5BEBF
 Hogyha bánat árja szemeimre zúdul,
 Szivem fájdalmára balzsamot ád az Úr,
 S felszárad a könny, a seb begyógyul.
/3
#B24415CF
 Kínok hogyha gyötrik lelkemet halálra,
 Sóhajom akkor is Istenemet áldja,
 Hiszen föltámaszt egy szebb világra.

;Debrecen, 1781
>403
/1
#224FEDD9
 Szomorú a halál a gyarló embernek,
 Halál követétől mindenek rettegnek,
 Kiváltképpen kik e világban örülnek,
 Nehéz e világtól megválni ezeknek.
/2
#D507F415
 Többet higgy a jégen rajzolt írásoknak,
 Mint a csalárd világ sok biztatásának,
 Szép szín alatt közli kis részét javának,
 De semmit sem hihetsz állandóságának.
/3
#404A8DE6
 Romlandó az élet s csak veszendő hívság,
 Hamar elenyészik, mint a gyenge virág,
 Minden ékessége csak nagy sanyarúság:
 Helyette vár reád a fényes mennyország.
/4
#45DCAC53
 Azért, gyarló világ, már maradj magadnak,
 Ám tartsd barátidnak, kik benned vigadnak,
 Tovább követője nem leszek utadnak,
 Nem leszek bús rabja mulandó javadnak.
/5
#27E3457C
 Tudom, kinek hittem és kinek szolgáltam,
 Kit szeretett lelkem, kihez folyamodtam,
 Kit hívtam Uramnak, ki mellett harcoltam,
 Azért hűségemnek jutalmát találtam.
/6
#0954AD29
 Elértem hitemnek egyetlenegy célját,
 Vitézkedésemnek reménylett pálmáját,
 Az örök életnek drága koronáját,
 Hogy abban tiszteljem a mennyek királyát.
/7
#E147BDF0
 Ó, édes Megváltóm, ne nézd bűneimet,
 Sok ellened való cselekedetimet,
 Idvességem árát: tekintsd érdemedet,
 Jövel, Jézus Krisztus, vedd hozzád lelkemet!
/8
#B7B20AE5
 Igazságod szerint ne ítélj meg engem,
 Mert semmi kegyelmet nem talál érdemem,
 Nem lehet kívüled soha idvességem:
 Jövel, Jézus Krisztus, édes reménységem!
/9
#C16D61A2
 Elvégeztem immár pályafutásomat,
 E világon való zarándoklásomat,
 Megtartottam hitem s igaz vallásomat:
 Jövel, Jézus Krisztus, add meg koronámat!

>404
/1
#239B2FA5
 Jaj,mely hamar múlik világ dicsősége,
 Hirtelen változik minden ékessége,
 Távozik szépsége!
/2
#4BFED05B
 Mint a füst és árnyék, csak olyan életed,
 Hirtelen elmúlik, észre sem veheted,
 Mint lesz enyészeted.
/3
#77800EFD
 Mint a szép virágok tavasszal újulnak,
 ékes nyílásukkal fénylenek, vidulnak,
 De hamar elmúlnak:
/4
#09D0568C
 Ifjak rettegjetek, vének megtérjetek:
 Ím rajtatok a sor, azért készüljetek,
 Mert el kell mennetek.
/5
#DC5691D6
 Ez úttól kincs, jószág senkit meg nem menthet,
 Egyedüli mentség csak Jézusod lehet,
 Ki velünk jót tehet.
/6
#144BD862
 Ő a testi halált életre fordítja,
 A hivő léleknek szennyét letisztitja,
 Végre boldogítja.
/7
#2A8E9B5F
 Mi is végóránkon hogy erre juthassunk,
 Tiszta kegyességet Istenhez mutassunk:
 Így lesz boldog sorsunk.
/8
#847CF015
 Ki-ki életében azért elkészitse
 Testét a férgeknek, lelkét az Istennek
 Örök dicsőségre.

;Darmstadt, 1687
>405
/1
#49606CE9
 Minden ember csak halandó,
 Minden test, mint fű, virág;
 Itt ami van, mind romlandó,
 És elmúlik e világ.
 Porrá kell e testnek lenni,
 Hogyha el akarja venni
 Az örök dicsőséget,
 Melyet Isten készített.
/2
#079F515C
 Azért e testi életem,
 Ha jön a sír éjjele,
 Bátran s örömmel leteszem,
 Semmit sem vesztek vele;
 Mert a Krisztus drága vére
 Utat nyit egy dicsőbb létre;
 A halálban biztatóm
 Jézus, az én Megváltóm.
/3
#48934A6C
 Ki szakaszthat el őtőle?
 Enyém ő, s övé vagyok.
 Tudom, el nem vet előle,
 Ígéreti mert nagyok;
 Sőt felvisz engem az égbe,
 Dicsőültek seregébe,
 Hol az Istent meglátom
 És mindörökké áldom.
/4
#7BFE97B4
 Ott van öröm s örök pálma,
 Hol sok ezren az égben,
 Isten trónja előtt állva
 Tündöklő fényességben,
 A dicső szent angyalokkal,
 Minden megboldogultakkal
 A Jézust magasztalják,
 Megtartójuknak vallják.
/5
#C6989171
 Nagy keresztet Kik hordoznak
 S harcolják a hit harcát,
 Győzelemben vigadoznak
 S zengnek ott halléluját.
 Ott öröm s dicső korona
 Én fejemet körülfonja;
 Ott élem az életet,
 Melynek vége nem lehet.

;Debrecen, 1781
>406
/1
#BC520A1C
 Én Istenem, benned bízom,
 Segélj, ne hagyj tántorodnom!
 Lelkem, testem, minden tagom
 Reád bízom:
 Vezess, mert rád támaszkodom.
/2
#C89B14FF
 Nem tudom: itt meddig élek,
 Mikor ér életem véget?
 Valamikor tetszik neked: hozzád megyek,
 Akaratodnak engedek.
/3
#269201E5
 Minden órában kész lelkem,
 Hogy veled legyek, Istenem.
 Nem választok időt, órát,
 Hozzád jutást,
 Ha akarod, megyek mindjárt!
/4
#7476729B
 Testemnek minden tagjait
 És a hajamnak szálait
 Tudod, Uram, mert fejemről
 Még csak egy szál
 Kedved nélkül alá nem száll.
/5
#646696D1
 Itt csak bánat, keserűség,
 Vagyon fájdalom, betegség;
 Életünk többnyire ínség,
 Kedvetlenség:
 még erőnk is erőtlenség.
/6
#E6B900DD
 Nincs orvosság halál ellen
 Patikában vagy más helyen,
 Széles mezőn, drága kertben
 Oly fű nincsen,
 Haláltól amely megmentsen.
/7
#D8F95672
 Te azért, ó, én Istenem,
 Add énnékem ezt értenem,
 Hogy nyomorult mi életünk,
 Meg kell halnunk,
 E világból ki kell múlnunk.
/8
#65DBD128
 A halálban biztatásom
 Nékem egyedül Jézusom,
 Ki énérettem szenvedett,
 Megfizetett:
 Szent Atyjának eleget tett.
/9
#DE042298
 Azért mikor, én Istenem,
 Akarod, vedd hozzád lelkem,
 Hogy veled örökké éljek,
 Reád nézzek,
 Az angyalokkal örvendjek.

>407
/1
#CC225570
 Itt nincs igazi boldogság,
 Ez az élet csak fényes fogság;
 Nincsen valóságos nyugodalom,
 Mig be nem fedez a sírhalom.
/2
#75497BF3
 E világnak kedvessége
 Tartatlan és keserű vége;
 A szenvedések és változások
 Uralkodnak, s itt van lakások.
/3
#D2DBDAB6
 Csak a sírban van csendesség,
 Ott van a legnagyobb egyesség;
 Ott az úr és szolga megegyeznek,
 Mint csak por, föld: nem különböznek.
/4
#283FC455
 Jézus porunkat felkölti,
 Mikor nyugvását itt kitölti;
 Lelkünkkel is újra egyesíti
 És idvességben részesíti.

;XIX. sz. eleje
>408
/1
#96D87B75
 Seregeknek szent Istene,
 Mennynek, földnek teremtője,
 Jövel, jövel, én Krisztusom,
 Ne hagyj utolsó órámon!
/2
#0D2F8338
 Végy be, kérlek, kegyelmedbe,
 Idvezítő szerelmedbe,
 Jövel, jövel, én Krisztusom,
 Ne hagyj utolsó órámon!
/3
#619AD466
 Az egekre tégy méltóvá,
 Szent hitemben állandóvá;
 Jövel, jövel, én Krisztusom,
 Ne hagyj utolsó órámon!
/4
#943E068A
 Életemben ki szerettél,
 Jóvoltodban részeltettél:
 Jövel, jövel, én Krisztusom,
 Ne hagyj utolsó órámon!
/5
#D22F6925
 Áldom azért szent nevedet,
 Hirdetem dicsőségedet:
 Jövel, jövel, én Krisztusom,
 Ne hagyj utolsó órámon!
/6
#732B512D
 Testem nyugtasd meg a földbe',
 Lelkem vidd a magas égbe;
 Jövel, jövel, én Krisztusom,
 Ne hagyj utolsó órámon!

;Bourgeois L., Genf, 1542
>409
/1
#2E41812A
 Utas vagyok e világban,
 Mennyországban
 Vár örök hazám készen;
 A testem csak lelkesült por
 És ha a sor
 Reá jön, porrá lészen.
/2
#1B1482F5
 Minden nap hoz rám fájdalmat:
 Nyugodalmat
 Szívem sehol nem talál.
 Majd kár ér, majd búbánat sért,
 Majd bűn kísért,
 Végre eljön a halál.
/3
#E3B74D85
 Uram, te látod végemet,
 Mert testemet
 S lelkemet te formáltad;
 Előbb, mint lettek napjaim,
 Hajszálaim
 Mind egyig megszámláltad.
/4
#56AB6214
 Lelkemnek földi társától,
 Sátorától
 Bizonyos megválása;
 De csak annak, ki Istent fél,
 Kegyesen él,
 Lesz boldog kimúlása.
/5
#CA1066AB
 Atyám, hogy meg ne rettenjek,
 S bátran menjek A minden test útjára,
 Teremts tiszta szívet bennem:
 Néked élnem Légyen lelkem fő vágya.
/6
#83954EC5
 Néked, napjaimnak Ura,
 Akár búra,
 Akár örömre juttatsz,
 Szent megnyugvással engedek;
 Bár szenvedek,
 Mindent jómra fordítasz.
/7
#090E6F2F
 Akármikor jön a halál,
 Készen talál,
 Fájdalmimnak vet véget;
 Bátran fogok vele kezet:
 Hozzád vezet
 S ád örök üdvösséget.

;Genf, 1562
>410
/1
#C0CF1490
 Csak vándorút az életem,
 Míg majd hazámba érkezem,
 Szent Jeruzsálem városába,
 Mit fönn az Isten készített,
 Szövetségvérre épített,
 Hol ajkam majd csak őt imádja;
 Csak vándorút az életem,
 Míg majd hazám elérhetem.
/2
#D950191B
 Árván megyek az élten át,
 Nem ismer itt a vak világ;
 Ott várnak rám a hű testvérek;
 Ott vár az égi szent sereg;
 Ujjongva szolgálok neked,
 És örökké csak érted égek;
 Ó, Megváltóm, jövel, siess,
 Szívem csak tégedet keres.

>411
/1
#99A06406
 Nem sokáig tart már földi bujdosásom,
 Atyám hajlékában lesz örök lakásom;
 A megfáradottnak jó lesz ott pihenni,
 Küzdelmei után nyugalomra lelni.
/2
#6723EFBE
 Repülj hát én lelkem, repülj bontott szárnnyal:
 Vár az örökélet a nagy boldogsággal;
 Nincsen elegyítve az szomorúsággal,
 Zeng az örömének angyalok karával.
/3
#C7D14BD2
 Szerelmes Jézusom, egeknek Királya!
 Majd ha üt éltemnek utolsó órája,
 Nyugtassad békében a sírban testemet,
 Irgalmadból add meg várt üdvösségemet!

>412
/1
#7AB00184
 Porok vagyunk, porrá lészünk.
 E világtól búcsút vészünk.
 Mint az árnyék, elenyészünk.
/2
#622F0922
 Itt nincsen semmi állandó,
 Minden változó, múlandó,
 Rövid életü, s halandó.
/3
#629BC022
 Emberek, tapasztaljátok:
 Egyaránt tartozik rátok
 A paradicsomi átok.
/4
#D24609A7
 Annyi nyugvó helyet adnak
 Neked is, mint portársadnak,
 Ha szemeid elszunnyadnak.
/5
#7B2C2A2E
 A holtak csendes kertjében
 Melletted alszik békében,
 Kit gyűlöltél életében.
/6
#9A2C870D
 Lelkem, ne bízz emberekben,
 A földi fejedelmekben:
 Nincsen segítség ezekben.
/7
#34303153
 El kell múlni a porrésznek;
 Ha kimegy lelkünk, enyésznek:
 Szándékink mind semmik lésznek.
/8
#0251FE81
 Így hát múlandó mindenem,
 Amivel, bírok, csak te nem,
 Mert örökké élsz, Istenem.
/9
#63F72E40
 Te vagy lelkem bizodalma,
 Szívem örök nyugodalma,
 Hitemnek boldog jutalma.
/10
#8C0AACA5
 Ha ostromol a test és vér,
 Szegény lelkem csak hozzád tér,
 Uram, kegyelmet tőled kér.
/11
#A9637818
 Ha reám kereszt tétetett,
 Ha megúnom az életet:
 Vígasztalsz, örök szeretet.
/12
#ED1D59F8
 Ha sajtolgat a szegénység:
 Hozzád vezet a reménység,
 Édes Atya, nagy Istenség.
/13
#158A6F8E
 Ha rettent halálfélelem:
 Édes Jézus, te légy velem;
 Megtartó kezed ölelem.
/14
#384DE8F5
 Erősítsd gyenge hitemet,
 Halálos küzdéseimet,
 Segítsd, idvezitsd lelkemet.
/15
#544C4B09
 Nyugodt lesz így porba szállnom,
 Végső szómmal ezt kiáltom:
 Jővel, Úr Jézus, Megváltóm!

>413
/1
#68B504B0
 Menj el a te nyugalmadba,
 Boldog lélek, követünk;
 Hogy a kívánt nyugalomba'
 Részt vehessünk, sietünk.
 Készíts, Uram, e jóra,
 Hogy midőn amaz óra
 Eljő, örömmel mehessünk
 Hozzád, jobbodra ülhessünk.

>414
/1
#A168C820
 Megszabadultam már én a testi haláltól,
 És megmenekedtem minden nyavalyáimtól,
 Bűntől, a haláltól, e csalárd világtól,
 Az örök kárhozattól.
/2
#3D313389
 Lelkemet ajánlom a hatalmas Istennek,
 És testemet hagyom ő anyjának, a földnek;
 Ez világot pedig az én feleimnek,
 És a benne élőknekl
/3
#5E036A75
 A testi halálból megyek örök életre,
 És megmondhatatlan örömre, dicsőségre,
 Kit kezdettől fogva a Krisztus megszerzett
 Az ő benne bízóknak.
/4
#0CCDF257
 Nincsen már hatalma én rajtam az ördögnek,
 E csalárd világnak, sem a kegyetlen bűnnek,
 Mert Krisztus elrontá ezeknek hatalmát
 Az ő szent halálával.
/5
#9E7D1B49
 Az Atya Istennek vagyok én szerelmében,
 Az ő szent Fiának kedvében, kegyelmében:
 Részessé tett engem minden örökében,
 Az örök dicsőségben.
/6
#BCD5AF72
 Dicsőség tenéked, örök Atya Úr Isten,
 És tisztesség néked, megváltó Fiú Isten:
 Teljes Szentháromság egy bizony Istenség
 Az örök dicsőségben!

;Debreceni halottas könyvből
>415
/1
#ADC82688
 Jer, temessük el a testet,
 Melyről kétségünk nem lehet,
 Hogy az ítéletnek napján
 Fel fog támadni igazán.
/2
#07D6E91E
 Porból való eredete,
 Azért porrá kell lennie,
 De majd feltámad a sírból,
 Mihelyt az Úrnak szava szól.
/3
#CC490C3F
 Az ő lelke örökké él
 A más világon Istennél,
 Ki szent Fiának általa
 Őt a bűntől megváltotta.
/4
#B5894593
 Lelke csendességben nyugszik,
 Teste a földben aluszik,
 Honnan ítélet napjára
 Feltámad nagy vigasságra.
/5
#88155E2B
 Itt volt ő nagy félelemben,
 De ott lészen csendességben,
 Örökkévaló örömben
 És hatalmas fényességben.
/6
#55CD7316
 Jer, hagyjuk itt őt aludni,
 Krisztus Jézusban nyugodni,
 És mi szüntelen vigyázzunk,
 Mert nékünk is meg kell halnunk.
/7
#2EF9EA68
 Erre Krisztus adjon erőt,
 Ki vérével minket kivett
 A veszedelmes pokolból
 És kínból, örök halálból.
/8
#8A8F7DAC
 Ő mitőlünk dicsértessék,
 Örökké magasztaltassék
 Egyetemben az Atyával
 És Szentlélek Úr Istennel.

;XIX. század eleji kéziratból
>416
/1
#985907D7
 Krisztus, én életemnek
 Te vagy reménysége,
 Szegény bűnös lelkemnek
 Örök üdvössége.
 Lészek hát én csendességben,
 Bár a halál fúlánkjával rettentsen.
/2
#DB669D6A
 Bátran éltem letészem,
 Mert jutalmát vészem,
 Elkészítve már nékem
 Királyi szent székem.
 Lészek hát én csendességben,
 Bár a halál fúlánkjával rettentsen.
/3
#0D720A65
 Megyek hát én örömmel
 Sion királyához;
 Ó, Jézusom, vezess el
 Égi szent Atyádhoz!
 Örvendj, szívem, repess, lelkem!
 Mert léssz mennyben angyali dicsőségben!

>417
/1
#B4BC3532
 Gyarló testünk porrá lészen,
 Mivel porból vétetett,
 A halál vár kit-kit készen.
 Mert halálra született.
 A bennünk levő bűnért
 Ezzel fizetjük a bért:
 Minden nap. sőt minden óra
 Emlékeztet koporsóra.
/2
#5E395F56
 Sírva jöttünk e világra.
 Sirás közt költözünk el.
 De ha jutunk boldogságra.
 Semmit se vesztettünk el.
 Bár panasz tölti szánkat.
 Míg megfut juk pályánkat.
 Bár e földön csak bujdosunk:
 Lesz mennyben örök városunk.
/3
#7099220A
 Hadd menjen hát kiköltözött
 Lelkünk jobb hazájába.
 Testünk a halottak között
 Hadd nyugodjék sírjába':
 Nem alszik el örökre,
 Még boldogabb időkre
 Tartatik a sír fenekén,
 Biztatást innen veszek én.
/4
#161E8031
 Eljön egykor amaz óra,
 Melyben a gyász sírhalmok
 Megnyílnak trombitaszóra
 S elvész minden hatalmok.
 Akkor majd lelkünk, testünk
 Elveszi,mit kerestünk;
 Szemünk téged is meglát ott.
 Kit most innen elbocsátott.

>418
/1
#E888F6D1
 Uram, majd egyszer feltámadnak,
 Kik a sírhalmokban szunnyadnak,
 Áldott kezed tágas tért nyitott,
 Hol a sírdomb húzott kárpitot,
 Jő majd a boldog kikelet,
 Elűzi a hosszú telet.
/2
#D0854D6A
 Boldog, aki nyugszik sirjában
 A Megváltó szent oltalmában;
 Bár rágja testét féregsereg,
 A rög közt is élte szendereg;
 S jő majd a boldog kikelet:
 A Góel áll sírja felett.
/3
#8F2A20CF
 Áldott légy hát, holtak országa,
 Az Istennek nagy majorsága;
 Megfogjuk azt élő hitünkkel,
 Hogy a halálban nem veszünk el:
 Jő majd a boldog kikelet,
 Ha kinyugosszuk a telet.
/4
#3372FC49
 Jézus, ki isteni erővel
 Megvlvtál a gyász temetővel,
 S feltámadtaknak vagy zsengéje,
 Kössön hozzád szád szent igéje;
 Téged követvén, úgy éljünk,
 Hogy sírunktói mi se féljünk.

>419
/1
#52917B17
 Már elmégyek az örömbe,
 Paradicsomnak kertébe,
 Lészek Istennek kedvébe:
/2
#EF2406BC
 Immáron bételjesedék,
 Mit éltemben sokszor hallék:
 Hogy még halálnak fizetnék.
/3
#39DB9D23
 Keresztyén hitnek jutalmát,
 Már elvészem diadalmát:
 Az életnek koronáját.
/4
#3E4B65DA
 Lelkem Ábrahám keblében,
 Testem is nyugszik a földben,
 De vagyok oly reménységben:
/5
#3D4327B0
 Onnét hogy még feltámadok,
 Krisztus Jézussal vigadok;
 Azért lelkemnek így szólok:
/6
#00A91EF5
 Siess a boldog városba,
 Aholott bé vagyok írva:
 A mennyeknek országába.
/7
#BBEDF1E1
 Drága szép lakó városunk,
 Hol Istennel együtt lakunk;
 Knsztus is örökös társunk.
/8
#2E32F375
 E városnak boldogságát,
 Tündöklését, gazdagságát,
 Szent Irás mondja mivoltát.
/9
#D1A88B46
 Abban, úgymond, nem szükséges
 Napfény, s hold nem kellemetes:
 lsten benne fényességes.
/10
#95FACA7A
 Keserűség, sem irígység,
 Nincs ott sántaság, betegség,
 Nincsen ott nyomorult vénség.
/11
#7393B4FF
 De mindnyájan feltámadván,
 Lészünk erős állapotban
 Krisztus ítéleti után.
/12
#B20FAC11
 Ott nincs továbbá kevélység,
 Nem uralkodik az éhség:
 Szeretet lesz az elégség.
/13
#685CBAEC
 Lészünk hívek egy seregben,
 Mind egyenlő dicsőségben,
 Szent Pál mondja levelében.
/14
#2A00D7E5
 Gerjedez együtt szerelmünk,
 Istenünkhöz tiszteletünk;
 Arról el nem feledkezünk.
/15
#67647A3A
 Az angyalok, pátriárkák,
 Apostolok és próféták:
 Megismernek atyánk s anyánk.
/16
#DC38BC60
 De még nagyobbnak mondatik,
 Hogy Isten velünk lakozik,
 Kinek dicsőség adassék!

>420
/1
#DCDBE3F2
 Életemnek végső napját,
 Minthogy már elértem óráját.
 Testem leteszi bűnének zsoldját.
 Bátran megvívja halálbaját.
/2
#8AFDB93F
 Lelkem érzi, hogy testemtől,
 Meg kell válnia ez élettől,
 Tudja: búcsút vészen földiektől,
 Mert így van rendelve Istentől.
/3
#D071892D
 Azért dolgát ő is bízza Istenre, ki magához hívja;
 Elhitte, hogy Krisztus őt nem hagyja,
 E harcban segedelmét nyújtja.
/4
#653AE3D2
 Támadjon fel bár a Sátán,
 De nincs hatalom botja vállán;
 Nem örülhet soha lelkem kárán:
 Megváltott vagyok Krisztus árán.
/5
#2F81A62F
 Isten ellen hogy bűnt tettem,
 Ezért akkor pereljen velem.
 Tudom: fortélya nem árt ellenem,
 Mert részes velem a kegyelem.
/6
#502C12E4
 Igyekezzék érdememtől,
 Eltiltani Idvezítőmtől,
 Kifogni a hívek számok közül:
 Isten már nem ád ki kezéből.
/7
#9F58D316
 Ennyi ellenségim között Krisztus, ki magának gyűrűzött
 Nem hágy, mert jó reménységet szerzött,
 Kit nálam mindeddig megőrzött.
/8
#D889507F
 Édes Jézus, segedelmem!
 Kérlek, már légy jelen mellettem,
 Pályafutásom ha elvégeztem,
 Letett jutalmát add meg nekem I
/9
#BA2E910A
 Angyali tábor ne késsék,
 Kik lelkem hozzád felkísérjék,
 Ábrahámnak kebelébe vigyék,
 Hol Lázárral együtt nyúgodjék.
/10
#5EE5BD55
 Feltámadás felől hitem,
 Vagyon állhatatos értelmem,
 Jó bizodalom, reménység engem
 Táplál, mert kész életre mentem.
/11
#B3D14C4F
 Ha porrá változik testem,
 De hiszem, hogy én Idvezítőm
 Ezen testben szemeimmel nézem,
 Mikor eljő, mint én ítélőm.
/12
#2A75E853
 Nem lesz nekem félelmemre,
 Ha mondatik, hogy ítéletre,
 Keljetek fel, holtak, számvetésre,
 Mert eljött Bíró fizetésre.
/13
#CDE09DA5
 Bátran a trombita szóra,
 Angyaloktól fúvallására
 Feltámad testem, mert vígasságra,
 Mennybe vitetik boldogságra.
/14
#9F7DA801
 Áldott Jézus, végső szómat,
 Halljad, elégeld meg harcomat,
 Terjeszd ki lelkemnek szent markodat,
 Részeltesd velem országodat!
/15
#AF0D2DCC
 Hol színről-színre Atyánkkal,
 Ezt kiáltsam boldog juhokkal:
 Dícsértessél, Atyám, szent Fiaddal,
 Szent, szent, szent légy ajándékoddall

;Stebbins C.G. után
>421
/1
#C92C5447
 Tudom, az én Megváltóm él,
 Hajléka készen vár reám;
 Már int felém és koronát
 Ígér a földi harc után.
 Bár a világ gúnyol, nevet,
 A honvágy tölti lelkemet,
 Mert nemsokára hív az Úr:
 Jöjj haza, jővel, gyermekem!
 Kitárt karjával vár az Úr:
 Jer, pihenj, nyugodj keblemen!
/2
#C41A8903
 Remélek Jézusomban én,
 Ő törlé el sok bűnömet;
 Ajkáról hívón zeng felém:
 „Jer haza, vár rád Mestered!"
 Bár a világ gúnyol, nevet,
 A honvágy tölti lelkemet,
 Mert nemsokára hív az Úr:
 Jöjj haza, jővel, gyermekem!
 Kitárt karjával vár az Úr:
 Jer, pihenj, nyugodj keblemen!
/3
#60DC25D9
 Előttem oly csudálatos,
 Hogy értem szállt a földre le;
 Miattam annyit szenvedett,
 Bűnömért annyit véreze.
 Bár a világ gúnyol, nevet,
 A honvágy tölti lelkemet,
 Mert nemsokára hív az Úr:
 Jöjj haza, jővel, gyermekem!
 Kitárt karjával vár az Úr:
 Jer, pihenj, nyugodj keblemen!
/4
#7F7CE84F
 Tudom, hogy közel Mesterem,
 Az óra fut, a nap közel;
 Elébe állok csakhamar,
 Megváltó Jézusom, jővel!
 Bár a világ gúnyol, nevet,
 A honvágy tölti lelkemet,
 Mert nemsokára hív az Úr:
 Jöjj haza, jővel, gyermekem!
 Kitárt karjával vár az Úr:
 Jer, pihenj, nyugodj keblemen!

;Mason Lowell, 1856
>422
/1
#DD15379B
 Hadd menjek, Istenem, Mindig feléd,
 Fájdalmak útjain Mindig feléd.
 Ó, sok keresztje van, De ez az én utam,
 Mert hozzád visz, Uram, Mindig feléd.
/2
#FD400245
 Ha este száll reám
 S csöndes helyen
 Álomra hajtanám
 Fáradt fejem:
 Nem lesz hol nyughatom,
 Kő lesz a vánkosom,
 De álomszárnyakon
 Szállok feléd.
/3
#106B2CC3
 Szívemtől trónodig - Mily szent csoda -
 Mennyei grádicsok Fényes sora,
 A szent angyalsereg Mind nékem integet;
 Ó, Uram, hadd megyek Én is feléd!
/4
#2A0C57C5
 Álomlátás után
 Hajnal ha kél,
 Kínos kővánkosom
 Megáldom én.
 Templommá szentelem,
 Hogy fájdalmas szívem,
 Uram, hozzád vigyem,
 Mindig feléd!
/5
#C4653057
 Csillagvilágokat Elhagyva már,
 Elfáradt lelkem is Hazatalál.
 Hozzád ha eljutok, Lábadhoz roskadok:
 Ottan megnyugodhatok Örökre én!

>423
/1
#2E8F8FBF
 Akik bíznak az Úr Istenben,
 El nem vesznek életükben:
 Megtartatnak kegyelemben.
/2
#6960994F
 Az Istennek szent beszédében,
 Kik maradnak szerzésében,
 Hitük által idvességben:
/3
#0271C036
 Ezek lésznek nagy erősségben,
 Mint egy kőszál, keménységben:
 Megtartatnak kegyelemben.
/4
#7E5923AA
 Sokszor esnek kísértésekben,
 De ők el nem vesznek ebben:
 Megmaradnak igaz hitben.
/5
#F5D890BD
 Akik pedig hitetlenségben
 Élnek és nagy kevélységben:
 Mind elvesznek azok ebben,
/6
#DF13F643
 De a hívek nagy békességben,
 Lésznek örök dicsőségben:
 Hitük által idvességben.

>424
/1
#542AD693
 Ó, mely boldog az oly ember éltében,
 Akit az Isten bévett kegyelmélben,
 Sok vétkeiről elfeledkezett,
 Bűnös fejének megkegyelmezett,
 Boldog az s lehet teljes bizodalma,
 Kinek a Krisztus paizsa s oltalma,
 Mert nem árt annak a bűn s a halál,
 Ámbár a törvény ellene fennáll.
/2
#7F1BA1CC
 A törvény súlyát hogy kezdém vizsgálni,
 Eröm mivoltát azon megpróbálni,
 Azonnal lelkem, jaj, elrémüle,
 Testem és vérem bennem elhűle.
 Mert tudom nyilván, mely nagy gyarlóságom,
 Ó, mely számtalan bűnöm s adósságom;
 Ha törvény szerint bán az Úr velem,
 Soha nem fordul napra éjjelem.
/3
#F4454EDB
 De hogy szívemet emelém végtére
 Az én Uramnak elégtételére,
 Érdemét hittel hozzám kapcsoltam,
 Legottan lelkemben megújultam.
 Mert igazsága reám kiterjedett,
 Melyért a nagy Úr nékem megengedett,
 A törvény átkát rólam elvévé,
 Nagy kegyelmének tőn részesévé.
/4
#B6C3967C
 Hát benned vagyon minden bizodalmam,
 Ó, áldott Jézus, paizsom s oltalmam,
 Tudom, vétkeim sokak és nagyok,
 Magamban féreg, por, hamu vagyok.
 De te kedves vagy, Uram, az Istennél,
 Érdemed drágább előtte mindennél,
 Ha igazságod pártomat fogja,
 Soha nem lészek a halál foglya.
/5
#89E9E040
 Azért hitemmel tehozzád elmégyek,
 Nem hogy érette bért kívánjak s végyek,
 Mert tudom: Istennek ajándéka,
 Hát szívemnek ez teljes szándéka,
 Hogy érdemedet elfogadjam ezzel,
 Mint drága jóra kinyújtatott kézzel,
 És megmutassa együgyűségem,
 Hogy benned légyen én reménységem.
/6
#DAEF7F7B
 De mivel gyarló szolgád e hűségben,
 Te, ki lakozol odafent az égben,
 Küldd el lelkedet, hogy vezéreljen,
 Tanítson: hozzád mennem mint kelljen.
 Ó, ne hagyj Uram, ne hagyj el engemet,
 Oltalmazd híven én gyarló fejemet,
 Hogy holtom után hozzád mehessek,
 Az igazak közt bő részt vehessek.

;Jiddis dallam, 1770.
>425
/1
#2DE2275C
 Ó, Ábrahám Ura,
 Hadd áldjuk szent neved,
 Mert mindenható vagy és örök szeretet.
 Nagy Isten a neved,
 Ezt vallja föld és ég,
 Csak téged illet tisztelet és dicsőség.
/2
#6230D4FA
 Ó, Ábrahám Ura,
 Ím, hallom szent szavad;
 Csak azt az üdvöt keresem, mit kezed ad.
 A múló földi jót
 És vágyát elhagyom,
 S őt választom, ki őrizőm és pásztorom.
/3
#06ECBA9C
 Ó, Ábrahám Ura,
 Szent kegyelmed nekem
 Az én örömöm, utamon ez vezessen.
 Te barátod lettem,
 Én Istenem te vagy:
 Tarts meg a Jézus véréért és üdvöt adj!
/4
#FC18625A
 Megesküvél, Uram,
 És igédben bízom,
 Hogy égbe viszed gyermeked sasszárnyakon.
 Meglátom Jézusom
 És áldom hatalmát,
 Szent kegyelmének éneklek halleluját.

;Walesi népénekdallam
>426
/1
#FCD82B37
 Már keresztem vállra vettem
 S érted mindent elhagyok.
 Mindenem vagy, árva lettem,
 Honjavesztett szív vagyok.
 Vágyat, célt a múltnak adtam,
 Nincs már bennem vak remény,
 Mégis gazdag úr maradtam:
 Isten és a menny enyém.
/2
#95740AAC
 Ember bánthat és zavarhat:
 Szíved áldott menedék;
 Sorsom próbál és sanyargat:
 Édes csenddel vár az ég.
 Nincsen búm, mely könnyet adjon,
 Míg szerelmed van velem,
 Nincs öröm, mely elragadjon,
 Hogyha nem benned lelem.
/3
#959A9559
 Lelkem, teljes üdv a részed,
 Hagyd a bút s a gondot el;
 Légy vidám, ha meg-megérzed:
 Tenni kell még s tűrni kell.
 Gondold el: ki Lelke éltet,
 Milyen Atya mosolya;
 Megváltód meghalt teérted:
 Mit bánkódnál, menny-fia?
/4
#27BF553B
 Kegyelemből dicsőségbe
 Szállj, hited majd szárnyat ad,
 S az örök menny fénykörébe
 Bévezet majd szent Urad.
 Véget ér itt küldetésed,
 Elszáll vándoréleted,
 Üdvösséggé lesz reményed,
 Égi látássá hited.

>427
/1
#9AAF4850
 Lelkem hozzád kíván menni,
 Veled lenni,
 Ó, életem kővára,
 Mert nálad talál oltalmat,
 Nyugodalmat,
 Te lehetsz várt javára.
/2
#86045B7C
 Nem bízhatom érdememhez,
 Mert fejemhez
 Köttetett a gyarlóság;
 Ha vizsgálom testem s vérem,
 Megesmérem:
 Nincs bennem semmi jóság.
/3
#EC92E7A1
 Az Úrnak parancsolatját,
 Akaratját,
 Ó hányszor hágtam által!
 Hányszor volt munkám éretlen
 És rendetlen
 Erőtlenségem által!
/4
#0ED5BB19
 Ha lelkem gerjed a jóra,
 Majd sátora,
 A halandó romlott test
 Készségét balra fordítja,
 Megtompítja,
 Hogy légyen a jóban rest.
/5
#F65180FB
 Ezért gyakran azt mívelem,
 Ha nincs velem
 Az Úrnak segedelme,
 Miről tudom, hogy helytelen,
 Törvénytelen
 S lelkemnek veszedelme.
/6
#462762A1
 Ily bűnös gyarló voltommal,
 Oltalommal
 ügyemben nem lehetek,
 Tettető igazságomért
 Jutalmat s bért
 Hasztalan követelek.
/7
#B27ED812
 Mert ki saját erejének, érdemének
 Nagyságát veti, hányja,
 Szükség: légyen feddhetetlen
 Okvetetlen,
 Mint a törvény kívánja.
/8
#6F7D141D
 Tudom, hogy a kegyességnek
 És hűségnek
 Vagyon drága jutalma;
 De azt adja kegyelemből,
 Nem érdemből
 Az Istennek irgalma.
/9
#1364E47A
 Én azért ez oltalomhoz,
 Mint kőfalhoz
 Egyedül támaszkodom;
 Jézusomnak kegyelméhez,
 Érdeméhez
 Jó hittel ragaszkodom.
/10
#C87CD4C4
 Te pedig, ki által élek,
 Ó Szentlélek,
 Úgy igazgasd dolgomat,
 Hogy e jót nálam tarthassam
 S megmutassam
 Hálaadó voltomat.

>428
/1
#6AB7C6A1
 Látod, Úr Isten, szívünket,
 Tudod erőtlenségünket,
 Nem rejtjük el bűneinket:
 Várjuk kegyelmességedet.
/2
#77903DF2
 Mindnyájan arcra borulunk,
 Szent Atyánk, hozzád kiáltunk:
 Felejtsed el bűneinket,
 Tengerbe el-bévesd őket.
/3
#FE69871B
 Világositsad elménket,
 Gerjeszd fel a mi szívünket,
 Hogy dícsérhessünk tégedet,
 És áldhassuk szent nevedet.
/4
#A24A4364
 Hajlékoddá mi szívünket
 Szenteljed, kérünk tégedet;
 Szíveinkben lakodalmat
 Szerezz, és nagy nyugodalmat.
/5
#08B4C5FA
 Így cselekedjél mivélünk
 Szent Fiadért, ki minékünk
 Nálad bizonyos reményünk,
 Hozzád egyenes vezérünk.

>429
/1
#87531B35
 Az Úr énnékem hűséges vezérem,
 Főpásztoromnak csak őtet ösmérem;
 Drága jóvoltát elrejtem szívemben,
 Szent nevét áldom egész életemben,
 Mert kegyelméből én érdemem felett
 Sok lelki jókkal engemet körülvett.
/2
#E0074F9B
 Szent beszédének gyönyörűségével,
 Áldott lelkének titkos erejével
 Sötét álomból híven felébresztett,
 Az én szívemben oly hitet gerjesztett,
 Mely bánatában lelkem vidámítja,
 Jó reménységgel naponként újítja.
/3
#0D938655
 Sőt, hogy még inkább érezzem kegyelmét,
 Várjam bizonnyal igért segedelmét:
 Jósággal teljes, édes indulatja
 Sákramentomit vélem kóstoltatja,
 Így édesíti szívemet magához,
 Hogy erös hittel fussak oltalmához.
/4
#E3845BC5
 Csak az szívemnek hozzád óhajtása,
 Ó, idvességnek fakadó forrása:
 Ha kegyelmedből már Atyámmá lettél,
 Végezd el a jót, amit elkezdettél;
 Reménységemnek add meg kívánt végét:
 Az én lelkemnek örök idvességét.

>430
/1
#783DCC0B
 Úr Jézus, idvesség fejedelme,
 Én életemnek hű segedelme,
 Megtérek hozzád hálaadással,
 Szent neved áldom énekmondással,
 Emlékezeted elrejtem szívemben,
 Rólad éneklek egész életemben.
/2
#B500B078
 Mert gyarló voltom bűnben eredett,
 De nagy kegyelmed rám kiterjedett,
 Istenem voltál kezdettöl fogva,
 Nem hagytad a bűnt, hogy tartson fogva,
 Sőt a világtól engem elszakasztál,
 Szent szerelmedből magadnak választál.
/3
#302EB379
 Eljegyzél kegyelmed pecsétjével,
 A keresztségnek drága vizével,
 Zászlód alá én nevem beírtad,
 Néped közé szolgád befogadtad,
 Erőtlenségem lelked segítette,
 Házad javait velem éreztette.
/4
#F43CE53D
 Testtől születtem s merő test voltam,
 Bűnöm sírjában szinte megholtam,
 Ámde szent véred drága hullása
 Lőn én lelkemnek megtisztulása,
 Újjászült Lelked, hathatós voltával,
 Felékesített égi koronával.
/5
#F1143439
 Hát világ, test, bűn, tőlem távozzál,
 Bennem nincs részed, bármit okozzál!
 Uram pecsétjét én reám vettem,
 Már kegyelméből tagjává lettem,
 Fejedelmemnek csak őtet esmérem,
 Igéje s Lelke lesz holtig vezérem.
/6
#5932B6F9
 Téged is kérlek, ó áldott Lélek,
 Vezess a jóban, amíglen élek,
 Hogy a bűnöknek épen meghaljak,
 Szent életet kövessek és valljak,
 Hogy az én Uram engem megkedveljen
 És végre bennem öröme beteljen.

;Greiter M., Strasbourg, 1525 (1539) után
>431
/1
#25C91244
 Úr Isten, kérünk tégedet:
 Keresztelj és moss meg minket,
 És tisztíts meg kegyesen;
 A Krisztusnak ő vérével
 Nagy bűneinket töröld el
 Szent lelked erejében.
 Amit e szent fürdő jegyez,
 Mindent mi bennünk megszerezz:
 Végy körül szerelmeddel,
 Hogy a te szövetségedben
 Megmaradjunk mindvégiglen,
 Minden mi gyermekinkkel.

;Énekelhető más szapphói dallamra is.
;Debrecen, 1774
>432
/1
#18DEF49C
 Mi kegyes Atyánk, bölcsességnek Ura,
 És mindeneknek nagy bölcs tanítója:
 Akinek nem vagy te igazgatója:
 Nincs oktatója.
/2
#75EF9BED
 Mi, te fiaid, tudatlanságunkban,
 Gyermekségünkben és ifjúságunkban,
 Néked könyörgünk mi tanulásunkban,
 Imádságunkban.
/3
#530BCD57
 Világosítsd meg tudatlan elménket,
 Tanulásunkban vezérelj bennünket,
 A tudományra gerjeszd fel szívünket
 És mi lelkünket.
/4
#1B374EB3
 Adj jó tanító mestereket nékünk,
 Adj bölcsességet általuk minékünk,
 Adj jó erkölccsel, értelemmel élnünk,
 Mi jó Istenünk!
/5
#F4090A9F
 Őrizz meg minket gonosz erkölcsöktől,
 Minden naponként fertelmes beszédtől,
 Szent színed előtt utált részegségtől,
 Éktelenségtől.
/6
#DD8096D4
 Hogy jövendőre mi felnevekedvén,
 Téged dicsérjünk, mindenkor tisztelvén,
 Felmagasztaljunk, nagy jámborul élvén,
 Néked engedvén.
/7
#AD429E6D
 Hogy szolgálhassunk te szent egyházadnak,
 És használhassunk felebarátinknak,
 De kiváltképpen atyánknak s anyánknak,
 Édes hazánknak.
/8
#832C95A1
 Hogy holtunk után a nagy iskolában,
 Szentháromságnak ő tanításában,
 Az angyaloknak szép társaságokban
 Tanuljunk jobban.
/9
#AD7D2FF3
 Dicsőség néked, kegyes Atya Isten,
 És uralkodó áldott Fiú Isten!
 Te, vigasztaló Szentlélek Úr Isten!
 Áldj meg hitünkben!

>433
/1
#15B09B78
 Áldjad, én lelkem, felséges Uradat,
 Dicséretére emeld fel szavadat,
 Jóvoltát hirdesd lelki örömmel!
 Terólad, Uram, csak rólad éneklek,
 Mindenek felett Tégedet szeretlek,
 Mert voltál hozzám nagy kegyelemmel.
/2
#37D46A2F
 Gyarló éltemnek kezdetitől fogva
 Tartott kegyelmed én kezemnél fogva,
 Te szent szemeid én reám néztek;
 Erőtlen voltom még gyenge korában
 Talált nyugalmat szárnyad árnyékában,
 Midőn mellőlem sokan elvesztek.
/3
#976A6F7D
 Keresztyén szülék által felneveltél,
 Engem, méltatlant, frigyedbe bévettél,
 Tettél házadnak élő tagjává.
 Szent Irgalmadból énreám is tére
 Áldott Fiadnak drágalátos vére,
 Hogy a bűn ne tenne prédájává.
/4
#65A2D93A
 A keresztségnek külső fürdőjével,
 Mint kegyelmednek bizonyos jelével
 Megmosád gyenge s erőtlen testem;
 Így szent házadban engem béavattál,
 A pogányoktól híven elszakasztál,
 Midőn én azt még nem is kerestem.
/5
#05A4C422
 Bízzál hát, lelkem, a te Istenedben,
 Ki Atyáddá lett még gyengeségedben.
 Mint a vízzel megmosá testedet,
 Úgy szent Fiának drága érdemével,
 Azzal érdemlett áldott Szentlelkével
 Bűneidből megtisztít tégedet.
/6
#F60DAB66
 Tehozzád is, csak tehozzád emelem
 Szívemet, ó nagy jóság és kegyelem;
 Idvességemet tetőled várom.
 Ne hagyj el engem, ha Istenem lettél,
 Végezd el a jót, amit elkezdettél,
 Ó én Uram, Atyám és kővárom!

;Drese A., Darmstadt, 1698
>434
/1
#64893AA3
 Vezess, Jézusunk,
 S véled indulunk.
 Küzdelemre hív az élet,
 Hadd kövessünk benne téged;
 Fogjad a kezünk,
 Míg megérkezünk.
/2
#8D4C9548
 Adj erős szívet,
 Hogy legyünk hívek.
 És ha terhet kell viselnünk,
 Panaszt mégsem ejt a nyelvünk;
 Rögös bár utunk,
 Hozzád így jutunk.
/3
#BB4C79C2
 Sebzett szívünk majd
 Mikor felsóhajt,
 Vagy ha másért bánat éget,
 Adj türelmet, békességet,
 Reménnyel teli
 Rád tekinteni.
/4
#6C80BA54
 Kísérd lépteink
 Éltünk végéig,
 És ha roskadozva járunk,
 Benned támaszt hadd találunk,
 Míg elfogy az út
 S mennyben nyitsz kaput.

;Debrecen, 1781
>435
/1
#CBD5F6D9
 Lelkem siet hozzád menni,
 Ámbár gyenge ereje,
 Kíván asztalodról enni,
 Ó, életnek kútfeje!
 Hogy megelégülhessen,
 Benned része lehessen.
/2
#BEF8D08E
 Kedveld, Uram, kegyességét
 A te szegény szolgádnak,
 Éreztessed édességét
 Elkészült vacsorádnak,
 Hogy véled egyesüljön
 S lelke benned örüljön.

;Kolozsvár, 1837
>436
/1
#F7CCC377
 Örülj, szívem, Vigadj, lelkem,
 Ékességed lett a hit;
 Vacsorához, Mégy Jézushoz,
 Hivatalos vagy te itt.
/2
#A167F734
 Ha bűnödért Halálos bért
 Érdemlettél lelkedre:
 E szent asztal Megvigasztal
 S válik idvességedre.
/3
#1A0E13AF
 Ez örömben, Reménységben,
 Jézus, ma hozzád jövök.
 Asztalodnál Lábam megáll:
 Testem, lelkem újítsd meg.
/4
#3E0CBB72
 Tisztogass meg, Bűnből moss meg
 Kegyelmedből, Istenem,
 És a Sátán lelkem kárán
 Nem örül, nem árt nekem.
/5
#E04F7F75
 E vacsora Égi módra
 Engem véled összead,
 Maradj Bennem, benned engem
 Hagyj lennem, hogy áldjalak.
/6
#4699FEDF
 Csakhogy immár Bűntől elválj
 S légy hívséggel Jézushoz:
 Halálával, Váltságával
 Minden bűnből feloldoz.
/7
#AB03C74A
 Hát jöjjetek, Bűnös lelkek,
 Orvosságot kik vártok!
 Jézus lelke, Szent kegyelme
 Kiárad ma reátok.

;Genf, 1542
>437
/1
#DE593A3D
 Ó, Jézus, mi idvességünk,
 Fejedelmünk, dicsőségünk,
 Szívünknek fő vigasztalója!
 Rólad, csak rólad énekel
 Szánk buzgó dicséretekkel,
 Lelkünknek édes táplálója;
 Egyedül csak hozzád térünk,
 Mert hű pásztornak ismerünk.
/2
#8A1CCEE9
 Ó, mily szent az az indulat,
 Mely veled örömest mulat,
 Ó, Jézus, idvesség forrása!
 Boldog, akit hívsz magadhoz,
 Méltóztatsz szent asztalodhoz,
 Hogy kegyelmed jelét ott lássa.
 Bizony, boldog az oly ember,
 Ki nálad kedvességet nyer.
/3
#EFF2E449
 Mert érdemedben részesül,
 Teveled, Uram, egyesül,
 Ki hív tagja e vendégségnek.
 Mint a bort s kenyeret veszi,
 Akképp részesévé teszi
 A hite őt az idvességnek,
 Melyhez ártatlan Jézusunk
 Vére által vagyon jussunk.
/4
#118F2CD6
 Azért, kit bűnök rongálnak,
 Járulj e kegyes királynak
 Víg örömmel szent asztalához!
 Nem vet el ez irgalmas szív,
 Mert könyörülő, igaz, hív,
 Csak hogy folyamodj oltalmához:
 Bévesz sebe rejtekébe,
 Mint ígéri szent jelébe.
/5
#8E7FC0FC
 Nincs is egyébben oltalmunk,
 Reménységünk, bizodalmunk,
 Édes Jézus, hanem csak benned.
 Szállj hozzánk, kedves vendégünk,
 Oltsd meg lelki nagy éhségünk,
 Mert eledelünk csak érdemed;
 Szívünket, ímé, kitárjuk,
 Szentségedet várván várjuk.

;Bourgeois L., Lyon, 1547
>438
/1
#85D00828
 Hallottuk, Jézus, miképpen hívogatsz,
 Sietünk hozzád, tudjuk, hogy megnyugtatsz.
 Vállunkat nyomja nagy terhünk súlya,
 Félünk, erőnket hogy felülmúlja.
 Fáradtak vagyunk, régóta emeljük,
 Mert együtt jött ez a világra velünk;
 Jézus, terhünktől légy szabadítónk,
 Fáradtságunkban légy megújítónk.
/2
#29E6DF79
 Törődött szívvel teszünk neked vallást:
 Hogy nem követtük a jó útmutatást,
 Melyet sugallt lelkünk ismerete,
 Mikor minket jóra serkentgete;
 Inkább követtük gyarló testünk kényét,
 Mint idvességes törvényid ösvényét;
 Ezért nagy lévén szívünk keserve,
 Állunk előtted, mellünket verve.
/3
#E4E13751
 De mégis lelkünk e hittel élesztjük:
 Hogy te, Megváltónk, nem akarod vesztünk;
 Hanem magadat adtad érettünk,
 Életedet letévén helyettünk.
 A törvény átka tégedet üldözött,
 Míg felfüggeszte az ég és föld között,
 Tulajdon tested ott megáldozád
 És e világ bűnét elhordozád.
/4
#BAC3DE99
 Ezért az Atya magával bennünket
 Megbékéltetvén s eltörlé bűnünket;
 Ellenségiből tett fiaivá;
 Szent Fia örökös társaivá;
 Jézus! te halállal tévén eleget,
 Megengesztelted a földhöz az eget;
 Általad, amit más nem tehetett,
 A teljes váltság elvégeztetett.
/5
#DCC8FEB4
 Azért már minket éleszt és vigasztal
 Ez az általad rendeltetett asztal;
 Mely annak nyilvánvaló tüköre:
 Véred mint omla, tested mint töre!
 Sőt tested jegyét itt nemcsak mutatod,
 De személy szerint nekünk is átadod:
 Zálog ez, hogy van nekünk is jogunk
 A tőled szerzett jókhoz, Jézusunk!
/6
#6B87B8C5
 Szállj le most mennyből, életnek kenyere!
 Tápláld lelkünket az örök életre!
 Tudjuk: aki e kenyérből eszik,
 Soha örökké meg nem éhezik.
 Életnek vize, nyiss magadnak utat,
 A szomjú hívek keresik e kutat;
 Szolgáltasd ingyen az italokat:
 Oltsd el végképpen szomjúságukat.
/7
#7CCD9363
 Óhajtunk, Jézus, egyesülni veled:
 Úgy lesz szívünk szent, ha te megszenteled.
 Adjad hát, hogy mint tagok a főnek,
 Engedjünk néked, bennünk élőnek.
 Olts be magadba, mint jó szőlőtőbe,
 Hogy jó nedvesség folyjon a vesszőbe,
 És légyen szívünk szívednek mása,
 Éltünkben élted hogy minden lássa.
/8
#7E74D1E1
 Adjad: egymás közt legyen egyességek,
 Akik e közös asztalnál vendégek;
 Mert bizony, aki másokat szeret,
 Csak az eszik itt méltán kenyeret.
 Fogjon hát ez az egész gyülekezet
 Egymással atyafiságosan kezet,
 Mert tudjuk: maga a hívogató
 Úr Jézus nem személyválogató.
/9
#1BAFC270
 Megelégítvén lelkünket ez étel,
 Hálaadással távozzunk innét el,
 Mondván: az Úrnak dicsőség legyen,
 Ki az éhezőt betölti ingyen.
 Tartsd fenn, Úr Isten, rajtunk jóvoltodat,
 Add nekünk immár kegyes válaszodat:
 „Bízzál már, fiam, bízzál, leányom,
 bűneidet szemedre nem hányom.”

;Bourgeois L., Genf, 1551
>439
/1
#E6B9508D
 Én lelkem, ébredj fel az Úrnak dicséretére,
 És szent emlékezetére
 Víg szívvel gerjedjél.
 Kegyelmét keressed
 E te lelki orvosodnak,
 Hű és édes táplálódnak
 Jóvoltát hirdessed.
/2
#FA609F7B
 Mert éhező valál,
 A szomjúság elepesztett,
 A bűn oly kútba rekesztett,
 Hol kész volt a halál.
 De lelke megesett
 Rajtad e kegyelmes Úrnak,
 Látván, nyavalyáid dúlnak,
 Téged felkeresett.
/3
#33618D98
 A kútból kihozott,
 Az igazságnak mezején
 És az életnek kútfején
 Vezérlett, hordozott.
 Végre vitt magához,
 Bíborban felöltöztetett,
 És kegyelméből ültetett
 Ő szent asztalához.
/4
#E071D84C
 Hol megelégített
 Megtöretett szent testével
 És kifolyt drága vérével,
 melyet elkészített.
 Boldog, ki ezzel él,
 Mert a Sátán annak nem árt,
 Semmi veszély nem tehet kárt,
 A haláltól sem fél.
/5
#1BD11B5A
 Azért, kik éheztek,
 Jertek az Úr asztalához,
 Siessetek jóvoltához,
 És megelégesztek.
 Mert az Úr nem hagyja
 A töredelmes lelkeket,
 Sőt táplálja mindezeket:
 Magát nekik adja.
/6
#D1931AA7
 Mint a száj részesül
 A látható szent jegyekben,
 Úgy e nagy Úr a hívekben
 Él, s vélek egyesül.
 Lesz édes dajkájuk:
 Megtöretett szent testének,
 Kiontott drága vérének
 Hasznát szabja rájuk.
/7
#5173E586
 Hát lelkem, tégy vallást,
 Hogy igaz szívvel szereted
 Ez Urat, s híven követed,
 És néki adsz szállást.
 Jöjj be hozzám, kérlek,
 Jézus, én kedves vendégem,
 Mert te vagy én dicsőségem,
 Csak téged kedvellek.

;Bourgeois L., Genf, 1551
>440
/1
#B76A92AD
 Jer, lássuk az Úr keresztjét,
 Melyet felvett érettünk,
 Megszánván embernek vesztét,
 Megfizete helyettünk.
 Úr lévén, lett szolgává,
 Mindeneknek csúfjává,
 Istenségét elrejtette,
 Midőn testünket felvette.
/2
#16A630D9
 A Gecsemáné kertjében
 Kezdette szenvedését
 Érezvén a kínt lelkében
 S a halál rettentését,
 Ivék keserű pohárt,
 Fizette értünk az árt;
 Véres cseppjei testének
 Nékünk gyógyulást szerzének.
/3
#6C411D88
 Itt szintén lelkéig hata
 Sebes vizek mélysége,
 Itt kezdődék a nagy csata
 S lelke keserűsége:
 Egyedül hagyattaték,
 Éjjel megfogattaték;
 Ki mindent kézen fogva tart,
 Megfogattatni így akart.
/4
#9D64FB6E
 Feszítésre ítélteték
 S gyalázatos kínokra,
 Pilátushoz vitetteték,
 Önként ment ez átokra.
 Ki ügyét hogy meghallá,
 Őt ártatlannak vallá,
 Mégis adá keresztfára,
 A legkínosabb halálra.
/5
#193F76AE
 Mint Prófétát, csúffá tették,
 Mert befedvén szent fejét,
 Verték, s ki verte? kérdezték,
 Kísértvén szent erejét.
 Mint Királyt, meggyalázák,
 Tövissel koronázák,
 Náddal verék a bársonyban,
 Csúfolák térdhajtásokban.
/6
#6EE31E97
 A fát adván szent vállára,
 Mint Pap, meggyaláztaték,
 Az is lőn gyalázatjára,
 hogy megostoroztaték.
 A vitézek kifoszták,
 És ruháit megoszták;
 A nagy bűnöst elereszték,
 A szentet fára függeszték.
/7
#917D5FC9
 Mily csuda buzgó szerelem:
 Meghalni barátiért,
 De e kegyes Fejedelem
 Meghalt ellenségiért.
 Ez lelkünk drága bére,
 Mert Isten fia vére:
 Mily drága az az áldozat,
 Mellyel romol a kárhozat.
/8
#BACE1242
 Ezért szerzé szent asztalát,
 Hogy e jót előadja,
 Szenvedését és halálát
 Szemeink előtt hagyja.
 A kenyér megtörése,
 A bornak kitöltése
 Lelke, teste szenvedését,
 Jelenti megöletését.
/9
#61ED168A
 Ez egyszerű vendégségben
 Jézussal egyesülünk,
 Egy kenyérből e szentségben
 Hívők mind részesülünk.
 E rövid szent vacsora
 Mutat mennyei jóra;
 Ezt az ő kínt látott teste
 Választottinak kereste.
/10
#CA7F8B09
 Az Úr sebei mi sebünk,
 Halálunk ő halála;
 Ő érdeme mi érdemünk,
 Részesülés ez nála.
 Pecsétli ezt az étel,
 Együtt a pohárvétel:
 Testét midőn hittel esszük,
 Magunk ő testévé tesszük.
/11
#C9D98532
 Ez eledel zálog nékünk,
 Hogy lesz mennyben menyegzőnk,
 Ezzel hitünk, reménységünk:
 Öregbül lelki erőnk.
 De mit ér sokszor enni,
 E szent vacsorát venni,
 Ha lélek szerint nem élünk,
 Büntetést enni nem félünk?!
/12
#B1898B16
 Ha meghaltunk Úr Jézussal,
 Nem illik már vétkeznünk;
 Feltámadván Úr Krisztussal,
 Le kell a bűnt vetkeznünk.
 Mint mennyei lakosok,
 Legyünk tiszták s okosok;
 Mint testének s vendégének,
 Szükséges lennünk szenteknek.
/13
#8A328D35
 Isten ártatlan Báránya,
 Méltó vagy, hogy végy áldást,
 Mert érdemed azt kívánja,
 Végy tőlünk hálaadást!
 Végy örök dicsőséget,
 Hatalmat, tisztességet,
 Mert értünk megölettettél,
 Atyádnak eleget tettél.
/14
#4954C03A
 Add, Úr Jézus, halálodban,
 Követnünk életedben,
 Szentséges tudományodban,
 Szelíd szenvedésedben.
 Légyünk alázatosak.
 Mindenhez irgalmasak;
 Segélj erre Szentlelkeddel,
 Oktass, vezérelj igéddel!

;Debrecen, 1774
>441
/1
#25D7980D
 Kegyes Jézus, én imádságomra,
 Hajtsad füled én kiáltásomra,
 Jusson hozzád kérésem én jómra,
 Ne nézz, Uram, méltatlan voltomra!
/2
#282BF8BF
 Szomorkodom undok bűneimért,
 Fohászkodom sok gonoszságimért,
 Félek, Uram, mert sok rútságimért
 Megutáltál, tudom, én bűnömért.
/3
#7A998CBD
 Tökéletes a te ítéleted,
 Tiszta, szent vagy, a bűnt nem nézheted,
 A vakmerő bűnöst megbünteted,
 Országodból méltán kirekeszted.
/4
#F451DD2B
 Félek, azért megvallom, Istenem,
 Hogy ha úgy bánsz, mint érdemlem, velem,
 Szent színedet elfordítod tőlem,
 Országodból kirekeszted lelkem.
/5
#B68E4F20
 Mert ha végignézek életemen,
 Látom, hogy a bűnt gyermekségemen
 Kezdettem, és azóta szüntelen
 Növekedett a teher lelkemen.
/6
#ABA0E2AA
 Múló hasznát a bűnnek szerettem,
 Törvényedet azért félretettem;
 Felebarátomat sértegettem,
 Melyért hozzád méltatlanná lettem.
/7
#0342F747
 De ez egyben biztatásom lelem,
 Hogy tenálad vagyon a kegyelem;
 Azért hozzád szívemet emelem:
 Cselekedjél irgalmasan velem.
/8
#83F14935
 Mert nem azért vetted fel testünket,
 Hogy megítélj bűnünkért bennünket,
 Hanem hogy megszabadítsd lelkünket
 és megszerezd örök életünket.
/9
#CD8590A3
 A megtérő bűnösnek kedveztél,
 Bűnbocsátó kegyelmet ígértél,
 Isten előtt érte kezes lettél,
 Minden átkot fejéről elvettél.
/10
#676CC7EA
 Kívánkozom én is hozzád térni,
 Segélj, Uram, e szent célt elérni,
 Mert így lehet kegyelmedet kérni,
 Így lehet csak tőled azt megnyerni.
/11
#BE844172
 Segélj, segélj, én édes Istenem!
 Néked adom s ajánlom ma lelkem:
 Kérlek, fogadd magadhoz, Jézusom,
 Idvességem s életem, Krisztusom!
/12
#A39B7192
 Míg itt élek, éltem vezéreljed,
 Tartóztasson a bűntől törvényed;
 Minden jóra segéljen Szentlelked,
 Kegyelemmel biztasson érdemed.
/13
#20363EEE
 Mikor érzem halálom óráját,
 Biztass, hogy jól kifutván a pályát,
 Majd meglátom az Isten orcáját,
 Elnyerem az élet koronáját.

;Kolozsvár, 1744
>442
/1
#5A4D5E28
 Idvességünk, váltságunk,
 Jézus! hozzád kiáltunk;
 Midőn a kegyelem asztalához lépünk:
 Jövel, maradj mivélünk!
/2
#22BE980F
 Nagy volt szüleink bűne,
 De Atyád könyörüle;
 És az örök halál tüzéből megváltott
 Engesztelő halálod.
/3
#2BE9D204
 Ez a terített asztal
 Bűneinkben vigasztal:
 Hogy könyörülsz rajtunk s kegyelmed velünk lesz,
 Ha bízunk érdemedhez.
/4
#B8FE32FF
 Szent testednek jegyei
 Annak itt a jelei:
 Hogy csak az igazán megtérő találhat
 Idvességet tenálad.
/5
#19A794E2
 Vérednek kiömlése
 Bűneink eltörlése;
 Emlékeztess, midőn ajkunk a bort issza:
 Légyen éltünk szent s tiszta!
/6
#971CA33C
 Jézus! a te véred szent!
 Légyen szívünk bűntől ment;
 mert örök életet veled az remélhet,
 Ki megveti a vétket.
/7
#DDF46BE6
 Szent törvényed fáklyája
 Világoljon pályánkra!
 Mert e földi úton, tudjuk, el nem téved,
 Ki híven követ téged.
/8
#8ABEEDD4
 Fogjad azért kezünket,
 S magad vezérelj minket;
 Mert a te utadon Istenünkhöz vezet
 A hit, remény, szeretet.
/9
#4B871528
 Ha szemeink meglátnak
 Jobbján te szent Atyádnak:
 Ott a Szentlélekkel örök egyességben
 Üdvözíts majd az égben!

>443
/1
#38360DD3
 Lelkem, adj dolgot magadnak,
 Hirdessed édes Uradnak
 Jóvoltát és felségét.
 Hirdessed ajándékinak,
 Melyeket ad barátinak,
 Nagyságát és szépségét.
 Uram, te reád gerjedez
 Szivem, és rólad zengedez
 Nyelvem kész indulattal,
 Mert voltál s vagy te népedhez,
 Áron vett örökségedhez
 Drága jó akarattal.
/2
#78FBAA46
 Kijelentéd irgalmadban,
 Szentséges végvacsorádban,
 Hogy áldozattá lettél;
 A benned bízó hívekért,
 Halált nemző vétkeikért
 Egyszer eleget tettél.
 Kikkel is végig lakozol
 És karjaidon hordozol,
 Sőt azokkal egyesülsz;
 Szentlelked által bennük vagy,
 Noha felséged dicső nagy,
 S az Atyának jobbján ülsz.
/3
#64D838E8
 Azért, akik tusakodnak,
 Bűneiken szomorkodnak
 És hittel reád néznek;
 Idvezítésedet várják,
 Bűn előtt szívük bezárják,
 Megtérni igyekeznek;
 Kívánnak néked engedni.
 Szent életben nevelkedni
 És futnak jóvoltodhoz:
 Azok, mint édes fiaid,
 Kebeleden nőtt juhaid,
 Méltók szent asztalodhoz.
/4
#6C7720B9
 De a színes képmutatók,
 Bűnt vetők, gonoszt aratók
 Innen távol menjenek;
 Megtérni nem tudó szívek,
 Álnokságra felvont ívek
 E jóval ne éljenek.
 Ti, álnokságnak fiai,
 Fussatok el, mert javai
 Az Úrnak nem tiétek;
 Ha éltek itt méltatlanul,
 A harag csalhatatlanul
 Reátok száll, higyjétek.
/5
#FC8FBD17
 Hát lelkem, hallgasd Uradat,
 Próbáld meg tennenmagadat,
 Úgy járulj ez asztalhoz.
 Higyj, térj meg, a bűnt útáljad,
 Az Istent féljed, szolgáljad,
 Fuss hozzá, mint kőfalhoz.
 Uram, ezekre kész vagyok,
 De hibáim sokak, nagyok,
 Rajtam nagy terhek vannak:
 Ó Jézus, engem ne hagyj el,
 Segélj, tőlem ne maradj el,
 Sőt válassz el magadnak.

>444
/1
#88A9B4A8
 Ó, Istennek választott kedves népe,
 Kit ékesít szentséges lelki képe:
 Vigyázz te belső ékességedre,
 Szépségedre.
/2
#BF60480E
 A te szemeid az Úrra nézzenek,
 Szent rendelési előtted légyenek,
 Az Úrnak kertje hogy virágozzék,
 Illatozzék.
/3
#727F1AEF
 Szent Igéjét híven s tisztán hirdessed,
 Háza tisztaságát híven szeressed,
 A jókat vedd és tartsd kebeledben,
 Szerelmedben.
/4
#F5CF576A
 Akik hitüknek szép jelét mutatják,
 Magukat szent áldozatul állatják,
 Azokat, mint ilIik, csókolgassad
 és biztassad.
/5
#7E09980F
 Az igaz útról kik eltévelyedtek,
 Feslett és rossz életre vetemedtek,
 Térítsd meg őket hasznos intéssel
 És feddéssel.
/6
#655AB98A
 Hogyha szeretetből lett intésednek,
 Megvetvén a bűnt, kész szívvel engednek,
 Jó szemmel tovább is nézz azokra,
 Járj javokra.
/7
#CBE608AC
 Igy az Úr háza nem romlik, sőt épül,
 Ékes voltában napról-napra szépül,
 Plántái nönek az Úr kedvére,
 Örömére.
/8
#B43B45DB
 Te pedig, hívek hűséges pásztora,
 Segéld gyenge népedet minden jóra,
 Hogy a szent igazságot szeresse,
 Azt kövesse.

;Urhan Ch., 1790-1845
>445
/1
#D57E1556
 Szólsz hozzám, Istenem, s én választ adni készen
 Mármár megindulok, hogy rád bízzam magam,
 De látod, köt s lehúz még régi csüggedésem,
 Áldd meg ma lelkemet több hittel, ó, Uram.
/2
#5F35D29A
 Sok szép ígéretem, ó, hányszor megtagadtam,
 A nagy fogadkozást, hogy csak tiéd szívem.
 A bűnös gyengeség bús rabjának maradtam,
 És törvényed szerint nem éltem semmiben.
/3
#994DEA83
 Ha jót tettél velem, ha áldva látogattál:
 Én nem dicsértelek s nem hirdettem neved;
 Nem értettem, mikor szenvedni, sírni hagytál,
 Hogy ha szeretsz, miért sújt vessződ engemet?
/4
#3E3DE79B
 Köt még a földi jó, a bűn, a földi örvény,
 S tehozzád bűnömért, lásd, el nem juthatok.
 A béklyó súlya nyom, levetném, összetörném,
 De lelkem gyenge még s jaj, összeroskadok.
/5
#60016265
 Más nem tanít meg rá, csak égi bölcsességed,
 Hogy bölcsen bízzak és szolgáljak úgy neked.
 Mit érek nélküled? Add, hogy imádva téged,
 Bús, gyarló bűnös én, hadd légyek gyermeked.
/6
#DDC2680F
 Nagy lelked élt, Uram, a prófétás időkben,
 Az fénylett át a szent s apostol életén;
 Áldj meg s kegyelmedet reám is töltsd ki bőven,
 Hogy Jézust nézzem és ővéle győzzek én.

;Bourgeois L., Strasbourg, 1545
>446
/1
#6ECC1D4D
 Uram, bocsásd el népedet békével,
 Idvezítésed kívánt örömével,
 Mert megtaláltuk, amit lelkünk kedvel,
 Megragadjuk hittel, nem bocsátjuk el;
 Már ő miénk, mi vagyunk az övéi,
 Véle egy testek, java részesei.
/2
#A9F07549
 E gyarló testben Jézus él már, nem mi,
 Hű szerelmétől nem szakaszt el semmi.
 Ó, Jézus, gyenge hitünket neveljed,
 Te magad lelkünk az égre emeljed,
 Hogy így lelkünket az ég dicső vára
 Fogja be, annak örökös javára.

;Genf, 1562
>447
/1
#CF1881DC
 Uram, bocsássad el Szolgádat békével,
 Szent ígéreted szerint,
 Mert te idvezítőd
 Én szemeim előtt
 Nékem nyilván megjelent.
/2
#F6E29310
 Kit világos fényül
 Pogányoknak küldél,
 Kinek fényességével
 Nyilván kijelenék
 A szent Izráelnek
 Nagy dicsősége széjjel.

;Debrecen, 1774
>448
/1
#1A39D01A
 Hirdetvén az Úr halálát,
 Uram, néked adunk hálát,
 Hogy nemcsak tartod testünket,
 Hanem táplálod lelkünket.
/2
#043A53E3
 Ma is ételt készítettél
 Ennek, s asztalt terítettél,
 Melyről Fiad szent testével,
 Elégítél szent vérével.
/3
#AE0F48A3
 Ezzel világos jelt adtál
 S minket arról megbiztattál,
 Hogy ha igaz hitünk lészen,
 Fiad miénk lesz egészen.
/4
#6B65838D
 Kérünk, Uram, adj kegyelmet,
 És nyújts nekünk segedelmet,
 Hogy Jézusunkat szeressük
 És őtet végig kövessük.
/5
#8960B871
 Az ő megtöretett teste,
 Mellyel váltságunk kereste,
 Lelkünk máskor is táplálja,
 A bűnért meg ne utálja.
/6
#DF4F3E36
 Segítsd gyarló tehetségünk,
 Véle kötött szövetségünk
 Hogy szent és állandó légyen,
 Mígnem lelkünk hozzá mégyen.
/7
#269D6FBF
 Amely fogadást ma tettünk,
 Midőn asztalodról ettünk,
 Uram, segélj, hogy megálljuk,
 A bűnöket megutáljuk.

>449
/1
#5350FA74
 Uram, téged tisztellek
 Méltó hálaadással.
 Áldom neved, míg élek,
 Szép dícséret-mondással,
 És hirdetem mindennek
 Nagy voltát kegyelmednek.
/2
#FFAEAB14
 Mert te voltál Istenem
 Anyám méhétől fogva.
 Felkölt a bűn ellenem,
 Hogy tőrbe ejtsen fogva,
 De lelked engem megszánt,
 Rólam minden tőrt elhányt.
/3
#D05EEAB6
 Igazságod palástját
 Reám kiterjesztetted,
 Szentlelkednek áldását
 Bennem felgerjesztetted,
 Hogy megújulást végyek:
 Hozzád hasonló légyek.
/4
#6CD707D5
 Hát vajjon hogy lehetne,
 Ó Jézus, egy drága kincs,
 Hogy lelkem ne szeretne,
 Holott nálad főbb jó nincs.
 Csak te vagy a valóság,
 Út, élet és igazság.
/5
#64592426
 Szeretlek is bizonnyal,
 Lelkem hozzád kiterjed.
 Bár halnék meg azonnal,
 Ha szívem rád nem gerjed:
 Mert halál és ítélet
 Nálad nélkül az élet.
/6
#ACC25826
 Szívem vallástételét
 Jó lelkiismerettel
 Megtartom, s adom jelét
 Kegyes és szent élettel,
 Hogy életem folyása
 Légyen másnak lámpása.
/7
#A9344F7A
 Szent lábaidnál ülök,
 Hallgatok törvényedre,
 Bűnt útálok, kerülök,
 Hogy lehessek kedvedre,
 S megmutassam valóban:
 Örül szívem a jóban.
/8
#AED306DA
 Mert tudom: akik járnak
 A romlott testnek útján,
 Tőled hiába várnak
 Áldást haláluk után.
 Jaj lesz vége dolguknak,
 El kell veszni azoknak.
/9
#5F8E39FB
 De nékem reménységem
 Vagyon az én Uramban,
 Hiszem: lesz idvességem,
 Csak Járjak jó útamban:
 Ha követem világát,
 Meglátom szent országát.
/10
#230C012B
 De mivel, Uram, e test,
 Melyben lelkem sátoroz,
 Gyakran csügged s igen rest,
 A Sátán is jár s oroz:
 Kérlek, oktass Lelkeddel,
 Végy körül kegyelmeddel.

;Debrecen, 1778
>450
/1
#F4990445
 Drága dolog az Úr Istent dicsérni,
 Színe előtt kegyesen énekelni,
 Ékes dolog szent nevét magasztalni,
 A híveknek őelőtte szolgálni.
/2
#0DEF049A
 Ő gyógyít meg szomorodott szíveket,
 Vigasztalja az elalélt lelkeket,
 Ő enyhít meg sok keserűséget,
 Mind testi, mind lelki betegségeket.
/3
#56A53B6B
 Az őnéki csodálatos tanácsa,
 Hogy a szelídeket felmagasztalja,
 De a kevély gonoszokat utálja,
 És azokat a földig megalázza.
/4
#25B1CD6C
 Énekeljünk néki hálaadással,
 Vigasságos hangicsáló szerszámmal:
 Beszélgessünk e mi kegyes Urunkkal,
 Magasztalván őtet imádságunkkal.
/5
#B1923BEE
 Dicsérd azért, Jeruzsálem, Uradat!
 Keresztyénség, magasztald Megváltódat!
 Áldjad, Sion, mennyei királyodat!
 Híveknek serege, te szent Atyádat!

>451
/1
#BB41C25C
 Száma nincsen, Uram, jótéteményidnek,
 Vége, hossza nincsen kegyelmességednek;
 Azért magasztallak felette mindennek,
 Mert hiszen én lelkem te ígéretednek.
/2
#9C696154
 Kiáltásom, Uram, tőled el nem vetéd,
 Az én nyavalyámat csak te magad nézéd;
 Atyai voltodat rajtam éreztetéd,
 Irgalmasságodat mikor megjelentéd.
/3
#DA848335
 A pokol torkából te kiszabadítál,
 És az én lelkemben engem megnyugtattál,
 Te igéreteddel mikoron bíztattál,
 Fiad által ismét fiaddá fogadtál.
/4
#B1C7BE7C
 Kész mindenha lelkem néked énekelni,
 Mind e világ előtt rólad vallást tenni,
 Sok jótételidért néked hálát adni,
 Örökkön örökké tégedet dicsérni.

;Debrecen, 1774
>452
/1
#2E7084A5
 Jézus, édes emlékezet,
 Te adsz szívünknek örömet:
 Már hozzád vontad szívemet;
 Vezéreljed lépésimet.
/2
#97B21E5A
 Jézus, megtérők reménye,
 Hozzád kiáltóknak szíve,
 Téged keresőknek kincse,
 Megtalálóknak mindene!
/3
#2A312181
 Jézus, szívnek édessége,
 Élő kútja, fényessége,
 Minden kincsének szekrénye,
 Kívánságának jó vége:
/4
#D1ECFA8A
 Sem nyelv azt meg nem mondhatja,
 Sem betű fel nem írhatja,
 Csak ki próbálta, tudhatja,
 Hogy milyen a Jézus lángja.
/5
#D3456C41
 Te vagy kegyelem kútfeje,
 Igaz hazánk fényessége,
 Búbánatnak enyhítője,
 Boldogságnak megszerzője.
/6
#4988CC75
 Tégedet áldnak mennyégben,
 Magasztalnak dicsőségben;
 Jézus, vigasztalj éltünkben,
 Juttass Atyádnak kedvében.
/7
#8A0C8E36
 Téged kövessünk egy szívvel,
 Énekléssel, könyörgéssel,
 Hogy mi is a Szentlélekkel
 Lehessünk egyek Istennel.
/8
#04BC322A
 Jézus, mennyben légy örömünk,
 Ki voltál földön érdemünk;
 Benned minden dicsőségünk,
 Ó, mi Urunk és Istenünk!

>453
/1
#3FB99E3A
 Hozzád kiáltok, én Istenem,
 Mert érzem, mily gyarló vagyok,
 Ellenségim sokak s nagyok,
 Az ó-ember harcol ellenem.
/2
#19094C98
 Szerez lelkemnek sokféle bajt,
 Mivel erősebb nálamnál,
 Lakozik szintén házamnál;
 Ha meg nem szánsz, jaj, tőribe hajt.
/3
#8DC5068F
 Öldököld meg, Uram, én bennem
 Ez ó-embert, hogy meghaljon,
 Maga pártjára ne vonjon:
 Ne kelljen igáját felvennem.
/4
#1F1EFE4B
 Küldd el hozzám áldott lelkedet,
 Ki által bánja, gyűlölje
 Szivem a bűnt, sőt megölje,
 Mely megbántotta Felségedet.
/5
#062C6B5F
 Támaszd fel, Uram, valójában
 Az új embert én szívemben;
 Hadd nyugodjék kebelemben,
 S örüljek az Úr jóvoltában.
/6
#0A7602C0
 Az igazságnak tudománya
 Járásomban vezéreljen,
 Hogy lelkem-testem úgy éljen,
 Mint szent akaratod kívánja.
/7
#8EDB91FB
 Taníts meg te szent törvényedből,
 Felségedet mint tiszteljem,
 Máshoz magam mint viseljem,
 Hogy ki ne essem szerelmedből.
/8
#8C9117F3
 Kész lészek, Uram, ím fogadom:
 Szent törvényed szerint élek,
 Járok, kelek, fekszem vélek:
 Zálogul szívem néked adom.

;Debrecen, 1781
>454
/1
#69222F8B
 Ez a világ csak baj halma,
 Nincs itt senkinek nyugalma,
 Minden naphoz új küzdelmet,
 Vállainkra rak sok terhet.
/2
#F1650CC4
 Mint őz, szarvas a vizekre,
 Vágyom azért oly helyekre,
 Ahol lelkem nyugtot talál,
 Jézusommal egy úton jár.
/3
#E6B3B073
 Legyen bár tövissel rakva,
 Nyugodt szívvel járok rajta,
 Jézus terhe könnyű s édes,
 Igája gyönyörűséges.

;Bastiaans J.G. 1866
>455
/1
#E0AD5C76
 Testvérek, menjünk bátran, hamar leszáll az éj,
 E földi pusztaságban
 Megállni nagy veszély.
 Hát merítsünk erőt
 A menny felé sietni,
 Nem állva megpihenni
 A boldog cél előtt.
/2
#9F70FB41
 A keskeny útra térünk,
 Ne rettentsen meg az;
 Ki elhívott, vezérünk,
 Tudjuk, hogy hű s igaz.
 Mint egykor Ő tevé,
 Most véle s benne bízva,
 Arcát ki-ki fordítsa
 A szent város felé.
/3
#1AA5E7E7
 Óemberünk ha szenved,
 Az jó nekünk, tudom;
 Ki vérnek, testnek enged,
 Az nem jár jó úton.
 A láthatót ne bánd,
 Csak rázd le, mi kötözne:
 Hadd törjön éned össze,
 Menvén halálon át.
/4
#9DBFC62E
 Zarándok módra járva, Legyen kezünk üres;
 Csak terhet vesz magára, Ki pénzt, vagyont keres.
 Hadd gyűjtsön a világ, Mi tőle el se kérjük,
 Kevéssel is beérjük, Bennünket gond se bánt.
/5
#6AA16CB2
 Az út el van hagyatva,
 Borítja sok tövis;
 Nehéz emelni rajta
 Még a keresztet is.
 De egy út van csupán,
 Így hát előre bátran,
 Keresztül minden gáton,
 Hű Mesterünk után.
/6
#2E53A9CD
 Úgy járunk itt, lenézve,
 Mint ismeretlenek;
 Sokan nem vesznek észre,
 Hangunk se hallva meg.
 De aki ránk figyel,
 Víg énekünket hallja.
 Szent reménység sugallja,
 Mit ajkunk énekel.
/7
#125EFF84
 Ha botlanak a gyöngék,
 Segítsen az erős;
 Hordjuk, emeljük önként,
 Kin gyöngesége győz.
 Tartsunk jól össze hát,
 Tudjunk utolsók lenni,
 A bajt vállunkra venni
 E földi élten át.
/8
#6170704A
 Menjünk vígan sietve,
 Hisz utunk egyre fogy;
 Nap megy napot követve,
 S a test majd sírba rogy.
 Csak még egy kis tűrés!
 Ha Őt híven követjük:
 A láncot mind levetjük
 S vár ránk az égi rész.
/9
#FFEC3DEB
 Elmúlik nemsokára a földi vándorút,
 És az örök hazába, ki hű volt, mind bejut.
 Ott vár angyalsereg, Ott várnak mind a szentek,
 S az Atyánál pihentek, Megfáradt gyermekek.

>456
/1
#349B5034
 Hatalmas Isten, könyörgünk
 Szent Felségedhez kiáltunk.
 Mert igen megnyomorodtunk,
 Fejünket nincsen már kihez hajtanunk.
/2
#ECA55A3E
 Irgalmas Isten, hallgass meg,
 Bűnünket nékünk bocsásd meg,
 A te hitedet építsd meg
 És Szentlelkeddel, kérünk, vígasztalj meg!
/3
#7CFA4FFB
 Jertek énhozzám, bűnösök,
 Fiaim nékem legyetek,
 Mert én is Atyátok lészek,
 Ha bűnötökből énhozzám megtértek!
/4
#41376247
 Értsük meg hát ő mondását,
 Ismerjük meg bűneinket,
 Kövessük meg Istenünket,
 Így megmenthetjük ördögtől lelkünket.
/5
#38A4D683
 Hagyjuk el a nagy vakságot,
 És kérjünk világosságot,
 Mert ő gyűlöl hamisságot
 És csak szereti az egy igazságot.
/6
#12ACED0F
 Nincs igaz más, mint a Krisztus,
 Mert igazság ő beszéde;
 Maradjunk azért ezekben,
 Ne kételkedj Isten beszédében.
/7
#1B283A2D
 Dícsérjük az Atya Istent,
 Magasztaljuk Fiú Istent;
 Vélek a Szentlélek Istent,
 Három személyben csak egy bizony Istent!

;Walesi dallam
>457
/1
#05B5D4CC
 Ó, Jézus, árva csendben az ajtón kívül állsz,
 Bejönnél már, de némán kulcsfordulásra vársz.
 Mi mondjuk, hogy miénk vagy, te vagy a név, a jel:
 Ó, szégyen, hogy te légy az, akinek várni kell.
/2
#1822D3E5
 Ó, Jézus, most kopogtatsz, sebhelyes még a kéz;
 Könnymarta kedves arcod oly búsan intve néz.
 Ó, áldott, drága jóság, mely ennyit tűrve vár!
 Ó, bűnök szörnyű bűne, mely téged így kizár!
/3
#79BFCC62
 Ó, Jézus, szólsz, s a szívhez a szó szelíden ér:
 „Így bánsz velem? — teérted hullt testemből a vér!”
 Bús szégyennel behívunk, az ajtónk nyitva már.
 Jöjj, Jézus, jöjj, ne hagyj el, a szívünk várva vár.

;Hastings I., 1830
>458
/1
#8F3A0490
 Aki értem megnyíltál,
 Rejts el, ó, örök kőszál!
 Az a víz s a drága vér,
 Melyet ontál a bűnér',
 Gyógyír légyen lelkemnek,
 Bűntől s vádtól mentsen meg!
/2
#FE8BF355
 Törvényednek eleget
 Bűnös ember nem tehet;
 Buzgóságom égne bár,
 S folyna könnyem, mint az ár:
 Elégtételt az nem ad,
 Csak te válthatsz meg magad.
/3
#7A0E9199
 Jövök, semmit nem hozva,
 Keresztedbe fogózva,
 Meztelen, hogy felruházz,
 Árván, bízva, hogy megszánsz;
 Nem hagy a bűn pihenést:
 Mosd le, ó, mert megemészt!
/4
#2AB770CA
 Ha bevégzem életem,
 és lezárul már szemem,
 Ismeretlen bár az út,
 Hozzád lelkem mennybe jut:
 Aki értem megnyíltál,
 Rejts el, ó, örök kőszál!

;Roy Krisztina, 1861-1937
>459
/1
#409D7848
 Az Isten Bárányára
 Letészem bűnöm én,
 És lelkem béke várja
 Ott a kereszt tövén.
 A szívem mindenestül
 Az Úr elé viszem,
 Megtisztul minden szennytül
 A Jézus vériben,
 A Jézus vériben.
/2
#4E4EC0BA
 Megtörve és üresen
 Adom magam neki,
 Hogy újjá ő teremtsen,
 Az űrt ő töltse ki.
 Minden gondom, keservem
 Az Úrnak átadom,
 Ő hordja minden terhem,
 Eltörli bánatom,
 Eltörli bánatom.
/3
#6D575991
 Örök kőszálra állva
 A lelkem megpihen;
 Nyugszom Atyám házába
 ' Jézus kegyelmiben.
 Az ő nevét imádom
 Most mindenek felett;
 Jézus az én királyom,
 Imámra felelet,
 Imámra felelet.
/4
#EC8162D9
 Szeretnék lenni, mint ő,
 Alázatos, szelíd,
 Követni híven, mint ő,
 Atyám parancsait.
 Szeretnék lakni nála,
 Hol mennyei sereg
 Dicső harmóniába'
 Örök imát rebeg,
 Örök imát rebeg.

;Barnby J., 1838-1896
>460
/1
#F8B8808D
 Amint vagyok, sok bűn alatt,
 De hallva hívó hangodat,
 Ki értem áldozád magad:
 Fogadj el, Jézusom!
/2
#39BACC27
 Amint vagyok - nem várva, hogy
 Lelkemnek terhe, szennye fogy,
 Te, aki megtisztíthatod: -
 Fogadj el, Jézusom!
/3
#694E8773
 Amint vagyok - bár gyötrelem,
 S kétség rágódik lelkemen,
 Kívül harc, bennem félelem: -
 Fogadj el, Jézusom!
/4
#B3D8C967
 Amint vagyok - vak és szegény,
 Hogy kincset leljek benned én,
 S derüljön éjszakámra fény: -
 Fogadj el, Jézusom!
/5
#17877464
 Amint vagyok - nincs semmi gát,
 Kegyelmed mit ne törne át;
 Hadd bízza lelkem rád magát: -
 Fogadj el, Jézusom!
/6
#D3E61825
 Amint vagyok - hogy a te szent
 Szerelmed tudjam, mit jelent
 Már itt s majd egykor odafent: -
 Fogadj el, Jézusom!

;Howard S., 1710-1782
>461
/1
#3B8BFD05
 Bár bűn és kín gyötör,
 És nehéz bár szívem,
 A Sátán életemre tör:
 Kétségbe nem esem.
/2
#FAA19E6D
 Bár vétkem súlya nagy,
 Mégis hozzád jövök.
 A bűnnek gyűlölője vagy,
 De kegyelmed örök.
/3
#8A58F9A2
 Az én erőm kicsiny,
 S a bűn erős nagyon:
 Te tudsz s akarsz segíteni,
 Hát segíts bajomon!
/4
#86C0F7CC
 Az ég oly messze van,
 Még messzebb tőled én,
 De szent igédben írva van,
 Hogy irgalmad enyém.
/5
#36D551F0
 Nem félek senkitől,
 Hisz te vigyázsz reám,
 Már bánat és gond sem gyötör:
 Meghallgatod imám.
/6
#536F1416
 Jézusban bízva én
 szívemet átadom,
 Mert így, tudom: akármi ér,
 Atyám szeret nagyon.

;Silcher F., 1789-1860
>462
/1
#DADFF989
 Csak vezess, Uram, végig, és fogd kezem,
 Míg boldogan a célhoz elérkezem,
 Mert nélküled az én erőm oly kevés,
 De hol te jársz előttem, nincs rettegés.
/2
#66E83AA4
 Szent irgalmaddal szívemet födjed bé,
 Tedd örömben és bánatban csöndessé,
 Hogy hadd pihenjen lábadnál gyermeked,
 Ki szemlehunyva téged híven követ.
/3
#5418A375
 Ha gyarlóságom meg nem is érzené:
 A vak homályból te mutatsz ég felé;
 Csak vezess, Uram, végig, és fogd kezem,
 Míg boldogan a célhoz elérkezem.

;Iverson Daniel
>463
/1
#798A1A45
 Isten élő Lelke, jöjj, Áldva szállj le rám,
 Égi lángod járja át szívem és a szám!
 Oldj fel, küldj el, Tölts el tűzzel!
 Isten élő Lelke, jöjj, áldva szállj le rám!
/2
#C8955707
 Isten élő Lelke, jöjj, légy vezérem itt,
 Ó, segíts, hogy hagyjam el bűnök útjait!
 Oldj fel, küldj el, tölts el tűzzel!
 Isten élő Lelke, jöjj, légy vezérem itt!
/3
#A76EE281
 Isten élő Lelke, jöjj, hadd lehessek szent,
 S Jézusommal légyek egy már e földön lent!
 Oldj fel, küldj el, tölts el tűzzel!
 Isten élő Lelke, jöjj, hadd lehessek szent!
/4
#979759E0
 Isten élő Lelke, jöjj, győzedelmet adj,
 S majd a végső harcon át mennybe fölragadj!
 Oldj fel, küldj el, tölts el tűzzel!
 Isten élő Lelke, jöjj, győzedelmet adj!

;Finn népi dallam
>464
/1
#C957647E
 Jöjj, királyom, Jézusom!
 Szívem, íme, megnyitom.
 A gonosztól óvj te meg,
 Meg ne rontson engemet.
/2
#6DC600B3
 Véreddel, mely el-kifolyt,
 Mosd le rólam, ami folt;
 Élet útját megmutasd,
 Én meg nem találom azt.
/3
#978ABC6F
 Gyógyítsd meg sok nyavalyám,
 Enyhíts szívem bánatán;
 Kétség, gond ha gyötrenek,
 Biztasd nádszál hitemet.
/4
#A3F1A5CD
 Van hatalmad rá, tudom,
 Míveld, édes Jézusom:
 Hit, remény és szeretet
 Töltse be a szívemet.
/5
#0842F92B
 A keresztet te adod,
 Adj hozzá alázatot:
 Hordjam olyan csendesen,
 Mint egykor te, Mesterem.
/6
#21660BE9
 Majd ha véget ér a harc
 S megpihentetni akarsz:
 Megragadom jobbodat,
 S mennyországod béfogad.

;Nyberg Huugo
>465
/1
#D9D51249
 Szelíd szemed, Úr Jézus,
 Jól látja minden vétkemet,
 Személyemet ne vesse meg
 Szelíd szemed, Úr Jézus.
/2
#DD24F5A1
 Szelíd szemed, Úr Jézus,
 Tekintsen rám, ha roskadok,
 Adjon békét, bocsánatot
 Szelíd szemed, Úr Jézus.
/3
#58C3A6F0
 Szelíd szemed, Úr Jézus,
 Tudom, hogy vádat is emel;
 Vétkeztem én, ítéljen el
 Szelíd szemed, Úr Jézus.
/4
#04D7ABF6
 Szelíd szemed, Úr Jézus,
 Elítél bár, lásd, én megint
 Csak várom, hogy majd rám tekint
 Szelíd szemed, Úr Jézus.

;Mason Lowell, 1792-1872
>466
/1
#2B87643C
 Rád tekint már hitem,
 Megváltóm, Istenem,
 A Golgotán: Halld könyörgésemet,
 És vedd el vétkemet;
 Mostantól hadd legyek
 Tied csupán.
/2
#E6A4AC27
 Szívemet töltse be
 Kegyelmed ereje
 Buzgósággal!
 Meghaltál érettem;
 Add: szívem s életem
 Teérted éghessen
 Forró lánggal!
/3
#F1F03215
 Ha elfog utamon
 Félelem s fájdalom:
 Fogd kezemet!
 Derítsd fel éjemet,
 Szárítsd fel könnyemet:
 Tévelygésben ne hagyd
 Én lelkemet!
/4
#1A12C1E0
 Éltem ha fogyva fogy,
 És a halál ahogy
 Jön már felém:
 Megváltóm, ments te meg
 Kétségtől engemet,
 Nálad hogy üdvömet
 Meglássam én.

;Reinagle A.R., 1799-1877
>467
/1
#BCDAD5C0
 Mily jó, ha bűntől már szabad,
 Az Úr szolgája vagy;
 A bűn szolgája gyáva rab,
 A Krisztusé szabad.
/2
#7456B6BF
 A bűn sötétben tévelyeg
 És bajba dönt vakon;
 De Krisztus kézen fog s vezet
 Világos utakon.
/3
#EC519FDB
 A bűnben kín van s gyűlölet,
 Mi mást, más minket öl;
 Öröm köt egybe s szeretet
 Az Úr szívén belöl.
/4
#4B63BC69
 Már szolgád lettem, Jézusom,
 Ki értem áldozál;
 Más uram nincsen, jól tudom,
 Mert bűnből kihozál.
/5
#9479E0E5
 Légy áldott, Krisztusom, te nagy!
 Hadd adjam át szívem:
 Vedd szívesen, hogy hol te vagy,
 E szív is ott legyen.

;Shrubsole W., 1760-1806
>468
/1
#8B4FC5BC
 Zengd Jézus nevét, zengd, világ, őt, angyalok, áldjátok!
 Felékesítve homlokát, Királylyá Jézust,
 Jézust koronázzátok!
/2
#FA9C3B5F
 Ti vértanúi Istennek,
 Kik mennyben szolgáljátok
 A Bárányt, ki megöletett:
 Királlyá Jézust, Jézust koronázzátok!
/3
#FDD6100D
 Ti választottak, szent hívek,
 Mind akit ő megváltott,
 Szent irgalmát dicsérjétek:
 Királlyá Jézust, Jézust koronázzátok!
/4
#4494E3A6
 Ti bűnösök, mert ő hordott
 Tiértetek kínt s átkot,
 És szent vérével áldozott:
 Királlyá Jézust, Jézust koronázzátok!
/5
#AF02BBC3
 Ti népek, törzsek, kik bárhol
 Az ő szavát halljátok:
 Nagy jóvoltáért hálából
 Királlyá Jézust, Jézust koronázzátok!
/6
#7A7D7C13
 Mily boldogság lesz majd, ha fenn
 A Jézus előtt állok
 És mindörökké zenghetem:
 Királlyá Jézust, Jézust koronázzátok!

;Finlay K.G., 1882
>469
/1
#AB0DCFF9
 Jézus, nyájas és szelíd,
 Láss meg engemet,
 Hallgassad meg, hű Megváltóm, gyermekedet!
/2
#F64EC430
 Bűnöm láncát oldja fel
 Kegyelmed s a hit;
 Törjed össze balga szívem bálványait!
/3
#829FC00D
 Szabadságot adj nekem
 És tiszta szívet,
 Vonj magadhoz, Jézusom, hogy járjak veled!
/4
#9F6EB608
 Vezess engem utadon:
 Magad légy az út,
 Melyen lelkem a halálból életre jut.
/5
#72A0090A
 Jézus, nyájas és szelíd,
 Láss meg engemet:
 El ne engedd, hű Megváltóm, már kezemet!

;Damon's Psalmes, 1579
>470
/1
#792B3CFC
 Úr Jézus, nézz le rám,
 Jöjj, mosd le bűnömet,
 Sok földi szenvedély kötöz: jöjj, oldj fel engemet.
/2
#C5BFE133
 Úr Jézus, nézz le rám,
 Gond és bú látogat;
 Hű szolgád: ízleljem ígért, szent nyugodalmadat.
/3
#E97AA185
 Úr Jézus, nézz le rám,
 Ne tévedhessek el;
 A menny felé sötéten át te légy az úti jel.
/4
#B8B9A6C3
 Úr Jézus, nézz le rám,
 Ha nő a félelem,
 Ár zúg és ellenség szorít, légy, Megváltóm, velem!
/5
#EA470473
 Úr Jézus, nézz le rám,
 Ha elvonult az ár,
 Te szent derűd derítsen és az örök napsugár.

;Bliss Ph., 1838-1876
>471
/1
#39BAC106
 Fel, barátim, drága Jézus zászlaja alatt,
 Rajta, bátran! megsegít és győzedelmet ad.
 Bízzatok, mert Jézus eljön, ő a fővezér,
 Zengje ajkunk: hozzád esdünk győzedelemér'!
/2
#5DEFBDC1
 Lám, a Sátán serge talpon, szembetörni kész,
 A legbátrabb harcosoknak bátorsága vész.
 Bízzatok, mert Jézus eljön, ő a fővezér,
 Zengje ajkunk: hozzád esdünk győzedelemér'!
/3
#E46C315F
 Szóljon a kürt, fenn lobogjon győzedelmi jel,
 Így előre Jézusunkkal: néki győzni kell!
 Bízzatok, mert Jézus eljön, ő a fővezér,
 Zengje ajkunk: hozzád esdünk győzedelemér'!
/4
#0EA86987
 Harci zajban, küzdelemben oldalunkon áll,
 Benne higgyünk, ő segít meg szívünk harcinál.
 Bízzatok, mert Jézus eljön, ő a fővezér,
 Zengje ajkunk: hozzád esdünk győzedelemér'!

>472
/1
#67C10916
 Mennyit zengi a lelki békét,
 A szivek csendjét énekünk,
 Bár künn kín és ínség üzenget
 Sok hangos jajszóval nekünk.
 Az óra, testvér, gyorsan elfut
 S az ember útja, sorsa zord,
 Nincs rá idód, hogy álmodozva
 A lelked meddőn tékozold.
/2
#BDA14D58
 Ebből az éhség csal ki könnyet,
 Az ott a fagy miatt remeg,
 Házad táján sorvadva élnek
 Éhes, lerongyolt gyermekek.
 Az apjuk gyötrött mély szemében
 A meghasonlás láza él,
 Értük megmentő harcra kelni:
 Elhivatás, testvéri cél.
/3
#B8F7B018
 Lelkek éjét oszlatni fénnyel:
 Reád az Úr ezt bízta, lásd,
 S azt, hogy hozz Krisztus szent nevével
 A bűnösnek szabadulást.
 Az Isten vérét adta értünk;
 Láttasd, hogy szent példája hat,
 S hogy áldott útját járva hittel
 Te is adod egész magad.
/4
#9F54235C
 Adj éjt, napot, add át erődet,
 Add kincsedet, add szivedet,
 Rontó lelkek gáncsa nem árthat,
 Győztes csak egy: a szeretet.
 A Mester lelke járt előtted
 S kitűzi most eléd a célt,
 Munkára hát, ne késs, serénykedj
 Megváltódért, Királyodért.
/5
#B69B6D19
 Áldott Jézus, a földre jöttél
 S vállaltál értünk szörnyű kínt;
 Bárcsak tudnánk, Igéd követve,
 Mi is szenvedni, tűrni mind.
 Sok szolgád, férfi, nő, ma bízva
 Átadjuk szívünket neked;
 Hogy munkánk éljen és teremjen:
 Gyújtsd fel bennünk szerelmedet!

>473
/1
#738EA9C8
 Emeld fel szíved, füled nyisd meg
 Te kemény nyakú Izrael!
 Isten parancsolatát értsd meg,
 Törvényire figyelmezzél.
/2
#DD73CBF6
 ÉN VAGYOK ISTENED EGYEDŰL,
 Ki a szolgálatból téged
 Kihoztalak nagy ínségedből:
 Ne légyenek más istenid.
/3
#7E358E84
 NE CSINÁLJ SEMMI BÁLVÁNYKÉPET
 Akárminémű dologból;
 Ne imádjad, ne tiszteld őket,
 Mert nagy haragom megindúl.
/4
#0601165A
 URADNAK ISTENEDNEK NEVÉT SOHA NE VEGYED HIÁBA,
 Mert aki nem becsüli nevét,
 Megbünteti azt valóba'.
/5
#8F48C376
 HAT NAP MINDEN DOLGODAT TEGYED,
 NYUGODJÁL A HETEDIKEN,
 Mert hogy teremte mindeneket,
 E nap megnyugovék Isten.
/6
#0F05CF78
 A TE ATYÁDAT ÉS ANYÁDAT
 TISZTELJED szeressed híven,
 Hogy hosszú életet, sok jókat
 Adjon tenéked az Isten.
/7
#9F0920A6
 Embert NE ÖLJ, ne tégy vérontást,
 PARÁZNASÁGBA NE TÉVEDJ,
 NE OROZZ, gyűlöljed a lopást,
 HAMIS BIZONYSÁGOT NE TÉGY.
/8
#244BE755
 FELEBARÁTODNAK Ő HÁZÁT
 NE KÍVÁND, se házastársát,
 Szolgáját, szolgálóleányát,
 Se barmát, semmi marháját.
/9
#F850E502
 Uram, hatalmas a te Igéd,
 Még zengését is rettegjük,
 Irgalmadból velünk azt tégyed,
 Hogy akaratodat tégyük.

;Kolozsvár, 1837
>474
/1
#97C3D968
 Istennel járni, lakozni,
 Szent élettel illatozni,
 Igaz hitben nem habozni:
 Jézus Krisztus, taníts,
 Taníts imádkozni!

;Tinódi Lantos Sebestyén, 1546
>475
/1
#24148DF5
 „Imádkozzatok és buzgón kérjetek!„
 Bűnös voltunkért, Uram, ó, ne vess meg!
 Tiszta szívet és Szentlelket adj nékünk,
 Hallgass meg Fiad nevébe', ha kérünk.
/2
#F165DF2B
 „Keressetek buzgón és megtaláltok!” –
 Téged keresünk, Uram: hogy bűn s átok
 Erőt ne vegyen mirajtunk, légy nékünk
 Égi utunk, igazságunk, életünk!
/3
#ACAE707C
 „Zörgessetek buzgón Isten ajtaján!” –
 Elfáradtunk, Uram, e világ zaján;
 Ó, nyisd meg az égi béke szép honát,
 Add, hogy zenghessünk örök halleluját!

>476
/1
#7B517E0C
 Siess nagy Úr Isten én segítségemre.
 És légy figyelmetes az én beszédemre.
 Emlékezzél reá szent ígéretedre.
 Hajtsad füleidet én könyörgésemre.
/2
#A57C4B07
 Hitből, tiszta szívből azért arra kérlek:
 Engedd, szent igédből hogy ismerhesselek,
 Teljes életemben hogy dicsérhesselek;
 Minden dolgaimban téged nézhesselek.
/3
#F63AADE8
 Az én életemet tartsd meg szent nevedért,
 A te szent Fiadnak kedves szerelméért;
 Nagy hálákat adok tenéked ezekért,
 Ha meghallgatsz Uram, engem jóvoltodért.
/4
#74F30DE5
 Nem kérem a végre hosszú életemet,
 Hogy itt e világon hízlaljam testemet,
 Sem hogy öregbítsem híremet-nevemet,
 De hogy dícsérhessem te Istenségedet.
/5
#07BE4C13
 Világi életem csak árnyék énnékem,
 Mint a szép folyóvíz, elmúlik éntőlem,
 De a te országod örök lakóhelyem,
 Melybe Jézus Krisztus által helyheztetem.
/6
#7B32E8DD
 Szent Fiad vállára rakád bűneimet,
 Ördög rabságából megváltál engemet,
 Érette fiaddá fogadál engemet,
 Örökössé tevél, mint édes gyermeket.
/7
#C5C00D9B
 Bízvást merek azért néked könyörögni,
 Az én ajakimmal áldozatot tenni;
 Mint kegyes Atyámmal, te veled szólani,
 Minden szükségemet szépen megbeszélni.
/8
#E5BBE96A
 A te népeidet, Uram, immár szánd meg,
 Súlyos ostorodat tőlünk immár vond meg,
 Te nagy haragodat, szent Atyánk, enyhítsd meg:
 Gyülekezetedet immáron építsd meg!
/9
#753DAC60
 Rútságát bűnünknek Úr Isten, ne nézzed,
 Hanem szent Fiadnak halálát tekintsed,
 És áldozatjának szent érdemét nézzed:
 Szegény fiaidat immár megsegéljed.
/10
#9896B17E
 A te jóvoltodért mindörökké áldunk,
 Irgalmasságodért is felmagasztalunk.
 Háládatlanságot távoztass el tőlünk,
 Új és tiszta szívet teremts már mi bennünk!
/11
#760A0AB5
 Tied legyen, Isten, mindenben dicsőség
 A te szent Fiaddal, és minden tisztesség,
 Szentlélek Istennel egyenlő felségben,
 A Szentháromságban mind örökké, Amen.

>477
/1
#217E5CBF
 Halld meg, Uram, esedezésünket,
 Irgalmadból szánd meg mi ügyünket,
 Mert érezzük számtalan bűnünket,
 Melyekkel bémocskoltuk lelkünket.
/2
#2BACBE8F
 Törvényedet minden nap hallottuk,
 Szántszándékkal mégis megrontottuk,
 Jó Atyánkat így haragítottuk,
 Ő szent lelkét megszomoritottuk.
/3
#4E4E2F37
 Tisztünkben is hányszor restelkedtünk,
 Gyarlóságból mennyi hibát tettünk,
 Szeretettel akikhez köttettünk,
 Azok ellen mely sokat vétettünk.
/4
#CA6B2196
 Méltók vagyunk, hogy kárhozatban hagyj,
 De te, Uram, jó és Irgalmas vagy,
 Minden megtérőkhöz kegyelmed nagy;
 Kérünk, abba minket is béfogadj.
/5
#142DF583
 Szent Fiadnak kiontatott vére
 Lett lelkünknek váltsága és bére,
 Tekints, kérünk, az ö érdemére.
 Bűneinkért ne vonj ítéletre.
/6
#42FFEA60
 Megígérted, kegyelmes Istenünk,
 Hogy e felől jó hitben kell lennünk,
 Ha azoknak mi is megengedünk.
 Kik valaha vétettek ellenünk.
/7
#42B83590
 Segíts, Uram, hogy ezt mívelhessük,
 Feleinket híven szerethessük,
 Naponként a bűnt levetkezhessük.
 Az örök életet elvehessük.

;Strasbourg, 1539
>478
/1
#6CEA52FC
 Ó, irgalmas Isten,
 Én könyörgésemben
 Füledet hozzám hajtsad;
 Ó, igen jó Isten,
 Minden szükségemben
 Áldásod szaporítsad.
/2
#2DE68A72
 Ó, hatalmas Isten,
 Keserűségemben
 Szívemet vidámítsad;
 Ó, nagy és szent Isten,
 Minden félelmemben
 Elmémet bátorítsad.
/3
#556DF9BA
 Ó, örök Úr Isten,
 A veszedelmekben
 Segedelmedet nyújtsad;
 Ó, igaz Úr Isten,
 Kételkedésemben
 Hitemet gyámolítsad.
/4
#E10AF5FE
 Ó, áldott Úr Isten:
 Rossz testiségemben
 Lelkem hozzád hódítsad;
 Ó, erős Úr Isten,
 A világ fényében
 Szemem világosítsad.
/5
#CB53C777
 Ó, teremtő Isten,
 A kísértetekben
 A Sátánt elfordítsad;
 Ó, megváltó Isten,
 Sok vétkezésimben
 Irgalmad bizonyítsad.
/6
#73460CFB
 Ó, szentelő Isten,
 Erőtlenségimben
 Kegyelmedet újítsad;
 Ó, kegyelmes Isten,
 Éltemnek végében
 Lelkemet boldogítsad.

;Ramsey M.
>479
/1
#6F204384
 Hinni taníts, Uram, kérni taníts!
 Gyermeki, nagy hitet kérni taníts!
 Indítsd fel szívemet,
 Buzduljon fel, neked
 Gyűjteni lelkeket! Kérni taníts!
/2
#E89CD397
 Hinni taníts, Uram, kérni taníts!
 Lélekből, lelkesen kérni taníts!
 Üdvözítőm te vagy,
 Észt, erőt, szívet adj.
 Lelkeddel el ne hagyj!
 Kérni taníts!
/3
#6D266432
 Hinni taníts, Uram, kérni taníts!
 Gyorsan elszáll a perc: kérni taníts!
 Lásd gyengeségemet,
 Erősíts engemet,
 Míg diadalt nyerek: Kérni taníts!
/4
#160A2689
 Hinni taníts, Uram, kérni taníts!
 Jézus, te visszajössz: várni taníts!
 Majd ha kegyelmesen
 Nézed az életem:
 Állhassak csendesen. Hinni taníts!

;Genf, 1562
>480
/1
#75B0C1C6
 Ó, könyörgést meghallgató
 Édes Atya, mindenható,
 Ha lelkem hozzád emelkedik,
 És a buzgóság szárnyain,
 Ajakim óhajtásain
 Elődbe érvén, reménykedik:
 Érzem, hogy az örök élet
 Már e földön az enyém lett.
/2
#8AE4DB66
 Ha örömmel gerjedezem
 És rebegni igyekezem,
 Tetőled mennyi áldást vészek!
 Feltekintvén rád, Atyámra,
 Könny csordul már az orcámra;
 Azáltal olyan újjá lészek,
 Mint a plánta, mikor arra
 Harmatcseppet szülsz hajnalra.
/3
#7DAE8B14
 Ha szívemet bánat járja,
 Szemem keserűség árja:
 Előtted való zokogásom;
 Titkon ajtómat behajtva
 És magánosan sóhajtva
 Akkor is édes újulásom,
 Mert minden könnyűvel, jajjal
 Könnyebbül sorsom egy bajjal.
/4
#7C6CD675
 Ha templomban megjelenek,
 Ahol összesereglenek,
 Felséges neved imádói:
 Velük együtt fohászkodva
 Úgy tetszik, mintha vigadva
 Ott volnék, hol a menny lakói
 A te királyi székednek
 Előtte letelepednek.
/5
#AABBB5BC
 Te, ki a szív mozdulásit,
 Mint a vizeknek folyásit,
 Szabadon hajtod ide s tova,
 Neveld ezt az érzést bennem,
 És taníts jól könyörgenem;
 Majd egyszer pedig, Mennyek ura,
 Vigy be az egek egébe,
 Imádásod szent helyébe.

>481
/1
#D85B9ED8
 Ó, felséges Úr, egek királya!
 Nagyságod a menny és föld csudálja;
 Nevedre minden térdek meghajolnak,
 Széked előtt angyalok udvarolnak.
 Mégis hozzánk úgy leereszkedtél,
 Hogy Fiad által Atyánkká lettél,
 És mikor szükségünkben esedezünk,
 Gyönyörködöl, ha Atyánknak nevezünk.
/2
#2D7500E7
 Nagyobb kegyelem ennél mi lehet,
 Hogy a bűnös tehozzád úgy mehet,
 Mint Atyjához, aki könnyen könyörül
 És szerelmes fia kárán nem örül.
 A terhes kereszt alatt levőknek,
 Ínség, betegség alatt nyögőknek
 Az az egy gondolat ad könnyebbséget,
 Hogy Atyjuk vagy s látsz bajukban jó véget.
/3
#62E4FE17
 Ó, édes Atyánk, nyújtsd kegyelmedet,
 Hogy mi is úgy szeressünk tégedet,
 Mint a fiak, kik érzik, hogy illetlen,
 Ha szívük az atyjukhoz tiszteletlen.
 Emeld fel lelkünk gyakran mennyekbe,
 Hogy ne merüljünk a földiekbe,
 Hanem addig is, míg majd haza mégyünk,
 Hozzád vágyódásunkkal otthon légyünk.

>482
/1
#6DE1BFC7
 Mi kegyes Atyánk, kit vallunk hitünkben:
 Légy segítségül íly nagy szükségünkben,
 Építs meg minket mi természetünkben,
 Ki vagy mennyekben.
/2
#ABE9485B
 A te szent neved szenteltessék köztünk,
 Ne uralkodjék hamisság mibennünk;
 A te szent Igéd hirdettessék köztünk,
 Hogy el ne vesszünk.
/3
#83DD5C01
 Jöjjön miközénk a te szent országod,
 Csak te bírj minket, bűnös fiaidat;
 Ismerhessük meg a te szent Fiadat,
 Krisztus Urunkat.
/4
#B00D6998
 Ne légyen bennünk ördögnek országa,
 Ne bírjon minket a bűnnek soksága;
 Szálljon mi közénk hitnek igaz ága,
 Megtisztulása.
/5
#714A81EF
 Természetünket ne hagyjad követnünk;
 Akaratodat épltsed mi bennünk,
 Hogy mindenekben tenéked engedjünk:
 Ne hagyj elesnünkl
/6
#5CC2D56F
 Mi kenyerünket mindennap megadjad,
 Dolgunkban áldást, csendes nyugodalmat;
 És mindenekben testünket tápláljad
 És oltalmazzad.
/7
#1C7DFFBA
 Megbocsásd nékünk gonosz bűneinket,
 Álnokságinkat, hitetlenséginket:
 Végy ki közülünk minden irígységet,
 És gyűlölséget.
/8
#EEB8B53E
 És ne vígy minket gonosz kísértetbe',
 Ne hagyj elvesznünk a mi bűneinkbe';
 Oltalmazz minket, ne essünk kétségbe,
 Ördög kezébe.
/9
#11C6278C
 Szabadítsál meg minket a gonosztól,
 Bűntől, ördögtől, minden álnokságtól,
 Ellenséginktől, hirtelen haláltól,
 És kárhozattól.
/10
#257A713B
 Dicsőség néked mennyben, örök Isten!
 Ki mihozzánk vagy engedelmességben:
 Hallgass meg minket e könyörgésünkben,
 Áldj meg hitünkben!

;Lipcse, 1539
>483
/1
#096FEC4A
 Mennybéli felséges Isten,
 Kinek dicsőséged ott fenn
 Boldog lelkek seregitül
 Láttatik véghetetlenül:
 E teljes világ általad
 Teremtetett, áll és marad.
/2
#22D35A6D
 Te noha ily felséges vagy,
 Erőd, méltóságod ily nagy,
 Mégis minket, kik föld pora
 S hitvány férgek vagyunk, arra
 Méltóztatsz, hogy fiaidnak
 Hívassunk, s te MI ATYÁNKnak.
/3
#3686739B
 Fiúi bizodalomban
 Azért te elődbe mostan
 Járulunk imádságunkkal,
 Reggeli áldozatunkkal;
 MI ATYÁNK, tiszteletünket
 Vedd bé, halld meg kérésünket.
/4
#E9E8709C
 SZENTELTESSÉK MEG TE NEVED,
 Azaz: mivelünk azt tegyed,
 Hogy igazán megismerjünk
 Téged, féljünk és tiszteljünk,
 Szemlélvén nagy bölcs munkáid
 S minden tökéletességid.
/5
#2A9A7457
 Gondolatunk és beszédünk,
 Sőt a mi egész életünk
 Úgy folyjon s úgy tündököljék,
 Hogy mindenek megismerjék.
 Hogy te vagy, szent Isten,
 Atyánk, Fiaidnak te vagy példánk.
/6
#3118657D
 JÖJJÖN EL A TE ORSZÁGOD,
 Töltse bé uralkodásod,
 Ó, mi királyunk, e földet;
 Szaporítsad seregedet,
 Drága Igédnek kész szállást,
 Adj mindenütt szabad folyást.
/7
#76CA477A
 Ha egyházad ellenségi
 Igyekeznek azt rontani:
 Te velünk egy táborba szállj,
 És előnkbe vezérül állj;
 Szégyenítsd ellenséginket,
 Tartsd meg, bírj, vezérelj minket.
/8
#CA8207F5
 LEGYEN A TE AKARATOD;
 Ami jó s rendes, te tudod;
 Azért mi akaratunkat
 Tetszésed után hordozd,
 Szeressük, amit szeretsz,
 De gyűlöljük, amit gyűlölsz te.
/9
#30C251C6
 Engedjünk néked mindenben,
 MINT dicső seregid MENNYBEN:
 Ha velünk keményen bánsz is,
 Szenvedjük békével azt is;
 Tiéd mind testünk, mind lelkünk,
 Teremtőnk, szabad vagy velünk.
/10
#E3DF3877
 ADD MEG NEKÜNK KENYERÜNKET
 NAPONKÉNT, eledelünket:
 Viselj gondot életünkről,
 E mi halandó testünkről;
 Szolgáltass jó egészséget,
 Termő időt, békességet.
/11
#7432F58F
 Szállíts áldást munkáinkra,
 Minden marhánk s jószágunkra;
 Javaiddal pedig nékünk
 Adj mértékletesen élnünk,
 Rád nézve háládatosan,
 Mások iránt irgalmasan.
/12
#EDC20F13
 BOCSÁSSAD MEG BŰNEINKET:
 Mind ama nagy veszettséget,
 Amely első szüléinktől
 Szállott reánk örökségül,
 Mind amely rossz gyümölcsöket
 Bosszúdra e rossz fa termett.
/13
#B2DFC2AA
 Nézvén szent Fiad vérére,
 Ne vonj minket ítéletre,
 És vétkeinkért meg ne feddj,
 Sőt mindenekben megengedj,
 AMINT MI IS MEGENGEDÜNK,
 HA KIK VÉTETTEK ELLENÜNK.
/14
#9561A5E0
 ÉS NE VÍGY A KÍSÉRTETBE,
 Mely rajtunk erőt vehetne!
 Jól tudod, mily gyarlók vagyunk,
 Könnyen tántorodik lábunk;
 SZABADÍTS AMA GONOSZTÓL,
 Ki lest hány nekünk akárhol.
/15
#E295DCEC
 Hogyha pedig dicsőséges
 Néked s nekünk épületes
 A kísértőt ránk bocsátni,
 Hitünket próbára vonni:
 Add nekünk a győzedelmet,
 Ezzel koronázd hitünket.
/16
#31668A76
 Hogy tőled kérjük ezeket,
 Ily okok indítnak minket:
 MERT TIED AZ ORSZÁG: néped
 Vagyunk mi és örökséged,
 Jó királyunkhoz szükségben
 Hogyne folyamnánk merészen?
/17
#02D1F691
 Velünk jól tenni akarván,
 Arra tehetséged is van,
 TIED A teljes HATALOM,
 Égi s földi birodalom;
 Jól tenni méltó is hozzád,
 DICSŐSÉG tér ebből reád.
/18
#579AA770
 Aki kérésében haboz,
 Az tőled üres választ hoz,
 De mi, Krisztus érdeméből
 Nyugodt szívvel lévén erről,
 Hogy megadsz, valamit kértünk,
 Kérésünkhöz ÁMEN-t kötünk.

;Lipcse, 1539
>484
/1
#E2529B09
 Mi Atyánk, ó, kegyes Isten,
 Ki vagy a magas mennyekben:
 Szenteltessék neved szívvel;
 A te országod jöjjön el;
 Te akaratod meglégyen
 Ez földön, miképpen mennyben.
/2
#2D7CF134
 Mindennapi kenyerünket
 Add meg, bocsásd meg vétkünket,
 Amint mi is megbocsátunk
 Azoknak, kiktől bántattunk;
 És ne vigy a kísértetbe,
 De szabadíts gonosz ellen.
/3
#C1987D01
 Mert tied, Uram, az ország,
 Tied minden hatalmasság,
 Megadhatsz azért minékünk
 Mindent, amit tőled kérünk;
 Tied a dicsőség, Ámen.
 Most és örökké úgy légyen.

;Kolozsvár, 1744
>485
/1
#09B7E267
 Jézus Krisztus, szép fényes hajnal,
 Ki feltámadsz új világgal,
 És megáldasz minden jókkal:
/2
#4C4E7640
 Te vagy nékünk egy reménységünk,
 Isten előtt közbenjárónk,
 Szép koronánk, ékességünk.
/3
#FD1C4467
 Világosítsd a mi szívünket:
 Ismerhessünk meg tégedet;
 Tanulhassuk szent Igédet.
/4
#1E92BBFD
 Az ördögnek csalárdságától,
 Lelki-testi nyavalyától,
 Őrizz hamis tudománytól.
/5
#3D2DF7B2
 Az Atyával és Szentlélekkel,
 Hogy tégedet tiszta szívvel
 Mindörökké áldjunk! Ámen.

;Debrecen, 1774
>486
/1
#4C50A1E8
 Szívem megalázván, tehozzád megyek,
 Elődbe, Istenem, hál'adást viszek
 És szent Fiad által néked könyörgök.
/2
#D0EEC7F4
 Áldott légy, én Uram, hogy megtartottál,
 Bút és kárt ez éjjel rám nem bocsátál,
 Angyali sereggel oltalmam voltál.
/3
#0A1F1E73
 E reggeli időt megadtad érnem,
 Melyben egészséges elmém és testem;
 Kérlek, minden jóra vezérelj engem.
/4
#DE0461C9
 Mai nap ezt velem cselekedd, Uram:
 Mindenféle bűntől magam óvhassam,
 Hitem jó gyümölcsét hogy el ne rontsam.
/5
#0055030A
 Ételem, italom mérsékelt legyen,
 Tobzódás, részegség ne nehezítsen,
 Rossz gondolat bennem erőt ne vegyen.
/6
#5A7D9578
 Lágyítsd meg énbennem az indulatot,
 Meg ne háborítsak soha másokat,
 És haragtartásra ne adjak okot.
/7
#ED1F7836
 Jól tudod, Istenem, mily gyarló vagyok,
 Különb-különb bűnre mily könnyen hajlok!
 Adjad: megtérhessek, mikor bűnt vallok!
/8
#9616331C
 Igaz ítéleted ne ostorozzon,
 Bűnből a Krisztusért ingyen oldozzon,
 Érdemem szerint rám átkot ne hozzon.
/9
#90B6837F
 Magamat egészen neked szentelem:
 Kegyelmes oltalmad legyen mellettem,
 Szentlelked és Igéd legyen vezérem.
/10
#B1F1B243
 Külső károktól is ments meg engemet,
 Ne eméssze bánat és bú szívemet;
 Tartsd jó egészségben gyarló testemet!
/11
#01C5190A
 Istenem, tenéked legyen dicsőség,
 A Szentháromságban ki vagy egy Felség:
 Csak tégedet illet minden tisztesség!

;Német dallamból, Kolozsvár, 1744
>487
/1
#69A6AF23
 Magasztallak én téged,
 Isten, egeknek királyát,
 Hogy tőlem meszsze űzted
 A sötét éjnek homályát.
 Nem küldél rám betegséget,
 Sem egyéb ínséget,
 Épségben megtartottál,
 E napra engem juttattál.
/2
#FD376944
 Szívből könyörgök néked, Kegyes teremtő Istenem:
 E napot is engedd meg Békességgel véghezvinnem,
 Akaratodat tanulnom, Útaidban járnom;
 Oltalmad béfedezzen, Kedvem kedved szerint légyen.
/3
#CDB1AD1F
 Igaz utadra taníts,
 Hogy veled együtt járhassak,
 És tőled el ne taszíts,
 Hogy kísértetbe ne jussak.
 Jóvoltodból tarts meg engem,
 Én édes Istenem,
 Hogy a bűn csalárdságát,
 Észre vegyem undokságát.
/4
#246F76E0
 Az igaz hitnek tüzét
 Bennem Krisztusért élesszed,
 Gyarlóságomnak vétkét
 Soha szememre ne vessed;
 Fogadásodat tekintsd meg,
 Szent Fiadért tarts meg,
 Ki értem eleget tett,
 Törvény átkától megmentett.
/5
#A001C1C1
 Reménységgel ruházz fel,
 Ördög tőrébe ne essem,
 Szívem hozzád gerjeszd fel,
 Ne csak hasznomat keressem.
 Atyafi szent szeretetet,
 Adj jámbor életet;
 Szeress, mint sajátodat,
 Kövessem akaratodat.
/6
#227B143F
 Szent Igédet vallanom
 Adjad tisztán, homály nélkül
 És szolgádnak mondatnom
 Minden képmutatás nélkül;
 Kinccsel semmit sem gondolok,
 Más jutalmat várok:
 A szentek seregében,
 Tarts meg gyülekezetében.
/7
#CFEA836E
 E napot is engedd meg,
 Uram, békével megfutnom,
 Hivatalom hasznával
 Magam s cselédim táplálnom,
 Hogy szent nevedet dicsérjem,
 Oltalmadat nyerjem:
 Testemnek megmaradást,
 Lelkemnek adj boldog szállást.
/8
#17F98D62
 Uram, egyedül vagy jó.
 Te vagy út, élet, igazság,
 De én életem gyarló.
 Gondolatom csak hamisság.
 Táplálj Krisztus szent testével,
 Itass szent vérével,
 Hogy örökké nevednek,
 Énekeljek Felségednek.
/9
#3C74FA0C
 Hála légyen Atyának, Fiúnak és Szentléleknek,
 Kitől minden jók vagynak: Ne engedj azért ördögnek
 Engem, igaz megváltottad, Gyenge juhocskádat:
 Lelkem a boldogságba, Vigyed a paradicsomba.

;Dykes I.B., 1823-1876
>488
/1
#95464EC5
 Szent vagy, szent vagy, szent vagy,
 Mindenható Isten,
 Énekünk jó reggel száll hozzád szívesen.
 Szent vagy, szent vagy, szent vagy,
 Végtelen kegyelem,
 Három személyben áldott egy Isten!
/2
#6224E4A1
 Szent vagy, szent vagy, szent vagy,
 Kit a szentek áldnak,
 Koronájukat letészik teelőtted;
 Angyali seregek térdelve imádnak,
 Ki voltál, vagy s nem érsz soha véget.
/3
#9DDE55BA
 Szent vagy, szent vagy, szent vagy,
 Földi köd bár elfed,
 És bűnös szem nem látja dicsőségedet,
 Csak te vagy szent, Isten,
 és senki kívüled:
 Teljes hatalmú szentség, szeretet!
/4
#D4462553
 Szent vagy, szent vagy, szent vagy,
 Nagy és erős Isten;
 Minden műved dicsér
 az ég-, föld- s tengeren!
 Szent vagy, szent vagy, szent vagy,
 áldott, véghetetlen:
 Három személyben egy áldott Isten!

;Ahle J. R., 1662
>489
/1
#A7C5CCAA
 Örök élet reggele,
 Fény a véghetetlen fényből,
 Egy sugárt küldj ránk te le,
 Kik új napra ébredénk föl;
 Fényed lelkünk éjjelét
 Űzze szét.
/2
#526EE6A7
 Jóságodnak harmata
 Gyarló életünkre hulljon,
 Szívünk, mely kiszárada,
 Vigaszodtól felviduljon,
 S híveid közt légy jelen
 Szüntelen.
/3
#C661D09B
 Bűn ruháját vessük el
 A szövetség vére által,
 Vétkeink fedezzük el
 Tőled nyert fehér ruhánkkal,
 Hogy hitünk legyőzze majd
 Mind a bajt.
/4
#66067669
 S majd vezess az égbe föl,
 Irgalomnak napvilága,
 Könnyek gyászos völgyiből
 Üdvösségnek szép honába,
 Hol az üdv és béke majd
 Egyre tart.

>490
/1
#4576AC57
 Igaz Biró, nagy Úr Isten,
 Ki állatsz mindent jó rendben,
 Reggelt, délt, estvét rendelted
 S híveidnek megszentelted:
/2
#855681D6
 Krisztus, hívek megtartója,
 Egeknek igazgatója,
 Mindeneknek alkotója,
 Te vagy bölcsen formálója:
/3
#5C35321F
 Szentlélek, egy igaz Isten,
 Ki lakol minden hívekben,
 Tartasz mindent igaz rendben:
 Tőled függ minden keresztyén.
/4
#9E69C5DB
 Szentháromság, egy Istenség,
 Ki vagy mindenre elégség:
 Csak tiéd minden tisztesség,
 Néked adassék dicsőség!
/5
#A71F4D9A
 Te teremtettél bennünket:
 Bírd hát jóra életünket;
 Ne nézd gyarló esetünket:
 Segéld erőtlenségünket.
/6
#AA59F4BD
 Hogy mi csak tiéid légyünk,
 Téged szüntelen dícsérjünk:
 Légyen szeretet mi bennünk,
 Egymás közt egyesek légyünk.
/7
#32CF7A6B
 Oltsd meg patvarinknak tüzét,
 Vedd el bűneinknek terhét:
 Adj minékünk csendességet,
 Lelki-testi békességet.
/8
#6AD741BC
 E világon úgy élhessünk,
 Hogy bűnünkben el ne vesszünk,
 Hanem mennyekbe mehessünk,
 S véled örökké élhessünk.
/9
#644E4E4B
 Dicsőség légyen tenéked,
 Menny, föld magasztaljon téged,
 Mert mindenekre kegyelmed,
 Mint a bő árvíz, kiterjed.

>491
/1
#896D1A3F
 A nap immár felvirradván,
 Kérjük az Istent mindnyájan:
 Hogy ezen az egész napon
 Bűntől minket oltalmazzon.
/2
#B4B6B1E3
 Nyelvünket megtartóztassa
 És a bűnre ne bocsássa,
 Szemeinket bétakarja,
 Gonoszra nézni ne hagyja.
/3
#E06FD93E
 Szíveink tiszták legyenek,
 Gonosz bűnt be ne vegyenek;
 Kevélységet gyűlöljenek,
 Dobzódást ne szeressenek.
/4
#545CB25C
 Hogy, mikor a nap elnyugszik
 És az éj béalkonyodik,
 Ml egészen tiszták légyünk
 És őtet vígan dícsérjük.
/5
#61B3DE2D
 Dícséret légyen Atyánknak,
 Es egyetlen-egy Fiának,
 És a mi Vígasztalónknak,
 Kitől minden jók származnak.

>492
/1
#ECBDBB04
 Felséges Atya Úr Isten,
 Ez napnak ő kezdetiben
 Könyörgünk: tarts meg hitünkben
 És az igaz szeretetben.
/2
#4DE09C07
 Bűneinket megbocsássad,
 És szívünket vígasztaljad,
 Vallásunkat bátorítsad,
 És Szentlelked nékünk, adjad:
/3
#0F54D65C
 Hogy ne ártson az ellenség,
 Távol légyen minden kétség,
 Testi-lelki fertelmesség,
 És a gonosz eretnekség.
/4
#EED67AA4
 Adjad meg nékünk éltünkben
 Egymást szeretnénk szívünkben;
 Élhessünk nagy békességben,
 És hitnek jó gyümölcsében.
/5
#D34F95BD
 Lelkünket és jószágunkat,
 Néked ajánljuk magunkat:
 Vezéreld a mi útunkat,
 És viseld jóra gondunkat.
/6
#FA1720DD
 Engedd ezt nékünk Krisztusért,
 Te egyetlenegy Fiadért,
 És érettünk Szószólónkért,
 Egy elég áldozatunkért.
/7
#613A3523
 Hálákat adunk, Úr Isten,
 Te szent Fiadnak akképen,
 Szentlélekkel egyetemben,
 Mostan és minden időben.

>493
/1
#0DB0C5F0
 Dícsérlek, Uram, téged,
 Hogy ez elmúlt éjjel,
 Életem megőrizted
 Hatalmas kezeddel;
 Setétségnek tőriből,
 Mely környülvett vala,
 Íme, engemet kivől,
 Ó mindennek Ura!
/2
#8665A568
 Néked adok hálákat,
 Én kegyes Istenem,
 És kérlek, mint Atyámat:
 Ma is tarts meg engem,
 Hogy szolgállhassak néked
 Akaratod szerint:
 Éltemet vezéreljed,
 Légy velem óránkint!
/3
#F75DF0E3
 Hogy el ne tévelyedjem
 A te ösvényidtől,
 Ne ártson ellenségem,
 Őrizkedjem bűntől:
 Őrizz meg jóvoltodból,
 Uram, téged kérlek,
 Ördög álnokságától
 Hogy semmit ne féljek.
/4
#068BDF9A
 Adjad, hogy erős hittel
 Bízzam szent Fiadban;
 Bűneimet töröld el.
 Tarts meg irgalmadban.
 Hiszem, hogy ezt megadod,
 Amint megigérted;
 Bűnünket megbocsátod,
 Ha könyörgünk néked.
/5
#589EB0EC
 Szeress jó reménységgel,
 Mely nem ejt szégyenben;
 Atyafi szeretettel
 Építsél éltemben,
 Hogy szívből szerethessem
 Felebarátomat,
 És ebben ne keressem
 Csak az én hasznomat.
/6
#7B77ABCC
 Engedd, hogy szent Igédet
 Bátran megvallhassam,
 Tisztemben hűségemet
 Néked megtarthassam;
 Senki engem e földön
 Tőled el ne vonjon,
 Híveid seregében
 Felséged megtartson.
/7
#F09F2620
 Hogy szent dícséretedre
 Végezzem e napot,
 Tőled ne térjek félre,
 Vezérljed útamot;
 Áldd meg termő földünket,
 Őrizz meg éltünkben,
 Mert minden szerencsénket
 Ajánljuk kezedben.

>494
/1
#82AEB9B3
 Tenéked, Uram, hálát adok
 Hozzám való jóvoltod ért,
 És teljes szívemből vígadok
 Rám árasztott jó kedvedért,
 Mert ím meghallgattál,
 Engem kiragadtál
 Nyavalyáimból,
 Szívbéli vídulást,
 Adtál szabadulást
 Sok bánatimból.
/2
#3F83DBD3
 Ím, feltámasztád napom fényét
 Mely homályba béborult volt,
 Elmosád rólam lelkem szennyét,
 Melyben hevertem, mint egy holt.
 Nyomorúság között,
 Mint egy lekötözött,
 Mikor hányattam:
 Voltál segítséggel,
 Hogy én több ínséggel
 Ne bántattassam.
/3
#BE9C7ECF
 Atyai gondviselésedet
 Sok ezerszer megpróbáltam,
 Reám kinyújtott kezeidet,
 Jó Istenem, által láttam;
 Tégedet dicsérlek
 Azért, míg itt élek
 E gyarló testben;
 Szájamnak szózatja
 Eged általhatja:
 Hiszek Igédben.
/4
#F6F95F2D
 Ma kezeidbe ajánlottam,
 Amit csak adtál énnékem;
 Felköltömkor is azt mondottam:
 Magad bírj engem, Istenem!
 Hogy a bűnnek rabja,
 Ne legyek szolgája;
 Adj estét érnem!
 Ki uralkodol s élsz
 És senkitől nem félsz:
 Ne hagyj elvesznem!

;Bourgeois L., Strasbourg, 1545
>495
/1
#03D8A647
 Hálát adok néked, mennybéli Isten,
 Szent Fiadnak, a Krisztusnak nevében,
 Hogy engemet ez éjjel megőriztél,
 Minden veszedelemtől megmentettél;
 Tartsd meg, kérlek, e napon is éltemet,
 Bűntől, minden kártól ments meg engemet.
/2
#788D2DA3
 Magam, Uram, ajánlom szent kezedbe,
 Mind testem, lelkem vedd őrizetedbe,
 Szent angyalidat ne vedd el mellőlem,
 Szent Lelked se távozzék el éntőlem,
 Hogy vakmerő bűnöktől ma is menten
 Megőriztessék épen testem, lelkem.
/3
#1FDC8996
 Könyörgök, Uram, minden emberekért,
 De főképpen a hív keresztyénekért,
 Minden rokonimért, kik téged félnek,
 Akik vagy itt, vagy messze földön élnek;
 Minden gonosztól őket is őrizd meg,
 És minden javaiddal látogasd meg.
/4
#CDEC3AF5
 A szegény rabokat és betegeket,
 Kik ínségükben óhajtnak tégedet,
 Uram, vigasztald meg bágyadt szívükben,
 Szenvedésükből mentsd ki kegyelmesen;
 És térítsd meg a szegény bűnösöket,
 Add, hogy jó véghez vigyük életüket.

>496
/1
#E31D2238
 Hálát adunk néked, mennyei Atyánk,
 Ki mindenkor kegyesen vigyázsz reánk.
/2
#FC76D88E
 Ím az elmúlt éjjel is megnyugtattál,
 E napra jó kedvedből virrasztottál.
/3
#E9C53708
 Áldj meg, kérünk, e napon is bennünket,
 Védelmezd kártól lelkünket, testünket.
/4
#E5F68B86
 Segélj meg hivatalos munkáinkban,
 Részeltess lelki, testi áldásidban.
/5
#AD6FD398
 Oltalmazz meg minden bűnbe eséstől,
 Téged félhessünk, tisztelhessünk szentül.
/6
#77E28227
 Erősits, nevelj naponként a hitben,
 Tarts meg végig atyai szerelmedben.
/7
#07BAD003
 Hogy Fiaddal s Lelkeddel egyetemben
 Néked víg szivvel szolgálhassunk! Ámen.

>497
/1
#504C15CA
 Mi szent Atyánk, ki lakozol mennyégben,
 És uralkodol az egy Istenségben,
 Szent Fiaddal, Lelkeddel dicsőségben
 Tisztelettel földön és a mennyégben:
/2
#093BC997
 Szenteltessék, Uram, a te szent neved,
 Dicsértessék mitőlünk, mert érdemled,
 Hogy ez éjjel minket te szent Felséged,
 Kegyelmesen őrizett Istenséged.
/3
#2773408D
 A gonosznak tőribe nem eresztél,
 Sőt a te szent angyaliddal őrzettél,
 És álmunkból életben felköltöttél,
 Lelki, testi veszélytől megmentettél.
/4
#008D604F
 Szent kezedbe ajánljuk mi lelkünket,
 Mai nap is tápláljad mi testünket,
 Védelmezzed, Uram, a mi éltünket,
 Lelki-testi veszélytől ments meg minket.
/5
#981E9D70
 Te vagy Uram, egyedül, mi Istenünk,
 Mennyen, földön nincs nekünk több reményünk;
 Légy te azért, Uram Isten, pásztorunk,
 Szükségünkben táplálónk és oltalmunk.
/6
#A0E1BB90
 Kegyelmeddel tartóztasd meg lelkünket,
 Hogy bűnnel ne fertőztessük testünket;
 Világosítsd inkább sötét elménket,
 Hogy megtartsuk, Uram, te Szent törvényed.
/7
#E42F1BC7
 A te Szentlelked minket ösvényidben,
 Vezéreljen, Uram, szent törvényidben,
 Ma és életünknek minden rendiben,
 Hogy így hozzád juthassunk az egekben.

>498
/1
#9C6B7152
 Ki hívta az éjszakát elő,
 Hogy megnyugodjatok,
 Ti, a napnak terhét viselő
 Test és állatok?
 Ki fordította fényességre
 Az éjnek homályát?
 Ki hozta fel napját az égre,
 Hogy fusson új pályát?
/2
#4BB7C259
 Te vagy az, mindenek Atyja,
 Mert a te kezednek,
 Mely az ég seregit forgatja,
 Nap és hold engednek.
 Téged hát imádni nem késünk
 És mihelyt hajnallik,
 Dícséret-mondó éneklésünk
 Buzgó szava hallik.
/3
#E637CB44
 Bézártuk volt, mikor aludtunk,
 Érzékenységinket;
 Mi magunkról semmit se tudtunk,
 De te tudtál minket.
 Ha felénk kár, vagy veszedelem
 Közelítni akart,
 A tenálad lakó kegyelem
 Oltalma betakart.
/4
#0C4054FD
 A te erőd, ó Mindenható,
 Légyen nekünk ma is
 Minden életünk háborgató
 Gonosz ellen paizs!
 Légy segítőnk, ha munkálkodunk
 S igazán keresünk;
 Vígasztalónk, ha szomorkodunk,
 Atyánk, ha elesünk.
/5
#1553F72B
 Adjad, hogy tartsuk fődolgunknak
 Szent nevedet félni,
 Hasznáért felebarátunknak
 Semmink sem kimélni.
 És szívünk szerint elfelejtsük,
 Ha megbánt valaki;
 A másoktól vett jót ne ejtsük
 Elménkből soha ki!

;Bourgeois L., Strasbourg, 1545
>499
/1
#E6031E77
 Megújult testtel és erővel
 Fölébredvén az álomból,
 Nyugodt szívvel, felemelt fővel
 Felkelvén csendes ágyamból,
 Életnek Ura, hozzád térek,
 Magasztalom jóvoltodat,
 Minden jót ismét tőled kérek:
 Kérem újra áldásodat.
/2
#015E38EC
 Mint a feljövő nap világa
 Elűzi az éj homályát:
 Kegyelmednek világossága
 Az igazságnak fáklyáját
 Sötét elmémben jobban gyújtsa,
 Hogy a jónak ismerete
 Határit benne messzebb nyújtsa,
 S az igazság szeretete.
/3
#BFFE7D1A
 Engedd, hogy mint a nap futása
 Soha meg nem állapodik,
 S bár felhő jő néha útjába,
 De meg nem homályosodik:
 Én is az igazság ösvényén
 Tántorodás nélkül menjek,
 És a rút testiségnek kényén
 Soha csak meg se pihenjek.
/4
#E19681FE
 Minthogy míg e világon élek,
 E mulandóságnak helyén,
 Test is vagyok, és nemcsak lélek:
 Viselj gondot jó idején
 Mértékletes eledelemről,
 Hogy életem fenntarthassam,
 S tisztességes öltözetemről,
 Hogy testemet ruházhassam.
/5
#72689D63
 Ha pedig még többet vehetek
 Ingyen való jóvoltodból
 És másokkal is jót tehetek
 Velem közlött áldásodból:
 Engedd, hogy legyek hű sáfára
 Nálam letett javaidnak,
 Lehessek vigasztalására
 Velem testvér fiaidnak.
/6
#8B3A8F39
 Dolgaimnak követésére
 Adj testemnek egészséget,
 A bajoknak meggyőzésére
 Lelkemnek elevenséget!
 Indíts szívemben akaratot
 És készséget minden jóra,
 Hogy bennem vidám indulatot
 Leljen az estvéli óra.

;Debrecen, 1774
>500
/1
#A7D07776
 Krisztus, ki vagy nap és világ,
 Minket sötétségben ne hagyj!
 Igaz világosság te vagy,
 Kárhozatra mennünk ne hagyj!
/2
#A2D1F71D
 Téged kérünk, szent Úr Isten:
 Oltalmazz minket ez éjen;
 Nyugodalmunk benned légyen,
 A mi lelkünk el ne vesszen!
/3
#B25CB6A4
 Nehéz álom el ne nyomjon,
 Az ellenség meg ne csaljon;
 Testünk hozzá ne hajoljon
 És haragodba ne hozzon!
/4
#1EB44273
 Mi szemeink ha alusznak,
 Szíveink rád vigyázzanak;
 Te hatalmadnak ereje
 Légyen híveid őrzője!
/5
#13F08467
 Úr Isten, hozzád kiáltunk:
 Gondviselőnk, légy oltalmunk!
 Őrizz meg ellenségektől,
 Lelki, testi ínségektől!
/6
#49C032EC
 Parancsoljad angyalidnak,
 Hogy mireánk vigyázzanak;
 A mi gonosz ellenségünk
 Messze távol járjon tőlünk!
/7
#4EDAF099
 Emlékezzél meg mirólunk:
 Jól tudod, mily gyarlók vagyunk;
 Kiket megváltál véreddel:
 Úr Jézus, kérünk, ne hagyj el!
/8
#EF89717C
 Dicsőség légyen Atyának,
 Ő szent Fiának, Krisztusnak,
 Szentlélekkel egyetemben,
 Örökkön-örökké! Ámen.

;Német dallam, 1507
>501
/1
#3BEC74C4
 Adjunk hálát az Úrnak, mert érdemli,
 Mert minden gazdagságát velünk közli.
/2
#784C5159
 Ő, mint bő irgalmú kegyelmes Atya,
 Fiait testben, lélekben megáldja.
/3
#09370BD6
 Énekeljünk néki egy akarattal:
 Dicsőség, Atya Isten, szent Fiaddal!
/4
#2D3DD0EE
 Ki ételt adtál alkalmas időben,
 S felruháztál
 mezítelenségünkben.
/5
#96A4A355
 Adjad, hogy tégedet megismerhessünk
 És szent Fiad által üdvöt nyerhessünk!
/6
#D2F510A6
 Szent Igédet hirdettessed közöttünk,
 Hogy éhen meg ne haljon a mi lelkünk!
/7
#D30D051F
 Hálát adunk ezekért mi Atyánknak,
 Jézus Krisztusnak, mi Közbenjárónknak.
/8
#08266256
 Szentlélek Istennel egyenlőképpen,
 Ki minket vigasztaljon, mondjuk: Ámen.

;Schulz J.A. Péter, 1790
>502
/1
#1313EFB9
 Fölkelt immár a szép hold,
 A csillagezres égbolt
 Oly tisztán tündököl;
 Az erdő áll sötéten,
 S fehér köd künn a réten
 Csodásan száll a légbe föl.
/2
#FDD65396
 Mély csend borult a földre,
 S mit alkony leple föd be,
 Meghitten integet,
 Mint nyájas, tiszta hajlék,
 Hol nappalodnak terhét
 Kialhatod, feledheted.
/3
#744D8785
 Nem látod-é a holdat?
 Fél arca int, mosolyg csak,
 Pedig kerek, egész.
 Van sok, mit itt az ember
 nem lát jól földi szemmel,
 És oktalan nevetni kész.
/4
#88D3D8EF
 Mint gyenge, földi férgek,
 Kik bűn útjára tértek,
 Mi nem sokat tudunk,
 Sok csalfa képet űzünk,
 Sok furfangot kifőzünk,
 S a céltól csak messzebb jutunk.
/5
#C0CA2CCA
 Ó, add, üdvöd keressük,
 Ne a mulandót lessük,
 Ne kössön fénye meg;
 Hagyj egyszerűvé válni
 S előtted élni, járni,
 Mint vidám, boldog gyermekek.
/6
#1A5ACF0F
 Ha jő a végső óra,
 Fordítsd a kínt is jóra.
 És adj szelíd halált;
 Ha innen elvezetsz te,
 Ó, hadd jutunk egedbe,
 Úr Istenünk és jó Atyánk!

;Isaac H. (kb. 1490) után
>503
/1
#B17E5386
 Már nyugosznak a völgyek,
 Az erdők s minden földek,
 Már alszik a világ,
 De míg eljő az álom,
 A szívemet kitárom,
 S az Úrhoz küldök hő imát.
/2
#FAFB425D
 Ó, nap, hová tűnél el,
 Hová űzött az éjjel,
 Mely harcban áll veled?
 Te nem ragyogsz az égen,
 De más napom van nékem:
 Betölti Jézus szívemet.
/3
#DC5DE5EB
 Már rám borul az éjjel,
 De biztatnak reménnyel
 Ott fenn a csillagok,
 Hogy engem is a mennybe
 Felvisz az Úr kegyelme,
 Ha a földtől megválhatok.
/4
#31261344
 A test nyugalmat áhít,
 És leveti ruháit,
 Halandóság jelit,
 De Jézusom az égbe'
 Öltöztet dicsőségbe,
 Ha végórám elközelít.
/5
#0D954461
 Ti, fáradt tagok, mostan
 Tegyétek le nyugodtan
 A napnak terheit.
 Vigadj, szívem, te is majd
 Levetheted a bút, bajt
 S a bűn terhét, mely keserít.
/6
#259CC576
 Már álom jő szememre;
 Ki vigyáz életemre,
 Ha most elszunnyadok?
 Izráel őrizője
 Lesz házamnak védője:
 Nem érhetnek károk, bajok.
/7
#50ADAD20
 Te légy, Jézus, oltalmam,
 Nálad lesz jó jutalmam,
 Hű szárnyaid alatt.
 Te vigyázz csak, Uram, rám,
 Nem árt akkor a Sátán: Testem-lelkem békén marad.
/8
#8816A9C1
 Ti is távol s közelben,
 Akik szerettek engem:
 Békén pihenjetek;
 A sötét éjszakába'
 Az Úr világossága
 Őrködjék hűn fölöttetek.

;Bourgeois L., Genf, 1551
>504
/1
#02BB1C1B
 A nap immár elenyészett,
 Az ég besötétedik,
 Nyugvóra vált a természet,
 És minden csendesedik;
 Engemet is az álom
 Megújít, feltalálom
 Fáradt testemnek nyugvását,
 Elmémnek megvidulását.
/2
#9DE6E8D4
 Míg hát fejem lehajtanám
 A szükséges álomra,
 Gondjaim elbocsátanám,
 Menvén nyugodalomra:
 Gondviselőm, táplálóm!
 Jóságodat hálálom;
 Aki ma is úgy szerettél,
 Hogy sok jóban részeltettél.
/3
#BE09907A
 Adtál erőt, tehetséget,
 Adtál ösztönt a jóra,
 De vajon bennem készséget
 Talált-e minden óra?
 Magam is jól érezem,
 Hogy sokra nem érkezem,
 A lélek kész, ámde a test
 Sokszor tehetetlen és rest.
/4
#99F93B8D
 Kegyelmednek köszönhetem
 E nagy ajándékot is,
 Hogy a jóra érezhetem
 Már csak a szándékot is.
 Amit azért kezdettél,
 Bennem felélesztettél,
 Kérlek, hogy félbe ne hagyjad,
 Sőt gyarapodását adjad.
/5
#10887675
 Bocsássad meg hibáimat,
 Melyeket ma ejtettem,
 Szaporítsad javaimat,
 Melyeket tőled vettem.
 Engedjed, hogy halálom,
 melynek a csendes álom
 Kiábrázoló példája:
 Légyen idvesség órája.

;Bourgeois L., Strasbourg, 1545
>505
/1
#4F9E8998
 Adjunk hálát megtartó Istenünknek!
 Ismét egy napját eltölténk éltünknek,
 Melyen ő nékünk békességet adott,
 Gondviselő oltalmába fogadott;
 Megadá mindennapi kenyerünket,
 Megtartá erőnket, egészségünket.
/2
#F715D9E1
 Ajánljuk magunk ismét oltalmába
 A már bekövetkező éjszakába'
 Mert nem lehet annak semmi félelme,
 Akinek a Mindenható védelme;
 Az övéinek ő ád nyugodalmat,
 Akik csak benne keresnek oltalmat.
/3
#EB5B91AC
 Köszönjük, Uram, hogy mirajtunk ma is
 Gondviselésed volt hűséges paizs,
 Hogy megmentetted veszélytől éltünket,
 Szánkat panasztól, sírástól szemünket;
 Munkánkra áldást s elég erőt adtál,
 És az estére békében juttattál.
/4
#5D6CCA8A
 Hányan, kik hosszú életet reméltek
 Reggel tetőled: még estét sem értek;
 És egészségért sokan esedeztek,
 Kiket fájdalmak ágyukba szegeztek;
 Árvákká lettek sokan s özvegyekké,
 Vagy gazdagokból váltak szegényekké.
/5
#34163D7B
 Mennyivel voltunk mi ezeknél jobbak,
 Hű védelmedre mennyivel méltóbbak?
 Mégis te minket ím külön választál,
 Ránk semmi romlást, kárt, veszélyt nem hoztál;
 Hordozott ingyen nagy kegyelmességed:
 Áldunk örökké ezért, Uram, téged!
/6
#EE4050A9
 Kérünk, ez éjjel is tartsd meg éltünket,
 Csendes álommal újíts meg bennünket,
 Több örvendetes reggelre virrassz fel,
 Új életre új napoddal támassz fel,
 Hogy új erővel tégedet szolgáljunk:
 Az Úrnak éljünk és az Úrnak haljunk.

>506
/1
#693F17E9
 Jézus Krisztus, szentek reménye,
 Híveid gondviselője,
 Benned bízók szent Istene!
/2
#07EF0D31
 Áldattassék, Uram, szent neved,
 Mindenektől istenséged,
 Mitőlünk is azt érdemled.
/3
#EC2547A9
 Kárhozattól ma is lelkünket,
 És veszélytől mi testünket,
 Megtartottad életünket.
/4
#828948D0
 Ételünket és italunkat
 Megadtad, és munkáinkat,
 Megáldottad dolgainkat.
/5
#F1B6B4AD
 Ez éjjel is mi életünket,
 Istenünk, a mi lelkünket,
 Ajánljuk néked testünket.
/6
#97256C21
 Parancsoljad szent angyalidnak,
 Hogy azok tábort járjanak,
 Mi mellettünk vigyázzanak.
/7
#1F0C5D41
 A setétségnek fejedelme,
 Lelkünk-testünk ellensége:
 Ne ártson ördög serege.
/8
#E3879B49
 Jézus Krisztus, igazság napja:
 Ez éjszakában csillaga,
 Légy lelkünk fényes hajnala!
/9
#EEE8B787
 Uram, virrassz fel egészségben,
 Lelki-testi békességben,
 Hogy reggel téged örömben
/10
#872BBC3D
 Dícsérhessünk te szent Atyáddal,
 És a Szentlélek Istennel
 Tisztelhessünk teljes szívvel.

>507
/1
#02340344
 Néked, mennyei Atyánk, hálát adunk,
 Szent Fiaddal s Szent Lelkeddel imádunk,
/2
#3CC89A3E
 Hogy minket e napon is megtartottál,
 Lelki, testi eledellel tápláltál.
/3
#DB9A582E
 Bocsásd meg, kérünk, minden bűneinket,
 Mikkel ma is megbántottunk tégedet.
/4
#656A5A0D
 Ez éjszakának is sötétségében
 Védelmezz minket minden gonosz ellen!
/5
#41F082D6
 Őrizz a Sátán nagy csalárdságától,
 És minden lelki-testi nyavalyától!
/6
#FCF40EED
 Áldd meg csendességgel nyugodalmunkat:
 Örömmel érjük felvirradásunkat!
/7
#5EF28596
 Tarts meg bennünket mindvégig a hitbe',
 Holtunk után vígy az örök életbe!

>508
/1
#7BC398FC
 Nagy hálát adunk, kegyes Atyánk, néked,
 Hogy te ez napon nékünk megengedted
 Nagy-szép békével élnünk teelőtted:
 Dicsőség Néked!
/2
#119A5028
 Immár e napnak kimenetelében
 Néked könyörgünk, Atyánk, igaz hitben:
 Segíts meg minket minden szükséginkben,
 Áldj meg lelkünkben!
/3
#6938B948
 Hogy megnyugodjunk a mi munkáinktól,
 Törődésinktől és fáradságinktóI,
 Néha pediglen mi nagy siralminktól
 És bánatinktól.
/4
#AB60D477
 Hogy tiszta szívből áldhassunk tégedet,
 Szép énekszóval dicsérjük nevedet,
 Józan elmével imádjunk tégedet,
 Mint Istenünket.
/5
#E8B01C55
 A sötét éjnek reánk jövésében
 Adjad, hogy lelkünk ne legyen sötétben,
 Sem pedig hitünk tökéletlenségben,
 És tévelygésben.
/6
#7275B061
 Ha a mi testi szemeink alusznak,
 Lelki szemeink reád vigyázzanak,
 A mi bűneink mind elaludjanak,
 És meghaljanak.
/7
#8129DA37
 Tartsd, Atyánk, tisztán testünket, lelkünket,
 Őrizz meg bűntől álmunkban is minket,
 Az álnok ördög ne bírjon el minket,
 Erőtleneket.
/8
#CA11CE30
 Adj békés nyugodalmat nékünk,
 És tennen­magad vigyázz,
 Urunk, értünk,
 Nagy-szép békével légyen felkelésünk,
 Teljes életünk.
/9
#FDD94A1A
 Tégedet kérünk, Istennek szent Fia!
 Néked könyörgünk, mi Urunknak Atyja!
 Hogy amit kérünk, Szentlelked megadja:
 Megkoronázza!

;Genf, 1562
>509
/1
#B12DD5F0
 Ne jöjjön addig szememre álom,
 Míg Teremtőmnek és Gondviselőmnek,
 Kitől minden jó adományok jőnek,
 Jótéteményit meg nem hálálom.
/2
#46EBE429
 Ha elgondolom, mennyi jót vettem
 Tetőled, Atyám, méltatlan létemre:
 Csudálom, mint vigyázol életemre,
 Ki csak eltűrést sem érdemlettem!
/3
#84E5A5F4
 Áldalak, hogy e megrepedezett
 Nádszálat ma eltörni nem engedted;
 Gyarló életem híven vezérelted,
 Melyet sok veszély megkörnyékezett.
/4
#1A9F98D4
 Számnak mind étele, mind itala,
 Az ép elme az egészséges testben,
 Az öröm és a békesség szívemben:
 Ingyen való adományod vala.
/5
#F128A9F0
 Kihez menjek több kegyelmet kérni?
 Szemem és szívem tehozzád emelem,
 Mert a tenálad lakozó kegyelem
 Mélységeit nem lehet megmérni.
/6
#C14A7E5A
 Bizton hajtom le fejem ez éjjel,
 Ha te, hű pásztor, tartasz engem szemmel,
 Így szemem nem lát s nem hallok fülemmel:
 Nem bánthat senki semmi veszéllyel.
/7
#15E7CFC4
 Légy hát őrállóm ez éjszakában!
 Hogyha tőreit útaimba hányja,
 Aki vagy vesztem, vagy károm kívánja:
 Ne menjen elő rossz szándékában.
/8
#B3665581
 Adjad, hogy véghezvivén munkáim,
 Lelkem és testem újra erőt végyen,
 És nyugodalmam mértékletes légyen,
 Hogy munkára szánhassam óráim.
/9
#7167125B
 Nem bocsátlak el, Atyám, tégedet,
 Míg meg nem áldasz engemet, fiadat:
 Dicsőítsd meg hát bennem irgalmadat,
 És én magasztalom Felségedet.

;Dykes J. B., 1823-1876
>510
/1
#72713869
 Ó, lelkem szent napsugara!
 Ahol te vagy, nincs éjszaka.
 Bár földi köd szakadna fel,
 S látás elől ne fedne el.
/2
#303435C3
 Ha csendes est szememre száll
 S szelíd álomharmat szitál,
 Gondoljam azt, Egyetlenem:
 Nyugodni jó a szíveden.
/3
#AA81544D
 Reggeltől estig légy velem,
 Nincs nálad nélkül életem;
 Légy velem, ha az éj leszáll,
 Nélküled rémít a halál.
/4
#B8CF05E7
 Ha egy bolygó, bús gyermeked
 Gúnyolta szód, mert tévedett,
 Ne hagyd bűnben, Kegyelmes, őt,
 Emeld fel azt a vétkezőt.
/5
#95893F38
 Virraszd, akit kór súlya nyom,
 Ki koldus, áldd meg gazdagon;
 Kit kín szorít, gyász keserít,
 Légy álma, könnyű és szelíd.

;Monk W. H., 1823-1889
>511
/1
#A7FA246F
 Maradj velem, mert mindjárt este van,
 Nő a sötét, ó, el ne hagyj, Uram;
 Nincs senkim és a vigaszt nem lelem,
 Gyámoltalannal, ó, maradj velem.
/2
#81EE5AF2
 Kis életem fut s hervadásba hull,
 Bú lesz a vígság, fényesség fakul,
 Csak változást és romlást lát a szem;
 Változhatatlan, ó, maradj velem.
/3
#3AEEC412
 Minden múló perc Hozzád visz közel,
 Kegyelmed űzi kísértőmet el,
 Nincs más vezérem, Nincs más Mesterem,
 Fényben, borúban, ó, maradj velem.
/4
#209526DB
 Ellenség ellen áldásod fedez,
 A könny nem sós, a kór is könnyű lesz,
 Sír, halál-fúlánk, hol a győzelem?
 Győztes leszek, csak légy, Uram, velem.
/5
#74BDB092
 Hunyó szemembe vésd keresztedet,
 Ködöt foszlatva láttasd szent eged.
 Föld árnya fut, menny fénye megjelen:
 Halálban is Te légy, Uram, velem.

;Hannover, 1648
>512
/1
#6C2376D4
 „Szólj, szólj hozzám, Uram, mert szolgád hallja szódat!”
 Így mondom, mert magam rég annak érezem.
 Hadd járjak utadon, hadd várjam égi jódat
 Hű szívvel szüntelen, hű szívvel szüntelen.
/2
#1D409408
 Adj lelkedből erőt, hogy értsem és szeressem
 Elrendelt utamat s minden parancsodat.
 Egy vágyat hagyj nekem: hogy halljam és kövessem
 Szent igazságodat, szent igazságodat.
/3
#BD820FCB
 Nincs oly tudós sehol, ki megtanít utadra,
 A bölcs nem fejti meg törvényedet sosem;
 Te fejted meg nekünk, te, hű szíveknek Atyja,
 Kinek szavát lesem, kinek szavát lesem.
/4
#1295110F
 Te nagy csodáidról bár fennszóval beszélnek
 És fennen hirdetik felséges rendedet,
 Ha nem te szólsz, Uram, a szó fülig ha érhet,
 De szívig nem mehet, de szívig nem mehet.
/5
#B37C6A36
 Szólj, szólj, én Istenem! - szól hangodból a jóság,
 A lelkem megfeszül s a hallásban segít,
 És szódban meglelem az örökkévalóság
 Jó édességeit, jó édességeit.
/6
#BF126DE2
 Szólj és csitítsd a bút, mert bú és kín gyötörnek,
 Szólj, hogy legyen szavad ír s gyógyító erő;
 Szólj, dicsőséged úgy még szebben tündökölhet,
 És mindörökre nő, és mindörökre nő.
