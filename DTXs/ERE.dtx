;Magyar Református Énekeskönyv - Kolozsvár 1999, 
;(a Magyarországi Református Egyház Kálvin Kiadójánál 1996-ban megjelent kiadvány felhasználásával)
;A szöveget gondozta Hatházy Ferenc. 
;Kiadta az: Erdélyi Református Egyházkerület, 2013
;ISBN: 978-973-7971-41-8
;A református énekeskönyv erdélyi változata.
;Az ötödik, javított kiadás alapján begépelte: Nandor Kantor
NErdélyi Református Énekeskönyv
RE.Ref
CEgyházak

;Kétféle életút
;Bourgeois L., Strasbourg, 1539
;
>1
/1
#E46D75E8
 Aki nem jár hitlenek tanácsán,
 És meg nem áll a bűnösök útán,
 A csúfolóknak nem ül ő székében,
 De gyönyörködik az Úr törvényében,
 És arra gondja mind éjjel, nappal:
 Ez ily ember nagy boldog bizonynyal.
/2
#59EA3963
 Mert ő olyan, mint a jó termőfa,
 Mely a víz mellett vagyon plántálva,
 Ő idejében meghozza gyümölcsét,
 És el nem szokta hullatni levelét;
 Ekképpen amit ez ember végez,
 Minden dolgában megyen jó véghez.
/3
#1566C13E
 De nem ígyen vannak a gonoszak;
 Hanem mint az apró por és polyvák,
 Melyek a széltől széjjelragadtatnak:
 Így az ítéletben meg nem állhatnak
 A gonoszok és kik bűnben élnek,
 Az igazak közt helyet nem lelnek.
/4
#020E2EF7
 Mert az Isten ismeri útukat,
 Az igazaknak érti dolgukat;
 Azért mindörökké ők megmaradnak;
 De akik csak a gonoszságban járnak,
 Azoknak nyilván mind elvész útuk:
 Mert Istennek nem kell az ő dolguk.

;Isten Felkentjének diadalmas uralkodása
;Bourgeois L., Genf, 1542
;
>2
/1
#199F4416
 Miért zúgolódnak a pogányok?
 Mit forgatnak ő bolond elméjükben?
 A földi népeknek mi szándékok?
 Csak hiába valót űznek szívükben.
 E világi királyok egybegyűlnek,
 A fejedelmek tanácsot tartnak,
 Az Isten ellen erős kötést tesznek,
 És az ő Krisztusára támadnak.
/2
#008F8197
 Nagy fennen mondják: mit késünk ezzel?
 Jer, szaggasuk el ezeknek kötelét,
 És minden igájukat rontsuk el,
 Ne viseljünk többé rajtunk effélét!
 De az Úr Isten a magas mennyekben
 Csak neveti dolgukat azoknak,
 Csúfolja őket, ülvén szent székében,
 Kinek ezek semmit nem árthatnak.
/3
#9802F979
 Szolgáljatok e hatalmas Úrnak
 Jámbor élettel, igaz félelemmel,
 Örvendezzetek ő nagy voltának,
 De ezek is légyenek rettegéssel:
 Csókoljátok e néktek küldött fiat,
 Hogy erősen meg ne haragudjék;
 El ne mulasszátok parancsolatját,
 Mert szörnyűképpen el kell vesznetek!

;Reggeli ének nehéz időkben
;Bourgeois L., Genf, 1551
>3
/1
#A645E096
 Ó, mely sokan vannak,
 Akik háborgatnak
 Engemet, én Istenem!
 Nagy sok ellenségim
 És sok gyűlölőim
 Tusakodnak ellenem.
 Sokan azt állítják,
 Lelkemről azt mondják:
 Elveszett ennek dolga,
 Segítsége nincsen,
 Mert elhagyta Isten,
 Így szólnak bolond módra.
/2
#8D0F680A
 Mert te, én Istenem,
 Paizsom vagy nekem,
 Ki életem megmented,
 És nagy tisztességre,
 Fejem dicsőségre
 Idővel fölemeled.
 Tehozzád, Úr Isten,
 Kiáltok szüntelen,
 És te megvidámítasz;
 Meghallgatsz kedvedből,
 Sion szent hegyéről
 Nagy segedelmet nyújtasz.
/3
#059B05B7
 Ha ágyamban nyugszom,
 Csendesen aluszom,
 Nincsen semmi félelmem.
 Midőn felserkenek,
 Semmit sem kesergek,
 Mert Isten őriz engem.
 Ha százezer népek
 Mind körülvennének
 Jobb és bal kezem felől,
 Ha rám ütnének is,
 Nem rettegnék mégis
 Semmi veszedelemtől.
/4
#A0208B9B
 Kelj föl, Uram, tarts meg,
 Ellenségim vond meg,
 Megtörvén ő fogokat!
 Mind összepaskolod
 És arcul csapdosod
 Az Isten-utálókat.
 Csak te vagy az Isten,
 Ki minden szükségben
 Meg tudsz szabadítani,
 Ki a te sereged
 Megtartod, szereted,
 És meg szoktad áldani.

;Esti ének nehéz időkben
;Bourgeois L., Genf, 1542
>4
/1
#8CD79F8A
 Én igazságomnak Istene,
 Hallgasd meg én kiáltásom!
 Szánj meg és tekints ínségemre,
 Te vigy engemet tágas helyre,
 Midőn itt szorongattatom!
 Ti nagy urak, míglen gyaláztok
 Engemet tisztességemben?
 És ily hívságban míg maradtok?
 Hazudozásra mit vágyódtok?
 Mit gyönyörködtök ezekben?
/2
#6160045A
 De vegyétek jól eszetekbe,
 Hogy az Úr Isten engemet
 Bévett kedvébe, kegyelmébe,
 Csuda mód megmentett engemet,
 És meghallja kérésemet!
 Rettegjetek hát és lássátok,
 Hogy ellene ne vétsetek;
 Magatokat meggondoljátok,
 Az ágyasházban ha nyugosztok,
 Hogy lehessen csendességtek!
/3
#9CEFA22F
 Az igazak áldozatjával
 Áldozzatok az Istennek;
 Jó élettel és igazsággal,
 És az Istenben bátorsággal
 Bízzatok és örvendjetek!
 De sokan mondják azt minékünk:
 Ki vezérl minket a jóra?
 Azért téged, Úr Isten, kérünk,
 Mutasd kegyes orcádat nékünk,
 Jöjjön el az áldott óra!
/4
#758BE625
 Mellyel inkább vigasztalsz engem
 És örvendeztetsz szívemben,
 Hogynem kiknek sok mustjok terem,
 Búzájokkal rakva sok verem,
 Mikkel élnek bőségesen.
 Azért élek jó békességben,
 Fekszem, aluszom kedvemre,
 Uralkodván lakom földemben
 Bátorságos örvendezésben,
 Mert az Úr vigyáz éltemre.

;Reggeli könyörgés az esztelenek ellen
;Bourgeois L., Genf, 1542
>5
/1
#9A0D8DFB
 Úr Isten, az én imádságom,
 Kérlek, vegyed füleidbe
 És hallgass meg kérésembe'!
 Én Istenem és én királyom,
 Értsd meg mondásom.
/2
#CFB6A56E
 Tekintsed meg esedezésem,
 És halld meg kiáltásomat,
 Midőn hívlak, királyomat!
 Meghallgatod én könyörgésem,
 Bizonnyal hiszem.
/3
#AF2AE82C
 Jó reggel meghallgatsz engemet,
 Uram, még virradta előtt,
 Idején a nap hogy feljött;
 Elődbe számlálom ügyemet,
 Várván kegyelmet.
/4
#905188AD
 Mert egyedül te vagy oly Isten,
 Kinek a gonoszság nem kell;
 És aki gonosz bűnben él,
 Nem mehet hozzád semmiképpen,
 Míg él vétekben.
/5
#9648439B
 Én pedig nagy jó reménységgel
 Bémegyek szent templomodba,
 És imádlak szent házadba;
 Nagy jóvoltodért félelemmel
 Szolgállak szívvel.
/6
#5171A4F1
 Uram, vezérelj igazságban
 Ellenségimnek láttokra,
 Kik igyekeznek káromra;
 Oktass, hogy a te útaidban
 Járhassak jobban.
/7
#2FA16325
 És hogy azok mind örüljenek,
 Akik bíznak csak tebenned,
 Szívből szeretik szent neved:
 Engedd, hogy vígan
 Felségednek Énekeljenek.
/8
#FDF47FAC
 Az igazat mert te megáldod,
 Te nagy irgalmasságoddal
 Körülveszed, mint paizzsal;
 A gonosz ellen őt megtartod
 És oltalmazod.

;Lelki-testi nyomorúságban (Első bűnbánati zsoltár)
;Bourgeois L., Genf, 1542
>6
/1
#460EB1E0
 Uram, te nagy haragodban,
 Mely miatt vagyok búban,
 Engemet ne feddj meg!
 És haragodnak tüze,
 Szűnjék meg sebessége,
 Melyben ne büntess meg!
/2
#6D93DC40
 Nékem, Uram, légy irgalmas,
 Mert vagyok nagy fájdalmas.
 Ne hagyj, Uram, kérlek!
 Gyógyítsd meg sérelmimet,
 Elrettent tetemimet
 Újítsd meg, hogy éljek!
/3
#663398DF
 Térj, Uram, kegyesen hozzám,
 Mert, jaj, elfogyatkozám!
 Te nagy irgalmadból
 Szánj meg nagy nyavalyámban
 És keserves kínomban:
 Ments meg a haláltól!
/4
#6B423976
 Én szomorúságim miatt
 Én két szemem elsorvadt
 És elhomályosult;
 Ezt szerzik ellenségim,
 Vigadnak gyűlölőim,
 Min szívem elbúsult.
/5
#11B22FA6
 Azért minden ellenségim
 És én háborgatóim,
 Pironkodjatok el!
 Már mind hátra térjetek,
 És megszégyenüljetek
 Nagy hirtelenséggel!

;Könyörgés segedelemért
;Bourgeois L., Genf, 1551
>7
/1
#6D585D10
 Ó, én Uram és én Istenem,
 Tebenned vagyon reménységem,
 Én oltalmamra légy jelen,
 Tarts meg ellenségem ellen,
 Hogy engemet el ne ragadjon,
 Mint éh oroszlán, meg ne rágjon,
 Amidőn nincs segítségem,
 Aki megmentene engem.
/2
#0EABCD90
 Hogyha én ezt tettem, Úr Isten,
 Avagy hamisság van kezemben;
 Hogyha gonoszt tettem ennek,
 Ki örült békességemnek;
 Hogyha valaki abban megért,
 Hogy gonoszt fizettem a jóért;
 Sőt, ha jól nem tettem azzal,
 Ki nekem volt bosszúsággal:
/3
#5E2F35BF
 Ámbár kergessen ellenségem,
 És bátor megragadjon engem,
 Életemet földhöz verje,
 Dicsőségem porrá tegye!
 Kelj föl azért nagy haragodban,
 Ellenségem ellen támadván;
 Add meg az előbbi tisztem,
 Kit rendeltél, Uram, nekem!
/4
#DB8EB1D2
 Verd meg az istenteleneket,
 És védelmezd meg híveidet,
 Mert mindeneknek titkait,
 Látod, Uram, szívit-lelkit.
 Te vagy paizsom, igaz Isten,
 És nem hagysz el veszedelmemben,
 Ki a híveket megtartod,
 A gonosztól takargatod.

;Estvéli ének a mennyboltozatról és az emberről
;Bourgeois L., Genf, 1542
>8
/1
#5B72D976
 Ó, felséges Úr, mi kegyes Istenünk,
 Mely csudálatos a te neved nékünk!
 Nagy dicsőséged ez egész földre
 Kiterjed és felhat az egekre.
/2
#A07A7814
 Dicsérnek téged még a csecsszopók is,
 Szájukban viselik nevedet ők is,
 Kik által ellenséget megejtesz,
 És bosszúállót megszégyenítesz.
/3
#92508610
 Nagy voltát ha megnézem dolgaidnak,
 Melyeket a te kezeid formáltak,
 Az eget, holdat, a fényes napot,
 És szép renddel a sok csillagokot:
/4
#C068D328
 Csudálván mondom: micsoda az ember,
 Ki tőled ennyi sok dicsőséget nyer?
 De micsoda az embernek fia,
 Kiről Felségednek van ily gondja?
/5
#E0A9F153
 Az angyaloknál noha egy kevéssé
 Kisebbé tőd, de nagy dicsőségessé
 Teremtéd őtet és magasztalád,
 Nagy dicsőségre felkoronázád.
/6
#2D6482BC
 Kezed munkáin őtet úrrá tevéd,
 Hogy azokkal bírna, néki engedéd,
 Valamit e világra teremtél,
 Mindeneket lába alá vetél.
/7
#B0B32106
 Ó, felséges Úr, mi kegyelmes Urunk,
 Mely csudálatos a te neved nálunk!
 Felségednek mely nagy dicsősége,
 Mellyel teljes e föld kereksége.

;Hálaének Isten ítéletéért
;Bourgeois L., Genf, 1542
>9
/1
#7318DBCE
 Dicsérlek téged, Úr Isten,
 És áldlak teljes szívemben,
 És a te csudatételidet,
 Hirdetem jótéteményidet.
/2
#61ED3AFB
 Tebenned, Uram, vigadok,
 Nagy örömömben tombolok,
 És a te felséges nevednek
 Szép dicséreteket éneklek.
/3
#DBD416AD
 Mert az én ellenségimet
 Veréd, megtérítéd őket,
 Kik rettegvén, hátra esének,
 Szent színed elől elveszének.
/4
#5F30681B
 Én ügyemet megtekintéd,
 És kegyelmesen felvevéd;
 Ülvén a törvénytevő székben,
 Megmentél igaz ítéletben.
/5
#6A90539E
 Énekeljetek az Úrnak,
 A Sion hegyén lakónak!
 Sokságát cselekedetinek
 Hirdessétek el minden népnek.
/6
#A4170270
 Aki nyilván megkeresi,
 Az igaz vért nem felejti,
 A szegény népet ő nem hagyja,
 Akiknek kiáltását hallja.
/7
#C8B81856
 Én Uram és én Istenem,
 Tekintsed meg nagy ínségem:
 Életemet a gonosz gyötri;
 A halál kapuiból végy ki!
/8
#AFCB9738
 Kelj fel, Uram, és légy jelen,
 Hogy ember erőt ne vegyen;
 A pogányokat hívd elődbe,
 Ítéld meg erős törvényedbe!
/9
#C4ECE08E
 Szívökben, Uram, rettentsd meg,
 Hogy magukat gondolják meg,
 És ismerjék a pogány népek,
 Hogy ők is halandó emberek.

;Nyomorultak kiáltása segedelemért
;Bourgeois L., Genf, 1551
>10
/1
#5677CE59
 Mire távozol tőlünk, Úr Isten,
 Ily meszsze, és magad mit rejted el?
 E sok ideig való ínségben,
 Miért hogy minket így felejtesz el?
 Ím, az istentelen kevélységgel
 Kergeti a szegényt és nyomorgatja:
 Hálója essék az önnön nyakába!
/2
#2448C156
 Mert az istentelen dicsekedik,
 És gyönyörködik kívánságában.
 Dicséri a fösvényt, hízelkedik,
 Istent káromol felfuvalkodván,
 Akit megutál nagy hívságában,
 Sőt kevélységében így gondolkodik:
 Hogy nincsen Isten, azzal csúfolódik.
/3
#84B0E348
 Ő bolondságában úgy elmégyen,
 Erős ítéletedet nem féli.
 És kevélységében szól nagy fennen,
 Ellenségit is semminek véli.
 Könnyen elfúhatja, azt ítéli,
 Végre mond: bátran lakom és csendesen,
 Soha nem esem szerencsétlenségben.
/4
#92DD13F1
 Átokkal, szitokkal rakva szája,
 Szól az ő nyelve csak álnokságot,
 E nyelvet szoktatta csalárdságra,
 Mellyel szerez sok bút és bánatot.
 Tolvajok módján megáll barlangot,
 Leshelyből a szegényre ólálkodik,
 Hogy megfoghassa, csak azon forgódik.
/5
#0E9D9EAA
 Uram, tekintsd meg a nyomorultat,
 A szelídeket kegyesen tartsd meg;
 Vedd füledbe ő kiáltásukat,
 Erősítsd szívüket, vigasztald meg.
 A szegény árvákat védelmezd meg,
 Zabolázd meg az erőszaktevőket:
 E földön óvd a gonoszoktól néped.

;Kishitűség ellen
;Bourgeois L., Genf, 1551
>11
/1
#A0B66408
 Az Istenben bízom jó reménységgel,
 Miért szóltok hát így én lelkemnek,
 Hogy én a ti hegyetekre fussak el,
 És mint egy madár messze repüljek?
 Mert a kegyetlenek ívüket szegzik,
 Hogy a tiszta szívűket meglőjék,
 Már a nyilakat az idegbe tették.
/2
#69CE7350
 Mert igaz ő, és igazságot szeret,
 És kedves orcát mutat azoknak,
 Kik igazságban rendelik éltüket.
 Az Istennek temploma mennyekben,
 Ott fenn vagyon szent széke, melyből széjjel
 Szemei mindent néznek élesen,
 Földi népet is megnézell és szemlél.

;Imádság szorongattatásban
;Bourgeois L., Genf, 1551
>12
/1
#26BDF7BA
 Szabadíts meg és tarts meg, Uram Isten,
 Mert szentid elfogytak, nincs jól tévő,
 És a földön már sok a tökéletlen,
 Nincs emberek közt igaz beszédű.
/2
#C17292D7
 Ezek egymásnak szólnak csak hívságot,
 Dolgukat festik szép beszédekkel;
 Hízelkedvén, mutatnak nyájasságot,
 De nyelvük nem egyez ő szívükkel.
/3
#20750817
 És ezt mondják: jer, tegyük azt nyelvünkkel,
 Hogy minket minden nagyra becsüljön;
 Ajkunkkal, szánkkal elbírunk ezekkel,
 És senki sincs, ki velünk pöröljön.
/4
#3D6A6D75
 Azért mond Isten: ímé, a szegények
 Elhagyatnak, öletnek, pusztulnak;
 Azért őket megszánom és felkelek,
 És kezükből kimentem azoknak.
/5
#714F4A5E
 Az Istennek mondási oly igazak,
 Mint a drága ezüst, kit a tűzben
 Az ötvösök kohókban tisztítottak,
 És hétszer megeresztettek szépen.
/6
#750C2881
 Tartsd meg azért népedet kegyelmesen,
 Kérünk, jóvoltodból reánk tekints!
 Őrizz meg örökké e nemzet ellen,
 Hozzánk mindenkor jókedvet jelents!

;Meddig még?
;Bourgeois L., Genf, 1542
>13
/1
#F166E449
 Míglen felejtesz el, Uram,
 Míg nem emlékezel rólam?
 A te orcádat énelőlem,
 Örökké elrejted e tőlem?
 Mire nem könyörülsz rajtam?
/2
#BBB688F3
 Tekints reám, kegyes Uram!
 Szánjad meg az én nyavalyám!
 Szemeimet világosítsd meg,
 Hogy életemben viduljak meg,
 Halálban el ne aludjam.
/3
#8EBDE5A0
 Mert én bízom jókedvedben,
 Ki megvigasztalsz szívemben.
 Örvendez azért az én lelkem,
 Hogy Isten az én segedelmem,
 Melyért dicsérem énekben.

;Isten erősebb, mint ellenségei
;Bourgeois L., Genf, 1542
>14
/1
#19BD7CE1
 A bolond így szól az ő szívében:
 'Nincs Isten', azért nagy gonoszságban él;
 Utálatos bűnt teszen, semmit nem fél.
 Ez egész földön aki jót tegyen,
 Senki nincsen.
/2
#D1060FE8
 Az Úr az égből alátekinte
 E földön az emberek fiaira,
 Hogy meglássa, ha kinek esze volna,
 Ha valaki az Istent keresné
 És tisztelné.
/3
#89A5D6F4
 De azt jól látja dicsőségében,
 Hogy a jó útról eltértek mindenek,
 Mindnyájan fertelmes bűnben hevernek;
 Aki az Istent tisztelné híven,
 Csak egy sincsen.
/4
#E3176CBE
 A Sionról vajon ki jövend el,
 Ki a szent Izráelt megszabadítja?
 Ha Isten fogságból népét kihozza,
 Örvend a Jákób és az Izráel
 Teljes szívvel.

;Ki mehet az Úr elé?
;(1539) Bourgeois L., Genf, 1542
>15
/1
#6FBB6F9E
 Uram, ki lészen lakója
 A te felséged sátorának?
 Jelentsd meg és add tudtomra,
 Ki lészen örök lakosa
 Szent hegyednek és hajlékodnak?
/2
#C4A3E68B
 Aki jár igaz életben,
 Szól és szolgáltat igazságot,
 Forgódik híven mindenben,
 És hűséges ő szívében,
 És szereti a jámborságot.
/3
#0B542E13
 Ki az ő felebarátját
 Ő nyelvével nem rágalmazza,
 Kárral nem bántja szomszédját,
 Nem rútolja atyjafiát,
 És tisztességét nem gyalázza.
/4
#C234F6CC
 Az istentelent utálja,
 Az istenfélőt megbecsüli,
 Őt nagy tisztességben tartja,
 Esküvését meggondolja,
 Ha kárt is vall, azt meg nem szegi.
/5
#CB65AF34
 Aki pénzével él híven,
 Kölcsön ád, de nem kér uzsorát;
 Az ártatlan ember ellen
 Ajándékot ő nem veszen:
 Aki így tészen, az megállhat.

;A hívek öröksége: Isten
;Bourgeois L., Genf, 1551
>16
/1
#00261736
 Tarts meg engemet, ó, én Istenem,
 Mert reménységem vetem csak tebenned!
 Azért az Úrnak mondjad ezt, lelkem:
 Én Uram vagy te, örvendek tenéked!
 E kívül nem kérkedhetem semmivel,
 Hogy néked használhatnék jótétemmel.
/2
#7C74A4E2
 Az Úr Isten az én örökségem,
 Mely nekem kiváltképpen részeltetett,
 És ő megtartja híven azt nékem.
 Az én sorsom a legjobb részre esett;
 Hogy az örökség zsinórral osztatott,
 Azzal a legszebbik rész nekem jutott.
/3
#E42C24A8
 Dicséreteket mondok az Úrnak,
 Ki tanácsával engem jól vezérel;
 Még veséim is erre tanítnak,
 És az ágyban is megintenek éjjel.
 Az Urat szüntelen előttem tartom:
 Mert jobbom felől van, nem ingadozom.
/4
#6E460971
 Öröme vagyon szívemnek ezen,
 Örvendez lelkem és megnyugszik testem:
 Noha a sírban fekvése leszen,
 De azt szívemben nyilván én elhiszem:
 A koporsóban nem hagyod szentedet,
 De rothadás ellen megmented őtet.
/5
#5E05CDF2
 Az életre nekem utat mutatsz,
 Mely életben van az örök boldogság,
 Hol szent színednek gyönyörű voltát
 Láttatod, mely nagy öröm és vigasság.
 Nagy dicsőséged, jobbod erőssége
 Örökké megmarad, soha nincs vége.

;Könyörgés oltalomért
;Bourgeois L., Genf, 1551
>17
/1
#F2592094
 Hallgasd meg igazságomat,
 Én kiáltásom, Uram, értsd meg!
 Hozzám figyelmezz és tekintsd meg
 Szívbéli imádságomat!
 Ítéletet tetőled várok,
 Nézd meg ügyem, láttass törvényt:
 Ítélj meg igazság szerint,
 Mert én álnoksággal nem járok.
/2
#A59C3A32
 Szívem éjjel megpróbáltad,
 És megvizsgáltad teljességgel,
 Láttad, hogy egyez én nyelvemmel:
 Csalárdság nélkül találtad.
 Amit ember szól, avagy művel,
 Én ajakid beszédire,
 Gondot tartok szent Igédre,
 Nem járok a gonosztevővel.
/3
#68B22934
 Vezérljed én járásomat,
 És tarts meg a te ösvényiden,
 Máshova senki ne térítsen;
 Erősítsed lábaimat!
 Téged hívlak én segítségre,
 Uram, láss meg szükségembe’,
 Kérésem vedd füleidbe,
 És légy figyelmes beszédemre.
/4
#2336834F
 Te vagy bizonyos oltalmok,
 Akik tebenned reménylenek;
 Jóvoltod mutasd meg ezeknek,
 Hogy lássák a rád támadók:
 Mint a te szemeidnek fényét,
 Úgy őrizz meg, Uram, engem,
 Szárnyad árnyékában fejem
 Takargasd, őrizvén ösvényét.

;Hálaadás a csodálatos szabadításért
;Bourgeois L., Lyon, 1547
>18
/1
#1B3E459D
 Ó, én Uram, ki erőt adsz énnékem,
 Szeretlek téged, míg leszen életem.
 Én magas kőszálam, ó, én Uram,
 Erős bástyám és erős kőváram!
 Én erős Istenem és bizodalmam,
 Idvességemnek bizonysága, pajzsom!
 Midőn az Urat dicsérvén kérem,
 Ellenségimtől ő megtart engem.
 Halál fájdalmi hogy körülvennének,
 Béliál folyói rettegtetnének,
 Pokol kötele vala körülem,
 Csaknem a halál tőribe esém.
/2
#A22A6590
 Térhelyre hoza s kimente engemet,
 És mutata énhozzám nagy szerelmet,
 Megfizete az igazság szerint,
 Kezeimnek tisztasága szerint.
 Mert az Úrnak útától nem tértem el,
 Az én Istenemtől nem szakadtam el,
 Ítéletire szüntelen néztem,
 És szent törvényét meg nem vetettem.
 De mindenkor híven előtte jártam,
 Gonosztételtől magamat megóvtam:
 Megfizete az igazság szerint,
 Kezeimnek tisztasága szerint.
/3
#3577183B
 Szent vagy és jóltevő a jóltevőkkel,
 És igazat téssz igazán élőkkel,
 Tiszta vagy azokhoz, akik tiszták,
 Elfordulsz tőlük, akik gonoszak.
 Nyomorult szegényeket megsegíted,
 A kevély szeműket megszégyeníted;
 Nékem, Uram, szövétneket gyújtasz,
 És a setétben világot nyújtasz.
 Teáltalad ellenségim seregin
 Általfutok, átszököm kerítésin.
 Mely tökéletes az Isten úta,
 Tiszta és próbált az ő mondása.

;Isten dicsősége a mennyen és az Igében
;Bourgeois L., Genf, 1542
>19
/1
#B18F5D22
 Az egek beszélik
 És nyilván hirdetik
 Az Úrnak erejit.
 Az ég menynyezeti
 Szépen kijelenti
 Kezének munkáit.
 A napok egymásnak
 Tudományt mutatnak
 Az ő bölcseségéről,
 Egy éj a más éjnek
 Beszél az Istennek
 Ő nagy dicsőségéről.
/2
#D74C1212
 Nincs szó, sem tartomány,
 Holott e tudomány
 Nem prédikáltatnék;
 Mindenfelé mégyen
 E földkerekségen
 Beszédük ezeknek.
 Írásuk kimégyen
 Mind e világ végén,
 Holott a fényes napnak
 Hajlékot az Isten
 Helyheztetett szépen
 Ő lakó szállásának.
/3
#0C0304A7
 Melyben mint vőlegény
 Reggel felkél szépen
 Ő ágyasházából,
 És ugyan örvendez,
 Mint az erős vitéz,
 Ha futásra indul.
 Az égnek egy végén
 Felkél és elmégyen
 Gyorsan a más végére;
 Sehol semmi nincsen
 Ő hévsége ellen,
 Ki magát elrejthesse.
/4
#DE9AF4AA
 Az Isten törvénye
 Tiszta, ő beszéde
 Lelkeket megtérít,
 Hűség bizonysága,
 Kisdedeket abba’
 Bölcsességre tanít.
 Ő parancsolati,
 Igazak mondási,
 Mik szívet vigasztalnak.
 Minden ő törvényi,
 Tiszták szent beszédi,
 Szemet világosítnak.
/5
#07A4926B
 Az Isten félelme
 Tiszta, mindörökké
 Megmarad és megáll.
 Az ő ítélete
 Igaz mindenekbe’,
 Teljes nagy jósággal.
 Aranynál, ezüstnél
 És drágaköveknél
 Sokkal becsületesebb;
 Ő szerelmessége
 És gyönyörűsége
 A méznél is édesebb.
/6
#73DA4D02
 Aki szolgál néked,
 Tanul, Uram, tőled
 Nagy jó tanulságot.
 És ha azt megtartja,
 Jól lészen ő dolga,
 Mert veszen jutalmat.
 Ki tudná bűninek,
 Számát esetinek,
 És ki gondolhatná meg?
 Én sok bűneimet,
 Titkos vétkeimet,
 Uram, nékem bocsásd meg!
/7
#AE2CFF0C
 Szolgádat őrizd meg,
 Kevélységtől tartsd meg:
 Ne essék e bűnbe,
 És én tiszta lészek,
 Semmi bűnt nem tészek,
 Járván te kedvedbe’.
 Szájamnak szólása,
 Szívem gondolatja
 Kedves legyen tenéked!
 Adjad, ó, én Uram,
 Kősziklám, megváltóm,
 Hogy ne vétsek ellened.

;Imádság a Messiás-király szabadításáért
;Bourgeois L., Genf, 1551
>20
/1
#114162CA
 Az Úr tégedet meghallgasson
 Te nagy ínségedben,
 A Jákób Istene megtartson
 Te veszedelmedben!
 Küldjön tenéked segedelmet
 Az Úr ő szent házából,
 Tereád fordítsa kegyelmét,
 Tartson meg a Sionból!
/2
#E77CC556
 Áldozatidat megtekintse,
 Mikkel őt tiszteled,
 Égő áldozatodat tüze
 Égesse meg neked.
 Amit a te szíved kívánhat,
 Adja meg ő tenéked,
 Hogy mindennemű szándékodat
 Te jó véghez vihessed.
/3
#2E49E740
 Adjad, Uram, hogy te nevedben
 A mi zászlóinkat
 Felemeljük nagy örömünkben
 És adjunk hálákat,
 Mondván: Az Úr Isten megőrzi
 Fölkentjét kegyelmével,
 A mennyből őtet erősítí
 Jobb keze erejével.
/4
#B3A0630D
 Némelyek az ő szekerükben,
 És bíznak lovukban;
 De mi a nagy Isten nevében
 Bízunk mint Urunkban.
 Azért ők keményen megesnek,
 Mi pediglen megállunk;
 Ők mind a földhöz verettetnek,
 De mi épen maradunk.

;Imádság szabadulás után (Messiási)
;Bourgeois L., Genf, 1551
>21
/1
#6D2D51C6
 Örvendez, Uram, a király
 A te nagy hatalmadban
 És szabadításodban;
 És vigad nagy buzgósággal,
 Hogy őt megsegítéd,
 Ínségből kimentéd.
/2
#A97C2001
 Úgy viseled néki gondját,
 Hogy amit tőled kérend,
 Mindeneket megnyerend.
 Mihelyt felnyitja ő száját,
 Szól alig egy igét:
 Már hallod kérését.
/3
#7E3220D9
 Elébb, hogynem könyörgene,
 Meghallgatod, meglátod,
 Irgalmaddal megáldod,
 És feltéssz az ő fejére
 Sárarany koronát
 Mint királyi pompát.
/4
#112C6088
 Őt felvőd nagy dicsőségre
 A te segedelmeddel,
 Örök üdvösségeddel.
 E nagy királyi felségre
 Tőled emelteték
 És ékesítteték.
/5
#AFD42B27
 Szereted őt minden jókkal,
 És a te szent áldásod
 Néki örökké nyújtod.
 Örvendetes vígságokkal
 Őt gyönyörködteted,
 Színedre nézeted.
/6
#D1955B9E
 E király mindenkor bízik
 Csak az ő Istenében,
 És nem fél veszélyében.
 A Magasságosnak nyugszik
 Irgalmasságában
 És megmarad abban.
/7
#D29A6DCF
 Megtalálja kezed őket,
 Akik reád támadnak,
 Bosszúságodra járnak.
 És a te gyűlölőidet
 Kezeidből senki
 Soha ki nem menti.
/8
#96057FC9
 Mint a hév tüzes kemence,
 Haragod körülveszi
 És őket mind ellepi.
 Haragos orcádnak színe
 Őket megemészti,
 Mint a láng, elnyeli.
/9
#BFF0618F
 Mert gonoszra igyekeztek,
 Szándékoztak ellened,
 Hogy bosszantsanak téged.
 Sok csalárdságot terveztek,
 De hogy véghez menjen,
 Erejükben nincsen.
/10
#9E2317BA
 Azért, Uram, már támadj föl:
 Mutasd meg hatalmadat,
 Lássuk erős voltodat,
 Hogy dicsőséges erődről
 Vígan énekeljünk,
 És benned örvendjünk.

;Miért hagyál el engemet? – Krisztus szenvedéséről
;Bourgeois L., Genf, 1542
>22
/1
#132120C7
 Én Istenem, én erős Istenem!
 Miért hagyál el enynyire engem?
 Kiáltásomtól a segedelem
 Nagy távol vagyon.
 Én kiáltok tehozzád egész napon,
 De mégsem felelsz meg, nincs ki megtartson;
 Még éjjel sem hallgatok semmi módon
 Ez ínségben.
/2
#ADE05DB3
 De te szent vagy és az Izráelben
 Te szentséged megmarad mindenben:
 Dicsértetel e gyülekezetben
 Szívvel, lélekkel.
 A mi régi atyáink teljességgel
 Tebenned bíztanak jó reménységgel,
 Szükségükben őket segedelmeddel
 Megtartottad.
/3
#3A864BA3
 Ha ők szívből kiáltottak hozzád,
 Mindjárt őket megszabadítottad;
 Benned bíztak és őket nem hagytad
 Esni szégyenben.
 Én nem vagyok ember, de féreg lévén,
 Minden népeknél vagyok nevetségben:
 Csúfolnak, utálnak és megvet minden
 És keserget.
/4
#1E48337B
 Aki lát, minden csúfol engemet,
 Száját elvonssza, szól merő mérget,
 Fejét rázza, hunyorgatja szemét,
 Reám néz szörnyen,
 Mondván: ez ember bízott az Istenben,
 Szabadítsa meg azért őtet innen;
 Ha szereti az Isten, néki légyen
 Segítségül.
/5
#149B50FE
 Hogy te engemet anyám méhéből
 Kihozál, ottan segedelmem lől,
 Csecsemő koromban is egyedül
 Csak benned bíztam.
 Sőt, mihelyt anyám méhéből származtam,
 Istenem voltál, reád támaszkodtam,
 Bátorsággal tehozzád ragaszkodtam,
 Én Istenem!
/6
#D12BF836
 Ne távozzál azért messze tőlem,
 Ne hagyj el, mert nagy az én gyötrelmem!
 Nincs segítőm, és az én sérelmem
 Nem fáj senkinek.
 Sok erős bikák engem körülvettek,
 A básáni nagy ökrök reám törnek,
 Megölni, megtaposni igyekeznek
 Nagy méltatlan.
/7
#E38A602F
 Az ő szájukat énreám tátván,
 Mint ragadozó, sívó oroszlán,
 Agyarkodnak, hogy engem torkukban
 Béfalhassanak.
 Könnyhullatásim, mint a vizek, folynak,
 Én csontjaim helyükből kimozdulnak.
 Szívem, mint viasz, olvad, bélim fájnak
 Sebek miatt.
/8
#5CF3A61D
 Mint cserép, minden erőm elszáradt,
 Száraz nyelvem az ínyemhez ragadt;
 Porba vetél engem, érzem kínját
 Halál mérgének.
 Mert engemet sok ebek körülvettek,
 Gonosz népek ellenem összegyűltek,
 Kezeimet és lábaimat ezek
 Általszúrták.
/9
#39AC2B62
 Én csontjaimat megolvashatnák,
 Szörnyű szemeket reám fordítnak.
 Kínomban nem szánnak, de gúnyolnak,
 Űznek csúfságot.
 Elosztották egymás közt én ruhámat,
 És öltözetemre vetettek sorsot,
 Hogy abból ne metélnének foltokat,
 Osztván részre.
/10
#4F1F57E0
 Azért tőlem, Uram, ne légy messze,
 Ne késsél, életemnek ereje!
 Kérlek, siess, tekints ínségemre:
 Légy segedelmül!
 Mentsd meg életem az éles fegyvertül,
 Védelmezz meg e sok dühös ebektül,
 Egyedül-voltomat mentsd meg ezektül,
 Jóvoltodbul!
/11
#CB3B58E6
 Tarts meg az éh oroszlán torkátul,
 És az egyszarvú fenevadaktul,
 Mik mostan körülvettek nagy mordul,
 Ó, tarts meg engem!
 Melyért nevedet híven hadd dicsérem,
 Az én atyámfiainak hirdetem,
 És a szent gyülekezetben tisztelem
 Felségedet.
/12
#40D49C67
 Istenfélők, dicsérjétek őtet,
 Jákób fiai, áldjátok nevét;
 Izráel népe, féld e Felséget
 Mint Istenedet!
 Mert nem utálja a szegénynek ügyét,
 És tőle el nem fordítja szent színét,
 De ha kiált, meghallgatja kérését
 Nagy kegyesen.
/13
#727CD18B
 Azért dicséretem rólad leszen
 Minden előtt a gyülekezetben,
 És én fogadásim semmiképpen
 Meg nem töretnek.
 A szegények esznek, megelégesznek,
 Téged az Istent keresők dicsérnek,
 Él az ő lelkük, és benned örvendnek
 Mindörökké.
/14
#E921CB20
 E földi népnek minden serege
 Az Úrhoz gyűl ez emlékezetre.
 És a pogányoknak nemzetsége
 Néki hajt fejet.
 Mert egyedül az Úr bír mindeneket,
 Övé az ország, és a pogány népet
 Bírja, rajtuk megmutatván erejét
 Kezeinek.
/15
#0EAC21F8
 A kövérek, kik megelégedtek,
 És kik már porrá lenni készültek,
 Dicsérnek téged minden szegények
 És nyomorultak,
 És maradéki mindnyájan azoknak
 Néked szolgálván, térdet-fejet hajtnak
 És firól fira téged ők uralnak,
 Ó, nagy Isten!

;A jó Pásztor
;Bourgeois L., Strasbourg, 1545
>23
/1
#B1D6CB2C
 Az Úr énnékem őriző pásztorom,
 Azért semmiben meg nem fogyatkozom.
 Gyönyörű szép mezőn engemet éltet,
 És szép kies folyóvízre legeltet;
 Lelkemet megnyugtatja szent nevében,
 És vezérl engem igaz ösvényében.
/2
#F8B1B04B
 Ha a halál árnyékában járnék is,
 De nem félnék még ő sötét völgyén is,
 Mert mindenütt te jelen vagy énvelem,
 Vessződ és botod megvigasztal engem,
 És nekem az én ellenségim ellen
 Asztalt készítesz, eledelt adsz bőven.
/3
#C9DEC6E3
 Az én fejemet megkened olajjal,
 És engemet itatsz teljes pohárral;
 Jóvoltod, kegyességed körülvészen
 És követ engem egész életemben.
 Az Úr énnékem megengedi nyilván,
 Hogy mind éltiglen lakjam ő házában.

;Isten tiszteletére hív a mindenség
;Bourgeois L., Lyon, 1547
>24
/1
#288DE7FF
 Az Úr bír ez egész földdel
 És minden benne élőkkel;
 Övé a földnek kereksége,
 Mit a tengeren épített,
 Folyóvizekkel körülvett,
 Melyben meglátszik bölcsesége.
/2
#509502A3
 Ki mégyen fel a szent helyre,
 Az Úrnak ő szent hegyére?
 És vajon kitől tiszteltetik?
 Akinek tiszta ő szíve,
 És ártatlan az ő keze,
 Aki hamisan nem esküszik.
/3
#9FFD8823
 Ezt az Úr megáldja szépen,
 És az idvezítő Isten
 Adja igazságát őnéki.
 Az pedig a boldog nemzet,
 Mely, nézve az ő szent színét,
 Jákóbnak Istenét keresi.
/4
#DC94F7D1
 Ti szent kapuk, kinyíljatok,
 Fejeteket feltartsátok:
 E dicső király hadd térjen be!
 Micsoda dicső király ez?
 A seregek Istene ez,
 Nagy ennek hadi erőssége.
/5
#0FADFB07
 Ti kapuk, emelkedjetek,
 Fejeteket felvessétek,
 Hogy e király belétek térjen!
 Kicsoda e nagy királyság?
 Ez a Zebaoth uraság!
 Mely nagy az ő dicsőségében!

;Könyörgés oltalomért, vezetésért és bűnbocsánatért
;Bourgeois L., Genf, 1551
>25
/1
#6D46C446
 Szívemet hozzád emelem,
 És benned bízom, Uram;
 És meg nem szégyeníttetem,
 Nem nevet senki rajtam,
 Mert szégyent nem vallanak,
 Akik hozzád esedeznek,
 Azok pironkodjanak,
 Akik hitetlenül élnek.
/2
#2BDF7DE6
 Útaid, Uram, mutasd meg,
 Hogy el ne tévelyedjem;
 Te ösvényidre taníts meg,
 Miken intézd menésem.
 És vezérelj engemet
 A te szent igaz Igédben;
 Oltalmazd életemet,
 Mert benned bízom, Úr Isten.
/3
#E8FB432E
 Emlékezzél jóvoltodból
 Nagy kegyelmességedre,
 Emlékezzél irgalmadról,
 Mely megmarad örökre.
 Ifjúságomnak vétkét,
 Kérlek, hogy meg ne említsed.
 Sőt nagy kegyességedet,
 Én Istenem, megtekintsed.
/4
#34A42706
 Jó és igaz az Úr Isten
 Mind örökkön-örökké,
 A bűnösöket térítvén
 Ő igaz ösvényire;
 És a nyomorultakat
 Életükben igazgatja,
 Nagy kegyesen azokat
 Az ő útában megtartja.
/5
#88E96404
 Az Istennek minden úta
 Kegyesség és nagy hűség
 Azoknak, kik mondására
 Gondot tartnak mindvégig.
 Énnékem kegyelmezz meg,
 Uram, a te szent nevedért.
 És bűnömet bocsásd meg,
 Ne ostorozz nagy voltáért!
/6
#1A52B2FF
 Aki az Úr Istent féli
 És tiszteli szívében,
 Azt ő nagy híven vezérli
 Igaz ösvényeiben.
 Nagy békességben annak
 Minden jó bőven adatik,
 És ő maradékinak
 Gazdag örökség hagyatik.
/7
#6E7E8F37
 Az igaz istenfélőknek
 Megjelenti titkait,
 És az őbenne hívőknek
 Megmutatja kötésit.
 Istenhez szemeimet
 Felemelem szüntelenül,
 Ő megőriz engemet,
 Lábam kivonssza a tőrbül.
/8
#5A58B219
 Térj azért hozzám, Istenem,
 Tekints reám kegyesen,
 És kegyelmezz meg énnékem,
 Mert élek szegénységben.
 Nyavalyája szívemnek
 Napról napra mind öregbül;
 Uram, add végét ennek,
 Végy ki engem ez ínségbül!

;Hűséges szív reménysége
;Bourgeois L., Genf, 1551
>26
/1
#57BE197D
 Légy ítélőm, Uram,
 Mert hűséggel jártam
 És éltem nagy ártatlanul!
 Azért hiszem az Istent,
 Hogy ő engemet megment
 Minden háborúságomtól.
/2
#7E5BE8E1
 Próbálj és kísérts meg,
 Ügyemet jól nézd meg:
 Meglátod tisztaságomat!
 Vizsgáld meg veséimet,
 És próbáld meg szívemet,
 Hogy értsed indulatomat.
/3
#E6048D1C
 Látom szemem előtt
 Kegyelmességedet,
 És azon vagyok szüntelen,
 Hogy minden dolgaimban
 Járjak igazságodban,
 Ne vétsek Felséged ellen.
/4
#7DCA5E72
 A hazug emberek
 Nálam nem kedvesek;
 A tettető csalárdokat
 Szívem szerint gyűlölöm,
 És nagy távol kerülöm
 Az álnoksággal járókat.
/5
#063F00AE
 A gonosztevőknek
 És hamis népeknek
 Társaságukat gyűlölöm.
 Ő gyülekezetükben
 Nem ülök semmiképpen,
 Sőt előttem sem szenvedem.
/6
#3F3CF10A
 Belső tisztaságban
 És ártatlanságban
 Én kezeimet megmosom;
 Tisztán téged dicsérlek,
 Áldozván Felségednek,
 Oltárod körül forgódom.
/7
#B966B511
 Éneki felszóval
 És víg hangossággal
 Magasztalom Felségedet;
 Néked adván hálákat,
 Hirdetvén csodáidat,
 Mindenütt áldlak tégedet.
/8
#120DC183
 Uram, hajlékodat,
 Szeretem házadat,
 Holott lakol dicsőséggel,
 Szent helyedet kedvelem,
 És azt feljebb becsülöm
 Minden e világi kincsnél.
/9
#CFBF26B7
 Ostorodat, Uram,
 Fordítsd el énrólam,
 Ne büntess a bűnösökkel!
 Vélük ne verd lelkemet,
 Ne vedd el életemet
 A vérontó emberekkel!
/10
#498D12D9
 Az én lábam megáll,
 Tágas helyet talál,
 És megmarad ösvényiben:
 Azért, Uram, dicsérlek,
 És örömmel tisztellek
 A hívek szent seregiben.

;Elég nékem az Isten kegyelme (II. Kor. 12:9)
;Bourgeois L., Genf, 1551
>27
/1
#28370D5C
 Az Úr Isten az én világosságom,
 És idvességem, hát kitől félnék?
 Ő életemnek ereje, jól tudom:
 Ki volna hát, akitől rettegnék?
 Midőn a kegyetlen gonosztévők,
 Mint ellenségim, énreám törnek,
 Hogy engemet ugyan megégyenek:
 Megbotolnak és mind elesnek ők.
/2
#F01FCC47
 Hogyha táborral körülvennének is,
 De mégsem félne semmit én szívem;
 Ha szintén az ellenség közt volnék is:
 Őbenne vetném mégis reményem.
 Egy dolgot kívántam én az Úrtól,
 Melyet még most is kérek nagy bízván:
 Hogy lakhassam az Úrnak házában,
 Míg e földön élek jóvoltából.
/3
#190629DD
 Melyet én azért kérek, hogy meglássam
 Az Úrnak felséges dicsőségét,
 És ő szent templomát látogathassam,
 Mely tisztességére építtetett.
 Mert engem hajlékába takarít
 Én háborúságimnak idein,
 És elrejt engem rejtekhelyein,
 Magas kősziklára felemelít.
/4
#2BCB0BDF
 Mert mind én atyám s anyám elhagy engem,
 De az Úr kegyesen hozzá vészen.
 Mutasd meg, Uram, te utadat nékem,
 Ellenség ellen tarts ösvényedben!
 Kívánságukra ellenségimnek
 Ne adj engemet, mert sokan vannak,
 Kik ellenem hamisságot szólnak,
 Hazudnak és erőszakot tesznek.
/5
#94B5F6BE
 Ha nem hittem volna, hogy még éltemben
 Jóvoltát az Úrnak meglátandom
 Az élőknek földén: hát immár régen
 Odalett volna minden én dolgom.
 Várjad azért bizonnyal az Urat,
 Légy víg és bátorságos szívedben,
 Mert téged megtart a nagy Úr Isten,
 Csak tőle várjad hát oltalmadat.

;Ellenség ellen
;Bourgeois L., Genf, 1551
>28
/1
#E38F9629
 Hozzád kiáltok, kegyes Uram,
 Én segítségem és kőváram!
 Hallgass meg kegyelmesen engem!
 Ne hallgass el, mert el kell vesznem!
 Azokhoz hasonló lészek,
 Kiknek a koporsó helyek!
/2
#F16406F8
 Midőn tehozzád esedezem,
 És kezeimet felemelem
 A te szentséges templomodban:
 Hallgass meg én imádságomban!
 Ne büntess a hitlenekkel,
 Ne verj a gonosztevőkkel!
/3
#0A93501D
 Áldott légyen a nagy Úr Isten,
 Ki meghallgata kérésemben!
 Az Úr énnékem erősségem,
 Én paizsom és segedelmem;
 Örvend szívem és énekben
 Dicsérem őtet szüntelen.
/4
#55618DB4
 Az Úr én népemnek ereje,
 A Krisztusnak nagy erőssége.
 Tartsd meg azért a te népedet,
 És áldjad meg örökségedet:
 Legeltessed és vigasztald,
 És örökké felmagasztald!

;A hét mennydörgés zsoltára (Jel. 10:3 skk.)
;Bourgeois L., Genf, 1551
>29
/1
#4A422230
 Mostan, ti hatalmasak,
 Tekintetes nagy urak,
 Adjatok az Istennek,
 Dicsőséget nevének!
 Mint hatalmas Istenteket;
 Féljétek, tisztelvén őtet!
 Szent templomában áldjátok,
 És térdet, fejet hajtsatok!
/2
#7A8586FF
 Az Úr szava megzendül,
 A vizeken megdördül;
 Mennydörgő dicsősége
 Elhat a nagy tengerre.
 Az Úrnak rettentő szava
 Nagy hatalmát megmutatja.
 Az Úrnak dördülő szaván
 Nagy volta meglátszik nyilván.
/3
#4D4BAA32
 De az ő templomában
 Ő hívei mindnyájan
 Hirdetik nagy erejét,
 Beszélik dicsőségét.
 Az Úr ült az özönvizen,
 Mint bíró ítélőszéken;
 Az Úrnak ő királysága,
 Örökké megáll országa.

;Hálaének a haláltól való szabadulásért
;Bourgeois L., Genf, 1551
>30
/1
#96E9F417
 Dicsérlek, Uram, tégedet,
 Mert te megtartál engemet,
 És kegyesen felemelél,
 Ellenségimtől megmentél,
 És meg nem engedéd azoknak,
 Hogy nyavalyámon vigadjanak.
/2
#D16971AD
 Hogy felkiálték tehozzád,
 Nyavalyámat meggyógyítád,
 És hogy én csaknem a sírba,
 Esném a halál torkába:
 Ismét feltámasztál engemet,
 Pokoltól megmentéd lelkemet.
/3
#ACC7195E
 Istenes hívek és szentek,
 Az Úrnak énekeljetek,
 Áldjátok őtet mindvégig,
 Mert nem haragszik sokáig,
 Az ember alig gondolhatja,
 Mily hamar elmúlik haragja.
/4
#7AB6CE94
 De az ő kegyelmessége
 Rajtunk megmarad örökre.
 Néha oly dolgom érkezik,
 Min este szívem bánkódik,
 De reggel, mihelyen felkelek,
 Azonnal víg örömet lelek.

;Bizodalmas könyörgés nagy nyomorúságban
;Bourgeois L., Genf, 1551
>31
/1
#0B3B2F81
 Uram, én csak tebenned bíztam,
 Őrizz meg kegyesen,
 Ne essem szégyenben!
 Te igazságod fordítsd hozzám,
 És tarts meg jóvoltodból,
 Ments ki nagy nyavalyámból!
/2
#41D53B2D
 Hajtsd énhozzám, Uram, füledet,
 Ó, én idvességem!
 Siess, tarts meg engem!
 Mutasd meg nagy erősségedet,
 Légy én erős kőváram,
 Melyben bátran lakhassam!
/3
#09AF05A5
 Te vagy kősziklám, erősségem:
 Szent nevedért kérlek,
 Vezérelj, hogy éljek!
 Szabadíts ki a tőrből engem,
 Melyet énnékem vetnek,
 Mert megtartómnak hiszlek!
/4
#69DFF3D0
 Lelkemet kezedbe ajánlom,
 Mert nagy ínségemben
 Megtartál, Úr Isten.
 Szívemet azoktól megvonszom,
 Akik élnek hívságban;
 És csak bízom Uramban.
/5
#7DE14197
 Örvendezek nagy vigasságban,
 Vigadok szívembe’,
 Irgalmadra nézve.
 Ha megtekintesz nyavalyámban,
 Szívemet megismered,
 Sérelmit megtekinted.
/6
#C8E9231A
 Dicsőség adassék az Úrnak,
 Ki tart oltalmában,
 Mint egy szép városban,
 Melynek erős bástyái vannak,
 Hogy aki abban lakik,
 Senkitől nem bántatik.
/7
#B356077A
 Félelmes futásomban mondék:
 Tőled elvetettél,
 Rám nem nézsz szemeddel;
 De tenálad kegyelmet lelék,
 Meghallád könyörgésem,
 Megadád, amit kértem.
/8
#80B9AE24
 Az Urat szeressétek, szentek,
 Ki megtart híveket,
 Büntet kevélyeket!
 Ő legyen néktek reménységtek!
 Higgyétek szívetekben:
 Megvigasztal az Isten.

;A bűnbocsánat útja (Második bűnbánati zsoltár)
;Bourgeois L., Lyon, 1547
>32
/1
#1F639CAA
 Ó, mely boldog az oly ember éltébe',
 Akit az Isten bevett kegyelmébe,
 És megbocsátá az ő vétkeit,
 És befedezte minden bűneit.
 Boldog, akinek ő nagy hamissága
 Istentől néki nincs tulajdonítva,
 És csalárdság nincsen ő szivében,
 Tettetés nélkül jár életében.
/2
#00415B9D
 Hogy bűnömet el akarám hallgatni,
 És neked meg nem akarám vallani,
 Csontjaim ottan elszáradának
 Soksága miatt én siralmimnak,
 Mert éjjel-nappal kezed nehéz volta
 Nagy bűneimért rajtam fekszik vala,
 Elfogya bennem minden erősség,
 Mint nyári hévségben a nedvesség.
/3
#56EB0B45
 De hogy bűnömet előtted megvallám,
 Nagy vétkeimet el nem hallgathatám,
 De híven előbeszélém neked:
 Ott bocsánatot nyerék tetőled.
 Azért az Úr Istennek minden hívek
 Könyörögjenek, míg vagyon idejek,
 Mert ha nagy árvizek jönnének is,
 De nem árthatna ezeknek mégis!
/4
#B7932201
 Te vagy oltalmam, őrizz meg engemet,
 Minden gonosz ellen tartsd meg lelkemet,
 Vigasztalj meg, hogy örvendezhessek,
 És vígan néked énekelhessek!
 Tanítlak téged, úgy mond az Úr Isten,
 És vezérellek az igaz ösvényen,
 Szememmel mindig reád vigyázok,
 És igazgatlak, rád gondot tartok.

;Isten a teremtő és gondviselő
;Bourgeois L., Strasbourg, 1545
>33
/1
#2CDD495B
 Nosza, istenfélő szent hívek,
 Örvendezzetek az Úrnak,
 Mert illik, hogy őtet dicsérjék,
 Kik örülnek igazságnak!
 Áldjátok azértan
 Hangos citerákban!
 Az Úr áldassék!
 Lantban, hegedűben,
 Cimbalmi zengésben
 Magasztaltassék.
/2
#99C51587
 Énekeljetek néki vígan
 Gyönyörű szép új éneket!
 Szép hangicsáló szerszámokban
 Mondjatok ékes verseket!
 Igaz ő mondása,
 Állhatatos dolga:
 Amit az Úr szól,
 Megáll igazságban,
 Minden dolgaiban
 Cselekeszik jól.
/3
#5ADF2ED8
 Ő szereti az igazságot,
 Az ítélet nála kedves,
 Dolgában tart irgalmasságot,
 Mellyel mind e világ teljes.
 Az Úrnak Igéje
 Egeket teremte,
 Melyeket ott fenn
 Szájának lelkével
 Nagy szép seregekkel
 Szerze ékesen.
/4
#E4CEDFDB
 Mert mihelyt ő csak egy igét szól,
 Azonnal megleszen minden,
 És valamit ő megparancsol,
 Nagy hamarsággal meglészen.
 Pogányok tanácsát
 És minden szándékát
 Az Úr megtöri,
 Magukban a népek
 Amit elvégeznek,
 Semmivé tészi.
/5
#1493DCCD
 De az Úr Istennek tanácsa
 Megmaradánd mindörökké,
 És megáll minden gondolatja,
 Kiterjed minden időkre.
 Boldog az a nemzet,
 Ki ily Urat szeret,
 Mint ő Istenét;
 Boldog a nemzetség,
 Kit kedvel e Felség
 Mint örökségét.
/6
#E089185B
 Gondot tart rájuk s a haláltól
 Megtartja őket éltükben,
 Szükségtől és éhenhalástól
 Őrzi a drága időben.
 Lelkünk azért várja,
 Szüntelen óhajtja
 Az Úr oltalmát,
 Ki paizsul végre
 Eljő segítségre,
 Ád diadalmat.
/7
#6FB0FCD2
 Őbenne azért a mi szívünk
 Igen örvendez szüntelen,
 Mert ő minékünk reménységünk,
 És bízunk ő szent nevében.
 Nagy kegyelmességed
 Mirajtunk bővítsed,
 Légy mi gyámolunk!
 Ne hagyj szükségünkben,
 Segíts meg éltünkben,
 Mert téged várunk!

;Isten megvédi az övéit
;Bourgeois L., Genf, 1551
>34
/1
#40F6EF73
 Mindenkoron áldom
 Az Urat, míg engem éltet,
 És az ő szent dicséretét
 Szájamban hordozom.
 Dicsekedvén áldja
 Lelkem jó voltát az Úrnak,
 Mit a szegények hallanak,
 És örülnek rajta.
/2
#5D915CE9
 Magasztaljuk vígan
 Az Úrnak áldott szent nevét,
 És véghetetlen kegyelmét
 Dicsérjük mindnyájan!
 Mert midőn keresém
 És kérém az én Uramat,
 Meghallgatá nagy felszómat,
 És megtarta engem.
/3
#A419E377
 Az Úrra kik néznek,
 Tőle megvigasztaltatnak,
 Soha orcáik azoknak
 Meg nem szégyenülnek.
 A szegény kiálta,
 És meghallgatá az Isten,
 El nem hagyá ínségében,
 De megszabadítá.
/4
#147EE216
 Az Isten angyali
 Hívek körül tábort járnak,
 Istenfélőket megtartnak
 Mint Isten követi.
 Jó voltát az Úrnak
 Kóstoljátok és lássátok,
 Mert bizony azok boldogok,
 Őbenne kik bíznak.
/5
#5B6BF455
 Az Isten szemei
 Látják a gonosztévőket,
 És ő emlékezetöket
 A földről eltörli.
 A jókat nem hagyja;
 Kik hozzá tiszta szívükből
 Kiáltnak nagy ínségükből:
 Mind megszabadítja.
/6
#7A2A8133
 Közel az Úr Isten
 A töredelmes szívekhez,
 És a sérelmes lelkekhez
 Lészen segítséggel.
 Az igaznak itten
 Ő nyavalyája sokasul,
 De nyomorúságaibul
 Kimenti az Isten.

;„Perelj, Uram, perlőimmel!”
;Bourgeois L., Genf, 1551
>35
/1
#E7AC8A3B
 Perelj, Uram, perlőimmel,
 Harcolj én ellenségimmel,
 Te paizsodat ragadd elő!
 Én segedelmemre állj elő!
 Dárdádat nyújtsd ki kezeddel,
 Ellenségimet kergesd el!
 Mondjad ezt az én lelkemnek:
 Tégedet én megsegítlek!
/2
#17CD284D
 Gyászban jártam lehorgadva,
 Mint ki az anyját siratja,
 De ők szomorú esetemen
 Örülnek, és gyűlnek seregben.
 Hátmögül a gonosz népek
 Engemet szörnyen nevetnek;
 Ártatlan lévén, nem szánnak,
 Sőt csúfolnak és szaggatnak.
/3
#BDBB3F23
 A képmutató galibák
 Fogukat rám csikorgatják;
 És rajtam nagy csúfságot űznek,
 Kik csak zabálódást keresnek.
 Uram, míg nézed ezeket?
 Jövel, tartsd meg én lelkemet!
 Egyedül voltom tekintsd meg,
 Ez oroszlánoktól ments meg!
/4
#9E8A9E53
 Dicsérlek téged szüntelen
 Nagy sűrű gyülekezetben,
 És nagy roppant sereg nép előtt
 Téged dicsérlek minden fölött.
 Ne engedd, hogy örüljenek,
 Akik ok nélkül gyűlölnek;
 Ellenségimet fordítsd el,
 Ne gúnyoljanak szemükkel.
/5
#88ACD56D
 Már azok énekeljenek,
 Kik igazamnak örülnek,
 Mondván: hála legyen az Úrnak,
 Ki nyugalmat ád szolgájának!
 Én nyelvem igazságodat,
 Hirdeti nagy jóvoltodat;
 Dicséretedet nagy híven
 Éneklem minden időben.

;Ember gonoszsága – Isten jósága
;Greiter M., Strasbourg, 1525 (1539) után
>36
/1
#13C7736E
 A gonosztévőknek dolgán
 Eszembe veszem azt nyilván,
 Hogy Istenre nincs gondja.
 Magában felfuvalkodik,
 Bűneitől meg nem szűnik,
 A híveket utálja.
 Hamis és hazug beszéde,
 Jó tanúsághoz nincs kedve,
 És nem jár igazsággal;
 Hívságot gondol ágyában,
 Foglalatos gonosz útban,
 Semmi bűnt ő nem utál.
/2
#A36853C1
 Uram, a te nagy hűséged
 Égig ér, kegyelmességed
 Mind a felhőkig felhat.
 Mint a hegy, te igazságod,
 Törvényed mélység, megtartod
 Az embert és a barmot.
 Te kegyességed mily drága!
 Azért a te szárnyad alá
 Emberek folyamodnak,
 Kik jól megelégíttetnek,
 Mint bő vízzel, legeltetnek
 Javaival házadnak.
/3
#5FFFA2C2
 Nálad az élet kútfeje,
 Világodnak nagy ő fénye,
 Mely nekünk szépen fénylik.
 Bővítsd rajtuk kegyességed,
 Akik jól ismernek téged,
 Szívvel neved tisztelik!
 Ne hagyd, hogy a kevély lába
 Rám rohanjon, és hatalma
 Letapodjon a földre;
 Adjad, hogy a hitetlenek
 Megessenek, süllyedjenek,
 Fel se keljenek többé!

;A gonosz szerencséje hiábavalóság
;Bourgeois L., Lyon, 1547
>37
/1
#F6CD064A
 Ne boszszankodjál a gonosztévőkre,
 Midőn őnékik jól vagyon dolguk!
 Ne nézz búskodva ő szerencséjökre,
 Ha látod nékik jó állapotjuk!
 Mert mint a szénafű, levágattatnak,
 És mint a zöld fű, hamar elhullnak.
/2
#D8748EC6
 Tégy jól és bízzál erősen Istenben:
 Békével élhetsz itt ez országban:
 Hűséggel járj el egész életedben,
 Örvendj az Úrnak nagy jóvoltában,
 És valamit kérsz tőle, mind megnyered,
 Mindent megád, amit kíván szíved.
/3
#B5C0E753
 Csak az Istenre támaszd minden dolgod,
 És kétség nélkül bízzál őhozzá,
 Mert megcselekszi, bizonnyal meglátod:
 Ártatlanságod világra hozza,
 Hogy igazságod úgyan lássa minden,
 Mint a fényes nap fénylik délszínben.
/4
#9D434420
 Bízzál az Úrban, csendes légy szívedben,
 És reménységed vessed őbenne!
 Ne haragudjál jó szerencséjeken
 Azoknak, akik élnek kedvükre!
 Ne gondolj semmit az ő életükkel,
 Hogy velük együtt bűnbe ne ess el!
/5
#CE354561
 Mert a gonoszok mind eltöröltetnek,
 De akik a nagy Istenben bíznak,
 E földnek azok örökösi lesznek:
 A gonosztévők szörnyen elhullnak.
 Majdan, ha ő helyüket megtekinted,
 Aholott laktak, üresen leled.
/6
#BE21B70C
 Élj igazán, légy hű és tökéletes,
 És nagy jól lészen tenéked dolgod,
 Békességed lészen nagy örvendetes.
 A gonoszok mind vallnak csúfságot,
 Mert ők szertelen nagy ínségbe esnek,
 És teljességgel végre elvesznek.
/7
#90E53F24
 Mert az Úr oltalma az igazaknak,
 Megmenti őket sok ínségükből,
 Vélük vagyon és tőle megtartatnak.
 És hogy őhozzá fordulnak szívből,
 A gonosztévőktől megszabadítja,
 És jelenvoltával vigasztalja.

;Lelki-testi nyomorúságban (Harmadik bűnbánati zsoltár)
;Bourgeois L., Genf, 1542
>38
/1
#56AF7B5B
 Haragodnak nagy voltában
 Megindulván,
 Ne feddj meg, Uram, engem!
 Búsult gerjedezésedben
 Rám tekintvén,
 Ne büntess meg Istenem!
/2
#B24C9E7A
 Nyilaid belém lövettek,
 Mik szereznek
 Énnekem nagy sérelmet;
 Kezeidnek súlyossága
 Hátam nyomja,
 És sanyargat engemet.
/3
#E4D26487
 Testemnek semmi részében
 Épség nincsen
 Haragodnak miatta;
 Nincs békesség tagjaimban,
 Csontjaimban
 Bűneimnek miatta.
/4
#9B641A1D
 Mert az én nagy gyarlóságim
 És bűneim
 Fejem felülhaladták,
 Miknek nehéz, terhes voltát,
 Súlyosságát
 Tagjaim nem bírhatják.
/5
#6701F619
 Minden mostani kérésem,
 Én Istenem,
 Vagyon szemeid előtt,
 És minden fohászkodásom,
 Óhajtásom
 Tőled el nem rejtetett.
/6
#240470E0
 Szívem nyugalmat nem lelhet,
 Igen reszket,
 Minden erőm elfogyott;
 Szemeim világossága,
 Vidámsága
 Éntőlem eltávozott.
/7
#6EA14500
 De én Istenemben bízom,
 És elvárom,
 Hogy kérésem meghallja,
 Mert szívem hozzá emelem
 És elhiszem,
 Hogy szükségem meglátja.
/8
#B0A3ED45
 Uram, ne hagyj el engemet!
 Nézd ügyemet,
 Egyedül mint hagyattam!
 Kérlek, légy irgalmas nekem,
 Én Istenem,
 Mert csak tebenned bíztam!
/9
#3FDD81DF
 Azért tőlem ne állj messze,
 Szánj meg végre,
 Én kegyelmes Istenem!
 Segedelmeddel ne késsél,
 Siess, jöjj el,
 Én édes idvességem!

;Imádság a lélek nyugalmáért
;Bourgeois L., Genf, 1551
>39
/1
#9C39661C
 Magamban elvégezém, és mondám:
 Hogy dolgom megtartóztatnám,
 Hogy nyelvem oly igét nem ejtene,
 Mely énnekem bút szerzene,
 Én szájamra zabolát vetettem,
 Míg a hitlen áll előttem.
/2
#E4164ABA
 Én mint a néma, veszteg hallgaték,
 Még a jóról sem beszélék,
 Sőt fájdalmam is titkolnom kelle,
 Min sérelmem öregbüle;
 Ég vala szívem, hogy meggondolám,
 Eltüzesülvén ezt mondám:
/3
#C11B167B
 Mutasd meg, Uram, éltemnek végét,
 És meddig éltetsz engemet?
 A napok számát jelentsd meg nekem!
 Míg e világon kell élnem,
 Mert időm nálad csak egy arasznyi,
 Előtted életem semmi.
/4
#A398F13E
 Bizony mulandó semmi az ember,
 Ki magának sokat ígér!
 Mint az árnyék, az ember elmúlik,
 Mégis szorgalmatoskodik,
 Sokat gyűjt és sok kincset rak össze,
 Nem tudja, kié lesz végre.
/5
#33C04A61
 Uram, hát nékem kiben kell bíznom?
 Nincs kívüled vigasságom!
 Ments ki engemet minden vétkemből,
 És a bolondok nyelvétől
 Őrizz meg, hogy ők ne csúfoljanak,
 Midőn ez ínségben látnak.
/6
#01C24592
 Mint a néma, hallgatok erősen,
 Szájam fel sem nyitom, mert én
 Tudom, hogy ezt mind te cselekedted.
 Ostorod rólam elvegyed!
 Mert kemény kezed nagy volta miatt
 Minden életem ellankadt.
/7
#14CC12FF
 Mert midőn te megfedded az embert
 Az ő nagy gyarlóságáért,
 Azonnal elvész szép ábrázatja,
 Mint a molytól a szép ruha.
 Lám, az ember mely igen mulandó,
 Semmi dolga nem állandó.
/8
#DEE3DAEA
 Hallgasd meg, Uram, könyörgésemet,
 Kérésemre ne légy siket!
 Mert előtted vendég és zarándok,
 Mint atyáim, olyan vagyok.
 Szűnjél meg tőlem, hadd vegyek erőt
 Az én kimúlásom előtt!

;A hit nemes harca
;Bourgeois L., Genf, 1551
>40
/1
#861E3D29
 Várván vártam a felséges Urat,
 És íme, hozzám fordula,
 Kegyelmesen meghallgata,
 És rajtam megmutatá jó voltát.
 Kivőn a mély veremből,
 És a sáros fertőből,
 És én lábaimat
 Szép egyenes kőre
 Elfelhelyeztette,
 Vezérlvén utamat.
/2
#D1039476
 Ada én számba új énekeket Istenünk dicséretire,
 Hogy akik hallgatnak erre,
 Higgyék és féljék ő Istenüket.
 Boldog, aki az Úrban
 Bízik, szemét elhajtván
 A kevély népektül,
 Kiknek minden dolgok
 Hazugságra hajlók:
 Tőlük távol kerül.
/3
#3FAB8441
 Csudatételidnek sok ő száma,
 És nagy bölcs gondolatidnak,
 Hozzánk való jóvoltodnak
 Sokságát senki meg nem mondhatja.
 Ha elkezdem számlálni,
 Nem tudom kimondani,
 Mert te nem kívántad
 A sok áldozatot,
 De hogy fogadjak szót,
 Fülemet alkottad.
/4
#F1787A69
 Égő áldozat nincsen kedvedben,
 A bűnért valók sem kellők.
 Akkor mondom: ím, eljövök,
 Rólam írás van a törvénykönyvben:
 Hogy akaratod tegyem,
 Én kegyelmes Istenem!
 A te törvényedben
 Gyönyörködik lelkem
 És örvendez szívem
 A te szent igédben.
/5
#1036D2D9
 Már tebenned mind örvendezzenek,
 Akik keresnek tégedet,
 Kívánják idvességedet,
 Mondván: dicsőség légyen Istennek!
 Noha én szegény vagyok,
 És én szükségim nagyok,
 De rám gondot visel
 Az Úr, én megtartóm,
 Jövel, szabadítóm,
 Úr Isten, ne késsél!

;Hűtelen barátok ellen
;Bourgeois L., Genf, 1551
>41
/1
#851C2787
 Boldog, aki a nyavalyást híven
 Szánja ínségében,
 Mert szükségében őtet ismétlen
 Megmenti az Isten,
 Megtartja éltét, és ez országba'
 Lesz jó állapotja,
 Ellenséginek kívánságába,
 Nem adja markába.
/2
#D2FEEEAB
 Fájdalmában az Isten megtartja,
 Szépen felgyógyítja;
 Betegágyát fordítja örömre,
 És jó egészségre.
 Azért így szólok neked, Istenem:
 Kegyelmezz meg nekem!
 Gyógyítsd meg, Uram, én betegségem,
 Mert igen vétkeztem.
/3
#1EAC313F
 Tisztaságomban engem megtartasz,
 És megszabadítasz,
 És szemeid eleibe állatsz,
 Örökké el nem hagysz.
 Áldott légy, Izráelnek Istene,
 Most és mindörökké!
 Szent neved dicsértessék mindenben,
 Ámen és úgy légyen!

;Óhajtozás Isten után
;Bourgeois L., Genf, 1551
>42
/1
#A051EAB4
 Mint a szép, híves patakra
 A szarvas kívánkozik,
 Lelkem úgy óhajt Uramra,
 És hozzá fohászkodik,
 Tehozzád, én Istenem,
 Szomjúhozik én lelkem,
 Vajon színed eleiben
 Mikor jutok, élő Isten?
/2
#D4F8F93E
 Könnyhullatásim énnékem
 Kenyerem éjjel-nappal,
 Midőn azt kérdik éntőlem:
 Hol Istened, kit vártál?
 Ezen lelkem kiontom,
 És házadat óhajtom,
 Hol a hívek seregében
 Örvendek szép éneklésben.
/3
#13B71411
 Én lelkem, mire csüggedsz el?
 Mit kesergesz ennyire?
 Bízzál Istenben, nem hágy el,
 Kiben örvendek végre,
 Midőn hozzám orcáját,
 Nyújtja szabadítását;
 Ó, én kegyelmes Istenem,
 Mely igen kesereg lelkem!
/4
#1EA998C8
 Sebessége árvizednek,
 És a nagy zúgó habok
 Énrajtam összeütköznek,
 Mégis hozzád óhajtok;
 Mert úgy megtartasz nappal,
 Hogy éjjel vigassággal
 Dicséreteket éneklek
 Néked, erős őrizőmnek.
/5
#34B99EB0
 Mondván: Isten, én kőszálam,
 Mire felejtesz így el?
 Ellenségim vannak rajtam,
 Gyászban járok veszéllyel.
 Mert az ő hamis nyelvek
 Csontjaimban megsértnek,
 Mert így bosszantnak ellened:
 Lássuk, hol vagyon Istened?
/6
#26BFFA9A
 Én lelkem, mire csüggedsz el:
 Mit kesergesz ennyire?
 Bízzál Istenben s nem hágy el,
 Kiben örvendek végre.
 Ki nekem szemlátomást
 Nyújt kedves szabadulást,
 Nyilván megmutatja nekem,
 Hogy csak ő az én Istenem.

;(Folytatás)
;Bourgeois L., Lyon, 1547
>43
/1
#09B78A49
 Ítélj meg engemet, Úr Isten,
 És fogadd fel én ügyemet
 E kegyetlen nemzetség ellen!
 A hamis embernek kezében
 Ne bocsáss, Uram, engemet,
 Tartsd meg én fejemet!
/2
#810486AB
 Uram, engem miért hagyál el?
 Lám, te vagy én erősségem!
 Miért járok keserűséggel?
 Minden örömöm távozék el,
 Mert nyomorgat ellenségem
 És sanyargat engem.
/3
#DEACEECC
 Igazságodat add értenem,
 Világosságod küldd alá,
 Mely megvilágosítson engem!
 Szent hegyedre légyen vezérem;
 Bémenésem igazgassa
 A te hajlékodba!
/4
#F0AEEFC9
 Isten oltárához bemégyek
 Az én Uram eleiben,
 Aki öröme én szívemnek.
 Hegedűvel neked éneklek,
 És hálát adok szüntelen
 Tenéked, Úr Isten.
/5
#2A4339C3
 Miért vagy szomorú, én lelkem?
 Mit kesergesz ily szertelen?
 Bízzál az Istenben, mert hiszem,
 Hogy őtet én még dicsőítem,
 Midőn híven megment engem
 Megváltó Istenem.

;Ne taszíts el örökre!
;Bourgeois L., Genf, 1551
>44
/1
#099C9997
 Hallottuk, Isten, füleinkkel,
 Amit régenten cselekedtél,
 Nékünk atyáink mondották,
 Kik nagy dolgaidat látták;
 A pogány népet kezeddel
 Elvesztéd, földét elpusztítád;
 Néped más helyre vitted el,
 Holott ismét megszaporítád.
/2
#65BDC62F
 Mert nem ő fegyverük által lett,
 Hogy ők megülték e jó földet;
 Nem az ő kezük, sem karjuk
 Volt nekik szabadítójuk,
 De te orcád tekintése
 És a te karod és jobb kezed őket így megsegítette,
 Mert őhozzájuk volt jó kedved.
/3
#30D9BAF5
 Úr Isten, te vagy én királyom
 És az én teljes vigasságom!
 Jákóbnak küldd segedelmed,
 Amint régenten mívelted.
 Általad ellenséginket
 Megökleldezzük és megrontjuk,
 És a mi gyűlölőinket
 A te nevedben letapodjuk.
/4
#B90BB2C8
 Mert én nem bízom kézívemben,
 Sem az én éles fegyveremben;
 Az engem meg nem szabadít,
 Ha az ellenség megszorít,
 De te tartasz meg bennünket
 Minden mi ellenségünk ellen,
 És a mi kergetőinket
 Elveszted, és ejted szégyenben.
/5
#E30D8F89
 Azért az Istent magasztaljuk,
 És szent nevét örökké áldjuk;
 Mindennap dicsérvén őtet,
 Hirdessük nagy kegyességét!
 De minket te megvetettél,
 És juttattál nagy szégyenségben,
 A hadba velünk nem jövél,
 Hogy megtartottál volna épen.
/6
#69CE2EBF
 De mi teéretted naponkint,
 Üldöztetünk e világ szerint;
 Miként az ártatlan juhok,
 Kik a mészárszékre valók.
 Kelj fel azért, mit aluszol?
 És álmodból már serkenjél föl,
 Támadj föl, és hatalmadból
 Ments ki minket e nagy ínségből!

;Az Isten Felkentjének menyegzője (dicsősége)
;Bourgeois L., Genf, 1551
>45
/1
#EF5B6C1E
 Egy szép dolgot hoz elő az én szívem,
 A dicső királyról lesz éneklésem,
 Kit nyelvem dicsér nagy szép felszóval,
 Mint egy deák az írópennával.
 Sokkal szebb vagy te minden embereknél,
 Mindent felülmúlsz te szép termeteddel,
 Ajakidnak nagy ő kedvessége,
 Mert megáldott az Isten örökké.
/2
#78E5874F
 Te kegyes, erős vitéz, készüljél fel,
 Vedd fegyvered és oldaladra kösd fel,
 Úgy, mint királyi ékességedet,
 Ez ékességben végy győzedelmet!
 A jó igazság vezérlje utadat,
 Jóság, kegyesség bírja járásodat,
 És a te karodnak erejével
 Nagy csudákat téssz, megládd szemeddel!
/3
#23C4D9F9
 Mert hegyesek a te sebes nyilaid,
 Mit megéreznek minden ellenségid,
 Szívüket midőn általszegezed,
 És őket hatalmad alá veted.
 Ó, Úr Isten, a te királyi széked,
 Megmarad mindörökké dicsőséged;
 Pálcája a te királyságodnak:
 Pálcája bizony igazságodnak.
/4
#C9EA9C2C
 Az igazságot te igen szereted,
 A hamisságot viszontag gyűlölöd,
 Azért az Isten mindenek felett
 Víg olajjal megkenett tégedet.
 A te nevedet mindenkor hirdetem,
 Nemzetségről nemzetségre beszélem:
 És minden maradéki ezeknek
 Mindörökké tégedet dicsérnek.

;Erős vár a mi Istenünk!
;Bourgeois L., Genf, 1551
>46
/1
#8EE5EC54
 Az Isten a mi reménységünk,
 Midőn reánk tör ellenségünk,
 Minden háborúságinkban
 Megtart erős hatalmában.
 Azért a mi szívünk nem félne,
 Ha az egész föld megrendülne,
 És a hegyek a tengernek
 Közepibe bedűlnének.
/2
#6D020AF2
 Ha a tenger szörnyen zúgna is,
 Minden víz felzavarodnék is,
 És ha a sebes szélvésszel
 A hegyek hányatnak széjjel:
 A szép folyóvíz mindazáltal
 Az ő szép tiszta folyásával
 Az Istennek szent városát,
 Megvigasztalná hajlékát:
/3
#B5D5FB12
 Mert közepén lakik az Isten,
 Azértan romlása nem lészen;
 Semmi ínségbe nem ejti,
 Az Isten jókor megmenti.
 A pogány népek dúlnak-fúlnak,
 Nagy sok országok feltámadnak,
 De az ő haragos szava
 Mind e földet elolvasztja.
/4
#D4FBFAB1
 De az Isten minden időben
 Mivelünk vagyon ínségünkben;
 Jákób Istene oltalmunk,
 A Zebaóth erős várunk!
 Jertek, lássátok e nagy Úrnak,
 Csuda dolgait hatalmának,
 Ki mind e föld kerekségét,
 Elpusztítja ékességét!
/5
#36F2E12C
 E földön széjjel nagy hadakat,
 Ő megcsendesít háborúkat,
 Ívet, kopjákat megrontat,
 Társzekereket felgyújtat,
 Így szólván: mindnyájan halljátok,
 Hogy erős Istenetek vagyok,
 És hogy birodalmam vagyon
 Minden népen e világon!

;Isten a világ királya
;Bourgeois L., Genf, 1551
>47
/1
#0A93251D
 No, minden népek,
 Örvendezzetek
 És tapsoljatok,
 Istent áldjátok
 Szép hangossággal
 És nagy felszóval,
 Mert az Úr Isten
 Nagy felségében
 Ő királysága
 És nagy országa
 Kiterjed meszsze,
 Ez egész földre.
/2
#5F011948
 Hatalmunkba vet
 Nagy sok népeket,
 És meghódoltat sok pogányokat
 Minékünk végre
 Ő nagy ereje.
 Minket ő felvett,
 Örökévé tett,
 Nékünk engedte
 Ő kegyessége
 Jákóbnak tisztét,
 Kit igen szeret.
/3
#3D6DE3E4
 Íme, az Isten
 Szépen felmegyen
 Nagy vigasságban,
 Trombitaszókban;
 Urunk felmegyen
 Nagy dicsőségben.
 Énekeljetek
 Az Úr Istennek
 Zengő verseket,
 Szép énekeket,
 E nagy királynak,
 Mint mi Urunknak!
/4
#C78C173A
 Őnéki minden
 Jól énekeljen!
 Mert minden népek
 Néki engednek.
 Királyi székben
 Ül nagy kegyesen.
 A fejedelmek
 Őhozzá gyűlnek,
 Mint Ábrahámnak
 Istenét, áldják,
 Alázatosan
 Néki szolgálván.

;Sion zarándokainak éneke
;Genf, 1562
>48
/1
#A1A99441
 Nagy az Úr méltóságában,
 Az Isten szent városában,
 Holott lakozik dicsősége,
 És dicsértetik ő szent neve.
 A szent Sionnak hegyén,
 Annak északi szélén,
 E gyönyörűséges helyen,
 Holott helyheztetett szépen
 A nagy királynak városa,
 Melynek nincs e földön mása.
/2
#A9FAC885
 Mit megerősített Isten,
 Hogy romlása ne lehessen.
 Nézünk itt te kegyességedre
 Szent templomodnak közepette.
 Dicsőséges szent neved
 Mind e földre kiterjed;
 Ekképp a te dicséreted
 Nagy messzire kiterjeszted;
 Jobbod teljes igazsággal,
 Mindenben irgalmassággal.
/3
#679EE223
 Örül a Sionnak hegyi
 És a Judának leányi,
 Te igaz ítéleteiden
 Szíve szerint örvendez minden.
 A Siont járjátok meg,
 Tornyait lássátok meg,
 Nézzétek meg ő bástyáit,
 Szépen épített házait,
 Hogy ezt megbeszélhessétek
 A jövendő nemzedéknek.

;A gazdagságban bízni balgaság
;Genf, 1562
>49
/1
#FF00AA3E
 Hallgassátok meg, minden nemzetek,
 E föld lakói, jól figyeljetek,
 Köznépek és a főrenden valók,
 Minden szegények és a gazdagok!
 Az én szájam szól nagy bölcses-séget,
 És elmém gondol jó értelmeket;
 E példára magam is figyelmezek,
 Hegedűszóban szép mesét jelentek.
/2
#392C6150
 Mit félnék én a gonosz időben,
 Hogy nyomorgatóm engem elejtsen,
 Ha ellenségem mind azon vagyon,
 Hogy engemet láb alá tapodjon?
 Némelyek igen bíznak pénzükben,
 És dicsekednek az ő kincsükben;
 De senki meg nem váltja atyjafiát,
 Nem adhatja meg Istennél váltságát.
/3
#876D6902
 Mert drága a léleknek váltsága,
 Életét senki meg nem válthatja,
 Hogy a halált ő elkerülhesse,
 És a sírba menni ne kellene.
 Mert látja, hogy sem bolond, sem eszes,
 A halál ellen nincsen mentséges,
 És holtuk után minden gazdagságuk,
 Más emberre száll az ő sok jószáguk.
/4
#4E370F85
 Ez őnékiek fő gondolatjuk,
 Hogy mindörökké tartson szép házuk,
 Hogy el ne fogyjon az ő nemzetük,
 És megmaradjon örökké nevük.
 Mert noha vagyon pénzük és tisztük,
 De nem sokáig tartnak mindezek,
 Mert végre minden jóktól elszakadnak,
 Mint oktalan állatok, ők meghalnak.
/5
#B22CFE30
 Mert oda lészen minden ő dolguk,
 A koporsó lészen ő hajlékuk,
 De a haláltól engem az Isten
 Megment, és hozzá vészen kegyesen.
 Annak okáért azzal ne gondolj,
 Hogy némely ember igen gazdagul,
 Mert minden kincset másoknak kell hagyni,
 A dicsőségtől meg kell néki válni.

;Az igazi hálaáldozat
;Bourgeois L., Lyon, 1547
>50
/1
#D0A90AA8
 Az erős Isten, uraknak Ura,
 Szól és e földet mind elő hívja,
 Támadatról és napenyészetről,
 Nagy szépséggel a Sion hegyéről
 Eljő az Isten ő fényességében,
 Semmit el nem hallgat ítéletében.
/2
#9FC8BD4F
 Emésztő tűz mégyen őelőtte,
 Nagy forgó szélvész lészen körüle;
 Szólítja az eget és a földet,
 Hogy megítélje minden ő népét,
 Mondván: gyűjtsétek ide a híveket,
 Kik áldozattal vették kötésemet!
/3
#6E9D95DE
 Az egek hirdetik igazságát
 Az Istennek, mert ítél igazat.
 Én népem, hallgasd meg, szólok neked,
 Mert bizonyságot teszek ellened,
 Én tenéked erős Istened vagyok,
 Áldozatiddal keveset gondolok.
/4
#61907999
 Szükségben tőlem segítséget kérj,
 Én megsegítlek, hogy engem dicsérj!
 A gonosznak mond Isten: de mire
 Vészed törvényem a te nyelvedre?
 Kötésem szájjal vallod, de gyűlölöd
 Intésemet, és igémet megveted.
/5
#8C823269
 És mikoron a lopót te látod,
 Együtt futsz véle, dolgát javallod;
 A paráznákkal örömest mulatsz,
 Rossz társaságnak gyakran helyet adsz;
 Szájaddal szólasz nagy sok gonoszságot,
 Nyelveddel szerzesz sok háborúságot.
/6
#17131811
 Leülsz és atyádfiát megszólod,
 Az anyád fiát is rágalmazod;
 Ezt míveled, de én csak hallgatok.
 Azt véled, én is csak olyan vagyok,
 Mint szintén te, de téged előveszlek,
 És szemlátomást bűnödről megfeddlek.
/7
#0946402A
 Ezt mostan eszetekbe vegyétek,
 Kik az Istenről elfelejtkeztek,
 Hogy mentség nélkül el ne rántsalak!
 Az becsül engem, aki hálát ad;
 Mond Isten: az ily ember jár jó úton,
 És segedelmem néki jelen vagyon!

;Dávid bűnbánati imádsága (Negyedik bűnbánati zsoltár)
;Bourgeois L., Genf, 155
>51
/1
#2F2524A9
 Úr Isten, kérlek, kegyelmezz nékem,
 És kegyelmedből könyörülj énrajtam,
 Bűneimből tőled megtisztíttassam,
 Nagy irgalmaddal végy körül engem!
 Törüld el az én nagy bűneimet,
 Mosogass jól meg fertelmességimből,
 Amikkel fertéztetem éltemet:
 Tisztítsad el nagy kegyelmességedből!
/2
#D77F2C91
 Mert ismerem jól gyarló voltomat,
 És bűnöm mindenkor forog előttem,
 Melyet csak teellened cselekedtem,
 Mely miatt tészek én nagy siralmat.
 Vétkeztem a te szemeid előtt,
 Melyért engemet méltán megbüntethetsz,
 Rám vethedd kemény ítéletedet,
 Mindazonáltal igazán ítélhetsz.
/3
#80AECB8C
 Mert íme, látom, nyilván jól értem,
 Hogy én a gyarlóságban fogantattam,
 Vétekben az én anyámtól származtam,
 Szüleim bűnös véréből lettem.
 De te az igazságot szereted,
 És igen kedveled a tiszta szívet,
 És a te titkos bölcsességedet
 Nagy kegyelmesen nékem megjelented.
/4
#EB2A8407
 Hints meg, Uram, engemet izsóppal,
 És én azontúl tisztán megújulok;
 Moss meg engem, és szépen megtisztulok,
 És fejérb lészek a tiszta hónál.
 Hogy örvendezhessen az én szívem,
 Add, hogy megértsem te nagy irgalmadat,
 És megvidul minden én tetemem,
 A te haragod melyet összerontott.
/5
#06930EFE
 Színedet rejtsd el vétkeim elől,
 Fertelmes bűneimet ne tekintsed,
 Haragos orcád énrólam elvégyed,
 Tisztíts meg engem minden bűnömből!
 És teremts tiszta szívet énbennem,
 Dolgomat mindig jóra vezéreljed,
 Az erős lelket újítsd meg bennem,
 Hogy kedves légyen életem előtted!
/6
#1495DD13
 Ne vess el engem szent színed elől,
 És ne vedd el szent Lelkedet éntőlem,
 Sőt szerezz teljes örömet énbennem,
 Tégy bizonyossá engedelmedről!
 Indíts szívemben nagy vigasságot,
 És engem vidám lélekkel erősíts,
 Értsem örömmel nagy irgalmadat,
 Kegyelmességgel engemet bátoríts!
/7
#E56D6E84
 És példa lészek erről mindennek,
 A bűnösöket utadra tanítom,
 Bűnükből hozzád térésre indítom,
 Hogy csak tebenned reménykedjenek.
 Ó, én Istenem, én idvességem,
 Szabadíts meg e vérrel buzgó bűntől,
 Hadd énekeljen örökké nyelvem
 Te igazságos szent ítéletedről!
/8
#E90A43F1
 Nyisd meg azért az én ajakimat,
 Hogy az én szájam dicsérhessen téged!
 Ha az áldozat kedves volna néked,
 Nem kíméleném áldozatomat,
 De nem kell néked égő áldozat:
 Az alázatos lelket te szereted,
 Az tenálad a kedves áldozat;
 A töredelmes szívet meg nem veted.

;Jaj a gonosz nyelvnek!
;Genf, 1562
>52
/1
#5C91FD6D
 Mit dicsekedel gonoszságban,
 Te hatalmaskodó?
 Mit fuvalkodol fel magadban,
 Nagyravágyakozó?
 Mert az Istennek jóvolta
 A jókat megtartja.
/2
#738DA62F
 Én téged örökké dicsérlek,
 Mert megtartál engem,
 És a te nevedben remélek,
 Míg leszen életem,
 Mert jó a te híveiddel
 Lakoznom örömmel.

;A balgák megbűnhödnek
;Bourgeois L., Genf, 1542
>53
/1
#A4226945
 A bolond így szól az ő szívében:
 'Nincs Isten', azért nagy gonoszságban él;
 Utálatos bűnt teszen, semmit nem fél.
 Ez egész földön aki jót tegyen,
 Senki nincsen.
/2
#B5604214
 Az Úr az égből alátekinte
 E földön az emberek fiaira,
 Hogy meglássa, ha kinek esze volna,
 Ha valaki az Istent keresné
 És tisztelné.
/3
#25A23001
 De azt jól látja dicsőségében,
 Hogy a jó útról eltértek mindenek,
 Mindnyájan fertelmes bűnben hevernek;
 Aki az Istent tisztelné híven,
 Csak egy sincsen.
/4
#F5D913DC
 A Sionról vajon ki jövend el,
 Ki a szent Izráelt megszabadítja?
 Ha Isten fogságból népét kihozza,
 Örvend a Jákób és az Izráel
 Teljes szívvel.

;A hit diadalma
;Genf, 1562
>54
/1
#0519C422
 Tarts meg, Uram, én Istenem,
 És szent nevedért védelmezz meg,
 Ártatlan ügyemet tekintsd meg,
 Hatalmaddal támadj mellém!
 Kérem te szent Felségedet,
 Hallgass meg én könyörgésembe'
 És kegyesen vedd füleidbe
 Az én szájamnak beszédét!
/2
#C72FD49E
 Mert ellenségim kevélyen
 Reám támadnak és kergetnek,
 És engem halálra keresnek,
 Eszükbe sem jut az Isten.
 De az Isten megtart engem,
 Kegyelmességét megmutatja,
 És segedelmét hozzám nyújtja,
 Jóra vezérli életem.
/3
#B1B67519
 Ez nékem szánt nagy nyavalyát
 Én ellenségimre téríti,
 És ezt nékiek megfizeti,
 Megmutatván igazságát.
 Akkoron neked víg szívvel
 Hálaadásokat áldozom,
 És szent nevedet magasztalom,
 Mert teljes vagy kegyességgel.

;Könyörgés hamis atyafiak ellen
;Genf, 1562
>55
/1
#9025397C
 Hallgasd meg az én könyörgésem,
 Úr Isten, ne fordulj el tőlem,
 Imádságom vedd füleidbe,
 Mert nagy kínokat szenvedek,
 Szívemben igen kesergek,
 Előtted panaszlok reszketve!
/2
#8F10C8BC
 Kesereg szívem nagy ínségben,
 Élek halálos félelemben,
 Teljességgel elszomorodtam;
 Rettegek, félek, gyötrődöm,
 Reszketvén szörnyen vesződöm,
 Úgy, hogy immár sokszor kívántam:
/3
#D61097E3
 Szárnyaim, ó, ha lehetnének,
 Mint a galamb, ha repülhetnék:
 Én elrepülnék messze földre,
 Elmennék e népek közül.
 Pusztát keresnék, ezektül
 Ahol nyugodalmam lehetne.
/4
#8FB9BEE6
 Mind este, reggel őt óhajtom,
 Délkorban is őtet kiáltom,
 És meghallgatja könyörgésem,
 Megtart engem békességben
 Minden ellenségim ellen,
 Kik seregbe gyűltek ellenem.

;Könyörgés üldöztetésben
;Genf, 1562
>56
/1
#778C22FC
 Kegyelmezz meg nekem, én Istenem,
 Mert az ember igen kerget engem,
 Nagy hatalommal támad ellenem,
 Hogy engem elejthessen.
 Sok ellenségim háborgatnak szörnyen,
 Elszánták, hogy bényeljenek hirtelen,
 Úr Isten, én ilyen nagy félelmemben
 Benned reménységem!
/2
#A7C7868D
 Én az Úr Istenben dicsekedem,
 Szent Igéjében nem kételkedem.
 Mit tehetne az ember énnekem,
 Aki kerget engemet?
 Visszafordítják az én beszédemet,
 És naponkint abban hányják eszüket,
 Hogy nékem szerezzenek veszedelmet,
 Csak gonoszt gondolnak.
/3
#29F46254
 Énellenem ők összejárulnak,
 Hogy megkapjanak, azon forgódnak,
 És életemtől megfoszthassanak:
 Ez minden ő szándékuk.
 A gonoszságban nagy ő bizodalmuk,
 Azt vélik, hogy jól leszen minden dolguk,
 De haragod ha esik, Uram, rajtuk:
 Őket mind levered.
/4
#685BCC69
 Minden futásimat megemlíted,
 Könnyhullatásim tömlődbe szeded:
 A te könyveidbe fel is jegyzed
 Minden én ínségemet.
 Könyörgésemet midőn hozzád nyújtom,
 Ellenségimet ottan futni látom,
 Mert velem vagy, és te vagy megtartásom,
 Kegyelmes Istenem!
/5
#6C211AB3
 Úr Isten, Fölségedet dicsérem,
 És szent igédet nagynak becsülöm;
 Áldom az Urat, míg lesz életem,
 Bízván ő beszédében.
 Reménységemet vetem az Istenben,
 Irgalmasságára nézek szüntelen,
 Azért félelmem senkitől nem leszen:
 Ki árthatna nékem?
/6
#F4E6C2A1
 Szent fogadásom tartja azt nékem,
 Hogy jóvoltodért neved dicsérjem,
 Mert kegyelmesen megmentéd lelkem
 A halál köteléből.
 És lábaimat megtartád eséstől,
 Hogy én élhessek tenéked kedvesül;
 Az élők fényes világában szentül
 Járjak teelőtted.

;Bizodalom veszedelemben
;Genf, 1562
>57
/1
#0EE4ABD2
 Irgalmazz, Uram, irgalmazz nekem,
 Mert tebenned bízik az én lelkem!
 Több segedelmem sehol nincsen nékem!
 Szárnyadnak árnyékában védelmem,
 Míg veszedelmem eltávozik tőlem.
/2
#DBBAA69A
 Én a felséges Istenben bízom,
 Ki jóra vezérli minden dolgom;
 Segedelmet küld alá ínségemben.
 Az ellen lészen nékem oltalmam,
 Aki engem kerget, hogy benyelhessen.
/3
#4B7BE0A6
 Én nyelvem és lantom már serkenj föl,
 Véletek ím a hajnalt keltem föl!
 Jó reggel én is ágyamból fölkelek,
 És éneklek nagy dicsőségedről.
 Téged, Uram, minden népnek hirdetlek!
/4
#420842EF
 Mert nagy jóvoltod, örök irgalmad
 És igazságod az égig fölhat!
 Nagy dicsőséged láttasd meg az égen,
 És jelentsd meg nagy hatalmasságod
 Az embereknek mind az egész földön!

;Igaztalan ítélkezés ellen
;Genf, 1562
>58
/1
#59711B76
 Ti tanácsban ülő személyek,
 Kik zúgolódtok ellenem:
 Mondjátok csak meg énnekem,
 Ha igaz e, amit beszéltek,
 És ha igazat ítéltek,
 Ádámtól származott népek?
/2
#6EEBAF11
 Sőt ha ember jól megtekinti:
 Álnoksággal jár szívetek,
 Mér hamis fonttal kezetek;
 Nincs dolgotoknak tökéleti.
 A gonoszok eltévedtek,
 Mihelyt anyjuktól születtek.
/3
#970EE41B
 Végre is azt mondhatja minden,
 Hogy jó az igaznak dolga,
 Sok és nagy az ő jutalma.
 Azt vegye mindenki eszében,
 Hogy Isten mindent megítél,
 Aki gonoszul vagy jól él.

;Könyörgés szorongattatásban
;Genf, 1562
>59
/1
#F1952ACD
 Szabadíts meg engem, Úr Isten,
 És tarts meg ellenségim ellen!
 Mentsd meg azoktól életem,
 Kik feltámadtak ellenem!
 Oltalmazz meg a hamis néptül,
 Ki én veszedelmemnek örül
 És szomjúzik ártatlan vért!
 Ments meg attól szent nevedért!
/2
#24D64F18
 Én ellenségimnek ereje,
 Kezedben minden tehetsége.
 Benned bízom, én Istenem,
 Mert te vagy én segedelmem.
 Az Isten jóvoltát jelenti,
 Nyavalyámnak elejét veszi,
 És megláttatja énvélem,
 Hogy elvész én ellenségem.
/3
#AD58A13E
 Mert te vagy, Uram, én oltalmam,
 Reménységem és bizodalmam:
 Azért, ó, én erősségem,
 Mindenütt neved dicsérem,
 Hogy nekem az én szükségemben,
 Segedelmem vagy ínségemben;
 Te vagy én erős kőváram,
 Kegyességed nagy énhozzám.

;Imádság nehéz időkben
;Bourgeois L., Genf, 1562
>60
/1
#627E9FAF
 Minket, Úr Isten, elhagyál,
 És reánk megharagudtál,
 Tőled széjjel eloszlatánk,
 Térj kegyesen ismét hozzánk!
 Mind e földet megindítád,
 Nagy hatalmaddal megrontád,
 Építsd meg az ő nagy romlását,
 Mert romlás miatt alig állhat.
/2
#DD75CDFA
 Népedet keményen tartád,
 És igen megsanyargatád;
 Minket mintegy csípős borral,
 Itatál keserű búval,
 De akik tégedet félnek,
 A zászlót adád nékiek,
 Hogy fölemelnék igazságban,
 Bízván a te szent oltalmadban.
/3
#B59570B2
 Légy minékünk segítségül,
 Őrizz meg ellenséginktül,
 Mert az emberi segítség
 Hiábavaló epedség.
 Az Isten által minekünk
 Lészen erős győzedelmünk,
 És megszabadít ő bennünket,
 Megtapodja ellenségünket.

;Kiáltás Istenhez az idegenben
;Genf, 1562
>61
/1
#626E321D
 Kiáltásom halld meg, Isten!
 Vedd füledbe
 Az én könyörgésemet,
 Mert én szívem nagy ínségből,
 Meszsze földről
 Kiáltja Felségedet!
/2
#58D87700
 Végy fel engemet kőszálra,
 Magasságra,
 Hol bátorságom legyen,
 Mert te vagy én erős tornyom,
 Vigasságom
 Én ellenségim ellen!
/3
#F5251B3C
 Te hajlékodban lakásom
 Én kívánom
 És óhajtom szüntelen;
 Szárnyaidnak árnyékába
 Kívánkozva
 Vagyok jó reménységben.
/4
#31690FA5
 Meghallgatsz fogadásomban
 Engem, Uram,
 Nyújtván kegyességedet.
 Örökségüket megadod,
 És megáldod,
 Akik félik nevedet.
/5
#E09ADBE3
 És aztán vígan éneklek
 Szent nevednek
 Mostan és mindörökké,
 És amely fogadást tettem,
 Megfizetem
 Naponként őnékie.

;Csendes bizodalom Istenben
;Bourgeois L., Lyon, 1547
>62
/1
#0C4172BF
 Az én lelkem szép csendesen
 Nyugszik csak az Úr Istenben,
 Mert csak ő az én idvességem.
 Ő nékem erős kőváram,
 Megtartóm és én oltalmam,
 Minden gonosztól megment engem.
/2
#465F551B
 Azért, szívem, reménységed
 Csak az Úr Istenben vessed,
 Élj csak az ő segedelmével!
 Ő nekem magas kőszálam,
 Oltalmam és erős bástyám,
 Hogy soha ne tántorodjam el.
/3
#0CCF28CA
 Az Isten én idvességem,
 Erősségem, dicsőségem;
 Bízzatok azért csak őbenne!
 Előtte ti szíveteket, töltsétek ki lelketeket,
 Ő legyen lelkünk hiedelme!
/4
#F7E4C232
 Ne bízzatok erőszakban,
 Hamis ragadozástokban,
 Mulandó dolgon ne kapjatok!
 Ha sokasul ti kincsetek,
 Ahhoz ne bízzék szívetek,
 Mert nem állandó gazdagságtok.
/5
#A83107CB
 Az Isten egy szót szólt egyszer,
 Melyet én hallottam kétszer:
 Hogy nagy ereje vagyon néki.
 A kegyesség, Uram, tied,
 És te nyilván megfizeted,
 Akinek dolga mint érdemli.

;Vágyódás Isten és a szent hajlék után
;Bourgeois L., Genf, 1551
>63
/1
#52539827
 Isten, te vagy én Istenem,
 Jó reggel kereslek tégedet,
 Hozzád óhajtván, elepedett
 Szomjúság miatt én lelkem.
 Én testem hozzád áhítozik,
 Szomjúságban elhalt szintén
 E puszta és száraz földön,
 Hol semmi víz nem találtatik.
/2
#6CF81E1C
 Mert látni igen kívánja
 A te nagy erős dicsőséged,
 És isteni tiszteletedet
 A te dicső templomodba’.
 Mert nekem kedvesb életemnél
 A te nagy kegyelmességed,
 Melyért én ajakim téged
 Dicsérjenek szép énekléssel.
/3
#3A91B0AD
 Magasztallak én tégedet
 Életemnek minden rendiben,
 Én kezeimet fölemelvén
 Áldom te dicső nevedet.
 Örül, mintha drága étkekkel
 Jóllakott volna én szívem;
 Szent Fölségedet dicsérem,
 Éneklek rólad nagy örömmel.
/4
#FC93EEB0
 Rólad el nem felejtkezem,
 Még ágyamban is emlegetlek,
 És midőn reggel én fölkelek,
 Csak terólad emlékezem.
 Mert te énvelem sokszor jól től,
 És megszabadítál engem,
 Azért most is én életem
 Szárnyaid árnyékában örül.

;A gonoszok meglakolnak
;Bourgeois L., Genf, 1542
>64
/1
#4E0864FC
 Hallgasd meg, Uram, könyörgésem,
 Tarts meg ellenségem ellen,
 Aki reám dühödt szörnyen;
 Félelmétől őrizz meg engem,
 Mentsd meg életem!
/2
#59E068C2
 Rejts engem el a gonoszoktól,
 Akik reám fenekednek,
 Csak gonoszra igyekeznek;
 Tarts meg ő hamis tanácsuktól,
 Háborgásuktól!
/3
#A62574B1
 Akik nyelvüket élesítik,
 Mint öldöklő fegyvereket;
 Mint a nyilat, beszédüket
 Az ártatlan emberre lövik,
 És azt megsértik.
/4
#FEA10BF5
 Őket az önnön gonosz nyelvek
 Ejti a veszedelembe,
 És kik ezt veszik eszükbe,
 E dolgon igen megijednek
 És elrémülnek.
/5
#747A6DE4
 És nagy félelemmel mindenek
 Hirdetik az Isten dolgát,
 Beszélik annak nagy voltát,
 Melyet innen eszükbe vesznek
 És megértenek.
/6
#C391138E
 De szíve igaz embereknek
 Örvend az erős Istenben
 És az ő nagy kegyelmében.
 Az igaz hívek örvendeznek
 És dicsekednek.

;Hálaadás lelki-testi jókért (Aratási ének)
;Bourgeois L., Strasbourg, 1545
>65
/1
#CFB3878E
 A Sionnak hegyén, Úr Isten,
 Tied a dicséret,
 Fogadást tesznek néked itten,
 Tisztelvén tégedet,
 Mert kérésüket a híveknek
 Meghallod kegyesen,
 Azért tehozzád az emberek
 Jönnek mindenünnen.
/2
#D5E19100
 Rajtam a bűn elhatalmazék,
 Terhelvén engemet,
 De nagy volta kegyességednek
 Eltörli vétkünket.
 Boldog, akit te elválasztál,
 Fogadván házadba,
 Hogy előtted nagy buzgósággal
 Járjon tornácodba.
/3
#52341F25
 Javaival a te házadnak
 Megelégíttetünk,
 Szép dolgain te templomodnak
 Gyönyörködik szívünk.
 A te csuda igazságodból
 Megfelelsz minekünk,
 Hallgass meg, Isten, velünk tégy jól,
 Ó, mi segedelmünk!
/4
#ABB1AB29
 Tengeri habok nagy zúgását
 Te megcsendesíted,
 A pogány nép zúgolódását
 Ottan elenyészted.
 Nagy félelmükben elbágyadnak
 Mindenek e földön,
 Nagy voltán a te csudáidnak,
 Miknek számuk nincsen.
/5
#5A6E0347
 Megkoronázod az esztendőt
 Nagy sok javaiddal,
 Lábaid nyoma kövérséget
 Csepegtet nagy zsírral.
 Lakóhelyei a pusztáknak
 Folynak kövérséggel,
 Hegyek és halmok vigadoznak
 Nagy bő termésekkel.
/6
#6A2B7EB9
 A szép sík mezők ékeskednek
 Sok baromcsordákkal;
 Villognak a szép szántóföldek
 Sűrű gabonákkal.
 A hegyoldalak, mezőföldek
 Szép búzanövéssel,
 Örvendeznek és énekelnek
 Nagy gyönyörűséggel.

;Hálaének Isten csodálatos szabadításáért
;Bourgeois L., Strasbourg, 154
>66
/1
#92118CE2
 Örvendj, egész föld, az Istennek,
 És énekelj szép zengéssel!
 Nagy dicsőségét szent nevének
 Mindenek dicsérjék széjjel.
 Mondjátok ezt az Úr Istennek:
 Csudálatosak dolgaid,
 Erősséged nagy, hozzád esnek
 Hízelkedvén ellenségid.
/2
#B59AC3E2
 A te isteni felségedet
 E földön mindenek áldják,
 És dicsőséges szent nevedet
 Énekléssel magasztalják.
 Jertek, és ezt jól meglássátok,
 Minden jól ide figyelmezz:
 Istennek mily csudálatosak
 Dolgai az emberekhez.
/3
#171DC254
 Áldjátok a mi Istenünket,
 E földön minden emberek,
 Dicsérjétek az ő szent nevét
 Nagy zengéssel, minden népek.
 Mert életünket ő megadá
 Az ő nagy kegyességéből,
 Lábainkat meggyámolítá,
 Oltalmazván eleséstől.
/4
#330BE3D5
 Mi fejünkre népet ültetél,
 Mint a barmok, terhelteténk;
 Nagy árvizeket ránk eresztél,
 Sebes tűzön általmenénk.
 De te kihoztál, Uram, minket,
 Megenyhítél, megnyúgotál,
 Templomodban azért tégedet
 Dicsérlek szép áldozattal.
/5
#DB33210E
 Jertek, halljátok, hadd beszéljem
 Tinéktek, istenfélőknek,
 Amiket Isten tett énvélem,
 Mily kegyesen tett lelkemnek!
 Hogy szájammal hozzá kiálték,
 Ottan meghallgata engem,
 Azért beszédével nyelvemnek
 Mindenkor őtet dicsérem.

;Isten a népek Ura
;Bourgeois L., Strasbourg, 1545
>67
/1
#9F77167F
 Úr Isten, áldj meg jóvoltodból
 És kegyesen fordulj hozzánk,
 Oltalmazz meg minden gonosztól,
 Szent színedet fordítsd reánk,
 Hogy e földön minden
 Megismerje szépen
 A te utadat,
 És a pogány népek
 Téged tiszteljenek,
 Megtartójukat.
/2
#EDCE5832
 És akkoron dicsérnek téged,
 Dicsérnek téged a népek,
 Nagy tisztességet tesznek néked.
 A pogányok is örülnek,
 Midőn mindeneket
 És a pogány népet
 Szent igazsággal
 Bírod és ítéled
 És jóra vezérled
 Nagy hatalmaddal.
/3
#4B44704F
 Dicsérjen téged minden nemzet,
 Úr Isten, téged dicsérjen!
 A föld teremjen bő gyümölcsöt,
 Áldjon meg minket az Isten!
 Adja szent malasztját,
 Nyújtsa áldomását,
 És ő felségét
 Félje és rettegje
 E föld kereksége,
 Mint ő Istenét.

;Diadalének (A hugenották harci zsoltára)
;Greiter M., Strasbourg, 1525 (1539) után
>68
/1
#73DBBDA6
 Hogyha felindul az Isten,
 Elkergettetnek szertelen
 Minden ő ellenségi.
 És elfutamodnak széjjel
 Ő haragos színe elől
 Minden ő gyűlölői.
 Úgy elűzetnek hirtelen,
 Mint a füst semmivé leszen,
 Elvész minden ő dolgok;
 Mint viasz olvad a tűztől,
 Úgy az ő kemény színétől
 Elvesznek a gonoszok.
/2
#5B136DB3
 De az igazak mindnyájan
 Örvendeznek nagy hívságban
 A kegyes Isten előtt;
 És víg szívvel énekelnek,
 Hogy ő kevély ellenségek
 Megszégyenült, elveszett.
 Nagy örömmel az Istennek,
 Énekeljetek nevének,
 Dicsérjétek, áldjátok;
 Ki puszta földeken megyen,
 Kinek neve erős Isten:
 Őtet magasztaljátok!
/3
#5AF4C21B
 Énekeljetek Istennek,
 Ki lakója az egeknek,
 Kiket teremtett régen.
 Holott nagy hatalmával ül,
 Honnan szava úgy megdördül,
 Hogy zeng és harsog minden.
 Dicsérjétek nagy hatalmát,
 Ki felséges méltóságát
 Izráelen láttatja!
 Kinek nagy erejét ott fönn,
 Az égen és a felhőkön
 Senki sem tagadhatja.

;Kiáltás a mélységből
;Bourgeois L., Genf, 1551
>69
/1
#7A888036
 Úr Isten, segíts és tarts meg engem,
 Mert a vizek szinte lelkemig érnek,
 Közepén vagyok a sáros mélységnek,
 Amelyben csaknem elsülylyed fejem!
 Az árvizek öszszeütnek rajtam,
 A kiáltás miatt torkom elrekedt,
 Én szemeimben megfogyatkoztam,
 Midőn várom a te segedelmedet.
/2
#7AB7E691
 Én hajam szálánál többen vagynak,
 Akik engemet ok nélkül gyűlölnek;
 Én ellenségim szertelen erősek,
 És engem eltörleni akarnak.
 Noha semmit nem vettem senkitől,
 De mégis énnékem kell megfizetnem;
 Nincs, Uram, elrejtve színed elől
 Én bolondságom és minden én vétkem.
/3
#C0FE3EDA
 Úr Isten, nagy a te kegyességed,
 Hallgasd meg azért, amit tőled kérek:
 Irgalmas szemeid reám nézzenek,
 Hadd láthassam nagy kegyelmességed!
 Ne rejtsd el, Uram, kegyes orcádat
 Szegény szolgádtól, mert szorongattattam;
 Ne késsél, halld meg kiáltásomat,
 Add meg kérésem, vigasztalj meg, Uram!

;Uram, ne késlekedjél!
;Bourgeois L., Genf, 1551
>70
/1
#2A305880
 Siess, ments meg, Uram Isten,
 Mert benned bízom teljes szívvel,
 Azért hamar légy segítséggel
 Minden ellenségim ellen!
 Akik törnek én életemre,
 Mind megszégyeníttessenek;
 Kik nyavalyámon örülnek,
 Térjenek meg nagy szégyenükre!
/2
#7C6636FA
 Nyomorult és szegény vagyok,
 Tarts meg azért, ó, én Istenem,
 Mert csak te vagy én segedelmem:
 Ne késsél, mert majd elfogyok!
 Hogy azok benned örvendjenek,
 Kik segedelmedet várják
 És magukat reád bízzák,
 Ezt mondván: dicsőség Istennek!

;Ne hagyj el engem agg koromban!
;Bourgeois L., Genf, 1551
>71
/1
#C4498634
 Tebenned bízom, én Istenem,
 Kérlek, oltalmazz meg,
 Gyalázattól ments meg,
 Hogy örök szégyenbe ne essem!
 A te nagy jóvoltodból
 Tarts meg minden gonosztól!
/2
#4C8C2163
 Hajtsd hozzám füled, tarts meg engem,
 Nyújtsad segítséged,
 Amint megígérted,
 Hogy segítségül léssz énnékem;
 Légy azért én kősziklám
 És én erős kőváram!
/3
#0DB0501E
 Ments ki a hamisnak kezéből,
 És annak markábul,
 Aki él csalárdul;
 Oltalmazz meg a kegyetlentől!
 Benned bíztam, Uramba’,
 Gyermekségemtől fogva.
/4
#A27F43A1
 Hogy származám anyám méhéből,
 Legottan énnékem
 Valál reménységem.
 Anyám méhéből engem kivől,
 Azért neked éneklek
 És szüntelen dicsérlek.
/5
#CD84312E
 Ez én nyavalyás vén koromban,
 Erőtlenségemben
 Ne vess el, Úr Isten!
 Ne hagyj el én sok nyavalyámban,
 Midőn szegény testemben
 Semmi erősség nincsen!
/6
#3A973085
 Gyermekségemtől fogva engem
 Híven tanítottál
 Csuda dolgaiddal.
 És hogy immáron megvénhedtem,
 És a hajam megőszül:
 Légy most is segítségül!
/7
#C8F4FC4A
 A te hűségedet, Úr Isten,
 Hirdetem, éneklem,
 Mindennek beszélem.
 Dicsérlek lantnak zengésében,
 Ó, szentek dicsősége,
 Izráelnek szentsége!
/8
#5D111589
 Ajkaimmal vígan dicsérem
 A te hatalmadat
 És nagy irgalmadat.
 Lelkemet hozzád fölemelem:
 Megtartád életemet,
 Hogy dicsérjelek téged!

;Áldáskívánás a béke fejedelmére
;Bourgeois L., Strasbourg, 1545
>72
/1
#0490F149
 Uram, a te ítéletedet
 Adjad a királynak,
 És igazságodnak értelmét
 A király fiának,
 Hogy ő a te nagy seregedet
 Igazán ítélje,
 És a te sok szegény népedet
 Törvénynyel vezérlje.
/2
#A802A142
 Nagy alázatosan imádják
 Őt minden királyok;
 Minden népek őtet szolgálják,
 És ő lészen urok,
 Mert ő a szegényt megsegíti,
 Aki őtet hívja,
 És a nyomorultat megmenti,
 Kinek nincs gyámola:
/3
#D4662A8B
 A szűkölködőket megszánja
 Nagy kegyességében,
 És a szegényt hozzá fogadja,
 Megőrizvén híven.
 Megtartja erőszaktól őket,
 És a csalárdságtól,
 Nagyra becsüli ő véröket,
 Megmenti gonosztól.
/4
#8ED41E7F
 E földön minden nemzetségek
 Őtet áldják vígan,
 És szívük szerint dicsekednek
 E kegyes királyban.
 A pogányok is így dicsérik:
 Dicsőség Istennek,
 Aki nagy csudát cselekeszik,
 Ura Izráelnek.

;Jó nékem az Isten közelsége
;Bourgeois L., Genf, 1551
>73
/1
#C306D3F0
 Bizonyára jó az Isten,
 Híveihez kivált képpen,
 Akik szívüket tisztán tartják,
 Az ő jóvoltát azok látják.
 De én már csaknem elhajlék,
 Úgy megtántorodék lábam,
 Járásomban úgy megbotlám,
 Hogy csaknem szörnyen ledűlék.
/2
#F6A4BF32
 Mert bosszankodom e népre,
 Kinek nagy esztelensége,
 Midőn látom, hogy a hitlenek
 Jó szerencsében, vígan élnek;
 A halállal nem bajlódnak,
 Semmi fájdalmat nem látnak,
 És kövérek ő testükben,
 Élnek nagy jó egészségben.
/3
#EBC5038D
 Mit használ tehát énnékem,
 Hogy tisztán tartom én szívem?
 És micsoda hasznom van abban,
 Hogy kezem mosom tisztaságban?
 Ímé, mind hiába marad,
 Hogy én így kesergettetem,
 Ottan marad büntetésem
 Reggel, mihelyen megvirrad.
/4
#93D8BD48
 De én mindig nálad vagyok,
 És közeledben maradok,
 Mert te megtartál jobb kezeddel,
 Nagy ínségemben nem hagyál el.
 Tanácsoddal vezérlj engem,
 És igazgass ösvényedre,
 Aztán végy fel dicsőségre,
 Tégy ily kegyesen énvelem!
/5
#1BCDF8ED
 A mennyekben te vagy nekem
 Csak egyedül én Istenem;
 Ez egész földön senki nincsen,
 Kit kívüled lelkem kedveljen.
 Ha elfogy testem és lelkem,
 Mégis vigasztalsz szívedben;
 Egyéb részem nekem nincsen
 Tenáladnál, én Istenem!
/6
#E5474D98
 Mert kik tőled eltávoznak,
 Azok elvesznek, romlanak;
 Elveszted azokat hirtelen,
 Kik bíznak bálvány-istenekben.
 Azért én néked ezt mondom:
 Közelséged oly jó nékem,
 Mert te vagy én menedékem,
 És jótéted magasztalom!

;Az ellenség a templomban
;Genf, 1562
>74
/1
#C2F466AF
 Miért vetsz minket így el, Úr Isten?
 Mire haragszol mireánk enynyire?
 Míglen gerjed föl haragodnak tüze
 Juhaidnak nyájára ily igen?
/2
#02A33287
 Emlékezzél meg te seregedről,
 Melyet régente magadnak szerzettél;
 Felvett örökségedről emlékezzél,
 A Sion hegyén lakóhelyedről!
/3
#549A9D4C
 Kelj föl, és végre jövel, Úr Isten:
 Szent templomodat rútul összedönték,
 Mindent elrontott benne az ellenség,
 Pusztaságot tőn te szent helyeden!
/4
#BB77973D
 De az Isten én királyom régen,
 Aki engemet bírt és jól vezérlett.
 Hatalmát jelentvén e világ előtt,
 Hogy segedelmem ő mindenekben.
/5
#1E7436AD
 Te isteni nagy bölcsességeddel
 Bizonyos határt vetettél a földnek,
 És különbségét a nyártól a télnek
 Elosztád hévséggel és hideggel.
/6
#9E3CC5B9
 Ne engedd hátra térni szégyennel
 A te nyomorult szegény szolgáidat;
 Fordítsd hozzájuk te nagy jóvoltodat,
 Hogy nevedet dicsérjék víg szívvel!

;Isten megítéli a kevélyeket
;Genf, 1562
>75
/1
#78586D5C
 Dicsérünk téged, Isten,
 Dicséret légyen neked,
 Mert a te dicső neved
 Hozzánk közel jött híven,
 Mi azért csudáidat,
 Hirdetjük jó voltodat.
/2
#8F6868C9
 Mert a tiszt és az erő
 Nem a napkelet felől,
 Sem a napenyészetről,
 Sem nem a pusztából jő:
 Minden az Istenen áll,
 Ő aláz, ő magasztal.
/3
#7BF2E99F
 Én örökké dicsérem
 A Jákóbnak Istenét,
 Hirdetem dicsőségét,
 És szarvukat megszegem
 Az istenteleneknek,
 Hogy a jók felkeljenek.

;A mennyei Bíró dicsérete
;Bourgeois L., Genf, 1551
>76
/1
#123EA3E6
 Ismeretes az Úr Isten
 A Júdában és szent neve
 Az Izráelnek földében;
 Meszsze terjed dicsősége,
 A Sálemben szép ő sátora,
 A Sionon lakó hajléka.
/2
#A826972A
 Ott meglátja minden ember,
 Hogy isteni hatalmával
 Megtörik íj, paizs, fegyver,
 Hadat megállít azonnal.
 Nagyobb felségednek ereje,
 Hogynem az ellenség fegyvere.
/3
#0414A9FA
 Ki állhatna meg előtted,
 Aki ily rettenetes vagy,
 Midőn haragod felgerjed?
 Mennybéli szentenciád nagy,
 Melyet hallatsz az emberekkel,
 A föld megrémül csendességgel.
/4
#83EDAD14
 Midőn felkél az Úr Isten,
 Tartván kemény ítéletit,
 Hogy a szegényeket itten,
 E földön megtartsa népit:
 Dicséretedre fordul néked,
 Ha a nép haragszik ellened.
/5
#CF591B52
 Még egyszer nyilván elveszted
 E maradék dúlót-fúlót.
 Jer, dicsérjük Istenünket,
 Megállván fogadástokat,
 Kik mindenkor vagytok ővéle,
 És el nem távoztok őtőle!
/6
#438B73CB
 Adjatok szép ajándékot
 E rettenetes Istennek,
 Ki megtöri hatalmukat
 A gonosz fejedelmeknek!
 A nagy királyok itt e földön
 Előtte lesznek rettegésben.

;Jelen, múlt és jövendő
;Bourgeois L., Strasbourg, 1545
>77
/1
#484B1815
 Az Istenhez az én szómat,
 Emelém kiáltásomat;
 Hogy felkiálték hozzá,
 Beszédem meghallgatá.
 Mindennémű szükségemben
 Reménységem csak az Isten;
 Éjjel kezem feltartom,
 Az égre hozzá nyújtom.
/2
#34A3A127
 Lelkem nagy bánatba esett,
 Minden vigasztalást megvet,
 Az Isten rettent engem,
 Ha róla emlékezem.
 Noha Istennek szívemben
 Panaszlok nagy ínségemben,
 Lelkem mégsem találhat
 Semmiben nyugodalmat.
/3
#61489C7A
 Szemeimet nyitva tartod,
 Aludni éjjel sem hagyod;
 Erőmben úgy elfogyék,
 Hogy egy szót sem szólhaték.
 Gondolám a régi időt,
 És forgatám szemem előtt
 Az elmúlt esztendőket,
 Hányám és vetém őket.
/4
#F224BD04
 Ó, erős és kegyes Isten,
 Szent vagy cselekedetidben,
 És sehol senki nincsen
 Hozzád hasonló Isten.
 Csuda, Isten, a te dolgod,
 Amint gyakran megmutatod,
 Minden népek jól látják,
 Nagy voltát hatalmadnak.

;Istenről beszél a történelem
;Bourgeois L., Genf, 1551
>78
/1
#900E725B
 Hallgass, én népem, az én törvényemre,
 Füledet hajtsad az én beszédemre,
 Amelyek az én szájamból származnak.
 Hogy jól megérthesd mivoltát azoknak,
 Mert én neked oly dolgot beszélek,
 Mit titoknak tarthatnak mindenek.
/2
#D839BBDC
 Oly dolgot, amit a mi atyáinktól
 Hallottunk, és megértettünk azoktól.
 Nem azért, hogy csak mi megemlékeznénk,
 De fiainknak is jól elbeszélnénk;
 Hirdessük azért nagy dicsőségét
 És az ő sok csuda téteményét!
/3
#6A2DAF25
 A pusztán a kősziklát meghasítá,
 És a vízzel, amely abból kifolya,
 Népét megitatá, és azon helybe’
 A kősziklából kútfejet ereszte,
 Melyből bőséges forrás buzdula,
 Mely, mint a patak, sebesen folya.
/4
#F2D95F10
 Bátorsággal ő seregét kivivé,
 Az ellenséget a tengerbe veszté;
 Mindenütt nyilván szabadon menének,
 Míg a szent föld határába érének,
 Mind a nagy hegyig a dicsért földön,
 Melyet erős jobb kezével megvőn.

;Panaszos ének Jeruzsálem elpusztításáról
;Bourgeois L., Strasbourg, 1545
>79
/1
#ECDB3DC2
 Öröködbe, Uram, pogányok jöttek,
 És szent templomodat megfertőztették,
 Jeruzsálem városát elrontották,
 És széjjel nagy kőrakásokra hányták.
 Szolgáidnak testek,
 Akik megölettek,
 Adattak a hollóknak;
 Húsok te szentidnek
 Ételül vettetnek
 A mezei vadaknak.
/2
#7712B27F
 Míg haragszol, Uram, reánk ekképpen?
 Haragod míglen gerjedez ily igen?
 Meddig terjeszted bosszúállásodat,
 Mely minket, mint a sebes tűz, elfogyat?
 Bosszúdat azokra,
 Ontsd a pogányokra,
 Kik téged nem tisztelnek!
 Dűtsd az országokra,
 Hol nevedet soha
 Nem tisztelik a népek!
/3
#2AE2021D
 Tekintsd meg, Uram, kegyelmességedet,
 A te szent nevedért segíts meg minket!
 Szabadíts és tarts meg minket kegyesen,
 Bűneinket bocsásd meg szent nevedben!
 Hogy ne nevettessünk,
 Kérdvén: hol Istenünk?
 Verd meg a pogány népet!
 Vérét szolgáidnak,
 Mit ők kiontának,
 Nékik el ne engedjed!

;Isten, állíts helyre minket!
;Genf, 1562
>80
/1
#838124D6
 Hallgasd meg, Izráel pásztora
 És Józsefnek vezérlő Ura,
 Kit őrizsz, mint egy juhnyájat,
 Fordítsd hozzánk szent orcádat!
 Aki ülsz a Kérubimon,
 Jelenjél meg világoson!
/2
#05A240FF
 Úr Isten, térj hozzánk ismétlen,
 Őrizz meg minden gonosz ellen!
 Nyújtsad világosságodat,
 Fordítsad ránk irgalmadat,
 Térítsd hozzánk szent orcádat,
 És semmi nekünk nem árthat!
/3
#848598D9
 Ó, Úr Isten, vajon míg tartod
 Mirajtunk a te nagy haragod?
 Míg veted meg kérésünket?
 Könnyhullatást kenyér helyett
 Adál nékünk, és itattál
 Minket nagy könnyhullatással.
/4
#62806FEC
 Uram, ismétlen térj mihozzánk,
 Mennyekből szemed fordítsd reánk,
 És látogasd meg a szőlőt,
 Melyet jobb kezed ültetett,
 Nézd meg a csemetét végre,
 Kit szerzél dicsőségedre!
/5
#C575F064
 Tűzzel megint megégettetik,
 Teljességgel elpusztíttatik
 Haragodnak nagy tüzében.
 Nyújtsd kezedet, ó, Úr Isten,
 Az emberre, kit kezeddel
 Magadnak erősítettél.
/6
#9169928B
 És mi el nem térünk tetőled,
 Csak életünket erősítsed,
 És dicsérjük szent nevedet,
 Úr Isten, vigasztalj minket!
 Világosítsd színed rajtunk,
 És mi nyilván megtartatunk.

;Ünnepi ének
;Genf, 1562
>81
/1
#1245001B
 Örvendezzetek
 Az erős Istennek!
 Énekeljetek
 Dicséreteket,
 Szép énekeket
 Jákób Istenének!
/2
#69D62AD6
 Szép dicséretet
 Néki mondjon minden!
 Lantokban őtet
 És citerákban
 Dicsérjük vígan
 Néki zengedezvén!
/3
#00EF23D5
 Most ez új hóban (évben)
 Néki örvendezzünk
 Trombitaszóban!
 Rendelt időnkben,
 Víg ünnepünkben
 Illik énekelnünk!
/4
#2A9DC13C
 Hallgasd meg, népem,
 És közlöm tevéled
 Vádló beszédem!
 Halld meg, Izráel,
 Amit szám beszél,
 És azt jól megértsed:
/5
#6416F067
 Néked ne legyen
 Idegen istened,
 De egyedül én!
 Csak engem tisztelj,
 Senki mást ne félj.
 Nevemet becsüljed!
/6
#D777B797
 Én vagyok neked
 Istened egyedül;
 Ínségben téged
 Én megtartálak
 És kihozálak
 Egyiptom földébül.
/7
#179CA116
 Tátsd föl csak szádat,
 És megtöltöm bőven;
 Menten meglátod,
 Hogy nagy bőséggel
 Lesz az eledel
 Csudálatosképpen.
/8
#04A85CB8
 De az én népem
 Engem nem hallgata,
 Noha intettem
 Sűrű intéssel,
 De az Izráel
 Füleit bedugta.
/9
#9F591E17
 Min megbúsulék,
 És őket elhagyám,
 Hogy bár menjenek
 Önnön kedvükre,
 És ösvényükre
 Szabadon bocsátám.
/10
#A2D2ED50
 Ha népem szívvel
 Szót fogadott volna,
 És ha Izráel
 Én útaimban
 És tanácsomban
 Járni akart volna,
/11
#B8EF6997
 Én is legottan
 Az ő ellenségét
 Nagy hatalmamban
 Megvertem volna,
 Vetettem volna
 Rájok én kezemet.
/12
#06BA02B9
 Ő ellenségét
 Néki adtam volna,
 Jó szerencséjét
 Én őnékie
 Nagy sok időkre
 Terjesztettem volna.
/13
#7C609F49
 Búzát nékie
 Szépet adtam volna
 Eledelére.
 És nagy bőséggel
 A kősziklából
 Mézet adtam volna.

;Isten az igaz Bíró
;Bourgeois L., Genf, 1551
>82
/1
#191FCAEF
 Az Isten áll ő seregében,
 A bírák gyülekezetében,
 És köztük igazat ítél.
 És nékiek ígyen felel:
 Míglen ítéltek ti ekképpen
 A törvény igazsága ellen,
 Hogy személyét a hamisnak
 Nézitek, kedvezvén annak?
/2
#906E882A
 Ítéljetek szegényt igazán,
 És könyörüljetek az árván!
 Ügyét a szegény embernek
 Erőszaktól megmentsétek!
 A nyomorultat a hamistól,
 Kiszabadítsátok markából,
 Hogy a szegény erőt vegyen
 A hatalmaskodó ellen!
/3
#40A80AF4
 De intésemet nem fogadják,
 Sőt értetlenül megutálják.
 Nagy setétségben mennek el,
 Min a föld csaknem süllyed el.
 Ímé, én mondom, hogy ti vagytok,
 Kik isteneknek hívattattok,
 És a magasságos Úrnak
 Ti mondattok fiainak.
/4
#7A5DE6CB
 De mindnyájan meg kell halnotok,
 Mint egyebek, ti is kimúltok,
 Végre ti mind oda lesztek,
 Úgy mint egyéb fejedelmek.
 Támadj fel azért, ó, Úr Isten,
 És ítélj meg mindent e földön,
 Mert neked hatalmad vagyon
 Minden népen e világon!

;Körös-körül veszedelem
;Genf, 1562
>83
/1
#D8CBA93D
 Uram, ne hallgass ily igen,
 Ne maradj e veszteglésben,
 Ne nyugodjál, ó, erős Isten!
 Mert dühösködő ellenségid
 És minden gonosz gyűlölőid
 Felemelték fejüket fennen.
/2
#37A6E7E7
 Kiváltképpen néped ellen
 Csalárd árultatásképpen
 Sok álnok tanácsokat lelnek,
 És akikre te gondot tartasz,
 És rejtekhelyedben oltalmazsz,
 Azok ellen összeesküsznek.
/3
#8A02CDC2
 Mondván: jertek, ím ezeket,
 Veszessük el e nemzetet,
 Töröljük őket el e földről;
 Fogyasszuk el e népet szintén,
 Hogy emlékezet se lehessen
 Az Izráel népe nevéről!
/4
#16A1FB3B
 Kergesd el forgó szeleddel,
 Nagy rettegéssel ijeszd el,
 Szélvésszel háborítsd meg őket!
 Ő orcájukat szégyenítsd meg,
 Hadd ismerjenek tégedet meg,
 És keressék te szent nevedet.
/5
#B8D69C98
 Taszítsd őket nagy szégyenbe,
 És ejtsd be nagy félelembe!
 És megismerjék minden népek,
 Hogy te az élő egy Isten vagy,
 Akinek hatalma igen nagy,
 Akit felségesnek neveznek.

;Sóvárgás a szent hajlék után
;Genf, 1562
>84
/1
#5E0E8759
 Ó, seregeknek Istene,
 Mely kedves gyönyörűsége
 A te szerelmes hajlékidnak!
 Az én lelkem fohászkodik,
 Tornácodba kívánkozik.
 Ó, Istene a magasságnak!
 Áhítozik testem, lelkem
 Tehozzád, élő Istenem!
/2
#34ABBAD7
 A verébnek is van fészke,
 És honjában költ a fecske.
 Én királyom, Zebaoth Isten:
 Hol vannak a te oltárid
 És te szentséges hajlékid,
 Hol dicsértetel felségesen?
 Bizony boldog az oly ember,
 Ki téged házadban dicsér.
/3
#58BB080F
 Ó, boldog az ember nyilván,
 Aki a te útaidban
 Kíván járni szívvel-lélekkel!
 Menvén a siralom völgyén,
 Ahol merő száraz minden,
 Ott is ő nagy hiedelemmel
 Kutakat ás és csatornát,
 Melybe esővizet bocsát.
/4
#28DB75D2
 Mennek erőről erőre,
 Segítségről segítségre,
 Míg hozzád jutnak a Sionra.
 Ó, erős Zebaoth Isten,
 Hajtsd hozzám füled kegyesen,
 És figyelmezzél az én szómra!
 Jákób Istene, nagy Isten,
 Hallgass meg én szükségemben!
/5
#DD7B14A9
 Mi paizsunk, ó, Úr Isten,
 Fölkentedre nézz kegyesen,
 Mert jobb egy nap a te házadban,
 Hogynem ezer nap egyebütt!
 Az Isten tornáca előtt
 Kapunálló lennék inkábblan,
 Hogynem mint sok időt éljek
 Házukban a hitleneknek.
/6
#9B8939A2
 Mert minékünk fényes napunk
 Az Isten, és mi paizsunk,
 Nagy dicsőséggel szeret minket.
 Azokkal kegyelmet tészen,
 Kik járnak a jó ösvényen:
 Sok javaival áldja őket,
 Boldog az ember éltében,
 Ki bízik az Úr Istenben.

;Vigasztalás szenvedésben
;Genf, 1562
>85
/1
#4696B7FA
 Nagy kegyesen től, Uram, földeddel,
 Jákób nemzetivel a fogságban,
 Kiket rabságból haza engedtél,
 Megengedél nékik irgalmasan.
 Megbocsátál, bűnüket elfedéd,
 Ellenük haragodat enyhítéd.
 Uram, kegyesen végy hozzád minket,
 Vedd el rólunk nagy haragod tüzét!
/2
#E684C910
 Nemde mind örökké haragszol-é?
 Mind éltig nyújtod-é haragodat?
 Népedet már meg nem enyhíted-é?
 Hogy benned lelhesen vigasságot,
 És noha nagyok a mi bűneink,
 Mégis kegyelmed mutasd minékünk,
 És noha tettünk sok gonoszságot,
 De tőled kérünk irgalmasságot.
/3
#F43E9C8F
 Én meghallgatom, mit szól az Isten
 Az ő népének és ő szentinek:
 Békességet beszél kegyelmesen,
 Hogy bolondságból ne vétkezzenek.
 Kik őtet félik, minden meghiggye,
 Közel azokhoz ő segedelme;
 Hogy dicsősége lakjék földünkben,
 Minden ínséget rólunk elveszen.
/4
#D0A97A9E
 Jóság hűséggel összebékélik,
 Békesség, igazság egymást szépen
 Csókolják, e földön nevekedik
 A hit, látszik igazság az égen.
 Mindennemű jót ád az Úr Isten,
 És sok jó gyümölcs terem e földön.
 Az igazság megyen őelőtte,
 És ő járása tart mindörökké.

;Szorongatott ember imádsága
;Bourgeois L., Strasbourg, 1545
>86
/1
#32553BCF
 Hajtsd hozzám, Uram, füledet,
 És hallgasd meg kérésemet,
 Mert igen szegény vagyok,
 Az én szükségim nagyok.
 Tartsd meg testemet, lelkemet,
 Tekintsd kegyes életemet,
 Szolgádhoz térjen kedved,
 Ki bízik csak tebenned!
/2
#8A8B57E9
 Hozzád mindennap óhajtok,
 Nagy szükségemben kiáltok;
 Te nagy irgalmad szerint
 Kegyelmezz meg óránkint!
 Szolgád lelkét vigasztald meg,
 Uram, kiáltásom halld meg,
 Mert szívemet e végre
 Emeltem fel az égre!
/3
#4EF060F1
 Uram, jókedvű s édes vagy,
 A te irgalmasságod nagy,
 Minden emberhez pedig,
 Ki hozzád esedezik.
 Hallgasd meg azért kérésem,
 És nézd meg esedezésem:
 Tekintvén kegyelmedre,
 Figyelmezz beszédemre!
/4
#1FDA5879
 Nagy szükségemben óhajtok,
 Tehozzád szívből kiáltok,
 És te engem meghallgatsz,
 Nyavalyámban el nem hagysz.
 Ugyan nincs sehol több Isten,
 Ki hozzád hasonló légyen,
 Nincsen több erős Isten,
 Ki ily dolgot tehessen.
/5
#E4535377
 E világon minden népek,
 Kiket teremtél, eljőnek,
 És imádnak tégedet,
 Dicsőítik nevedet.
 Mert te nagy és hatalmas vagy,
 Csudatételed sok és nagy;
 Te vagy egyedül Isten,
 Sehol Isten több nincsen.
/6
#F23293BD
 Vezess, Uram, útaidban,
 Hogy járjak igazságodban,
 És csak arra hajtsd szívem,
 Hogy szent nevedet féljem.
 Néked, Uram, hálát adok,
 És teljes szívből vigadok,
 Mindörökké nevednek
 Dicséretet éneklek.

;A szent város dicsérete
;Genf, 1562
>87
/1
#C6754817
 Az Úr Isten az ő lakó hajlékát
 Az ő szentséges hegyén helyezi,
 És a Siont ő inkább szereti,
 Hogynem Jákóbnak akármely sátorát.
/2
#01B704A0
 Azért ott énekelnek szép éneket,
 Mond az Isten, és itt örvendeznek,
 Mert én dicsőségére e helynek
 Támasztok nagy-szép élő kútfejeket.

;Könyörgés halálos veszedelemben
;Genf, 1562
>88
/1
#EF65FD86
 Úr Isten, én idvességem,
 Éjjel-nappal kiáltok hozzád!
 Könyörgésemet meghallgassad,
 És tekintsd meg nagy ínségem!
 Kegyesen hajtsd hozzám füledet,
 Értsd meg én esedezésemet!
/2
#BD3A0746
 Az én lelkem nyavalyákkal
 Teljességgel eláradt, eltölt:
 Mint aki már a sírba készült,
 És a pokolra alászáll,
 Vagyok ahhoz szinte hasonló,
 Akinek már kész a koporsó.
/3
#9A00BDB9
 Megfosztattam életemtől,
 Mint akiket agyonvertenek,
 Kik a halottak közt hevernek,
 Kikről már nem emlékezel.
 Kik eltemettetvén feküsznek,
 És te kezedből kiestenek.
/4
#BA3DF44D
 A koporsóba től engem,
 Bévetél a setét mélységben,
 Holott haragod nyom keményen;
 Elborítád szegény fejem
 Nagy árvizednek habjaival,
 Mik rám rohannak nagy zúgással.
/5
#A226D334
 Engem te megutáltatál,
 És tőlem én ismerőimet
 Elvitted, elhagytak engemet,
 És a tömlöcbe taszítál,
 Holott kemény fogságban vagyok,
 Melyből ki nem szabadulhatok.
/6
#B2455AFD
 Uram, miért vetsz el engem,
 Miért rejted el szemeidet?
 Szegény vagyok, erőm elveszett,
 Jaj, mely igen gyötrettetem!
 Ez én régi nagy ínségemben
 Előtted vagyok rettegésben.
/7
#9429BDFA
 Nagy haragod reám borult.
 Nagy rettegés engem körülvett,
 És teljességgel elmerített,
 Mint az árvíz, reám tódult.
 Sanyargat engem minden dolog,
 Valamely énkörültem forog.

;Isten ígéretei beteljesednek (A Messiás-királyról)
;Genf, 1562
>89
/1
#E2DED85D
 Az Úrnak irgalmát örökké éneklem,
 És hűséges voltát mindenkor hirdetem,
 Mert mondom, hogy megáll mindörökké irgalma,
 Melyet úgy megépít, hogy megálljon mindenha,
 És hogy mind az égig erősíted, megtartod
 Te szent igazságod és a te fogadásod.
/2
#15D33B11
 Az egek hirdetik sok csuda dolgodat,
 A szent gyülekezet te igazságodat,
 Mert vajon kicsoda volna ott fenn az égben,
 Ki nagy hatalommal hozzád hasonló légyen?
 Az erős angyalok seregében vagyon-e,
 Ki e dicső Úrhoz hasonlatos lehetne?
/3
#9087B266
 Igen rettenetes e fölséges Isten,
 Őtet féli minden a szent gyűlésekben,
 Ó, seregek Ura, minden enged tenéked,
 Te nagy erős Isten, vajon ki érne véled?
 Tenálad lakozik a te nagy igazságod,
 Soha el nem múlik a te igaz mondásod!
/4
#9A3BEB98
 Boldog a nép, amely tenéked örvendez,
 Minden dolgát, Uram, ez viszi jó véghez.
 Fényes orcád előtt ezek járnak merészen,
 És a te nevedben örvendeznek szüntelen,
 Mert nagy dicsőségre őket felmagasztalod,
 És jótéteményed rajtuk megszaporítod.
/5
#B54C65FE
 Te vagy ékessége az ő erejüknek,
 Minden hatalmukat te adtad nékiek.
 A te kegyelmedből orcánkat fölemeljük,
 Tetőled, Úr Isten, mi paizsunkat vettük,
 És a mi királyunk a te fegyvered nélkül,
 Ó, Izráel Ura, nem lehet segítségül!
/6
#CD97F647
 Gondold szolgáidnak nagy gyalázatjukat,
 És hogy sok népeknek iszonyú szidalmát
 Keblemben hordozom, kik tégedet bosszantnak,
 Gyalázván homlokát fölkenett királyodnak.
 Dicsőség tenéked és áldassál, Úr Isten,
 Melyre minden néptől mondassék: Ámen, ámen!

;Isten az örökkévaló hajlék
;Bourgeois L., Genf, 1551
>90
/1
#2B8F5A43
 Tebenned bíztunk eleitől fogva,
 Uram, téged tartottunk hajlékunknak!
 Mikor még semmi hegyek nem voltanak,
 Hogy még sem ég, sem föld nem volt formálva,
 Te voltál és te vagy, erős Isten,
 És te megmaradsz minden időben.
/2
#8F427215
 Az embereket te meg hagyod halni,
 És ezt mondod az emberi nemzetnek:
 Légyetek porrá, kik porból lettetek!
 Mert ezer esztendő előtted annyi,
 Mint a tegnapnak ő elmúlása
 És egy éjnek rövid vigyázása.
/3
#8B29FD8F
 Kimúlni hagyod őket oly hirtelen,
 Mint az álom, mely elmúlik azontól,
 Mihelyt az ember felserken álmából,
 És mint a zöld füvecske a mezőben,
 Amely nagy hamarsággal elhervad,
 Reggel virágzik, estve megszárad.
/4
#669A67BC
 Midőn, Uram, haragodban versz minket,
 Ottan meghalunk és földre leesünk,
 A te kemény haragodtól rettegünk,
 Ha megtekinted mi nagy bűneinket,
 Titkos vétkünket ha előhozod
 És színed eleibe állítod.
/5
#38FD92E1
 Haragod miatt napja életünknek
 Menten elmúlik nagy hirtelenséggel,
 Mint a mondott szót elragadja a szél.
 A mi napink, miket nekünk engedtek,
 Mintegy hetven esztendei idő,
 Hogyha több, tehát nyolcvan esztendő.
/6
#AFFDED60
 És ha kedves volt is valamennyire,
 De többnyire volt munka és fájdalom;
 Elkél éltünknek minden ékessége,
 Elmúlik, mint az árnyék és az álom.
 De ki érti a te haragodat?
 Csak az, aki féli hatalmadat.
/7
#71594EEE
 Taníts meg minket azért kegyelmesen,
 Hogy rövid voltát életünknek értsük,
 És eszességgel magunkat viseljük!
 Ó, Úr Isten, fordulj hozzánk ismétlen!
 Míg hagyod, hogy éltünk nyomorogjon?
 Könyörülj már a te szolgáidon!
/8
#41117EB6
 Tölts bé minket reggel nagy irgalmaddal,
 Hogy jókedvvel vigyük véghez éltünket,
 Ne terheltessünk nyomorúságokkal!
 Vigasztalj minket és adj könnyebbséget,
 És haragodat fordítsd el rólunk,
 Mellyel régóta ostoroztatunk!
/9
#699695F8
 Szolgáidon láttassad dolgaidat,
 Dicsőségedet ezeknek fiain!
 Add értenünk felséges hatalmadat,
 Mi kegyes Urunk, ó, irgalmas Isten!
 Minden dolgunkat bírjad, forgassad,
 Kezeink munkáit igazgassad!

;Isten szárnyainak árnyékában
;(1539) Bourgeois L., Genf, 1542
>91
/1
#E7578C4A
 Aki a felséges Úrnak
 Lakozik oltalmában,
 És e nagy hatalmasságnak
 Nyúgoszik árnyékában,
 Az ilyen nyilván mondhatja:
 Isten az én kővárom,
 Ő életemnek oltalma,
 És csak őbenne bízom.
/2
#319C2A89
 A vadásznak ő tőritől
 Téged megment féltedben;
 A pusztító betegségtől
 Megoltalmaz kegyesen;
 Téged ő kedves szárnyával
 Takargat és béfedez,
 És az ő igazságával
 Mint paizzsal védelmez.
/3
#7DAE36FF
 Azért egyedül Istenben
 Vetem hiedelmemet,
 Aki ül a magas mennyben,
 Abban vesd reménységed,
 És semmi kár téged nem ér,
 Sem nem esel veszélyben,
 És minden gonosz hátra tér,
 Házad felé sem mégyen.
/4
#46968CB8
 Mert az ő szent angyalinak
 Megparancsolta nyilván,
 Hogy téged oltalmazzanak
 Minden te utaidban.
 Tégedet ezek nagy szépen
 Ő kezükben hordoznak,
 Hogy lábad meg ne üsd kőben,
 Oly híven igazgatnak.
/5
#A8B3B06C
 Sárkányon és oroszlánon
 Minden kár nélkül járhatsz,
 Oroszlánkölykön és kígyón
 Lábaiddal tapodhatsz.
 Mond Isten: őtet megtartom,
 Mert engem szívből szeret;
 Én őtet megoltalmazom,
 Mert ismeri nevemet.
/6
#7BA6956E
 Mihelyt hív könyörgésében,
 Őtet menten segítem;
 Véle leszek ínségében,
 Melyből hamar kivészem;
 És nagy dicsőségre őtet
 Emelem, magasztalom,
 És az én segedelmemet
 Őneki megmutatom.

;Az igazságos Isten dicsérete
;Genf, 1562
>92
/1
#BFF4E939
 Ékes dolog dicsérni,
 Uram, Felségedet,
 És a te nevedet
 Énekkel magasztalni.
 Hogy ember áldja reggel
 Te nagy jó voltodat,
 És igazságodat
 Dicsérje minden éjjel.
/2
#F9E5C8AE
 Lantban és hegedűben
 És szép cimbalmokban,
 Hangos citerákban,
 Dicsértessél zengésben!
 Dolgaidon örvendek
 Hatalmadat látván;
 Kezednek csudáján
 Örömömben éneklek.
/3
#2DE01A9D
 Sok és nagy csudálatos
 Te cselekedeted,
 Mélység bölcsességed,
 Beszéded drágalátos,
 E dolgot az esztelen
 Egy szálnyit sem érti,
 És meg sem tekinti,
 Hogy ez miképpen légyen.
/4
#544D403B
 Hogy a gonoszok nőnek,
 Mint a fű a mezőn;
 Virágoznak szépen
 A sok istentelenek,
 Hogy örökké essenek
 A veszedelemben.
 Te vagy örök Isten
 Fölötte mindeneknek.
/5
#78A085C6
 Virágoznak a hívek,
 Mint a szép pálmafák,
 És mint a cédrusfák,
 Mik Libánonon nőnek.
 És az Úr hajlékában
 Ezek plántáltatnak,
 Szépen virágoznak
 Az Isten tornácában.
/6
#EE41A0A0
 Ámbátor megőszülnek,
 De mindazonáltal
 Nagy szaporasággal
 Szép gyümölcsöt teremnek,
 Hogy igazságát híven
 Mindenütt hirdessék
 Az én Istenemnek,
 Kiben hamisság nincsen.

;Isten a világ dicsőséges királya
;Genf, 1562
>93
/1
#1537FEF1
 Nagy hatalmával regnál az Isten,
 Öltözvén felséges erejében,
 E föld kerekségét úgy helyezte,
 Hogy mozdulása nem lesz nékie.
/2
#14AC2218
 Királyi széked kezdettől megvolt,
 Istenséged örökké országolt.
 A folyóvizek erősen zúgnak,
 A magas habok igen harsognak.
/3
#AFFBF1BC
 És bár a tenger zúgjon erősen,
 És a habok hánykódjanak fennen,
 De ők mind semmik az Isten ellen,
 Mert ő hatalmasb a magas mennyben.
/4
#D5FC5776
 Uram, megmarad a te mondásod,
 Merő hűség a te tudományod,
 Házadnak szentség ő ékessége,
 Melynek örökké nem leszen vége.

;Az igazság diadala
;Genf, 1562
>94
/1
#02DD4009
 Ó, erős boszszúálló Isten,
 Ki bűnünkért büntetsz erősen:
 Jelentsd meg már hatalmadat!
 Mind e világnak bírája,
 Támadj fel, add meg valóba'
 A kevélyeknek jutalmát!
/2
#C37B2054
 Míglen marad ez büntetetlen,
 Míg fuvalkodnak föl kevélyen
 A gonosz istentelenek?
 Míg élnek ily vigassággal,
 És az ő gonoszságukkal
 Vajon meddig dicsekednek?
/3
#A3A66B86
 Uram, a te népedet rontják,
 És örökséged sanyargatják
 Minden irgalmasság nélkül.
 Özvegyet, árvát megölnek,
 Szegényt, jövevényt elvesztnek;
 Ezt mondják szentségtelenül:
/4
#F86A09E1
 Az Isten e dolgot nem látja,
 A Jákób Istene nem tudja,
 Amiket mi szerzünk mostan.
 Én csudálkozom rajtatok,
 Hogy ily oktalanok vagytok!
 Fontoljátok meg valóban:
/5
#EDE6B251
 Aki a fület teremtette,
 És a látó szemet szerzette,
 Hogy ne látna, sem hallana?
 A pogányok büntetője
 Titeket hogy ne büntetne:
 Ki mást tanít, hogy ne tudna?
/6
#FF6E5A3B
 Az Isten minden szívnek titkát,
 Jól tudja minden gondolatját:
 Semmirekellők, jól látja.
 Boldog, akit te pirongatsz,
 És törvényedre tanítasz,
 És kinek vagy oktatója.
/7
#A109F127
 Mert nem hagyja Isten ő népét,
 El nem taszítja örökségét,
 Sőt vagyon rájok nagy gondja.
 És midőn ideje eljő,
 Mindent igazán ítél ő,
 És a híveket megtartja.
/8
#65C13887
 Ki ment meg a gonosztól engem?
 És ki támad fel énmellettem
 E gonosztévő nép ellen?
 Ha Isten nem őrzött volna,
 Már régen meghaltam volna,
 És most feküdném a sírban.
/9
#1A460CF8
 Midőn mondom: ím, el kell esnem,
 Legottan megsegítesz engem
 Te nagy irgalmasságoddal.
 Mikor nagy bánatban volnék,
 És szívemben kesergenék,
 Megvigasztalál azonnal.
/10
#5E35AFC5
 Ítéletedhez hogy férhetne
 Az istentelenek törvénye,
 Kik jót hamisra fordítnak?
 Nagy sereggel összegyűlnek,
 Hogy az igazat megöljék
 És ártatlan vért ontsanak.
/11
#E820684F
 De én csak az Istenben bízom,
 Ő énnékem erős kővárom,
 És ezeket megbünteti,
 Az ő számtalan bűnükért
 És gonosztéteményükért
 Az Isten őket elveszti.

;Hívogatás Isten imádására
;Bourgeois L., Lyon, 1547
>95
/1
#1B7397A1
 Jertek, örvendjünk mindnyájan
 Az Úrban, mi kősziklánkban!
 Vigadozzunk szép énekekkel,
 Menjünk színe eleiben,
 És magasztaljuk kegyesen
 Örvendetes dicséretekkel!
/2
#79795C09
 Mert hatalmas az Úr Isten,
 És nagy király mindeneken,
 Ő a fejedelmeknek ura.
 Ez egész földnek mélységét
 És a hegyeknek tetejét
 Hatalmas kezébe foglalja.
/3
#28C9E787
 Övé a tenger, amelyet
 Erős kezével teremtett,
 A szárazt is ő szerzé szépen.
 Az Úrnak, jer, imádkozzunk,
 Néki térdet, fejet hajtsunk,
 Aki minket teremtett bölcsen!
/4
#3F35D951
 Mert ő Istenünk és Urunk,
 Mi pedig juhai vagyunk;
 Ő legeltet minket mint nyáját.
 Lágyítsátok szíveteket:
 Hogy ha hívand ma titeket,
 Vajha meghallanátok szavát!

;Isten a pogányoknak is Ura
;Genf, 1562
>96
/1
#EB977C10
 Énekeljetek, minden népek,
 Új éneket az Úr Istennek!
 E földön őnéki minden
 Dicséretet énekeljen,
 Jóvoltát hirdesse mindennek.
/2
#1175B970
 A pogányok közt dicsőségét,
 Hirdessétek jótéteményét!
 Mert nagy és erős az Isten,
 Hát inkább tisztelje minden,
 Hogynem mint egyéb isteneket!
/3
#694B9E6D
 Nagy dicsőségét látja minden
 Ő tiszteletes szent helyében.
 Jertek tehát, minden népek,
 Adjunk hálát az Istennek,
 Mert nagy az ő dicsőségében!
/4
#D1A1369A
 Az Úrnak dicsőség adassék,
 És az ő neve dicsértessék!
 Adjunk őnéki hálákot,
 Minden kedves ajándékot
 Tornácaiban szent helyének!
/5
#41138659
 Jer, menjünk az Úr eleiben
 És imádjuk szentséges díszben!
 Templomában szentségének
 Tőle mindenek féljenek
 Ez egész föld kerekségében.
/6
#7A8AF6BE
 Örvendjen az ég hangossággal,
 A föld örüljön vigassággal,
 A tenger zúgjon, a mező
 Zengedezzen és a erdő
 Az Úr előtt nagy hál’adással!
/7
#90BB9825
 Eljő az Isten törvényt tenni
 És mind e földet megítélni!
 És igazsággal e földet,
 Tisztasággal a népeket
 Ítéli: a jókat megmenti.

;Isten eljön ítéletre
;Genf, 1562
>97
/1
#99BBEB46
 Az Úr Isten regnál,
 Ő az erős király,
 Kin mind e föld örvendjen,
 Minden sziget örüljön!
 Felhő áll előtte,
 És homály körüle,
 Ő törvényszékinek,
 És ítéletinek
 Áll erős törvénye.
/2
#EDF4C33F
 Tűz megyen előtte,
 Mellyel ellensége
 Szörnyen megégettetik
 És hamuvá tétetik.
 Hírük sem lesz soha,
 Az ő villámlása
 Fénylik világoson
 Ez egész világon,
 A föld fél, ha látja.
/3
#B1363BDC
 Az Isten színének
 Előtte a hegyek,
 Mint viasz, elolvadnak,
 Mert Ura e világnak.
 Hirdetik az egek
 Mindenféle népnek
 Az ő igazságát
 És dicső nagy voltát
 Az ő hatalmának.
/4
#D6FB84AC
 Hát pironkodjanak,
 Kik bálványt imádnak
 És tisztelnek képeket,
 Mikre vetik szívüket!
 Ti minden istenek,
 Őtet tiszteljétek,
 Állván széke előtt,
 Kit a Sion hallott,
 És örült e hírnek.
/5
#D6A5A1D9
 Ti, Istent szeretők,
 Gonoszt gyűlöljetek,
 Hogy részetek ne légyen
 Hamis cselekedetben.
 Mert az ő szolgáit,
 Megmenti híveit
 A gonosz kezéből;
 Ő nagy erejéből
 Megtartja népeit.
/6
#13F1B813
 Szentihez világát,
 Nyújtja nagy irgalmát;
 Kik egyenes szívűek,
 Tőle örömet nyernek.
 Szent hívek, ez Úrban
 Örvendjetek vígan,
 És az ő szentségét,
 Dicsőséges nevét
 Áldjátok mindnyájan!

;Népek szabadítója az Úr
;Bourgeois L., Strasbourg, 1545
>98
/1
#12D5ABC9
 Énekeljetek új éneket
 Az Úr Istennek örömmel,
 Mert nagy csudákat cselekedett
 Karjának nagy erejével.
 Mivelünk az ő üdvösségét
 Kétség nélkül megérteté,
 Szent igazságát, kegyességét
 Minden népnek kijelenté.
/2
#F3CACFB7
 Meggondolá irgalmasságát,
 És kegyességét tekinté,
 És az ő nagy hűséges voltát
 Az Izráelhez téríté.
 A nekünk küldött üdvösséget
 Ez egész földkerekségen
 Nyilván meglátta minden nemzet.
 Örvendjen néki hát minden!
/3
#9CB58B5F
 Örvendjetek és vigadjatok,
 Mondjatok szép zsoltárokat!
 Cimbalmokkal hangicsáljatok,
 Zendítsetek citerákat!
 A trombitákat fújjátok meg,
 E király előtt zengjetek!
 Csendüljetek, zúduljatok meg
 Tengeren-földön mindenek!
/4
#14453B7A
 Az Úr előtt a folyóvizek
 Örvendezzenek mindnyájan,
 A magas hegyek az Istennek
 Tapsoljanak víg voltukban!
 Mert íme, eljő törvényt tenni,
 Megítél e földön mindent,
 Mind e világot jól rendeli,
 A tiszta igazság szerint!

;A háromszor szent Isten dicsérete
;Genf, 1562
>99
/1
#CD9948CB
 Az Úr országol
 És regnál nagy jól,
 A nép megrémül,
 Hogy ő ott fenn ül
 A Kérubimon,
 Ki előtt nagyon
 Félnek és rettegnek
 E földön mindenek.
/2
#E1C2467C
 Nagy az Úr Isten
 Ő erejében.
 A Sion hegyen
 Minden népeken
 Vagyon hatalma,
 Minden őt áldja,
 Mert nagy az ő neve
 És dicső szentsége.
/3
#D7C8A0FA
 Fönn a felhőben
 Oszlop képében
 Őket vezérlé
 A puszta helybe’,
 Kik ő törvényét
 És szent igéjét
 Híven megőrizték,
 Frigyét kedvellették.
/4
#EEF683A6
 Meghallád, Isten,
 Ő kérésükben,
 Hozzájuk térél,
 És megengedél
 Kegyességedből.
 De hogy bűnükből
 Ők ki nem térnének:
 Megbüntettetének.
/5
#A8F7B126
 Áldjátok őtet
 Mint Istenünket,
 Térdet hajtsatok
 És imádjátok!
 A Sion hegyén,
 Ő lakóhelyén,
 Dicsértetik itten,
 Mert szent az Úr Isten!

;Hálaének a templomban
;Bourgeois L., Genf, 1551
>100
/1
#84B801BA
 E földön ti minden népek,
 Az Istennek örvendjetek,
 Előtte szép énekekkel
 Szolgáljátok őt víg szívvel!
/2
#54092149
 Tudjátok, hogy ez az Isten,
 Ki minket teremtett bölcsen,
 És mi vagyunk ő népei
 És ő nyájának juhai.
/3
#28F86FCE
 Ő kapuin menjetek be,
 Hálát adván szívetekbe’!
 Jer, menjünk be tornácába,
 Néki nagy hálákat adva!
/4
#4CEFD318
 Mert nagy az ő kegyessége,
 És megmarad mindörökre,
 És ő hűsége mindenha
 Megáll és el nem fogy soha.

;Uralkodó tüköre (Messiási zsoltár)
;Bourgeois L., Strasbourg, 1545
>101
/1
#5A9391E5
 Mindennek előtte irgalmasságról,
 Lészen éneklésem az igazságról,
 És dicséretet mondok szüntelen
 Néked, Isten.
/2
#F3333F5A
 Okossággal járok minden utamban,
 Vajon mikor jössz már el, Isten, hozzám?
 Hogy házamat híven vezérlhessem,
 Igyekezem.
/3
#58C095F9
 Én semmi gonosz dolgot nem kedvelök,
 De minden csalárdságot én gyűlölök,
 Ezekre semmiképpen nem vetem
 Az én kezem.
/4
#0634EED2
 A hamis szívű távol menjen tőlem,
 A gonosz emberhez nem lészen kedvem,
 Nem jön a csalárd ember előmben
 Semmi helyen.
/5
#A1A3BF24
 Titkon ki ő feleit rágalmazza,
 Nem lészen nálam annak maradása,
 A fuvalkodó kevélyt előttem
 Nem szenvedem.
/6
#9818FD00
 Szemeim inkább azokra nézzenek,
 Akik e földön igazságban élnek,
 Lakjanak nálam, és mint hű szolgák,
 Szolgáljanak.

;Imádság Sion helyreállításáért (Ötödik bűnbánati zsoltár)
;Genf, 1562
>102
/1
#00536D4B
 Hallgasd meg, Uram, kérésem,
 Tekintsd meg esedezésem!
 Beszédem jusson hozzád,
 Ne rejtsd el tőlem orcád!
 Hajtsd énhozzám te füledet,
 Enyhítsd meg nagy ínségemet!
 Midőn kiáltok, Úr Isten,
 Siess, hallgass meg kegyesen!
/2
#DF7D89D4
 Mert napjai életemnek
 Oly hirtelen elkelének,
 Mint a füst és a pára,
 És mint a tűzhely pora.
 Minden csontom úgy elszáradt,
 Szívem, mint a fű, elhervadt,
 Úgy, hogy az én ételemet,
 Elfelejtem kenyeremet.
/3
#782F7353
 Porhamu kenyerem nékem,
 Melyet étel gyanánt észem;
 Italom könnyeimmel
 Elegyítem, mint vízzel,
 A te nagy haragod miatt,
 Melynek tüze úgy fellobbant,
 Hogy engemet fölemeltél,
 Ismét a földhöz ütöttél.
/4
#B2742E02
 Az én időm úgy elmúlék,
 Mint az árnyék, elenyészék;
 Minden testem elasza,
 Mint a lekaszált széna,
 Amely meg nem éled többé;
 De te megmaradsz örökké,
 És a te emlékezeted,
 Mindörökre megtart híred.

;A könyörülő Isten
;(1539) Bourgeois L., Genf, 1542
>103
/1
#6C2131A7
 Áldjad, lelkem, Uradat, Istenedet,
 Minden énbennem dicsérje szent nevét
 És az ő mondhatatlan jóvoltát!
 No, dicsérd, lelkem, és az Urat áldjad,
 Feledékenységben el ne hallgassad
 Ő jótéteményinek sok voltát.
/2
#06B05BF5
 Adj hálát néki, aki bűneidet
 Megbocsátja, gyógyítja sérelmidet,
 Kiment minden nagy bajodból híven,
 És életedet veszélytől megmenti,
 A halál veszedelmitől megőrzi:
 Irgalmával megkoronáz szépen.
/3
#853EE45E
 Aki lelkedet kegyesen táplálja,
 Ami kell szádnak, bőséggel megadja;
 Mint a sast, megifjít és megújít.
 És akik méltatlanságot szenvednek,
 Tőle kegyesen azok megmentetnek,
 Hozzájuk nyújtja az ő jókedvit.
/4
#063CD6A1
 Mint az atya fiaihoz kegyelmes,
 Ő is azokhoz igen engedelmes,
 Kik őt igazán félik, tisztelik,
 Mert jól tudja, mily gyarló a mi voltunk,
 És hogy mi oly romlott emberek vagyunk,
 Mint a por, mely a széltől hintetik.
/5
#4222C200
 Ember élete a fűhöz hasonló,
 Felnő és zöldül, de hamar elmúló,
 Mint a gyenge virág a sík mezőn,
 Melyet mihelyt megfúval a meleg szél,
 Elhull és hervad, ékessége elkél,
 Ember nem tudja, hol volt, hova lőn.
/6
#B82CA30D
 De az Úr kegyelme örökké megáll
 Azokon, kik őt félik igazsággal,
 És firól fira terjed irgalma
 Azokon, kik megtartják ő kötésit,
 Akik gyakran megemlítik törvényit,
 És azok szerint járnak mindenha.
/7
#7FBD1183
 Dicsérjétek őt, minden ő seregi,
 Kik a mennyekben szolgáltok őnéki
 És cselekszitek szent akaratját!
 Áldjátok az Urat, minden állatok,
 És birodalmát fenn magasztaljátok,
 Örökké áldjad, lelkem, az Urat!

;A teremtő dicsősége
;Bourgeois L., Genf, 1542
>104
/1
#4B8447DE
 Áldjad, lelkem, az Urat és tisztöld,
 Dicsőségével rakva menny és föld.
 A te felséged, Uram, nagy és erős,
 A te ékességed nagy szép és fényös.
 Te öltözeted ékes és tiszta,
 Szép világosság származik róla.
 Az egeket szélesen kiterjesztéd,
 Mint egy kárpitot, úgy felékesítéd.
/2
#1C36AB81
 A vizet körüled, mint kamarát,
 Jól megépítéd, mint szép palotát.
 A felhőkön úgy jársz, mint egy szekéren,
 A szelek szárnyukon hordoznak szépen.
 Angyalidat sebes széllé teszed,
 Száguldó postáidként kiküldöd;
 Mennydörgés, tűzláng, villámlás előtted,
 Mint kész szolgáid, úgy függnek tetőled.
/3
#C0E00F94
 A föld fundámentomát megvetéd,
 Amelyre erősen helyeztetéd,
 Hogy azon mindenkoron megállana,
 És helyéből soha ki nem mozdulna.
 Mely azelőtt a nagy mélységekben
 Vízzel mint ruhával volt elfödvén;
 Őrajta a nagy vizek felül folytak,
 Kik a nagy hegyeken is felülmúltak.
/4
#9FD13797
 De mihelyen te megfeddéd őköt,
 Megfutamának feddésed előtt;
 Hogy meghallák mennydörgését te szódnak,
 A földről sietvén elszaladának.
 A nagy hegyek fölemelkedének,
 És a mély völgyek mind kitetszének;
 Minden megtartja ő tulajdon helyét,
 Melyet Felséged nékiek engedett.
/5
#9AED47F1
 Az Úrnak légyen örök tisztesség,
 És övé légyen minden dicsőség!
 Örvend az Úr ő csuda dolgaiban,
 Gyönyörködik ő minden munkáiban.
 Tekintésétől a föld megrémül,
 És az ő haragjától megrendül;
 Reszketvén a nagy hegyek füstölögnek,
 Hogyha az Úrtól ők megillettetnek.
/6
#5F14533D
 Dicséretet az Úrnak éneklek,
 Valamíg én e világon élek;
 Az Úr Istent én egész életemben
 Dicsérem és áldom szép éneklésben.
 De viszontag azt kérem őtőle,
 Hogy éneklésem jókedvvel vegye,
 És aztán teljes szívből örvendezek,
 Szép énekeket mondván szent nevének.

;Isten csodatettei velünk
;Genf, 1562
>105
/1
#76A1E029
 Adjatok hálát az Istennek,
 Imádkozzatok szent nevének!
 Hirdessétek dicséretét
 És minden jótéteményét!
 Beszéljétek a nép előtt
 Nagy csudáit, melyeket tött!
/2
#800921D6
 Néki vígan énekeljetek,
 Sok csuda dolgát dicsérjétek!
 Magasztaljátok szent nevét,
 Kik szívből félitek őtet!
 Örvendjen azoknak szívek,
 Kik az Úrról emlékeznek!
/3
#C289F61F
 Keressétek e kegyes Urat
 És az ő színét és hatalmát!
 Meggondoljátok dolgait,
 Ne felejtsétek csudáit!
 Ítéletit hirdessétek,
 Melyek ő szájából jöttek!
/4
#6C30E6CD
 Népét vígsággal ő kihozta,
 Választott népét vigasztalta.
 A pogányok tartományát,
 Ezeknek adta országát,
 Mit kezükkel munkálkodván,
 Szerzettek volt ez országban.
/5
#8AD9CB68
 Ezt nékiek azért mívelte,
 Hogy gondjuk légyen törvényére,
 Hogy fogadják meg ő szavát,
 Megtartsák parancsolatát,
 És örökké megőrizzék,
 Melyért dicséret Istennek!

;Isten kegyelme népéhez
;Genf, 1562
>106
/1
#64C36D54
 Az Urat áldjátok, mert jó,
 Irgalma örökkévaló!
 Vajon kicsoda mondhatná ki
 Az ő nagy erős hatalmát?
 Sok és nagy az ő dicséreti,
 Melynek ki tudhatná számát?
/2
#A2F8DF42
 Boldog, aki az Úr szavát,
 Megőrzi parancsolatát.
 Uram, énrólam emlékezz meg!
 Népedhez való kedvedért,
 Kérlek, engemet látogass meg
 Üdvözítő szerelmedért!
/3
#13C23C11
  Javaival hogy élhessek
 Te választott híveidnek,
 És hogy szívem örvendezhessen
 Örömén a te népednek,
 És örökséged örömében
 A te népeddel örvendjek.
/4
#9BB12E52
 Az Úr felmagasztaltassék,
 Istene az Izráelnek:
 Dicsértessék az ő szent neve!
 És hogy örökké úgy legyen,
 Minden nép ígyen szóljon erre:
 Az Úrnak dicsőség! Ámen.

;A megváltottak hálaéneke
;Bourgeois L., Lyon, 1547
>107
/1
#D8689469
 Dicsérjétek az Urat,
 Mert nagy ő jóvolta,
 És örökké megmarad
 Az ő nagy irgalma.
 Akik megváltattak
 Őáltala kegyesen,
 Kíntól megtartattak,
 Őtet dicsérjék híven.
/2
#64FFD686
 Kiket ő támadatról
 És napenyészetről,
 Dél felől és északról
 Bégyűjte sok földről,
 Kik a szörnyű pusztán
 Idestova bujdostak,
 Semmi várost ottan
 Lakásra nem találtak.
/3
#53FD135D
 Holott nekik nem vala
 Ételük, italuk,
 Min lelkük elbágyada,
 Nagy bú szállott rájuk.
 Ez ő ínségükben
 Istenhez kiáltának,
 Kitől nagy kegyesen
 Megszabadíttatának.
/4
#EEBF6CEB
 És igaz úton őket
 Nagy szépen hordozá,
 Holott lelnének helyet,
 Városra juttatá.
 Ezek hát víg szívvel
 Az Úr Istent dicsérjék,
 Minden népnek széjjel
 Nagy csudáit beszéljék.
/5
#4C6C3494
 Ezek áldják az Istent,
 Dicsérjék kegyelmit,
 Minden nép közt óránként
 Hirdessék csudáit.
 Istennek szívükből
 Áldozzanak hűséggel,
 Csudatételiről
 Énekeljenek széjjel.
/6
#A43D3D6A
 Kegyesen fölemeli
 Ő a nyomorultat,
 Cselédit kiterjeszti,
 Mint a sereg nyájat.
 A jók, kik ezt látják,
 Örvendez az ő szívek;
 De szájukat dugják,
 Akik gonoszul élnek.

;Isten jósága és hűsége (Reggeli ének)
;Bourgeois L., Genf, 1562
>108
/1
#404BB608
 Úr Isten, kész az én szívem,
 És azon vagyon én lelkem,
 Hogy tenéked énekeljen,
 Dicséretet zengedezzen.
 Nosza, lantok és citerák,
 Zendüljetek fel muzsikák,
 Mert igyekezem jó reggelen
 Így menni az Úr eleiben.
/2
#BFBCECE7
 Dicsérlek, Uram, tégedet
 Minden nemzetségek előtt,
 Tisztellek szép énekekkel
 Minden nép előtt víg szívvel.
 Mert a te kegyelmességed
 A széles égre kiterjed,
 Felségednek szent igazsága
 A felhőket mind felülmúlja.
/3
#8FE133F7
 Légy minekünk segítségül,
 Őrizz meg ellenséginktül,
 Mert az emberi segítség
 Hiábavaló epedség.
 Az Isten által minekünk
 Lészen erős győzedelmünk,
 És megszabadít ő bennünket,
 Megtapodja ellenségünket.

;Panasz az istentelenek ellen
;Bourgeois L., Genf, 1551
>109
/1
#CD3C4961
 Ó, Úr Isten, én dicsőségem,
 Ne hallgass, ne felejts el engem!
 Mert rágalmaz az istentelen,
 Száját reám tátotta szörnyen,
 Hazugságot szól ellenem,
 Nyelvével sérteget engem.
/2
#FF638398
 Ok nélkül rólam gonoszt szólnak,
 És nagy ellenségüknek tartnak;
 Azért, hogy én őket szerettem,
 Kegyetlenül gyűlölnek engem;
 Én csak Istenhez szüntelen
 Fohászkodtam ez ínségben.
/3
#C4F95E87
 Irgalmaddal biztatom lelkem,
 Szent nevedért őrizz meg engem,
 Mert szegény szűkölködő vagyok,
 Én szívemnek fájdalmi nagyok!
 Ím, el kell múlnom hirtelen,
 Mint az árnyék a setétben.
/4
#20105610
 Ez én nagy keserűségemben
 Csúfolnak és gyaláznak szörnyen;
 Fejüket rázzák, midőn látnak,
 És engem gúnyolnak, bosszantnak.
 Azért, Úr Isten, segíts meg,
 Nagy kegyességedért tarts meg!
/5
#593DEE21
 Az Úr Istent én az én számmal,
 Dicsérem szép énekmondással,
 Magasztalom őtet szüntelen,
 Mert ő könyörül a szegényen,
 És azok ellen megtartja,
 Akik ítélik halálra.

;Krisztus örök királysága és papsága
;Bourgeois L., Genf, 1551
>110
/1
#16796F15
 Az Úr Isten monda az én Uramnak:
 Ülj az én hatalmamnak jobbjára,
 Míg ellenségidet, kik rádtámadnak,
 Zsámolyul vetem lábadnak alá.
/2
#62AD5F36
 Dicsőségére a te szentségednek
 A nép örvendez győzedelmeden.
 Oly sok fiaid tenéked születnek,
 Mint a hajnali harmat a földön.
/3
#C2D6028A
 Mert az Úr Isten megesküdt tenéked,
 Melyet meg nem bán soha örökké,
 Melkisédeknek rendi szerint (értsed)
 Te vagy a főpap most és örökké.
/4
#27485735
 Az Úr, aki ül a te jobb kezeden,
 Ha megharagszik egykor valóban,
 A hatalmas királyokat erősen
 Hatalmával megrontja legottan.
/5
#5E172F50
 A pogányokon ítéletit tartja,
 Megtölti a földet holttestekkel,
 Ellenségidnek fejüket megrontja,
 Országa kihat e földre széjjel.
/6
#2BA668A3
 Az úton iszik a tiszta patakból,
 Melynek vize foly nagy harsogással,
 Ennek okáért ő nagy hatalmából
 Emeli fejét nagy méltósággal.

;Hálaének Isten testi és lelki áldásaiért
;Bourgeois L., Lyon, 1547
>111
/1
#403BA802
 Hálát adok, Uram, néked,
 Teljes szívből áldlak téged
 A hívek gyülekezetében;
 Megvallom nagy dicsőséged
 És hirdetem dicséreted
 Életemnek minden rendében.
/2
#058F4675
 Nagyok az Úrnak csudái,
 És aki azt megtekinti,
 Örvendez annak az ő szíve.
 És az ő szent igazsága
 És ő dicső méltósága
 Megmarad mind örök időkre.
/3
#6900C1FC
 Csuda dolgait az Isten
 Szerzette emlékezetben
 Kegyes és irgalmas kedvéből.
 Azoknak ételt ád bőven,
 Akik őtet félik híven;
 Megemlékezik kötéséről.
/4
#F19A4E27
 Népével nagy csudákat tett,
 Hogy a pogányok örökét
 Kezükbe adá őnékiek.
 Merő hűség és tisztaság
 És állhatatos igazság
 Minden dolga az ő kezének.
/5
#42050E04
 Igaz minden ő hagyása,
 Minden ő parancsolata,
 Melyben semmi változás nincsen.
 Az ő népét ő megmenté,
 És ővélük kötést szerze,
 Mely megmarad minden időben.
/6
#74E1D912
 Szent és dicső az ő neve,
 És az Úrnak ő félelme,
 A jó bölcsességnek kezdete.
 Ki megtartja ő törvényét,
 És megőrzi szent Igéjét,
 Annak megmarad dicsérete.

;Az istenfélők boldogsága
;Genf, 1562
>112
/1
#41D976BB
 Boldog az ember, ki az Istent
 Féli, tiszteli szíve szerént,
 És az ő törvényét szereti.
 Nagy lesz e földön ő nemzete,
 Öregbül a hívek serege,
 Mert az Úr megáldja és őrzi.
/2
#CEADEEDC
 Gazdagsággal őtet meglátja,
 Mellyel bővölködik ő háza,
 Áll igazsága mindörökké.
 A híveknek a sötétségben
 Támaszt világot a jó Isten,
 Hogy láttassék rajtuk kegyelme.

;Az alázatosokat Isten felemeli
;Bourgeois L., Genf, 1551
>113
/1
#2BC53AF5
 Az Urat ti, ő szolgái,
 Dicsérjétek, mert érdemli,
 Áldjátok szent nevét mindnyájan!
 Dicsértessék szent felsége,
 Most és örökkönörökké
 Ő szent neve áldassék tisztán.
/2
#411BECBE
 Napkelettől enyészetig
 Áldassék neve mindvégig,
 Mert az Úr Isten a mennyekben
 Regnál minden pogányokon,
 Nagy dicső hatalma vagyon,
 Mely felülhat a szép egeken.
/3
#33E2FAD6
 De ki volna hasonlatos
 E mi hatalmas Urunkhoz,
 Kinél felségesb sehol nincsen?
 Aki a mennyből alánéz
 Mindenre, ami van és lesz
 Itt e földön és fenn az égben.
/4
#309543AD
 A szegényt porból felveszi,
 És a sárból felemeli,
 Állapotát felmagasztalván.
 Felülteti végezetre
 A nagy fejedelmek közé
 Az ő népe közt igazában.
/5
#34A8A261
 Az asszony szomorúságát
 Ő magtalansága miatt
 Nagy vigasságra megfordítja.
 Gyermekek anyjává tészi,
 Szép fiakkal körülvészi,
 Házát gyümölccsel szaporítja.

;Isten csodálatos szabadítása
;(1539) Bourgeois L., Genf, 1542
>114
/1
#5CD154E6
 Hogy Izráel kijött Egyiptomból,
 Az idegen népnek országából,
 Megtére Jákób háza.
 Júdát Isten magának szentelé,
 Az Izráelt országul fölvevé,
 Ő lőn nékie Ura.
/2
#A17368F6
 A tenger ezt látván hátra álla,
 A Jordán vize visszafordula,
 Mind hátra sietének.
 A hegyek szökdöstek, mint a kosok,
 És a halmok, mint a juhbárányok,
 Magasan szökdösének.
/3
#E12EF16F
 Mi lelt téged, tenger, mit térsz hátra?
 Mi lelt téged, Jordán, ki űz vissza,
 Hogy elszaladsz ily igen?
 Mit szöktetek, hegyek, mint báránykák,
 És ti halmok, mint a kis juhocskák,
 Mért szöktök ilyen fennen?
/4
#A2A32D83
 Az Úrnak haragos színe előtt,
 Jákób Istene haragja előtt
 Mind e föld megrettenjen!
 Ki a kősziklát tóvízzé teszi,
 Forrásnak útját kőben repeszti
 Hatalmas erejében.

;Egyedül Istené a dicsőség!
;Bourgeois L., Genf, 1542
>115
/1
#4C394E0A
 Nem nekünk, Uram, nem nekünk engedd,
 Hanem adj nevednek dicsőséget
 A te nagy hűségedért!
 Mit csúfolnának a pogány népek,
 Mondván: hol vagyon az ő Istenek,
 Ki megmentené őket?
/2
#B56DBAED
 De Istenünk ő nagy erejével,
 Amit csak akar, mindent megmível
 Mind mennyen és e földön.
 De sok bálványuk a pogányoknak
 Aranyból, ezüstből csináltattak
 Embereknek kezében.
/3
#DA658D91
 De te, Izráel, Istenben bízzál,
 És az Úr Istenhez ragaszkodjál:
 Ő paizsod tenéked!
 Áronnak háza, Istenben bízzál,
 A nagy Úr Istenhez ragaszkodjál:
 Ő megsegíthet téged!
/4
#F41CE935
 Istenfélők, bízzatok Istenben,
 Higgyetek az ő segedelmében,
 Bízzatok e paizsba’!
 Megemlít az Úr, mert szeret minket,
 És Izráelhez nyújtja kegyelmét,
 Áron házát megáldja.
/5
#8C3E3B2F
 Ő megáldja a kicsinyt, a nagyot,
 Kik őtet félik, mint ő Urokot,
 És kik szolgálják őtet.
 Az Úr titeket bőven megáldjon,
 Sok áldásival megszaporítson,
 És minden nemzedéket!
/6
#0EFB0717
 Megáld a nagy Úr Isten titeket,
 Ki teremté a mennyet és földet
 Minden szép ékességgel.
 Ő magának az eget megtartja,
 Földet az emberfiaknak adja,
 Hogy azt meglakják széjjel.

;Hála és fogadástétel halálból való szabadításért
;Genf, 1562
>116
/1
#59307324
 Szeretem és áldom az Úr Istent,
 Mert meghallgatá az én beszédemet,
 Könyörgésemre hajtá kegyes fülét,
 Melyért imádom őtet naponként.
/2
#757E3752
 Midőn a halál körülvőn engem
 És csaknem szörnyű kötelibe ejte,
 A pokol kínja engemet rettente,
 Nagy volt bánatom és nagy sérelmem:
/3
#7D0953B0
 Segítségül hívám az Úr nevét:
 Lelkemet tartsd meg! ottan megsegíte;
 Az Úr igaz, hív, és nagy ő jó kedve,
 Ő megőrzi az együgyűeket.
/4
#7D55B771
 Nagy nyavalyámban mikor én valék,
 Ottan megmenté nyomorult életem.
 Légy csendességben te azért, én lelkem,
 Látván kegyelmét szent felségének!
/5
#EFC35A3A
 Te megmentél a haláltól engem,
 Szemem sírástól, lábamat eséstül;
 Az élők földén járok szünetlenül
 A te színed előtt, én Istenem.
/6
#7B98C6E8
 Hittem Istenben, mikor így szólék,
 Én szegény lelkem vala nagy ínségben,
 És ezt mondám én csüggedezésemben,
 Hogy hazugok már minden emberek.
/7
#154EB5A1
 Mit adjak az Úrnak jótettiért?
 A hálaadó pohárt én felveszem,
 És az Úr jótéteményét hirdetem,
 Szent nevét áldom segítségiért.
/8
#F225CD84
 Fogadásom az egész nép előtt
 Hálaadással megadom nékie.
 A híveknek halála, minden higgye,
 Drágalátos az Úr szeme előtt.
/9
#2E8B8B71
 Imádlak téged, Idvezítőmet,
 Ki engem választál szegény szolgádnak;
 Nékem, szolgáló leányod fiának,
 Föloldozád minden kötelimet.
/10
#75540F5D
 Áldozom néked hálaadással,
 Nevedet minden nép előtt hirdetem,
 És fogadásom, mit előtted tettem,
 Megállom minden előtt vígsággal.

;Felhívás Isten dicséretére
;Bourgeois L., Genf, 1551
>117
/1
#9201407A
 Az Urat minden nemzetek,
 Dicsérjétek minden népek,
 Mert nagy az ő kegyessége,
 Mit rajtunk megerősíte,
 És igazsága mindenha
 Áll és marad. Alleluja.

;Isten népének öröméneke (Templomszenteléskor)
;Bourgeois L., Strasbourg, 1545
>118
/1
#689D8BA6
 Adjatok hálákat az Úrnak,
 Mert nagy az ő kegyessége,
 És nagy volta szent irgalmának
 Megmarad most és örökre!
 Izráel, bátorsággal mondjad,
 Hogy megáll kegyelmessége;
 Irgalmas voltáról azt valljad,
 Hogy megmarad mindörökre!
/2
#84BA5106
 Az Úrhoz én nagy ínségemben
 Könyörögvén felkiálték,
 És meghallgata kérésemben,
 Tőle segedelmet nyerék.
 Az Úr Isten vagyon énvélem,
 És tőlem másuvá nem tér,
 Hát kitől kelljen nékem félnem,
 Mit árthat nékem az ember?
/3
#86C9AB85
 Az Úr énnékem erősségem,
 És én őróla éneklek;
 Csak ő énnékem segedelmem,
 És én őbenne reménylek.
 A hívek vígan énekelnek
 Az ő hajlékukban széjjel,
 Mert jobb keze az Úr Istennek
 Nagy erős dolgokat mível.
/4
#1E6A055D
 Az Úr megbüntete engemet,
 És igen megostoroza,
 De nem akará elvesztemet,
 És a halálnak nem ada.
 Nyissátok meg azért kapuit
 Az igazság templomának,
 Hogy bemenvén, nagy dicsőségit
 Dicsérhessem e nagy Úrnak.
/5
#8214B456
 E kő, amit a házépítők
 Ítéltek megvetendőnek,
 Az épületbe helyezteték,
 Lőn feje a szegeletnek.
 Ez pedig az Úr Istentől lett,
 Aki ezt ígyen rendelte,
 E dolognak szemeink előtt
 Csudálatos hossza-vége.
/6
#F4DD23E8
 Áldott, aki az Úr nevében
 Eljöve nagy dicsőséggel!
 Áldunk, dicsérünk egyetemben
 Az Úrnak háza népével!
 Adjatok hálát e nagy Úrnak,
 Mert nagy az ő kegyessége,
 És nagy volta szent irgalmának
 Megmarad most és örökre!

;Az Úr Igéjének és törvényének dicsősége
;Aranyábécének is szokták nevezni, mivel héber versei a bibliában nyolcanként – itt négyenként – egyforma betűvel kezdődnek ábécérendben
;Bourgeois L., Genf, 1551
>119
/1
#73BB63D4
 Az oly emberek nyilván boldogok,
 Kik igazsággal járnak életökben,
 Isten törvényére vagyon gondjok,
 És aszerint élnek minden időben,
 Szent bizonyságit akik megőrzik,
 És az Istent szívök szerint keresik.
/2
#E7106D08
 Boldogok azok is, mondom nyilván,
 Akik hamisságot nem cselekesznek,
 De mindenkor az Úr útaiban
 Járnak és szent ártatlanságban élnek.
 Meghagytad, hogy a te parancsodat,
 Jól megőrizzük minden mondásodat.
/3
#F40C18E5
 Vajha én oly boldoggá lehetnék,
 Hogy járhatnék a te szent útaidban
 És engedhetnék szent törvényednek!
 Ha parancsodat nézhetném valóban,
 És azt szívemben bizonnyal hinném,
 Hogy soha semmi szégyenbe nem esném!
;Boldog az ifjú, ki az Urat féli
;
/4
#C72FA2E5
 Hálát adok néked teljes szívből,
 Hogy megtanítasz te ítéletidre,
 Melyek tiszták minden hiba nélkül!
 Megtartom és gondom lesz törvényidre,
 De kérlek téged, ó, én Istenem,
 Hogy soha örökké ne hagyj el engem!
/5
#E9160D9E
 Beszéld meg, mit tégyenek az ifjak,
 Hogy élhessenek ők feddhetetlenül?
 Szent Igéd szerint útjukat szabják.
 Én téged kerestelek szüntelenül;
 Kérlek, Úr Isten, teljes szívemből:
 Eltévelyednem ne hagyj törvényedtől.
/6
#6D26EC9A
 A te igédet rejtem szívemben,
 Hogy semmi bűnnel ne bántsalak téged,
 De megmaradjak te ösvényedben,
 Minden dolgomban megtartom törvényed. Ó, áldott Isten, taníts engemet, Hogy igazán értsem rendelésedet!
/7
#F473E140
 Ítéletedet én ajakimmal,
 És a te szádnak ő kegyes beszédét
 Előszámlálom hálaadással,
 Szent kötéseddel biztatom szívemet;
 Bizonyságidon örvendez lelkem,
 Mik gazdagságnál kedvesebbek nékem.
/8
#FE1BCA12
 Szüntelen nékem gyönyörűségem
 Vagyon csak a te parancsolatidban,
 Te útaidat gyakran említem,
 Hogy el ne essem valaha azokban.
 Szent igazságodban minden kedvem,
 És te ösvényed én el nem tévesztem.
;Csak Isten törvényére tekintek
/9
#778AA346
 Cselekedd ezt szolgáddal kegyesen,
 Hogy én élhessek tovább e világban
 És szent igédet megtartsam híven!
 Én szemeimet nyisd meg világosan,
 Hogy a te törvényed megtekintsem
 És annak csudáit eszembe végyem!
/10
#5F118736
 Míglen én e földön járok-kelek,
 Ne rejtsd el tőlem parancsolatidat,
 Kívánság miatt mert elepedek,
 Igen óhajtom szent igazságodat!
 Ítéletedhez az én szívemben
 Nagy kívánságom volt minden időben.
/11
#174AC7D5
 A kevélyeket, Uram, megrontod,
 Átkozottak és büntetésre méltók,
 Akik megvetik parancsolatod.
 Forduljon el rólam ő gyalázatjok,
 Kik csak azért gyalázzák szolgádat,
 Hogy megőrzöm a te bizonyságidat.
/12
#FCCE6E05
 A fejedelmek énreám törnek,
 Ha összegyűlnek, de a te hű szolgád
 Szentségét nézi ítéletednek,
 Amely szívemnek igaz örömet ád.
 Bizonyságid nékem vigasságim
 És minden dolgomban tanácsadóim.
;Drága orvosság az Úrnak beszéde
;
/13
#FB009F17
 De lám, a porban hever életem,
 Mintha vitetném majd a koporsóba;
 Szent igéd szerint élessz meg engem!
 Midőn útaimat előszámlálva
 Felkiálték, te legott feleltél;
 Rendelésidre taníts meg jókedvvel.
/14
#90543281
 Add értenem parancsolatidat,
 Hogy elmélkedjem a te csudáidrul!
 És elmémben foglalom azokat,
 Szívem keserűség miatt kibuzdul!
 Ígéreted szerint segíts engem,
 Hogy tőled ismét megerősíttessem!
/15
#D4572203
 A hamis útról, Uram, téríts el,
 Törvényeidnek vezérelj útára,
 Min ember járhat szép csendességgel!
 Juttass kegyesen szent igazságodra!
 Ítéletedet én elválasztom,
 És igazságod szemem előtt tartom.
/16
#7691C4EB
 Bizonyságidra hajtom szívemet,
 És életemet a szerint rendelem.
 Szégyenvallástól ments meg engemet!
 Mivel most kitárod bennem én szívem,
 Parancsolatidra lesz nagy gondom,
 És víg örömmel azokat megfutom.
;Életnek ösvényén vezet a Úr
;
/17
#F1EDBC5D
 Én Istenem, taníts útaidra,
 Hogy szent törvényedre gondot viseljek,
 És azokat megtartsam mindenha!
 Adj értelmet, Uram, igazgass, kérlek,
 Hogy törvényedet őrizzem híven,
 És mindenkor megtartsam én szívemben.
/18
#6E9EBD03
 Vezess, hogy benned leljem örömöm,
 Mutasd meg parancsolatid ösvényét,
 Mert azokban igen gyönyörködöm!
 Te rendelésedre hajtsad szívemet!
 Szent bizonyságid végyem eszembe,
 És ne hagyj esnem telhetetlenségbe!
/19
#ACB39DF5
 Fordítsad el az én szemeimet,
 Hiábavalókat hogy ne nézzenek;
 A te utadban éltess engemet!
 Szolgáddal láttasd szent ígéretednek
 Bétöltését, aki téged tisztel
 És mindenkor fél alázatos szívvel.
/20
#F71916BC
 Végy el rólam minden gyalázatot,
 Melytől én igen félek és rettegek!
 Ítéleteid jók, és azokat
 Én megtanulni igen örvendezek.
 Gyönyörködik törvényedben szívem,
 És igazságban éltess, Uram, engem!
;Foglaljam szívembe az Úr törvényét
;
/21
#A1012B76
 Forduljon hozzám, Uram, kegyelmed,
 És segedelmed adjad, hogy láthassam!
 Ígéretedből vélem ezt tegyed,
 Hogy szájukat azoknak bedughassam,
 Akik engemet gyaláznak szörnyen,
 Mert bízom a te szent ígéretedben!
/22
#816523B5
 Szent igaz igéd ne vedd el tőlem,
 Hogy az mindenkor legyen az én számban!
 A te beszéded én reménységem,
 Te törvényedet szívemben foglalván.
 És azt megtartom én minden módon,
 A szerint élvén most és mindenkoron.
/23
#B7F9F3E6
 Szüntelen járok én nagy örömmel,
 Mert parancsolataidat követem;
 Szívem mindenkor azokra szemlél.
 A királyok előtt bízvást beszélem
 A te bizonyságidat, melyektől
 Meg nem rettenek, nem félek szégyentől.
/24
#FA17D245
 Gyönyörködöm a te törvényedben,
 Szent parancsolataidat szeretem
 Mindenek felett egész éltemben.
 Én kezeimet készen felemelem
 A te kedves parancsolatidra,
 És én azokról beszélek mindenha.
;Gondolkodjunk Isten dolgairól
;
/25
#6C39C81C
 Gondold meg azt és jusson eszedbe,
 Amit szolgádnak egyszer megígértél,
 Jó reménységet adván szívembe!
 Minden ínségben vagyok bátor szívvel,
 Mert szent beszéded bizony engemet
 Megújít és megtartja életemet.
/26
#18C868BD
 A kevély népek csúfolnak engem
 És nevetnek, de nem gondolok vélek,
 Hogy törvényedet azért megvetném,
 De a te ítéletidre tekintek,
 Melyeknek örök voltát jól tudom,
 És magamat azokkal vigasztalom.
/27
#24032572
 És miként az istentelen népek,
 Kik elszakadtak a te törvényedtől,
 Gondolatimnak nagy bút szereznek:
 Emlékezésem a te szerzésedről
 Ének volt nékem nagy örömömben,
 Bujdosásimnak minden ő helyökben.
/28
#86657433
 Sem éjjel, sem nappal meg nem szűnöm
 A te nevedről gyakran emlékezni;
 Szent parancsolatidat keresem,
 Főképpen erre szoktam vágyakozni.
 Minden előtt magamban elszántam,
 Hogy a te törvényedet én megtartsam.
;Hiszek Isten ígéreteiben
;
/29
#E87FDD70
 Hiszem, te vagy az én örökségem,
 Teljes erőmmel azért azon lészek,
 Hogy a te igédet megőrizzem.
 A te színed előtt szívből könyörgök:
 Kegyelmezz meg, Úr Isten, énnekem,
 Mert ígéreted megvigasztal engem!
/30
#AF505C01
 Jól meggondolom az én utamat,
 Hogy a jó útról el ne tévelyedjem;
 Arra vezérlem minden gondomat,
 Bizonyságidra lábamat térítem.
 Igen sietek, nem kések semmit,
 Hogy megtarthassam szent parancsolatid.
/31
#F7852F8E
 Megfosztottak az istentelenek,
 Elpusztítának, de mindazonáltal
 Törvényid tőlem nem felejtetnek.
 Még éjfélkor fölkelek vigassággal
 És tégedet dicsérlek és áldlak
 Ítéletiért szent igazságodnak.
/32
#EACD6645
 Az oly népekhez adom magamat,
 Kik téged félnek és tereád néznek
 És megtartják parancsolatidat.
 Bőségével te kegyelmességednek
 Teljes e világ, azért, én Uram,
 Szent törvényedre tőled taníttassam!
;Isten igéjére építek
;
/33
#546E2128
 Íme, szegény szolgáddal sok jót től,
 Szent ígéreted szerint megsegítél,
 Min most is örvendek tiszta szívből.
 Taníts és áldj engem jó értelemmel!
 Engedd meg nékem ismeretedet,
 Mert igaznak ismerem törvényedet.
/34
#DD81F63D
 Minekelőtte megbüntettetném,
 Az igaz utat elvétettem vala,
 Most életem igédhez rendelem,
 Szorgalmatossággal tekintek arra.
 Uram, jókedvű vagy és irgalmas,
 Azért szerzésedre engemet oktass.
/35
#C034A987
 A kevélyek rólam hamisságot
 Költnek, de én a te szent törvényedet,
 Megőrizem parancsolatidat.
 Kövér ő szívük és megkeményedett,
 Én pedig a te szent törvényedben
 Gyönyörködöm mind egész életemben.
/36
#59D21923
 Jómra lett nékem, hogy megalázál,
 Hogy megtanuljam a te törvényedet,
 Min igyekezem nagy óhajtással.
 Aranyt, ezüstöt és egyéb effélét,
 Mit az emberek nagyrabecsülnek,
 Törvényedhez képest tartok semminek.
;Könyörgés megtérésért
;
/37
#0EB8DB7D
 Kezeiddel formáltál engemet,
 Taníts meg azért parancsolatidra,
 Hogy törvényidnek tudjam értelmét.
 És ezen indulnak nagy vigasságra
 Az istenfélők, látván e dolgot,
 Hogy igédbe vetem bizodalmamat.
/38
#32F61D09
 Igaz vagy, Uram, ítéletidben,
 Tudom, hogy senkit nem büntetsz méltatlan,
 Engem is méltán büntetsz ekképpen.
 Kérlek, cselekedjed ezt irgalmadban,
 Hogy én megvigasztaltassam megint
 Szolgádnak mondott ígéreted szerint!
/39
#3BC0E29F
 Nagy irgalmadat mutasd meg nekem,
 Hogy éljek, mert csak a te törvényedbe’
 Vagyon minden én gyönyörűségem.
 A kevélyek essenek szégyenségbe,
 Kik engem hamis okkal terhelnek,
 De én a te törvényedről beszélek.
/40
#36E690E1
 Térjenek hozzám mostan mindenek,
 Kik téged félnek, törvényed tisztelik
 És a te bizonyságidnak hisznek!
 Tiszta én szívem, el sem tévelyedik,
 De megtartja a parancsolatot,
 Hogy ne valljak szégyent, se gyalázatot.
;Lelki vigasztalásért való könyörgés
;
/41
#3F23628F
 Lelkem elfogy nagy kívánságában,
 Midőn várom a te segedelmedet;
 Bízván igédnek fogadásában.
 Ugyan elfárasztom az én szememet
 Nagy várakozás miatt, így szólva:
 Mikor vigasztalsz meg engem valóba’?
/42
#863AF6A0
 Noha én csaknem hasonló vagyok
 Füstön elaszott, megszáradt tömlőhöz,
 Szerzésidre mégis gondot tartok.
 Míglen kell várnom, mikor látsz ügyemhöz?
 Míg halasztod el ítéletedet,
 Ha bünteted meg ellenségeimet?
/43
#32837CA2
 A kevélyek, kik szent törvényedet
 Megvetik, titkon nekem vermet ásnak,
 De ha tekintjük szent szerzésidet,
 Parancsolatid mind jók és igazak.
 Nagy méltatlanul kergetnek engem,
 Tarts meg azért és légy én segedelmem!
/44
#D0551289
 Csaknem elvesztének ők engemet,
 És majd ugyan eltörlének e földről,
 Mégsem hagyom el szent törvényedet,
 Tartsd meg éltemet kegyelmességedből,
 Hogy megtartsam Felséged kötésit,
 És megőrizzem minden bizonyságit.
;Megáll az Istennek igéje
;
/45
#01334B00
 Mindörökké, Uram, a te igéd
 Megáll és megtart a magas mennyekben,
 Azonképpen isteni hűséged
 Megmarad örökre minden időben,
 Mint az álló föld, mit te fundáltál,
 Mely ő helyében mindenkoron megáll.
/46
#0D553F7C
 Mind ma és mindörökké megállnak,
 Amiket te rend szerint teremtettél,
 És teneked mindenek szolgálnak.
 Hogyha magamat a te törvényeddel
 Nem vigasztaltam volna, már régen
 Elvesztem volna én nagy ínségemben.
/47
#463DF109
 Nem felejtem el szent törvényedet,
 És gondot tartok parancsolatidra,
 Mert te azokban éltetsz engemet.
 Tekints, Uram, kegyelmesen szolgádra!
 Légy segítségem, mert tied vagyok,
 És törvényidnek őrzésire vágyok.
/48
#93C3E5E7
 Istentelenek énreám titkon
 Leselkednek és törnek életemre.
 Én elmélkedem bizonyságidon,
 És minden dolgot ha megnézek végre,
 Látom, hogy mindenek elmúlandók,
 De a te törvényid megmaradandók.
;Nincs Istenen kívül bölcsesség
;
/49
#1C8A28C0
 Nagy szerelmem vagyon törvényedhez,
 Melyről naponként örömest beszélek,
 Mert ez nékem víg örömöt szerez;
 Te parancsolatid bölcsebbé tésznek
 Engemet minden ellenségimnél,
 Mert soha tőlem ők nem távoznak el.
/50
#61D25C34
 Tudósabb vagyok tanítóimnál,
 Akiket már nagy bölcseknek tart minden,
 Mert bizonyságod elmémben megáll.
 Még a véneknél is bölcsebb vagyok én,
 Mert te törvényed szem előtt tartom,
 És elmémet attól el nem fordítom.
/51
#7D1F0520
 Minden hamis utat elkerülök,
 Lábam ne járjon a gonosz ösvényen;
 Igéd megtartásának örülök,
 Ítéletidet tekintem szüntelen.
 Azoktól soha el nem távozom,
 Kik által én tetőled taníttatom.
/52
#12E28C98
 A te beszéded ékes és drága,
 Még a méznél is édesebb én számban,
 Kimondhatatlan gyönyörű volta.
 Igédben van bölcsességem fundálván;
 Bölcsességemet abban keresem,
 A hamisságnak ösvényét gyűlölöm.
;Óránként fáklyám az Ige
;
/53
#34ED79ED
 Óránként fáklyám nékem szent igéd,
 Mely világot tart nékem útaimban,
 Hogy egyenesen járjam ösvényed,
 Amelyen én járhatok bátorságban.
 Megesküszöm és néked megállom,
 Hogy igazságodnak jussát megtartom.
/54
#CD16E556
 Felette igen megnyomorodtam,
 Enyhíts meg és végy fel engem ismétlen,
 Amint nekem megígérted, Uram!
 Az áldozat, mit szájam neked tészen,
 Kérlek, hogy légyen kedves tenálad,
 Ítéletedet én tudtomra adjad!
/55
#C26BE7CE
 Oly nagy veszélyben forog életem,
 Hogy tenyeremben hordozom lelkemet,
 Szent törvényedet mégsem felejtem.
 A hitlenek, kik gyalázzák nevedet,
 Tőrt vetnek nékem mindenütt széjjel,
 Szent szerzésidtől mégsem távozom el.
/56
#08612842
 Bizonyságidat örökül bírom,
 És azokat tartom drága kincsemnek,
 Azokban lesz minden vigasságom!
 Szívemet hajtom a te törvényednek
 Megtartására minden időben,
 És azokat megőrzöm mindvégiglen.
;Ösztönösen gyűlölöm a gonoszt
;
/57
#346FC430
 Ösztönösen gyűlölöm azokat,
 Akik mindenkor kétfelé gondolnak,
 De szeretem én a te utadat,
 És csak tégedet tartlak oltalmamnak.
 Ígéretedben van reménységem,
 A te szent igéd paizsom énnékem.
/58
#1D5E22A1
 Gonosztevők, menjetek el tőlem,
 Mert én azt mind meg akarom tartani,
 Amit az Isten parancsol nekem!
 Igéd szerint siess engem táplálni,
 Hogy élhessek, légy velem, Úr Isten,
 Ne szégyenüljek meg reménységemben!
/59
#1B9B73F5
 Légy gyámolom, adj jó békességet,
 Úgy lészen kedves énnékem törvényed,
 Abban keresem én örömömet!
 Az olyakat te mind a földhöz vered,
 Kik elhajolnak igazságodtul
 És járnak minden dolgukban álnokul.
/60
#26AB20CA
 Gonoszakat te elvetsz a földről,
 Mint a salakot vagy ércnek szemétjét.
 Szent bizonyságid szeretem szívből!
 Félelmében szívem előtted reszket!
 Testem elepedt nagy rettegésbe’,
 A te kemény ítéletedre nézve.
;Pörpatvarnál jobb az Úr törvénye
;
/61
#6B1AB32F
 Pörpatvar nincsen nékem kedvemre,
 És igazságát megadom mindennek;
 Ne adj azért azoknak kezekre,
 Akik engemet szüntelen kergetnek!
 Szolgádat minden jóra vezéreld,
 És a kevélyek ellen védelmezzed!
/62
#84C2B28B
 Az én szemeim elfogyatkoztak,
 Úgy nézik, várják te segedelmedet,
 Óhajtják igazságát szavadnak,
 Ne késsél, Uram, segíts meg engemet!
 Szegény szolgáddal tégy kegyelmesen,
 Taníts igédre, oktass törvényedben!
/63
#9DF7081B
 Szolgád vagyok, adj értelmet nekem,
 Hogy érthessem a te bizonyságidat,
 És jó értelmében gyönyörködjem!
 Ideje, Uram, hogy láttasd dolgodat,
 Mert helye nincs már az igazságnak,
 A te törvényid mind eltiportatnak.
/64
#955B4DA4
 Azért a te szent tanításidat
 Tiszta aranynál is inkább szeretem,
 Mindennél feljebb tartom azokat,
 Életemet én a szerint rendelem,
 Mert igaznak tartom mindenképpen,
 A hamis ösvényt gyűlölöm erősen.
;Rakva csudákkal az Isten igazsága
;
/65
#110ADFF9
 Rakvák bizonyságid nagy csudákkal,
 Hogy azért megtarthassam én szívemben,
 Azon igyekszem nagy buzgósággal.
 A te igéd, ha kik veszik eszükbe,
 Setét szíveket megvilágosít,
 Együgyűeket bölcsességre tanít.
/66
#B6BF4381
 Felfohászkodom gyakran én számmal,
 Mert én azt nagy szívem szerint kívánom,
 Hogy törvényedet értsem bizonnyal.
 Tekints reám és könyörülj szolgádon!
 Irgalmazz nékem, lám, nagy jókedved
 Azokhoz, akik szeretik szent neved!
/67
#FC52A057
 Szent igédben vezéreld utamat,
 És őrizz meg engem a hamisságtól,
 Hogy az rajtam ne vegyen hatalmat;
 Ments meg a népek nyomorgatásától,
 Hogy törvényedet örömmel nézzem
 És parancsolatidat megőrizzem.
/68
#30D12FB9
 Világosítsd meg orcád szolgádon,
 És taníts meg, hogy én jól meggondoljam,
 Beszéded engem mire tanítson!
 Könnyhullatásom szememből azértan
 Mint patak ömlik, mert hogy a népek
 Becsületet nem tesznek törvényednek.
;Summa szerint igaz az Isten
;
/69
#8A1BDB35
 Summa szerint, Uram, te igaz vagy
 Mindennémű te cselekedetedbe’,
 Ítéletednek igazsága nagy.
 Te igazságod vehetik eszükbe,
 Bizonyságidat akik megnézik,
 És parancsolatidat megőrizik.
/70
#5E4116C6
 Megöl a bú nagy indulatomban,
 Midőn tekintem a te szent igédet,
 Hogy az ellenség csúfolja bátran,
 És elfelejti minden beszédidet.
 A te szent igéd igen szép tiszta,
 Szolgád azért szereti és megtartja.
/71
#910A4B50
 Én kicsiny és megvettetett vagyok,
 De mégsem felejtem el törvényedet,
 Sőt mindenütt arra gondot tartok.
 Szent igazságidnak nem látják végét,
 Mert mindörökké ők megmaradnak;
 Törvényed törvénye az igazságnak.
/72
#F5C2494D
 Én kergettetem, vagyok ínségben,
 De nem gondolván semmi nyavalyámmal,
 Nagy örömöm van te törvényedben.
 Te igazságod mindörökké megáll,
 Melyet jelents meg nekem kegyesen,
 És bátorságos leszek életemben.
;Tőle várok szabadítást
;
/73
#B12A52F5
 Teljes szívből hozzád esedezem,
 Uram, hallgass meg engem kegyelmesen,
 Hogy rendelésidet megőrizzem!
 Kérlek, szabadíts meg engem, Úr Isten,
 És legottan igyekezem azon,
 Hogy bizonyságid megtartsam jó módon.
/74
#B0F1B5DE
 Gyakorta reggel virradat előtt
 Könyörgésemben tehozzád kiáltok,
 Igédbe vetvén reménységemet.
 Előbb, hogynem elmennek a virrasztók,
 Az én szemeim vigyáznak, néznek,
 És a te szent igédről elmélkednek.
/75
#E0FF1393
 Kegyességedért halld meg beszédem,
 Tartsd meg életem a te jóvoltodból,
 Hadd vidámuljon meg az én szívem,
 Mert sok hitetlen nép énreám tódul,
 És engem szertelen sanyargatnak,
 De a te törvényedtől távol vannak.
/76
#65B99293
 De te énhozzám, Uram, közel vagy,
 És te igédből tudom réges-régen:
 Szent törvényednek igazsága nagy,
 Te bizonyságid fundáltattak szépen.
 Hogy örökké megmaradnak, tudom,
 És jól értem, azért nyilván kimondom.
;Védelmezz meg a te igazságodért!
;
/77
#560E6855
 Vedd eszedbe én nagy ínségemet,
 Én nyavalyámból, Uram, szabadíts meg,
 Mert nem felejtem el törvényedet!
 Fogadd fel ügyemet és védelmezz meg,
 Megtekintvén szent ígéretedet,
 Éltess a te szent igéddel engemet!
/78
#E4A1A875
 A gonoszoktól, minden meghiggye,
 A segítség s üdvösség távol vagyon,
 Mert ők nem néznek szent szerzésidre.
 Szent irgalmasságod nagy mindenkoron!
 Uram, tarts meg engemet, hogy éljek
 Nagy igazságából ítéletednek.
/79
#6382281E
 Bizonyságidat én nem hagyom el,
 Noha énreám nagy sok népek törnek,
 Kik gyűlölnek és kergetnek széjjel;
 Ó, mily sérelmes ez az én szívemnek,
 Hogy ellenségi az igazságnak
 A te igéddel semmit nem gondolnak!
/80
#E7D20E4F
 Törvényidet én igen szeretem,
 És soha el nem távozom azoktól;
 Kegyességedből tartsd meg életem!
 Te igaz beszéded mindent felülmúl;
 Szent ítéleti igazságodnak
 Mostan és mindörökké megmaradnak.
;Üldöztetésben is Rólad gondolkodom
;
/81
#D5A06C61
 Űznek, kergetnek a fejedelmek,
 Noha senkinek semmit nem vétettem;
 Szent igédtől szívemben rettegek,
 Ígéretedben örvendez én lelkem,
 Mint aki talál nagy gazdag prédát,
 Avagy mint aki nyer sok drága marhát.
/82
#23A89864
 A hazugságot igen gyűlölöm,
 Semmit e földön inkább nem utálok,
 De a te törvényedet szeretem,
 És igazságán oly igen vigadok,
 Hogy meggondolván ítéletedet,
 Naponként hétszer dicsérlek tégedet.
/83
#A9406FFF
 Nagy békességük vagyon azoknak,
 Kik szeretik a te szent törvényedet,
 Semmi veszélyben el nem botolnak.
 Várom, Uram, a te üdvösségedet!
 Abban forgatom minden gondomat,
 Hogy cselekedjem parancsolatidat.
/84
#4AAEDA65
 Bizonyságidra gondot tart lelkem,
 Mert én oly igen szeretem azokat,
 És én csak azokban gyönyörködöm,
 Rendelésidben gyakorlom magamat.
 Előtted vagyon minden életem,
 Nincsen elrejtve tőled én ösvényem.
;Zendüljön fel a szabadító Isten dicsérete!
;
/85
#447786FD
 Zengő kiáltásom jusson hozzád,
 És igazságodat adjad értenem,
 Mint szent igédben te felfogadtad!
 Jusson elődbe én esedezésem,
 Szabadíts meg minden ínségemből,
 A te régen tett szent ígéretedből!
/86
#895015CF
 Ha én megtanulom szerzésedet,
 Az én ajakimmal dicsérlek téged,
 Hirdeti nyelvem te szent igédet,
 Mert minden törvényed és ígéreted
 Merő hűség és tiszta igazság,
 Nem találtatik abban semmi hívság.
/87
#C4153BD7
 Ments meg, Uram, engem kezeiddel,
 Légy erősségem, segedelmem nékem,
 Mert törvényedet szeretem szívvel!
 Üdvözítésed várom, én Istenem,
 Melyben vetettem reménységemet,
 Mert igen kedvelem szent törvényedet.
/88
#B5027B48
 Éltét csak azért kívánja lelkem,
 Hogy ő tégedet, Uram, dicsérhessen,
 Ítéleted légyen segedelmem!
 Mint elveszett juh bujdosom e földön.
 Uram, keress meg engem, szolgádat,
 Nem felejtem el parancsolatidat!

;A rágalmazók ellen
;Bourgeois L., Genf, 1551
>120
/1
#1B3C9DE9
 Én az Úr Istenhez kiálték,
 Mikoron nagy ínségben valék,
 És be nem dugá az ő fülét.
 Hallgasd meg, Uram, kérésemet!
 A népeknek hazug szájától,
 Hamis nyelvek gyalázatjától,
 Életemet e veszélytől
 Tartsd meg kegyelmességedből!

;Isten az ő népének hű őrizője
;Bourgeois L., Genf, 1551
>121
/1
#40549307
 Szemem a hegyekre vetem,
 Onnan felül nékem
 Minden segedelmem.
 Isten az én reménységem,
 Ki az eget formálta
 És e földet alkotta.
/2
#C55AE8B0
 Lábad botlani nem hagyja,
 És aki rád vigyáz,
 Nem szunnyadozik az
 Izráelnek vigyázója,
 Mert az nem aluszik el,
 De rájuk gondot visel.
/3
#B566990B
 Az Úr téged megőrizzen,
 Kezet rád terjesztvén,
 Árnyékkal befedjen!
 Hogy a nap ő hévségében,
 Néked ne ártson éjjel
 A hold az ő fényével!
/4
#139A54E0
 Az Úr őrizze örökké
 Lelkedet esettől,
 Mentse meg veszélytől!
 Az Úr híven megőrizze
 A te kimenésedet
 És te bejövésedet!

;Áldáskívánás az anyaszentegyházra
;Bourgeois L., Genf, 1551
>122
/1
#C0292ABC
 Örülök az én szívembe'
 E kívánatos hírt hallván,
 Hogy mi bémegyünk ezután
 Az Isten lakó helyébe,
 Jeruzsálem, kapuidban
 Mi lábainkkal megállván.
 A Jeruzsálem jól megépült
 Sok ékes épületekkel,
 Holott szép polgári renddel
 Mindenféle nemzet egybegyült.
/2
#08E3E4E9
 Légyen te kőfalaidban
 Csendesség és jó békesség,
 A község közt egyenesség;
 Jó szerencse házaidban!
 Az én atyámfiaiért
 És ott lakó feleimért
 Adjon Isten jó békességet!
 Szentségiért én e helynek,
 Mely szerzetett az Istennek,
 Minden jót kívánok tenéked!

;Könyörülj rajtunk!
;Bourgeois L., Genf, 1551
>123
/1
#4421EDD5
 Tehozzád szemeimet, Úr Isten,
 Emelem az égben!
 Mint szolgának szemei nyitva vannak,
 És az Úrra vigyáznak,
 Mint szolgáló leány ő aszszonyára
 Tekint minden órába,
 Úgy néznek szemeink az Istenre
 És ő kegyelmére.
/2
#BD7FA707
 Kegyelmezz, Uram, kegyelmezz nékünk,
 Mert nincs becsületünk!
 Minden oly szertelenül gyaláz minket,
 Melynél feljebb nem lehet.
 Kevély népek minket szörnyen nevetnek,
 Rajtunk csúfságot űznek;
 Gyalázó szókkal úgy illettetünk,
 Mikkel eltölt lelkünk.

;Isten velünk van a szükségben
;Bourgeois L., Genf, 1551
>124
/1
#56BF8084
 Az Izráel ezt nyilván mondhatja:
 Ha az Isten nem lett volna velünk,
 Ha ő nem lesz vala segítségünk,
 Midőn az embereknek soksága
 Nagy kegyetlen támada ellenünk.
/2
#E1F31A49
 Elnyelnek vala ők elevenen,
 Úgy reánk gerjedt vala haragjok.
 Minket teljességgel a nagy habok
 Vízzel elborítnak vala szörnyen,
 Ránk omlanak vala nagy patakok.
/3
#CB22DBF1
 A nagy árvíz reánk rohan vala,
 És elnyeli vala életünket.
 Dicséret Istennek, hogy ő minket
 Az ő foguktól megszabadíta,
 Hogy ők meg ne ennének bennünket.
/4
#890B5DC1
 Miképpen elszalad a madárka
 A madarász tőriből, akképpen
 Megszabadulánk, a tőrt elszegvén.
 Mert életünknek az Úr oltalma,
 Ki mennyet, földet teremtett bölcsen.

;A reménység meg nem szégyenít
;Bourgeois L., Genf, 1551
>125
/1
#3F509848
 Akik bíznak az Úr Istenben
 Nagy hiedelemmel,
 Azok nem vesznek el
 Semminémű veszedelemben.
 Mint a Sion hegye, megállnak,
 Nem ingadoznak.
/2
#B4BF886F
 És mint a nagy Jeruzsálemet
 Hegyek körülvették
 És nagy kerítések,
 Akképpen Isten az ő népét
 És ő híveit körülveszi
 És megőrizi.
/3
#E2CEE299
 Mert az övéit ő nem hagyja
 A hamis kezében
 És semmi ínségben,
 Hogy ki-ki önmagát megóvja,
 Hogy a hitlen népekkel egybe
 Ne essék bűnbe.
/4
#C5B2F9BF
 Jelen van a jószívűekkel,
 De a hitleneket
 És ő ösvényüket
 Elhagyja a gonosztévőkkel.
 Ő híveinek békességet,
 Ád csendességet.

;Sion foglyainak szabadulása
;Bourgeois L., Genf, 1551
>126
/1
#296EC0F3
 Mikor a Siont az Isten
 Fogságból kihozá híven,
 Úgy mentünk, mint egy álomban,
 Víg nevetés volt szájunkban.
 Dicsekedtünk a mi nyelvünkkel,
 És énekeltünk nagy örömmel;
 A pogány közt minden mondta:
 Ez nyilván az Isten dolga.
/2
#C707FA34
 Az Úr mutatá hatalmát,
 Mivelünk tőn nagy csudákat,
 Azért őt dicsérjük itten,
 Örvendezzünk szíveinkben!
 Hozd ki, Úr Isten, a többit is,
 Vesd végét ő fogságuknak is,
 Mint az erős zúgó széllel
 Mind e föld megszárad széjjel.
/3
#3168AB7F
 Akik nagy könnyhullatással
 Magot vetnek nagy bánattal,
 Aratáskor azok széjjel
 Aratnak majd víg örömmel.
 Sírva mennek ki a vetésre,
 A magot hintik keseregve,
 De a jó kévéket aztán
 Béhordják nagy vigasságban.

;Emberé a munka, Istené az áldás
;Bourgeois L., Genf, 1551
>127
/1
#44C32D8A
 Hogyha ember házat épít
 Isten segedelme nélkül,
 Ott a munka hiába kél.
 Ha Isten nem őrzi népét
 És a várost meg nem óvja,
 Az őrzőknek semmi haszna.
/2
#F5A18CF4
 Hiába keltek fel reggel,
 Nagy későn feküdni menvén,
 Alig esztek a kenyérben,
 Mit kerestek verejtékkel,
 Holott kit az Isten szeret,
 Annak könnyen ád eleget.
/3
#B8BDE83C
 Akinek gyermeki vannak,
 Szép ajándékkal láttatik,
 Mely az Istentől adatik.
 Kedves áldása az Úrnak:
 Láthatja ő magzatait,
 Önnön méhének gyümölcsit.
/4
#61E85484
 Ó, mely drága adományok,
 Hogy az ily apró gyermekek
 Ékesen felnevelkednek,
 És ők szintén olyaténok,
 Mint nyilak a jó kézívben,
 Az erős vitéz kezében.

;Az istenfélő házán áldás van
;Bourgeois L., Strasbourg, 1545
>128
/1
#581A676B
 Boldog az ember nyilván,
 Ki az Istent féli,
 Ő útaiban járván,
 Ösvényit kedveli.
 Mert magadat táplálod
 Kezed munkájával,
 Isten megáldja dolgod,
 Lát jó állapottal.
/2
#7FA51BBA
 Házadban feleséged,
 Mint a szőlővessző,
 Szép gyümölcsöt hoz néked,
 Ha eljő az idő.
 Meglátod gyermekidet
 Te asztalod körül,
 Renddel, mint olajvesszőt,
 Kikben szíved örül.
/3
#1A34A3C8
 Ez igen szép ajándék,
 Mit az Isten enged
 Az őbenne hívőknek,
 Akiket ő szeret.
 És végre te meglátod
 Fiadnak fiait,
 A Sionnak csudálod
 Nagy szép békességit.

;A nyomorgatók megszégyenülnek
;Bourgeois L., Genf, 1551
>129
/1
#C18B3385
 Én ifjúságomtól fogva engem
 Sanyargattak, mondhatja az Izráel,
 Régtől fogván sok bút tettek nékem,
 Mégsem fogyathattak el teljességgel.
/2
#7B7D055E
 Széjjel-keresztül az én hátamat
 Megszántották és megszaggatták szörnyen,
 Vontanak rajtam nagy barázdákat,
 Úgy hogy testemben semmi ép tag nincsen.
/3
#5F43D7D5
 De az igazságnak jó Istene
 Elszaggatá az ellenség kötelét;
 Kik irigykednek Sion hegyére,
 Nagy szégyenség térítse hátra őket!
/4
#9E9998BB
 Mint a hitvány fű, olyak legyenek,
 Amely szokott nőni az eszterhákon,
 Elszáradnak, mielőtt kinőnek,
 Amelyben nincsen sehol semmi haszon.
/5
#800337C1
 Min az arató nem talál annyit,
 Hogy egy marokkal valót arathatna,
 Nemhogy köthetne valami kévét,
 Amit az ember ölébe szorítna.
/6
#44DA5DBF
 És aki itt elmégyen, ne mondja:
 „A kegyes Isten áldjon meg titeket,
 Ti aratástok légyen szapora,
 Az Úr nevében áldunk benneteket!”

;A mélységből kiáltok, Uram! (Hatodik bűnbánati zsoltár)
;(1539) Bourgeois L., Genf, 1542
>130
/1
#4A7AD0B1
 Tehozzád teljes szívből
 Kiáltok szüntelen:
 E siralmas mélységből
 Hallgass meg, Úr Isten!
 Nyisd meg te füleidet,
 Midőn téged hívlak,
 Tekintsd meg én ügyemet,
 Mert régen óhajtlak.
/2
#5304F764
 Ha, Uram, bűnünk szerint
 Minket büntetnél meg:
 Uram, e világ szerint
 Ki állhatna úgy meg?
 De a te irgalmad nagy
 A téged félőkön,
 És te engedelmes vagy,
 Hogy dicsérjen minden.
/3
#67EBF816
 Énnékem reménységem
 Vagyon csak Istenben,
 És bízik az én szívem
 Ő szent igéjében.
 Én lelkem erős hittel
 Az Urat óhajtja,
 Mint a virrasztó éjjel
 A virradtát várja.
/4
#E4085418
 Izráel, az Istenben
 Vesd reménységedet,
 Mert szent irgalma bőven
 Nagy messze kiterjedt.
 Megsegít ő mindenben,
 Hívein könyörül,
 Az Izráelt kegyesen
 Kimenti bűnébül.

;Hívő lelki alázatosság
;Bourgeois L., Genf, 1551
>131
/1
#5FF06D19
 Uram, nem űz nagyra szívem,
 Szemem nem hányom magasan;
 Nem leledzem oly dolgokban,
 Mik felülhaladnak engem.
/2
#6CC51EE1
 De kivetettem szívemből
 Mindennemű kevélységet,
 Mint midőn a kisgyermeket
 Elfogják anyja tejétül.
/3
#4340918D
 Hogyha olyanná nem lettem,
 Mint az együgyű kis gyermek,
 Kit a csecs mellől elvesznek,
 Meg se kegyelmezz énnékem.
/4
#54057A71
 Izráel, az Úr Istenben
 Vessed minden reménységed!
 Megtartódnak őtet higgyed
 Mostan és minden időben!

;Könyörgés a Sionért
;Bourgeois L., Genf, 1551
>132
/1
#F67E65AA
 Emlékezzél meg, Úr Isten,
 Dávidról és ínségéről,
 Ki néked megesküdt szívből,
 És fogadást tett hűségben
 Az Úrnak, kit Jákób tisztöl.
/2
#9E0195FF
 Ezt, úgymond, én fölfogadom,
 Hogy nem megyek be házamba,
 Se le nem fekszem ágyamba,
 Szemeimet be sem hunyom,
 Szempillámat le sem zárva;
/3
#7C6416EE
 Nyugalmam addig nem lészen,
 Míglen helyet nem keresek
 A Jákób nagy Istenének,
 Holott sátort szerzek szépen,
 Hol dicsősége lakozzék.
/4
#0D34490A
 Mert e Siont az Úr Isten
 Választá lakóhelyének,
 Mondván: e hely kell kedvemnek,
 Itt lakom minden időben;
 Ez helye jótetszésemnek.
/5
#D8B9DD93
 Én őket megelégítem,
 Kenyért adok a szegénynek,
 Ruhájába üdvösségnek
 Papjait felöltöztetem;
 Örömük lesz a híveknek.

;A testvéri egyetértés áldása
;Bourgeois L., Genf, 1551
>133
/1
#51C3AD8E
 Ímé, mily jó és mily nagy gyönyörűség
 Az atyafiak közt az egyenesség,
 Ha békével együtt laknak.
 Mint a balzsamolaj, ők olyanok!
 Megáldja az Úr az ilyeneket,
 Nékik ád hoszszú életet.

;Éjtszakai dicséret a templomban
;Bourgeois L., Genf, 1551
>134
/1
#D5717C96
 Úrnak szolgái mindnyájan,
 Áldjátok az Urat vígan,
 Kik az ő házában éjjel
 Vigyázván, vagytok hűséggel.
/2
#10CA7B3B
 Felemelvén kezeteket,
 Dicsérjétek Istenteket,
 Szívből néki hálát adván,
 Őt áldjátok minduntalan!
/3
#1488DA7A
 Megáldjon téged az Isten
 A Sionról kegyelmesen,
 Ki teremtette az eget,
 A földet és mindeneket!

;Az egy igaz Isten dicsérete
;Genf, 1562
>135
/1
#D8FA92ED
 Áldjátok az Úr nevét,
 Akik néki szolgáltok!
 Magasztaljátok őtet,
 Kik hív szolgái vagytok,
 Kik állotok házában
 És jártok tornáciban.
/2
#B37F3508
 Dicsérjétek, mert jó ő,
 Áldjátok ő szent nevét,
 Mert ő igen jókedvű!
 Jákóbot, mint ő népét,
 Izráelt elválasztá,
 Örökévé foglalá.
/3
#FE2AC5B0
 Mert tudom, hogy ez Isten
 Erősb más isteneknél,
 Hatalma nagy mindenen;
 Isteni beszédével,
 Amit akar az égen,
 Megtészi földön, vízen.
/4
#E98D7746
 Megáll örökké neve
 És szent emlékezete.
 Ítéli hatalmában,
 Népét igazságában;
 Szolgáinak az Isten
 Megkegyelmez kegyesen.
/5
#35C440B2
 Kik az Urat félitek,
 Szent nevét dicsérjétek!
 A Sionról áldjátok
 És őt magasztaljátok
 Az ő lakóhelyében,
 A szent Jeruzsálemben!

;Isten örök jósága és csodái
;Genf, 1562
>136
/1
#5D112A65
 Dicsérjétek az Urat,
 Mert ő jókedvet mutat,
 És az ő kegyessége
 Megmarad mindörökre.
/2
#AF1C0406
 Áldjátok Istenteket,
 Isteneknek Istenét,
 Mert az ő kegyessége
 Megmarad mindörökre.
/3
#2A0169C8
 Dicsérje őtet minden,
 Mert nagy csudákat tészen,
 És az ő kegyessége
 Megmarad mindörökre.
/4
#6C106D5E
 Minden őt magasztalja,
 Mert ő Uraknak Ura,
 És az ő kegyessége
 Megmarad mindörökre.
/5
#4BB1F6E6
 Ki a mennyet teremté
 És bölcsen ékesíté,
 Mert az ő kegyessége
 Megmarad mindörökre.
/6
#6A038773
 Aki a széles földet
 Szerzette a víz felett,
 És az ő kegyessége
 Megmarad mindörökre.
/7
#48E934CA
 Ki nagy szép lámpásokat
 Fenn az égen alkotott,
 És az ő kegyessége
 Megmarad mindörökre.
/8
#EB774DD8
 Nappal vezérlésére
 A napfényt teremtette,
 És az ő kegyessége
 Megmarad mindörökre.
/9
#4E36FF83
 Ellenségünk kezéből
 Megmente jókedvéből,
 És az ő kegyessége
 Megmarad mindörökre.
/10
#A0BC7A39
 E földön minden testet
 Táplál és bőven éltet,
 És az ő kegyessége
 Megmarad mindörökre.
/11
#4B45C23F
 Áldjátok az Úr Istent
 Minden népek naponként,
 Mert az ő kegyessége
 Megmarad mindörökre.

;Babilon vizei mellett
;(1539) Bourgeois L., Genf, 1542
>137
/1
#D6E7B6E4
 Hogy a babiloni vizeknél ültünk,
 Ott mi nagy siralomban keseregtünk,
 A szent Sionról megemlékezvén,
 Melynél gyönyörűségesb hely nincsen.
 A nagy búnak és bánatnak miatta
 Hegedűnket függesztettük fűzfákra.
/2
#46FAD8FA
 Akik minket fogva tartottak, kértek,
 Hogy valamit hegedülnénk nékiek,
 És mondanánk sioni éneket.
 Felelvén mondtuk: Miképpen lehet?
 Hogy dicsérhetnénk az Úr Istent vígan,
 Énekelvén ez idegen országban?
/3
#06A5977B
 Íme, neked én azt nyilván felelem,
 Hogy hegedülésemet elfelejtem
 Előbb, hogynem a Jeruzsálemet.
 Míg e fogságban tartnak engemet,
 Inkább ínyemhez ragadjon én nyelvem,
 Ha Jeruzsálemre nem óhajt szívem!
/4
#8D46E3EA
 Az Édom fiairól emlékezzél,
 És nékiek azt, Uram, ne engedd el,
 Amit ők akkoron kiáltottak,
 Midőn Jeruzsálemet rontották:
 „Fosszad, fosszad Jeruzsálemnek népét,
 Földig lerontsad minden épületjét!”
/5
#41A21719
 Te, Babilonnak leánya, meghiggyed,
 Hogy végre még por-hamuvá kell lenned!
 Boldog, aki tenéked e dolgot,
 Megfizeti e méltatlanságot,
 Ki öledből gyermekidet kirántja,
 És az erős kősziklához paskolja!

;Isten az én szabadítóm
;(1539) Bourgeois L., Lyon, 1542
>138
/1
#BD8B31E3
 Dicsér téged teljes szívem,
 Én Istenem, Hirdetem neved.
 Dicsérlek istenek felett
 Én tégedet, Mert azt érdemled.
 És a te szentegyházadban
 Imádkozván, Neved tisztelem,
 Áldásodra én kész vagyok,
 Hálát adok Neked, Istenem.
/2
#01D452FA
 Öregbül nagy dicsőséged,
 Mert megtészed, Amit megmondasz.
 Ha könyörgök ínségemben,
 Engem menten Megszabadítasz.
 Téged minden földön lakók,
 Nagy királyok, Uram, dicsérnek,
 Mert szent igéd tiszta voltát,
 Igazságát Eszükbe vették.
/3
#6072D67B
 Ez Urat, ki felségesen
 Csudát teszen, Felmagasztalják.
 Mondván: nagy ő dicsősége
 És ereje, Őt azért áldják.
 Mert noha ül nagy magasan,
 De lát nyilván Alatt valókat;
 Magas dolgokat is könnyen,
 Lát élesen, Mind fent, mind alatt.
/4
#7D28ADCD
 Mindennemű szükségemben,
 Ínségemben Megerősítesz;
 Láttukra ellenségimnek,
 Kik gyűlölnek, Engem megmentesz.
 Amit az Úr egyszer végez,
 Az jó véghez megyen mindenütt.
 Jókedved megáll: ne hagyd el,
 Sőt végezd el Kezed munkáit!

;Isten mindentudó és mindenütt jelenvaló
;Bourgeois L., Genf, 1551
>139
/1
#6C8D191C
 Uram, te megvizsgálsz engem,
 Megismersz mindent énbennem,
 Vagy állok, ülök, vagy megyek,
 Mind tudod, amit művelek;
 Valamit gondolok szívemben,
 Te azt mind jól érted meszszünnen.
/2
#B80F3820
 Vagy járok-kelek, vagy fekszem,
 Te mindenütt vagy körülem.
 Jól látod minden utamat,
 Érted minden dolgaimat;
 Egy szó sem jő az én nyelvemre,
 Mit előbb ne tudtál volna te.
/3
#9A448CDB
 Körös-körül rajtam minden
 Tőled vagyon teremtetvén,
 Reám bocsátád kezedet,
 Felülmúlja értelmemet.
 Bölcseséged meg nem foghatom,
 Dolgaidon csak álmélkodom.
/4
#05BFA530
 Lelked előtt hová mennék,
 Holott elrejtve lehetnék?
 Előtted hova szaladjak?
 Égbe menjek? Ott talállak!
 Ágyamat ha vetném pokolban,
 Ott látnálak téged legottan.
/5
#8BC2E53A
 Ha a hajnal szárnyát venném,
 És az égre emelkedném,
 És elrepülnék nagy messze,
 A külső tenger szélire,
 Ott is meglelnél, Uram, engem,
 Kezedet el nem kerülhetem.
/6
#34EE9044
 Mondék: talán setétséggel
 Bétakarózhatom éjjel.
 Az sem lenne használatos,
 Mert még az éj is világos,
 Mely körülem annyira fénylik,
 Hogy nékem fényes napnak látszik.
/7
#93D87C20
 Nálad ismeretes voltam
 Előbb, hogynem formáltattam,
 Hogy még nem voltam, ismertél,
 És napjaimra szemléltél,
 Mik könyvedben voltak felírva,
 Midőn még nem voltak formálva.
/8
#932CF3FA
 Mily drágák a te tanácsid,
 Ha megnézem gondolatid,
 Számtalan soknak találom,
 Ha kimondani akarom;
 Többnek lelem én a fövénynél,
 Mely sok a tengerparton széjjel.
/9
#70524619
 Vizsgálj meg és jól próbálj meg,
 Szívemet valóban nézd meg,
 És lásd meg, minémű vagyok!
 Ha gonosz ösvényen járok,
 És ha olyannak találsz engem,
 Vezess a jó útra, Istenem!

;Könyörgés ellenségtől való szabadításért
;Bourgeois L., Strasbourg, 1545
>140
/1
#DB0EF8CD
 Szabadíts meg engem, Úr Isten,
 A gonosz csalárd embertől;
 Őrizz erőszaktevők ellen,
 Ments meg a vakmerő néptől!
/2
#F1DBE7C8
 Kik csak hamisságot gondolnak
 Mindenkor az ő szívükben,
 És hogy hadakat indítsanak,
 Azon vannak mindenképpen.
/3
#8CBD6BEC
 Ments meg a gonoszok kezébül,
 Akik erőszakot tesznek,
 És igyekeznek szüntelenül,
 Hogy engemet megejtsenek.
/4
#03EB97EE
 Uram, segítség vagy te nékem
 Mindennémű ínségemben,
 Azért védelmezd meg én fejem
 A hadakozó időben!
/5
#CF767EEC
 Ne engedd a hitetleneknek,
 Hogy elővigyék dolgukat!
 Hogy inkább ne kevélykedjenek,
 Rontsd meg gonosz szándékukat!
/6
#5D167469
 Tudom, hogy Isten a szegénynek
 Felfogja ügyét kegyesen,
 Megkegyelmez az erőtlennek,
 Ő igazságát jelentvén.
/7
#BBFD646C
 Az igazak szép énekkel
 Dicsérik te szent nevedet,
 És örökké jó reménységgel
 Megmaradnak színed előtt!

;Esti könyörgés megszentelődésért
;Genf, 1562
>141
/1
#F9DA35A7
 Tehozzád kiáltok, Úr Isten,
 Siess énhozzám, ne késsél,
 Mert téged óhajtlak szívvel,
 Azért hallgass meg kegyelmesen!
/2
#12D90E30
 Könyörgésem elődbe menjen,
 Mint jó illat füstölgése,
 Kezeim felemelése
 Estvéli áldozatul légyen!
/3
#D72C14EA
 Őrizettel tartsd meg én szájam,
 Amely gondot tartson arra;
 Závárt szerezz ajakimra,
 Hogy tőlük ne légyen nyavalyám!
/4
#D6FF8394
 Ne bocsássad szívem gonoszra,
 Hogy a hamisan élőkkel
 Soha ne ereszkedjem el,
 Az ő nyájas kívánságokra.
/5
#003B2EB7
 A hív ember megfeddjen engem,
 És feddése kedves légyen,
 Mint balzsamolaj fejemen;
 Még verése sem árt énnékem!
/6
#55832D76
 Mert én elhiszem s mondom nyilván,
 Hogy még a hitetlenekért,
 Imádkozom mentségekért,
 Nyavalyájukon szánakozván.
/7
#0F55137C
 Szemeim reád néznek, Uram,
 Reménységem vagyon benned,
 Lelkemet hát el ne vessed,
 Mert te vagy minden bizodalmam!

;Segedelemkérés nagy szorongattatásban
;Bourgeois L., Genf, 1551
>142
/1
#5D1DA3B8
 Én az Úrhoz felkiálték,
 Kiáltván néki könyörgék,
 Jelentvén én panaszimat,
 Megbeszélém nyavalyámat.
/2
#E4AB55CD
 Ha szorongattatik lelkem,
 Te utat mutatsz énnekem,
 Módot találsz, amely által
 Megszabadulok jó móddal.
/3
#93B9BB1F
 Utamra tőrt mernek hányni,
 Amelyen szoktam én járni.
 Tekintek széjjel mellőlem,
 De senki nem ismer engem.
/4
#27ABF683
 Bezárattak utak, ajtók,
 Egy felé sem szaladhatok,
 És ez ilyen ínségemben
 Senki nincsen, ki segítsen.
/5
#FE71B7FA
 Tehozzád kiálték, Uram,
 Mondván: te vagy bizodalmam,
 És egyedül reménységem
 E földön csak te vagy nékem.
/6
#6D80CEE4
 Halld meg az én imádságom,
 Mert igen sanyargattatom!
 Ments meg ellenségim ellen,
 Mert erősbek, hogynem mint én!
/7
#B921D876
 Vedd ki fogságból lelkemet,
 Hogy dicsérjem szent nevedet,
 Hogy jól téssz velem, a hívek
 Ottan engem körülvesznek.

;Könyörgés szabadításért és vezetésért (Hetedik bűnbánati zsoltár)
;(1539) Bourgeois L., Genf, 1542
>143
/1
#9A2BC341
 Hallgasd meg, Uram, kérésemet,
 Vedd füledbe könyörgésemet
 A te ígéreted szerint!
 Hallgass meg és tarts meg engemet
 Te szent igazságod szerint!
/2
#C6F154E1
 Az ellenség kerget engemet,
 A földhöz verte életemet,
 Helyheztetett a setétbe.
 Setétségben engem elrejtett,
 Mint halottat sír mélyébe.
/3
#3ACBD634
 Ez ilyen nagy ínségim között
 Említem a régi időket,
 Nézvén sok csuda dolgodat,
 Melyeket kezed cselekedett,
 Mikkel biztatom magamat.
/4
#FE48F7C2
 Tehozzád, én Uram, Istenem,
 Nagy siralommal felemelem
 És kinyújtom kezeimet.
 Tégedet úgy óhajt én lelkem,
 Mint a száraz föld a vizet.
/5
#96B89D09
 Ne késsél, hallgasd meg kérésem,
 Mert elfogy bennem az én lelkem!
 Színed tőlem el ne térjen,
 Mert immár olyanná kell lennem,
 Mint aki száll sír mélyében!
/6
#7790CEBC
 Uram, engemet erősíts meg,
 Te szent nevedért vigasztalj meg,
 És a te kegyességedből
 Az én életem szabadítsd meg
 Mindennémű ínségekből!

;Háború és béke
;Bourgeois L., Genf, 1562
>144
/1
#2CC6303B
 Áldott az Úr, ki kezemet tanítja,
 És ujjaimat a hadakozásra,
 Áldott legyen az én jó Istenem,
 Aki mindenkor megőriz engem!
 Ő az én kőváram és szabadítóm,
 Ő én paizsom és én oltalmazóm.
 Tebenned vetem reménységemet,
 Mert alám vetéd az én népemet.
/2
#C1F9B5A1
 De micsoda az ember életében,
 Hogy őreá gondot tartasz ekképpen?
 Micsodák az emberek fiai,
 Hogy felséged őket így kedveli?
 Az emberek dolgát ha megtekinted,
 Legottan őket csak semminek leled;
 Az ő napjai hamar elfogynak,
 És mint az árnyék, ottan elmúlnak.

;Isten jó és hatalmas
;Genf, 1562
>145
/1
#5A20679D
 Magasztallak téged, én Istenem,
 Ó, én királyom, neved tisztelem!
 Mindennap hirdetem dicséreted,
 És örökké áldom te szent neved!
 Igen dicséretes és nagy az Isten,
 Megfoghatatlan ő dicsőségében.
 Nemzetről nemzetre te dolgaidat,
 Mindenkor hirdetik nagy hatalmadat.
/2
#B6FB0F41
 Éneklem a te dicsőségedet,
 A te felséges dicséretedet!
 Csudatétidet minden népeknél
 Én kibeszélem mindenütt széjjel,
 Hogy hatalmadat mindenek hirdessék,
 Melyet dolgaidban eszükbe vesznek.
 Dicsőségedet én el nem hallgatom,
 De mindeneknek jól előszámlálom.
/3
#30C54690
 Megáll örökké a te országod,
 És mindenkor tart uralkodásod.
 Az Úr megtartja az eldűlőket,
 És felemeli az elesteket.
 Mert mindeneknek szemei rád néznek,
 Idején eledelt adsz őnekiek,
 Midőn felnyitod áldott kezeidet,
 Bőven megelégítesz mindeneket.
/4
#AB1E2549
 Az Úr igaz minden útaiban,
 És szentséges minden dolgaiban.
 Közel vagyon ő azokhoz nyilván,
 Akik hozzá kiáltnak valóban.
 És amit az istenfélők kívánnak,
 Őnékik nagy bőségben megadatnak;
 Nagy kegyelmesen őket meghallgatja,
 És üdvösségét nékik megmutatja.

;Ne bizakodjunk emberekben!
;Genf, 1562
>146
/1
#75D05225
 Áldjad, én lelkem, az Urat,
 Hirdessed dicséretét!
 Az Istennek adok hálát,
 Valamíg engem éltet.
 Én az Úrnak éneklek,
 Mindaddig, míglen élek.
/2
#0CD37782
 Ki a mennyet és a földet
 És a tengert teremté,
 És ezekben mindeneket
 Nagy hatalmával szerze,
 Igazsága, hűsége megmarad mindörökre.
/3
#8E4EA843
 Akik méltatlan szenvednek,
 Megmenti a jó Isten,
 A nyomorult éhezőknek
 Ő ád eledelt bőven,
 A foglyokat kihozza,
 Rabságból kioldozza.
/4
#9545A186
 És a világtalanoknak
 Megnyitja ő szemöket;
 Akik dűlőfélben vannak,
 Meggyámolítja őket,
 Mert az igazat híven
 Szereti az Úr Isten.
/5
#C0020CB9
 Veszedelmében megmenti
 A nyavalyás jövevényt;
 Az árvákat megsegíti,
 Rájuk kegyesen tekint;
 A sérelmes özvegyek
 Tőle megenyhíttetnek.
/6
#1EC50E81
 A hitleneket megrontja,
 Ösvényüket elvesztve,
 De megáll az ő országa
 Most és minden időbe’.
 Ó, Sion, a te Urad
 Mindörökké megmarad!

;Isten hatalma és gondviselése
;Genf, 1562
>147
/1
#489D4BA0
 Az Urat dicsérjétek, mert jó,
 És az Istent dicsérni méltó;
 Kedves dolog az Úr Istennél,
 Hogy őtet dicsérjék egy szívvel,
 Mert Jeruzsálemet az Isten
 Megépíti nagy kegyelmesen,
 Ismét az Izráel nemzetjét,
 Öszszegyűjti eloszlott népét.
/2
#0D3F56B0
 Töredelmes szívet meggyógyít,
 Sérelmes lelket ő megenyhít,
 Békötözi az ő sebüket,
 Megkönnyebbíti sérelmüket.
 A csillagokat megszámlálja,
 És azoknak számát jól tudja,
 És mindeniket úgy ismeri,
 Hogy tulajdon nevén nevezi.
/3
#09CE0925
 Nagy a mi Urunk, az Úr Isten,
 Akinél nagyobb semmi nincsen;
 Erőssége kimondhatatlan,
 Bölcsessége számlálhatatlan.
 Az együgyűket felemeli,
 A szelídeket erősíti,
 Megalázza a hitleneket,
 A földre lealázza őket.
/4
#870C9A6C
 Inkább azokban gyönyörködik,
 Kik igazán csak őtet félik,
 Kik teljes szívből kegyes voltát,
 Reménylik mindenkor irgalmát.
 Dicsérd, Jeruzsálem, az Urat,
 Felmagasztaljad nagy hatalmát,
 E kegyes Urat, ó, szent Sion,
 Te is dicsérd áhítatoson!
/5
#655BC201
 Jákóbnak adá szent igéjét,
 Hogy ahhoz szabják életüket;
 Törvényit adá Izráelnek,
 Hogy ők azok szerint éljenek.
 Nem tőn így semmi pogány néppel,
 Nem látá őket ily szentséggel,
 Szent szerzésit ők nem ismerték,
 Ezért Alleluja mondassék!

;Egész világ Istent dicsérje!
;Genf, 1562
>148
/1
#8FD1D1FE
 No, dicsérjétek mindnyájan
 Az Urat a mennyországban!
 Őt dicsérjétek az égben,
 Az ő felséges székében!
 Minden angyalok őt dicsérjék,
 És minden menynyei seregek,
 A nap és hold őt dicsérje
 Minden csillagokkal egybe'!
/2
#A6F992B5
 Az Urat ti magas egek,
 Dicsérjétek minden vizek,
 Mik fenn az égben lebegtek,
 Az Úr nevét dicsérjétek!
 Mert mindeneket igéjével,
 Ő teremte nagy erejével,
 Mindent úgy megerősíte,
 Hogy megálljon mindörökre.
/3
#D253B5B2
 Ifjak, leányok és vének,
 Az Úr Istent dicsérjétek,
 Mert nevének dicső volta
 Mennyet-földet felülmúlja!
 Arcát felemeli népének,
 Minden szentek, őt dicsérjétek!
 Izráel választott népe
 Az Úrnak nevét dicsérje!

;Győzedelmi ének
;Genf, 1562
>149
/1
#45DAC1EA
 Az Úrnak, no, énekeljetek
 Új éneket szent felségének!
 A szentek gyülekezetében
 Dicsérje Őtet minden!
 Az Izráel örvendezzen
 Ő teremtő Istenében,
 Királyuknak örüljenek
 A Sionbéliek!
/2
#4F83071B
 Az Úr nevét síppal és dobbal,
 Dicsérjétek lanttal, kobozzal,
 Légyen hegedűknek zengése
 Neve dicséretére!
 Mert az Isten az ő népit,
 Igen szereti híveit,
 A szegényeket megmenti,
 Sok jókkal szereti.
/3
#033A8891
 Szent hívei az Úr Istennek
 Nagy tisztességet tőle nyernek;
 Örülnek az ő hálóházukba’,
 Az Urat magasztalva.
 Dicséretét az Istennek
 Az ő szájukban viseljék,
 És kétélű fegyver lészen
 Nékiek kezükben!

;Minden lélek dicsérje az Urat!
;Genf, 1562
>150
/1
#9167EF6A
 Dicsérjétek az Urat!
 Áldjátok ő szent voltát!
 Dicsérjétek menynyekben,
 Hol országol kegyesen
 Az ő nagy dicsőségébe'!
 Dicsérjétek hatalmát,
 Melyből dicső nagy voltát
 Minden veheti eszébe!
/2
#DA84D30D
 Dicsérjétek őt kürtben,
 Ékes éneklésekben;
 Hegedűkben, lantokban
 És hangos citerákban
 Az Úrnak zengedezzetek!
 Sípokban, orgonákban
 És más szép muzsikákban
 Örvendjetek az Istennek!
/3
#3CF3A4E4
 Az Urat cimbalmokban
 És egyéb szerszámokban
 Mindnyájan dicsérjétek,
 Citerát pengessetek,
 Az Úr szent nevét dicsérvén!
 Minden lelkes állatok,
 Istent magasztaljátok:
 Dicsőség Istennek! Ámen.

>Kezdő énekek

;Clausnitzer Tóbiás, 1618–1684
>151. Kegyes Jézus, itt vagyunk
/1
#7D1EEAB5
 Kegyes Jézus, itt vagyunk
 Te szent Igéd hallására,
 Gyúljon fel kívánságunk
 Idvesség tanulására,
 Hogy a földtől elszakadjunk,
 Csak tehozzád ragaszkodjunk.
/2
#00E60991
 Elménket, értelmünket
 Lelki sötétség fogta bé,
 De szent Lelked szívünket
 Tiszta fénnyel úgy töltse bé,
 Hogy jót gondoljunk és szóljunk,
 Mert csak tőled kell azt várnunk.
/3
#D52B9387
 Dicsőségnek napfénye,
 Istentől jött világosság!
 Indítsd lelkünk készségre,
 Nyisd meg fülünk, szívünk és szánk!
 Hitvallásunk, könyörgésünk,
 Urunk Jézus, halld meg, kérünk!

;Debrecen, 1590
>152a. Ím, béjöttünk nagy örömben
/1
#0891EDB1
 Ím, béjöttünk nagy örömben,
 Felséges Isten,
 A te szentidnek gyülekezetébe,
 A te templomodba,
 Felséges Atya Isten.
/2
#1A91383F
 Itt megállunk teelőtted,
 Felséges Isten!
 És igaz hitből áldozunk előtted,
 Vallást teszünk rólad,
 Felséges Atya Isten!
/3
#297787CB
 Vágyik lelkünk szent Igédhez,
 Felséges Isten!
 Mint a szomjúhozó szarvas a vízhez,
 A hideg kútfőhöz,
 Felséges Atya Isten!
/4
#93933C60
 Örvendezünk mi szívünkben,
 Felséges Isten!
 Mert bejutottunk immár teelődbe,
 A te templomodba,
 Felséges Atya Isten!
/5
#236F85D3
 Csak ez nékünk vigasságunk,
 Felséges Isten!
 Hogy lakozik a te neved közöttünk,
 Dicsértetel tőlünk,
 Felséges Atya Isten!
/6
#F6D4770A
 Tartsd meg azért békességben,
 Felséges Isten,
 E kicsiny seregecskét igaz hitben,
 Te tiszteletedben,
 Felséges Atya Isten!
/7
#910C4CD8
 Prédikáltasd szent igédet,
 Felséges Isten!
 Ne hagyd szomjúhozni a mi lelkünket,
 Áldd meg életünket,
 Felséges Atya Isten!
/8
#7F46150A
 Zengedeznek mi ajakink,
 Felséges Isten!
 Örvendetes szókban, dicséretekben,
 Ékes énekekben,
 Felséges Atya Isten!
/9
#AEEACD7B
 Áldott vagy te magas mennyben,
 Felséges Isten!
 Kit illet dicséret a szent templomban,
 Anyaszentegyházban,
 Felséges Atya Isten!

;Tersteegen G., 1697–1769
;F.: Csomasz Tóth Kálmán
>153. Itt van Isten köztünk, Jertek őt imádni
/1
#98BF0843
 Itt van Isten köztünk, Jertek őt imádni,
 Hódolattal elé állni,
 Itt van a középen, minden csendre térve
 Őelőtte hulljon térdre. Az, aki Hirdeti
 S hall -ja itt az Ígét: Adja néki szívét!
/2
#1B08CA53
 Itt van Isten köztünk: Ő, kit éjjel-nappal
 Angyalsereg áld s magasztal.
 Szent, szent, szent az Isten!
 Néki énekelnek A mennyei fényes lelkek.
 Halld, Urunk, Szózatunk,
 Ha mi, semmiségek Áldozunk Tenéked!
/3
#B45A6B97
 Csodálatos Felség, Hadd dicsérlek Téged:
 Hadd szolgáljon lelkem Néked!
 Angyaloknak módján Színed előtt állván,
 bárcsak mindig orcád látnám!
 Add nékem Mindenben
 Te kedvedben járnom, Istenem, Királyom!
/4
#266E2B95
 Általjársz Te mindent; Rám ragyogni engedd
 Életadó, áldott Lelked!
 Mint a kis virág is Magától kibomlik,
 Rá ha csöndes fényed omlik:
 Hagyj, Uram, Vidáman
 Fényességed látnom S országod munkálnom!
/5
#7DD59A88
 Egyszerűvé formálj Belső, lelkiképpen,
 békességben, csöndességben.
 Tisztogasd meg szívem: Tisztaságod lássam
 Lélekben és igazságban.
 Szívemmel Mindig fel- Szállhassak sasszárnyon:
 Csak Te légy világom!
/6
#F51909BD
 Jöjj és lakozz bennem: Hadd legyen már itt lenn
 Templomoddá szívem-lelkem!
 Mindig közellévő: jelentsd Magad nékem,
 Ne lakhasson más e szívben;
 Már itt lenn Mindenben
 Csakis Téged lásson, Leborulva áldjon!

;Földvári József, 1766–1830
>154. Jövel, ó, áldott Szentlélek!
/1
#AACA6B9D
 Jövel, ó, áldott Szentlélek!
 Gerjeszd fel híveidet,
 És szent munkájokban vélek
 Közöld kegyelmeidet,
 Hogy szívünk egyenesen
 Az Úrra függesztessen.

;Lengyel József, 1770–1822
>155. Ó, mely boldog ember az
/1
#5A0C408F
 Ó, mely boldog ember az, Ki téged, élő, igaz,
 Egy Istent megismerhet,
 És te szent hajlékodban Lélekben, Igazságban
 Dicsérhet és tisztelhet.
 Ó, mi Istenünk, Adtad e boldogságot nekünk;
 Adjad, hogy ismerhessünk Még jobban,
 Naponként dicsérhessünk Buz -góbban!
 Szent Lelkedet hozzánk e végre
 Küldd el segítség -re!

;Debrecen, 1736
>156. Atya, Fiú, Szentlélek, Készíts szent Igédre
/1
#F6306014
 Atya, Fiú, Szentlélek,
 Készíts szent Igédre,
 Jézus érde-mét ruházd
 Áldott híveidre.

;Kolozsvár, 1837
>157. Nem vagyunk mi magunkéi
/1
#36669BB6
 Nem vagyunk mi magunkéi,
 De Jézus vére bére;
 Lelkünk, testünk Istenéi
 Az ő tiszteletére.
 Urunk Jézus, jöjj most el,
 Lelked által segíts fel!
 Dicsőítünk mi testünkben,
 Magasztalunk mi lelkünkben!

;Földvári József, 1766–1830
>158. Ó, Úr Isten, légy közöttünk
/1
#B0349491
 Ó, Úr Isten, légy közöttünk,
 Kik imádásodra jöttünk,
 Szívünk egyedül rád hallgasson,
 Hogy igédnek áldott szava,
 Melytől függ lelkünknek java,
 Minket minden jóra bírhasson:
 Szent e hely, légyenek szentek,
 Kik e házban megjelentek.

;Szegedi Gergely, 1569 előtt
;a 122. zsoltár alapján
>159. Örvend mi szívünk, Mikor ezt halljuk
/1
#D7A2DCE2
 Örvend mi szívünk, Mikor ezt halljuk:
 A templomba mégyünk,
 Hol Úr Istennek Szent Igéjét halljuk.
/2
#425DA9C2
 Megállunk hittel, Örök Úr Isten,
 A te templomodban,
 És tiszta szívvel Dicsérünk mi téged.
/3
#27A150EB
 Áldd meg, Úr Isten, A te népedet,
 Kik téged szeretnek,
 Tartsd meg közöttünk A gyülekezetet.
/4
#8234C8F9
 Légyen békesség, Felséges Isten,
 Anyaszentegyházban,
 Oltalmazz minket Minden háborúnkban.
/5
#3D57397E
 Dicséret neked, Atya Úr Isten,
 A te szent Fiaddal,
 És Szentlélekkel, Mi vigasztalónkkal!

;Földvári József, 1766–1830
>160. Szent Isten, no-ha néked
/1
#D7FDB61F
 Szent Isten, no-ha néked Az egek ü-lő széked,
 Ott dicsérnek tégedet, Biztat szent ígéreted,
 Hogy azért meg nem veted Földön lakó népedet.
 Együtt vagy vélek, Kikben alázatos a lélek,
 Mi is hát térdet hajtunk,
 Hogy rajtunk Könyörülj, kegyelmeddel
 Fogadd el, Mikor tehozzád emel kezet
 Ez a gyülekezet.

;Szücs György, †1809
>161. Szentlélek, végy körül bennünket
/1
#BF9878D4
 Szentlélek, végy körül bennünket,
 Szenteld meg szívünket,
 Készíts neved imádására, Magasztalására,
 Hogy téged szívből imádhassunk,
 Hálákat adhassunk:
 Hiszszük, a mi szánknak szózatja
 Egeid meghatja.
/2
#3BBCA1B6
 Szentlélek, mi imádunk téged,
 Valljuk istenséged.
 Hisszük, hogy az új ember szíve saját kezed míve.
 Te vagy a hitnek mind szerzője,
 Mind elvégezője,
 Te gyújtasz szívünkben világot,
 Forró buzgóságot.
/3
#D94F474C
 Szakaszd el hát most is szívünket,
 Minden érzésünket
 A sok hiábavalóságtól, E csalárd világtól,
 Hogy az Igének hallgatói,
 Légyünk megtartói;
 Mely szívünkben gyökeret verjen,
 Gyümölcsöt teremjen.

;Nagyenyed, 1785
>162. Szűkölködünk Nagymértékben
/1
#63B4E9B8
 Szűkölködünk Nagymértékben
 Segedelem nélkül,
 Reménykedünk, Örök Isten,
 Te légy segítségül.
 Dicsérhessünk és lehessünk
 Jézus szava hallgatói, Igaz megtartói.

;Szentes István
>163. Terólad zeng dícséretünk
/1
#3BE832E9
 Terólad zeng dícséretünk,
 Nagy Isten, téged tisztelünk
 Szent házadban szívvellélekkel,
 Imádkozó szent érzésekkel;
 Szenteld meg igyekezetünk,
 Lelkesítsd gyülekezetünk:
 Hogy vigyen előbb ez óra
 Mindnyájunkat a fő jóra.

;Szentes István
>164. Téged áldunk, nagy Isten!
/1
#C83CC7C0
 Téged áldunk, nagy Isten!
 E szent napnak reggelén,
 Kik megjelentünk itten,
 Szent tornácid küszöbén;
 Te vagy Atyánk, jóltevőnk,
 Hűséges gondviselőnk.
/2
#3FAED9DB
 Szent Igédnek világát
 Tündököltesd lelkünkben,
 Annak hathatósságát
 Gyümölcsöztesd éltünkben;
 Hogy lélek szerint higgyünk,
 s igazán cselekedjünk.
/3
#2C61D8A0
 Tartsd meg köztünk, Úr Isten,
 A híveknek seregét,
 Közöld velünk kegyesen
 Szereteted kegyelmét,
 Hogy országod terjedjen,
 Vég nélkül növekedjen!

;Huszár Gál, 1574
>165. Uram Isten, siess
/1
#D8D694B9
 Uram Isten, siess
 Minket megsegíteni
 Ily nagy szükségünkben,
 Krisztus Jézusért,
 Mi Urunkért És Megváltónkért.

;Lengyel József, 1770–1822
>166. Úr Isten, mi sok szükséget
/1
#9DE93903
 Úr Isten, mi sok szükséget
 Érezvén körülöttünk,
 Segedelemkérés végett
 E szent helyre feljöttünk.
 Uram, a könyörgőnek
 Illesd meg szívét és száját,
 A téged tisztelőnek
 Áldd és szenteld meg munkáját.

;Lengyel József, 1770–1822
>167. Úr Jézus, mely igen drága
/1
#E6D9E522
 Úr Jézus, mely igen drága
 A te igédnek világa,
 Mely bölcscsé tévén az elmét,
 Szüli az Úrnak félelmét.
 Gerjeszd fel most indulatunk,
 Hogy míg igédre hallgatunk,
 Végyen bennünk épületet
 A hit, reménység és szeretet:
 Tégy bölcsekké, tégy szentekké!

;Névtelen szerző, 1648. F.: Schulek Tibor
>168. Urunk Jézus, fordulj hozzánk
/1
#0976D958
 Urunk Jézus, fordulj hozzánk,
 Szent Lelkedet ma töltsd ki ránk,
 Kegyelmeddel minket segélj,
 Az egy Igazságra vezérlj.
/2
#03F09914
 Nyisd meg szánkat hál’adásra,
 Készítsd szívünk buzgóságra,
 Hitünk s értelmünk neveljed,
 Neved velünk ismertessed.
/3
#E2DEB1AD
 Míglen éneklünk mennyégben:
 Szent, szent, szent az erős Isten,
 És színről színre láthatunk,
 A fényességben vigadunk.
/4
#6B4471D4
 Dicsőség Atya Istennek,
 Fiúnak és Szentléleknek:
 A dicső Szentháromságnak
 Mindenek áldást mondjanak!

>Záró énekek

;Csomasz Tóth Kálmán
;a II. Kor 13,13 alapján
>169. Az Úr Jézus Krisztusnak kegyelme
/1
#162C6B26
 Az Úr Jézus Krisztusnak kegyelme,
 És Atyánknak, Istennek szerelme,
 Szentlélekkel áldott közösségben
 Légyen mindig mindnyájunkkal. Ámen.

;Új Zengedező Mennyei Kar, 1743
>170. Áldjuk Istent végével
/1
#E4D118E6
 Áldjuk Istent végével
 Isteni tiszteletünknek,
 Mondjunk immár örömmel
 Dicséretet szent nevének!
 Mi Istenünk legyen áldott,
 Hogy lelkünkben gazdagított.
/2
#AFE1C3FB
 Immár Isten áldása
 Kimondatott mindnyájunkra,
 Menjünk örömmel haza
 Tisztünkre és dolgainkra.
 Ő Szentlelke erősítsen,
 Minket tovább is vezessen.
/3
#3D63F8D4
 Áldja meg kimentünket,
 Áldja bemenetelünket,
 Áldja meg kenyerünket,
 Szentelje meg életünket;
 Áldjon meg boldog halállal,
 Végre örök boldogsággal.

;F.: Csomasz Tóth Kálmán
;Liturgikus szöveg a IV. századból
>171. Dicsőség az Atyának, A Fiúnak
/1
#6FE65388
 Dicsőség az Atyának,
 A Fiúnak és Szentlélek Istennek,
 Valamint volt kezdetben,
 Azonképpen most és mindörökké,
 Ámen, Ámen.

;Stegmann Józsué, 1588–1632
;F.: Árokháty Béla és Payr Sándor
>172. Ó, maradj kegyelmeddel Mivelünk
/1
#4BC411E3
 Ó, maradj kegyelmeddel
 Mivelünk, Jézusunk,
 Hogy a bűnös világnak
 Tőribe ne jussunk.
/2
#E3638327
 Ó, maradj szent igéddel
 Mivelünk, Megváltónk,
 E földi vándorlásban
 Te légy útmutatónk.
/3
#CD7DE584
 Ó, maradj, világosság,
 Mivelünk fényeddel,
 Te vezess a sötétben,
 Hogy ne tévedjünk el.
/4
#A51C89B2
 Ó, maradj áldásoddal
 Mivelünk, Úr Isten,
 Szent kegyelmed áraszd ránk
 Minden szükséginkben.
/5
#FBE651B4
 Ó, maradj oltalmaddal
 Mivelünk, Hű pajzsunk,
 Hogy e világ diadalt
 Ne vehessen rajtunk.
/6
#1745A62D
 Ó, maradj hűségeddel
 Mivelünk, szent Isten,
 Adj erőt, hogy megálljunk
 Mindvégig a hitben.

;Bornemisza Péter énekeskönyve
>173.  Ti keresztyének, dicsérjétek Istent
/1
#DD8D5A7D
 Ti keresztyének, dicsérjétek Istent,
 Kik Úr Jézusnak e földön szolgáltok
 És mindenkoron a szent helyen vagytok:
 Dicsérjétek Istent!
/2
#9D4CD86B
 Imádkozzatok az Atya Istennek,
 Jézus nevében hozzá siessetek,
 És tiszta szívet hozzá emeljetek:
 Dicsérjétek Istent!
/3
#557D6DA1
 Áldjon meg minket az Atya Úr Isten,
 Megigazítson a Fiú Úr Isten,
 És megszenteljen a Szentlélek Isten
 Mindenkoron! Ámen.

;Szentgyörgyi Kemény János
>174. Uram, bocsásd el népedet békével
/1
#48623F97
 Uram, bocsásd el népedet békével,
 Idvezítésed kívánt örömével,
 Mert megtaláltuk, amit lelkünk kedvel,
 Megragadjuk hittel, nem bocsátjuk el;
 Már ő miénk, mi vagyunk az övéi,
 Véle egy testek, java részesei.
/2
#1C6119A2
 E gyarló testben Jézus él már, nem mi,
 Hű szerelmétől nem szakaszt el semmi.
 Ó, Jézus, gyenge hitünket neveljed,
 Te magad lelkünk az égre emeljed,
 Hogy így lelkünket az ég dicső vára
 Fogja be, annak örökös javára.

;Marot K., 1496–1544
;Lk 2,29–32 alapján
;F.: Szenczi Molnár Albert
>175. Uram, bocsássad el Szolgádat békével
/1
#E1DC88DA
 Uram, bocsássad el
 Szolgádat békével,
 Szent ígéreted szerint,
 Mert te idvezítőd
 Én szemeim előtt
 Nékem nyilván megjelent.
/2
#6833643A
 Kit világos fényül
 Pogányoknak küldél,
 Kinek fényességével
 Nyilván kijelenék
 A szent Izráelnek
 Nagy dicsősége széjjel.
/3
#0CA70A29
 Őt, ki kegyelmedből
 Leszállt a mennyekből
 Minden népek üdvére,
 Én immár megláttam
 És szívembe zártam
 Lelkemnek örömére.

>Advent

;Szilágyi Bálint, †1807
>176. Álmélkodással csudáljuk
/1
#446EE574
 Álmélkodással csudáljuk
 Véghetetlen szerelmed,
 Ó, Isten, ha megvizsgáljuk
 Kijelentett kegyelmed;
 Ezt száj ki nem mondhatja,
 Nyelv nem magyarázhatja.
/2
#5B553E67
 Mert az emberi nemzetet
 Annyira becsülötted,
 Hogy te egyetlenegyedet
 Érette elküldötted
 Emberi ábrázatban,
 Hogy élne gyalázatban.
/3
#6C7FCD8F
 Ó, Isten bölcsességének
 Megfoghatatlan titka!
 Mély tengerét szerelmének
 Ki értené, mily ritka;
 Ki ezt eszébe venné,
 Mélyen szívébe tenné.
/4
#03F46FF7
 Mi azért vígan dicsérünk,
 Ó, jó Atyánk, tégedet,
 Magasztalunk s arra kérünk,
 Hogy te szeretetedet
 Gerjesszed fel szívünkben,
 Jobban-jobban lelkünkben.

;Herman Miklós, 1480–1561
>177. Dicsérd Istent, keresztyénség
/1
#B6461ECB
 Dicsérd Istent, keresztyénség,
 Ő dicsőségében,
 Kitől árad rád idvesség,
 Fia érdemében, Fia érdemében.
/2
#99008308
 Leszállván Atyja kebléből,
 Kicsiny gyermek korban,
 Szegénységben, ruha nélkül
 Fekszik a jászolban,
 Fekszik a jászolban.
/3
#A18ECF44
 Letette minden hatalmát,
 Erőtelen vala, Felvette szolga formáját
 Mindeneknek Ura, Mindeneknek Ura.
/4
#BC6980E1
 Anyjának ölében nyugszik,
 Tejével táplálja;
 Az angyalok ezt örvendik,
 Mert Dávidnak fia, Mert Dávidnak fia.
/5
#8D04C902
 Ki az utolsó időben
 Eljövendő vala,
 Hogy építtessék hívekben
 Az Isten országa, Az Isten országa.
/6
#D26EAC6A
 Csuda változással testet
 Ő magára felvőn,
 Adván nékünk idvességet,
 Mennyben részessé tőn,
 Mennyben részessé tőn.
/7
#BE330222
 Én úr, ő pedig lett szolga,
 Ó, csuda változás!
 Jobbat Jézus mit adhatna:
 Nékünk boldogulás,
 Nékünk boldogulás.
/8
#6A1F9642
 Ma Paradicsom kapuját
 Ismét megnyitotta,
 Kérub nem állja ajtaját,
 Ezért minden áldja,
 Ezért minden áldja.

;Uray Sándor
>178. Drága Advent, köszöntünk
/1
#3320B881
 Drága Advent, köszöntünk,
 Lelkünk téged epedve várt,
 Hajnalfényed közöttünk
 Hintsen szerte aranysugárt;
 Jövel, Adventnek királya,
 Népek Messiása! Jézusunk! alázattal
 Fogadunk szent hódolattal.

;Nagy István, 1770–1831
>179. Igaz Isten, ígéretedben
/1
#6151F769
 Igaz Isten, ígéretedben
 Változhatatlan valóság!
 Amit te a te beszédedben
 Megmondasz, az  mind valóság.
 Köny-nyebb megavulni,
 Vég-képpen elmúlni
 A természetnek, Mint semmibe menni
 Az igaz isteni Szent ígéretnek.
/2
#81779112
 Megmondottad volt még kezdetben
 Az első egy pár embernek,
 Hogy ők a szomorú esetben
 Mindörökké nem hevernek:
 Küldesz vigasztalót,
 Ő magvukból valót, Ki eredeti
 Elvesztett jussukba
 És boldogságukba Visszahelyezi.
/3
#41EF91FC
 Meglett ez, és a Szűz méhében
 Fogantatott az, akinek Jézusi felséges nevében
 Áldás szól a föld népinek.
 Így az ígértetett És kijelentetett
 Teljes időnek Hajnala felderült
 A sötétségben ült Sok kesergőnek.
/4
#C9377E73
 Lelki örömmel megújulva
 Imádjuk szent Felségedet,
 Hogy ismét mireánk fordulva
 Szemléljük régi kedvedet.
 Megnyitjuk szívünket,
 Tárjuk kebelünket,
 Hogy Jézusunkat
 Ekképpen fogadjuk,
 Híven általadjuk Néki magunkat.
/5
#64F617FB
 Tarts meg bennünket az országban,
 Melyet ő köztünk állított,
 Adj részt abban a boldogságban,
 Melyet a földre szállított!
 Majd éltünk végével
 Bocsáss el békével,
 Hogy oda térjünk,
 Hol az ő hívei Száma közt mennyei
 Országlást érjünk.

;Szabolcska Mihály, 1862–1930
>180. Isten, aki népedet
/1
#41BF2392
 Isten, aki népedet
 Bűnben, gyászban nem hagyod,
 Rád tekintve várjuk a Betlehemi csillagot;
 Feltűnése lesz nékünk
 Új világunk, reggelünk.
/2
#FD10FBF3
 Ez a fény ha felragyog
 Napkelettől nyugotig,
 Hervadó mezők felett
 Angyalének hallatik:
 Szíveket vigasztaló,
 Könnyeket szárítgató.
/3
#074D5808
 Idvözítő Jézusunk,
 Te vagy a mi reggelünk,
 Bátorító szódra ha
 Bűneinkből felkelünk:
 A te égi szózatod
 Hoz nekünk bocsánatot.
/4
#DA0C1A81
 Vágyva vágyunk, nagy Király,
 ünnepelni jöttödet;
 űnagy örömmel fogadunk,
 Jöjj, siess, te szent követ!
 Mi hideg jászol helyett
 Szívünkben adunk helyet.
/5
#96CEE88D
 Isten, aki szót adál
 Prófétáid ajkira,
 Téged áldva néz szemünk
 Betlehem határira;
 Várva várjuk őt, aki
 Jön minket megváltani.

;Luther Márton,
;Ambrosius milánói püspök (340–397)
;éneke alapján
>181a. Jöjj, népek Megváltója
/1
#1796CF51
 Jöjj, népek Megváltója,
 Szűznek ékes virága,
 Mind e világ csudálja,
 Mint jöttél, Isten Fia.
/2
#91719D90
 Nem emberi erőtől,
 De Szent Lélek Istentől
 Ige testbe öltözék,
 Szűz méhe megvirágzék.
/3
#05FF6A50
 Jöve ágyas házából,
 Tiszta szűz szent méhéből;
 Isten, ember ő egyben,
 Eljött hozzánk már testben.
/4
#6CE6BDC3
 Szent Atyjától földre jött,
 Szállt poklokra és győzött,
 Atyjához emelteték,
 Ő székibe ülteték.
/5
#3FC7C411
 Atya Isten szent Fia,
 E világnak istápja:
 Gyarló néped sok baja
 Rád szállt, légy bajvívója!
/6
#DBF3EB64
 Jászolod immár fénylik,
 Új világa tündöklik,
 Melytől éj elenyészik,
 Hitünk megerősödik.
/7
#FC98AD36
 Dicsőség néked, Urunk,
 Mi kegyelmes Megváltónk!
 Dicsőség szent Atyádnak
 És mi Vigasztalónknak!

;Luther Márton,
;Ambrosius milánói püspök (340–397)
;éneke alapján
>181b. Jöjj, népek Megváltója
/1
#DE29E452
 Jöjj, népek Megváltója,
 Így kér a föld lakója,
 Jöjj, lelkünk drága fénye,
 Szívünk édes reménye.
/2
#C8218A39
 Jöjj Te, az égnek szentje,
 Az Atya egy felkentje,
 Jessének bimbós ága,
 Szűz testének virága!
/3
#6982B350
 Te érkezel Atyádtól
 Mennybéli palotádból,
 Hogy itt megvívj miértünk
 És légy mi pajzsunk, vértünk.
/4
#BE473DF5
 Ím, készen vár a jászol,
 A szív többé nem gyászol,
 Az éjszaka világos,
 Az ösvény barátságos.
/5
#AC5E0746
 Ó, szállj közénk, királyunk!
 Íme, eléd kiállunk:
 A sötétség elmúljon,
 A hit fénye felgyúljon.

;Szőnyi Benjámin, 1717–1794
;
>182. Kapuk, emelkedjetek! Kiáltó szó hallik
/1
#6A37AD49
 Kapuk, emelkedjetek!
 Kiáltó szó hallik,
 Ím, jő fejedelmetek,
 Az idő hajnallik:
 Harmattal rakott feje,
 Véle sok áldása,
 Bétölt teljes ideje,
 Hogy minden test lássa.
/2
#3E167657
 Ímhol jő a Vőlegény,
 Lelkem, menj elébe,
 Keresd nyugtod, mint szegény,
 Gazdag kebelébe’.
 Kedves vendéget várok,
 Szívem ajtajárul
 Hulljanak a závárok,
 Mert már közel járul.
/3
#1F55303F
 Már az ég harmatozzon,
 A föld igazságot,
 Mint bő gyümölcsöt, hozzon:
 Indíts vigasságot,
 Ó, kegyes Immánuel,
 Mert várlak valóba’;
 Ó, te kisded Sámuel,
 Jöjj el már Silóba!
/4
#FFD7BD87
 Az utat egyengessed,
 Szívemben a mérget
 S ürömgyökért égessed,
 Lelkemről a kérget,
 A keménységet vedd ki,
 Hogy meglágyulhassak,
 A gazt s gerendát szedd ki
 Szememből, hogy lássak.
/5
#B8FAD82F
 Az Illésnek lelkével
 Ruházz fel engem is,
 Szent szerelmed tüzével
 Égjen én lelkem is.
 Áldj meg oly kegyességgel,
 Hogy higgyek és szóljak,
 Hűségedre hűséggel
 Másokat unszoljak.
/6
#E59A67F9
 Isten Báránya, jövel,
 Mutasd szelídséged;
 Uram, felemelt fővel
 Várom idvességed.
 Igazság napja, támadj,
 Adj világosságot,
 Magamnak nincs: reám adj
 Örök igazságot.
/7
#316478B3
 Dávid gyökere s ága,
 Fényes hajnalcsillag,
 Pogányok kívánsága,
 megígért áldott mag;
 Isai törzsökéből
 Származott vesszőszál:
 Nékem Atyád kedvéből
 Erős torony s kőszál!
/8
#106A52E3
 Bennem az Úr temploma
 Általad készüljön,
 Vesszen a bűnnek nyoma,
 Lelked újjászüljön.
 Méltóztass személyedre
 E gyarló világon,
 Dicső jelenlétedre
 Dűljön le a Dágon.
/9
#C5BDFBEF
 Tégy szívedre pecsétül,
 Bélyegül karodra:
 Így lelked erejétül
 Élek csak számodra.
 Mindvégig velem maradj
 Mennyei erővel,
 Holtom óráján ne hagyj,
 Jövel, Ámen, jövel!

;Hegymegi Kiss Áron
;vagy Láczai Szabó József
>183. Kegyes lelkek, az Urat dicsérjétek
/1
#D554D1BD
 Kegyes lelkek, az Urat dicsérjétek,
 Áldott Idvezítőnket tiszteljétek,
 Ki bételjesíté, amit ígére,
 Magát megalázván Isten létére,
 Felvevé testünket, Eltörlé bűnünket
 Szent ártatlan voltával;
 Fogadjuk hát, midőn
 Mostan is hozzánk jön,
 Dicséret mondásával.
/2
#812E082C
 Lelkünk sebe már nem gyógyulhatatlan,
 Mert meggyógyítá azt a halhatatlan;
 Magára vett minden viszontagságot,
 Hogy így készítsen lelkünknek váltságot.
 Felkeresé nyáját, Elvégzé munkáját,
 Végyen hát dicsőséget!
 Zengjen annak ének,
 Ki hozott népének
 Ily kívánt idvességet!
/3
#972D2CE9
 Ne vess meg, Jézusunk, tovább is kérünk,
 Légy életünkben hatalmas vezérünk;
 Szent igéd és Lelked segítsen minket,
 Hogy meggyőzzük lelki ellenséginket.
 Aki úgy szerettél, Hogy emberré lettél,
 Isten lévén, érettünk:
 Add viszont szeretnünk,
 Szent példád követnünk,
 Valamíg tart életünk.

;Thilo Bálint, 1607–1662
;F.: Czeglédy Sándor
>184. Várj, ember szíve, készen! Mert jő a Hős, az Úr
/1
#6B603E9B
 Várj, ember szíve, készen!
 Mert jő a Hős, az Úr,
 Ki üdvösséged lészen.
 Szent győztes harcosúl,
 Fényt, éltet hozva jő,
 Megtört az ősi átok:
 Kit vágyakozva vártok,
 Betér hozzátok Ő.
/2
#93686B6D
 Jól készítsétek útát!
 A Vendég már közel!
 Mi néki gyűlölt, útált,
 Azt mind vessétek el!
 A völgyből domb legyen,
 Hegycsúcs a mélybe szálljon,
 Hogy útja készen álljon,
 Ha Krisztus megjelen.
/3
#9492E520
 Az Úr elé ha tárod A szív alázatát,
 Őt nemhiába várod:
 Betér hozzád, megáld.
 A testi gőg: halál!
 De bűnödet ha bánod,
 Szent Lelke bőven árad,
 S a szív üdvöt talál.
/4
#3B9E6C77
 Ó, Jézusom, szegényed
 Kér, vár, epedve hív:
 Te készítsd el: tenéked
 Lesz otthonod e szív.
 Jer hű szívembe hát!
 Habár szegény e szállás,
 De mindörökre hálás,
 Úgy áldja Krisztusát.

;A „Veni Immanuel” középkori himnusz XIII. századi változatából
;Szedő Dénes és Hamar István fordította
>185. Ó, jöjj, ó, jöjj, Immánuel
/1
#4CEE353E
 Ó, jöjj, ó, jöjj, Immánuel,
 Hogy mentsd meg, kérlel Izráel.
 Ím, fogságból kél sóhaja,
 Oly meszsze Isten szent Fia.
 Örvendj, örvendj, ó, Izráel,
 Mert eljövend Immánuel!
/2
#47494659
 Ó, jöjj el, Jesse vesszeje,
 És állj a Rossznak ellene:
 A mélyből, mely már eltemet,
 A sírból hozd ki népedet!
 Örvendj, örvendj, ó,
 Izráel, mert eljövend Immánuel!
/3
#419F8830
 Ó, jöjj, ó, jöjj el, Napkelet,
 Hogy megvigasztald népedet:
 Törd át a sűrű éj ködét,
 És oszlasd gyászát szerteszét!
 Örvendj, örvendj, ó, Izráel,
 mert eljövend Immánuel!
/4
#CE00CE73
 Jöjj, Dávid kulcsa, nyisd ki hát
 Az égi honnak ajtaját,
 A mennyországba tárj utat,
 És rekeszd el a poklokat!
 Örvendj, örvendj, ó, Izráel,
 mert eljövend Immánuel!
/5
#E9B4648F
 Ó, jöjj, ó, jöjj el, Adonáj,
 Ki forgószélből szólalál,
 Így adtál törvényt népednek,
 Jöjj, fenségedben jelenj meg!
 Örvendj, örvendj, ó, Izráel,
 mert eljövend Immánuel!

;Abaelard Péter (1079–1142) éneke után
;Kassa, 1662
>186. Küldé az Úr Isten Hűséges szolgáját
/1
#BBE029B7
 Küldé az Úr Isten Hűséges szolgáját
 Szűzhöz Názáretben Gábriel angyalát
 Hozzánk jókedvében:
/2
#10AFA717
 Menj el Máriának
 E jót megmondani,
 A régi Írásnak
 Titkát jelentsed ki
 Angyali erőddel.
/3
#B01106CA
 Mondd ezt: ó,
 szent, kegyes,
 Üdvözlésem végyed;
 Ajándékkal teljes,
 Az Úr van tevéled;
 Szűnjék hát félelmed.
/4
#48A5BC55
 Szent szűz,
 méhedbe vedd
 Az Úr Isten Fiát,
 Melyben megőrizzed
 A szüzesség jussát,
 Minden tisztaságát.
/5
#7332BDEB
 Hallá s elfogadá
 E parancsolatot,
 Hivé és fogada
 Méhében magzatot,
 De nagy Csudálatost,
/6
#52D3E80B
 Emberi nemzetnek
 Hű tanácsadóját,
 Jövendő életnek
 Maradandó atyját
 Örök békességben;
/7
#3DAE8066
 Kinek erőssége
 Minket úgy őrizzen,
 Hogy a világ vétke
 Minket meg ne sértsen,
 Pokolra ne vessen;
/8
#423B096D
 De a Bűnbocsátó
 Végye el vétkünket,
 Légyen igazítónk
 S adjon örökséget
 Mennyek országában.

;Nicolai Fülöp hamburgi lelkész, 1556–1608
;
>187. Szép tündöklő hajnal csillag
/1
#DD50F653
 Szép tündöklő hajnal csillag,
 Ki kegyelemmel felragyog,
 Jessének szép veszszeje,
 Dávid királynak magzatja,
 Sok királyoknak királya,
 Lelkemnek vőlegénye:
 Kedves, kegyes, kellemetes,
 ékes és gyönyörűséges,
 Gazdagsággal dicsőséges.
/2
#D7EBE6A6
 Ó, gyöngyös, drága korona,
 Embernek, Istennek fia,
 Menny áldott, szent gyümölcse!
 Szívemnek vagy lilioma,
 Édes evangélioma,
 Kedvességeknek kincse!
 Eggyem, lelkem,
 Szép violám, égi mannám, eledelem:
 Rólad el nem feledkezem.
/3
#36767F0E
 Öntsd mélyen az én szívembe,
 Szerelmed tüzét lelkembe,
 Ó, drága jáspis kövem!
 Fogadj hozzád s vigasztalj meg,
 Hogy eleven tagod legyek,
 Benned legyen örömem.
 Hozzád kiált
 Sebhedt szívem, lelki rózsa:
 légy orvosa;
 Nélküled nincs vigassága.
/4
#C31D666F
 Mennyből nagy öröm fénylik rám,
 Midőn szemeddel énreám
 Kegyelmesen tekintesz,
 Ó, Uram Jézus, szent
 Lelked, Szent igéd,
 tested és véred
 Vélem közölvén, éltetsz.
 Ilyen híven Tarts öledben,
 táplálj engem mindvégiglen,
 Mert fogadtad szent igédben.
/5
#43DB2814
 Teremtő Atyám, Úr Isten,
 Ki engem örök időkben
 Szent Fiadban szerettél:
 Ő engem magának jegyzett,
 Szentlélekkel elpecsétlett,
 Megtisztított vérével.
 Vigadj és adj Szívem hálát,
 Istent imádd, hogy mennyégben
 Idvességet szerzett nékem.
/6
#03110213
 Szóljon tehát az orgona,
 Gyönyörűséges muzsika,
 Ujjongva énekeljek,
 Hogy megváltó Jézusomnak,
 Drága szép kegyes
 Uramnak Szerelmében örvendjek.
 Zengjen, pengjen
 Vigasságos, buzgóságos hál’adó szó,
 Nagy az Úr, ezekre méltó.
/7
#52D75E0A
 Ily kedvvel vigad én szívem,
 Mert drága kincsem az Isten,
 Ki kezdet s vég mindenben.
 Ő engem dicséretire,
 Mennybe viszen nagy örömre,
 Min tapsolok szívemben.
 Ámen, ámen! Jövel,
 Uram, szép koronám, jöjj sietve:
 Téged várlak reménykedve!

;Szabolcska Mihály, 1862–1930
;
>188. Szállj, szállj magasra, szíveink reménye
/1
#92DED2A9
 Szállj, szállj magasra, szíveink reménye,
 Vezess el minket Jézusunk elébe;
 Ragyogj előttünk fénynek oszlopával,
 Szent biztatással.
/2
#ACF28DF1
 Jön már a Jézus, a mi sziklavárunk,
 Nem tart soká már bűnben bujdosásunk,
 Az ígéretnek földére érkezünk,
 Ő lesz mivelünk.
/3
#5CFFE9CE
 Ő lesz a váltság, élet birodalma,
 Fordulj örömre, szívünk aggodalma!
 Az elhagyottnak lesz már pártfogója,
 Oltalmazója.
/4
#191D8CC0
 Hitünknek lesz majd diadalma teljes:
 Hozzánk az Isten irgalmas, kegyelmes;
 A szeretetnek fényes napja jő fel
 Idvezítőnkkel!
/5
#4B9920AD
 Szállj, szállj magasra, szíveink reménye,
 Vezess el minket Jézusunk elébe!
 Mert a váltságot ő hozza földünkre
 Nagy örömünkre.

>189. Új világosság jelenék
/1
#4970FB21
 Új világosság jelenék,
 Ó, tévelygés csendesedék;
 Isten igéje jelenék,
 Újonnan nékünk adaték.
/2
#0A2D0764
 Evangéliom erejét,
 Krisztust, áldott szent
 Igéjét, Atya Isten nagy jó kedvét,
 Megmutatá ő kegyelmét.
/3
#A700F70A
 Kit sok száz esztendeiglen
 Eltitkolt volt Atya Isten,
 Mint megmondá jövendölvén
 Ámos próféta könyvében.
/4
#276C9444
 Ezt a mi hitetlenségünk
 És nagy telhetetlenségünk,
 Érdemlette tévelygésünk,
 Emberbeli reménységünk.
/5
#EFE7804E
 Igaz az  Isten Igéje,
 Kivel él ember elméje,
 Kinek megmarad ereje
 És el nem vész ő reménye.
/6
#12F006CF
 Kérünk, Úr Isten, tégedet,
 Erősítsd meg híveidet,
 Hogy vehessük szent Igédet
 És vallhassuk te hitedet.
/7
#41130391
 Mert csak te vagy bizodalmunk,
 Ördög ellen nagy gyámolunk,
 Testünk ellen diadalmunk,
 E világ ellen oltalmunk.
/8
#E7D7F4CA
 Dicsőség légyen Atyának
 És egyetlenegy Fiának,
 Ezeknek Ajándékának;
 A dicső Szentháromságnak.

>Karácsony

>190. Jer, mindnyájan örüljünk
/1
#45BA95AD
 Jer, mindnyájan örüljünk,
 És szívünkben vigadjunk,
 Mert született Úr Jézus nekünk.
/2
#054D6B82
 Kit az Atya Úr Isten,
 Könyörülvén emberen,
 Elbocsájta teljes időben.
/3
#5808CAD2
 Elhagyá gazdagságát,
 Véghetetlen országát,
 Hogy érettünk adja önmagát.
/4
#EE5A4187
 Ő életnek adója,
 Szívek vigasztalója,
 Lelkünk megvilágosítója.
/5
#79A151D1
 Azért jöve, hogy éljünk,
 Isten kedvébe essünk,
 Érdeméből kegyelmet nyerjünk.
/6
#E8115CF6
 Nagy szeretet mindenhez,
 Hogy Isten emberekhez
 Jöve, fertelmes bűnösökhöz.
/7
#CECC6E69
 Akik benne nem bíznak,
 Sőt bízni sem akarnak,
 Örök halállal mind meghalnak.
/8
#9150B28E
 Mi azért e felségben,
 Emberré lett Istenben:
 Higgyünk mi egy reménységünkben.

>191. Az Istennek szent angyala
/1
#DD9F7B10
 Az Istennek szent angyala
 Menynyekből hogy alászálla,
 És a pásztorokhoz juta,
 Né-kiek ekképpen szóla:
/2
#AF8CF369
 Mennyből jövök most hozzátok,
 És íme, nagy jó hírt mondok,
 Nagy örömet majd hirdetek,
 Melyen örvend ti szívetek.
/3
#6DF55558
 E mai nap egy kis gyermek
 Szűztől született tinéktek,
 A gyermek szép és oly ékes,
 Vigasságra kellemetes.
/4
#18277BBC
 Már lehozta az életet,
 Mely Istennél volt készített,
 Hogy ti is véle éljetek,
 Boldogságban örvendjetek.
/5
#B2024F9B
 Ez lesz néktek a jegy róla:
 Ímé, fekszik a jászolba’,
 Ott megtaláljátok őtet,
 Kitől menny, föld teremtetett.
/6
#BFE2987B
 Ez Úr Krisztus mi Istenünk,
 Nyavalyáinkból kimentőnk,
 Ő lészen az Idvezítő,
 Minden bűnünkből kivévő.
/7
#BF92F089
 Nyílj meg, szívem, lásd meg jobban,
 Ki fekszik itt e jászolban?
 Ez a gyermek bizonyára
 Az Úr Jézus, Isten fia.
/8
#4965F6D2
 Jertek hát, mi is örvendjünk,
 A pásztorokkal bémenjünk,
 Lássuk, mit adott az Isten
 Hozzánk való szerelmében.
/9
#7B5449B2
 Mindeneknek teremtője,
 Miért vagy ily szegénységbe’?
 Hogy fekszel az aszú szénán,
 Szamár s ökrök közt aludván?
/10
#3E6561B9
 Nincs-é senki e világon,
 Ki tégedet béfogadjon?
 Nincsen-é meleg helyecskéd,
 Sem gyengén rengő bölcsőcskéd?
/11
#6BC46120
 Néked bársonyod s tafotád,
 Aszú széna lágy párnácskád;
 Noha nagy dicső király vagy,
 Mostan ímé, mily szegény vagy!
/12
#F7FE2FEF
 Ó, én szerelmes Jézusom,
 Édes megváltó Krisztusom!
 Jövel, csinálj csendes ágyat,
 Szívemben magadnak házat!
/13
#B541FF97
 Ó, kedves vendég, nálam szállj,
 Bűnömtől ne iszonyodjál,
 Jöjj be hozzám, te szolgádhoz,
 Szegény megtérő juhodhoz!
/14
#2E2BB5AE
 Én lelkemnek rejtekébe,
 zárkózz emlékezetébe,
 Hogy el ne felejthesselek,
 Sőt örökké dicsérjelek!
/15
#CDD580BE
 A mennyei magas égben
 Istennek dicsőség légyen,
 Ki szent Fiát küldé értünk,
 Hogy Megváltónk lenne nékünk.

>192. Dicsőség a magas menynyekben
/1
#C6C7A674
 Dicsőség a magas menynyekben
 Istennek és ide alatt,
 A földi alacsony helyekben
 Békesség és jóakarat!
 Így énekelnek az Istennek
 Az égi karok, buzdítván
 Az élőket, kik örvendeznek,
 Ez éneklést megújítván.
/2
#FBEB92E7
 Dicsőség, dicsőség az égben
 Istennek, ki úgy szerette
 E világot, hogy szegénységben
 Szent Fiát eleresztette,
 Hogy minket, gyarló halandókat,
 Kiket szomorú fogságba’
 Tart vala a bűn, mint rabokat,
 Helyheztessen szabadságba.
/3
#CC13A16D
 Ó, emberi testbe öltözött
 Jézusunk, lásd, mint gerjedez
 A mi szívünk az öröm között,
 Úgy újul és úgy éledez,
 Mint mikor a nap feljöttével,
 Kilövellvén az életet,
 Elűzi az éjjelt s fényével
 Felkölti a természetet.
/4
#42D73A0B
 Jövel, fogadd el te magadnak
 E szívet és lakozz ebben;
 Ez az, amit adhatnak s adnak
 Híveid legszívesebben
 Azért a csuda szeretetért,
 Amelyet hozzánk mutattál,
 Midőn a mi már vesztére tért
 Lelkünkért alászállottál.
/5
#C2330E05
 Hozd el mihozzánk te magaddal
 Az isteni békességet;
 Bűnös lelkünknek irgalmaddal
 Nyújts biztatást, reménységet,
 Hogy bús háborúnk, amely belől
 Régóta szaggat már és tép,
 Ne kezdődjék elől, meg elől,
 Hanem hallgasson el végképp.
/6
#FB4596B3
 Szülj újjá, értünk ma született
 Jézusunk, a te lelkeddel
 Ezen a néked szenteltetett
 Ünnepnapon, s kegyelmeddel
 Úgy igazgass és bírj bennünket,
 Hogy nyomdokid követhessük,
 És e mi földi életünket
 Mennyeivel cserélhessük.

>193. Dicsőség az Istennek, Dicsőség az Istennek
/1
#8C853C3C
 Dicsőség az Istennek, Dicsőség az Istennek
 Fenn a magas egekben!
 Jóakarat, békesség, Jóakarat, békesség
 Az emberi szívekben!
 Zengjen a hála, Mert földre szállt az ég királya!
 Szeretet és igazság jön vele,
 Kezében a szabadság fegyvere,
 Nyissatok hát előtte szívet,
 Ti megváltott hívek!

>194. Dicsőség mennyben az Istennek!
/1
#BF335F1A
 Dicsőség mennyben az Istennek!
 Dicsőség mennyben az Isten -nek!
 Az angyali seregek Vígan így énekelnek:
 Dicsőség, dicsőség Istennek!
/2
#02D6B998
 Békesség földön az embernek!
 Békesség földön az embernek,
 Kit az igaz szeretet A Jézushoz elvezet,
 Békesség, Békesség Embernek!
/3
#1A5A647D
 Dicsérjük a szent angyalokkal,
 Imádjuk a hív pásztorokkal
 Az isteni Gyermeket,
 Ki minket így szeretett,
 Dicsérjük, Imádjuk És áldjuk!
/4
#852CB64B
 Ó, Jézus! ne vess meg bennünket,
 Hallgasd meg buzgó kérésünket!
 Jászolodnál fogadjuk,
 Hogy a vétket elhagyjuk,
 Ó, Jézus, Ne vess meg:
 Hallgass meg!
/5
#E2988039
 Dicsőség az örök Atyának
 És értünk született Fiának,
 Mindkettő Szent Lelkének,
 Áldások kútfejének:
 Dicsőség, Dicsőség Istennek!

>195. Ez nap nékünk dicséretes nap
/1
#2597DDB1
 Ez nap nékünk dicséretes nap,
 Bizony vigasságnak napja,
 És idvességnek bizodalma,
 Mert született ez nap nékünk mi váltságunkra
 A Krisztus Jézus, Istennek Fia.
 Áronnak veszszeje megvirágozék,
 Tiszta szűztől gyermek születék,
 Menynyei királyul nékünk adaték,
 Krisztus Jézusnak nevezteték.
/2
#460F44ED
 Ez Gyermek volt a megígért mag
 A mi első atyáinknak, Ádám atyánknak,
 Ábrahámnak,
 Kiben minden nemzetségek megáldatnának,
 Örök életre feltámadnának.
 Áronnak vesszeje megvirágozék,
 Tiszta szűztől gyermek születék,
 Mennyei királyul nékünk adaték,
 Krisztus Jézusnak nevezteték.
/3
#C58ED310
 Megtöreték e Gyermek miatt
 Az ördögnek nagy hatalma,
 A halál, ördög, bűn országa;
 Megnyittaték mennyországnak erős kapuja:
 Istennek kedve mireánk szálla.
 Áronnak vesszeje megvirágozék,
 Tiszta szűztől gyermek születék,
 Mennyei királyul nékünk adaték,
 Krisztus Jézusnak nevezteték.
/4
#A2063982
 Nincsen immár semmi félelmünk
 A mi nyomorúságinktól,
 Bűntől, haláltól, kárhozattól,
 Sem a Mózes törvényének kemény átkától,
 Ördögnek rajtunk nagy bosszújától.
 Áronnak vesszeje megvirágozék,
 Tiszta szűztől gyermek születék,
 Mennyei királyul nékünk adaték,
 Krisztus Jézusnak nevezteték.
/5
#7478312A
 Megtöretnek a pogány népek,
 Kik e Gyermekben nem hisznek;
 A nagy Istennek nem kellenek,
 Az ördögnek hatalmába örökké esnek,
 Akik a bűnnek véget nem vetnek.
 Áronnak vesszeje megvirágozék,
 Tiszta szűztől gyermek születék,
 Mennyei királyul nékünk adaték,
 Krisztus Jézusnak nevezteték.
/6
#A4E44FE3
 Hálát adjunk az Úr Istennek,
 Atya-Fiú-Szentléleknek,
 És örüljünk mind e Gyermeknek;
 Nagy örömet az angyalok nékünk hirdetnek,
 Dicsérvén Istent, így énekelnek:
 Áronnak vesszeje megvirágozék,
 Tiszta szűztől gyermek születék,
 Mennyei királyul nékünk adaték,
 Krisztus Jézusnak nevezteték.

>196. Hogy eljött az időknek teljessége
/1
#0FB2C5E5
 Hogy eljött az időknek teljessége,
 Bétölt már minden szentek reménysége;
 Kit régtől fogva minden szent vára:
 A Fiú testet öltött magára.
 Nyilván lett hozzánk Isten jó szándéka:
 Ím, emberek közt van az ő hajléka;
 E világ éppen mármár megavult,
 De Jézus eljöttével megújult.
/2
#EA60E84F
 Uraknak Ura értem lett szolgává,
 Tévén én dolgom a maga dolgává.
 Ami a testnek nem volt lehető,
 Elvégzi Jézus, mindent tehet ő.
 Ó, mint szerette Isten e világot!
 Számára nevelt egy szép szál virágot.
 Ma fakadt fel az élet kútfeje,
 És megvirágzott Áron vesszeje.
/3
#96D61A32
 Ezáltal a kegyelmi frigy felépült,
 Hogy halál árnyékában amely nép ült,
 Láthasson szép nagy világosságot,
 A bűnös is nyerhessen váltságot.
 Eljött, hogy a békességet hirdesse,
 Hogy az elveszett juhot megkeresse,
 Hogy az ördögnek dolgát elrontsa,
 Hogy értem drága vérét kiontsa.
/4
#57888114
 Ó, Isten, hozzám kötéd így magadat,
 Hogy értem küldéd világra Fiadat,
 Sok gonoszságom nem tekintetted,
 Veszendő sorsom szívedre vetted.
 Nem gondolál szent gyönyörűségeddel,
 Csak azzal, hogy jót tégy ellenségeddel;
 Csuda, hogy annak Istene lettél,
 Kinek teljességgel nem kellettél.
/5
#F40CB23B
 Már megítélted szegény lelkem perét,
 Rám árasztottad szerelmed tengerét;
 Végét nem érem én e mélységnek,
 Angyalok is csak rebegnek ennek:
 Ó, nagy szeretet, melyhez hasonló nincs!
 Ha lett volna még Istennél nagyobb kincs,
 Nem tartózkodott volna iránta;
 Ez volt a legtöbb; nékem ezt szánta.
/6
#EB8D493C
 E kedves vendéget már mint fogadjam?
 Dávid fiának immár mimet adjam?
 Ha nincs a vendégházakban helyed,
 Ímhol van szívem, itt hajtsd le fejed.
 Hagyd ott a barmot, jászolt és istállót,
 Hadd nyerjelek meg, mennyből hozzám szállót!
 Itt szállj, galambom, karjaim készek,
 E száraz fán vár egy üres fészek.
/7
#0A8E1BB4
 Kereslek, Uram, engem te is keress;
 Szeretlek, tudod, ó, hát te is szeress!
 Tedd egy szívvé szívemet szíveddel,
 Ejts rabul engem hívó szemeddel.
 Vonj, hogy atyádhoz általad mehessek,
 Élj bennem, benned hogy én is élhessek!
 Ó, Uram, tőled hová mehetnék?
 Elveszném, tiéd ha nem lehetnék.

>197. Itt állok jászolod felett
/1
#49CDE576
 Itt állok jászolod felett,
 ó, Jézusom, szerelmem,
 Eljöttem, elhoztam neked,
 amit kezedből nyertem;
 Vedd elmém, lelkem és szívem,
 Hadd adjam néked mindenem,
 Hogy kedves légyek néked!
/2
#32F9D179
 Nem éltem még e föld színén:
 te értem megszülettél;
 Még rólad mit sem tudtam én:
 tulajdonoddá tettél;
 Még meg sem formált szent kezed,
 Már elválasztál engemet,
 Hogy társam légy e földön.
/3
#1D5A2B55
 Halálban, éjben vártam én:
 fölkelt a nap rám véled.
 Terólad ömlik rám a fény:
 a béke, boldog élet,
 A lélek ékességei;
 Belőlük hitnek mennyei
 Szép tisztasága árad.
/4
#9B247E79
 Csak nézlek boldog szívvel,
 és nem győzlek nézni téged,
 Szóm és erőm mind oly kevés,
 hogy elmondhassa néked:
 Bár felfoghatna tégedet
 Az emberszív és ismeret,
 Hogy megfejthesse titkod!
/5
#C970A39A
 Megváltóm, egy kérésemet
 nem vetheted meg nékem:
 Hogy szívem mélyén tégedet
 hordozhatlak, remélem,
 És bölcsőd, szállásod leszek;
 Jövel hát, tölts el engemet
 Magaddal: nagy örömmel!

>198. Jézus, születél idvességünkre
/1
#CFDE91CD
 Jézus, születél idvességünkre,
 Amint régenten vala ígérve,
 Atya Istennek nagy szerelme,
 Emberi nemhez tetszik ebbe
 Kegyessége.
/2
#E7DF572D
 Első szüleink vétkei miatt
 És az ördögi csalárdság miatt
 Hoztunk magunkra örök halált,
 És kárhozatot, pokol kínját,
 sok nyavalyát.
/3
#147301A8
 Nem volt senki sem,
 ki megmentene,
 Ámde szent Atyád megkönyörüle,
 Téged ígére, s idő telvén
 Szabadítóul nékünk
 mennyből Alákülde.
/4
#44EB6BD0
 Értünk felvevél a Szűztől testet,
 Melyben szenvedél s tettél eleget;
 Elvévéd a mi bűneinket,
 Nyervén minékünk örök éltet,
 Dicsőséget.
/5
#8A3E8FCA
 Siess már hozzánk,
 megváltó Urunk!
 Adjad, hogy végig benned bízhassunk,
 És szent Atyáddal megláthassunk,
 Mennyben tevéled lakozhassunk,
 Vigadhassunk.
/6
#7F5E531D
 Légyen tisztesség te Felségednek
 És szent Atyádnak, mi teremtőnknek,
 És egyetemben Szentléleknek:
 Dicsőség a Szentháromságnak,
 Egy Istennek!

>199. Jöjjetek Krisztust dicsérni
/1
#0EDB29CC
 Jöjjetek Krisztust dicsérni,
 Bízó szívvel hozzá tér -ni,
 Énekekkel zengve kérni,
 Krisztus népe, jöjjetek.
/2
#B4E510E8
 Bűn, pokol már búban éljen,
 Ördögöt hadd ölje szégyen,
 Üdvösségünk szent ölében
 Már levetjük mind a bút.
/3
#4A879332
 Küldte Őt az Úr kegyelme
 Öröklétre, győzelemre,
 Hogy szívünket felemelje
 Boldogságos ég felé.
/4
#5E099DE9
 Irgalommal szánva minket,
 Nagy jósága ránk tekintett,
 S ördögcsalta bús szívünket
 Mennymagasból látni jött.
/5
#E9E3F162
 Áldott óra, boldog óra,
 Nagy hitünknek meghozója,
 Ajkunk zengő hálaszóra
 Nyílik, édes Jézusunk.
/6
#A14DA8C1
 Jászol-ölben drága Gyermek,
 Ég felé vigyen kegyelmed,
 Hol dicsérve énekelnek
 Édes hangú angyalok.

>200. Karácsony ünnepében, Karácsony ünnepében
/1
#F44F2BEC
 Karácsony ünnepében,
 Karácsony ünnepében
 Örvendezünk szívünkben:
 Mert Isten ő szent Fiát,
 Mert Isten ő szent Fiát
 Adta meg nékünk testben,
 Ki az ő népét Megszabadítá a bűnöktől,
 És a régi kígyótól Megmenté,
 Nékünk a dicsőséget Megnyeré:
 Légyen dicsőség Királyunknak
 Most és mindörökké!

>201a. Krisztus Urunknak áldott születésén
/1
#3C0A968A
 Krisztus Urunknak áldott születésén,
 Angyali verset mondjunk szent ünnepén,
 Mely Betlehemnek mezejében régen
 Zengett ekképpen:
/2
#FF06B366
 A magasságban dicsőség Istennek,
 Békesség légyen földön embereknek,
 És jóakarat mindenféle népnek
 És nemzetségnek!
/3
#92301E62
 A nemes Betlehemnek városába’
 Gyermek született szűztől e világra, ű
 Örömet hozott Ádám árváira,
 Maradékira.
/4
#124E7FC5
 Eljött már, akit a szent atyák vártak,
 A szent királyok akit óhajtottak,
 Kiről jövendőt próféták mondottak,
 Nyilván szólottak.
/5
#0DDBAC92
 Ez az Úr Jézus, igaz Messiásunk,
 Általa vagyon bűnünkből váltságunk,
 A mennyországban örökös lakásunk,
 Boldogulásunk.
/6
#182EA0D4
 Hála legyen mennybéli szent Atyánknak,
 Hála legyen született Jézusunknak,
 És Szentléleknek, mi vigasztalónknak,
 Bölcs oktatónknak!
/7
#5C6B9AE2
 Ó, örök Isten, dicső Szentháromság,
 Szálljon mireánk mennyei vigasság,
 Távozzék tőlünk minden szomorúság,
 Légyen vidámság!

>201b. Krisztus Urunknak áldott születésén
/1
#247103DC
 Krisztus Urunknak áldott születésén,
 Angyali verset mondjunk szent ünnepén,
 Mely Betlehemnek mezejében régen
 Zengett ekképpen:

>202. Örvendjetek, keresztyének
/1
#196E8AA2
 Örvendjetek, keresztyének,
 Nyíljatok meg nyelvek és szívek,
 Az idvesség Is -tenének
 Mondjatok áldást, minden hívek!
 Felváltatott nagy örömmel
 A haláltól való félelem,
 Mit véghetetlen érdemmel
 Meggyőz az isteni kegyelem.
/2
#043CABDE
 A Megtartó ma született,
 Az örök Isten emberré lett,
 És ma visszaszereztetett
 Az igazság s elvesztett élet.
 Kiküldé szerelmes
 Fiát Istenünk a teljes időben,
 Hogy a kezes a bűn díját
 Fizesse s szenvedje testében.
/3
#C3870498
 Ó, imádandó titkai
 Ama békesség tanácsának,
 Ó, nagyhatalmú dolgai
 A menny és föld szabad Urának!
 Aki által teremtetett
 S lett minden serege az égnek,
 Az Ige testté született,
 Személye az egy Istenségnek.
/4
#26A98366
 Ennek örülnek az egek,
 A mély titkon elcsudálkoznak,
 Hirdetik angyalseregek Jézust,
 s előtte leborulnak.
 A Magasságost tisztelik,
 A földön békességet zengnek,
 És ünnepnappá szentelik
 megjelenését az Istennek.
/5
#99819C2C
 Áldom én is szent nevedet,
 Én királyom, szenteknek szente,
 És vígan ülöm ünneped,
 Ó, Jézus, Istennek felkentje!
 Magasztalom Felségedet
 És imádlak szent félelemmel,
 De egyszersmind szerelmedet
 Megölelem igaz hitemmel.

>203. Ó, jöjjetek, hívek, ma lelki nagy örömmel
/1
#631A3FDF
 Ó, jöjjetek, hívek, ma lelki nagy örömmel,
 A jászolhoz Betlehembe jöjjetek el!
 Megszületett az angyalok királya:
 Ó, jöjjetek, imádjuk, Ó, jöjjetek, imádjuk,
 Ó, jöjjetek, imádjuk az Úr Krisztust!
/2
#57CB9CD1
 Az életnek szent Ura, dicsőség Királya
 Itt fekszik a jászol mélyén nagy szegényen.
 Nagy dicsőséges, szent és örök Isten!
 Ó, jöjjetek, imádjuk, Ó, jöjjetek, imádjuk,
 Ó, jöjjetek, imádjuk az Úr Krisztust.
/3
#3A97BB0B
 Ti angyali lelkek, ma zengjetek az Úrnak
 És vigadva örvendjetek, buzgó hívek!
 A magas mennyben dicsőség Istennek!
 Ó, jöjjetek, imádjuk, Ó, jöjjetek, imádjuk,
 Ó, jöjjetek, imádjuk az Úr Krisztust.
/4
#D6A17511
 Úr Jézus, ki ez napon érettünk születtél,
 Csak tégedet illet szívünk tisztelete!
 Isteni Gyermek, testet öltött Ige!
 Ó, jöjjetek, imádjuk, Ó, jöjjetek, imádjuk,
 Ó, jöjjetek, imádjuk az Úr Krisztust!

>204. Szívünk vígsággal ma bétölt
/1
#76167DA2
 Szívünk vígsággal ma bétölt,
 Mert ígéret szerint felkölt
 Istenfélők számára
 Az igazság fényes napja:
 Újtestámentomnak papja
 Eljött, kit sok szent vára.
 Csillag villog, Ra-gyog már rám, melyet
 Bálám láta régen,
 Fénylik, mint szép nap az égen.
/2
#17D65C63
 Megszáná az Úr estünket,
 Felöltözé portestünket,
 Szent kegyelme mily drága!
 Győzelmei már megszűnnek
 A kárhozatszerző bűnnek,
 Törvénynek nem sújt átka;
 Mérge, férge a pokolnak megromolnak ma végképpen,
 Nem árthatnak semmiképpen.
/3
#ACA36A5D
 Próféta, kinek nincs mása,
 Jött hozzánk, hogy népét lássa,
 Bölcsességre oktassa;
 A főpap jött, hogy áldozzon,
 Tiszta szívet, lelket adjon,
 Világ bűnét elmossa;
 Eljött s meglett a szenteknek,
 felkenteknek fejedelme,
 Királya, fősegedelme.
/4
#AFF5DE8C
 Szelídség volt minden dolga,
 Önként leve értünk szolga,
 Megalázta önmagát;
 Földi fényt, hírt nem kergete,
 Szerény munkás volt élete
 S vérén szerzé birtokát;
 De épp ez szép bizonysága,
 hogy országa lelki, belső,
 Királysága égi, első.
/5
#66A12BA1
 Jézus, engedj hozzád mennem,
 Éljek benned, te énbennem,
 Ha te hozzánk eljöttél;
 Adjad, legyek igaz híved,
 Ó, essék meg rajtam szíved,
 Ha kegyedbe bevettél;
 Mert nincs más kincs, mely hívekkel,
 bús szívekkel jót tehetne,
 Boldogságot szerezhetne.

>Óév

>205. Ismét egyik esztendeje
/1
#ED9EF3B2
 Ismét egyik esztendeje,
 Istentől kimért ideje
 Telék el a mulandóságnak:
 Az égi testek órája
 Lefolyván, lett új példája
 A közös változandóságnak;
 Elmúlt vége, mint kezdete,
 Már van csak emlékezete.
/2
#F8DB549D
 Boldog, ki csendes lélekkel,
 És nem könnyező szemekkel
 Tekinthet vissza folyására,
 Ki, ha magát megkérdezi,
 Belső örömmel érezi,
 Hogy szolgált ez jobbulására,
 Hogy ment mind az ismeretben
 Előbb, mind a szeretetben.
/3
#1EA26903
 Boldog, kinek nem kell szánni
 Elvesztett idejét s bánni
 Megbecsülhetetlen óráit,
 Kinek hív emlékezeti
 Elébe nem helyezteti
 Helyrehozhatatlan hibáit,
 Ki az időt megbecsülte,
 A hivalkodást kerülte.
/4
#D4BF0B32
 De vajon melyik halandó
 Volna annyira állandó
 A jóban, aki meg ne esne?
 Kinek volna olyan szíve,
 Hogy a kísértésnek íve
 Rajta oly nyílást ne keresne,
 Melyen lopva hozzá férjen
 És az elevenig érjen?
/5
#400E20DD
 Tökéletességnek Atyja,
 Az öröm szívemet hatja,
 Valahányszor azt elgondolom,
 Mely sok jókat vettem tőled!
 De elrejtőzném előled,
 Ha meg másrészről megfontolom:
 Azokkal mily rosszul éltem,
 Elvesztegetni nem féltem.
/6
#968088E5
 Minthogy azért te teheted,
 Véghez egyedül viheted,
 Hogy a jót akarjam és tégyem:
 Engedd, hogy a múlt esztendőt,
 Mint már vissza nem jövendőt,
 Magamnak tükörül felvégyem;
 A jót benne követhessem,
 A rosszat elkerülhessem.

>Újév

>206. Az Úrnak jó volta
/1
#B900B98A
 Az Úrnak jó volta napjainkhoz napokat told,
 Melyeket bizonyos részekre oszt a nap és hold;
 Nyomain e két vezetőnek
 Tél, tavasz, nyár és ősz eljőnek,
 Engedvén a bölcs Teremtőnek.
/2
#0E1598E9
 Jó Atyánk, az elmúlt esztendő minden szakasza
 Jóságod tanúja, szívünknek nincs rád panasza;
 Bizony, az Úr minket kedvellett,
 mert lelki adományi mellett megadta,
 ami nékünk kellett.
/3
#D51B8BCA
 Úr Isten, ki minket
 sok áldásiddal töltél be,
 Ez új esztendővel jó kedved
 ne szakaszd félbe!
 Áldd meg kezdetét s végét ennek,
 Hogy midőn napjai lemennek,
 Mondhassuk: dicsőség Istennek!
/4
#56B6EDAE
 Vigyázz híveidre s hab közt
 hánykódó hajódra,
 Vigyázz országunkra s
 minden elöljáróinkra,
 Akik népedet úgy vezessék,
 Hogy igazság és jó békesség
 Egymást csókolva ölelgessék.
/5
#6B703A05
 E gyülekezeten, mely
 e helyre telepedett,
 Könyörülj, Úr Isten,
 Bővítsd rajta kegyelmedet,
 Áldd meg nagyjait, kicsinyjeit,
 Mind köz, mind tanácsos rendjeit;
 Töröld el a sírók könnyeit!
/6
#35874659
 Adj vidám órákat,
 ha nekünk azt jónak látod,
 Békességes tűrést,
 ha vessződet ránk bocsátod;
 Ha több esztendőt nem számlálunk,
 És ma vagy holnap el kell válnunk:
 Add, hogy légyen boldog halálunk!

>207. Ez esztendőt áldással
/1
#C4616787
 Ez esztendőt áldással,
 Ez esztendőt áldással,
 Koronázd meg, Úr Isten,
 Hogy víg hálaadással,
 Hogy víg hálaadással
 Dicsérje neved minden.
 Tetőled jőnek Kedves
 napjai az időnek;
 Nyújts hát, kérünk,
 bőséget Ke-zeddel,
 Tőlünk a békességet
 Ne vedd el; De midőn
 megtartod testünket,
 Ne hagyd el lelkünket!

>208. Ez esztendőt megáldjad
/1
#997FBE60
 Ez esztendőt megáldjad,
 Ez esztendőt megáldjad
 Kegyelmedből, Úr Isten!
 Bőséggel ékesítsed,
 Bőséggel ékesítsed,
 Te szent Jehova Isten!
 Tégedet áldnak,
 Kik lakoznak e földnek színén;
 És a puszta helyek is Bőséggel,
 Halmok áhítozódnak Víg kedvvel;
 Boldog, kit magadnak választál,
 Sionnak királya!

>209. Örök Isten, kinek esztendők
/1
#8E04F4DF
 Örök Isten, kinek esztendők
 Nincsenek létedben,
 Jelen vannak múltak, jövendők
 Egy tekintésedben;
 Minket pedig, mihelyt születünk,
 Már a koporsó vár,
 A bűn miatt lefoly életünk
 a halál velünk jár.
/2
#8B233AB0
 Ó, mely sokan elaluvának
 A múlt esztendőbe’!
 Kik nálunknál jobbak valának,
 Szálltak temetőbe,
 Minket pedig, kedvező Atyánk,
 Eddig takargattál,
 S ez új esztendőt derítvén ránk,
 Erre eljuttattál.
/3
#95FE975B
 Uram, a te lelked ereje
 Vélünk ily jót tégyen,
 Hogy jókedvednek esztendeje
 Ez az új is légyen;
 Atyai karoddal forgassad
 A mi dolgainkat,
 Dicsőségedre igazgassad
 Minden szándékinkat.
/4
#D8BA94BB
 Adj minékünk megújult szívet
 És új indulatot,
 Tehozzád mindenekben hívet
 És szent akaratot.
 Újítsd meg rajtunk a te képed,
 Mely áll szent életben,
 Hogy lehessünk választott néped,
 Élvén szeretetben.
/5
#1CB9852E
 Ez esztendőt testi jókkal is,
 Uram, koronázzad,
 A szükséges mulandókkal is
 Házunk felruházzad;
 Földünket bő gyümölcsözéssel
 Munkálkodásunkra
 Áldd meg, hogy szükséges terméssel
 Szolgáljon javunkra.
/6
#DF26417E
 Kegyelmed s áldásod újítsad
 Te szent egyházadon:
 Közelebbről azt szaporítsad
 Itt lévő nyájadon.
 Gátold hazánkban áradását
 A sok gonoszságnak,
 Fordítsd el eshető romlását
 Nemzetnek, országnak.
/7
#7147508E
 Tekintsd meg a szűkölködőket,
 Légy az árvák atyja,
 Vigasztalja a kesergőket
 Szád édes szózatja,
 Akik betegágyukban nyögnek,
 Vidámítsd lelküket;
 Ínségükben kik könyörögnek,
 Add meg kérésüket,
/8
#51AD2D71
 Végre midőn mind megavúlunk
 Az esztendeinkkel,
 Egymás után mi is kimúlunk
 Emlékezetinkkel:
 Uram, sorsunkon könyörülvén,
 Vigy a dicsőségbe,
 Minden bűnünket eltörölvén,
 Újíts meg az égbe’!

>210. Nékünk születék menynyei király
/1
#3D31DD57
 Nékünk születék menynyei király,
 Idvezítőnek kit mond az angyal:
 Új esztendőben mi vigadjunk,
 Született Jézust mi imádjuk!
/2
#5BEE1802
 Régen megírák ezt a próféták,
 Hogy fiat szül majd
 egy nemes virág:
 Új esztendőben
 mi vigadjunk,
 Született Jézust mi imádjuk!
/3
#5A513D11
 Szűz Máriától gyermek születék,
 De Szentlélektől ő fogantaték:
 Új esztendőben mi vigadjunk,
 Született Jézust mi imádjuk!
/4
#B24142C6
 Nagy hatalmas lőn e kisded gyermek,
 Megtöretének sok ellenségek:
 Új esztendőben mi vigadjunk,
 Született Jézust mi imádjuk!
/5
#4103F926
 Győzedelmet vőn a kárhozaton,
 Győzedelmet vőn örök halálon:
 Új esztendőben mi vigadjunk,
 Született Jézust mi imádjuk!
/6
#FE3FB57B
 Nincsen a bűnnek hatalma rajtunk,
 Kiknek e gyermek lészen oltalmunk:
 Új esztendőben mi vigadjunk,
 Született Jézust mi imádjuk!
/7
#2B546B3A
 Adjunk nagy hálát az Úr Istennek
 És idvezítő Jézus Krisztusnak:
 Új esztendőben mi vigadjunk,
 Született Jézust mi imádjuk!
/8
#2C77CDC2
 Dicsőíttessék a Szentháromság,
 Adja minékünk szent ajándékát:
 Új esztendőben mi vigadjunk,
 Született Jézust mi imádjuk!

>Virágvasárnap

>211a. Örvendezzen már e világ
/1
#661AD998
 Örvendezzen már e világ,
 Légyen mindenben vigasság,
 Mert Krisztus mindenért váltság,
 Világ bűnéért orvosság.
/2
#8C7FB493
 Zakariás szent próféta
 Már ezt régen megmondotta:
 Örvendezz, Sion leánya,
 És örülj nagy vigasságba’!
/3
#33252D71
 Ne félj, mert íme örömed:
 A te királyod jő neked;
 Szamár vemhén telepedett,
 Mutatván nagy szelídséget.
/4
#80744E98
 Véle vagyon istensége,
 mondhatatlan nagy kegyelme,
 Jóvolta, idvezítése,
 Mert ő mindeneknek feje.
/5
#AE3CF02E
 Elöl s utol nagy serege
 Szentlélek teljességében
 Hozsánnát kiált a mennybe,
 Dávid fiának örömben.
/6
#1D7B04DD
 Némelyek ruházatjokkal,
 Krisztus útját méltósággal,
 Mások hintik zöld ágakkal:
 Királyt tisztelnek azokkal.
/7
#8260D189
 Mi is azért királyunknak,
 Menjünk elébe Urunknak;
 Vigyünk szép pálmaágakat:
 Hitünk győzedelmes voltát.
/8
#ED41EFBB
 Dicsőség Atyánknak mennyben,
 Mi királyunknak akképpen
 Szentlélekkel egyetemben
 Mostan és minden időben.

>211b. Örvendezzen már e világ
/1
#D2C86F85
 Örvendezzen már e világ,
 Légyen mindenben vigasság!
 Mert Krisztus mindenért váltság,
 Vi-lág bűnéért orvosság.
/2
#CE3C8CD3
 Ó! mily nagy alázatosság,
 Istentől nagy irgalmasság,
 Hogy emberek közé leszállt,
 Kié mind e széles világ.
/3
#7E36FB43
 Jeruzsálem városába
 Minden népek sokasága
 Virágokat szór útjára,
 S ajka nyílik hozsánnára,

>211c. Örvendezzen már e világ
/1
#2A5F358C
 Örvendezzen már e világ,
 Légyen mindenben vigasság,
 Mert Krisztus mindenért váltság,
 Világ bűnéért orvosság.
/2
#3FADA403
 Csodatevő erejével,
 Könyörülő szerelmével
 Felméne Béthániába,
 Hol Lázárt feltámasztotta.
/3
#43BCDDA6
 Hol Mária, a szent asszony,
 Buzgóságra indult azon.
 Megadta neki hűségét:
 Drága kenet tisztességét.
/4
#0A86A5B8
 Onnan indult Jézus útra
 Jeruzsálem városába.
 Szegényen és szelíden jár;
 Mégis olyan, mint egy király.
/5
#8B5065C2
 Nagy sokaság körülötte
 Hozsánnát kiált a mennybe.
 Dávid fiának kiáltják,
 Hűségüket bemutatják.
/6
#373E5283
 Útjára ifjak és vének
 Zöld ágakat szeldelének.
 Leteríték ruhájukat,
 Tisztelik, mint királyokat.
/7
#E6B7E9EC
 Mi is azért királyunknak,
 Menjünk elébe Urunknak,
 Vigyünk pálmaágak helyett
 Alázatos, tiszta lelket.

>212. A nagy király jön: Hozsánna!
/1
#AD226C77
 A nagy király jön:
 Hozsánna! Hozsánna!
 Zeng e kiáltás előtte, utána;
 Zöld ágakat szeldelnek útára,
 Békességet hoz népe javára.
 Áldott, aki jött az Úrnak nevébe'
 Általa léptünk az Isten kedvébe;
 Békesség ott fenn a mennyországban,
 Áldott az Isten a magasságban!
/2
#33F9BF18
 Ó, szentegyház,
 hívek boldog országa!
 Mily édes ez a Jézus királysága!
 Szelíd, szegény ez és alázatos,
 De nagy hatalmú és csodálatos.
 Igaz ez és a bűnből szabadító,
 A bűnt, halált és népeket hódító;
 Vasvesszővel bírja ellenségét,
 De szelíden őrzi örökségét.
/3
#569FA679
 Jézus király és magát annak vallja,
 De hogy királlyá tegyék, nem javalja;
 Sőt, noha Isten, Sion királya:
 Lett az időben szolgák szolgája.
 Jön szamárháton alázatossággal,
 Aki pedig bír az egész világgal;
 Nem jön királyi fényes bíborban,
 Nem fegyverekkel zörgő táborban.
/4
#D58F5045
 Ó, édes Jézus, Atyádnak szent Fia!
 Ó, Isten, néped kegyelmes királya!
 Vezéreld jóra egész éltünket,
 Tégy tulajdon népeddé bennünket.
 Légy segítségül, ki a magasságban
 Ülsz drága véreden szerzett országban;
 Tégy engedelmes, hű polgárokká
 S nyert kincseidben birtokosokká.
/5
#CDEAEC58
 Hogy csak a Jézus és az ő szent
 Atyja Törvénye légyen és szent akaratja
 Cselekedetink zsinórmértéke:
 Áldott királyunk királyi széke
 Hű szíveinkben légyen felemelve,
 És hűségére életünk szentelve;
 Tégyen méltóvá a Jézus vére
 A boldog lelkek lakóhelyére.

>Nagypéntek

>213. Áll a Krisztus szent keresztje
/1
#464F3712
 Áll a Krisztus szent keresztje
 Elmúlás és rom felett,
 Krisztusban beteljesedve
 Látom üdvösségemet.
/2
#23EDA1C0
 Bánt a sok gond, űz a bánat,
 Tört remény vagy félelem:
 Ő nem hágy el, biztatást ad:
 Békesség van énvelem.
/3
#2EA14219
 Boldogságnak napja süt rám;
 Jóság, fény jár utamon:
 A keresztfa ragyogásán
 Fényesebb lesz szép napom.
/4
#94F20544
 Áldássá lesz ott az átok,
 Megbékéltet a kereszt;
 El nem múló boldogságod,
 Békességed ott keresd!
/5
#A72A051E
 Áll a Krisztus szent keresztje
 Elmúlás és rom felett,
 Krisztusban beteljesedve
 Látom üdvösségemet.

>214. Győzhetetlen én kőszálom
/1
#251D06C7
 Győzhetetlen én kőszálom,
 Védelmezőm és kővárom,
 A keresztfán drága áron
 Oltalmamat tőled várom.
/2
#EA0DA47A
 Sebeidnek nagy voltáért,
 Engedj kedves áldozatért,
 Drága szép piros véredért,
 Kit kiöntél ez világért.
/3
#5A5E1DFD
 Reád bíztam én ügyemet,
 Én Jézusom, én lelkemet,
 Megepedett bús szívemet,
 Szegény árva bús fejemet.
/4
#068F3557
 Irgalmazz meg én lelkemnek,
 Ki vagy ura mennynek, földnek,
 Könyörgök csak Felségednek,
 Én megváltó Istenemnek.
/5
#95B05ED5
 Mutass, Jézus, kies földet,
 Lakásomul adj jó helyet,
 Ez életben csendességet,
 Jövendőben idvességet.

>215. Jézus, világ megváltója
/1
#5FC8CD9D
 Jézus, világ megváltója,
 Üdvösségem megadója,
 Megfeszített Isten Fia,
 Bűnömnek fán függő díja:
 Jézus, engedd hozzád térnem,
 Veled halnom, veled élnem.
/2
#2C470542
 Szent kereszteden kereslek,
 Szomorú szívvel szemléllek,
 Mert így gyógyulást reménylek:
 Moss meg szent véredben s élek.
 Jézus, engedd hozzád térnem,
 Veled halnom, veled élnem.
/3
#1674B25F
 E keresztről, én Reményem,
 Tekints reám szerelmesen,
 Épen téríts hozzád engem,
 Mondván: ”bűnöd elengedtem”.
 Jézus, engedd hozzád térnem,
 Veled halnom, veled élnem.
/4
#DCA8C72C
 Ne gerjedezz vétkem ellen,
 Sőt véreddel kegyelmesen,
 Ily mocskos beteget, híven
 Moss meg, s bűntől üres lészen.
 Jézus, engedd hozzád térnem,
 Veled halnom, veled élnem.
/5
#173C3DA3
 Engem ily nagy szerelmedből,
 Végy hozzád szent kegyelmedből,
 Vegyek erőt keresztedből,
 És végbúcsút bűneimtől.
 Jézus, engedd hozzád térnem,
 Veled halnom, veled élnem.
/6
#0F143ADE
 Ajánlom magamat néked:
 Sebeidben szívemet vedd;
 Ó, nyíljál fel, piros forrás,
 mert nagy bennem rád a vágyás!
 Jézus, engedd hozzád térnem,
 Veled halnom, veled élnem.
/7
#00C36596
 Ez keserves halálodért,
 Melyet felvettél éltemért,
 Vedd szívemet mindezekért,
 Megérdemlett jutalmadért.
 Jézus, engedd hozzád térnem,
 Veled halnom, veled élnem.
/8
#C7766B4E
 Nyílj fel, édes szív rózsája,
 Jó illatú violája,
 Hogy lehessen maradása
 Szívemnek, s benned lakása.
 Jézus, engedd hozzád térnem,
 Veled halnom, veled élnem.
/9
#55ADFF50
 Engem, bűnöst, kérlek, ne hagyj,
 Halál rabját kínra ne adj;
 Sőt, ha eljő majd halálom:
 Szent jobbodra engedj állnom.
 Jézus, engedd hozzád térnem,
 Veled halnom, veled élnem.

>216. Jézus, Istennek Báránya
/1
#57246BB2
 Jézus, Istennek Báránya,
 Kínjaidat ég s föld szánja.
 A nap, a nap sötétté változik,
 A föld, a föld reng és ingadozik.
/2
#F141D451
 Hegyek, halmok süllyedeznek,
 A kősziklák repedeznek,
 Holtak, holtak Sírból feltámadnak,
 A szent, a szent Városban jelt adnak.
/3
#9130D15E
 Íme, a templom kárpitja
 Kettéhasad és megnyitja
 Helyét, helyét A szentek szentének,
 Jelét, jelét Jehova frigyének.
/4
#4683377C
 Mindezekből, ó, mit értsünk?
 És szívünkbe vajh’ mit véssünk?
 Isten, Isten Végtelen kegyelmét,
 Hozzánk, hozzánk
 Csuda nagy szerelmét.
/5
#4F6BEE8F
 Fiát a szeretet Atyja
 Kereszt kínjaira adja,
 Értünk, értünk
 Hogy eleget tégyen,
 Urunk, urunk
 S üdvözítőnk légyen.
/6
#D3A23499
 Köztünk van a Szentek Szentje:
 A híveknek ezt jelentse
 Az ép, az ép Kárpit hasadása,
 Földig, földig Kettéhasadása.
/7
#1D9A91D2
 Jézus, ki értünk szenvedtél,
 Hogy éljünk, halálra mentél:
 Néked, néked
 Szívből hálát adunk,
 Holtig, holtig
 Híveid maradunk.

>217. Királyi zászlók lobognak
/1
#548004D5
 Királyi zászlók lobognak,
 Fénylik titka keresztfának;
 Az élet ottan halni tért,
 Holtával nyervén drága bért.
/2
#CCDD096D
 Átverve szeggel tagjai,
 És kinyújtva szent karjai;
 Idvességünknek ára lett,
 Bűnünkért ő tett eleget.
/3
#D4D79409
 Dávidnak bétölt írása,
 Minden népeknek mondása:
 Isten, kit felfeszítetek,
 Úr lesz e fán felettetek.
/4
#DE206EC6
 Légy áldott, oltár s áldozat,
 Mellyel Krisztusunk áldozott:
 Holton az élet elesett,
 S holtával adott életet.
/5
#17D91340
 Élő kútfő, Szentháromság!
 Téged áldunk, legfőbb Jóság!
 Kérünk: a kereszt győzelmét,
 Közöljed vélünk érdemét!

>218. Krisztus, ártatlan Bárány
/1
#7A17B73C
 Krisztus, ártatlan Bárány,
 Ki miértünk megholtál,
 A keresztfa oltárán
 Nagy engedelmes voltál.
 Hordozván bűneinket,
 Te váltottál meg minket:
 Irgalmazz nékünk, ó, Jézus!

>219. Lelkem, nézz a Golgotára
/1
#C4372BD1
 Lelkem, nézz a Golgotára,
 Jézusodnak nézd fejét!
 Tövissel van koronázva,
 Mégis áldva néz feléd;
 Szenved Isten bajnoka,
 És bár vérzik homloka,
 A világot nem átkozza;
 Meghal érte imádkozva.

>220. Ó, Isten, ki a törődött
/1
#7BC29D73
 Ó, Isten, ki a törődött
 Szívet meg nem utálod,
 Sőt a bánatból ejtődött
 Könynyeket megszámlálod:
 Kedveljed érzésimet
 És elmélkedésimet,
 Melyek szívemben támadnak
 Halálán te szent Fiadnak.
/2
#02C79B53
 Mint sír a kertben magában,
 Mint küszködik a harccal,
 Mely a sötét éjszakában
 A földre ejti arccal,
 Hol kínjait lelkében
 Érezvén és testében
 Vércseppel verejtékezik,
 Elalél és csüggedezik.
/3
#481263AD
 De új erőt vesz magának,
 Lelkét megbátorítja,
 Mint a vitéz, ha harcának
 Kezdete tántorítja.
 Az eláruló csókra
 Önként megy a kínokra
 A fegyverekbe öltözött
 Vérengző kísérők között.
/4
#6778B394
 A bíró bár elismeri,
 Hogy ő vétkét nem látja,
 Elereszteni nem meri,
 Ostor alá bocsátja.
 Látja a nép véresen,
 Nincs szíve, mely megessen,
 Fakad ily gyilkos lármára:
 Feszítsd fel a keresztfára!
/5
#D88CA19C
 Függ már a fán kiterjesztett
 Kezekkel felszegezve,
 A föld testéből eresztett
 Vérrel van béfedezve.
 Elalélván végtére
 A kínok érzésére,
 Meghal, Atyjához sóhajtva,
 Fejét keresztjére hajtva.
/6
#BC8D432D
 Ó, kínok közt elenyészett
 Megváltóm a keresztfán!
 Téged az egész természet
 Búslakodva sirat s szán.
 Reng a föld alkotmánya,
 A holtakat kihányja;
 A fényes nap is bújába’,
 Borul sötét éjszakába.
/7
#4D4A1701
 Elhal énbennem is a szív
 Ez iszonyú látásra,
 Bánatba merült lelkem hív
 Bűneimért sírásra,
 Mert tudom, hogy helyettem,
 Ki sokképpen vétettem,
 Szenvedéd a vereséget,
 Mely így vet éltednek véget.
/8
#6EB71C5A
 Mégis biztat engem a hit,
 Hogy értem ontottál vért,
 Rettegésemben megenyhít,
 Tudván, hogy érdemedért
 Részeltetem az égben
 Ama nagy dicsőségben,
 Hol te, ki értem szenvedtél,
 Örök méltóságot vettél.

>221. Ó, Krisztusfő, te zúzott
/1
#44FC6235
 Ó, Krisztusfő, te zúzott,
 Te véres szenvedő,
 Te töviskoszorúzott
 Kigúnyolt drága fő,
 Ki szépség tükre voltál,
 Ékes, csodás remek,
 De most megcsúfolódtál:
 Szent fő, köszöntelek!
/2
#7B646B68
 Ékességed, te drága,
 Melytől máskor remeg
 Világ hatalmassága,
 Köpés mocskolta meg.
 Milyen halványra váltál!
 Szemed fényét, amely
 Szebb volt minden sugárnál,
 Ki rútította el?
/3
#CC1A4200
 Mind, ami kín, ütés ért,
 Magam hoztam Reád;
 Uram, e szenvedésért
 Lelkemben ég a vád.
 Feddő szót érdemelve
 itt állok én, szegény,
 S kérlek, lelked kegyelme
 Sugározzék felém.
/4
#B7153DD1
 Itt állok - ó, ne vess meg -
 A gyötrelmek helyén;
 Amíg ki nem hűl tested,
 El nem mozdulok én.
 S ha életed kilobban,
 Alácsuklik fejed,
 Ölemben és karomban
 Lesz nyugtató helyed.
/5
#A0D22B90
 Ó, légy érette áldott,
 Jézus, Egyetlenem,
 Hogy szörnyű kínhalálod
 Nagy jót akar velem.
 Add, hogy  hódolva híven
 Tőled ne térjek el,
 S ha hűlni kezd a szívem,
 Benned pihenjek el.
/6
#FEFE6684
 Mellőlem el ne távozz,
 Ha majd én távozom,
 A kínban, mit halál hoz,
 Állj mellém, Jézusom.
 Ha lelkem félve reszket,
 S rettent a meghalás,
 Nagy kínod és kereszted
 Legyen vigasztalás.
/7
#B283F681
 Légy pajzsom és reményem,
 Ha kétség látogat,
 Véssem szívembe mélyen
 Kereszthalálodat.
 Rád nézzek, Rád szünetlen,
 S ha majd szívem megáll,
 Öleljen át a lelkem -
 Így halni: jó halál.

>222. Ó, ártatlanság báránya
/1
#64E5CD3B
 Ó, ártatlanság báránya,
 E világnak ki vagy ára,
 Megtartója, táplálója,
 Áldott légy, egek királya!
/2
#E2D0D921
 Hogy érdemlettük ezt tőled,
 Hogy miértünk ezt felvégyed?
 Hogy szent tested vereséget,
 Szenvedjen ennyi sérelmet?
/3
#2874CF1F
 Bár tiszta ártatlanság vagy,
 Káromlónak is mondanak,
 Ámde te mégis vesztegelsz,
 Ily hamis vádra nem felelsz.
/4
#488BA304
 Mivelhogy Isten fiának,
 Mondod magad Messiásnak:
 Káromlónak elneveznek,
 Nem hisznek téged Istennek.
/5
#DD3EBB24
 Áldott légy ezért, Jézusunk,
 Édes megváltó Krisztusunk,
 Fő tanítónk és orvosunk,
 Megszabadító Királyunk!

>223. Paradicsomnak te szép élő fája
/1
#D31FBB0A
 Paradicsomnak te szép élő fája,
 Ó, kegyes Jézus, Istennek Báránya,
 Te vagy lelkünknek igaz
 Megváltója, Szabadítója.
/2
#5C81553E
 Értünk egyedül szörnyű
 kínt szenvedtél,
 Megfeszíttetvén töviset viseltél,
 Mi bűneinkért véreddel fizettél
 És megölettél.
/3
#62D121AC
 Csudánkra vannak a
 te szép gyümölcsid,
 Nagy kínjaid közt való
 szent beszédid,
 Kiket keresztfán szóltál,
 szent mondásid,
 Hét szép szavaid.
/4
#7A0382DA
 Első szódban így
 könyörgél Istennek:
 ATYÁM, BOCSÁSD
 MEG BŰNÖKET EZEKNEK,
 MERT NEM TUDJÁK ŐK MOST,
 MIT CSELEKESZNEK,
 Kegyetlenkednek.
/5
#C25015E5
 Lőn második szód
 a szegény tolvajhoz,
 Bűnén kesergő s
 törődő latorhoz,
 Mondván: VELEM LéSSZ MA
 PARADICSOMBAN,
 SZENT ORSZÁGOMBAN.
/6
#ABBF2337
 Keserves szívű
 szentséges anyádnak,
 Harmadik szódat nyújtád
 Máriának, És ugyanakkor
 szóltál szent Jánosnak,
 Mondván azoknak:
/7
#282C239B
 ÍMHOL AZ ANYÁD,
 kedves tanítványom,
 ÍMHOL A FIAD, ASSZONY,
 már ajánlom, Gyámolítódul
 ezt a Jánost hagyom,
 Kegyesen adom.
/8
#23FA98ED
 Ártatlan Bárány, keserűségedben,
 Negyed szód ez lőn
 nagy kísértetedben:
 ÉN ISTENEM, ÉN ISTENEM,
 ÜGYEMBEN MIÉRT HAGYÁL EL?
/9
#478592D6
 Jövendölések mind
 beteljesedvén,
 És minden dolgok
 már elvégeztetvén,
 Mondád ötödször:
 SZOMJÚHOZOM igen,
 Szíved epedvén.
/10
#7FF6EDEE
 Mikoron, Uram, a mérget elvévéd,
 Ottan hatodszor mondál
 ilyen igét:
 Bételjesedett és
 ELVÉGEZTETETT
 Váltságnak dolga.
/11
#435F69F1
 Reád érkezvén a
 szomorú halál,
 Hetedszer ily szót
 s imádságot mondál:
 ATYÁM, KEZEDBE
 TÉSZEM LE LELKEMET,
 Én életemet.
/12
#EC840358
 Édes Jézusunk,
 szenteld meg lelkünket,
 Hogy mi is megbocsáthassuk
 bűnüket Mindeneknek,
 kik ellenünk vétettek
 És elestenek.
/13
#2ABE875B
 Adjad, hogy mi is
 értük könyörögjünk,
 Téged követvén,
 szívből esedezzünk,
 Hogy sok szentekkel
 tehozzád mehessünk,
 Idvezülhessünk.
/14
#3E58683D
 A pályafutást mi is
 elvégezvén,
 Lelkünket ajánlhassuk
 szent kezedben;
 Mint megváltottak,
 mondhassuk nagy szépen
 Éltünk végében:
/15
#69D80990
 Hála légyen a mennybéli
 Istennek,
 Ki megváltója lőn
 bűnös embernek,
 És megszerzője szent
 békességünknek:
 Idvességünknek.

>224. Te drága Jézus, mi történt tevéled
/1
#32EDDC07
 Te drága Jézus,
 mi történt tevéled,
 Hogy oly keményen
 sújt a zord ítélet?
 A szörnyű vétket
 el mivel követted?
 Mi volt a tetted?
/2
#E9C20A56
 Megostoroznak,
 tövissel csúfolnak,
 Arcodba vágnak,
 gúnyolódva szólnak,
 Epét ecettel kínálgatni
 mernek,
 Keresztre vernek.
/3
#86E8B909
 Mondd, ennyi kínnak
 mi az eredetje?
 Jaj, vétkeimmel
 vertelek keresztre!
 Amit Te szenvedsz,
 Jézus, én okoztam,
 Fejedre hoztam.
/4
#6EEF42EC
 S mily büntetés,
 mit a világ Reád mért?
 A jó nyájőrző szenved
 a juháért;
 A bűnért, melyet szolgák
 elkövettek,
 Az Úr fizet meg.
/5
#EADC29EA
 Meghal a jó, ki
 hűség volt s alázat,
 Az él, ki Isten
 bántására lázadt;
 A vétkes ember
 sértetlen, s bilincsben
 Ott áll az Isten.
/6
#D5951A32
 Ó, mérhetetlen
 szeretet, csodás hit,
 Amely a kínok
 zord útjára rávitt!
 Én vigadozva
 élek és örömben,
 Te kín-özönben.
/7
#FA17D02E
 Ó, nagy Királyom,
 minden kor Királya,
 Hűségedet hogy
 hirdethesse hála?
 Nincs emberszív,
 melyben tanács fakadhat:
 Néked mit adhat?
/8
#B7407DE6
 Ha trónusodnál, Jézusom,
 Vezérem, Fejem ragyogva
 fürdik majd a fényben:
 Énekelek, hol szentül
 zeng az ének,
 Dicsérve Téged.

>Nagyszombat

>225. Ó, Krisztus, láttam szenvedésed
/1
#0D88FD2E
 Ó, Krisztus, láttam szenvedésed,
 S borzongásom véget nem ért,
 Jaj, hogy halálkín lett a részed
 Érettem, árva bűnösért.
/2
#8850BF97
 A természet velünk zokogja
 Halálodért fájdalmait,
 Elbújt a nap, gyászukba rogyva
 Siratnak választottaid.
/3
#C794A6CF
 Szent áldozat, tedd, meg ne szűnjünk
 Kereszten látni tégedet!
 Szent véred mossa csak le bűnünk,
 Lelkünk előtt az tár eget.
/4
#71EACF3B
 Jóságod mély és mély a hála,
 Amellyel hozzád fordulunk;
 Bűnünkért mentél kínhalálba,
 Egyetlen Megváltó Urunk.

>226. Sirasd meg, sirasd meg bűneidet, ó, ember
/1
#809F01CA
 Sirasd meg, sirasd
 meg bűneidet, ó, ember,
 A Krisztus sírjához úgy
 járulj ma könynyeddel;
 Gyászruhát öltözve álljatok
 a Golgotán, Úgy nézzetek
 onnan sírjába a holt után.
/2
#DE03526B
 Jövünk, ó, Úr Jézus,
 gyászban jövünk sírodhoz:
 Roskadozva jövünk,
 mert lelkünk nagy terhet hoz.
 Leomlunk fáradtan elődbe
 a kereszthez: Ó, Jézus,
 Megváltónk,
 bűneinknek kegyelmezz!
/3
#4FA58E22
 Te, kinek sírodból
 támadt fel az új élet,
 Ma veled töltjük el
 ezt a napot és éjet.
 Imádkozva várjuk új
 életünk hajnalát:
 Úr Jézus, irgalmad e
 szent napra vigyen át!
/4
#4873FD91
 Édes Idvezítőnk,
 bízva bízunk tebenned,
 Hogy mi még általad
 hálát adunk Istennek.
 Könyörgésed értünk,
 ha előtte megjelen,
 Lesz a mi lelkünknek
 bocsánat és kegyelem.

>Húsvét

>227. Felvirradt áldott szép napunk
/1
#BBC0CDDF
 Felvirradt áldott szép napunk,
 Ma teljes szívvel vigadunk,
 Ma győz a Krisztus, és ha int,
 Rab lesz sok ellensé -ge mind.
 Halléluja!
/2
#6003ABC5
 Az ősi kígyót, bűnt, halált,
 Kínt, poklot, szenvedés jaját
 Legyőzte Jézus, Mesterünk,
 Ki most feltámadott nekünk.
 Halléluja!
/3
#8A53D00D
 Az élet győz, a mord halál
 A prédát visszaadta már,
 Nagy úrságának vége lett,
 Krisztus hozott új életet.
 Halléluja!
/4
#3BDA660E
 A nap s a föld s minden, mi él,
 Ma bút örömmel felcserél,
 Mert a világnak zsarnoka
 Nem kelhet többé fel soha.
 Halléluja!
/5
#BA085770
 Mi is éljünk vigadva hát,
 Daloljunk szép halléluját,
 Hadd zengje Krisztust énekünk,
 Ki sírból feltámadt nekünk!
 Halléluja!

>228. E húsvét ünnepében
/1
#7ED174AE
 E húsvét ünnepében,
 E húsvét ünnepében
 Dicsérjük Istent szívvel,
 Ki értünk megholt Fiát,
 Ki értünk megholt Fiát
 Feltámasztotta testben.
 Ennek örül föld,
 Tenger és a menny víg kedvében;
 Minden élő állatok
 Sok rendben,
 Mik égben, földön vannak,
 Sok részben,
 Fák, füvek és minden virágok
 Újulnak örömben.

>229. E húsvét ünnepében
/1
#F584150F
 E húsvét ünnepében,
 E húsvét ünnepében
 Örvendjünk, keresztyének!
 Szívünk teljességében,
 Szívünk teljességében
 Illik szánkba víg ének.
 A feltámadott Jézus
 nékünk zálogot adott,
 Hogy bár a föld gyomrába
 Tétetünk, Megújul valójába'
 Életünk; E hittel midőn ünnepelünk,
 Te légy, Jézus, velünk!

>230. Uram, közel voltam hozzád, mégis elszakadtam
/1
#E80F50B8
 Uram, közel voltam hozzád,
 mégis elszakadtam.
 Megvallatták hűségemet:
 neved megtagadtam.
 Halálodnak harmadnapján
 vádolnak a fények,
 Halálodnak harmadnapján
 sírva hajtok térdet.
/2
#0902FE9D
 Uram, téged kerestelek:
 céltalan futottam,
 Csak magamig, másokig,
 majd sírodig jutottam.
 Halálodnak harmadnapján
 megszűnik az átok,
 Halálodnak harmadnapján
 szívem megtalálod.
/3
#22BEDB27
 Uram, taníts hinni benned,
 várni új csodára;
 Tedd életem hűségessé,
 s legyen szavam hála!
 Halálodnak harmadnapján
 zendülnek a fények,
 Halálodnak harmadnapján
 én is áldlak téged.

>231. Feltámadt a mi életünk
/1
#1E6F2367
 Feltámadt a mi életünk,
 Vígan méltó énekelnünk,
 Úr Krisztust dicsérnünk,
 E szent napon is áldanunk,
 Angyalokkal őt imádnunk,
 Mint Urunkat, félnünk,
 Mert őt magasztalják nap,
 hold s égi seregek,
 A menynyei szentek.
/2
#76CCE628
 A földben minden gyökerek,
 Fáknak bimbói terjednek,
 Mezők megzöldülnek,
 Ég madarai zengenek,
 Fákon vígan énekelnek,
 Szárnyukon repdesnek;
 Minden illatozó fűvek
 gyönyörködtetnek,
 Dicséretre intnek.
/3
#06E37030
 Mert feltámadt ő igazán,
 Angyala jelenté nyilván
 Koporsónak jobbján;
 Tanítványival vigadván
 megjelent Galileában,
 Pétert vigasztalván:
 Örvendj te, ki voltál
 gyakran bűnért sírásban,
 Alázatosságban.
/4
#2E14C619
 Dicséret a nagy Istennek,
 Életet ki nyert népének,
 A bűnös embernek,
 Őt részeltetvén egeknek,
 Gyönyörűségében minden
 Lakóhelyeinek;
 Lelki javaival népét
 meglátogatja, megtérését várja.

>232. Győzelmet vettél, ó, Feltámadott!
/1
#24DD2C66
 Győzelmet vettél, ó, Feltámadott!
 Dicsőséggel fényes a diadalod!
 Magas égből szállt le angyali követ,
 Hogy elhengerítse a nehéz követ.
 Győzelmet vettél, ó, Feltámadott!
 Dicsőséggel fényes a diadalod!
/2
#C75A7A5E
 Lásd, ott az Jézus, az Üdvözítő!
 Ne kételkedj többé: ő jelent meg, ő!
 Ujjongj, Isten népe, Hirdesd szüntelen,
 Jézusé a végső, döntő győzelem!
 Győzelmet vettél, ó, Feltámadott!
 Dicsőséggel fényes a diadalod!
/3
#C039E6BD
 Mért félne szívem? Él az én Uram,
 Békesség Királya, Benne nyugta van,
 Ő a diadalmam, pajzsom, életem,
 Szívemben már nincsen semmi félelem.
 Győzelmet vettél, ó, Feltámadott!
 Dicsőséggel fényes a diadalod!

>233. Jézus, ki a sírban valál
/1
#48ABDB68
 Jézus, ki a sírban valál,
 Általad megholt a halál,
 Az élet pedig feltámadott,
 Mert szent tested meg nem rothadott.
 Él a Jézus, a mi fejünk,
 Keresztyének, énekeljünk,
 Ülvén húsvét ünnepeket,
 Új győzedelmi éneket.
/2
#9600A6C0
 Hol van, koporsó, hatalmad?
 Elveszett nyert diadalmad.
 Hová lett, ó, halál, a fúlánk,
 Melyet fensz már régóta reánk?
 Már nem rettegünk miatta,
 Mert Jézus meghódoltatta
 Ama félelmek királyát,
 megnyitván sírjának száját.
/3
#F6924E2D
 Nincs már szívem félelmére
 Nézni sírom fenekére,
 Mert látom Jézus példájából,
 Mi lehet a holtak porából.
 Szűnjetek meg, kétségeim,
 Változzatok, félelmeim,
 Reménységgé, örömökké,
 mert nem alszom el örökké.
/4
#43B53881
 Sőt hiszem, hogy e tört cserép
 Edény leend még egyszer ép,
 És tetemim megépíttetnek,
 Bár veséim megemésztetnek.
 Gyalázat elvettetésem,
 De pompás lesz kikelésem,
 Új eget látván ezekkel
 Az újra megnyílt szemekkel.
/5
#453B9E0A
 Jézus, segíts engem ebben,
 Hogy éltem folyjék szentebben,
 És hogy ne menjek ítéletre,
 Támassz fel engem új életre.
 A te lelkednek ereje
 Az új életnek kútfeje;
 Hogy hadd legyek élő személy,
 Lelked által énbennem élj!

>234. Jézus meghalt bűneinkért
/1
#67C0E36A
 Jézus meghalt bűneinkért,
 Harmadnap feltámadott;
 Mi megigazulásunkért
 Mindenről számot adott.
 Menynyei szent Atyjának,
 És ő igazsá -gának
 A váltságot megfizette
 A bűnösöknek helyette.
/2
#8B549D1A
 Életét maga letette
 Önként és jó kedvéből,
 Azt maga ismét felvette
 Isteni erejéből.
 Ó, csudáknak csudája!
 Íme, az Isten fia
 Értem életet áldozott,
 Élővé holtból változott.
/3
#1BDD4875
 A meghalt Jézus vére szólt
 Nékünk oly drága dolgot,
 Mert e vér Isten vére volt,
 Bűnt, halált ezzel oldott.
 Vér, melynek nincsen mása,
 Ó, Betlehem forrása!
 Ó, Jézus egy áldozatja,
 Isten gyönyörű illatja!
/4
#B80B13B3
 Hogy ez kedves volt Istennek,
 Azt azzal megmutatta,
 Hogy szent testét ő
 Szentjének Sírban
 soká nem hagyta.
 És a halál kötelét,
 A koporsónak tőrét
 Harmadnap széjjel oldozta,
 Fiát fogságból kihozta.
/5
#38F83FEB
 Jézus, én megholt életem,
 Jézus, feltámadásom,
 Benned bűntől mentté lettem,
 Benned igazulásom.
 Ó, adj hát segítséget,
 Lelki elevenséget
 Az első feltámadásra:
 Az új életben járásra.
/6
#3D400043
 Istenséged megmutatád
 Életedben, holtodban;
 Hatalmasan bizonyítád
 Sírból kiszállásodban.
 Ez isteni erővel
 Jövel, ó, Jézus, jövel!
 Adj új életet s meghalást,
 S majd második feltámadást!

>235. Krisztus feltámada Igazságunkra
/1
#FF973175
 Krisztus feltámada Igazságunkra,
 Utat szerze mennyországra,
 Örök boldogságra.
/2
#D904C88C
 Mind e világ terhét
 Vállára vette,
 A hatalmas Atya Istent
 Értünk megkövette.
/3
#54031216
 De lám, ezt nem érti
 A hálátlanság,
 Emberekre honnan szállott
 Ennyi nyomorúság.
/4
#BE0FD8F5
 Azért nem fogadják
 Isten beszédét,
 Jóra intő szent Igéjét:
 Krisztust, idvességét.
/5
#A75135D0
 ”Krisztus feltámada” -
 sokan kiáltjuk,
 De a bűnnek undokságát
 Mi meg nem utáljuk.
/6
#9307DF5F
 Tudva, bűnben élünk,
 Semmit nem félünk,
 Azért a Krisztus halála
 Nem használ minékünk.
/7
#F62E6741
 Támadjunk fel testben
 Azért a bűnből,
 Melyért mi kirekesztettünk
 A nagy dicsőségből.
/8
#F132637A
 Vegyük nagy jó kedvvel
 Krisztus jóvoltát,
 Atya Isten előtt
 való kedves áldozatját.
/9
#3D1E5A19
 Dicsőség mennyégben
 Az Úr Istennek, Atya,
 Fiú, Szentléleknek,
 Mindörökké, Ámen.

>236. Krisztus feltámadott, Kit halál el ragadott
/1
#0B2D9C0B
 Krisztus feltámadott,
 Kit halál el ragadott;
 Örvendezzünk, vigadjunk,
 Krisztus lett a vigaszunk,
 Alleluja! Ha ő fel nem támad,
 Nincs többé bűnbocsánat,
 De él, ezért szent nevét,
 Zengjük ő dicséretét,
 Alleluja, Alleluja! Alleluja!
 Örvendezzünk, vigadjunk,
 Krisztus lett a vigaszunk. Alleluja!

>237. Krisztus ma feltámada
/1
#45051451
 Krisztus ma feltámada,
 Mi bűnünket elmosá,
 Ő szent vére hullása
 Lőn vétkünknek romlása:
 Csak ő minékünk az
 Atya Istennél
 Kedves szószólónk!
/2
#A5B8A110
 Csodálatos harc vala,
 Hol az élet meghala,
 De nagyobb csoda vala,
 Hogy az Úr feltámada;
 Csak ő minékünk az
 Atya Istennél
 Kedves szószólónk!
/3
#354FA512
 Haljunk meg bűneinknek,
 Végét vessük vétkünknek,
 Éljünk már új életnek,
 Adjunk hálát Istennek;
 Csak ő minékünk az
 Atya Istennél
 Kedves szószólónk!
/4
#E285861E
 Méltó azért vigadnunk,
 Mi szívünkben örülnünk,
 Krisztus Jézust dicsérnünk,
 És őt felmagasztalnunk:
 Csak ő minékünk az
 Atya Istennél
 Kedves szószólónk!

>238. Örvendezzetek, egek
/1
#2FBB58AF
 Örvendezzetek, egek,
 Ti is, földi seregek!
 Mindnyájan örüljetek,
 Vígan énekeljetek,
 Mert Urunk feltámadott,
 Nékünk életet adott.
/2
#53B39BAE
 Jézus él, mi is élünk,
 A haláltól nem félünk,
 Mert legyőzte a halált,
 Örök váltságot talált
 Isteni erejével,
 Hathatós érdemével.
/3
#384D2A79
 Nékünk megigazulást
 És a bűnből gyógyulást,
 Istennel békességet
 És boldog reménységet
 Nyert feltámadásával,
 örök igazságával.
/4
#4D6B1E87
 Előtted arcra esünk,
 S kérünk, édes kezesünk:
 Részeltess halálodnak
 És feltámadásodnak
 Drága érdemeiben,
 Édes gyümölcseiben.
/5
#A396CBA0
 Cselekedd Szentlelkeddel,
 Végtelen érdemeddel,
 Hogy új életet éljünk,
 Végre porból felkeljünk
 Örök, nagy boldogságra
 És halhatatlanságra.

>Krisztus mennybemenetele

>239. A Krisztus mennybe felméne
/1
#52E522FA
 A Krisztus mennybe felméne,
 Hogy nékünk helyet szerzene,
 Atyjával megbékéltetne,
 Életre bévinne.
/2
#5FEE77E0
 Ó, mi kegyelmes Mesterünk,
 Emlékezzél meg mirólunk,
 Ki meghaltál volt érettünk,
 Légy jelen mivelünk!
/3
#E96C0751
 Te látod mennyből éltünket
 És nagy keserűséginket;
 Vigasztald meg mi lelkünket,
 Hogy higgyünk tégedet.
/4
#D4744668
 Mert megfogadtad minekünk,
 Hogy léssz örökké mivelünk;
 Adjad Szentlelked minekünk,
 Hogy benned hihessünk.
/5
#A74A409D
 És oltalmazz meg mindentől,
 Szent Atyádnak haragjától,
 Ördögtől és kárhozattól,
 Minden dühösségtől.
/6
#59D5EBFB
 Tekints nagy gyarlóságinkra,
 E világ csalárdságára;
 Vigy bé a nagy boldogságba,
 Te szent országodba.
/7
#08A788BB
 Egyetemben keresztyének,
 Az Úr Krisztust dicsérjétek,
 Őnéki hálát adjatok,
 Felmagasztaljátok.
/8
#C775FB5D
 Dicséret légyen Atyának
 És ő Fiának, Krisztusnak,
 És a mi Vigasztalónknak,
 A Szentháromságnak.

>240. Úr Jézus, aki felséggel
/1
#164D6F7C
 Úr Jézus, aki felséggel,
 És dicsőséggel mentél égbe
 És ottan vettél hatalmat,
 Nagy birodalmat, teljességbe:
 Lelkünk áldja istenséged
 És híven téged magasztalunk,
 Bárha nem látunk szemünkkel,
 De hitünkkel megtapasztalunk.
/2
#E7400B20
 Te is e dicsőségedből,
 Szent székedből fordítsd le szemed,
 Erőtlen teremtésidre,
 Híveidre öntsd ki érdemed.
 Mennyben létednek hasznában,
 Javaiban részesekké tégy;
 Szent Atyád előtt érettünk,
 Kik vétettünk, esedezőnk légy.
/3
#F35D01B5
 Mivel te utat nyitottál
 És tanítottál mennybe menni,
 Adjad, hogy téged kövessünk
 És siessünk nyomodba’ lenni.
 Segítsd igyekezetünket,
 Vond szívünket te magad után,
 Hogy a te akaratodnak
 S nyomdokodnak járhassunk útján.
/4
#CBEAE5C2
 Míg bujdosunk e pusztában
 S mint hazánkba, az égbe érünk,
 Légy kezesünk, védj bennünket,
 És hitünket neveljed, kérünk,
 Hogy a veszedelmek között
 Megütközött köztünk ne légyen,
 Hitünknek elevenségét,
 Reménységét ne érje szégyen.
/5
#36FB030B
 Majd ha megfutjuk pályánkat,
 Várt pálmánkat a kezünkbe add,
 Lelkünket, ó, szerelmesünk,
 Hű kezesünk, magadhoz fogadd!
 Testünket is emeld végre
 Dicsőségre isteni karral,
 Hol tégedet szemlélhessünk,
 Dicsérhessünk az égi karral.

>241. Úr Jézus, ki mennybe fölmentél
/1
#AEBA6ECA
 Úr Jézus, ki mennybe fölmentél,
 Ott nékünk helyet készítettél.
 Örök otthon vár miránk,
 Azért dicsér most imánk,
 Úr Jézus, így áldunk, így várunk.
/2
#C0C8CEB2
 Úr Jézus, áraszd ki erődet:
 Szólhasson bizonyságot néped,
 Megtalálva a helyét,
 Néked szánja életét!
 Úr Jézus, így áldunk, így várunk.
/3
#4785937B
 Úr Jézus, hisszük: eljössz ismét,
 S hódol majd a világmindenség.
 Leborulunk előtted,
 Úgy dicsérjük szent neved,
 Úr Jézus, így áldunk, így várunk.

>Pünkösd, könyörgés Szentlélekért

>242. A pünkösdnek jeles napján
/1
#8028FF7E
 A pünkösdnek jeles napján
 Szentlélek Isten küldeték,
 Megerősítni szívüket
 Az apostoloknak.
/2
#7F8F9767
 Melyet Krisztus ígért vala
 Akkor a tanítványoknak,
 Mikor méne mennyországba
 Mindenek láttára.
/3
#4CA7A2D3
 Tüzes nyelveknek szólása,
 Úgy mint szeleknek zúgása
 Leszálla az ő fejükre
 Nagy hirtelenséggel.
/4
#7239155F
 Bételvén ők Szentlélekkel,
 Kezdének szólni nyelveken,
 Amint nékik a Szentlélek
 Ad vala szólani.
/5
#C286B856
 Örüljünk azért őneki,
 Mondván szép ékes éneket,
 Felmagasztalván ő nevét
 Örökkön-örökké.
/6
#9ADD72C1
 Dicsértessél Atya Isten,
 És megváltó Fiú Isten,
 Szentlélekkel egyetemben,
 Mindörökké, Ámen.

>243. Ó, áldott Szentlélek
/1
#B1411270
 Ó, áldott Szentlélek,
 kiaz ég dicsőségével,
 Leszállván a földre
 zúgó szél és tűz jelével
 A tanítványok gyűlésébe,
 Úgy munkálkodtál,
 hogy lelkükbe'
 Csuda erőkkel telének be.
/2
#3612033C
 Hatalmas erőddel őket
 egy szempillantásba’
 Megvilágosítád s hoztad
 olyan változásba,
 Hogy aznap, melyen előálltak
 És a népeknek prédikáltak:
 Anyaszentegyházat fundáltak.
/3
#B0E926F8
 Jövel mihozzánk is,
 részeltess ajándékidban,
 Lakozzál mibennünk,
 mint élő templomaidban!
 Adj hitet, adj jó reménységet,
 Adj szentid között egyességet,
 Békességet és idvességet.
/4
#6359C296
 Oszlasd el homályos
 elménknek tudatlanságát,
 Enyhítsd meg elepedt
 szívünknek szomorúságát;
 Éreztessed még itt létünkben
 Az örömöt a mi lelkünkben,
 Melyet adsz örök életünkben!

>244. E pünkösd ünnepében
/1
#548FE05E
 E pünkösd ünnepében,
 E pünkösd ünnepében
 Dicsér jük Istent szívvel,
 Ki Szentlelkét szívünkben,
 Ki Szentlelkét szívünkben
 Osztogatja bőséggel.
 Ennek örül föld,
 Tenger és a menny víg kedvében;
 Minden élő állatok Sok rendben,
 Mik égben, földön vannak
 Sok részben, Fák, füvek és
 minden virágok Újulnak örömben.

>245. E pünkösd ünnepében
/1
#E0D698AB
 E pünkösd ünnepében,
 E pünkösd ünnepében
 Zengnyel vünk dicséreti,
 Mert az Úr Szentlelkében,
 Mert Az Úr Szentlelkében
 Híveit részel-teti: Melynek ere-je
 Minden ismeretnek kútfeje;
 Világos-ságot ő  gyújt Szívünkben,
 E rőt, bátorságot nyújt
 Éltünkben, Áldott
 Lélek, te légy mellettünk,
 S újjá születtetünk.

>246. Jer, kérjük Isten áldott Szentlelkét
/1
#7045C525
 Jer, kérjük Isten áldott Szentlelkét
 Legfőképpen az igaz hitért,
 Hogyha jő a végóra, mellénk álljon,
 Hazatérésre készen találjon,
 Könyörüljön.
/2
#AF7AB990
 Jer, Világosság, ragyogj fel nekünk,
 Hogy csak Krisztus légyen mesterünk,
 El ne hagyjuk őt, mi hű Megváltónkat,
 Aki népének örökséget ad. Könyörüljél.
/3
#0E7071D4
 Ó, Szeretet, áraszd ránk meleged,
 Hadd kóstoljuk édességedet;
 Tiszta szívből mindenkit hadd szeressünk,
 Egyességben és békében éljünk.
 Könyörüljél.
/4
#1552099B
 Ínségeinkben fő Vigasztalónk,
 Halál ellen megbátorítónk,
 Össze ne hagyj esni, ha ellenségünk
 Reánk jő s romlást készít már nékünk.
 Könyörüljél.

>247.  Jézus, az ígéretet
/1
#B676B461
 Jézus, az ígéretet Ím, bételjesítetted,
 Bátorító Lelkedet
 Mihozzánk elküldötted,
 Aki által híveid Elnyerik érdemeid.
/2
#82BB0258
 Ama megfeszíttetett
 Test az égbe vitetett,
 és helyette küldetett
 E reánk kitöltetett Lélek:
 örökkévaló Gyámol és vigasztaló.
/3
#5862965C
 Isten, aki tűzben jött
 Mózest elbocsátani,
 S szélben ment Illés előtt
 Őtet bátorítani,
 Most kettős erőben:
 szent Tűzben s
 szélben megjelent.
/4
#54DCAC1B
 Bátorítja szívüket a Jézus híveinek;
 Tudományt és nyelveket
 Oszt kiküldötteinek.
 Tudatlanból tanítót Tesz,
 s betegből gyógyítót.
/5
#EC526755
 Egy halász, ha prédikál,
 fog sok ezer lelkeket,
 S míg dühösen űzi Pál
 Az eloszlott híveket,
 Útban éri leverő
  Mennyei tüzes erő.
/6
#1089B3D2
 Terjed e tűz az egész
 Föld színére hirtelen,
 Fú e szél s hatalmat vész,
 Ahol akar, szüntelen,
 Lelkesíti sorsosit,
 Szentel és világosít.
/7
#941940EC
 Add nékünk is, Istenünk,
 A te áldott Lelkedet,
 Szent tűz adja érzenünk
 Éltető kegyelmedet;
 Adj hitet, szeretetet:
 Lelki boldog életet.

>248. Jövel, Szentlélek Isten
/1
#EBDB3F77
 Jövel, Szentlélek Isten,
 Tarts meg minket Igédben,
 Ne légyünk setétségben:
 Maradjunk igaz hitben.
/2
#A71812AF
 Szenteld meg mi szívünket,
 Világosítsd elménket,
 Hogy érthessük Igédet,
 Mi édes Mesterünket.
/3
#DE6A6AF5
 Adj isteni félelmet
 És bizonyos értelmet;
 Igéddel taníts minket,
 Gerjeszd fel mi szívünket.
/4
#0170B27E
 Vigasztald meg elménket;
 Mindenben segíts minket;
 Öregbítsed hitünket,
 Távoztassad bűnünket.
/5
#FAE948A2
 Hogy téged az Atyával
 És az ő szent Fiával
 Dicsérhessünk mindnyájan
 A fényes mennyországban.

>249. Jövel, Szentlélek Úr Isten
/1
#826FBAD7
 Jövel, Szentlélek Úr Isten,
 Lelkünknek vigassága,
 Szívünknek bátorsága,
 Adjad minden híveidnek
 Te szent ajándékodat;
 Jövel, vigasztaló
 Szentlélek Isten!
/2
#8A3CD004
 Jövel megnyomorultaknak
 Nemes vigasztalója,
 Árváknak édes Atyja!
 Töltsd bé siralmas szívünket
 Mennyei nagy örömmel.
 Jövel, mi lelkünknek
 Édes vendége!
/3
#9180D7BF
 Távoztasd el mi lelkünknek
 Hitetlen sötétségét;
 Világosíts meg minket,
 Hogy az Istennek igéjét
 Hallhassuk és érthessük!
 Jövel és taníts meg
 Az igaz hitre.
/4
#24148B64
 Jövel, gerjeszd fel szívünkben
 Szent szerelmednek tüzét,
 Rontsd el a gyűlölséget,
 Hogy mi egyesek lehessünk
 Isteni szerelmedben:
 Jövel, mi lelkünknek
 Nagy vigassága!
/5
#5D6F88F5
 Te vagy bizony örök Isten,
 Ki Atyától s Fiútól
 Mindörökké származol.
 Te vagy mi urunk
 Krisztusnak Áldott,
 szent ígérete:
 Te vagy mi lelkünknek
 Megszentelője.
/6
#CF0782A2
 Te vagy, ki a prófétáknak
 Általa régen szóltál,
 Krisztust nekünk ígéréd;
 Te a szent apostoloknak
 Szívüket bátorítád;
 Te vagy erőssége
 Minden szenteknek.
/7
#8A16A6D5
 Te vagy a nagy Úr Istennek
 Mennyei ajándéka,
 Igazságnak mestere.
 Taníts minket a
 Krisztusnak Igaz ismeretire!
 Te vagy mi lelkünknek
 Bölcs tanítója.
/8
#1EDAF9D6
 Világosítsd meg elménket,
 Hogy hihessük a Krisztust
 Egy idvességnek lenni,
 És az áldott Atya Istent
 Kegyes atyánknak lenni;
 Téged ismerhessünk
 Vigasztalónknak.
/9
#13479D6B
 Adjad szent ajándékodat,
 Bátorítsad lelkünket,
 Hogy vallhassuk a Krisztust!
 Adjad, hogy mi meggyőzhessük
 Az ördög csalárdságát!
 Te vagy mi biztatónk,
 Minden oltalmunk.
/10
#0F010EBF
 Biztasd félelmes szívünket,
 Hogy kétségbe ne essünk
 Halálunknak idején,
 De nagy bátorsággal, vígan
 E világból kimúljunk!
 Jövel, vigasztaló
 Szentlélek Isten!

>250. Isten élő Lelke, jöjj, Áldva szállj le rám
/1
#1037AA4D
 Isten élő Lelke, jöjj,
 Áldva szállj le rám,
 Égi lángod járja át
 szívem és a szám!
 Oldj fel, küldj el,
 Tölts el tűzzel!
 Isten élő Lelke, jöjj,
 áldva szállj le rám!
/2
#66A5E72D
 Isten élő Lelke, jöjj,
 légy vezérem itt,
 Ó, segíts, hogy hagyjam el
 bűnök útjait!
 Oldj fel, küldj el,
 tölts el tűzzel!
 Isten élő Lelke, jöjj,
 légy vezérem itt!
/3
#C7C57612
 Isten élő Lelke, jöjj,
 hadd lehessek szent,
 S Jézusommal légyek egy
 már e földön lent!
 Oldj fel, küldj el,
 tölts el tűzzel!
 Isten élő Lelke, jöjj,
 hadd lehessek szent!
/4
#D7D712C7
 Isten élő Lelke, jöjj,
 győzedelmet adj,
 S majd a végső harcon át
 mennybe fölragadj!
 Oldj fel, küldj el,
 tölts el tűzzel!
 Isten élő Lelke, jöjj,
 győzedelmet adj!

>251. Jövel, Szentlélek Úr Isten
/1
#1874144D
 Jövel, Szentlélek Úr Isten,
 Töltsd bé szíveinket épen,
 Menynyei szent ajándékkal,
 Szívbéli szent buzgósággal,
 Melynek szentséges ereje
 Nyelveket egyező hitre
 Egybegyűjte sok népeket,
 Kik mondván, így énekeljenek;
 Alleluja! Alleluja!
/2
#2A2017C9
 Te, szentségnek új világa,
 Vezérelj Igéd útjára,
 Taníts téged megismernünk,
 Istent atyánknak neveznünk.
 Őrizz hamis tudománytól,
 Hogy mi ne tanuljunk mástól,
 És ne légyen több más senki,
 Hanem Krisztus, kiben kell bízni!
 Alleluja! Alleluja!
/3
#C14E48B2
 Ó, mi édes Vigasztalónk,
 Légy kegyes megoltalmazónk,
 Hogy maradjunk útaidban,
 ne csüggedjünk háborúnkban.
 Erőddel elménket készítsd,
 Gyenge hitünket erősítsd,
 Hogy halál és élet által
 Hozzád siessünk hamarsággal!
 Alleluja! Alleluja!

>252. Jövel, teremtő Szentlélek
/1
#0385516F
 Jövel, teremtő Szentlélek,
 És híveiddel légy vélek,
 Szent ajándékiddal szívek
 Újuljon és teljesedjék.
/2
#D8D3230C
 Hathatós Vigasztalónak
 És Isten ajándékának,
 Avagy hét ajándékúnak,
 Isten jobb keze ujjának;
/3
#2EBA798B
 Mondatol élő kútfőnek,
 Tűznek és lelki kenetnek,
 Atyának ígéretinek
 És az igaz szeretetnek.
/4
#C9B3CEFE
 Gerjessz világot elménkben,
 Önts szeretetet szívünkben,
 Erősíts minket hitünkben
 És nagy erőtlenségünkben.
/5
#263C066F
 Adj nékünk teljes örömet,
 Idvességhozó kegyelmet;
 Köztünk minden gyűlölséget
 Ronts el, és adj egyességet.
/6
#06B6D281
 Távoztasd ellenséginket,
 És add meg békességünket;
 Mindenkor vezérelj minket,
 Utálhassuk bűneinket.
/7
#648B4386
 Adjad ismernünk az Atyát,
 És az ő egyszülött Fiát,
 És hinnünk, hogy mindkettőtül
 Szentül származol, vég nélkül.
/8
#EE2C928A
 Dicsőség az egy Istennek,
 Atya, Fiú, Szentléleknek;
 Kedves ajándéka ennek
 Lakjék szívében mindennek.

>253. Könyörögjünk az Istennek Szentlelkének
/1
#BE23AAA2
 Könyörögjünk az Istennek
 Szentlelkének,
 Bocsássa ki magas mennyből
 fénylő világát,
 Végye el mi szívünknek
 minden homályát,
 Hogy érthessük Istenünknek
 Mindenben akaratját.
/2
#E74A693C
 Ó, Szentlélek, árváknak
 kegyelmes Atyja,
 Szegény gyarló
 bűnösöknek bátorítója,
 Hitükben tántorgóknak
 erős gyámola,
 És az ő reménységüknek
 Csak te vagy táplálója.
/3
#0D4584FD
 Te vagy a mi lelkünknek
 édes vendége,
 A mi szomorú szívünknek
 igaz öröme,
 Lelki háborúinknak
 csendesítője:
 Az örök életnek bennünk
 Csak te vagy elkezdője.
/4
#91FBC870
 Te tanítád régenten
 a prófétákat,
 Igazgatád ő nyelvüket
 és írásukat,
 Te tetted bölccsé
 a szent apostolokat,
 Hogy megtérítsék tehozzád
 Mind e széles világot.
/5
#43430918
 Vedd el a mi szívünknek
 hitetlenségét,
 Világosítsd meg elménknek
 nagy setétségét,
 Rontsd el a gyűlölségnek
 kegyetlenségét,
 Engedd a te szent
 hitednek
 Mindenütt egyességét!
/6
#639115FA
 Válassz minket magadnak
 élő templomul,
 Végyed mi könyörgésünket
 szent áldozatul;
 Vedd ki már e világot
 a kárhozatból,
 Engedj igaz hitben való
 Kimúlást e világból.
/7
#FC770130
 Dicsértessél, felséges
 Atya Úr Isten,
 Légyen áldott a te neved,
 Fiú Úr Isten,
 Ezekkel egyetemben,
 Szentlélek Isten:
 Maradjon a te áldásod
 A te bűnös népeden.

>Reformáció

>254a. Erős vár a mi Istenünk
/1
#05088616
 Erős vár a mi Istenünk,
 Jó fegyverünk és pajzsunk,
 Ha ő velünk, ki ellenünk?
 Az Úr a mi oltalmunk.
 Az ős ellenség
 Most is üldöz még,
 Nagy a serege,
 Csalárdság fegyvere;
 Nincs ilyen több a földön.
/2
#A32B9836
 Erőnk magában mit sem ér,
 Mi csakhamar elesnénk;
 De küzd értünk a hős vezér,
 Kit Isten rendelt mellénk.
 Kérdezed: ki az?
 Jézus Krisztus az,
 Isten szent Fia,
 Az ég és föld Ura,
 Ő a mi diadalmunk.
/3
#6D2584DE
 E világ minden ördöge
 Ha elnyelni akarna,
 Minket meg nem rémítene,
 Mirajtunk nincs hatalma.
 E világ ura Gyúljon bosszúra:
 Nincs ereje már,
 Reá ítélet vár:
 Az Ige porba dönti.
/4
#00BF5314
 Az Ige kőszálként megáll,
 Megszégyenül, ki bántja;
 Velünk az Úr táborba száll,
 Szent Lelkét ránk bocsátja.
 Kincset, életet,
 Hitvest, gyermeket
 Mind elvehetik,
 Mit ér ez őnekik?
 Miénk a menny örökre.

>254b. Erős várunk nékünk az Isten
/1
#7BF50EC8
 Erős várunk nékünk az Isten,
 És fegyverünk ellenség ellen,
 Megszabadít veszedelemtől,
 Amely ránk most mindenünnen tör.
 A mi régi ellenségünk
 Háborog velünk,
 Erővel, fegyverrel
 És sok csalárdsággal
 És minden nagy hatalmassággal.
/2
#81022B68
 Nincsen nekünk semmi hatalmunk,
 Mivel neki ellene álljunk.
 Viaskodik az Úr érettünk,
 Kit az Isten küldött le nekünk.
 Ha kérded, hogy: ki légyen az?
 Jézus Krisztus az!
 Seregeknek Ura,
 Kinél nincs több Isten,
 Nála vagyon a győzedelem!
/3
#DB29F4A7
 Ha a világ mind ördög volna
 És elnyomni minket akarna:
 Minket semmi meg nem rendíthet,
 Míg Krisztusba vetjük hitünket.
 Dühös tüzét bármint hányja
 A föld bálványa,
 Vége lesz és elvesz,
 Ledől megalázva
 Az ítélő Isten szavára.
/4
#8E66BC76
 Isten szava megáll mindenha,
 Nem dönti meg senki hatalma!
 Az Ige és Szentlélek bennünk
 Nekünk mindig erős védelmünk.
 Gyermeket, nőt, testi éltet,
 Hírt és jólétet
 Elvihet, benyelhet
 A föld gonoszsága,
 De megmarad Isten országa!

>Keresztelési énekek

>255. Ó, örök Isten! ki Atyánk vagy nékünk
/1
#7F6EBA6E
 Ó, örök Isten! ki Atyánk vagy nékünk,
 Ím, leborulva, esedezve kérünk:
 Ez újszülöttnek földi egész éltét
 Karjaid védjék!
/2
#FEA526F6
 Te is, ó, Jézus,
 Isten Egyszülöttje,
 Nézz le a mennyből
 e földi szülöttre!
 Légy te őnéki út,
 igazság, élet,
 S legyen ő híved!
/3
#06AE9772
 Isteni Lélek,
 te örök fényesség!
 Általad teljes az
 igaz keresztség;
 Bűnt, halált a te
 kicsinyedtől űzz el
 Mennyei tűzzel!
/4
#34003439
 Az Atya, Fiú és
 Szentlélek Isten
 Téged, te kisded,
 mindig úgy segítsen,
 Hogy hited által lelked
 üdvöt nyerjen
 Földön és mennyben!

>256. Úr Isten, kérünk tégedet
/1
#5DF4595F
 Úr Isten, kérünk tégedet:
 Keresztelj és moss meg minket,
 És tisztíts meg kegyesen;
 A Krisztusnak ő vérével
 Nagy bűneinket töröld
 el Szent lelked erejében.
 Amit e szent fürdő jegyez,
 Mindent mi bennünk megszerezz:
 Végy körül szerelmeddel,
 Hogy a te szövetségedben
 Megmaradjunk mindvégiglen,
 Minden mi gyermekinkkel.

>Úrvacsorai énekek

>257. Buzdulj mély hálára, lelkünk
/1
#74BC8F4E
 Buzdulj mély hálára, lelkünk,
 Dicséretet énekeljünk;
 Mert ma velünk az
 Úr jót tett,
 Nagy kegyelmet éreztetett.
/2
#ADC93E4B
 Szent Fiának emlékére,
 Kinek értünk omlott vére
 Asztalát elkészítette,
 Lelkünk megelégítette.
/3
#BCD0F1D6
 Áldott Jézus, lelkünk imád,
 Szeretettel gondolunk rád,
 Hisz először te szerettél,
 Amikor megváltónk lettél.
/4
#FE3516B1
 Mit adjunk, égi királyunk,
 Hogy ily jó voltál irántunk?
 Szívünket neked adjuk át,
 Végy abban örök lakozást!
/5
#BF15890C
 Jertek, hívek, buzduljunk fel,
 Hála poharát vegyük fel;
 Amit fogadtunk, tartsuk meg,
 Segítsen erre kegyelmed!
/6
#4FE4B036
 Te pedig, áldott jó Atyánk!
 Ki szent Fiad küldéd hozzánk;
 Lelkeddel úgy munkálj bennünk,
 Hogy Krisztussal hozzád menjünk.

>258. Hallottuk, Jézus, miképpen hívogatsz
/1
#1881ECFF
 Hallottuk, Jézus,
 miképpen hívogatsz,
 Sietünk hozzád, tudjuk,
 hogy megnyugtatsz.
 Vállunkat nyomja nagy
 terhünk súlya,
 Félünk, erőnket hogy felülmúlja.
 Fáradtak vagyunk,
 régóta emeljük,
 Mert együtt jött ez
 a világra velünk;
 Jézus, terhünktől légy szabadítónk,
 Fáradtságunkban légy megújítónk.
/2
#9CC60897
 Törődött szívvel teszünk neked vallást:
 Hogy nem követtük a jó útmutatást,
 Melyet sugallt lelkünk ismerete,
 Mikor minket jóra serkentgete;
 Inkább követtük gyarló testünk kényét,
 Mint idvességes törvényid ösvényét;
 Ezért nagy lévén szívünk keserve,
 Állunk előtted, mellünket verve.
/3
#E6A314AC
 De mégis lelkünk e hittel élesztjük:
 Hogy te, Megváltónk, nem akarod vesztünk;
 Hanem magadat adtad érettünk,
 Életedet letévén helyettünk.
 A törvény átka tégedet üldözött,
 Míg felfüggeszte az ég és föld között,
 Tulajdon tested ott megáldozád
 És e világ bűnét elhordozád.
/4
#5F869180
 Ezért az Atya magával bennünket
 Megbékéltete s eltörlé bűnünket;
 Ellenségiből tett fiaivá;
 Szent Fia örökös társaivá;
 Jézus! te halállal tévén eleget,
 Megengesztelted a földhöz az eget;
 Általad, amit más nem tehetett,
 A teljes váltság elvégeztetett.
/5
#059D1628
 Azért már minket éleszt és vigasztal
 Ez az általad rendeltetett asztal;
 Mely annak nyilvánvaló tüköre:
 Véred mint omla, tested mint töre!
 Sőt tested jegyét itt nemcsak mutatod,
 De személy szerint nekünk is átadod:
 Zálog ez, hogy van nekünk is jogunk
 A tőled szerzett jókhoz, Jézusunk!
/6
#83F97B38
 Szállj le most mennyből, életnek kenyere!
 Tápláld lelkünket az örök életre!
 Tudjuk: aki e kenyérből eszik,
 Soha örökké meg nem éhezik.
 Életnek vize! nyiss magadnak utat,
 A szomjú hívek keresik e kutat;
 Szolgáltasd ingyen az italokat:
 Oltsd el végképpen szomjúságukat.
/7
#8A459C9C
 Óhajtunk, Jézus, egyesülni veled:
 Úgy lesz szívünk szent, ha te megszenteled.
 Adjad hát, hogy mint tagok a főnek,
 Engedjünk néked, bennünk élőnek.
 Olts be magadba, mint jó szőlőtőbe,
 Hogy jó nedvesség folyjon a vesszőbe,
 És légyen szívünk szívednek mása,
 Éltünkben élted hogy minden lássa.
/8
#C90ACBE0
 Adjad: egymás közt legyen egyességek,
 Akik e közös asztalnál vendégek;
 Mert bizony, aki másokat szeret,
 Csak az eszik itt méltán kenyeret.
 Fogjon hát ez az egész gyülekezet
 Egymással atyafiságosan kezet,
 Mert tudjuk: maga a hívogató
 Úr Jézus nem személyválogató.
/9
#A7A66340
 Megelégítvén lelkünket ez étel,
 Hálaadással távozzunk innét el,
 Mondván: az Úrnak dicsőség legyen,
 Ki az éhezőt betölti ingyen.
 Tartsd fenn, Úr Isten, rajtunk jóvoltodat,
 Add nekünk immár kegyes válaszodat:
 „Bízzál már, fiam, bízzál, leányom,
 bűneidet szemedre nem hányom.”

>259. Én lelkem, ébredj fel az Úrnak dicséretére
/1
#E06BA786
 Én lelkem, ébredj fel az
 Úrnak dicséretére,
 És szent emlékezetére
 Víg szívvel gerjedjél.
 Kegyelmét keressed
 E te lelki orvosodnak,
 Hű és édes táplálódnak
 Jóvoltát hirdessed.
/2
#59071BB6
 Mert éhező valál,
 A szomjúság elepesztett,
 A bűn oly kútba rekesztett,
 Hol kész volt a halál.
 De lelke megesett
 Rajtad e kegyelmes Úrnak,
 Látván, nyavalyáid dúlnak,
 Téged felkeresett.
/3
#3278C5D7
 A kútból kihozott,
 Az igazságnak mezején
 És az életnek kútfején
 Vezérlett, hordozott.
 Végre vitt magához,
 Bíborban felöltöztetett,
 És kegyelméből ültetett
 Ő szent asztalához.
/4
#A0502E4F
 Hol megelégített
 Megtöretett szent testével
 És kifolyt drága vérével,
 melyet elkészített.
 Boldog, ki ezzel él,
 Mert a Sátán annak nem árt,
 Semmi veszély nem tehet kárt,
 A haláltól sem fél.
/5
#701D1029
 Azért, kik éheztek,
 Jertek az Úr asztalához,
 Siessetek jóvoltához,
 És megelégesztek.
 Mert az Úr nem hagyja
 A töredelmes lelkeket,
 Sőt táplálja mindezeket:
 Magát nekik adja.
/6
#77FF382F
 Mint a száj részesül
 A látható szent jegyekben,
 Úgy e nagy Úr a hívekben
 Él, s vélek egyesül.
 Lesz édes dajkájuk:
 Megtöretett szent testének,
 Kiontott drága vérének
 Hasznát szabja rájuk.
/7
#52499CD0
 Hát lelkem, tégy vallást,
 Hogy igaz szívvel szereted
 Ez Urat, s híven követed,
 És néki adsz szállást.
 Jöjj be hozzám, kérlek,
 Jézus, én kedves vendégem,
 Mert te vagy én dicsőségem,
 Csak téged kedvellek.

>260. Hirdetvén az Úr halálát
/1
#B0D7CF27
 Hirdetvén az Úr halálát,
 Uram, néked adunk hálát,
 Hogy nemcsak tartod testünket,
 Hanem táplálod lelkünket.
/2
#8BAC0B19
 Ma is ételt készítettél
 Ennek, s asztalt terítettél,
 Melyről Fiad szent testével,
 Elégítél szent vérével.
/3
#7EB64EC9
 Ezzel világos jelt adtál
 S minket arról megbiztattál,
 Hogy ha igaz hitünk lészen,
 Fiad miénk lesz egészen.
/4
#F287DA56
 Kérünk, Uram, adj kegyelmet,
 És nyújts nekünk segedelmet,
 Hogy Jézusunkat szeressük
 És őtet végig kövessük.
/5
#33E100D9
 Az ő megtöretett teste,
 Mellyel váltságunk kereste,
 Lelkünk máskor is táplálja,
 A bűnért meg ne utálja.
/6
#866F5EED
 Segítsd gyarló tehetségünk,
 Véle kötött szövetségünk
 Hogy szent és állandó légyen,
 Mígnem lelkünk hozzá mégyen.
/7
#A2B18B7C
 Amely fogadást ma tettünk,
 Midőn asztalodról ettünk,
 Uram, segélj, hogy megálljuk,
 A bűnöket megutáljuk.

>261. Jer, lássuk az Úr keresztjét
/1
#E428E3BD
 Jer, lássuk az Úr keresztjét,
 Melyet felvett érettünk,
 Megszánván embernek vesztét,
 Megfizete helyettünk.
 Úr lévén, lett szolgává,
 Mindeneknek csúfjává,
 Istenségét elrejtette,
 Midőn testünket felvette.
/2
#CCA41915
 A Gecsemáné kertjében
 Kezdette szenvedését
 Érezvén a kínt lelkében
 S a halál rettentését,
 Ivék keserű pohárt,
 Fizette értünk az árt;
 Véres cseppjei testének
 Nékünk gyógyulást szerzének.
/3
#09645731
 Itt szintén lelkéig hata
 Sebes vizek mélysége,
 Itt kezdődék a nagy csata
 S lelke keserűsége:
 Egyedül hagyattaték,
 Éjjel megfogattaték;
 Ki mindent kézen fogva tart,
 Megfogattatni így akart.
/4
#89FDEBDB
 Feszítésre ítélteték
 S gyalázatos kínokra,
 Pilátushoz vitetteték,
 Önként ment ez átokra.
 Ki ügyét hogy meghallá,
 Őt ártatlannak vallá,
 Mégis adá keresztfára,
 A legkínosabb halálra.
/5
#6EDA60CC
 Mint Prófétát, csúffá tették,
 Mert befedvén szent fejét,
 Verték, s ki verte? kérdezték,
 Kísértvén szent erejét.
 Mint Királyt, meggyalázák,
 Tövissel koronázák,
 Náddal verék a bársonyban,
 Csúfolák térdhajtásokban.
/6
#C721D257
 A fát adván szent vállára,
 Mint Pap, meggyaláztaték,
 Az is lőn gyalázatjára, hogy megostoroztaték.
 A vitézek kifoszták,
 És ruháit megoszták;
 A nagy bűnöst elereszték,
 A szentet fára függeszték.
/7
#5CBF35EE
 Mily csuda buzgó szerelem:
 Meghalni barátiért,
 De e kegyes Fejedelem
 Meghalt ellenségiért.
 Ez lelkünk drága bére,
 Mert Isten fia vére:
 Mily drága az az áldozat,
 Mellyel romol a kárhozat.
/8
#41239B16
 Ezért szerzé szent asztalát,
 Hogy e jót előadja,
 Szenvedését és halálát
 Szemeink előtt hagyja.
 A kenyér megtörése,
 A bornak kitöltése
 Lelke, teste szenvedését,
 Jelenti megöletését.
/9
#6FD4AF97
 Ez egyszerű vendégségben
 Jézussal egyesülünk,
 Egy kenyérből e szentségben
 Hívők mind részesülünk.
 E rövid szent vacsora
 Mutat mennyei jóra;
 Ezt az ő kínt látott teste
 Választottinak kereste.
/10
#F78CBED8
 Az Úr sebei mi sebünk,
 Halálunk ő halála;
 Ő érdeme mi érdemünk,
 Részesülés ez nála.
 Pecsétli ezt az étel,
 Együtt a pohárvétel:
 Testét midőn hittel esszük,
 Magunk ő testévé tesszük.
/11
#D124D23F
 Ez eledel zálog nékünk,
 Hogy lesz mennyben menyegzőnk,
 Ezzel hitünk, reménységünk:
 Öregbül lelki erőnk.
 De mit ér sokszor enni,
 E szent vacsorát venni,
 Ha lélek szerint nem élünk,
 Büntetést enni nem félünk?!
/12
#B4D36FAE
 Ha meghaltunk Úr Jézussal,
 Nem illik már vétkeznünk;
 Feltámadván Úr Krisztussal,
 Le kell a bűnt vetkeznünk.
 Mint mennyei lakosok,
 Legyünk tiszták s okosok;
 Mint testének s vendégének,
 Szükséges lennünk szenteknek.
/13
#2C3ADF37
 Isten ártatlan Báránya,
 Méltó vagy, hogy végy áldást,
 Mert érdemed azt kívánja,
 Végy tőlünk hálaadást!
 Végy örök dicsőséget,
 Hatalmat, tisztességet,
 Mert értünk megölettettél,
 Atyádnak eleget tettél.
/14
#C4997BAC
 Add, Úr Jézus, halálodban,
 Követnünk életedben,
 Szentséges tudományodban,
 Szelíd szenvedésedben.
 Légyünk alázatosak.
 Mindenhez irgalmasak;
 Segélj erre Szentlelkeddel,
 Oktass, vezérelj igéddel!

>262. Megterítve áll előttünk
/1
#96B464AD
 Megterítve áll előttünk
 A szeretet asztala;
 Maga szerzé nagy Mesterünk,
 Ki érettünk meghala:
 Hogy halála érdemében,
 A kenyér és bor jegyében
 Hit által részesüljünk, .
 Örök életet vegyünk.
/2
#AB44C58B
 Áldott Jézus, te tanítád:
 A test semmit sem használ.
 Lélek az, mi életet ád,
 S az élet magad voltál.
 Lélek és élet beszéded,
 S csak az egyesülhet véled,
 Ki törvényed szereti,
 Nyomdokodat követi.
/3
#A060D089
 Ó, idvesség fejedelme!
 Az volt a te életed,
 Az volt lelked igyekvése
 S buzgó törekedésed,
 hogy mennyei szent Atyádnak,
 Az ég felséges Urának
 Akaratja beteljen
 Itt alant és odafenn.
/4
#A5E6D177
 Te hű voltál egész végig
 Lelked érzeményéhez,
 Engedelmes mindhalálig
 Szent Atyád tetszéséhez;
 Ily hűségre köteleznek
 Minket a bornak s kenyérnek
 Étele és itala:
 A szeretet asztala.
/5
#04DA603A
 Kenyér és bor szent jegyei
 Kínszenvedéseidnek,
 Zálogai, pecsétei
 Örök üdvösségünknek.
 Ó, Jézus! Ez a vendégség
 Lelkünknek örök üdvösség;
 Előttünk a szent asztal:
 Lelkünk téged magasztal.
/6
#94B37DBD
 Halljuk hívó szózatodat,
 Melyben a fáradtaknak
 Nyújtod vigasztalásodat,
 Az elesett világnak:
 Szomjazónak, éhezőnek,
 Betegnek és szenvedőnek
 Életet s felújulást,
 Nyújtasz itt szabadulást.
/7
#B2E349D7
 Áldott Jézus, híveidre
 Töltsd ki mostan Lelkedet;
 E buzgó gyülekezetre
 Áraszd ki kegyelmedet,
 Hogy Lelkedtől lelkesülve,
 Asztalodban részesülve
 Veled eggyé lehessünk,
 És életet nyerhessünk!
/8
#DB6C1305
 Te szenvedtél kínt érettünk,
 És hoztál új életet;
 Te vagy a mi fejedelmünk:
 Ó, áldd meg híveidet!
 Hogy kereszten tört testednek,
 És kiontatott vérednek
 Jegyeivel úgy éljünk,
 Hogy benned egyesüljünk.
/9
#2C60822D
 Te az ég dicső Királya,
 Te vagy az Üdvözítő!
 Tied a mennyek országa,
 Te vagy odavezető.
 Te Király vagy, mi polgárid,
 Te Mester, mi tanítványid;
 Te vagy a Fő, mi tagok:
 Csak általad boldogok.
/10
#6271A807
 Az volt lelked hagyománya
 Utolsó vacsorádban,
 Hogy egy kenyér-
 s egy pohárban
 Részesüljünk mindnyájan.
 Szelíd Lelked szent emlékét,
 A szeretet vendégségét
 Hagytad tanítványidnak,
 Kik tégedet vallanak.
/11
#3D7171ED
 Egy kenyérben, egy pohárban
 Csak testvérek osztoznak,
 Kik az ég dicső Urában
 Közös Atyát imádnak:
 Aki mindeneknek Atyja,
 S Lelke jó kedvét szállítja
 Földi gyermekeire,
 A testvéri szívekre.
/12
#F95C29FD
 Asztalodnál egyesítsél
 Minket szívben, lélekben;
 Magas célunkra segítsél
 S köss össze szeretetben,
 Hogy így imádva Atyánkat,
 Szeretve embertársunkat
 Kövessük nyomdokodat,
 S nyerjük meg országodat!
/13
#3B3C14F6
 Asztalodnál erősíts meg
 Minket igaz hitünkben,
 Hogy gyümölcseit hozzuk meg
 Már e földi életben:
 Legyen igazság, békesség,
 Köztünk szeretet s egyesség;
 Jézus, hitünk védelme,
 Légy célunk segedelme!

>263. Ó, Jézus, mi idvességünk
/1
#8EB10CBE
 Ó, Jézus, mi idvességünk,
 Fejedelmünk, dicsőségünk,
 Szívünknek fő vigasztalója!
 Rólad, csak rólad énekel
 Szánk buzgó dicséretekkel,
 Lelkünknek édes táplálója;
 Egyedül csak hozzád térünk,
 Mert hű pásztornak ismerünk.
/2
#0208CAC5
 Ó, mily szent az az indulat,
 Mely veled örömest mulat,
 Ó, Jézus, idvesség forrása!
 Boldog, akit hívsz magadhoz,
 Méltóztatsz szent asztalodhoz,
 Hogy kegyelmed jelét ott lássa.
 Bizony, boldog az oly ember,
 Ki nálad kedvességet nyer.
/3
#5D7E4E33
 Mert érdemedben részesül,
 Teveled, Uram, egyesül,
 Ki hív tagja e vendégségnek.
 Mint a bort s kenyeret veszi,
 Akképp részesévé teszi
 A hite őt az idvességnek,
 Melyhez ártatlan Jézusunk
 Vére által vagyon jussunk.
/4
#44F9E76E
 Azért, kit bűnök rongálnak,
 Járulj e kegyes királynak
 Víg örömmel szent asztalához!
 Nem vet el ez irgalmas szív,
 Mert könyörülő, igaz, hív,
 Csak hogy folyamodj oltalmához:
 Bévesz sebe rejtekébe,
 Mint ígéri szent jelébe.
/5
#572E1E48
 Nincs is egyébben oltalmunk,
 Reménységünk, bizodalmunk,
 Édes Jézus, hanem csak benned.
 Szállj hozzánk, kedves vendégünk,
 Oltsd meg lelki nagy éhségünk,
 Mert eledelünk csak érdemed;
 Szívünket, ímé, kitárjuk,
 Szentségedet várván várjuk.

>264. Lelkem siet hozzád menni
/1
#2CB8E9F4
 Lelkem siet hozzád menni,
 Ámbár gyenge ereje,
 Kíván asztalodról enni,
 Ó, életnek kútfeje!
 Hogy megelégülhessen,
 Benned része lehessen.
/2
#953E745D
 Kedveld, Uram, kegyességét
 A te szegény szolgádnak,
 Éreztessed édességét
 Elkészült vacsorádnak,
 Hogy véled egyesüljön
 S lelke benned örüljön.

>265. Idvességünk, váltságunk
/1
#BFBD1EBE
 Idvességünk, váltságunk,
 Jézus! hozzád kiáltunk;
 Midőn a kegyelem asztalához lépünk:
 Jövel, maradj mivélünk!
/2
#04AD31F5
 Nagy volt szüleink bűne,
 De Atyád könyörüle;
 És az örök halál tüzéből megváltott
 Engesztelő halálod.
/3
#FB5FECBB
 Ez a terített asztal
 Bűneinkben vigasztal:
 Hogy könyörülsz rajtunk
 s kegyelmed velünk lesz,
 Ha bízunk érdemedhez.
/4
#1675AF5D
 Szent testednek jegyei
 Annak itt a jelei:
 Hogy csak az igazán
 megtérő találhat
 Idvességet tenálad.
/5
#9DB580DD
 Vérednek kiömlése
 Bűneink eltörlése;
 Emlékeztess, midőn
 ajkunk a bort issza:
 Légyen éltünk szent s tiszta!
/6
#99930672
 űJézus! a te véred szent!
 Légyen szívünk bűntől ment;
 mert örök életet veled az remélhet,
 Ki megveti a vétket.
/7
#171D79A4
 Szent törvényed fáklyája
 Világoljon pályánkra!
 Mert e földi úton, tudjuk,
 el nem téved,
 Ki híven követ téged.
/8
#83A699DA
 Fogjad azért kezünket,
 S magad vezérelj minket;
 Mert a te utadon
 Istenünkhöz vezet
 A hit, remény, szeretet.
/9
#8CD1148E
 Ha szemeink meglátnak
 Jobbján te szent Atyádnak:
 Ott a Szentlélekkel
 örök egyességben
 Üdvözíts majd az égben!

>266. Ó, Úr Jézus, áldassál
/1
#46A0BD46
 Ó, Úr Jézus, áldassál
 A te nagy szerelmedért;
 És felmagasztaltassál
 Áldott segedelmedért;
 Hogy midőn eltévedtünk,
 Testben jöttél érettünk.
/2
#3728C044
 Sőt egész életedben,
 Kivált midőn a halál
 Körülvett: személyedben
 Kínt szenvedni kész valál;
 Bűnünknek rettentésit,
 Hordozván büntetésit.
/3
#BAB1EA89
 Hisszük, hogy szenvedésed
 Váltság álnokságinkért,
 Vérrel lett fizetésed
 Elég adósságinkért;
 Félelmünket elvetted,
 Istent Atyánkká tetted.
/4
#ABE8100D
 Úr Jézus, táplálj minket
 Szenvedésed hasznával,
 Töröld el bűneinket \.
 Tested áldozatjával;
 Vigaszalj gyötrelmiddel,
 Gyógyíts meg sebeiddel!

>267. Örülj, szívem, Vigadj, lelkem
/1
#E87B4CF7
 Örülj, szívem, Vigadj, lelkem,
 Ékességed lett a hit;
 Vacsorához, Mégy Jézushoz,
 Hivatalos vagy te itt.
/2
#4B2B3BFE
 Ha bűnödért Halálos bért
 Érdemlettél lelkedre:
 E szent asztal
 Megvigasztal
 S válik idvességedre.
/3
#E2F463BB
 Ez örömben, Reménységben,
 Jézus, ma hozzád jövök.
 Asztalodnál Lábam megáll:
 Testem, lelkem újítsd meg.
/4
#3A9738CA
 Tisztogass meg, Bűnből moss meg
 Kegyelmedből, Istenem,
 És a Sátán lelkem kárán
 Nem örül, nem árt nekem.
/5
#B7444D7F
 E vacsora Égi módra
 Engem véled összead,
 Maradj Bennem, benned engem
 Hagyj lennem, hogy áldjalak.
/6
#D663A405
 Csakhogy immár Bűntől elválj
 S légy hívséggel Jézushoz:
 Halálával, Váltságával
 Minden bűnből feloldoz.
/7
#F63B78C8
 Hát jöjjetek, Bűnös lelkek,
 Orvosságot kik vártok!
 Jézus lelke, Szent kegyelme
 Kiárad ma reátok.

>Konfirmációi énekek

>268. Áldást kérünk, ó, nagy Isten!
/1
#E6A54495
 Áldást kérünk, ó, nagy Isten!
 Szent fogadást tett ma itten
 Házadnak ifjú népe,
 Felírva már fenn az égben,
 Az örök élet könyvében
 A szent frigy, melybe lépe.
 Tárva már a mennyországa
 boldogsága ő szívük-ben;
 Tartsd meg, áldd meg szent hitük-ben!
/2
#F15AF561
 Kitől jön ránk minden áldás,
 Legyen néked hálaadás,
 Ó, szent, ó, véghetetlen!
 Erős bástyánk, sziklavárunk!
 Tartsd meg anyaszentegyházunk,
 Hogy álljon rendületlen!
 S arcra hullva, Leborulva,
 halleluja Zengjen itten.
 Tarts meg, áldj meg, ó, nagy Isten!

>269. Szentháromság, egy Istenünk!
/1
#777A1545
 Szentháromság, egy Istenünk!
 Hozzád buzgón esedezünk:
 Nézz alá e kis seregre,
 Fogadd őket kegyelmedbe!
/2
#CCA6D639
 Jövel, jövel, áldott Isten!
 Gyújts szent lángot szíveikben,
 Hogy legyenek tiszták, szentek,
 Midőn rólad vallást tesznek.
/3
#BA38AD4D
 Legyen hitük rendületlen,
 Reménységük csüggedetlen!
 Jézus őket bűntől óvja,
 És segítse minden jóra!
/4
#6F0900DE
 Uram! Fiad szent egyházát,
 Idvességünk erős várát,
 Mint mennyei seregekkel,
 Növeld mindig új hívekkel!

>270. Szent hitünkről vallást tettünk
/1
#DC6D0964
 Szent hitünkről vallást tettünk,
 Te hallottad, jó Atyánk!
 Engedd mindig azt követnünk,
 A-mit itt most fogadánk!
 Jézus a mi vezérünk,
 Példája szerint élünk,
 Így élünk, így jutunk végre
 Örök élet üdvösségre.
/2
#5ED2D2C2
 Sorsunk boldog vagy mostoha,
 Ez a szent hit lesz velünk.
 El nem felejtjük, nem, soha,
 Mit tanít nagy Mesterünk!
 Amit Jézussal vallunk,
 Abban élünk és halunk;
 rád tekintünk, máshová nem,
 Isten úgy áldjon meg! Ámen.

>Egyházi tisztviselők választásakor

>271. Küldj áldást az őrállókra
/1
#C8A28A83
 Küldj áldást az őrállókra,
 Felséges Isten!
 Szolgáidat ne hagyd köztünk magukra,
 Siess oltalmukra: Felséges Atya Isten!
/2
#9BA4A137
 Szent egyházad kormányzóit
 Áldd meg, Úr Isten!
 Vezesse néped bölcs tanácsadóit
 Az igaz, buzgó hit, Felséges Atya Isten!
/3
#66B094B6
 A választott nemzetséget
 Áldd meg, Úr Isten!
 Lássék meg rajtuk a
 te fényességed
 És a te szentséged,
 Felséges Atya Isten!
/4
#705A5504
 Te királyi papságodat
 Áldd meg, Úr Isten!
 Hirdessék ők is a te hatalmadat
 És igazságodat,
 Felséges Atya Isten!
/5
#15343A03
 Áldd meg a te szent népedet,
 Felséges Isten!
 Bocsásd ki mennyből a te
 Szentlelkedet,
 Add rá kegyelmedet,
 Felséges Atya Isten!

>Tanévkezdésre

>272. Halleluja! Halleluja
/1
#3A41A2E3
 Halleluja! Halleluja,
 Iskoláink tárt ajtaja
 Várja ezt az ifjú népet,
 Mely áldoz most, Uram, néked.
 Munkakedvvel, víg énekkel
 Házadból így indulunk el.
/2
#FEB1155D
 Hű pásztorunk, jó Istenünk,
 Nagy utunkra, ó, jöjj velünk!
 Vigyázz reánk, nevelj minket,
 igazgassad lépteinket,
 Míg utunkat fedné homály,
 Tűzoszlopként előttünk járj!
/3
#88A3175F
 Templomodba hogy betértünk,
 Egy drága név volt vezérünk.
 Tekintettünk Jézusunkra,
 Ki zászlót bont ma számunkra.
 Lelkünknek add e hű vezért,
 Miértünk is ő onta vért.
/4
#8B5F58EB
 Áldd meg, óvd meg szép hazánkat,
 Virágoztasd egyházunkat!
 Tanítóink te szolgáid:
 Légyenek mind áldottaid!
 Igazság és jó békesség
 Köztünk egymást ölelgessék.

>273. Mi kegyes Atyánk, bölcsességnek Ura
/1
#A8B0AC95
 Mi kegyes Atyánk, bölcsességnek
 Ura, És mindeneknek
 nagy bölcs tanítója:
 Akinek nem vagy te igazgatója:
 Nincs oktatója.
/2
#EB936EA7
 Mi, te fiaid, tudatlanságunkban,
 Gyermekségünkben
 és ifjúságunkban,
 Néked könyörgünk
 mi tanulásunkban,
 Imádságunkban.
/3
#10587F3F
 Világosítsd meg tudatlan elménket,
 Tanulásunkban vezérelj bennünket,
 A tudományra gerjeszd fel szívünket
 És mi lelkünket.
/4
#6D70D510
 Adj jó tanító mestereket nékünk,
 Adj bölcsességet általuk minékünk,
 Adj jó erkölccsel, értelemmel élnünk,
 Mi jó Istenünk!
/5
#FCDFBF3F
 Őrizz meg minket gonosz erkölcsöktől,
 Minden naponként fertelmes beszédtől,
 Szent színed előtt utált részegségtől,
 Éktelenségtől.
/6
#F31F5AC6
 Hogy jövendőre mi felnevekedvén,
 Téged dicsérjünk, mindenkor tisztelvén,
 Felmagasztaljunk, nagy jámborul élvén,
 Néked engedvén.
/7
#8A852373
 Hogy szolgálhassunk te szent egyházadnak,
 És használhassunk felebarátinknak,
 De kiváltképpen atyánknak s anyánknak,
 Édes hazánknak.
/8
#E6B1BB74
 Hogy holtunk után a nagy iskolában,
 Szentháromságnak ő tanításában,
 Az angyaloknak szép társaságokban
 Tanuljunk jobban.
/9
#3E4DB008
 Dicsőség néked, kegyes Atya Isten,
 És uralkodó áldott Fiú Isten!
 Te, vigasztaló Szentlélek
 Úr Isten! Áldj meg hitünkben!

>Tanévzárásra

>274. Jertek, ifjak, gyermekek!
/1
#578743BC
 Jertek, ifjak, gyermekek!
 Istenünkhöz gyűljetek.
 Magasztaljuk szent nevét,
 Hűséges őrizetét;
 Ki tőlünk egy éven át
 Meg nem vonta oltalmát.
/2
#3BC55CAC
 ”Az Úr Isten félelme
 bölcsességnek kezdete.” -
 Ebben voltunk szorgosak,
 Most már háládatosak:
 Hogy Istenben élhettünk,
 Általa növekedtünk.
/3
#C4726913
 Atyánk, ha ez év alatt
 Elmúlt bár egy pillanat
 Hasztalanul s nélküled:
 Lelkünk ezért meg ne fedd!
 Jézus vére érdeme
 Bűneinket mossa le!
/4
#9E4FA122
 Áldást osztó jobb kezed
 Védje e hont s nemzetet.
 Szent egyház és iskola
 Híveidnek tábora:
 bizonysági legyenek
 Atyai szerelmednek.

>Aratási ének

>275. Nagy Isten! ki az aratásnak
/1
#3EDF3F58
 Nagy Isten! ki az aratásnak
 Heteit megtartod
 És a mezei gyors munkásnak
 Csűrét gazdagítod:
 Ím, megnyilván a nyári áldás
 A mi határinkon,
 Méltó, hogy öröm, hálaadás
 Zengjen ajakinkon.
/2
#0F07BCDF
 Megért mezők szép sárga színben
 Jóságod hirdetik
 Víg aratók nagy örömükben
 Neved emlegetik.
 Mi is dicsérünk,
 Atyánk, téged
 Ez áldott munkában,
 Felmagasztalván istenséged
 Minden búzaszálban.
/3
#891D842A
 Az aratás a te jóságod,
 Végtelen szeretet!
 Termő erejében te tartod
 A szép természetet.
 Rajtad kívül nem adhat senki
 Kenyeret számunkra,
 Te szolgáltatod nekünk azt ki
 A mi asztalunkra.
/4
#FD917AA5
 Áldott legyen,
 Atyánk, nagy neved,
 Hogy gabonát adtál.
 Magasztaljuk nagy istenséged,
 Hogy így megáldottál.
 Add: mértéklettel élvezhessük
 Szép ajándékidat,
 S éltünk végéig érezhessük
 A te javaidat.

>Reggeli énekek

>276. Alleluja, dicsérjétek
/1
#53F87216
 Alleluja, dicsérjétek,
 Istenfélő keresztyének,
 Dicsérjétek az Urat!
 Jelen van ő most is velünk,
 Őtőle van új reggelünk,
 Mely reánk most felvirradt:
 Háláljuk meg segedelmét,
 Kérjük ismét nagy kegyelmét,
 Mely örökké megmarad.

>277. Hálát adok néked, mennybéli Isten
/1
#36F183B1
 Hálát adok néked, mennybéli
 Isten, Szent Fi-ad-nak, a
 Krisztusnak nevében,
 Hogy engemet ez éjjel megőriztél,
 Minden veszedelemtől megmentettél;
 Tartsd meg, kérlek, e napon is éltemet,
 Bűn-től, minden kár-tól ments
 meg en-ge-met.
/2
#5A23C7EA
 Magam, Uram, ajánlom szent kezedbe,
 Mind testem, lelkem vedd őrizetedbe,
 Szent angyalidat ne vedd el mellőlem,
 Szent Lelked se távozzék el éntőlem,
 Hogy vakmerő bűnöktől ma is menten
 Megőriztessék épen testem, lelkem.
/3
#D0C4629A
 Könyörgök, Uram, minden emberekért,
 De főképpen a hív keresztyénekért,
 Minden rokonimért, kik téged félnek,
 Akik vagy itt, vagy messze földön élnek;
 Minden gonosztól őket is őrizd meg,
 És minden javaiddal látogasd meg.
/4
#76AB011B
 A szegény rabokat és betegeket,
 Kik ínségükben óhajtnak tégedet,
 Uram, vigasztald meg bágyadt szívükben,
 Szenvedésükből mentsd ki kegyelmesen;
 És térítsd meg a szegény bűnösöket,
 Add, hogy jó véghez vigyük életüket.

>278. Magasztallak én téged
/1
#08147424
 Magasztallak én téged,
 Isten, egeknek királyát,
 Hogy tőlem meszsze űzted
 A sötét éjnek homályát.
 Nem küldél rám betegséget,
 Sem egyéb ínséget,
 Épségben megtartottál,
 E napra engem juttattál.
/2
#8193727F
 Szívből könyörgök néked,
 Kegyes teremtő Istenem:
 E napot is engedd meg
 Békességgel véghezvinnem,
 Akaratodat tanulnom,
 Útaidban járnom;
 Oltalmad béfedezzen,
 Kedvem kedved szerint légyen.
/3
#DDC020CF
 Igaz utadra taníts,
 Hogy veled együtt járhassak,
 És tőled el ne taszíts,
 Hogy kísértetbe ne jussak.
 Jóvoltodból tarts meg engem,
 Én édes Istenem,
 Hogy a bűn csalárdságát,
 Észre vegyem undokságát.
/4
#6E72ACC5
 Az igaz hitnek tüzét
 Bennem Krisztusért élesszed,
 Gyarlóságomnak vétkét
 Soha szememre ne vessed;
 Fogadásodat tekintsd meg,
 Szent Fiadért tarts meg,
 Ki értem eleget tett,
 Törvény átkától megmentett.
/5
#0966FCB4
 Reménységgel ruházz fel,
 Ördög tőrébe ne essem,
 Szívem hozzád gerjeszd fel,
 Ne csak hasznomat keressem.
 Atyafi szent szeretetet,
 Adj jámbor életet;
 Szeress, mint sajátodat,
 Kövessem akaratodat.
/6
#4C13F390
 Szent Igédet vallanom
 Adjad tisztán, homály nélkül
 És szolgádnak mondatnom
 Minden képmutatás nélkül;
 Kinccsel semmit sem gondolok,
 Más jutalmat várok:
 A szentek seregében,
 Tarts meg gyülekezetében.
/7
#55EB3FA8
 E napot is engedd meg,
 Uram, békével megfutnom,
 Hivatalom hasznával
 Magam s cselédim táplálnom,
 Hogy szent nevedet dicsérjem,
 Oltalmadat nyerjem:
 Testemnek megmaradást,
 Lelkemnek adj boldog szállást.
/8
#306E0577
 Uram, egyedül vagy jó.
 Te vagy út, élet, igazság,
 De én életem gyarló.
 Gondolatom csak hamisság.
 Táplálj Krisztus szent testével,
 Itass szent vérével,
 Hogy örökké nevednek,
 Énekeljek Felségednek.
/9
#F0554878
 Hála légyen Atyának,
 Fiúnak és Szentléleknek,
 Kitől minden jók vagynak:
 Ne engedj azért ördögnek
 Engem, igaz megváltottad,
 Gyenge juhocskádat:
 Lelkem a boldogságba,
 Vigyed a paradicsomba.

>279.  Jézus Krisztus, szép fényes hajnal
/1
#AF1317AC
 Jézus Krisztus, szép fényes hajnal,
 Ki feltámadsz új világgal,
 És megáldasz minden jókkal:
/2
#CB6DEE57
 Te vagy nékünk egy reménységünk,
 Isten előtt közbenjárónk,
 Szép koronánk, ékességünk.
/3
#A006DF78
 Világosítsd a mi szívünket:
 Ismerhessünk meg tégedet;
 Tanulhassuk szent Igédet.
/4
#7A77F421
 Az ördögnek csalárdságától,
 Lelki-testi nyavalyától,
 Őrizz hamis tudománytól.
/5
#DC11983D
 Az Atyával és Szentlélekkel,
 Hogy tégedet tiszta szívvel
 Mindörökké áldjunk! Ámen.

>280. Megújult testtel és erővel
/1
#3C284EF7
 Megújult testtel és erővel
 Fölébredvén az álomból,
 Nyugodt szívvel, felemelt fővel
 Felkelvén csendes ágyamból,
 Életnek Ura, hozzád térek,
 Magasztalom jóvoltodat,
 Minden jót ismét tőled kérek:
 Kérem újra áldásodat.
/2
#7881E486
 Mint a feljövő nap világa
 Elűzi az éj homályát:
 Kegyelmednek világossága
 Az igazságnak fáklyáját
 Sötét elmémben jobban gyújtsa,
 Hogy a jónak ismerete
 Határit benne messzebb nyújtsa,
 S az igazság szeretete.
/3
#98D89845
 Engedd, hogy mint a nap futása
 Soha meg nem állapodik,
 S bár felhő jő néha útjába,
 De meg nem homályosodik:
 Én is az igazság ösvényén
 Tántorodás nélkül menjek,
 És a rút testiségnek kényén
 Soha csak meg se pihenjek.
/4
#7B985509
 Minthogy míg e világon élek,
 E mulandóságnak helyén,
 Test is vagyok, és nemcsak lélek:
 Viselj gondot jó idején
 Mértékletes eledelemről,
 Hogy életem fenntarthassam,
 S tisztességes öltözetemről,
 Hogy testemet ruházhassam.
/5
#6BD1936E
 Ha pedig még többet vehetek
 Ingyen való jóvoltodból
 És másokkal is jót tehetek
 Velem közlött áldásodból:
 Engedd, hogy legyek hű sáfára
 Nálam letett javaidnak,
 Lehessek vigasztalására
 Velem testvér fiaidnak.
/6
#7EAB98D3
 Dolgaimnak követésére
 Adj testemnek egészséget,
 A bajoknak meggyőzésére
 Lelkemnek elevenséget!
 Indíts szívemben akaratot
 És készséget minden jóra,
 Hogy bennem vidám indulatot
 Leljen az estvéli óra.

>281. Néked éneklek, Neved magasztalom
/1
#CBDAA7E0
 Néked éneklek,
 Neved magasztalom,
 Szívvel lélekkel,
 Hív érzésekkel
 Néked esdeklek,
 Ó, örökkévaló!
/2
#9EFEFC06
 Vajha nem volna
 Semmi bűn szívemben,
 Ami hűtelen Személyem ellen
 Vádakat szólna
 Majd az ítéletben!
/3
#EAA9F881
 Jó Atyám, segíts
 Sok jót véghez vinnem,
 E nap se légyen
 Rám nézve szégyen,
 Hagyj benned hinnem,
 S Szentlelkeddel őrizz!
/4
#64E4D274
 Lelked szózatja
 Legyen bölcs vezérem;
 Nem tévedek el,
 Sőt víg énekkel
 Zeng szívem hangja,
 Ha estvét megérem.
/5
#62C5E992
 Most a munkára
 vetem elmém, kezem;
 Ha súlyos lészen.
 Állj, Uram, készen,
 Mert szent szavadra
 Könnyül az én terhem!

>282. Örök élet reggele
/1
#0AA4E60D
 Örök élet reggele,
 Fény a véghetetlen fényből,
 Egy sugárt küldj ránk te le,
 Kik új napra ébredénk föl;
 Fényed lelkünk éjjelét
 Űzze szét.
/2
#25509CCA
 Jóságodnak harmata
 Gyarló életünkre hulljon,
 Szívünk, mely kiszárada,
 Vigaszodtól felviduljon,
 S híveid közt légy
 jelen Szüntelen.
/3
#246F4C86
 Bűn ruháját vessük el
 A szövetség vére által,
 Vétkeink fedezzük el
 Tőled nyert fehér ruhánkkal,
 Hogy hitünk legyőzze
 majd Mind a bajt.
/4
#F5FECFE1
 S majd vezess az égbe föl,
 Irgalomnak napvilága,
 Könnyek gyászos völgyiből
 Üdvösségnek szép honába,
 Hol az üdv és béke
 majd Egyre tart.

>283. Reménységemben hívlak, Uram Isten!
/1
#CE5B705E
 Reménységemben hívlak,
 Uram Isten!
 Reggeli órám rólad áll szívemben;
 Testem is óhajt hozzád e helyben,
 Hol éltető víz nincs e kietlenben.
/2
#C1182F32
 Csudámra vagyon
 szépsége házadnak,
 Vajha ott volnék,
 hol téged szolgálnak!
 Orvosság én lelkem
 fájdalmának,
 Az lenne öröm
 könnyező orcámnak!
/3
#99E61382
 Élesztő nékem
 nagy-drága beszéded,
 Melyben dicsőségedet kijelented,
 Embernek szívét
 azzal emeled,
 Szent biztatásra
 mikor kényszeríted.
/4
#B750064F
 Velem viselem
 ott is ízit annak,
 Valahol nékem
 nyugodalmat adnak,
 Csak jókat tőlem
 mindenütt hallnak,
 Édességedben
 ajakim mozognak.
/5
#7BB76BB1
 Sőt mind egész
 jövendő életemben
 Áldlak, dicsérlek
 tégedet örömben.
 Élet vagy: azért
 tiszteletedben
 Hozzád emelem
 kezeimet hitben.
/6
#ED76E357
 Azért ajakim
 csak téged dicsérnek,
 Szájam és nyelvem
 rólad énekelnek,
 Ágyamban is
 veled beszélgetnek,
 Éjjel és nappal
 téged emlegetnek.
/7
#3641711F
 Mint drágalátos
 illatú kenettel
 És kívánatos,
 ízes eledellel:
 Úgy vidul lelkem
 éltetéseddel,
 Mikor vigasztalsz
 angyali örömmel.
/8
#1E13961D
 Te vagy, ki nékem
 váltságot ígérhetsz,
 Ki engemet szent
 szárnyad alá rejthetsz,
 Vélem egyedül
 minden jót tehetsz,
 Előttem, Uram,
 soha el nem mehetsz.
/9
#1BDCAF1D
 Te ígéretedhez
 mert támaszkodom,
 Az én lelkemben
 csak tebenned bízom;
 Én árvaságom
 azzal táplálom,
 Hogy jobb karodnak
 árnyékában nyugszom.

>284. Szívem megalázván, tehozzád megyek
/1
#9F663455
 Szívem megalázván, tehozzád megyek,
 Elődbe, Istenem, hál'adást viszek
 És szent Fiad által néked könyörgök.
/2
#CDBFEC45
 Áldott légy, én Uram, hogy megtartottál,
 Bút és kárt ez éjjel rám nem bocsátál,
 Angyali sereggel oltalmam voltál.
/3
#6BA4D190
 E reggeli időt megadtad érnem,
 Melyben egészséges elmém és testem;
 Kérlek, minden jóra vezérelj engem.
/4
#20EC5FAF
 Mai nap ezt velem cselekedd,
 Uram: Mindenféle bűntől magam óvhassam,
 Hitem jó gyümölcsét hogy el ne rontsam.
/5
#CD14CF65
 Ételem, italom mérsékelt legyen,
 Tobzódás, részegség ne nehezítsen,
 Rossz gondolat bennem erőt ne vegyen.
/6
#31485CD9
 Lágyítsd meg énbennem az indulatot,
 Meg ne háborítsak soha másokat,
 És haragtartásra ne adjak okot.
/7
#7C14B6EC
 Jól tudod, Istenem, mily gyarló vagyok,
 Különb-különb bűnre mily könnyen hajlok!
 Adjad: megtérhessek, mikor bűnt vallok!
/8
#69F82C5B
 Igaz ítéleted ne ostorozzon,
 Bűnből a Krisztusért ingyen oldozzon,
 Érdemem szerint rám átkot ne hozzon.
/9
#814768E6
 Magamat egészen neked szentelem:
 Kegyelmes oltalmad legyen mellettem,
 Szentlelked és Igéd legyen vezérem.
/10
#C0D85DA1
 Külső károktól is ments meg engemet,
 Ne eméssze bánat és bú szívemet;
 Tartsd jó egészségben gyarló testemet!
/11
#72533F86
 Istenem, tenéked legyen dicsőség,
 A Szentháromságban ki vagy egy Felség:
 Csak tégedet illet minden tisztesség!

>Esti énekek

>285. Már nyugosznak a völgyek, Az erdők s minden földek
/1
#27F26691
 Már nyugosznak a völgyek,
 Az erdők s minden földek,
 Már alszik a világ,
 De míg eljő az álom,
 A szívemet kitárom,
 S az Úr-hoz küldök hő imát.
/2
#E28840E0
 Ó, nap, hová tűnél el,
 Hová űzött az éjjel,
 Mely harcban áll veled?
 Te nem ragyogsz az égen,
 De más napom van nékem:
 Betölti Jézus szívemet.
/3
#CCD31DEE
 Már rám borul az éjjel,
 De biztatnak reménnyel
 Ott fenn a csillagok,
 Hogy engem is a mennybe
 Felvisz az Úr kegyelme,
 Ha a földtől megválhatok.
/4
#81595CEF
 A test nyugalmat áhít,
 És leveti ruháit, Halandóság jelit,
 De Jézusom az égbe’
 Öltöztet dicsőségbe,
 Ha végórám elközelít.
/5
#AB2924C6
 Ti, fáradt tagok, mostan
 Tegyétek le nyugodtan
 A napnak terheit.
 Vigadj, szívem, te is majd
 Levetheted a bút, bajt
 S a bűn terhét, mely keserít.
/6
#3B55C231
 Már álom jő szememre;
 Ki vigyáz életemre,
 Ha most elszunnyadok?
 Izráel őrizője
 Lesz házamnak védője:
 Nem érhetnek károk, bajok.
/7
#05CC22E5
 Te légy, Jézus, oltalmam,
 Nálad lesz jó jutalmam,
 Hű szárnyaid alatt.
 Te vigyázz csak,
 Uram, rám, Nem árt akkor a Sátán:
 Testem-lelkem békén marad.
/8
#CFBDF3B5
 Ti is távol s közelben,
 Akik szerettek engem:
 Békén pihenjetek;
 A sötét éjszakába’
 Az Úr világossága
 Őrködjék hűn fölöttetek.

>286. Adjunk hálát megtartó Istenünknek!
/1
#5F412C9C
 Adjunk hálát megtartó Istenünknek!
 Ismét egy napját eltölténk éltünknek,
 Melyen ő nékünk békességet adott,
 Gondviselő oltalmába fogadott;
 Megadá mindennapi kenyerünket,
 Megtartá erőnket,  egészségünket.
/2
#6AE581EA
 Ajánljuk magunk ismét oltalmába
 A már bekövetkező éjszakába’,
 Mert nem lehet annak semmi félelme,
 Akinek a Mindenható védelme;
 Az övéinek ő ád nyugodalmat,
 Akik csak benne keresnek oltalmat.
/3
#3A68933D
 Köszönjük, Uram, hogy mirajtunk ma is
 Gondviselésed volt hűséges paizs,
 Hogy megmentetted veszélytől éltünket,
 Szánkat panasztól, sírástól szemünket;
 Munkánkra áldást s elég erőt adtál,
 És az estére békében juttattál.
/4
#1EE42BE7
 Hányan, kik hosszú életet reméltek
 Reggel tetőled: még estét sem értek;
 És egészségért sokan esedeztek,
 Kiket fájdalmak ágyukba szegeztek;
 Árvákká lettek sokan s özvegyekké,
 Vagy gazdagokból váltak szegényekké.
/5
#F14694A2
 Mennyivel voltunk mi ezeknél jobbak,
 Hű védelmedre mennyivel méltóbbak?
 Mégis te minket ím külön választál,
 Ránk semmi romlást, kárt,
 veszélyt nem hoztál;
 Hordozott ingyen nagy kegyelmességed:
 Áldunk örökké ezért, Uram, téged!
/6
#35983354
 Kérünk, ez éjjel is tartsd meg éltünket,
 Csendes álommal újíts meg bennünket,
 Több örvendetes reggelre virrassz fel,
 Új életre új napoddal támassz fel,
 Hogy új erővel tégedet szolgáljunk:
 Az Úrnak éljünk és az Úrnak haljunk.

>287. Lásd, Urunk, egy nap újra eltűnt
/1
#9DBA1609
 Lásd, Urunk, egy nap újra eltűnt,
 Ránk borul már az esthomály;
 Hajnali himnuszt néked zengtünk,
 Most hozzád es -ti ének száll.
/2
#69703D2A
 Hála, hogy ébren szentegyházad:
 Nappalokon és éjeken
 Szüntelen áldhat és imádhat,
 Nem hallgat el egy percre sem.
/3
#60C882A3
 Mert int a hajnal könnyű szárnya,
 Nem pihen, partról partra kel,
 Zeng új dicséret és új hála,
 Néped imája nem hal el.
/4
#841760F0
 Már a nap távol testvért ébreszt,
 Még alig hamvad fénye itt,
 Friss ajak dicsér egyre téged,
 És magasztalja tetteid.
/5
#80DCBF77
 Így legyen! Mert tiéd az ország,
 A hatalom és dicsőség.
 Jő a nap, melyen majd együtt áld
 És együtt dicsér minden nép.

>288. A nap immár elenyészett
/1
#A1446E13
 A nap immár elenyészett,
 Az ég besötétedik,
 Nyugvóra vált a természet,
 És minden csendesedik;
 Engemet is az álom
 Megújít, feltalálom
 Fáradt testemnek nyugvását,
 Elmémnek megvidulását.
/2
#0AFE6848
 Míg hát fejem lehajtanám
 A szükséges álomra,
 Gondjaim elbocsátanám,
 Menvén nyugodalomra:
 Gondviselőm, táplálóm!
 Jóságodat hálálom;
 Aki ma is úgy szerettél,
 Hogy sok jóban részeltettél.
/3
#6E9D63AA
 Adtál erőt, tehetséget,
 Adtál ösztönt a jóra,
 De vajon bennem készséget
 Talált-e minden óra?
 Magam is jól érezem,
 Hogy sokra nem érkezem,
 A lélek kész, ámde a test
 Sokszor tehetetlen és rest.
/4
#4E96CE62
 Kegyelmednek köszönhetem
 E nagy ajándékot is,
 Hogy a jóra érezhetem
 Már csak a szándékot is.
 Amit azért kezdettél,
 Bennem felélesztettél,
 Kérlek, hogy félbe ne hagyjad,
 Sőt gyarapodását adjad.
/5
#0C3A85A0
 Bocsássad meg hibáimat,
 Melyeket ma ejtettem,
 Szaporítsad javaimat,
 Melyeket tőled vettem.
 Engedjed, hogy halálom,
 melynek a csendes álom
 Kiábrázoló példája:
 Légyen idvesség órája.

>289. Csendesül a föld határa
/1
#95CD5A5A
 Csendesül a föld határa,
 Ránk borul az éj homálya;
 De mielőtt lepihenünk,
 Vedd hálánkat, jó Istenünk!
/2
#D0B836DE
 Vedd hálánkat védelmedért,
 Amely mellett kár, baj nem ért;
 amely mint egy erős paizs,
 Oltalmazott minket ma is.
/3
#16D4FDDB
 Gondviselő, jó Istenünk!
 Ez éjen is te légy velünk!
 Te takarj be szent kezeddel,
 Gondod rólunk, ó, ne vedd el!
/4
#CF362B76
 Add, hogy az éj csendes álmát,
 Mely fölöttünk gyorsan száll át,
 Semmi veszély ne zavarja,
 Virrassz fel majd új nappalra!
/5
#8BB74ACA
 Új nappalra, új erővel,
 Jó Istenünk! minket költs fel,
 Hogyha ismét minden ébred,
 Zengjük mi is dicsőséged!

>290. Krisztus, ki vagy nap és világ
/1
#197DD565
 Krisztus, ki vagy nap és világ,
 Minket sötétségben ne hagyj!
 Igaz világosság te vagy,
 Kárhozatra mennünk ne hagyj!
/2
#034CC871
 Téged kérünk, szent Úr Isten:
 Oltalmazz minket ez éjen;
 Nyugodalmunk benned légyen,
 A mi lelkünk el ne vesszen!
/3
#17FD9F10
 Nehéz álom el ne nyomjon,
 Az ellenség meg ne csaljon;
 Testünk hozzá ne hajoljon
 És haragodba ne hozzon!
/4
#F7AE7738
 Mi szemeink ha alusznak,
 Szíveink rád vigyázzanak;
 Te hatalmadnak ereje
 Légyen híveid őrzője!
/5
#3BBB58EE
 Úr Isten, hozzád kiáltunk:
 Gondviselőnk, légy oltalmunk!
 Őrizz meg ellenségektől,
 Lelki, testi ínségektől!
/6
#06D748AC
 Parancsoljad angyalidnak,
 Hogy mireánk vigyázzanak;
 A mi gonosz ellenségünk
 Messze távol járjon tőlünk!
/7
#B1584445
 Emlékezzél meg mirólunk:
 Jól tudod, mily gyarlók vagyunk;
 Kiket megváltál véreddel:
 Úr Jézus, kérünk, ne hagyj el!
/8
#A809315D
 Dicsőség légyen Atyának,
 Ő szent Fiának, Krisztusnak,
 Szentlélekkel egyetemben,
 Örökkön-örökké! Ámen.

>291. Maradj velem, mert mindjárt este van
/1
#700E70BD
 Maradj velem, mert mindjárt este van,
 Nő a sötét, ó, el ne hagyj, Uram;
 Nincs senkim és a vigaszt nem lelem,
 Gyámoltalannal, ó, maradj velem.
/2
#01D89726
 Kis életem fut s hervadásba hull,
 Bú lesz a vígság, fényesség fakul,
 Csak változást és romlást lát a szem;
 Változhatatlan, ó, maradj velem.
/3
#7E73DFBB
 Minden múló perc Hozzád visz közel,
 Kegyelmed űzi kísértőmet el,
 Nincs más vezérem,
 Nincs más Mesterem,
 Fényben, borúban, ó, maradj velem.
/4
#B5D5B33B
 Ellenség ellen áldásod fedez,
 A könny nem sós, a kór is könnyű lesz,
 Sír, halál-fúlánk, hol a győzelem?
 Győztes leszek, csak légy,
 Uram, velem.
/5
#D4414692
 Hunyó szemembe vésd keresztedet,
 Ködöt foszlatva láttasd szent eged.
 Föld árnya fut, menny fénye megjelen:
 Halálban is Te légy, Uram, velem.

>292. Maradj velünk, mi Krisztusunk
/1
#37120342
 Maradj velünk, mi Krisztusunk,
 Feltámadott fényes napunk!
 Mert már a nap, ím, alkonyul,
 Az éjszaka reánk borul.
/2
#5DC6D2A8
 Méltatlanul ha szenvedünk,
 Ha siralom az életünk,
 Ha a halál hozzánk közel:
 Te légy velünk, Jézus, jövel!
/3
#D98B74C3
 Nincs elhagyva, ki búban él,
 Bár örömet már nem remél;
 Mert teveled vigaszt talál
 Ott, hol nincs éj és nincs halál.
/4
#448474D0
 Már a halál nem lesz örök,
 diadalma már megtörött,
 A sírban is lesz virradat,
 Mert a Jézus velünk marad.
/5
#09E1EE2E
 Ó, ha az éj reánk borul,
 Földi napunk ha alkonyul,
 Maradj velünk, mi Krisztusunk!
 Te légy mindig fényes napunk.

>293. Nagy hálát adunk, kegyes Atyánk, néked
/1
#B4A6B4A3
 Nagy hálát adunk, kegyes
 Atyánk, néked,
 Hogy te ez napon nékünk
 megengedted
 Nagyszép békével élnünk teelőtted:
 Dicsőség Néked!
/2
#041633E3
 Immár e napnak kimenetelében
 Néked könyörgünk,
 Atyánk, Igaz hitben:
 Segíts meg minket
 minden szükséginkben,
 Áldj meg lelkünkben!
/3
#6A8C29AC
 Hogy megnyugodjunk
 a mi munkáinktól,
 Törődésinktől és fáradságinktól,
 Néha pediglen mi nagy siralminktól
 És bánatinktól.
/4
#61E8757C
 Hogy tiszta szívből
 áldhassunk tégedet,
 Szép énekszóval
 dicsérjük nevedet,
 Józan elmével imádjunk
 tégedet Mint Istenünket.
/5
#7BEF653F
 A sötét éjnek reánk
 jövésében
 Adjad, hogy lelkünk
 ne legyen sötétben,
 Sem pedig hitünk
 tökéletlenségben
 És tévelygésben.
/6
#E5BE31C6
 Ha a mi testi
 szemeink alusznak,
 Lelki szemeink reád
 vigyázzanak,
 A mi bűneink mind
 elaludjanak
 És meghaljanak.
/7
#A507812B
 Tartsd, Atyánk,
 tisztán testünket, lelkünket,
 Őrizz meg bűntől
 álmunkban is minket,
 Az álnok ördög
 ne bírjon el minket
 Erőtleneket.
/8
#FA1E4554
 Adj békességes
 nyugodalmat nékünk,
 És tennenmagad
 vigyázz, Urunk, értünk,
 Nagy-szép békével
 légyen felkelésünk,
 Teljes életünk.
/9
#4682F9DE
 Tégedet kérünk,
 Istennek szent Fia!
 Néked könyörgünk,
 mi Urunknak Atyja!
 Hogy amit kérünk,
 Szentlelked megadja:
 Megkoronázza!

>294. Ne jöjjön addig szememre álom
/1
#07AC3914
 Ne jöjjön addig
 szememre álom,
 Míg Teremtőmnek
 és Gondviselőmnek,
 Kitől minden jó
 adományok jőnek,
 Jótéteményit meg
 nem hálálom.
/2
#91365B7F
 Ha elgondolom,
 mennyi jót vettem Tetőled,
 Atyám, méltatlan létemre:
 Csudálom, mint
 vigyázol életemre,
 Ki csak eltűrést
 sem érdemlettem!
/3
#1F7A7831
 Áldalak, hogy e
 megrepedezett
 Nádszálat ma eltörni
 nem engedted;
 Gyarló életem híven
 vezérelted, Melyet
 sok veszély megkörnyékezett.
/4
#62FFF3E1
 Számnak mind étele, mind itala,
 Az ép elme az egészséges testben,
 Az öröm és a békesség szívemben:
 Ingyen való adományod vala.
/5
#D59C070F
 Kihez menjek több kegyelmet kérni?
 Szemem és szívem tehozzád emelem,
 Mert a tenálad lakozó kegyelem
 Mélységeit nem lehet megmérni.
/6
#966F1CF0
 Bizton hajtom le fejem ez éjjel,
 Ha te, hű pásztor, tartasz
 engem szemmel,
 Így szemem nem lát s nem hallok fülemmel:
 Nem bánthat senki semmi veszéllyel.
/7
#B79FD5AC
 Légy hát őrállóm ez éjszakában!
 Hogyha tőreit útaimba hányja,
 Aki vagy vesztem, vagy károm kívánja:
 Ne menjen elő rossz szándékában.
/8
#57167C24
 Adjad, hogy véghezvivén munkáim,
 Lelkem és testem újra erőt végyen,
 És nyugodalmam mértékletes légyen,
 Hogy munkára szánhassam óráim.
/9
#8BDBA606
 Nem bocsátlak el, Atyám, tégedet,
 Míg meg nem áldasz engemet, fiadat:
 Dicsőítsd meg hát bennem irgalmadat,
 És én magasztalom Felségedet.

>295. Ó, lelkem szent napsugara!
/1
#D49C04A2
 Ó, lelkem szent napsugara!
 Ahol te vagy, nincs éjszaka.
 Bár földi köd szakadna fel,
 S látás elől ne fedne el.
/2
#66498B3E
 Ha csendes est szememre száll
 S szelíd álomharmat szitál,
 Gondoljam azt, Egyetlenem:
 Nyugodni jó a szíveden.
/3
#E9597CAB
 Reggeltől estig légy velem,
 Nincs nálad nélkül életem;
 Légy velem, ha az éj leszáll,
 Nélküled rémít a halál.
/4
#D37269C4
 Ha egy bolygó, bús gyermeked
 Gúnyolta szód, mert tévedett,
 Ne hagyd bűnben, Kegyelmes,
 őt, Emeld fel azt a vétkezőt.
/5
#8F7B00EF
 Virraszd, akit kór súlya nyom,
 Ki koldus, áldd meg gazdagon;
 Kit kín szorít, gyász keserít,
 Légy álma, könnyű és szelíd.
/6
#EDD65A64
 Jöjj és áldj meg, ha ébredünk
 S világ útján járunk-kelünk,
 Míg jóságod szent tengerén
 Majd elmerít az égi fény.

>Bűnbánati énekek

>296. Amint vagyok, sok bűn alatt
/1
#2710FFAB
 Amint vagyok, sok bűn alatt,
 De hallva hí -vó hangodat,
 Ki értem áldozád magad:
 Fogadj el, Jézusom!
/2
#5E98A8B9
 Amint vagyok - nem várva, hogy
 Lelkemnek terhe, szennye fogy,
 Te, aki megtisztíthatod: -
 Fogadj el, Jézusom!
/3
#38E6AF08
 Amint vagyok - bár gyötrelem,
 S kétség rágódik lelkemen,
 Kívül harc, bennem félelem: -
 Fogadj el, Jézusom!
/4
#7EB56405
 Amint vagyok - vak és szegény,
 Hogy kincset leljek benned én,
 S derüljön éjszakámra fény: -
 Fogadj el, Jézusom!
/5
#70CCF5C9
 Amint vagyok - nincs semmi gát,
 Kegyelmed mit ne törne át;
 Hadd bízza lelkem rád magát: -
 Fogadj el, Jézusom!
/6
#F5F19300
 Amint vagyok - hogy a te szent
 Szerelmed tudjam, mit jelent
 Már itt s majd egykor odafent: -
 Fogadj el, Jézusom!

>297. A töredelmes szívet, Te, Uram, szereted
/1
#CCCBC01E
 A töredelmes szívet,
 Te, Uram, szereted,
 Az engedelmes lelket
 Soha meg nem veted.
 Ezzel a reménységgel
 Tehozzád óhajtunk,
 Légy, kérünk, segítséggel
 És könyörülj rajtunk.

>298. Bár bűn és kín gyötör, És nehéz bár szívem
/1
#FC1B449D
 Bár bűn és kín gyötör,
 És nehéz bár szívem,
 A Sátán életemre tör:
 Kétségbe nem esem.
/2
#B7FA055B
 Bár vétkem súlya nagy,
 Mégis hozzád jövök.
 A bűnnek gyűlölője vagy,
 De kegyelmed örök.
/3
#3C64A11E
 Az én erőm kicsiny,
 S a bűn erős nagyon:
 Te tudsz s akarsz segíteni,
 Hát segíts bajomon!
/4
#AD2FF5AB
 Az ég oly messze van,
 Még messzebb tőled én,
 De szent igédben írva van,
 Hogy irgalmad enyém.
/5
#452455DB
 Nem félek senkitől,
 Hisz te vigyázsz reám,
 Már bánat és gond sem gyötör:
 Meghallgatod imám.
/6
#8352DC80
 Jézusban bízva én szívemet átadom,
 Mert így, tudom: akármi ér,
 Atyám szeret nagyon.

>299. Az Isten Bárányára Le-té-szem bű-nöm én
/1
#9E778C20
 Az Isten Bárányára
 Letészem bűnöm én,
 És lelkem béke várja
 Ott a kereszt tövén.
 A szívem mindenestül
 Az Úr elé viszem,
 Megtisztul minden szennytül
 A Jézus vériben,
 A Jézus vériben.
/2
#A5E9B4C7
 Megtörve és üresen
 Adom magam neki,
 Hogy újjá ő teremtsen,
 Az űrt ő töltse ki.
 Minden gondom, keservem
 Az Úrnak átadom,
 Ő hordja minden terhem,
 Eltörli bánatom,
 Eltörli bánatom.
/3
#6304908C
 Örök kőszálra állva
 A lelkem megpihen;
 Nyugszom Atyám házába’
 Jézus kegyelmiben.
 Az ő nevét imádom
 Most mindenek felett;
 Jézus az én királyom,
 Imámra felelet, Imámra felelet.
/4
#BEE4C011
 Szeretnék lenni, mint ő,
 Alázatos, szelíd,
 Követni híven, mint ő,
 Atyám parancsait.
 Szeretnék lakni nála,
 Hol mennyei sereg
 Dicső harmóniába’
 Örök imát rebeg, Örök imát rebeg.

>300. Bocsásd meg, Úr Isten, ifjúságomnak vétkét
/1
#CE965A15
 Bocsásd meg, Úr Isten, ifjúságomnak vétkét,
 Sok hitetlenségét, undok fertelmességét,
 Töröld el rútságát, minden álnokságát,
 Könynyebbítsd lelkem terhét.
/2
#31391680
 Az én búsult lelkem én nyavalyás testemben
 Tétova bujdosik, mint madár a szélvészben;
 Tőled elijedett, Tudván, hogy vétkezett,
 Akar esni kétségben.
/3
#F1449BD3
 Akarna gyakorta hozzád ismét megtérni,
 De bűnei miatt nem mer elődbe menni,
 Tőled oly igen fél. Reád nézni sem mér,
 Színed igen rettegi.
/4
#A86BFFC5
 Semmije sincs pedig, mivel elődbe menjen,
 Mivel jóvoltodért viszont téged tiszteljen,
 Nagy alázatosan, Méltó haragodban
 Tégedet engeszteljen.
/5
#C1D6E46E
 Bátorítsad, Uram, azért biztató szóddal!
 Mit használsz szegénynek örök kárhozatjával?
 Inkább hadd dicsérjen E földön éltében
 Szép magasztalásokkal.
/6
#FE5D48EE
 Térj azért, én lelkem, kegyelmes Istenedhez,
 Szép könyörgésekkel béküljél szent kezéhez,
 Mert lám, hozzá fogad, Csak reá hagyd magad:
 Igen irgalmas Úr ez.
/7
#71015DFD
 Higgyünk mindörökké egyedül csak őbenne,
 Őrizkedjünk bűntől, ne távozzunk őtőle.
 Áldott az ő neve Örökké mennyekbe’,
 Ki már megkegyelmeze.

>301. Bűnösök, hozzád kiáltunk
/1
#F4386853
 Bűnösök, hozzád kiáltunk:
 Úr Isten, könyörülj rajtunk!
 Nyisd meg a te füleidet,
 Hallgasd meg könyörgésünket!
 Mert ha te mind megítéled,
 Amit vétettünk ellened,
 Mind elkárhozunk előtted.
/2
#27F85553
 Nagy a te irgalmasságod,
 A bűnt Te megbocsáthatod;
 Nincs nékünk semmi érdemünk,
 Már semmi ártatlanságunk.
 Nincs ki kérkedjék előtted,
 Félünk mindnyájan Tégedet,
 És könyörgünk mi Tenéked.
/3
#B4858605
 Bízzunk azért az Istenben,
 És nem a mi érdemünkben,
 Nyugodjék Őbenne lelkünk,
 Ő légyen mi reménységünk;
 Lám, nékünk nyilván ígéré,
 Hogy akar oltalmunk lenni,
 Azért higgyünk csak Őnéki.
/4
#2EFB10F6
 Erősek legyünk hitünkben,
 Bízzunk csak az Úr Istenben;
 Ne essünk Benne kétségben,
 Ne szomorkodjunk lelkünkben.
 Minden keresztyén hív légyen,
 Ki megújult Szentlélekben és
 bízik csak az Istenben.
/5
#41F84705
 Ha minékünk sok bűnünk van,
 Istennek több kegyelme van,
 Ő irgalmának nincs vége,
 bár sok az emberek vétke.
 Ő nékünk kegyes pásztorunk,
 Ki őriz minket pokoltól
 És megment a kárhozattól.

>302. Ím, nagy Isten, most előtted szívem kitárom
/1
#21F243D7
 Ím, nagy Isten, most előtted szívem kitárom,
 Menedékem nincs sehol e földi határon;
 Ha te nem jössz bánatomra biztató szóval,
 Italom könny, a kenyerem keserű sóhaj.
/2
#1ECA8780
 Ha a világ nem tudná is számos bűnömet,
 Teelőled elrejtenem semmit sem lehet;
 Látja Lelked minden bűnöm, melynek átka sújt:
 Vedd le rólam, ó, Úr Isten, vedd le ezt a súlyt!
/3
#88111152
 Jézusomra föltekintek a kereszt alatt,
 Nincs szívemnek nyugodalma vétkeim miatt;
 Ó, ne büntesd, Uram, azt, kit megtört a bánat:
 Szálljon reám irgalmadból béke, bocsánat!
/4
#70543E30
 Szent Fiadért, ki engemet vérén megváltott,
 Hallgass meg, ha bűnbánattal hozzád kiáltok!
 Vigaszoddal térj kegyesen beteg szívemhez,
 Hozzád térő gyermekednek, Atyám, kegyelmezz!

>303. Én Istenem! Sok nagy bűnöm
/1
#B4FC531B
 Én Istenem! Sok nagy bűnöm
 Lelkemet szorongatják;
 Itten nincsen, Ki segítsen,
 Hát kihez folyamodjam?
/2
#FEE90E90
 E világon Minden úton
 Akármerre indulok:
 Súlyos terhem, betegségem
 Ki elvegye, nem látok.
/3
#4CBF3F14
 Hozzád térek,
 Végy be, kérlek,
 Jó Atyám, s haragodban
 Meg ne büntess,
 Légy kegyelmes
 Hozzám te szent Fiadban.
/4
#96CFFFDB
 Ha terhelnem,
 Kell szenvednem
 Az igaz ítélettől:
 Bár itt büntess,
 Csak ott kedvezz:
 Ne vesszek el örökűl.
/5
#E84E3A0D
 Ha kereszted Reám veted,
 Adj engedelmes szívet,
 Hogy tűréssel
 És reménnyel
 Várjam üdvösségemet!
/6
#585D20C4
 Nyújtsd kegyelmed
 És Szentlelked,
 Hogy engem vezéreljen;
 Bűn, ellenség
 S hitetlenség
 Kárhozatra ne vigyen!
/7
#1E5CCF8E
 Mint kis madár,
 Ha szélvész jár
 Az égi csattogásban,
 Rejtez fákban És odúkban,
 Hol megmaradhat bátran:
/8
#498FC807
 Így a bűnök, Halál s ördög
 Engemet ha rettentnek:
 Rejtekhelye Vagy, reménye,
 Ó, Krisztus, én lelkemnek.
/9
#B9896C58
 Sebeidben Rejts el engem,
 Hol bízvást megmaradok;
 Bár kínt valljak,
 Vagy meghaljak,
 Jól tudom: hozzád jutok.
/10
#29EB3BB2
 Te énnékem Ellenségem
 Haláloddal meggyőzted,
 Boldog helyem, Idvességem
 A mennyben megszerezted.
/11
#7737E00F
 Áldott Isten Egy felségben,
 Tebenned szívem örül,
 Mert te szavad
 Áll s megmarad:
 Aki hiszen, idvezül.

>304. Én nem perlek, És nem merlek
/1
#171EF30E
 Én nem perlek,
 És nem merlek,
 Igaz bírám, vádolni,
 Ha elkezded Ítéleted
 Énrajtam gyakorolni.
/2
#17CFFB89
 Enyhén bántál,
 Nem kívántál
 Semmit sem erőm felett;
 Az izgató Csalogató
 Mégis engem bűnre vett.
/3
#FA1A3A89
 Megvettetést,
 Nagy büntetést
 Ezért előre látok;
 Szörnyű dolog
 Lesz, ha megfog
 A törvényben írt átok.
/4
#F9037CFF
 Sokszor hallom:
 Az irgalom Nálad éri az eget;
 De ismétlen
 Véghetetlen Igazságod fenyeget.
/5
#63444396
 E kettő közt Olyan eszközt
 Találnom lehetetlen,
 Hogy jó maradj
 És mégse hagyj
 Egy bűnt sem büntetetlen.
/6
#84EA7286
 De midőn én Kételkedvén
 E két mélység közt állok:
 Útmutatást, Megtartatást
 Szent igédben találok.

>305. Istenem, én nagy bűnös ember
/1
#DAED03C7
 Istenem, én nagy bűnös ember,
 Szent színed elé járulok,
 Vétkem oly mély már, mint a tenger,
 Mentségért hozzád fordulok.
 Én Istenem, én Istenem,
 Irgalmazz, kérlek, énnekem!
/2
#28C1451B
 Szívem szerint ím elkesergem
 Gonosz és sok bűneimet:
 Hogy tőled én eltévelyedtem,
 Elhagytalak, Teremtőmet.
 Én Istenem, én Istenem,
 Irgalmazz, kérlek, énnekem!
/3
#AB70B224
 Hallgasd meg én fohászkodásim
 Atyai nagy szerelmedből,
 Bocsásd meg minden rút bűneim,
 Mentsd ki szívem ez ínségből.
 Én Istenem, én Istenem, \.
 Irgalmazz, kérlek, énnekem!
/4
#46C231C5
 Ne büntess úgy, mint érdemlettem
 Tőled én undok bűnömmel,
 Mert akkor nyilván el kell vesznem:
 Térj hozzám hát jó kedveddel.
 Én Istenem, én Istenem,
 Irgalmazz, kérlek, énnekem!
/5
#8AEEE7E3
 Csak egy szót mondj, hogy újjá légyek,
 mondd ezt énnékem, bűnösnek:
 ”Megengedtem, menj el békével,
 Meglásd, többé ne vétkezzél.”
 Én Istenem, én Istenem,
 Irgalmazz, kérlek, énnekem!
/6
#592936EF
 Nincs kétségem, megvigasztaltál,
 Erősítéd én szívemet,
 Könyörgésemben meghallgattál,
 Érzem már szent kegyelmedet.
 Én Istenem, én Istenem,
 Irgalmazz, kérlek, énnekem!

>306. Mindenható Úr Isten, Mi, bűnös emberek
/1
#6D30CBE8
 Mindenható Úr Isten,
 Mi, bűnös emberek,
 Gyónást és vallást teszünk
 Mint töredelmesek.
/2
#6B32A4A8
 Mert mi igen vétkeztünk
 Istenséged ellen,
 Mint teremtő, megváltó
 És szent Atyánk ellen.
/3
#D079B5D2
 Életünknek rendiben
 Igen megbántottunk,
 Mi nagy sok bűneinkkel
 Téged bosszantottunk.
/4
#EF0CB798
 Gonosz szóval, szándékkal,
 Látással, hallással,
 Irigységgel, mordsággal,
 Rágalmazásokkal.
/5
#F643C720
 Szitkos, átkos voltunkkal,
 Haragtartásunkkal,
 Hamisan mi hitünket
 Gyakran mondásunkkal;
/6
#8C16A388
 Isteni káromlással
 És hitetlenséggel,
 Mi nyomorúságinkban
 Békételenséggel.
/7
#DEEE5577
 Istennek szent igéjét
 Noha gyakran halljuk,
 De mibennünk majd semmi
 Gyümölcsét nem látjuk.
/8
#5061650A
 Az isteni és atya-
 Fiúi szeretet Nincsen
 bennünk, de vagyon
 Éktelen, rút élet.
/9
#D68D286C
 Bujaság és torkosság,
 Megfojt a kevélység;
 Telhetetlen s átkozott,
 Izgat a fösvénység.
/10
#0B4B198E
 Mint tengernek fövénye
 Megszámlálhatatlan:
 Azonképpen mi bűnünk
 Nálunk tudhatatlan.
/11
#5E25E979
 Azért mi is magunkat
 Istenséged előtt,
 Bűnösöknek vádoljuk
 Mind e világ előtt.
/12
#48E5CE27
 Ne nézd mi bűneinket
 És gonoszságinkat,
 De tekintsd kegyelmesen
 Irgalmasságodat.
/13
#B318C7A9
 És el ne feledkezzél
 Te ígéretedről,
 A bűnösökhöz való
 Kegyelmességedről.
/14
#1976C458
 Azért néked könyörgünk,
 Felséges Úr Isten:
 Ne állj bosszút szertelen
 Mi gonosz bűnünkön.
/15
#769FEEF1
 De minékünk megbocsáss
 Te szent Fiad által,
 Hogy mi idvezülhessünk
 Ő irgalma által.
/16
#1E1FE239
 Hisszük, hogy meghallgattál
 Mi könyörgésünkben,
 Azért lelki örömmel
 Mi ezt mondjuk: Ámen.

>307. Ne szállj perbe énvelem
/1
#3B047858
 Ne szállj perbe énvelem,
 Ó, én édes Istenem!
 Mert meg nem igazul
 teelőtted lelkem, \.
 Elkárhoztathatsz engem.
/2
#97B337A7
 Mert én anyám méhében
 Fogantattam vétekben,
 E világra lettem eredendő bűnben,
 Kárhozatos esetben.
/3
#E979EE58
 Néked bűnöm megvallom,
 Mert én azzal tartozom;
 Te bocsáthatod meg,
 Bizonnyal jól tudom:
 Segítségedet várom.
/4
#BB4C7E2F
 Bűneimnek tőréből,
 Az ördögnek kezéből,
 Ments ki megérdemlett
 nagy veszedelmemből:
 Tekints reám az égből!
/5
#E6784953
 Szent Fiad haláláért,
 Keserves nagy kínjáért,
 Áldott szent vérének
 el-kifolyásáért:
 Kegyelmezz meg mindezért!
/6
#1F89DD1F
 Az ő tisztaságáért
 És ártatlanságáért,
 Szent kezein való
 sebeknek helyéért,
 töviskoronájáért!
/7
#1D897119
 Én fertelmességemet
 És oly nagy sok vétkemet,
 Mikkel megfertőztem
 undokul lelkemet,
 Rútítottam testemet:
/8
#A3CA3EE5
 Bocsásd meg, ó, Úr Isten!
 Ó, áldott Atya Isten!
 Kegyelemmel bőves,
 irgalmas jó Isten!
 Nagy türelmű szent Isten!
/9
#9DC91CAD
 A te áldott Szentlelked,
 Kérlek, tőlem el ne vedd,
 Sőt újítsd meg bennem,
 hogy dicsérjem neved,
 Szolgálhassak tenéked.
/10
#09416948
 Reád bíztam magamat,
 Te viseljed gondomat,
 Igazgasd jó útra az
 én lábaimat
 És minden szándékimat.
/11
#0DBC4016
 Vigasztald meg szívemet,
 Búban epedt lelkemet;
 Ne hányd szemeimre
 undok vétkeimet
 És cselekedetimet.
/12
#A15A74D9
 Csak tégedet dicsérlek,
 Míg e világon élek,
 Mert tudom, nyugalmat
 csak tenálad lelek,
 Mikor innét kikelek.
/13
#3AEAF090
 Mindörökké áldassál
 És felmagasztaltassál,
 Áldott Atya Isten,
 a te szent Fiaddal,
 Szentlélek áldásával!

>308. Istennek Báránya, Ki bűnünket elveszed
/1
#7C99CE3D
 Istennek Báránya,
 Ki bűnünket elveszed:
 Irgalmazz nékünk!
 Istennek Báránya,
 Ki bűnünket elveszed:
 Irgalmazz nékünk!
 Istennek Báránya,
 Ki bűnünket elveszed:
 Add ránk békességed!
 Á- men, Á- men.

>309a. Seregeknek hatalmas nagy királya
/1
#2D4B52EE
 Seregeknek hatalmas nagy királya,
 Könyörgésem székedet megtalálja,
 Mert szívemet sok bú állja,
 Olyan bűntől, mint megterhelt gálya.
/2
#74FF7AFA
 Ím, előtted megaláztam magamat,
 Földre hajtván szomorodott orcámat;
 Halld meg, Atyám, csendes szómat,
 Add meg szívből kívánt lelki jómat.
/3
#E1C7F355
 Ne vess el hát az én sok bűneimért,
 Ne is büntess háládatlanságomért,
 De sőt inkább szent Fiadért,
 Kegyelmezz meg érdemes kínjáért.
/4
#DD9E875E
 Kihez hajtsam búba merült fejemet?
 Ki tölti bé megsebhedett szívemet?
 Ha te is elhagysz engemet,
 Elveszek, ha nem szánod lelkemet.
/5
#7C1CC29B
 Az én lelkem reszket előtted állván,
 Mint a gyenge levél nyárfának ágán,
 Ellened tett bűnöm látván,
 Gyászban járván sír s kegyelmet kíván.
/6
#ECCE3418
 Gyakran szívből folyamodom elődbe,
 Könyörgésem hasson fel az egekbe;
 Uram, hadd jusson elődbe,
 Vétkeimet temesd a tengerbe.
/7
#9812ADED
 Örök Isten, felette irgalmas vagy,
 Megtérőkhöz kegyelmed is igen nagy;
 Engem, talpig bűnöst ne hagyj,
 Vigasztalást inkább szívemnek adj.
/8
#4C423BA9
 Engem bűnöst hiszen földből formáltál,
 Töredelmes cserép helyre állattál,
 Sokszor sok bűnért sujtoltál, \.
 Addig vertél, míg hozzád hajtottál.
/9
#5B0E21B3
 Reám vigyázz a szép felvont egekből,
 Dicsőséges királyi szent székedből;
 Halld meg, mert könyörgök szívből,
 Mosogass meg undok vétkeimből.
/10
#E31A5379
 Gyarló szívem magasztalja nevedet,
 Hogy pokolra nem eresztéd lelkemet;
 Megbocsátád vétkeimet,
 Melyért áldlak, édes Istenemet.

>309b. Seregeknek hatalmas nagy királya
/1
#0A73F036
 Seregeknek hatalmas nagy királya,
 Könyörgésem székedet megtalálja,
 Mert szívemet sok bú állja,
 Olyan bűntől, mint megterhelt gálya.

>310. Szelíd szemed, Úr Jézus
/1
#508F7C8D
 Szelíd szemed, Úr Jézus,
 Jól látja minden vétkemet,
 Személyemet ne vesse meg
 Szelíd szemed, Úr Jézus.
/2
#FF447090
 Szelíd szemed, Úr Jézus,
 Tekintsen rám, ha roskadok,
 Adjon békét, bocsánatot
 Szelíd szemed, Úr Jézus.
/3
#2DD3C923
 Szelíd szemed, Úr Jézus,
 Tudom, hogy vádat is emel;
 Vétkeztem én, ítéljen el
 Szelíd szemed, Úr Jézus.
/4
#2CF26DD8
 Szelíd szemed, Úr Jézus,
 Elítél bár, lásd, én megint
 Csak várom, hogy majd
 rám tekint Szelíd szemed, Úr Jézus.

>311. Tisztítsd meg szíved, Jeruzsálem népe!
/1
#A250F49F
 Tisztítsd meg szíved,
 Jeruzsálem népe!
 Hogy megtartassál,
 mosódj hófehérre!
 Karmazsin színű
 bűneidből tisztulj,
 Hogy ki ne pusztulj!
/2
#4DDF87DB
 Szeplőtelen légy a
 Krisztus napjára,
 Mert igen nagy
 volt váltságodnak ára:
 Ártatlan omlott
 Királyodnak vére;
 Térj meg kedvére!
/3
#D03A193D
 Nem járhatsz folyvást
 annyi undokságban,
 Részed így nem lesz
 mennyek országában;
 Vedd fel a harcot,
 vedd fel a keresztet,
 Vagy üdvöd veszted!
/4
#2F2ADC3B
 Élj botlás nélkül,
 szentül feddhetetlen,
 Vizsgáld a törvényt
 lelkiismeretben;
 Inkább fonj gyarló
 tested ellen ostort,
 Semhogy bemocskold!
/5
#CBEEE118
 Tisztítsd meg ajkad
 oltárról vett szénnel,
 Rútat ne illess, mit
 a lelked szégyell;
 Jót cselekedjél
 mindenekkel bőven,
 Minden időben!
/6
#9FD756DE
 Új teremtés vagy,
 vetkezd le a régit!
 Ördög ha üldöz,
 e világ ha rémít,
 Csak meg ne hátrálj,
 maradj meg az újban,
 Véled az Úr van!
/7
#6AA4B6B8
 Tisztítsd meg szíved,
 Krisztus szent egyháza,
 Őt temiattad pogány
 ne gyalázza!
 S hogy légy az
 Úrnak szent papsága,
 népe, Újulj meg végre!

>312. Uram, bűneink soksága
/1
#1D4FDCEC
 Uram, bűneink soksága,
 Undoksága
 Érdemli haragodat,
 Méltók vagyunk, hogy ellenünk,
 Szent Istenünk,
 Felemeld ostorodat.
/2
#54E5C918
 De tudjuk, hogy ki megvallja
 És megbánja
 Bűneit s hozzád megtér,
 Azt nem hajtod el előled,
 Sőt tetőled
 Bűnbocsánatot az nyér.
/3
#4B3A3DB5
 Azért hát mi is járulunk
 És borulunk
 Elődbe fájdalmasan,
 Bűnös lelkünknek
 kegyelmet, engedelmet
 Kérvén alázatosan.
/4
#D973F184
 Szánj meg, Uram, ily ügyünkben
 S megtértünkben
 Függeszd fel ostorodat.
 Ó, hajtson hozzánk
 békére Fiad vére,
 S felejtsd el haragodat.
/5
#A7ABC8B4
 Szólj hozzánk,
 Uram, csendesen
 És édesen;
 Félelmünk mindjárt széled,
 És lelkünk is e szózattól,
 Mint harmattól
 A hervadt virág, éled.
/6
#EC4123A0
 Uram, végy minket kedvedbe,
 Szerelmedbe,
 S vigasztald meg szívünket.
 Tőlünk soha ne maradj el,
 Se ne hagyj el,
 Hanem fogjad kezünket.

>Élet, halál, örök élet

>313. Az Úr Isten az én reménységem
/1
#798AF80A
 Az Úr Isten az én reménységem,
 Erősségem Mindenféle ínségben;
 Csak tőle várom Igaz boldogságom:
 S meg is találom.
/2
#95E8967B
 Benne élek, haláltól nem félek:
 Jót reménylek,
 Hogy tőle el nem térek;
 El nem enyészem
 A sírban egészen:
 Mennyben lesz részem.
/3
#BBDE9758
 Semmi engem tőle el nem választ,
 Jól tudván azt,
 Hogy sok jóval eláraszt;
 Erősít engem
 Erőtlenségemben
 És szükségemben.
/4
#F70A8D39
 Azért egész életem fogytáig
 Csodálom itt
 Szent kezének munkáit,
 S intem lelkemet:
 Áldjad Istenedet,
 Üdvözítődet.
/5
#BDDDF3E1
 Oltalmazzad,
 Uram, egyházadat,
 Szent nyájadat,
 mely vallja szent
 Fiadat, Ki bűneinkért
 Ártatlan bárányként
 Szenvedett sok kínt.
/6
#90309F01
 Hogy e földön szent gyülekezeted
 Dicséretet Zengvén,
 áldja nevedet,
 Míg szemtől szemben
 Magasztalunk mennyben
 Mind egyetemben.

>314. Az én időm, mint a szép nyár
/1
#564040E7
 Az én időm, mint a szép nyár,
 Menten lejár,
 Nem meszsze tőlem a vég.
 Ám a lélek el nem enyész,
 Sőt bére lész
 Jó vagy jaj: pokol vagy ég.
/2
#DCF55AB3
 Nem szükség hát
 veszteg ülnöm,
 Kell készülnöm:
 Égjen jól a szövétnek,
 Mert majd amaz
 öt szüzekkel, Mint ezekkel,
 Vélem is számot vetnek.
/3
#3DE76B91
 Ó, Uram, hová kell lennem,
 Ha kell mennem
 Veled, bírámmal szemben!
 Már ezt is alig állhatom,
 Ha forgatom
 Ezt előre eszemben.
/4
#98575433
 A kürtzengés máris hallik,
 Már hajnallik,
 Útban már az angyalok.
 Életem csak egy tenyérnyi,
 S számon kéri
 Jézusom, ha meghalok.
/5
#A83F25C4
 Taníts meg,
 Uram, hogy holtom
 S rövid voltom
 Soha el ne felejtsem,
 És a jövendő életet,
 Ítéletet
 Szívemből ki ne ejtsem.
/6
#999EA114
 Lelkem, mondj le hát e testről,
 Jóra restről;
 Egy úton ezzel futnál?
 Kérlek, vedd ezt jól szívedre:
 E vér vedre
 Majd eltörik egy kútnál.
/7
#62D4212A
 Segíts ezt megfeszítenem,
 Ó, Istenem!
 Magam nem bírok véle,
 Mert ha ma megöldöklöttem,
 Ura lettem:
 Holnap felkél új mérge.
/8
#158B21B1
 Uram, ha arra kell mennem,
 Hogy kell lennem
 Tanúnak a hit mellett:
 Láttass nyílt eget lelkemmel,
 Így testemmel
 Ám ne légyen kímélet.
/9
#519694B9
 Mindebből észbe vehetem:
 Harc életem,
 Sok ellenségim vernek,
 Lelkem sok ütközet vérzi,
 Ki nem érzi
 Vágtát ennyi fegyvernek?
/10
#6BDE3570
 Időm kevés, de sok a baj,
 Három a jaj,
 Ki ne kívánná végét?
 Boldog, aki pályát futott,
 Célra jutott
 S megtartotta hűségét.
/11
#73BFB53D
 Kezem én is feléd nyújtom,
 Szabadítóm,
 Jézusom, hozzád tartok!
 Bízzál, lelkem, nem süllyedsz itt,
 Kormány a hit,
 Várnak már a révpartok.

>315. Az élet nékem Krisztus, A halál nyereség
/1
#04C3DBF6
 Az élet nékem Krisztus,
 A halál nyereség.
 Ez az én reménységem,
 Kezdetté lesz a vég.
/2
#59625BCB
 Már indulok békében,
 Vár Krisztus, testvérem,
 És az ő közelében
 Szent célom elérem.
/3
#14F30F91
 Megküzdve életemnek
 Sok baját, ínségét,
 Meglátom Istenemnek
 Fényes dicsőségét.
/4
#0B87E195
 Ha erőtlenségemben
 Nem lesz szó ajkamon,
 Akkor is maradj vélem,
 Fogadd el sóhajom.
/5
#DD89ECA7
 Az életünk ellobban,
 Miként a gyertyaláng,
 A szívünk végsőt dobban,
 És éj borul reánk.
/6
#85374074
 Ám lelkem mégse bántsa
 A bú és félelem,
 Velem az Úr, és nála
 Üdvömet meglelem.

>316. Csak vándorút az életem
/1
#80CC0A04
 Csak vándorút az életem,
 Míg majd hazámba érkezem,
 Szent Jeruzsálem városába,
 Mit fönn az Isten készített,
 Szövetségvérre épített,
 Hol ajkam majd csak őt imádja;
 Csak vándorút az életem,
 Míg majd hazám elérhetem.
/2
#F6089B61
 Árván megyek az élten át,
 Nem ismer itt a vak világ;
 Ott várnak rám a hű testvérek;
 Ott vár az égi szent sereg;
 Ujjongva szolgálok neked,
 És örökké csak érted égek;
 Ó, Megváltóm, jövel, siess,
 Szívem csak tégedet keres.

>317. E világ mióta
/1
#5AD86163
 E világ mióta Fennáll,
 ő mivolta Sűrű változás;
 Minden dolga ínség,
 Bánat, keserűség
 És sóhajtozás;
 Nincsen benne állandóság:
 Minden hiábavalóság!
/2
#B3BABC9A
 Ó, ha meggondolná,
 Ember megfontolná
 Éltének végét!
 És minden dolgában
 Forgatná magában
 Nagy rövidségét:
 Bizony, e világ szépségét,
 Megutálná ékességét!
/3
#436E8492
 Nézd el: a virágok,
 Tavaszi újságok,
 Mint illatoznak! \.
 De ha ősznek dere
 Megüti, végtére
 Mind elváltoznak:
 Kórók lesznek, megavulnak,
 Hirtelenséggel elmúlnak.
/4
#6EE90F47
 Embernek is élte,
 Mint virág szépsége,
 Hamar elromlik;
 Rövid az ideje:
 Tíz, húsz esztendeje
 S azzal elmúlik;
 Akár gazda, akár szolga:
 Menni kell, ha üt az óra.
/5
#B8B14F46
 Nohát ezt felvégyed
 És szívedre tégyed:
 Rövid az élet!
 Légy mindenkor készen,
 Amíg meg nem lészen
 A nagy ítélet;
 Ne szabd magad e világhoz, \.
 Földi veszendő jószághoz!

>318. Én Istenem, benned bízom
/1
#9F1DA0ED
 Én Istenem, benned bízom,
 Se-gélj, ne hagyj tántorodnom!
 Lelkem, testem, minden tagom
 Reád bízom: Vezess, mert rád
 támaszkodom.
/2
#3A626AC1
 Nem tudom: itt meddig élek,
 Mikor ér életem véget?
 Valamikor tetszik neked:
 hozzád megyek,
 Akaratodnak engedek.
/3
#079249F8
 Minden órában kész lelkem,
 Hogy veled legyek, Istenem.
 Nem választok időt, órát,
 Hozzád jutást,
 Ha akarod, megyek mindjárt!
/4
#313CF0F7
 Testemnek minden tagjait
 És a hajamnak szálait
 Tudod, Uram, mert fejemről
 Még csak egy szál
 Kedved nélkül alá nem száll.
/5
#8323D75B
 Itt csak bánat, keserűség,
 Vagyon fájdalom, betegség;
 Életünk többnyire ínség,
 Kedvetlenség: még erőnk
 is erőtlenség.
/6
#8035D42C
 Nincs orvosság halál ellen
 Patikában vagy más helyen,
 Széles mezőn, drága kertben
 Oly fű nincsen,
 Haláltól amely megmentsen.
/7
#518D68EF
 Te azért, ó, én Istenem,
 Add énnékem ezt értenem,
 Hogy nyomorult mi életünk,
 Meg kell halnunk,
 E világból ki kell múlnunk.
/8
#5A5CB7D5
 A halálban biztatásom
 Nékem egyedül Jézusom,
 Ki énérettem szenvedett,
 Megfizetett:
 Szent Atyjának eleget tett.
/9
#2116CA3F
 Azért mikor, én Istenem,
 Akarod, vedd hozzád lelkem,
 Hogy veled örökké éljek,
 Reád nézzek,
 Az angyalokkal örvendjek.

>319. Jézus, én bizodalmam
/1
#142E9534
 Jézus, én bizodalmam
 És Megváltóm életemben,
 Benned van nyugodalmam.
 Nem kell semmitől rettegnem,
 Halál nagy éjszaká-ja,
 Bár ret-tentsen fú-lánk-ja.
/2
#5C26B071
 Jézus, én Megváltóm, él,
 Én is látom az életet:
 Lészek Idvezítőmnél,
 Immár semmi sem rettenthet:
 Ő a Fő, és nem hagyja,
 Hogy elvesszék egy tagja.
/3
#0E205F85
 Reménység kötelével
 Véle már összeköttettem,
 Erős hitem kezével
 Már őbelé helyheztettem;
 El nem szakaszt tőle már
 Sem élet, sem a halál.
/4
#A6D91FA7
 Földi porból vett porom
 Végre ismét porba tér meg,
 De föltámaszt egykoron
 Megváltóm, hogy véle éljek
 Végtelen dicsőségben:
 Nála lészek mennyégben.
/5
#B4D0D1A0
 Ami fáj itt és sóhajt,
 Az ott lészen dicsőséges;
 Földből szépen ott kihajt,
 Mi itt ínségekkel teljes.
 Erőtelenségemet,
 Ott letészem bűnömet.
/6
#A4E5505E
 Bátorságban légyetek:
 Jézus hordoz, mint övéit,
 Hát ne keseregjetek:
 Krisztus újonnan megépít.
 Angyalának szavára
 Felkeltek nemsokára.
/7
#2F748E1B
 Emeld fel hát lelkedet,
 hagyj el minden földi vágyat;
 Bízd rá arra szívedet,
 Kiből idvességed árad.
 Jézusnál tartsd kincsedet,
 Légyen Jézusé szíved!

>320. Krisztus, én életemnek Te vagy reménysége
/1
#DEBB608B
 Krisztus, én életemnek
 Te vagy reménysége,
 Szegény bűnös lelkemnek
 Örök üdvössége.
 Lészek hát én csendességben,
 Bár a halál fúlánkjával rettentsen.
/2
#73A88279
 Bátran éltem letészem,
 Mert jutalmát vészem,
 Elkészítve már nékem
 Királyi szent székem.
 Lészek hát én csendességben,
 Bár a halál fúlánkjával rettentsen.
/3
#28D5089F
 Megyek hát én örömmel
 Sion királyához;
 Ó, Jézusom, vezess el
 Égi szent Atyádhoz!
 Örvendj, szívem, repess,
 lelkem! Mert léssz mennyben
 angyali dicsőségben!

>321.Ments meg, Uram
/1
#901054C3
 Ments meg, Uram, engem
 az örök haláltól,
 Ama rettenetes napon
 minden bajtól!
 Midőn az ég és föld
 meg fognak indulni
 S eljössz a világot
 lángokban ítélni.
/2
#9C282557
 Reszket minden tagom,
 borzadok és félek,
 Én, e földön küzdő
 szegény bűnös lélek;
 Ama napon engem
 ítéletre vonnak,
 Midőn az ég és föld
 egyben megindulnak.
/3
#3EA02B15
 Haragnak napja az,
 kínok, ínség napja,
 Nagy nap, mely a
 bűnöst gyötri, szorongatja,
 Midőn te, ki mindig
 éltél és fogsz élni,
 Eljössz a világot
 lángokban ítélni.
/4
#A2F92A79
 Örök nyugodalmat adj,
 ó, Uram, nékünk,
 Örök világosság
 fényeskedjék nékünk,
 Hogy trónusod körül
 mi udvart állhassunk,
 S téged, boldogítót,
 örökké áldhassunk.
/5
#BFD57AA6
 Ments meg, Uram,
 engem az örök haláltól,
 Ama rettenetes napon
 minden bajtól,
 Midőn az ég és föld
 meg fognak indulni
 S eljössz a világot
 lángokban ítélni.

>322. Tudom, az én Megváltóm él
/1
#53F8FBD8
 Tudom, az én Megváltóm él,
 Hajléka készen vár reám,
 Már int felém, és gyermekének
 Koronát ád a harc után.
 Bár a világ gúnyol, nevet,
 Honvágy tölti el lelkemet,
 És nemsokára hív az Úr:
 Jövel haza, én gyermekem!
 Kitárt karjával vár az Úr:
 Pihenj, nyugodj a keblemen.
/2
#30508DB4
 Jézus nevében bízom én,
 Ő törli el sok bűnömet,
 Jézus ajkáról hallom én:
 Ó, jöjj haza, vár Mestered.
 Bár a világ gúnyol, nevet,
 Honvágy tölti el lelkemet,
 És nemsokára hív az Úr:
 Jövel haza, én gyermekem!
 Kitárt karjával vár az Úr:
 Pihenj, nyugodj a keblemen.
/3
#D152C3F2
 Előttem, ó, az oly csodás,
 Hogy értem szállt a földre le,
 Hogy érettem is szenvedett,
 Én bűnömért is véreze.
 Bár a világ gúnyol, nevet,
 Honvágy tölti el lelkemet,
 És nemsokára hív az Úr:
 Jövel haza, én gyermekem!
 Kitárt karjával vár az Úr:
 Pihenj, nyugodj a keblemen.
/4
#7FAB3720
 Tudom, hamar jő Mesterem,
 Az óra fut, a nap közel,
 Előtte állok csakhamar,
 Édes Jézus, jövel, jövel!
 Bár a világ gúnyol, nevet,
 Honvágy tölti el lelkemet,
 És nemsokára hív az Úr:
 Jövel haza, én gyermekem!
 Kitárt karjával vár az Úr:
 Pihenj, nyugodj a keblemen.

>323. Örök nagy hatalmú menynyei szent Isten
/1
#8553CC3A
 Örök nagy hatalmú
 menynyei szent Isten,
 Minden dolgaimban
 benned van reményem,
 Te vagy oltalmam,
 vigaszom nékem.
/2
#19CD0345
 Hogyha bánat árja
 szemeimre zúdul,
 Szívem fájdalmára
 balzsamot ád az Úr,
 S felszárad a könny,
 a seb begyógyul.
/3
#93BE06AC
 Kínok hogyha gyötrik
 lelkemet halálra,
 Sóhajom akkor is
 Istenemet áldja,
 Hiszen föltámaszt
 egy szebb világra.

>324. Utas vagyok e világban, Menny-or-szág-ban
/1
#1FA8520D
 Utas vagyok e világban,
 Menny-or-szág-ban
 Vár örök hazám készen;
 A testem csak lelkesült por
 És ha a sor Reá jön, porrá lészen.
/2
#DF7A96C2
 Minden nap hoz rám fájdalmat:
 Nyugodalmat
 Szívem sehol nem talál.
 Majd kár ér, majd búbánat sért,
 Majd bűn kísért,
 Végre eljön a halál.
/3
#ED8EEE19
 Uram, te látod végemet,
 Mert testemet
 S lelkemet te formáltad;
 Előbb, mint lettek napjaim,
 Hajszálaim
 Mind egyig megszámláltad.
/4
#7BD9FB4E
 Lelkemnek földi társától,
 Sátorától Bizonyos megválása;
 De csak annak, ki Istent fél,
 Kegyesen él,
 Lesz boldog kimúlása.
/5
#0012BF66
 Atyám, hogy meg ne rettenjek,
 S bátran menjek
 A minden test útjára,
 Teremts tiszta szívet bennem:
 Néked élnem
 Légyen lelkem fő vágya.
/6
#D8F95DB7
 Néked, napjaimnak Ura,
 Akár búra, Akár örömre juttatsz,
 Szent megnyugvással engedek;
 Bár szenvedek,
 Mindent jómra fordítasz.
/7
#F9F8B547
 Akármikor jön a halál,
 Készen talál,
 Fájdalmimnak vet véget;
 Bátran fogok vele kezet:
 Hozzád vezet
 S ád örök üdvösséget.

>Temetési énekek

>325. Jer, temessük el a testet
/1
#293F2729
 Jer, temessük el a testet,
 Melyről kétségünk nem lehet,
 Hogy az ítéletnek napján
 Fel fog támadni igazán.
/2
#624ACE0E
 Porból való eredete,
 Azért porrá kell lennie,
 De majd feltámad a sírból,
 Mihelyt az Úrnak szava szól.
/3
#91F513B7
 Az ő lelke örökké él
 A más világon Istennél,
 Ki szent Fiának általa
 Őt a bűntől megváltotta.
/4
#270CA8CC
 Lelke csendességben nyugszik,
 Teste a földben aluszik,
 Honnan ítélet napjára
 Feltámad nagy vigasságra.
/5
#D620DBC5
 Itt volt ő nagy félelemben,
 De ott lészen csendességben,
 Örökkévaló örömben
 És hatalmas fényességben.
/6
#B6D08C94
 Jer, hagyjuk itt őt aludni,
 Krisztus Jézusban nyugodni,
 És mi szüntelen vigyázzunk,
 Mert nékünk is meg kell halnunk.
/7
#4BB4E330
 Erre Krisztus adjon erőt,
 Ki vérével minket kivett
 A veszedelmes pokolból
 És kínból, örök halálból.
/8
#0A8CE4BD
 Ő mitőlünk dicsértessék,
 Örökké magasztaltassék
 Egyetemben az Atyával
 És Szentlélek Úr Istennel.

>326. Minden ember csak halandó
/1
#88936B94
 Minden ember csak halandó,
 Minden test, mint fű, virág;
 Itt ami van, mind romlandó,
 És elmúlik e világ.
 Porrá kell e testnek lenni,
 Hogyha el akarja venni
 Az örök dicsőséget,
 Melyet Isten készített.
/2
#AABDE10F
 Azért e testi életem,
 Ha jön a sír éjjele,
 Bátran s örömmel leteszem,
 Semmit sem vesztek vele;
 Mert a Krisztus drága vére
 Utat nyit egy dicsőbb létre;
 A halálban biztatóm
 Jézus, az én Megváltóm.
/3
#EF6CE77E
 Ki szakaszthat el őtőle?
 Enyém ő, s övé vagyok.
 Tudom, el nem vet előle,
 Ígéreti mert nagyok;
 Sőt felvisz engem az égbe,
 Dicsőültek seregébe,
 Hol az Istent meglátom
 És mindörökké áldom.
/4
#7E368C7A
 Ott van öröm s örök pálma,
 Hol sok ezren az égben,
 Isten trónja előtt állva
 Tündöklő fényességben,
 A dicső szent angyalokkal,
 Minden megboldogultakkal
 A Jézust magasztalják,
 Megtartójuknak vallják.
/5
#4C3ADC97
 Nagy keresztet
 Kik hordoznak \.
 S harcolják a hit harcát,
 Győzelemben vigadoznak
 S zengnek ott halléluját.
 Ott öröm s dicső korona
 Én fejemet körülfonja;
 Ott élem az életet,
 Melynek vége nem lehet.

>327. Seregeknek szent Istene
/1
#2C356619
 Seregeknek szent Istene,
 Menny-nek, föld-nek te-rem-tő-je,
 Jövel, jövel, én Krisztusom,
 Ne hagyj utolsó órámon!
/2
#8FA62F02
 Végy be, kérlek, kegyelmedbe,
 Idvezítő szerelmedbe,
 Jövel, jövel, én Krisztusom,
 Ne hagyj utolsó órámon!
/3
#10B89690
 Az egekre tégy méltóvá,
 Szent hitemben állandóvá;
 Jövel, jövel, én Krisztusom,
 Ne hagyj utolsó órámon!
/4
#04A89E9A
 Életemben ki szerettél,
 Jóvoltodban részeltettél:
 Jövel, jövel, én Krisztusom,
 Ne hagyj utolsó órámon!
/5
#42D47C8B
 Áldom azért szent nevedet,
 Hirdetem dicsőségedet:
 Jövel, jövel, én Krisztusom,
 Ne hagyj utolsó órámon!
/6
#3DB28639
 Testem nyugtasd meg a földbe',
 Lelkem vidd a magas égbe;
 Jövel, jövel, én Krisztusom,
 Ne hagyj utolsó órámon!

>328. Szomorú a halál a gyarló embernek
/1
#3D274EC5
 Szomorú a halál a gyarló embernek,
 Halál követétől mindenek rettegnek,
 Kiváltképpen kik e világban örülnek,
 Nehéz e világtól megválni ezeknek.
/2
#13CED13D
 Többet higgy a jégen rajzolt írásoknak,
 Mint a  csalárd világ sok biztatásának,
 Szép szín alatt közli kis részét javának,
 De semmit sem hihetsz állandóságának.
/3
#0EF27DDA
 Romlandó az élet s csak veszendő hívság,
 Hamar elenyészik, mint a gyenge virág,
 Minden ékessége csak nagy sanyarúság:
 Helyette vár reád a fényes mennyország.
/4
#9585C697
 Azért, gyarló világ, már maradj magadnak,
 Ám tartsd barátidnak, kik benned vigadnak,
 Tovább követője nem leszek utadnak,
 Nem leszek bús rabja mulandó javadnak.
/5
#E10C01B8
 Tudom, kinek hittem és kinek szolgáltam,
 Kit szeretett lelkem, kihez folyamodtam,
 Kit hívtam Uramnak, ki mellett harcoltam,
 Azért hűségemnek jutalmát találtam.
/6
#286CF8BA
 Elértem hitemnek egyetlenegy célját,
 Vitézkedésemnek reménylett pálmáját,
 Az örök életnek drága koronáját,
 Hogy abban tiszteljem a mennyek királyát.
/7
#D11D63CB
 Ó, édes Megváltóm, ne nézd bűneimet,
 Sok ellened való cselekedetimet,
 Idvességem árát: tekintsd érdemedet,
 Jövel, Jézus Krisztus, vedd hozzád lelkemet!
/8
#FF1437D2
 Igazságod szerint ne ítélj meg engem,
 Mert semmi kegyelmet nem talál érdemem,
 Nem lehet kívüled soha idvességem:
 Jövel, Jézus Krisztus, édes reménységem!
/9
#919CA5A5
 Elvégeztem immár pályafutásomat,
 E világon való zarándoklásomat,
 Megtartottam hitem s igaz vallásomat:
 Jövel, Jézus Krisztus, add meg koronámat!

>Könyörgések

>329. Adj már csendességet, lelki békességet
/1
#9E7DA431
 Adj már csendességet,
 lelki békességet, mennybéli
 Úr! Bujdosó elmémet,
 ódd bútól szívemet,
 kit sok kín fúr!
/2
#465BCB7F
 Sok ideje immár,
 hogy lelkem szomjan
 vár mentségére,
 Őrizd, ne hadd, ébreszd,
 haragod ne gerjeszd vesztségére!
/3
#EE53A871
 Nem kicsiny munkával,
 Fiad halálával váltottál meg,
 Kinek érdeméért most
 is szükségemet teljesítsd meg!
/4
#73ADDEDA
 Irgalmad nagysága,
 nem vétkem rútsága feljebb való;
 Irgalmad végtelen,
 de bűnöm éktelen s romlást valló.
/5
#E5090FAE
 Jó voltod változást,
 gazdagságod fogyást ereszthet-é?
 Engem, te szolgádat,
 mint régen sokakat, ébreszthet-é?
/6
#AC78830D
 Nem kell kételkednem,
 sőt jót reménylenem igéd szerint:
 Megadod kedvesen,
 mit ígérsz kegyesen, hitem szerint.
/7
#40FFE577
 Nyisd fel hát karodat,
 szentséges markodnak áldott zárját,
 Add meg életemnek,
 nyomorult fejemnek letört szárnyát!
/8
#C1D5E4F5
 Repülvén áldjalak,
 élvén imádjalak vétek nélkül;
 Kit jól gyakorolván,
 haljak meg nyugodván bú s kín nélkül!

>330a. Adj békességet, Úr Isten
/1
#8A640BE0
 Adj békességet, Úr Isten,
 A mi időnkben a földön,
 Mert nincsen nékünk több senki
 Bajvívónk és hadakozónk,
 Hanem csak te, Úr Isten.

>330b. Adj békességet, Úr Isten
/1
#6AFF2E92
 Adj békességet, Úr Isten!
 A mi időnkben e földön:
 Mert nincsen nékünk
 több bajvívónk
 És hadakozónk,
 Hanem csak te, Úr Isten!

>331. Egyedüli reményem
/1
#1C8F9790
 Egyedüli reményem,
 Ó, Isten, csak te vagy;
 Jövel és nézz meg engem,
 Magamra, ó, ne hagyj!
 Ne légy tőlem oly távol,
 Könyörülj hű szolgádon,
 Úr Isten, el ne hagyj!
/2
#270EAC64
 Ha a nehéz időkben
 Elcsügged a szívem,
 Vigasztalást igédben,
 Uram, te adj nekem!
 Ha kétség közt hányódom
 És mentségre nincs módom,
 Te tarts meg, Istenem!
/3
#F80A9FA2
 A földön ha elvesztem
 Szerelmem tárgyait,
 Maradjon meg mellettem
 Szerelmed és a hit;
 Csak azt el ne veszítsem,
 Mi benned, ó, Úr Isten,
 Remélni megtanít!
/4
#3829A4F3
 Földi jó és szerencse
 Mulandó, mint magunk,
 De a hit drága kincse
 Örök és fő javunk;
 Hitünk áll rendületlen,
 Hogy Isten véd szüntelen:
 Élünk vagy meghalunk.
/5
#A4DDAD48
 Uram, a nyomorultat,
 a gyöngét el ne hagyd,
 Az árvát, elhagyottat
 Gyámolítsd te magad!
 A szegényt, ki remélve
 Csak reád néz az égre:
 Úr Isten, el ne hagyd!

>332. Emlékezzél, Úr Isten, híveidről, Lelkitesti
/1
#1182353C
 Emlékezzél, Úr Isten, híveidről,
 Lelkitesti sokféle szükséginkről,
 Vi-selj gondot irgalmasságodból
 Különösen te szentegyházadról.
/2
#2AFA8335
 Adjad nékünk a kenyeret éltünkben,
 Amely táplál és erősít hitünkben,
 Nevekedést ád reménységünkben
 És megszentel lelkünkben-testünkben.
/3
#22D8E529
 Ne hagyj minket, Úr Isten, szomjúhoznunk,
 Az élő víz adassék most minékünk,
 Mely víz után soha nem szomjúzunk,
 Mert tebenned örökké vigadunk.
/4
#9D72B58A
 Akik hallják, Úr Isten, beszédedet,
 Ebből értik irántuk jó kedvedet,
 Szent Fiadért kegyelmességedet,
 Hogy közlöd vélünk örökségedet.
/5
#45FCA690
 Adjad nékünk most is te Szentlelkedet,
 Ne hallgassuk hiába szent Igédet,
 Sőt a szerint féljük szent nevedet,
 Magasztaljuk mindég Felségedet.
/6
#49F83096
 Dicsértessék már az Atya Úr Isten;
 Ő szent Igéje, a Fiú Úr Isten;
 Egyetemben Szentlélek Úr Isten:
 Szentháromság egy örök Úr Isten!

>333. Fohászkodom hozzád, Uram, Istenem!
/1
#B793DB90
 Fohászkodom hozzád, Uram, Istenem!
 Kérlek, kegyelmesen hallgass meg engem,
 Mert tebenned soha nem volt kétségem,
 Azért most is tehozzád esedezem.
/2
#380398B6
 Látod, Uram, igen megnyomorodtam.
 Előtted nagy nyavalyára jutottam,
 De míg te szent istenségedben bíztam,
 Soha semmiben el nem hagyattattam.
/3
#CE98CBCB
 Reménységem míg el nem fogyatkozott,
 A te ígéreted nálam nyilván volt,
 Hogy énnékem mind megadod azokat,
 Melyeket én szívem tőled óhajtott.
/4
#44A295EB
 Azért téged hívlak csak segítségre,
 És magamat nem is bízom senkire;
 Én lelkemet vigyed hálaadásra,
 És szívemet juttasd nagy vigasságra
/5
#CEE6CFC9
 Irgalmasságodat mikor hallhatom,
 Legott elfelejtem minden bánatom;
 Abban vagyon nékem nagy vigasságom,
 Bűneimnek bocsánatját hogy bírom.
/6
#EBA2BF8E
 Jelentsd nékem a te akaratodat,
 Fordítsd hozzám szent irgalmasságodat;
 Add meg nékem most, amit tőled várok,
 Melyért dicséretet örökké mondok.

>334. Hatalmas Isten! Népek közé szórva
/1
#465EF693
 Hatalmas Isten! Népek közé szórva,
 "Tebenned bíztunk eleitől fogva"!
 Mikor másunk nem volt, csak könny és zsoltár,
 Otthont keresve hajlékunk Te voltál.
 Szólj, ötágú síp, zengj, magyar ének,
 Adj hálát az ég nagy Istenének.
/2
#A3CFD262
 Bibliás ősök sírja fölött állva,
 Találjon e nép végre önmagára;
 Köszönjön meg mindent a „védő kar”-nak,
 S légyen testvére magyar a magyarnak!
 Szólj, ötágú síp, zengj, magyar ének,
 Adj hálát az ég nagy Istenének!
/3
#2768983D
 Kegyelmes Atyánk! Népek közé szórva,
 „Tebenned bíztunk eleitől fogva”!
 Add: éljünk egymásért és senki ellen,
 Bárhol a földön, de Isten-közelben!
 Szólj, ötágú síp, zengj, magyar ének,
 Adj hálát az ég nagy Istenének!

>335. Hallgasd meg, Jézus Krisztus
/1
#8B3CF8A3
 Hallgasd meg, Jézus Krisztus,
 Te megszomorodott
 S igen megkeseredett
 Szegény juhaidat,
 Hallgasd meg kegyelmesen
 A te szent egyházadat,
 Mely megnyomorodott.
/2
#199C6609
 Ne hagyd, édes Jézusunk,
 Nyomorult népedet,
 Szent véreddel megváltott
 Kicsiny seregedet;
 Ne hagyjad elpusztulni
 A te örökségedet:
 A keresztyénséget.
/3
#8EC2C680
 Kit a Sátán sokképpen
 Most megkörnyékezett,
 Sok fertelmes bűnökkel,
 És igen kísértget;
 Sok sanyarúságokkal
 És méreggel keserget,
 Szidalommal illet.
/4
#6033062C
 Siess, láss azért hozzánk,
 Kegyelmes Istenünk,
 El ne vess színed elől,
 Szerelmes jegyesünk!
 Ne hagyj el minket és ne
 Feledkezzél el rólunk:
 Messze ne menj tőlünk!
/5
#37DE7CD9
 Maradj meg, Uram, vélünk,
 Mert beestvéledik
 És immáron a nap is
 Majdan elnyugoszik;
 Nagy homály és setétség
 Ígyen majd következik:
 A bűn sokasodik.
/6
#C6204253
 Számtalan sok gonoszság
 Közöttünk megbővült,
 A te szent szereteted
 Igen meghidegült,
 És minden rendbéli nép
 Tőled elidegenült:
 A bűnben elmerült.
/7
#EF8ADFB6
 Hallgass meg azért minket
 És ments meg ezektől
 A pokolbéli ördög
 Nagy dühösségétől,
 Minden tévelygésektől,
 Oktalan, hamis hittől,
 Kétségbeeséstől.
/8
#7CD3CE59
 El ne távozzék tőlünk
 A Szentlélek Isten,
 Légyen ő gyámolítónk
 A mi életünkben;
 Tartsa meg szent
 Igédnek Szerelmét mi szívünkben:
 Hogy legyünk kedvében.
/9
#07CDE8CE
 Úgyszintén vezéreljen
 Minket igazságban,
 Tartson meg szent nevednek
 Erős vallásában,
 És vigasztaljon minket
 Minden háborúságban
 És nyomorúságban.
/10
#7EA225D2
 Hogy dicséretet mondjunk
 A te szent Atyádnak,
 Ővéle egyetemben
 Néked, mi Urunknak;
 Mindörökkön, örökké
 A Szentlélek Istennek,
 Mi megszentelőnknek.

>336. Hallgass meg minket, nagy Úr Isten
/1
#128F5AEB
 Hallgass meg minket, nagy Úr Isten,
 E mostani nagy szükségünkben,
 És tekints meg minket mi életünkben,
 Hogy ne essünk e földön hitetlenségbe,
 Ördög kezébe.
/2
#D33372D1
 Mert csak te vagy a világosság,
 Életünkben te vagy igazság,
 Mi setét szívünkben nagy világosság,
 És szomorú lelkünkben te vagy vigasság,
 Örök boldogság.
/3
#252F1BB6
 Távoztass tőlünk hamisságot,
 Add szívünkbe az igazságot,
 és a mi lelkünkbe nagy bátorságot,
 Hogy nyilván elhihessük a boldogságot:
 Örök országot.
/4
#F82B9D30
 Adj Szentlelket a tanítóknak,
 Egyetemben a hallgatóknak,
 Hogy mind engedhessünk akaratodnak,
 Dicséretet mondhassunk te szent
 Fiadnak, Jézus Krisztusnak.
/5
#5759093C
 Hála néked, mennybéli Isten,
 Ki vigasztalsz minket éltünkben,
 És el nem hagysz minket nagy szükségünkben,
 Megerősítesz inkább az igaz hitben,
 Ígéretedben.

>337. Mennybéli felséges Isten
/1
#B2192541
 Mennybéli felséges Isten,
 Kinek dicsőséged ott fenn
 Boldog lelkek seregitül
 Láttatik véghetetlenül:
 E teljes világ általad
 Teremtetett, áll és marad.
/2
#7F2FED69
 Te noha ily felséges vagy,
 Erőd, méltóságod ily nagy,
 Mégis minket, kik föld pora
 S hitvány férgek vagyunk,
 arra Méltóztatsz, hogy fiaidnak \.
 Hívassunk, s te MI ATYÁNKnak.
/3
#FD5DEB42
 SZENTELTESSÉK MEG TE NEVED,
 Azaz: mivelünk azt tegyed,
 Hogy igazán megismerjünk
 Téged, féljünk és tiszteljünk,
 Szemlélvén nagy bölcs munkáid
 S minden tökéletességid.
/4
#80E36FB6
 JÖJJÖN EL A TE ORSZÁGOD,
 Töltse bé uralkodásod,
 Ó, mi királyunk, e földet;
 Szaporítsad seregedet,
 Drága Igédnek kész szállást,
 Adj mindenütt szabad folyást.
/5
#051969E4
 Ha egyházad ellenségi
 Igyekeznek azt rontani:
 Te velünk egy táborba szállj,
 És előnkbe vezérül állj;
 Szégyenítsd ellenséginket,
 Tartsd meg, bírj, vezérelj minket.
/6
#1F6BCAF5
 LEGYEN A TE AKARATOD;
 Ami jó s rendes, te tudod;
 Azért mi akaratunkat
 Tetszésed szerint hordozzad,
 Hogy amit szeretsz: szeressük,
 Amit te gyűlölsz: gyűlöljük.
/7
#4594C915
 Engedjünk néked mindenben,
 E FÖLDÖN, MIKÉNT MENNYEKBEN:
 Ha velünk keményen bánsz is,
 Szenvedjük békével azt is;
 Tiéd mind testünk, mind lelkünk,
 Teremtőnk, szabad vagy velünk.
/8
#8DDA07CD
 ADD MEG NEKÜNK KENYERÜNKET NAPONKÉNT,
 eledelünket: Viselj gondot életünkről,
 E mi halandó testünkről;
 Szolgáltass jó egészséget,
 Termő időt, békességet.
/9
#E1EEE48D
 Szállíts áldást munkáinkra,
 Minden marhánk s jószágunkra;
 Javaiddal pedig nékünk
 Adj mértékletesen élnünk,
 Rád nézve háládatosan,
 Mások iránt irgalmasan.
/10
#EE4CCA96
 ÉS BOCSÁSD MEG VÉTKEINKET,
 Ne vonj ítéletre minket,
 És vétkeinkért meg ne feddj,
 Sőt mindenekben megengedj,
 AMINT MI IS MEGENGEDÜNK,
 HA KIK VÉTETTEK ELLENÜNK.
/11
#29DB449D
 NE VIGY MINKET KÍSÉRTÉSBE,
 Mely rajtunk erőt vehetne!
 Jól tudod, mily gyarlók vagyunk,
 Könnyen tántorodik lábunk;
 SZABADÍTS MEG A GONOSZTÓL,
 Ki lest hány nekünk akárhol.
/12
#469B7DD6
 Hogy tőled kérjük ezeket,
 Ily okok indítnak minket:
 MERT TIED AZ ORSZÁG:
 néped Vagyunk mi és örökséged,
 Jó királyunkhoz szükségben
 Hogyne folyamnánk merészen?
/13
#F11C19E9
 Velünk jól tenni akarván,
 Arra tehetséged is van,
 TIED A teljes HATALOM,
 Égi s földi birodalom;
 Jól tenni méltó is hozzád,
 DICSőSéG tér ebből reád.
/14
#ED511D28
 Aki nem könyörög hittel,
 Istentől az áldást nem nyer,
 De mi Krisztus által kérünk,
 Hozzád hittel énekelünk,
 Mennybéli Felséges Isten, \.
 Hallgasd meg imánkat, ÁMEN!

>338. Mikoron Dávid nagy búsultában
/1
#BB646285
 Mikoron Dávid nagy búsultában,
 Baráti miatt volna bánatban,
 Panaszolkodván nagy haragjában,
 Ilyen könyörgést kezde ő magában:
/2
#B6715BB6
 Istenem, Uram, kérlek tégedet:
 Fordítsad reám szent szemeidet;
 Nagy szükségemben ne hagyj engemet,
 Mert megemészti nagy bánat szívemet.
/3
#116CA353
 Csak sírok-rívok nagy nyavalyámban,
 Elfogyatkoztam gondolatimban,
 Megkeseredtem nagy búsultomban,
 Ellenségemre való haragomban.
/4
#542A3620
 Hogyha énnékem szárnyam lett volna,
 Mint a galamb, elrepültem volna;
 Hogyha az Isten engedte volna,
 Innét én régen elfutottam volna.
/5
#4E449A49
 Akarok inkább pusztában laknom,
 Vadon erdőben széjjel bujdosnom,
 Hogynem mint azok között lakoznom,
 Kik igazságot nem hagynak szólanom.
/6
#3FFF2857
 Egész e város rakva haraggal,
 Egymásra való nagy bosszúsággal;
 Elhíresedett a gonoszsággal,
 Hozzá fogható nincsen álnoksággal.
/7
#EC9669AE
 Gyakorta köztünk gyűlések vannak;
 Özvegyek, árvák nagy bosszút vallnak.
 Isten szavával ők nem gondolnak,
 Mert jószágukban felfuvalkodtanak.
/8
#F6AB47FD
 Én pedig, Uram, hozzád kiáltok,
 Reggel és délben s estve könyörgök;
 Megszabadulást tetőled várok,
 Az ellenségtől mert én igen tartok.
/9
#8ECAA0A4
 Te azért, lelkem, gondolatodat,
 Istenbe vessed bizodalmadat;
 Rólad elvészi minden terhedet
 És meghallgatja te könyörgésedet.
/10
#0261AD32
 Igaz vagy, Uram, ítéletedben:
 A vérszopókat ő idejükben
 Te meg nem áldod szerencséjükben;
 Hosszú életük nem lészen a földön.
/11
#C6B25FA1
 Az igazakat te mind megtartod,
 A kegyeseket megoltalmazod,
 A szegényeket felmagasztalod,
 A kevélyeket  aláhajigálod.
/12
#33AF1C85
 Ha egy kevéssé megkeseríted,
 Az égő tűzbe el-bétaszítod:
 Nagy hamarsággal onnét kivonszod,
 Nagy tisztességre ismét felemeled.
/13
#79C60BF1
 Szent Dávid írta a Zsoltárkönyvben,
 Ötvenötödik dicséretében,
 Melyből a hívek, keserűségben,
 Vigasztalásért szerzék így versekben.

>339. Lelki próbáimban, Jézus, légy velem
/1
#ECF03B86
 Lelki próbáimban,
 Jézus, légy velem,
 El ne tántorodjék tőled életem.
 Fé-lelem ha bánt, vagy nyereség kísért,
 Tőled elszakadnom ne hagyj sem-miért.
/2
#5BBD9E25
 űHa e világ bája engem hívogat,
 Nagy csalárdul kínál hitványságokat:
 Szemem elé állítsd szenvedésidet,
 Vérrel koronázott, szent keresztedet.
/3
#028DE5C5
 Tisztogass bár bajjal olykor engemet:
 Kegyelmeddel szenteld szenvedésemet;
 Bár e test erőtlen: te oltárodon
 Keserű pohárral, hittel áldozom.
/4
#AF630246
 Ha halálra válik testem egykoron:
 Ragyogjon fel lelked e hitvány poron;
 Ama végső harcon rád bízom magam:
 Örök hajlékodba fogadj be, Uram!

>340. Ne hagyj elesnem, felséges Isten
/1
#D6654D58
 Ne hagyj elesnem, felséges
 Isten, keserűségemben!
 Te szent Fiadért légy segítséggel:
 ne essem kétségben,
 Mert mindenfelől, látod,
 Úr Isten, Vagyok kísértetben.
/2
#F466073C
 Az írás rólad, Felséges Isten,
 bizonnyal azt mondja,
 Hogy valakinek tebenned
 vagyon szíve nyugodalma,
 Az olyan ember meg nem
 szégyenül, mert te vagy oltalma.
/3
#47B09284
 Gyermekségemtől fogva,
 Úr Isten, mind ez ideiglen
 Téged hívtalak én segítségül
 minden szükségemben,
 Mostan is nincsen több
 bizodalmam sem földön,
 sem mennyen.
/4
#862074FD
 Nincsen szívemnek több bizodalma,
 Úr Isten, náladnál,
 Valamíg gyötresz, szabad légy velem,
 csak ne haragudjál,
 Mint kegyes Atya, fiadat dorgálj,
 csak hogy meg nem utálj.
/5
#D6E6B5BC
 Csak te egyedül voltál,
 Istenem, énnékem gyámolom,
 Nagy fájdalmimban és
 romlásimban az én vigasztalóm;
 Ne hagyj elesnem s
 megszégyenülnöm,
 kegyes oltalmazóm!
/6
#EC6830A6
 Mely nagy örömem és
 bizodalmam vagyon nékem ebben:
 Hogy ígéreted, mint drága zálog,
 itt van én szívemben;
 Krisztus Jézusért engem
 meghallgaszt, tudom, kérésemben.
/7
#F2A7047D
 Jelentsd meg hozzám,
 felséges Isten, kegyelmességedet
 És véghetetlen, kegyes atyai
 te nagy szerelmedet,
 Hogy teljesítsd be
 könyörgésemre szent ígéretedet.
/8
#D9414E64
 Add meg, Úr Isten,
 te szent nevedért,
 amit tőled kérek,
 Szent Fiad által,
 teljes szívemből
 melyért most könyörgök,
 Mert csak tebenned bízom,
 Úr Isten, míg e testben élek.

>341. Ó, Atya Isten, irgalmas nagy Úr
/1
#D9DABD02
 Ó, Atya Isten, irgalmas nagy Úr,
 Bűnbánó szívvel ím eléd borul
 Hű néped, áldva felséges neved,
 Hogy esdve kérje nagy kegyelmedet.
/2
#32FCF17F
 Gondolsz ránk, híven oltalmaz kezed,
 Rólunk egy percre azt le nem veszed,
 Irgalmasságod mindig oly közel,
 És erős karod minket átölel.
/3
#BFAAEDC3
 Nagy jóságodra méltók nem vagyunk,
 Rossz útra térve gyakran elhagyunk;
 Áhítjuk mégis szent igéd szavát,
 Megtérő gyermekid fogadd be hát.
/4
#7122DF29
 Kérünk, Úr Isten, Krisztus-Jézusért,
 Vérrel pecsételt szent szerelmedért:
 Irgalmasságod közöld mivelünk,
 És tárd ki szíved, végy be, Istenünk!

>342. Irgalmazz, Úr Isten, immáron énnékem!
/1
#08E3D179
 Irgalmazz, Úr Isten,
 immáron énnékem!
 Irgalmazz, Úr Isten,
 immáron énnékem,
 Mert tebenned bízik,
 Uram, az én lelkem,
 És tebenned nyugszik,
 Uram, az én szívem.
/2
#E1537EDE
 Szárnyad alá vetem
 az én reménységim,
 Míg elmúlnak tőlem
 az én ellenségim,
 És míg eltávoznak
 tőlem én bűneim:
 Csak tebenned lésznek,
 Uram, én örömim.
/3
#0C9F88D7
 Tehozzád kiáltok,
 hatalmas Úr Isten,
 Mert nincsen, énvélem
 ki már jót tehessen,
 Én ellenségimtől
 engem megmenthessen,
 És én dolgaimban
 ki jóra vihessen.
/4
#67E97207
 Velem, én szent Atyám,
 nagy-sok jókat tettél,
 Mert énnékem mennyből
 őrizőt küldöttél,
 Én ellenségimtől engem
 megmentettél
 És az én szívemben
 örömöt szerzettél.
/5
#C9BD8B8F
 Azért téged, Uram,
 én felmagasztallak,
 Te dicsőségedben
 hatalmasnak mondlak,
 Mind e világ előtt
 irgalmasnak vallak;
 És jóakarómnak
 én tégedet hívlak.
/6
#B99511C9
 Kész már az én szívem
 néked énekelni,
 Kész most jóvoltodért
 nagy hálákat adni,
 És mindenek előtt
 téged megvallani,
 A te szent nevedet
 örökké dicsérni.
/7
#78F1FE03
 Kelj fel azért mostan,
 én nagy dicsőségem,
 Légy mindenben nékem
 kedves segítségem,
 Erőtlenségemben légy
 én erősségem
 És veszedelmemben
 légy oltalmam nékem.
/8
#33DD0228
 Én felmagasztalom
 irgalmasságodat,
 Mindenkoron vallom
 te igazságodat,
 Mind ez egész földön
 hatalmasságodat:
 Mindenkor hirdetem
 a te jó voltodat.

>343. Mi Atyánk, ó, kegyes Isten
/1
#DA6F5183
 Mi Atyánk, ó, kegyes Isten,
 Ki vagy a magas menynyekben:
 Szenteltessék neved szívvel;
 A te országod jöjjön el;
 Te akaratod meglégyen
 Ez földön, miképpen mennyben.
/2
#49179260
 Mindennapi kenyerünket
 Add meg, bocsásd meg vétkünket,
 Amint mi is megbocsátunk
 Azoknak, kiktől bántattunk;
 És ne vigy a kísértetbe,
 De szabadíts gonosz ellen.
/3
#B967B7B2
 Mert tied, Uram, az ország,
 Tied minden hatalmasság,
 Megadhatsz azért minékünk
 Mindent, amit tőled kérünk;
 Tied a dicsőség, Ámen.
 Most és örökké úgy légyen.

>344. Ó, könyörgést meghallgató
/1
#EA15DFB9
 Ó, könyörgést meghallgató
 Édes Atya, mindenható,
 Ha lelkem hozzád emelkedik,
 És a buzgóság szárnyain,
 Ajakim óhajtásain
 Elődbe érvén, reménykedik:
 Érzem, hogy az örök élet
 Már e földön az enyém lett.
/2
#61163D1D
 Ha örömmel gerjedezem
 És rebegni igyekezem,
 Tetőled mennyi áldást vészek!
 Feltekintvén rád, Atyámra,
 Könny csordul már az orcámra;
 Azáltal olyan újjá lészek,
 Mint a plánta, mikor arra
 Harmatcseppet szülsz hajnalra.
/3
#02DDD618
 Ha szívemet bánat járja,
 Szemem keserűség árja:
 Előtted való zokogásom;
 Titkon ajtómat behajtva
 És magánosan sóhajtva
 Akkor is édes újulásom,
 Mert minden könnyűvel, jajjal
 Könnyebbül sorsom egy bajjal.
/4
#D8ABAB53
 Ha templomban megjelenek,
 Ahol összesereglenek,
 Felséges neved imádói:
 Velük együtt fohászkodva
 Úgy tetszik, mintha vigadva
 Ott volnék, hol a menny lakói
 A te királyi székednek
 Előtte letelepednek.
/5
#7C7304F7
 Te, ki a szív mozdulásit,
 Mint a vizeknek folyásit,
 Szabadon hajtod ide s tova,
 Neveld ezt az érzést bennem,
 És taníts jól könyörgenem;
 Majd egyszer pedig,
 Mennyek ura,
 Vigy be az egek egébe, \.
 Imádásod szent helyébe.

>345. Ó, irgalmas Isten, Én könyörgésemben
/1
#34F5CA9F
 Ó, irgalmas Isten,
 Én könyörgésemben
 Füledet hozzám hajtsad;
 Ó, igen jó Isten,
 Minden szükségemben
 Áldásod szaporítsad.
/2
#AE06B331
 Ó, hatalmas Isten,
 Keserűségemben
 Szívemet vidámítsad;
 Ó, nagy és szent Isten,
 Minden félelmemben
 Elmémet bátorítsad.
/3
#2CC7FEEC
 Ó, örök Úr Isten,
 A veszedelmekben
 Segedelmedet nyújtsad;
 Ó, igaz Úr Isten,
 Kételkedésemben
 Hitemet gyámolítsad.
/4
#D1C8934C
 Ó, áldott Úr Isten:
 Rossz testiségemben
 Lelkem hozzád hódítsad;
 Ó, erős Úr Isten,
 A világ fényében \.
 Szemem világosítsad.
/5
#99119D5A
 Ó, teremtő Isten,
 A kísértetekben
 A Sátánt elfordítsad;
 Ó, megváltó Isten,
 sok vétkezésimben
 Irgalmad bizonyítsad.
/6
#0B59230D
 Ó, szentelő Isten,
 Erőtlenségimben
 Kegyelmedet újítsad;
 Ó, kegyelmes Isten,
 Éltemnek végében
 Lelkemet boldogítsad.

>346. Isten, áldd meg a magyart
/1
#F33076DE
 Isten, áldd meg a magyart
 Jó kedvvel, bőséggel,
 Nyújts feléje védő kart,
 Ha küzd ellenséggel.
 Balsors akit régen tép,
 Hozz rá víg esztendőt,
 Megbűn-hőd-te már e nép
 A mul-tat s jövendőt.
/2
#DC0FAAC5
 Őseinket felhozád
 Kárpát szent bércére,
 Általad nyert szép hazát
 Bendegúznak vére.
 S merre zúgnak habjai
 Tiszának, Dunának,
 Árpád hős magzatjai
 Felvirágozának.
/3
#CC1DA78A
 Értünk Kúnság mezein
 Ért kalászt lengettél,
 Tokaj szőlővesszein
 Nektárt csepegtettél.
 Zászlónk gyakran plántálád
 Vad török sáncára,
 S nyögte Mátyás bús hadát
 Bécsnek büszke vára.
/4
#BC82C42A
 Hajh, de bűneink miatt
 Gyúlt harag kebledben,
 S elsújtád villámidat
 Dörgő fellegedben,
 Most rabló mongol nyilát
 Zúgattad felettünk,
 Majd töröktől rabigát
 Vállainkra vettünk.
/5
#C956B676
 Hányszor zengett ajkain
 Ozmán vad népének
 Vert hadunk csonthalmain
 Győzedelmi ének!
 Hányszor támadt tenfiad
 Szép hazám, kebledre,
 S lettél magzatod miatt
 Magzatod hamvvedre?
/6
#F4F50A78
 Bújt az üldözött, s felé
 Kard nyúl barlangjában,
 Szerte nézett s nem lelé
 Honját a hazában,
 Bércre hág és völgybe száll,
 Bú s kétség mellette,
 Vérözön lábainál,
 S lángtenger felette.
/7
#38B4B941
 Vár állott: most kőhalom;
 Kedv s öröm röpkedtek:
 Halálhörgés, siralom
 Zajlik már helyettek.
 S ah, szabadság nem virul
 A holtnak véréből,
 Kínzó rabság könnye hull
 Árvák hő szeméből.
/8
#9EBC35C6
 Szánd meg, Isten, a magyart,
 Kit vészek hányának,
 Nyújts feléje védő kart
 Tengerén kínjának,
 Balsors akit régen tép,
 Hozz rá víg esztendőt,
 megbűnhödte már e nép
 A múltat s jövendőt!

>347. Ó, áldandó Szentháromság!
/1
#BD7F667D
 Ó, áldandó Szentháromság!
 Nyögésemre figyelmezz.
 Jövel, siess, Főboldogság,
 Én Istenem, kegyelmezz!
 Ó, Uram, Uram, ne hagyj!
 Mert reménységem te vagy.
/2
#744AF701
 Ím, körülvett már az ínség,
 Minden bűnöm rám tódult,
 Szorongat a végső szükség,
 Fetrengek, mint egy bódult.
 Ó, szerelmes szent Atyám!
 Légy nékem erős bástyám.
/3
#8C52CDA7
 Ölelgess szent szerelmeddel,
 Örök szövetségedből,
 Takargass be védelmeddel,
 Adj részt örökségedből:
 Nézz régi irgalmadra \.
 S egyetlenegy Fiadra.
/4
#8AC0A6DB
 Ó, egyetlenegy segítőm,
 Uram Jézus, irgalmazz!
 Légy vélem, én Idvezítőm,
 A gonosztól oltalmazz!
 Hiszen te értem jöttél,
 S értem is vért öntöttél.
/5
#41237550
 Gyógyítsd lelkem betegségét
 Véres verejtékeddel,
 A halálnak édességét
 Újítsd drága véreddel,
 Mert csak az lelkem bére,
 Más érdem álljon félre.
/6
#6B4F4304
 Ó, Isten Báránya, kedvezz,
 Mert most vagyon az óra;
 Mondd lelkemnek:
 Légy idvez, Dolga forduljon jóra.
 Ó, Úr Jézus, könyörülj,
 Könyvedből ki ne törülj!
/7
#4C5FB8D0
 Ó, vigasztaló Szentlélek,
 Jövel segítségemre,
 És míg itt mozgok és élek,
 Készíts idvességemre.
 Tartsd meg bennem a hitet,
 Mit kegyelmed készített.
/8
#416DE12A
 Bátoríts a halál ellen,
 És ha már elaluszom,
 Hogy a Sátán el ne nyeljen,
 Te légy, Jézus, végső szóm.
 Maradj velem mindvégig:
 kísértess bé az égig.
/9
#F40E416B
 Ó, Úr Jézus, légy Jézusom:
 Légy irgalmas lelkemnek!
 Ó, vedd hozzád egy orvosom,
 Míg megadod testemnek,
 Ó, Jézus, tőlem ne fuss!
 Ó, jövel, Uram Jézus!

>Isten dicsérete

>348. Áldjad, én lelkem, a dicsőség
/1
#0FF2444A
 Áldjad, én lelkem, a dicsőség erős királyát!
 Őnéki menynyei karokkal együtt zengj hálát!
 Zúgó harang, Ének és orgonahang,
 Mind az ő szent nevét áldják!
/2
#7C27F0BA
 Áldjad Őt, mert az Úr mindent oly szépen intézett!
 Sasszárnyon hordozott, vezérelt, bajodban védett.
 Nagy irgalmát Naponként tölti ki rád:
 Áldását mindenben érzed.
/3
#F1030D69
 Áldjad Őt, mert csodaképpen megalkotott téged,
 Elkísér utadon, tőle van testi épséged.
 Sok baj között Erőd volt és örömöd:
 Szárnyával takarva védett.
/4
#C4485FFE
 Áldjad Őt, mert az Úr megáldja minden munkádat,
 Hűsége, mint az ég harmatja, bőven rád árad.
 Lásd: mit tehet Jóságos Lelke veled,
 És hited tőle mit várhat.
/5
#AFA37977
 Áldjad az Úr nevét, Őt áldja minden énbennem!
 Őt áldjad, lelkem, és Róla tégy hitvallást, nyelvem!
 El ne feledd: Napfényed Ő teneked!
 Őt áldjad örökké! Ámen.

>349. A csillagos égnek seregei ott fent
/1
#3CE62ED4
 A csillagos égnek seregei ott fent
 Tündöklő fényükkel dicsérik az Istent.
 Az egek boltjai alatt forgó világok
 Mély bölcsességéről beszélő bizonyságok;
 Az ő szavára lett tenger, hegy, füvek és fák
 Jóvoltát erejét szüntelenül kiáltják.
/2
#EC9246A0
 Hát én, kibe lelket a jó Teremtő tett,
 Némán vesztegeljek s ne dicsérjem őtet?
 Nem, sőt minden lelki erőmet összeszedem,
 Elmémnek szárnyain székihez emelkedem;
 Szólok, s ha tétováz rebegése nyelvemnek,
 Könnyeim tanúi lesznek tiszteletemnek.
/3
#6810DB5E
 Nyelvem rebeg, de te, ki szívem vizsgálod,
 Abban oltárodat füstölve találod.
 Én, ha a napfénybe mártanám is ecsetem,
 Felséges voltodnak árnyékát sem festhetem;
 A te tisztelőid bár legyenek angyalok,
 Csak erőtlen hangon dicsértetel általok.
/4
#71D4E13E
 Ki ruházott fényt a sok ezer napokra,
 Színes sugárokat rakván orcájokra?
 Ki indítá útnak a számtalan földeket?
 Ki kötött egymáshoz ennyi ezer testeket?
 Kimért útjaikon ezeket ki mozgatja?
 Uram, a te szádnak mindenható szózatja.
/5
#D6AD02DA
 Te tartod a zsengés tavasznak kezeit,
 Míg kiteregeti színes szőnyegeit;
 Aranyszín ruhát adsz az ért gabonafőre,
 Fürtökből koronát tűzöl a szőlőtőre,
 A megfagyott földet hópelyhekbe pólyálod,
 Lakosit élesztő örömökkel kínálod.
/6
#5A651682
 Uram, ezer nyelvek kellenének számba,
 Jóságod csudáit hogy vehessem számba;
 Még ami gonosz is, használ nékünk más részbe,
 Istennel nem tartók, ezt vegyétek jól észbe,
 És ha nem vagytok megilletve jóvoltától,
 Féljetek, mint szökött rabszolgák, haragjától!
/7
#7C170146
 Felelj, kételkedő: ki mennydörög ott fent?
 A szélvész harsogó zúgása kit jelent?
 A sík tenger vizét ki emeli hegyekké?
 A szárazföld színét ki süllyeszti völgyekké?
 Mindegyik éltető elem ím, felemelte
 Szavát, hogy van Isten: hát mit kételkedel te?
/8
#2A954272
 Uram, munkáidról lészen dicséretem,
 Míg tart az e végre adatott életem.
 Kérlek, vedd jó néven, ha egy gyarló féregnek
 Ajkai töredék dicséretet rebegnek!
 Te látod, mily buzgók bennem az indulatok,
 Melyeket jól érzek, ki nem magyarázhatok.
/9
#EDF1FB13
 Ó, ha kivetkezvén e testből idővel,
 Állok széked előtt koronázott fővel,
 Több lesz énekemben az erő és méltóság,
 Magasztalván téged, ó, felséges valóság.
 Röpüljetek elő, kívánt idők, sebesen,
 Hogy örömeimben több változás ne essen!

>350. Zengjen hálaének, Minden ajkon zengjen!
/1
#EE9CE361
 Zengjen hálaének,
 Minden ajkon zengjen!
 Isten szent nevének
 Hő imát rebegjen
 Mindaz, aki tudja,
 Hogy az Úr kegyelme
 Minden bánatunkba'
 Gyógyírt hoz sebünkre!
/2
#A7ECD81C
 Isten szent szerelmét
 Jézus hozta hozzánk,
 Hogyne zengne mindig
 Róla szívünk és szánk?
 Ó, mi nagy dicsőség:
 Jézus a mi éltünk!
 Ó, csudálja föld s ég:
 Isten szenved értünk!
/3
#7E0B867E
 Ha a Bárány vérét
 Isten rajtunk látja,
 Életünk sok vétkét
 Szíve megbocsátja.
 Üdvösségünk árát
 Rejti Jézus vére;
 Áldjuk hát jóságát,
 Dicsőség nevére!

>351. Dicsértessék, Uram, Örökké szent neved
/1
#41C288E8
 Dicsértessék, Uram,
 Örökké szent neved,
 Hogy ez ideiglen
 Gondomat viselted.
 Méltatlanságomat,
 Úr Isten, nem nézted,
 Hogy eddig megtartál,
 Hála legyen néked.
/2
#32C53365
 Irgalmasságodnak
 Csudálatos volta,
 Emberi elme azt
 Soha meg nem fogja,
 Te bőséges kezed
 Táplál szüntelenül,
 Sok jót adsz gyakorta
 Reménységen felül.
/3
#BEC37312
 Oly nagy gondod vagyon
 A mi életünkre,
 Életünkben való
 Minden szükségünkre:
 Mikor elgondolnánk,
 Még annak előtte
 Felségednél dolgunk
 El vagyon rendelve.
/4
#52066036
 Gondolkodtam sokat
 A régi Atyákról,
 E világon való
 Sok bujdosásokról:
 Mely nagy gondot,
 Uram, Viseltél azokról,
 Még javaikról is,
 Nemcsak önmagokról.
/5
#90EF9CA5
 Jól tudom, mostan is
 Te azon Isten vagy,
 Irgalmasságod is
 Szintén oly igen nagy,
 Ki soha nem hagytad
 A benned bízókat,
 Mostan azért, Uram,
 Hallgassad meg szómat.
/6
#4166A961
 Benned bíztam, Uram,
 Mindenkor csak benned,
 Csak tetőled vártam
 Erőt, segedelmet,
 Teljes életemben
 Tebenned reméltem,
 Boldogulást, áldást
 Fölségedtől kértem.
/7
#F0F744D3
 Tekints reám azért
 Kegyelmességedben,
 Viselj gondot rólam
 Én nagy szükségemben.
 Ne nézd bűneimet,
 Én édes Istenem,
 Mutasd meg, hogy te vagy
 Gondviselőm nekem.
/8
#2F8064D4
 Bűneimet pedig
 Bocsássad meg nékem,
 Melyekkel megbántlak
 Naponként, Istenem.
 A földi gyarlóság
 Bűnbe viszi lelkem,
 De Te szent nevedért
 Légy kegyelmes nekem.
/9
#D05BCF55
 E földön, míg élek,
 Adj jó egészséget,
 Alázatos szívet,
 Lelki csendességet.
 Áldj meg én munkámban
 Kívánt jó sikerrel,
 S végre koronázz meg
 Örök üdvösséggel!

>352. Dicsérlek téged, Idvezítőmet
/1
#9A72F35D
 Dicsérlek téged, Idvezítőmet,
 Úr Jézus Krisztus,
 Istennek egy Fia!
 Megváltott nyájadnak
 hű atyjafia:
 Csak téged várlak,
 egy segítőmet.
/2
#1CA984DA
 Tenéked minden
 térdet s fejet hajt,
 Mert erős torony
 nékünk a te neved;
 Felvette szegény
 ügyünket jó kedved,
 A kárhozatban
 hogy ne lássunk bajt.
/3
#80976F1D
 Te vagy Istennek szelíd Báránya,
 Ki e világnak bűneit elvetted,
 Szent véred omlását nem kíméletted,
 Mellyel nem ér fel Ofir aranya.
/4
#7503CBF7
 Ó, örök Isten, lelkem reménye,
 Az Atya Istennek fő bölcsessége,
 Szentegyházadnak ő nagy dicsősége,
 Minden híveknek ragyogó fénye!
/5
#C705DB41
 Dicsérlek, mennyből leszállott
 Kenyér, Isten s Ember,
 Jézus, egy Közbenjáró!
 Bűnösök térését kegyesen váró,
 Kitől kegyelmet minden lélek nyér.
/6
#CFA87341
 Mely buzgó, Uram, a te szerelmed,
 Ki mi-érettünk magad megáldoztad,
 A halált, Sátánt, poklot megrontottad,
 Most is fenntartván hű segedelmed.

>353. Dicsőült helyeken
/1
#37100382
 Dicsőült helyeken,
 menynyei paradicsomban,
 Akik vigadoztok véghetetlen
 boldogságban,
 Szent és ártatlan állapotban;
 Az Úrnak nevét énekszóban,
 Magasztaljátok vigasságban.
/2
#E87F5AF2
 űLássátok, ti dicső angyali lelkek az égben,
 A fénylő mennyei testek roppant seregében:
 Mint uralkodik dicsőségben
 Az Úr, és őtet felségében
 Imádjátok tiszta szentségben!
/3
#92049ECB
 Nap, hold, csillagos ég!
 Ti az Urat dicséritek,
 Jóságát, hatalmát, bölcsességét hirdetitek;
 Mert amely rendet ő tinéktek
 Kiszabott, az a ti törvénytek
 És ti annak híven engedtek.
/4
#A0CC2FD7
 Felhők, égi vizek, harmat,
 hó, jég és záporok!
 Villám, égi tüzek, dörgések és csattogások!
 Szélvész és csendes fuvallások!
 Színekkel ékes szivárványok!
 Mind az Úrnak szolgái vagytok.
/5
#E34CCB78
 A földet kinyitó kies tavasz vidámsága,
 A mindent érlelő nyári idő forrósága,
 A gyümölcsös ősz gazdagsága,
 Sőt még a tél is bizonysága:
 Hogy nagy a teremtő jósága!
/6
#B8627063
 Tenger, föld, szigetek s a hegyek pompás szépsége,
 Erdők és ligetek, folyóvizek kiessége,
 Sík mezők, rétek ékessége,
 Vetések kellemetessége:
 Mind a teremtő dicsősége!
/7
#BFEEC022
 Fák, bokrok, csemeték, gyümölcsöket hozó kertek,
 Pompás liliomok s minden virági szépségek,
 Színekkel kihímezett rétek,
 Apróbb és nagyobb növevények:
 A gondviselést hirdessétek.
/8
#54BBB00B
 E földön lakozó minden állat boldogsága,
 A szárnyon repdeső madaraknak vidámsága,
 A halak s férgek sokasága,
 Egyszóval az élők világa
 Mondja: mily nagy az Úr jósága!
/9
#3B20BA4D
 Bölcs Isten remeki!
 Kik az ő képei vagytok
 És őtőle lehelt emberi lélekkel bírtok:
 Az örökkévalót lássátok
 Munkáiban és imádjátok;
 Egy szívvel és szájjal mondjátok:
/10
#4AF718CD
 Dicséret, dicsőség, tisztesség és hálaadás,
 A szentek Urának légyen örök magasztalás!
 Kiben soha nincs megváltozás,
 Vagy ígérettől elhanyatlás:
 Tőle fejünkre szálljon áldás!

>354. Megáll az Istennek Igéje
/1
#11CBB8FA
 Megáll az Istennek Igéje,
 És nem állhat senki ellene,
 A nagy Isten vagyon mivelünk,
 És Szentlelke lakozik bennünk.
 A nagy Isten vagyon mivelünk,
 És Szentlelke lakozik bennünk.

>355. Az Úr Istent magasztalom
/1
#02BE0929
 Az Úr Istent magasztalom,
 Jó voltáról emlékezem,
 Mindig hozzá folyamodom,
 Mert meghallgat, azt jól tudom.
/2
#F5CCEA92
 Számtalan kínokban valék,
 Bűneimért kit érdemlék:
 De ismét megkönnyebbülék,
 Mihelyt vigasztalást hallék.
/3
#9BD96C10
 Elterjesztvén kezeimet,
 Kiáltottam Istenemet,
 Csak bé sem hunytam szememet;
 Nem leltem sehol helyemet.
/4
#DE4F5883
 Magamban én dolgaimról,
 Elébbeni életemről,
 Nagy kedves nyájasságomról:
 Gondolkodtam énekimről.
/5
#1E31FAEB
 Nagy erősen, igaz hittel,
 Magam Isten beszédével,
 Kegyelmes ígéretével,
 Bátorítám Szentlelkével.
/6
#C10A2D97
 Csuda irgalmasságodat,
 Hiszem, Uram, hatalmadat,
 Onnan vészen bizodalmat,
 Én lelkem minden oltalmat.
/7
#22B870AF
 Kezeidnek karjaiban,
 Élet, halál birtokodban,
 Megmutatád, hogy markodban,
 Vagyon minden oltalmadban.
/8
#25EFF16F
 Erős vitéz mint népeit,
 Az ember ő két szemeit,
 Mint jó pásztor ő juhait:
 Úgy oltalmazza híveit.
/9
#7688C117
 A híveknek számuk vagyon,
 Nevük nála írva vagyon,
 Hajuk szála számon vagyon:
 Rájuk gondja van oly nagyon.
/10
#7EC7B1AD
 Ne félj azért háborúdban,
 Ó, én lelkem, nyavalyádban;
 Erős légy bizodalmadban,
 Mert vagy Isten oltalmában.

>356. Drága dolog az Úr Istent dicsérni
/1
#718EB45C
 Drága dolog az Úr Istent dicsérni,
 Színe előtt kegyesen énekelni,
 Ékes dolog szent nevét magasztalni,
 A híveknek őelőtte szolgálni.
/2
#5685452B
 Ő gyógyít meg szomorodott szíveket,
 Vigasztalja az elalélt lelkeket,
 Ő enyhít meg sok keserűséget,
 Mind testi, mind lelki betegségeket.
/3
#23D37AD5
 Az őnéki csodálatos tanácsa,
 Hogy a szelídeket felmagasztalja,
 De a kevély gonoszokat utálja,
 És azokat a földig megalázza.
/4
#732BD160
 Énekeljünk néki hálaadással,
 Vigasságos hangicsáló szerszámmal:
 Beszélgessünk e mi kegyes Urunkkal,
 Magasztalván őtet imádságunkkal.
/5
#008C47F8
 Dicsérd azért, Jeruzsálem, Uradat!
 Keresztyénség, magasztald
 Megváltódat! Áldjad,
 Sion, mennyei királyodat!
 Híveknek serege, te szent Atyádat!

>357. Isten kezét mutatja
/1
#4DBB2D05
 Isten kezét mutatja
 Az égnek boltozatja,
 Mely felettem kiterjed,
 Szívem örömre gerjed
 E remeknek látására.
 Alkotója lelkemet
 És minden érzésemet
 Ragadja csudálására.
/2
#26B6468D
 Ha elmémmel felhágok
 E számtalan világok
 Roppant alkotmányába:
 Ott látom valójába'
 Nagy voltát a Teremtőnek.
 Minél feljebb repülök,
 Annál jobban szédülök
 Magasságán e tetőnek.
/3
#21A73123
 Ki gyújtá meg ezeket
 Az örökös tüzeket?
 E nagy testeket fontba
 Ki vetette, hogy pontba
 Egymásnak megfeleljenek?
 Ki mérte ki útjukat
 És örök pályájukat,
 Hogy erről el ne térjenek?
/4
#C484EFE5
 Te vagy az, Mindenható,
 Kihez hasonlítható
 Nincs sem földön, sem égen!
 Te vagy, ki voltál régen,
 Kiben nincsen fogyatkozás,
 Kiben nincsen hajdani,
 Jövendő vagy mostani,
 Kihez nem járul változás.
/5
#098BD875
 Én hát mély tisztelettel,
 Mely teljes szeretettel,
 Előtted megnémulok,
 Zsámolyodhoz borulok,
 És imádom nagy voltodat,
 Hogy bármily kicsiny vagyok,
 Szintúgy, mint ezen nagyok,
 Tapasztalom jóságodat.

>358. Egek nagy Királya, Ma-gasztalunk téged
/1
#A076C29F
 Egek nagy Királya,
 Ma-gasztalunk téged,
 Térjen nevedre dicséret!
 Mulandó és gyarló
 Életünk átjárja
 Végtelen kegyelmed árja.
 Jöjj, segíts, Jóra ints;
 Rólad zengjen nyelvünk,
 Vígan énekeljünk!
/2
#E5B50315
 Dicsérjed, világom,
 Alkotód hatalmát,
 Mindenek felett uralmát!
 Hirdesse a csillag
 Ránk sugárzó fénye,
 Mily csodás az
 Isten lénye! Nagy égbolt,
 Nap és hold,
 Kik ott fennen jártok:
 Ti is Őt áldjátok!
/3
#FA8D437F
 Zengjen minden szívben
 Boldog hálaének
 A világ Teremtőjének!
 Nagy kegyelmessége
 Gondot visel rólunk,
 számon tartja minden dolgunk.
 Égben fenn, Földön lenn
 Áldassék jósága,
 Nagy irgalmassága!
/4
#458EE782
 Halleluját zengve,
 Magasztalva áldjad
 Krisztus által jó Atyádat!
 Ha az Istent élő,
 Igaz hittel féled
 És a szíved Jézusé lett:
 Boldogság, Üdv vár rád
 Fenn az ég honában,
 Hófehér ruhában.

>359. Istenemhez száll az ének
/1
#43700F12
 Istenemhez száll az ének
 És őbenne vigadok;
 Mert szerelme, bármit nézzek,
 Mindenünnen rám ragyog;
 Véget ér majd mindenem,
 Csak az ő szerelme nem!
 Mely szívemet égbe vonja,
 Onnan néz rám áldó gondja.
/2
#B880D554
 Mint a madár kis fiára
 Ráborítja szárnyait,
 Úgy fedez be bizonyára
 Engem Isten karja itt;
 Véget ér majd mindenem,
 Csak az ő szerelme nem!
 Jó Atyám ő kezdet óta,
 Minden bajtól szívem óvta.
/3
#5D30B197
 Uram, áldom szerelmednek
 Hűn vezérlő gondjait;
 Áldalak, hogy bíznom benned
 Szent beszéded megtanít;
 Véget ér majd mindenem,
 Csak az ő szerelme nem!
 Adj kegyelmet, hogy e hitben
 Megmaradjak mindvégiglen!

>360. Magasztalja az én szívem
/1
#2B56DB66
 Magasztalja az én szívem
 Az Urat mind éltiglen,
 És örvendez az én lelkem
 Megtartó Istenemben,
 Hirdetem kegyességét
 És nagy jótéteményét.
/2
#03716CA8
 Mert kegyesen megtekinté
 ő szolgáló leányát,
 Jóvoltából meg nem veté
 Hív, alázatos voltát,
 Mert íme, minden nemzet
 Boldognak mond engemet.
/3
#026553BE
 Mert énvelem nagy dolgokat
 Tőn a hatalmas Isten,
 Midőn együgyű voltomat
 Felemelé kegyesen,
 Szent azért az Ő neve,
 Áldandó dicsősége.
/4
#FC855FD3
 És az Ő irgalma vagyon
 Nemzetségről nemzetre,
 Örökké néz irgalmason
 Minden őket félőkre,
 Hatalmas dolgot szerze
 Ő karjának ereje.
/5
#1791EF9E
 A kevély gondolatúknak
 Menten eszeket veszté,
 Dühös szándékát azoknak
 A jókra nem ereszté,
 Engem is nagy irgalma,
 Megtarta nagy hatalma.
/6
#08DE4F07
 A hatalmas gonoszokat
 Ő székökből levoná,
 És az alázatosokat
 Híven felmagasztalá,
 Mert szeret együgyűket
 És gyűlöl kevélyeket.
/7
#7C3BBACD
 A nyomorult éhezőket
 Minden jókkal betölté,
 A gazdagsággal tölteket
 Üresen elereszté,
 Mert Őtőle mindenek
 Bölcsen vezéreltetnek.
/8
#4204F8AA
 Izráelt, az ő szolgáját
 Híven hozzá fogadá,
 Említvén irgalmasságát,
 Népét meglátogatá,
 Rájok nyújtván kegyelmét,
 Bétölté ígéretét.
/9
#88140DAD
 Amiképpen szólott vala
 Mi első atyáinknak,
 Ábrahámnak mit megmonda
 És ő maradékinak,
 Ím, mind megteljesíté,
 Ezért áldjuk örökké.

>361. Meghódol lelkem tenéked, nagy Felség
/1
#7D7BA587
 Meghódol lelkem tenéked, nagy Felség,
 Szentháromságban ki vagy egy Istenség.
 Csak téged illet minden tisztesség,
 Mert téged ural az egész föld s ég.
/2
#72CF8C66
 Imád a nagy ég ő teljességével,
 Mondván: szent, szent, szent az
 Úr felségével! Teljes a föld s ég dicsőségével,
 Seregek Ura erősségével!
/3
#97F7D840
 Imád a földnek kiterjedt nagysága,
 Mind e világnak nagy hatalmassága;
 A sok népeknek minden országa,
 Roppant táborok sűrű sokasága.
/4
#745371E1
 Imád téged a napnak fényessége,
 Az éjszakának titkos setétsége;
 Imád a holdnak ő teljessége:
 A csillagoknak szép ékessége.
/5
#9FE5D1C2
 Imád a tenger s a vizek folyása,
 A sok hegyeknek magas fennállása;
 Minden szeleknek széjjeloszlása
 S a madaraknak ékes szólása.
/6
#88EFEADB
 Imádlak én is téged, Teremtőmet,
 Gondviselőmet és Idvezítőmet,
 Megszentelőmet s erősítőmet:
 Én Istenemet, egy Segítőmet.
/7
#97708313
 Imádlak téged, Urát kezdetemnek,
 Urát végemnek s egész életemnek,
 Fő szerző okát a természetnek,
 Halálnak Urát és kegyelemnek.
/8
#6453EA79
 Imádlak téged, egyedül Uramat,
 Nem vetem másban én bizodalmamat;
 Mikor imádlak, halld meg én szómat,
 Írd be könyvedbe hódolásomat.

>362. Mely igen jó az Úr Istent dicsérni
/1
#8BA6DF01
 Mely igen jó az Úr Istent dicsérni,
 Felségednek, én Uram, énekelni,
 Szent nevedet dicsérvén magasztalni
 És mindenütt e világon hirdetni.
/2
#3FB24290
 Igen reggel irgalmadat hirdetni,
 Igazságodról éjjel gondolkodni,
 Hegedűvel, orgonával zengetni,
 Minden éneklő szerszámmal tisztelni.
/3
#98117FDA
 Csudaképpen én vigasztalást vészek,
 Cselekedetidre hogyha tekintek,
 Kezeidnek munkájában örvendek,
 Teremtőmnek, megváltómnak éneklek.
/4
#77009419
 Az esztelen ember ezt nem esméri,
 A hitetlen bolond ember nem érti;
 Kinek rólad nincs igaz esméreti:
 Szent Fiadban mert nincs hite őnéki.
/5
#619D19C6
 E világon gonoszok gyökereznek,
 Kik mindenkor hamisan cselekesznek;
 Mint a füvek, virágoznak, terjednek,
 Hogy örökül-örökké elvesszenek.
/6
#5B29F655
 Lám, ezeket, Uram, felséges Isten,
 Kik támadnak a te szent igéd ellen,
 Viaskodnak a te híveid ellen:
 Megbünteted, mert vagy örök Úr Isten.
/7
#AB1C73B5
 Rólad, Uram, akik megemlékeznek,
 Mint pálmafák úgy szintén ők zöldellnek;
 Mint cédrusfák, ugyan meggyökereznek
 Az igazak, kik igaz hitben élnek.
/8
#CA36D080
 Vallást tesznek Minden emberek előtt,
 Hogy az Isten igaz mindenek fölött;
 Hamisságot soha nem cselekedett,
 Mint kőszikla, ő ád nagy erősséget.

>363. Menynyei seregek, boldog, tiszta lelkek
/1
#9A950466
 Menynyei seregek, boldog, tiszta lelkek,
 Az Úrra örökké kik az égen néztek:
 Őtet teljes szívből ti mind dicsérjétek!
/2
#35EE324C
 Angyalok, az Úrnak követi kik vagytok,
 Szentek, kik ő székét mind körülálljátok:
 Örökké az Urat felmagasztaljátok!
/3
#8203447A
 Fényes nap világa, ez világ fáklyája;
 Szép hold, éj lámpása, csillagok nagy száma
 Az Úrnak szent nevét mindörökké áldja.
/4
#AAD6E545
 Mert csak ő egyedül minden teremtője,
 S mindent bír, valamint magában rendelte;
 Megmarad mindenek ellen ő szerzése.
/5
#945FE5FD
 Azért hát, ti hívek, Úrnak szent serege,
 Kik leginkább vagytok néki szerelmébe',
 Örökké szent nevét dicsérjétek mennybe'!

>364. Mindenkoron áldom az én Uramat
/1
#AEFADC83
 Mindenkoron áldom az én Uramat,
 Kitől várom én minden oltalmamat.
 Benne vetem minden bizodalmamat;
 Mindenkoron dicsérem, mint Uramat.
/2
#8B2A29D9
 Igen vigad és örvendez én lelkem,
 Az Istennek segedelmét hogy kérem,
 Nyomorultak meghallják, azt örvendem,
 Vigadjanak Istenben, arra intem,
/3
#65FFD8B8
 És mikoron Istenhez kiáltottam,
 Kegyelmesen tőle meghallgattattam,
 Őáltala hamar megszabadultam,
 Háborúságimban is megtartattam.
/4
#747ED644
 Lám, Istennek angyala mind tábort jár,
 Az istenfélő emberek körül jár.
 Az Istentől azért ki oltalmat vár,
 Útaiban mindenütt az nagy jól jár.
/5
#F8B5F332
 Segítségül azért Istent hívjátok,
 Ő jóvoltát kóstoljátok, lássátok!
 Igen nagy-jó, azt bizonnyal tudjátok:
 Benne bízó emberek mind boldogok.
/6
#6BCCEAEE
 Valamíglen élsz ez árnyék világban,
 Szántszándékkal ne élj a gonoszságban,
 Sőt életed foglaljad minden jóban,
 Hogy lakozzál Istennek oltalmában.
/7
#080D1744
 Sok jók közt a békességet szeressed,
 És éltedben mindenkor azt keressed;
 E világnak békességét ne nézzed,
 Az ördöggel ne légyen közösséged.
/8
#68354B0D
 A felséges Isten szemei vannak
 Igazakon, kik csak őbenne bíznak;
 Mindazok, kik tőle oltalmat várnak,
 Kérésükben mindig meghallgattatnak.
/9
#BA1A2CB0
 Igen közel az Úr Isten azoknak,
 Töredelmes szívvel akik óhajtnak;
 Alázatos lélekkel akik járnak,
 Sok ínségből bizton megszabadulnak.

>365. Mondjatok dicséretet
/1
#0A89D5DE
 Mondjatok dicséretet,
 Keresztyének, az Úr Istennek,
 Szentséges tiszteletet
 Adjatok ő nevének.
/2
#5C05C411
 Örvendjetek őneki
 Igaz isteni buzgóságban,
 Mert köztetek lakozik
 Az Anyaszentegyházban.
/3
#3D098AAD
 És ő megelégíti
 Lelketeket lelki kenyérrel;
 Biztat ő igéjével
 Édes ígéretivel.
/4
#60CF9949
 Szent, Uram, a te neved,
 Szent vagy te a magas mennyekben,
 Szent vagy és rettenetes
 Minden nemzetségekben.
/5
#C53A2F13
 Tégedet illet, Uram,
 Mi tiszteletünk, háladásunk,
 A mi gyülekezetünk,
 Ünnepet szentelésünk.
/6
#E66046BC
 Hála légyen tenéked,
 Hogy megjelentéd te magadat,
 Hogy megértettük, Uram,
 A te akaratodat.
/7
#8C6F9750
 Dicséret és dicsőség
 Légyen tenéked magasságban,
 Dicséret és tisztesség
 Az Anyaszentegyházban.

>366. Ó, Ábrahám Ura, Hadd áldjuk szent neved
/1
#60A08D16
 Ó, Ábrahám Ura,
 Hadd áldjuk szent neved,
 Mert mindenható vagy és örök szeretet.
 Nagy Isten a neved,
 Ezt vallja föld és ég,
 Csak téged illet tisztelet és dicsőség.
/2
#15F9BB63
 Ó, Ábrahám Ura, Ím, hallom szent szavad;
 Csak azt az üdvöt keresem, mit kezed ad.
 A múló földi jót És vágyát elhagyom,
 S őt választom, ki őrizőm és pásztorom.
/3
#7672DEF0
 Ó, Ábrahám Ura, Szent kegyelmed nekem
 Az én örömöm, utamon ez vezessen.
 Te barátod lettem, Én Istenem te vagy:
 Tarts meg a Jézus véréért és üdvöt adj!
/4
#49D217FE
 Megesküvél, Uram, És igédben bízom,
 Hogy égbe viszed gyermeked sasszárnyakon.
 Meglátom Jézusom És áldom hatalmát,
 Szent kegyelmének éneklek halleluját.

>367. Szent vagy örökké, Atya Úr Isten
/1
#E8B1A643
 Szent vagy örökké,
 Atya Úr Isten,
 A magas menynyekben,
 Ki teremtettél és megtartottál
 E nyomorult testben.
/2
#97AB54D5
 Szent a mi Urunk,
 Úr Jézus Krisztus,
 Kit értünk bocsátál,
 Kivel megváltál,
 Hogy testünk miatt
 Lelkünk el ne vesszen.
/3
#D12B3F50
 Szent a Szentlélek,
 Ki az Atyával
 Egy és a Fiúval,
 Aki vigasztal
 És tanít minket
 Az örök életre.
/4
#05EA49F0
 Szentség, dicsőség,
 Légyen tisztesség
 A Szentháromságnak,
 Ki uralkodik egy Istenségben
 Most és mindörökké.

>368. Teremtő Istenünk
/1
#E8AD8071
 Teremtő Istenünk,
 Édes Atyánk nékünk!
 Hatalmas vagy,
 Irgalmad nagy:
 Ínségemben,
 Szükségemben,
 Kérlek, engem ne hagyj!
/2
#161C00C2
 Szívem benned bízik,
 Hozzád fohászkodik;
 Nézz csendesen,
 Kegyelmesen,
 Fájdalmimon,
 Bánatimon \.
 Könyörülj kegyesen.
/3
#037B824C
 Tudom, hogy kegyelmed
 Enyhít mindeneket,
 Bús szíveket,
 Sebeseket,
 Szegénységben,
 Betegségben ellankadt fejeket.
/4
#3A8F12FE
 Jelen van az idő,
 Hogy te, jó segítő,
 Meghallgatod Siralmimat,
 Nékem adod Tanácsodat,
 Mert tudod sorsomat.
/5
#DED90C39
 Megváltó Istenünk,
 Szabadító Urunk:
 Vagy paizsom,
 Fő orvosom,
 Reménységem,
 Idvességem
 És gyönyörűségem.
/6
#8790E69D
 Azért, ó, Jézusom,
 Legbölcsebb orvosom,
 Siess, siess,
 És lelkemnek Segítője
 Légy testemnek,
 Megepedt szívemnek!
/7
#8E3187AD
 Nálad lehetetlen,
 Tudom, semmi sincsen.
 Nyújtsd kezedet:
 Bús szívemet,
 Erősítsd bágyadt lelkemet,
 Mert áldlak tégedet.
/8
#024E6110
 Szent Lélek Istenünk,
 Vigasztaló Urunk:
 Bátoríts meg,
 Oltalmazz meg,
 Igaz hitben,
 Szerelmedben
 Minden jókkal áldj meg!
/9
#10C5F4F7
 Keresztemben tűrést,
 Adj boldog szenvedést,
 Hogy abban ne Zúgolódjam,
 És búban ne Tántorodjam,
 Hozzád folyamodjam.
/10
#84455617
 Légy erős gyámolom,
 Csak rád támaszkodom.
 E világi Életemben,
 Halálom után mennyégben
 Véled legyek, Ámen.

>369. Téged, Úr Isten, mi keresztyének
/1
#8DE36908
 Téged, Úr Isten, mi keresztyének
 Dicsérünk és áldunk,
 És Atya, Fiú, Szentléleknek vallunk.
/2
#65DA5E17
 Néked a tiszta, ártatlan és szent
 Angyalok szolgálnak,
 Szüntelen hangos felszóval kiáltnak:
/3
#257B23D7
 Szent, szent, szent
 Isten, te seregednek
 Vagy Ura, Istene!
 Teljes a menny s föld
 nagy dicsőségeddel!
/4
#8A1E3934
 Téged a földön a keresztyének
 Szent gyülekezetje,
 Szent Atya Isten,
 mindenkoron dicsér.
/5
#0CFFFBDC
 És imádandó te szent fiadat,
 Az Úr Jézus Krisztust,
 És vigasztaló Szentlélek Úr Istent.
/6
#4B6DC209
 Oltalmazzad meg veszedelemtől
 A te népeidet:
 Oltalmazz bűntől s lelki kártól minket.
/7
#A40DAD70
 Dicsérünk téged, mennyei Atyánk,
 A te szent Fiaddal És mindörökké
 Szentlélek Istennel.

>370. Úr Isten, csak néked éneklek
/1
#4C64E633
 Úr Isten, csak néked éneklek,
 Ki volna más tehozzád fogható?
 Dicséretet csak néked zengek,
 Csak téged áldlak, ó, Mindenható!
 Jövel, segíts, én édes Jézusom,
 Hogy könyörgésem mennybe eljusson!
/2
#7CFB8985
 Ó, vonj, Atyám, Fiadhoz engem,
 Hogy szent Fiad is hozzád elvigyen;
 S míg jóvoltodat áldva zengem,
 Keblem Szentlelked hajléka legyen;
 Add békességed ízét érzeni
 És dicséreted vígan zengeni!
/3
#7BB6B7A5
 Ha semmi mást nem kérek tőled,
 Csak amit Lelked által kérhetek,
 Te szükségim beteljesíted,
 Mert Jézus hozzád utat készített,
 Ki gyötrelmével értem áldozott,
 És üdvösséget jussomul hozott.
/4
#EFEDB012
 Jó tudnom azt, hogy közbenjáróm
 Hű Jézusom, ki jobbod felől ül;
 Őáltala, ha hitből várom,
 Kérésem nálad bizton teljesül;
 Jó nékem, hogy míg tart ez életem,
 Szent neved vígan, áldva zenghetem!

>371. Végtelen irgalmú szentséges Úr Isten
/1
#C085DAD8
 Végtelen irgalmú szentséges Úr Isten,
 Ki engem segítesz minden én ügyemben,
 Csak te vagy énnékem győzhetlen fegyverem,
 Paizsom, kőfalam, minden reménységem!
/2
#54F5DFFA
 Hajtsd le füleidet a magas kék égből,
 Halld könyörgésemet kegyelmességedből,
 És ne csinálj törvényt az én érdememből,
 Hanem véghetetlen irgalmas szívedből!
/3
#F3F53852
 A te kezed által formáltattam földből,
 Nagy bűnnel jöttem ki anyámnak méhéből,
 De a te irgalmad megtisztított ettől,
 S minden földi jómat vettem te kezedből!
/4
#797847FF
 Mert az én érdemem nálad annyit teszen,
 Mennyi vizet fecske kis szájába veszen;
 A megmérhetetlen tengermélység ellen:
 Ennyi én érdemem te kegyelmed ellen.
/5
#E72B5B71
 Mégis én, nyomorult, háládatlan vagyok,
 Új bűnnek naponként sokszor helyet adok.
 Uram, költs fel, kérlek, amikor szunnyadok,
 Moss meg Szentlelkeddel, mert rút mocskos vagyok!
/6
#AEB1B32D
 Atyáink vétkéről meg ne emlékezzél,
 Sőt minden bűnükről, kérlek, feledkezzél:
 Mert tégedet nem holt, hanem ki most is él,
 Az dicsér, s nevedről tisztességet beszél.
/7
#010EE8A2
 Vedd hozzád lelkemet, mely téged alig vár,
 Miként segítséget hadtól megszállott vár;
 Vedd ki én testemből, melyen vagyon nagy zár,
 Ne fullassza bűntől nagyra növelt vízár!
/8
#F32FA3E7
 Nem nekünk, nem nekünk,
 Uram, tisztességet,
 De szent nevednek adj örök becsületet!
 Minket azért áldj meg, hogy hívjuk nevedet
 Mi segítségünkre, és bízzunk tebenned!

>372. Hű pásztorunk, vezesd a te árva nyájadat
/1
#D6451785
 Hű pásztorunk, vezesd a te árva nyájadat,
 E földi útvesztőben te mutass jó utat;
 Szent nyomdokodba lépve, a menny felé megyünk,
 Ó, halhatatlan Ige, vezérünk, Mesterünk.
/2
#7BFBB687
 Mert boldog az az ember, ki dicsér tégedet,
 És kóstolgatja mindennap szent beszédedet;
 Hát legeltessed igéddel bolygó nyájadat,
 És terelgessed Lelkeddel juhocskáidat.
/3
#5474F9D9
 Szentlelkedet töltsd ránk ki mint hajnal harmatát,
 És adj fejünkre tőled nyert ékes koronát,
 Hogy áldozatra felgyúlt, megszentelt életünk
 Oltárodon elégjen, Királyunk, Mesterünk!

>373. A mi szívünk csak tehozzád
/1
#DCB13E2D
 A mi szívünk csak tehozzád,
 Jézus, Isten Báránya,
 Óhajtozik, híveidnek
 Drága fényes aranya,
 Mert halálunk megrontója,
 Örök életnek adója
 Vagy egyedül,
 Krisztusunk,
 Idvezítő Jézusunk.
/2
#FEABE03F
 Csudálkozván nézi elménk
 Atyádnak nagy szerelmét,
 Álmélkodván is szemléli
 Hozzánk való jó kedvét,
 Melyet abban megmutatott,
 Hogy minékünk téged adott;
 Azért, édes Krisztusunk,
 Vagy szerelmes Jézusunk.
/3
#BC912C63
 Szegény bűnös embereknek
 Idvességes reménye,
 Benned hívőknek öröme,
 Boldogsága, szépsége:
 Jövel, jövel hozzánk, kérünk,
 Mindenkoron légy mivélünk,
 Mert csak te vagy Krisztusunk
 És kegyelmes Jézusunk.
/4
#4A9AE456
 Reád bízzuk mi magunkat
 Itt ez árnyék világban,
 Idvezítő szerelmedből
 Könyörülj fiaidon.
 Légy vezérünk, oltalmazónk,
 Kegyes mesterünk, tanítónk,
 Édes Atyánk, Krisztusunk,
 Gondviselő Jézusunk.
/5
#C891E1C9
 Erősítsd bennünk hitünket,
 Idvességünk eszközét,
 Vesd tengerbe bűneinket,
 Kerüljük pokol tüzét.
 Így mi halálunk óráján,
 Életünknek végső táján,
 Lelkünket, ó, Krisztusunk,
 Vedd kezedbe Jézusunk.
/6
#82428847
 Nincsen nékünk itt e földön
 Maradandó városunk,
 Hanem repeső elmével
 Jövendőt óhajtozunk,
 Ott sok ezer angyalokkal,
 Téged áldunk szent Atyáddal:
 Méltó vagy, ó, Krisztusunk,
 Erre, édes Jézusunk.
/7
#CD2F87D9
 Hozzád hajtjuk csak fejünket,
 Ó, mi édes Megváltónk,
 Koporsónkba ha beszállunk,
 Ott is léssz gyámolítónk.
 Benned édesen aluszunk,
 Eljöveteledig nyugszunk,
 És örökké, Krisztusunk, \.
 Véled élünk, Jézusunk.
/8
#32C92B2D
 Zörgetőknek megnyittatik
 Kegyelemnek ajtaja;
 Megígérte az Úr Jézus,
 Hogy kérésünk megadja.
 Jövel, Jézus, és ne késsél,
 Már mellőlünk el ne térjél:
 Idvezíts, ó, Krisztusunk,
 Mi kegyelmes Jézusunk!

>374. Úr Jézus, nézz le rám, Jöjj, mosd le bűnömet
/1
#06D015BA
 Úr Jézus, nézz le rám,
 Jöjj, mosd le bűnömet,
 Sok földi szenvedély kötöz:
 jöjj, oldj fel engemet.
/2
#F75C8C39
 Úr Jézus, nézz le rám,
 Gond és bú látogat;
 Hű szolgád: ízleljem ígért,
 szent nyugodalmadat.
/3
#6957377C
 Úr Jézus, nézz le rám,
 Ne tévedhessek el;
 A menny felé sötéten
 át te légy az úti jel.
/4
#E3AE6524
 Úr Jézus, nézz le rám,
 Ha nő a félelem,
 Ár zúg és ellenség szorít, légy,
 Megváltóm, velem!
/5
#A1EB404A
 Úr Jézus, nézz le rám,
 Mert harcom terhe nagy,
 Ily tenger kín és baj között
 Az életem Te vagy.
/6
#53CDBCB9
 Úr Jézus, nézz le rám,
 Ha elvonult az ár,
 Te szent derűd derítsen
 és az örök napsugár.

>375. Eltévedtem, mint juh, El-tévedtem, mint juh
/1
#AE3A93B0
 Eltévedtem, mint juh,
 El-tévedtem, mint juh,
 A bűnösök útjára,
 Ó, segíts, Jézusom,
 Őriző pásztorom,
 Hogy ne jussak romlásra.
 Te ontál drága vért
 Elveszett juhokért,
 Viselj gondot a nyájra.
/2
#CEEF7753
 Én is juhod vagyok,
 Én is juhod vagyok,
 Nyájadnak legkisebbje,
 Kit te megtéríthetsz,
 Bűnből kivezethetsz
 A szép kies helyekre.
 Kérlek azért hitből,
 Töredelmes szívből,
 Fogadj, végy kegyelmedbe.
/3
#44BC7982
 Ím, előtted állok,
 ím, előtted állok,
 Ajtód előtt zörgetek,
 Bár titkos bűnökkel
 És nyilván lévőkkel
 Vétkeztem teellened:
 Kérlek mindazáltal,
 Nagy irgalmassággal
 Fogadd vissza gyermeked.
/4
#5CA1E4AC
 Bár a hit szívemben,
 Bár a hit szívemben
 Oly kicsiny, mint mustármag,
 Mégis bármi gyenge,
 Szentlelked nevelje,
 Nevelje fel nagy fának,
 Hogy terjedjen ága
 És legyen virága
 Kedves néked, Urának.
/5
#60F480DE
 Míg porsátoromban,
 Míg porsátoromban
 Tartasz, mint egy tömlöcben,
 Nevelj igaz hitben,
 Munkás szeretetben,
 Hogy élhessek itt bölcsen
 És kimúlásomig,
 Utolsó órámig
 Tisztemet hűn betöltsem.
/6
#33B0E610
 Szállj le, Uram, hozzám,
 Szállj le, Uram, hozzám,
 Jöjj, ó, Jézus, sietve,
 Vágyakozó szívvel,
 Kiterjesztett kézzel
 Várlak immár epedve,
 Hogy veled mennyekbe,
 Örömmel menjek be
 Ábrahám kebelébe.

>376. Emeljük Jézushoz szemünk
/1
#1B4BC4CD
 Emeljük Jézushoz szemünk,
 Jön már királyi győztesünk,
 Mennyből leszáll s együtt leszünk.
 Lelkünk vigyázni meg ne szünjön,
 Felséges várástól feszüljön,
 Az álmot űzd el, készen állj,
 Krisztusnép, jön, jön a Király!
/2
#4E0CD1B9
 Azt mondta Jézus:
 Idelenn Új próba és
 új küzdelem
 A hívők sorsa szüntelen.
 Azért ne csüggedjünk, ne féljünk,
 Az út rövid, végére érünk.
 Az álmot űzd el, készen állj,
 Krisztus-nép, jön, jön a Király!
/3
#B4AC010F
 Éneklünk és a perc szalad,
 A nap, mely nesztelen halad,
 Az öröklét felé mutat.
 De míg hangunk majd zengve szárnyal
 Hozsánnás angyal-kar szavával,
 Az álmot űzd el, készen állj,
 Krisztus-nép, jön, jön a Király!
/4
#C1B76034
 Ó, kérünk, Jézus, jó Urunk,
 Te szabd meg életünk, utunk:
 Hány éjszakánk lesz s hány napunk.
 Bölcs szívedből töltsd meg szívünket.
 Te ismered jól kis hitünket:
 Küldj, küldj szent sóvárgást nekünk,
 Hogy majd ha jössz, készen legyünk.

>377. Isten szelíd Szolgája, te
/1
#C46CAAE3
 Isten szelíd Szolgája, te,
 Atyánknak édes gyermeke,
 Legfőbb kincsünk, szent ékszerünk,
 Maradj velünk!
/2
#F72BE233
 Éltél köztünk mint megvetett,
 Kaptál értünk fájó sebet.
 Bűnünk sötét, nincs érdemünk.
 Maradj velünk!
/3
#3683F414
 Kértél bennünk lakóhelyet,
 Lettél tető fejünk felett,
 Lettél kenyér, szent ételünk.
 Maradj velünk!
/4
#0BC27D78
 Ihatjuk mind véred borát,
 Mely szomjazónak enyhet ád;
 Ki nem fogyó szent serlegünk,
 Maradj velünk!
/5
#18F35067
 Meg nem rendül szerelmed, ó,
 Bűnünk tudó, irgalmazó!
 Szerelmeddel be nem telünk,
 Maradj velünk!
/6
#3907429F
 Javunkra, ím, adtad magad,
 Ontod reánk irgalmadat.
 Méltó nevet tőled nyerünk.
 Maradj velünk!
/7
#C777E85B
 Vágytad szívünk megáldani,
 Haláloddal megváltani.
 Nem lesz pokol végső helyünk!
 Maradj velünk!
/8
#0189B2F5
 Megváltottál kínok sokán,
 Értünk haltál a Golgotán.
 Keresztednél letérdelünk,
 Maradj velünk!
/9
#C8EE747F
 Krisztus, te szent, megkínzatott,
 Elsápadott, elcsúfított,
 Mégis nekünk szép Édenünk,
 Maradj velünk!
/10
#EF0417BB
 Érettünk meggyaláztatott,
 Taníts nekünk alázatot!
 Urunk, kitől üdvöt nyerünk,
 Maradj velünk!
/11
#F5A29F20
 Ó, boldog az, ki téged áld!
 Jaj, kárhozott, ki mást imád!
 Másutt üdvöt mi nem lelünk:
 Maradj velünk!
/12
#D688FB48
 Áldunk, Atyánk, nagy Istenünk,
 Áldunk, Fiú, hű Mesterünk,
 Áldunk, Szentlélek, s kérlelünk:
 Maradj velünk!

>378. Jézus, ó, mi Idvezítőnk
/1
#2147B7B6
 Jézus, ó, mi Idvezítőnk,
 Kiben vagyon hitünk,
 Ki miértünk születtél,
 És nagy halált szenvedtél;
 Irgalmazz minékünk!
/2
#3C77A169
 Rólunk elvetted haragját
 A te szent Atyádnak;
 Vérednek hullásával
 Megengesztelted hozzánk:
 Irgalmazz minékünk!
/3
#94176D6B
 Az ördögnek tömlöcéből,
 Kivettél a bűnből;
 A halál köteléből,
 Megmentettél rabságból:
 Irgalmazz minékünk!
/4
#DF3CC58D
 Te bűn nélkül születtetvén
 A Szűznek méhéből,
 Értünk válladra vetted
 A mi adósságinkat:
 Irgalmazz minékünk!
/5
#14A8D128
 Bűnön, poklon és halálon
 Birodalmad vagyon;
 Az élet és kegyelem
 Csak te kezedben vagyon:
 Irgalmazz minékünk!
/6
#A3B94E93
 Kérünk téged azért mostan:
 Tekints reánk mennyből,
 Kik hozzád esedezünk
 És segítséget várunk:
 Irgalmazz minékünk!

>379. Jer, dicsérjük az Istennek Fiát
/1
#B797F505
 Jer, dicsérjük az Istennek Fiát,
 A szép szűznek áldott szent magzatját,
 E világnak édes Megváltóját,
 Bűnösöknek kegyes szószólóját.
/2
#6E7B5AE4
 Jézus Krisztus, kegyelmes Megváltónk,
 Úr Istentől adott tanítónk,
 Szent Atyáddal felséges megtartónk,
 Szentlélekkel mi megigazítónk!
/3
#D2AC87E9
 Téged vallunk hatalmas Istennek,
 Kezdet nélkül való természetnek,
 Szent Atyáddal egy örök Istennek:
 Szentlelkétől lelkeztél Istennek.
/4
#EBE4809C
 Felöltözél az emberi testbe,
 E világra jövél miközinkbe,
 Igaz hitet adál mi szívünkbe,
 Új világot gerjesztél lelkünkbe'.
/5
#2813F4AD
 Szent véreddel minket te megváltál,
 Bűneinkből szépen kitisztítál,
 Szentlélekkel megvilágosítál,
 És Atyádnak kedvébe juttattál.
/6
#A5C03C4B
 A keresztfán érettünk meghaltál,
 Harmadnapra ismét feltámadtál,
 Mennyországban felmagasztaltattál,
 Tisztességgel megkoronáztattál.
/7
#A2C82E13
 Azért vallunk téged királyunknak
 És örökké való főpapunknak,
 Isten előtt minden gyámolunknak,
 E világon minden oltalmunknak.
/8
#1C919850
 Te vagy nékünk mi nagy igazságunk,
 Jámborságunk és ártatlanságunk,
 Te vagy nékünk szentségünk, váltságunk,
 Isten előtt örök boldogságunk.
/9
#637DD902
 Te vagy nékünk a mi reménységünk,
 Ez világon mi nagy tisztességünk,
 Isten előtt minden dicsőségünk,
 Mennyországban örök idvességünk.
/10
#077E6936
 Ó, Istennek drága kincstartója,
 Szentléleknek rajtunk nyugtatója,
 Nyomorodott árvák megtartója,
 Mi hitünknek megkoronázója.
/11
#BDA93FD2
 Ó, Istennek kedves áldozatja,
 Kit jó szemmel megtekint az Atya,
 És megmarad az ő jó illatja,
 Mindörökké kedves foganatja.
/12
#B3A877E8
 Édes Jézus, Elődbe járulunk,
 Széked előtt arccal leborulunk,
 Keservesen tehozzád óhajtunk,
 És nagy sírván mennybe felkiáltunk.
/13
#23E150CF
 Adjad nékünk a te Szentlelkedet,
 Terjesztessed a te szent igédet;
 Ismerjünk meg mindnyájan tégedet,
 Dicsérhessük a te szent nevedet.
/14
#A3AD21F6
 Tekints reánk most a magas mennyből,
 Látogass meg királyi székedből,
 Tégy jól vélünk kegyelmességedből,
 Ments meg minket keserűséginkből.
/15
#1F81221D
 Vigasztald meg a mi siralminkat,
 Teljesítsd bé mi kívánságinkat;
 Mi is néked hálaadásinkat,
 Bémutatjuk mi áldozatunkat.
/16
#9A124859
 Édes Jézus, néked hálát adunk
 Jóvoltodért és felmagasztalunk,
 Ajándékot elődbe mutatunk,
 Dicséretet te nevedre mondunk.
/17
#97F220D4
 Dicsértessék az Atya Úr Isten,
 Dicsértessék a Fiú Úr Isten,
 Dicsértessék a Szentlélek Isten,
 Szentháromság egy örök Úr Isten!

>380. Jézus, vigasságom! Esdekelve várom
/1
#EEBA2B23
 Jézus, vigasságom!
 Esdekelve várom Áldó szavadat!
 A te jelenléted Megvidámít, éltet,
 Bátor szívet ad. Légy velem,
 Ó, mindenem! Nálad nélkül nem is élek:
 Te vagy örök élet!
/2
#9370251E
 Jézus, menedékem!
 Hű oltalmam nékem
 Te vagy egyedül!
 Lelkem a viharból,
 Bűnből, minden bajból
 Hozzád menekül.
 Bár a föld Mind romba dőlt,
 S ha a pokol hada hány tőrt:
 Jézus maga áll őrt!
/3
#3C703F9B
 Jézus, üdvösségem!
 Te vagy földön-égen
 Örök örömem!
 Kik szeretjük Istent,
 Zengjünk neki itt lent
 S otthon: odafenn!
 Lelkem esd, Hogy
 Te vezesd! S hazahívó
 szavad várom,
 Jézus, Vigasságom!

>381. Jézus, édes emlékezet
/1
#95C6B511
 Jézus, édes emlékezet,
 Te adsz szívünknek örömet:
 Már hozzád vontad szívemet;
 Vezéreljed lépésimet.
/2
#69DB5DAD
 Jézus, megtérők reménye,
 Hozzád kiáltóknak szíve,
 Téged keresőknek kincse,
 Megtalálóknak mindene!
/3
#45C9F2AF
 Jézus, szívnek édessége,
 Élő kútja, fényessége,
 Minden kincsének szekrénye,
 Kívánságának jó vége:
/4
#A8156CF8
 Sem nyelv azt meg nem mondhatja,
 Sem betű fel nem írhatja,
 Csak ki próbálta, tudhatja,
 Hogy milyen a Jézus lángja.
/5
#305F336A
 Te vagy kegyelem kútfeje,
 Igaz hazánk fényessége,
 Búbánatnak enyhítője,
 Boldogságnak megszerzője.
/6
#20275024
 Tégedet áldnak mennyégben,
 Magasztalnak dicsőségben;
 Jézus, vigasztalj éltünkben,
 Juttass Atyádnak kedvében.
/7
#034C6E04
 Téged kövessünk egy szívvel,
 Énekléssel, könyörgéssel,
 Hogy mi is a Szentlélekkel
 Lehessünk egyek Istennel.
/8
#D2C78F4B
 Jézus, mennyben légy örömünk,
 Ki voltál földön érdemünk;
 Benned minden dicsőségünk,
 Ó, mi Urunk és Istenünk!

>382. Jézus Krisztus, egy Mesterünk
/1
#54BBBEC2
 Jézus Krisztus, egy Mesterünk,
 Menynyei szent bölcsességünk,
 És nékünk bizonyos idvességünk.
/2
#B9F3F075
 Mostan néked mi könyörgünk,
 Szent nevedért esedezünk:
 A te Szentlelkedet adjad nékünk!
/3
#A6F86DBB
 Mi szívünket megújítsa,
 És sebeinket gyógyítsa:
 Bűnös életünket megjobbítsa.
/4
#F8D94AD9
 Munkáinkat megszentelje,
 Bennünk a hitet nevelje,
 Utunkat hazánkba vezérelje.
/5
#AAA2D5F3
 Bölcsességre megtanítson,
 Hogy ördög meg ne csalhasson:
 Kísértések ellen bátorítson.
/6
#E267DD8C
 Plántáljon nagy egyességet,
 Igaz és szent szeretetet,
 Szívünkbe isteni szent félelmet.
/7
#94326B6C
 Hogy téged bátran vallhassunk,
 E földön néked élhessünk,
 Mennyben aztán veled lakozhassunk.

>383. Kegyes Jézus, én imádságomra
/1
#E7590B5D
 Kegyes Jézus, én imádságomra,
 Hajtsad füled én kiáltásomra,
 Jusson hozzád kérésem én jómra,
 Ne nézz, Uram, méltatlan voltomra!
/2
#2A6B54E2
 Szomorkodom undok bűneimért,
 Fohászkodom sok gonoszságimért,
 Félek, Uram, mert sok rútságimért
 Megutáltál, tudom, én bűnömért.
/3
#9490FB04
 Tökéletes a te ítéleted,
 Tiszta, szent vagy, a bűnt nem nézheted,
 A vakmerő bűnöst megbünteted,
 Országodból méltán kirekeszted.
/4
#774EFECA
 Félek, azért megvallom, Istenem,
 Hogy ha úgy bánsz, mint érdemlem, velem,
 Szent színedet elfordítod tőlem,
 Országodból kirekeszted lelkem.
/5
#6A7B3121
 Mert ha végignézek életemen,
 Látom, hogy a bűnt gyermekségemen
 Kezdettem, és azóta szüntelen
 Növekedett a teher lelkemen.
/6
#1A1A487C
 Múló hasznát a bűnnek szerettem,
 Törvényedet azért félretettem;
 Felebarátomat sértegettem,
 Melyért hozzád méltatlanná lettem.
/7
#B13C098D
 De ez egyben biztatásom lelem,
 Hogy tenálad vagyon a kegyelem;
 Azért hozzád szívemet emelem:
 Cselekedjél irgalmasan velem.
/8
#24966776
 Mert nem azért vetted fel testünket,
 Hogy megítélj bűnünkért bennünket,
 Hanem hogy megszabadítsd lelkünket
 és megszerezd örök életünket.
/9
#B9FC7D89
 A megtérő bűnösnek kedveztél,
 Bűnbocsátó kegyelmet ígértél,
 Isten előtt érte kezes lettél,
 Minden átkot fejéről elvettél.
/10
#CE1B9000
 Kívánkozom én is hozzád térni,
 Segélj, Uram, e szent célt elérni,
 Mert így lehet kegyelmedet kérni,
 Így lehet csak tőled azt megnyerni.
/11
#C0938875
 Segélj, segélj, én édes Istenem!
 Néked adom s ajánlom ma lelkem:
 Kérlek, fogadd magadhoz, Jézusom,
 Idvességem s életem, Krisztusom!
/12
#7D1DDC84
 Míg itt élek, éltem vezéreljed,
 Tartóztasson a bűntől törvényed;
 Minden jóra segéljen Szentlelked,
 Kegyelemmel biztasson érdemed.
/13
#D2EFA7ED
 Mikor érzem halálom óráját,
 Biztass, hogy jól kifutván a pályát,
 Majd meglátom az Isten orcáját,
 Elnyerem az élet koronáját.

>384. Hozzád jövök, ó, Mesterem
/1
#3E36605A
 Hozzád jövök, ó, Mesterem,
 Hogy halljam szavaid
 És értsem örök beszéded,
 Mely megtart, idvezít.
/2
#A7903E3F
 Igéd csodás világánál
 Mutassa meg a hit,
 Te vagy  váltság egyedül,
 Mely megtart, idvezít.
/3
#0140A588
 A te erőd a halálból
 Az új életre híd,
 Hogy hozzád jussak rajta át,
 Ki megtart, idvezít.
/4
#3D89CC9F
 Hogy utaidon járhassak,
 Szent kezed megsegít,
 Csak te vezéreld lábamat,
 Te vagy, ki idvezít.
/5
#F7A6F95F
 Jól tudom, kinek hittem én,
 S ez engem boldogít;
 Örökké áldom kegyelmed,
 Mely megtart, idvezít.

>385. Jézusom, ki árva lelkem
/1
#71234EB9
 Jézusom, ki árva lelkem
 Megváltottad véreddel,
 Kárhozattól óvtál engem,
 Bűnös szívem, ó, vedd el!
 Add, hogy néked megháláljam.
 Hogy nem hagytál a halálban, megmutattad:
 Bármit adj, Én oltalmam csak te vagy.
/2
#6FEF6382
 Jézus, benned bízva bízom,
 Elpusztulnom, ó, ne hagyj!
 Te, ki bűnön, poklon, síron
 Egyedüli győztes vagy:
 Gyönge hitben biztass engem,
 Készíts arra, hogy én lelkem
 Láthat majd fenn, ó, Uram,
 Mindörökké boldogan.

>386. Krisztusom, kívüled nincs kihez járulnom
/1
#2859B7F6
 Krisztusom, kívüled nincs kihez járulnom,
 Ily beteg voltomban nincs kitől gyógyulnom.
 Nincs ily fekélyemből ki által tisztulnom,
 Veszélyes vermemből és felszabadulnom.
/2
#9F2CEC07
 Gyújtsd meg szövétnekét áldott szent igédnek
 És bennem virraszd fel napját kegyelmednek;
 Igaz utat mutass nékem, szegényednek,
 Járhassak kedvére te szent Felségednek.
/3
#5888A7E8
 Várlak, Uram, azért reménykedő szívvel,
 Miként a vigyázó virradást vár éjjel;
 Hozd fel szép napodat nékem is jó reggel,
 Hogy szolgálhassalak serényebb elmével.
/4
#9AB5E280
 Dicsértessél. Atya Isten, magasságban,
 Mi Urunk Krisztussal mind egy méltóságban,
 És a Szentlélekkel mind egy hatalomban:
 Háromság egy Isten, áldj meg dolgainkban.

>387. Úr Jézus, hozzád kiáltok
/1
#2112EA87
 Úr Jézus, hozzád kiáltok,
 Ó, halld meg esdeklésem;
 Tetőled kegyelmet várok,
 Ne hagyj kétségbe esnem!
 Az igaz hitet kívánom,
 Ó, adjad azt énnékem:
 Néked élnem, Mindennek használnom,
 Szent Igédet követnem.
/2
#C2CA32BD
 Ezt is kérem, én Istenem,
 Megadhatod azt könnyen:
 Szégyenbe ne engedj esnem,
 Tarts meg reménységemben.
 Kiváltképp, ha ki kell múlnom,
 Hogy csak tebenned bízzam,
 Ne magamban, Sem én jóságomban;
 Ezt örökké megbánnám.
/3
#E7410911
 Adjad, hogy teljes szívemből
 Engedjek vétetteknek,
 Te is ingyen kegyelmedből
 Kegyelmezz vétkeimnek,
 Igéd légyen eledelem,
 Mellyel lelkem tápláljam
 És megóvjam, Ha ínség ellenem
 Támad, el ne nyomassam.
/4
#62BA9A32
 Gyönyörűség vagy félelem
 Tetőled el ne űzzön,
 Állhatatos mindvégiglen
 Légyek az igaz hitben.
 Az van kezedben: add ingyen,
 Nem cselekedetimből,
 De kedvedből Ments ki a halálból,
 Az örök kárhozatból!
/5
#C4BCD576
 Nagy harcban, ellenkezésben
 Te segélj meg, Istenem!
 Csak a te nagy kegyelmedben
 Van nyugtom és védelmem.
 Ha kísértet jő, védelmezz;
 Szent kezed megsegítsen,
 Meg ne ejtsen, Ami kárt szerezhet;
 Tudom: nem hagysz elvesznem.

>388. Lelkem drága Jézusa, Hozzád hajt a félelem
/1
#B171CA7E
 Lelkem drága Jézusa,
 Hozzád hajt a félelem,
 Míg üvölt a habtusa,
 S nő a vész a tengeren,
 Rejts el, rejts el, itt ne hagyj,
 Míg eláll a fergeteg;
 Biztonságos révet adj,
 S majd fogadd el lelkemet.
/2
#44876B14
 Nincs nekem más enyhelyem,
 Szívem Téged hív s keres,
 Ó, maradj itt, Mesterem,
 Őrizz, adj erőt, szeress!
 Véled állom a vihart,
 Hit s erő Te vagy, Te Szent,
 Szárnyad árnyával takard
 Fejemet, a védtelent.
/3
#3B22225A
 Csak Te kellesz, én Uram,
 Benned mindent meglelek;
 Támogasd, ki elzuhan, \.
 Gyógyítsd meg, ki vak s beteg.
 Szent szavadra hallgatok,
 Tévedés az én bajom,
 Én hamisság s bűn vagyok,
 Te igazság s irgalom.
/4
#33E0D566
 Kegyelem vagy, égi jó,
 Mely minden bűnt eltörül,
 Hagyd, hogy gyógyító folyó
 Tisztogasson meg belül.
 Élet-kút vagy, lüktetés,
 Vízmerítni drága hely,
 Ó, buzogj fel bennem és
 Öröklét felé emelj.

>389. Úr lesz a Jézus mindenütt
/1
#3BE449DB
 Úr lesz a Jézus mindenütt,
 Hol csak a napnak fénye süt,
 Úr lesz a meszsze tengerig,
 Hol a hold nem fogy s nem telik.
/2
#D905AC4E
 Őneki mondjunk hő imát,
 Díszítsük azzal homlokát,
 Jó illat légyen szent neve,
 Minden napon dicsérete.
/3
#D682CAB1
 Országok, népek és nyelvek,
 Ő dicsőségét zengjétek,
 Gyermekek hangja hirdesse:
 Áldott a Jézus szent neve!
/4
#422D1A8D
 Ő királysága bő áldás,
 Ott van a felszabadulás,
 Fáradtak ott megnyugszanak,
 Ínségesek megáldatnak.
/5
#E38CF841
 Minden teremtés dicsérje,
 A Király Krisztust tisztelje;
 Angyali ének zengjen fenn,
 S mind e föld mondja rá: Ámen.

>390. Zengd Jézus nevét, zengd, világ
/1
#800E0BF2
 Zengd Jézus nevét, zengd, világ,
 őt, angyalok, áldjátok!
 Felékesítve homlokát,
 Királylyá Jézust,
 Jézust koronázzátok!
/2
#F788C110
 Ti vértanúi Istennek,
 Kik mennyben szolgáljátok
 A Bárányt, ki megöletett:
 Királlyá Jézust,
 Jézust koronázzátok!
/3
#E5DBCF17
 Ti választottak, szent hívek,
 Mind akit ő megváltott,
 Szent irgalmát dicsérjétek:
 Királlyá Jézust,
 Jézust koronázzátok!
/4
#EEAB1416
 Ti bűnösök, mert ő hordott
 Tiértetek kínt s átkot,
 És szent vérével áldozott:
 Királlyá Jézust,
 Jézust koronázzátok!
/5
#39930203
 Ti népek, törzsek, kik bárhol
 Az ő szavát halljátok:
 Nagy jóvoltáért hálából
 Királlyá Jézust,
 Jézust koronázzátok!
/6
#F6485463
 Mily boldogság lesz majd, ha fenn
 A Jézus előtt állok
 És mindörökké zenghetem:
 Királlyá Jézust,
 Jézust koronázzátok!

>Gondviselés

>391a. Gondviselő jó Atyám vagy
/1
#55B5247F
 Gondviselő jó Atyám vagy,
 Ó, én édes Istenem!
 Hozzád vágyom, benned élek,
 Üdvöt mástól nem remélek.
 Látom én, hogy minden elhagy
 E világon, csak te nem!
/2
#B66DBB4B
 Mint az alélt bús virágra
 Megújító harmatot:
 Vérző szívem fájdalmára
 Csak te hintesz balzsamot.
 Könnyű sorsom terhe rajtam,
 Ha imára nyílik ajkam.
/3
#8D23F1DF
 Rám-rám derül ismeretlen
 Utamon egy kis öröm,
 Azt is a te véghetetlen
 Jóságodnak köszönöm;
 Hálakönnyem tündöklése
 A te neved hirdetése.
/4
#FE3C3E1D
 Gyenge vagyok, lankadoznak
 Buzgóságom szárnyai,
 Bármily híven vágyakoznak
 Színed elé szállani;
 Ó, adj erőt, hogy míg élek,
 Egyedül csak néked éljek!
/5
#D67BD15B
 Ó, add, hogy ha majd bevégzem
 E mulandó életet,
 Lelkem tisztán és egészen
 Egyesüljön teveled.
 El ne vonjon semmi többé,
 Tied legyek mindörökké!

>391b. Gondviselő jó Atyám vagy
/1
#930CBAF7
 Gondviselő jó Atyám vagy,
 Ó, én édes Istenem!
 Hozzád vágyom, benned élek,
 Üdvöt mástól nem remélek.(2)
 Látom én, hogy minden elhagy
 E világon, csak te nem!

>392. Mindenek meghallják és jól megtanulják
/1
#A73F28E0
 Mindenek meghallják és jól megtanulják,
 Kik segedelmüket nem Istentől várják:
 Nincsen Isten nélkül segítség és idvesség.
/2
#DE1F5CD8
 Ha nem az Úr Isten építi a házat,
 Ahány építője, mind hiába fárad.
 Nincsen Isten nélkül segítség és idvesség.
/3
#833F67D0
 Csak hiába lészen reggel felkeléstek
 Néktek, kik erős Istenben nem hisztek:
 Nincsen Isten nélkül segítség és idvesség.
/4
#418AFEBF
 Ekképpen történik mindnyájan tinektek,
 Munkával, bánattal kenyeret kik esztek:
 Nincsen Isten nélkül segítség és idvesség.
/5
#201F12F0
 Nagy könnyen az Isten mindent ád azoknak,
 Kik csak benne bíznak s hozzá fohászkodnak:
 Nincsen Isten nélkül segítség és idvesség.
/6
#2180BE58
 Mint a sebes nyilak az erős kezében,
 Erősek a hívek Isten kegyelmében:
 Nincsen Isten nélkül segítség és idvesség.
/7
#33FC8EFC
 Boldog, aki lelkét hittel erősíti,
 Minden ellenségét bizonnyal meggyőzi:
 Nincsen Isten nélkül segítség és idvesség.

>393. Mind jó, amit Isten tészen
/1
#307B558D
 Mind jó, amit Isten tészen,
 Szent az ő akaratja,
 Ő énvélem is úgy tégyen,
 Mint kedve néki tartja.
 Ő az Isten, Ki ínségben
 Az övéit megtart -ja,
 Hát légyen, mint akarja.
/2
#A54F4708
 Mind jó, amit Isten tészen,
 Ő engemet meg nem csal,
 De igaz ösvényen viszen,
 Én megelégszem azzal,
 Hogy kedvében, Kegyelmében
 Ő forgatja dolgomat,
 Csak rá hagyom magamat.
/3
#37E9CBF4
 Mind jó, amit Isten tészen,
 Ő engem meg nem utál,
 Mint jó orvosom, úgy tészen,
 És mérget ő nem kínál.
 Orvosságot, Boldogságot
 Énnékem készít, tudom,
 Azért csak benne bízom.
/4
#03ABF9DC
 Mind jó, amit Isten tészen,
 Ő az én idvességem,
 Ő velem rosszul nem tészen,
 Rábízom egész éltem.
 Örömömben, Keresztemben
 Mind nyilván megmutatja,
 Hogy javamat akarja.
/5
#FACE5165
 Mind jó, amit Isten tészen:
 Ha oly pohárt innék is,
 Amelynek íze szívemnek
 Nagy-keserűn esnék is,
 De eltűröm, Mert víg öröm
 Felváltja ezt végtére,
 Sok búm enyhítésére.
/6
#A079ACC6
 Mind jó, amit Isten tészen,
 Mind örökké ezt vallom,
 Ha rajtam bú, bánat lészen
 S kell bosszúságot látnom.
 Mindazáltal Megvigasztal,
 Mint édes Atyám, engem,
 Mert csak ő segítségem.

>394. Kegyelmes Isten, kinek kezében
/1
#055FFDF9
 Kegyelmes Isten, kinek kezében
 Életemet adtam, Viseld gondomat,
 vezérld utamat,
 Mert csak rád maradtam.
/2
#4EAC847C
 Add meg énnékem én reménységem
 Szerint való jódat!
 Áldd meg fejemet, ki bízik benned,
 S viseljed gondomat!
/3
#A1926D0B
 A szép harmatot miként hullatod
 Tavasszal virágra,
 Sok jódat, Uram, úgy hullasd reám,
 Te régi szolgádra.
/4
#3C9451BE
 Hogy mind holtomig szívem legyen víg,
 Téged magasztalván,
 Mindenek előtt s mindenek fölött
 Szent nevedet áldván.

>Bizodalom Istenben

>395. Úr Isten, te tarts meg minket
/1
#6DF7FCE7
 Úr Isten, te tarts meg minket,
 És szent igédben hitünket;
 Rontsd meg mi ellenségünket
 És minden kegyetleneket.
/2
#65BD76A2
 Kik a Krisztust háborgatni
 És szent székiből levetni
 Akarják, őtet rontani,
 Ő híveiben kergetni.
/3
#3162FC92
 Krisztus, ki vagy urak Ura
 És királyoknak Királya:
 Jelentsed istenségedet
 És törd meg ellenségidet.
/4
#39EA7C06
 Tartsd meg minden híveidet,
 Te szegény keresztyénidet,
 Hogy dicsérhessünk tégedet,
 Tartsd egyességben népedet.
/5
#5D124303
 Ó, áldott Szentlélek Isten,
 Vigasztalj minket e földön,
 Légy jelen mi szükségünkben,
 Minden keserűséginkben.
/6
#7DE7D06F
 Nevelj minket igaz hitben,
 Krisztusnak ismeretiben;
 Végy minket szent szerelmedbe
 És holtunk után örömbe.

>396. Forog a szerencse, Mit bízunk őbenne?
/1
#9CDE7479
 Forog a szerencse,
 Mit bízunk őbenne?
 Semmiben nem állandó.
 Csak ideig kedvez,
 Tündöklő üveghez
 Mindenképpen hasonló.
 Ki minthogy eltörik,
 Így ő is változik,
 Állapotja romlandó.
/2
#D058379F
 E rossz, csalárd világ,
 Ki merő hamisság,
 Engem már majd elveszte,
 Mint halat horogra,
 Úgy csala sok búra
 Hízelkedő beszéde.
 Étkével táplála,
 Melyben rejtve nála
 Iszonyú csalárd mérge.
/3
#2DC313FA
 Nyughatatlan sok gond
 Íme, annyira ront,
 Hogy nem soká megemészt.
 Ha nem szánja lelkem
 Teremtő Istenem,
 Sok bú, siralom elvészt,
 Sőt én ellenségem
 Csudálja, hogy engem
 A föld is el nem süllyeszt.
/4
#1D73CFC2
 Éljek-e? - nem tudom,
 Mert késik halálom
 S kínjaim nevekednek.
 Kétségbe essem-e,
 Halált szerezzek-e
 Nyomorodott fejemnek?
 Azt nem mívelhetem,
 Mert elveszti lelkem
 Haragja Istenemnek.
/5
#8F3ACC0A
 Tűrnöm hát jobb lészen,
 Mert még elővészen
 Istenem, szent Fiáért.
 Bűnöm elfelejti,
 Pokolra sem veti
 Lelkem, ki hozzá megtért.
 Bár súlyos kereszttel
 Próbál és megterhel
 Méltatlan életemért.
/6
#8A1FA40A
 Szenvedek békével,
 Magamat is ezzel
 Biztatom mindenekben:
 Szenvedett Megváltóm
 Többet, én jól tudom,
 Üdvösségemért testben.
 Nem egyedül vagyok
 Földön, ki nyomorgok:
 Szenvedik ezt is többen!
/7
#35276003
 Megszán még Istenem,
 Ennyi sok siralmim,
 Tudom, jóra fordulnak.
 Múlnak sóhajtásim,
 Noha most könnyeim
 Szemeimből csordulnak.
 Mint nap eső után,
 Így bánatim után
 örömim megújulnak.

>397. HAGYJAD az Úr Istenre Te minden utadat
/1
#2B6A53A0
 HAGYJAD az Úr Istenre
 Te minden utadat,
 Ha bánt szíved keserve,
 Ő néked nyugtot ad.
 Ki az eget hordozza,
 Oszlat felhőt, szelet,
 Napját rád is felhozza,
 Atyád ő, áld, szeret.
/2
#D95AADBA
 AZ ÚRRA bízzad dolgod:
 Könnyebbül a teher;
 Ezer baj közt is boldog,
 Aki nem csügged el.
 Minek a gond, a bánat?
 Mit gyötröd lelkedet?
 Az Istent kérjed, várjad,
 S megnyered ügyedet.
/3
#E0BF5F40
 A TE irgalmasságod
 Van rajtam, Istenem,
 Te jól tudod, jól látod,
 Hogy mi használ nekem.
 Sorsomat úgy intézed,
 Amint te akarod; \.
 Bölcs a te végzésed,
 Ha áld, ha sújt karod.
/4
#B1BD8A30
 UTAD van számtalan sok,
 Uram, és eszközöd;
 Reánk is szent áldásod
 Bőséggel öntözöd.
 Művednek akadálya,
 Szünetje nincs soha;
 Úgy téssz, amint kívánja
 Gyermekeid java.
/5
#98EE68B8
 BíZZáL, bánatos lélek!
 Mit bánt a bú, a gond?
 Él még, ki annyi vészek
 Torkából már kivont.
 Bajaidból kiment ő,
 Szűnnek keserveid;
 Rád még a jó Teremtő
 Víg napot is derít.
/6
#E34A764D
 ŐBENNE vesd halálig
 Jó reménységedet:
 Ő biztos révbe szállít
 A bajból tégedet.
 Bár késik a segítség
 És nem találsz vigaszt:
 Eloszlik gond és kétség
 Előbb, mint véled azt.
/7
#4C4472CC
 Ő MEGCSELEKSZI  végre
 Velünk azt, ami jó;
 Ösvényünk erőssége
 Te vagy, Mindenható!
 Bár nehéz földi pályánk,
 Könny lepi és tövis,
 De örök pálma vár ránk:
 Utunk a mennybe visz.

>398. Istenre bízom magamat
/1
#08E6857F
 Istenre bízom magamat,
 Magamban nem bízhatom;
 Ő formált, tudja dolgomat,
 Lelkem ezzel biztatom.
 E világ szép formája
 Az ő keze munkája.
 Mit félek? - mondom merészen:
 Istenem és Atyám lészen. -
/2
#28D60278
 Öröktől fogva ismerte,
 Hogy mire lesz szükségem,
 Éltem határát kimérte,
 Szükségem s elégségem.
 Lelkem, hát ne süllyedezz,
 A hitben ne csüggedezz!
 Egy kis bajt nem győznél-e meg?
 Hogy tántorítana ez meg?
/3
#65A9FC85
 Tudja Isten kívánságod,
 Ád is, mert csak ő adhat,
 De bölcs, Uram, te jóságod,
 Tudod, sok elmaradhat.
 Tudom: gondod reám nagy,
 Mivel édes Atyám vagy;
 Mint akarod, hát úgy légyen!
 Másként hinnem volna szégyen.
/4
#93FF84B6
 A valóságos igaz jót
 Az Úr meg nem tagadja;
 Nagy gazdagság és rakott bolt
 Nem fő jó, ritkán adja.
 Ki az Isten tanácsát,
 Megszívleli mondását,
 Azt ő Lelkével serkenti,
 Gondját is megédesíti.
/5
#94514A30
 E világnak dicsősége
 Igen hamar elmarad,
 Kit ma gondok sújtnak, végre
 Holnap diadalt arat.
 Csak Atyámban bízhatom,
 Ő megsegít, jól tudom,
 Mert az igazaknak Atyja
 Hű szolgáját el nem hagyja.

>399. Ki Istenének átad mindent
/1
#AC14D9B2
 Ki Istenének átad mindent,
 Bizalmát csak beléveti,
 Azt csudaképpen őrzi itt lent,
 Ínség, baj közt is élteti.
 Ki mindent szent kezébe tett,
 Az nem fövényre épített.
/2
#F6BBC888
 A súlyos gondok mit használnak,
 A sóhaj, sok jajszó mit ér,
 Ha sebeink még jobban fájnak
 S mindennap kínunk visszatér?
 Így terhünk egyre súlyosabb,
 Ha lelkünk búnak helyet ad.
/3
#69F15239
 Csak légy egy kissé áldott csendben:
 Magadban békességre lelsz,
 Az Úr rendelte kegyelemben
 Örök, bölcs célnak megfelelsz.
 Ki elválasztá életünk,
 Jól tudja, hogy mi kell nekünk.
/4
#C9BE794E
 Zengj hát az Úrnak s járd az utat,
 Mit éppen néked Ő adott;
 A mennyből gazdag áldást juttat
 S majd Jézus ád szép, új napot.
 Ki Benne bízik és remél,
 Az mindörökké Véle él.

>400a. Légy csendes szívvel és békével
/1
#745D176E
 Légy csendes szívvel és békével
 Életednek Istenében!
 Ő bír örömnek bőségével,
 Véle boldogulsz mindenben,
 Ő kútfőd és ő fényes napod,
 Őtőle jön minden vígságod:
 Légy csendes szívvel!
/2
#94C35469
 Ő bír vígsággal, kegyelemmel,
 Oly szívvel, mely feddhetetlen,
 Ahol ő van, nem sért semmivel,
 Hordozzon bár mélységekben;
 Sok nyavalyádat elfordítja,
 A halált is kezében tartja:
 Légy csendes szívvel!
/3
#B8A9F727
 Mint légyen néked s másnak dolga,
 Az nála nincsen elrejtve,
 Arra néz szemének világa,
 Ki búval van megterhelve.
 Ő számlálja fohászkodásid,
 Összefogja könnyhullatásid:
 Légy csendes szívvel!
/4
#4F2FA684
 Ha egy sem volna e világon,
 Kire magad rábízhatnád,
 Akkor is állna oldaladon,
 És hűségét tapasztalnád.
 Ő tudja titkos bánatidat,
 Elvenni mikor kell azokat:
 Légy csendes szívvel!
/5
#379500AC
 Ő hallja lelked óhajtásit,
 Mit nem mernél megmondani,
 Néki szíved titkos panaszit
 Éppen bízvást megvallani.
 Nincs messze ő, sőt itt áll köztünk,
 Meghallja s adja, amit kérünk:
 Légy csendes szívvel!
/6
#88354DD1
 Ki a mezei madaraknak
 Megadja eledelüket,
 Ki juhoknak s egyéb barmoknak
 Szépen tartja életüket,
 Ő téged, egyet eltartani,
 Éhség ellen meg tud menteni:
 Légy csendes szívvel!
/7
#CBD66641
 A segítség ha kissé késik,
 De ugyan csak eljő végre,
 A várakozás, rosszulesik,
 De majd válik üdvösségre;
 Ami lassan jő, bizonyosabb,
 Ami késik, kívánatosabb:
 Légy csendes szívvel!
/8
#3CFFCAF0
 Ne vedd szívedre, az ellenség
 Amit rád költ hazugsággal.
 Hadd jöjjön csúfság, keserűség:
 Ítél az Úr igazsággal.
 Ha Isten a te jó barátod,
 Nem lehet emberektől károd:
 Légy csendes szívvel!
/9
#BFB54F40
 Hogy is lehetne másként dolgunk?
 Szenvedni kell az embernek,
 Amíg e földnek útján járunk,
 Csak nyavalyák környékeznek.
 A kereszt botja bárha terhel,
 Azt is letesszük életünkkel:
 Légy csendes szívvel!
/10
#64F6637D
 Van Isten népének szombatja,
 Akkor az Úr megszabadít,
 Ínség kötelét elszaggatja,
 Teljes szabadságra állít,
 S az idvezültek seregében
 Vég nélkül élünk békességben:
 Légy csendes szívvel!

>400b. Légy csendes szívvel és békével
/1
#B891D66D
 Légy csendes szívvel és békével
 Életednek Istenében;
 Ő bír örömnek bőségével,
 Véle boldogulsz mindenben.
 Ő kútfőd és fényes napod.
 Őtőle jön minden vígságod:
 Légy csendes szívvel!
/2
#3EAA5D3A
 Ő bír végtelen kegyelemmel,
 Atyai szívvel van hozzánk;
 Sehol sem sért minket semmivel,
 Kegyelmes szemmel néz reánk;
 S bárha Kereszttel látogat,
 Kész segedelemmel támogat:
 Légy csendes szívvel!
/3
#57FCCAAD
 Ő hallja lelked óhajtásit,
 Miket nem mersz elmondani;
 Szívednek titkos sóhajtásit
 Neki meg lehet vallani;
 Tudja minden bánatidat,
 S bölcsen vezérli dolgaidat:
 Légy csendes szívvel.
/4
#08C9E0E8
 A segítség, bár néha késik,
 De bizonnyal eljön végre;
 A várakozás rosszulesik,
 De majd válik idvességre;
 Mi lassan jő, bizonyosabb,
 Ami késik, kívánatosabb:
 Légy csendes szívvel.
/5
#DCD00E28
 Ne emésztődjél, ha ellenség
 Rossz hírt költ rád hazugsággal;
 Bár érjen csúfság, keserűség,
 Meggyőzheted azt jósággal;
 Ha Isten a te oltalmad,
 Embertől nem lehet ártalmad:
 Légy csendes szívvel.
/6
#67F1C22D
 Valamíg mi e földön élünk,
 Sokszor kereszt ér bennünket;
 És amitől legjobban félünk,
 Sorsunk gyakran azzal büntet;
 A halál mindent elvégez,
 Bajnak, búnak akkor vége lesz:
 Légy csendes szívvel!

>401.Ne csüggedj el, kicsiny sereg
/1
#F67883AF
 Ne csüggedj el, kicsiny sereg,
 Ha rád zúdul vad ellened,
 Hogy végképp öszszetörjön;
 Bár elpusztításodra tör,
 Gond, kételkedés mit gyötör?
 Nem lesz ez így örökkön!
 Bízzál: ügyed az Istené,
 Népét ő el nem ejtené:
 Ő áll majd boszszút érted;
 Ő állít Gedeont melléd,
 Általa harcodban megvéd,
 Szent igéjét és téged.
/2
#CF15D932
 Él az Úr, áll ígérete:
 Ördög s világ minden csele
 Megszégyenül mirajtunk!
 Vélünk az Úr, mi ővele,
 Végtelen az ő ereje:
 Győzelmet kell aratnunk.
 Krisztusunk, segélj, el ne hagyj,
 Pártfogónk végig te maradj:
 Oltalmazz neved által,
 Hogy mint hű nyájad, teneked
 Zenghessünk dicséreteket
 Víg, boldog hál'adással!

>402. Ó, én két szemeim, ti az Úrra nézzetek
/1
#C59FA6D6
 Ó, én két szemeim, ti az
 Úrra nézzetek,
 Hogy kegyelmes hozzám,
 mindenkor elhigygyétek;
 Az ő nagy szerelmét,
 Hozzám nagy jó voltát
 Mindenkor hirdessétek.
/2
#E313CC2D
 Megragadlak, Uram, az én igaz hitemmel,
 Reád támaszkodom erős reménységemmel,
 El sem is bocsátlak, Amíg meg nem hallgatsz,
 Az én lelki kezemmel.
/3
#031925AF
 Felindítlak téged nagy nyomorúságimmal,
 Az én bűneimnek számtalan sokságával;
 Mindaddig kiáltok, Míg bé nem kötsz engem
 Szent irgalmasságoddal.
/4
#B28922BA
 Ó, mennybéli Isten, te vagy én reménységem,
 Kinek hatalmában, kezében ellenségem;
 Ott vagy te, Úr Isten, Én nagy segítségem,
 Ahol nincs reménységem.
/5
#1231B0EF
 Tanítsál meg engem a te igaz utadra,
 Hadd lássak elmenni a te igazságodra;
 Én ellenségimet Gonosz szándékukban
 Ne bocsássad szájukra.
/6
#21358540
 Ha tebenned, Uram, nem reménylettem volna,
 A nagy bánat miatt megemésztettem volna;
 Az élőknek földén A te javaidat
 Meg nem láthattam volna.
/7
#A4BB65A4
 Azért, ó, én lelkem, serkentsd fel te magadat:
 Mit töröd, fárasztod nagy bánatban magadat?
 Majd meg fogod látni Uradnak jóvoltát,
 Csak el ne hagyd magadat.

>403. Semmit ne bánkódjál, Krisztus szent serege
/1
#D81BFDCA
 Semmit ne bánkódjál,
 Krisztus szent serege,
 Mert nem árthat néked senki gyűlölsége,
 Noha e világnak rajtad dühössége,
 De nem hágy szégyenben
 Krisztus ő Felsége.
/2
#B12EA6FC
 Királyi nemzet vagy, noha te kicsiny vagy,
 Az Atya Istennél bizony te kedves vagy;
 Ő szent Fia által már te is fia vagy,
 Minden dicsőségben, higgyed,
 hogy részes vagy.
/3
#5F6190A8
 Hogyha te igazán a
 Krisztusban bízol,
 Higgyed, hogy lélekben
 Istennél gyarapszol,
 Ha Krisztus véréből
 igaz hittel iszol:
 Higgyed, hogy örökké
 meg nem szomjúhozol.
/4
#61D997BE
 Akármint halásszon az ördög utánad,
 Az ő tagjaiban dühösködjék rajtad,
 Mind tőrrel, fegyverrel siessen utánad,
 Ha Krisztusban bízol, higgyed,
 az sem árthat.
/5
#DED4803D
 Rajtad semmit sem fog a pogány ellenség,
 Noha nehéz néked a rettenetesség;
 Bátor koncra hányjon a hitlen ellenség:
 Feltámaszt a Krisztus, néked nagy reménység.
/6
#C9A848A8
 Oltalmazza Krisztus az ő szent egyházát,
 Miként a jó pásztor saját juhocskáját;
 Valaki hallgatja a Krisztus mondását,
 Viseli mindenkor szorgalmatos gondját.
/7
#212AD1DF
 Siess most mihozzánk,
 Krisztus, segélj minket,
 A te szent igéddel neveljed hitünket
 És te Szentlelkeddel bírjad életünket,
 Hogy minden dolgunkban dicsérhessünk téged.
/8
#D829185A
 Igaz fogadásod és minden beszéded,
 Nyilván vagyon immár minden dicsőséged;
 Ne hagyd elpusztulni e kicsiny sereget,
 Melyért a keresztfán vallál nagy gyötrelmet!
/9
#ED85EAF6
 Ne nézzed, Úr Isten, e világ vakságát,
 Te nagy jóvoltodról háládatlanságát,
 A te igéd ellen ily nagy káromlását:
 Nézzed a Krisztusnak ártatlan halálát.
/10
#3AA6001C
 Vedd el már mirólunk a sok ellenséget,
 Vedd ki miközülünk a sok gyűlölséget;
 Essenek szégyenbe minden ellenségink,
 Kik dicsőségedet tagadják s nevetik.
/11
#4E96D9BF
 Senkiben nem bízik az anyaszentegyház,
 Hanem csak tebenned, ki minket oltalmazsz,
 És igazságoddal mindenkoron táplálsz,
 Te szent sebeiddel minket megvigasztalsz.
/12
#7B4CD92D
 Hálát adunk néked, mennybéli nagy
 Isten, Ki vagy egy szentségben és három személyben.
 Ne hagyj elrettennünk keserűségünkben,
 Halálunk óráján ne essünk kétségben!

>404. Siess, keresztyén, lelki jót hallani
/1
#E8DC4CF0
 Siess, keresztyén, lelki jót hallani,
 Régi törvényből harcolni tanulni,
 Az igaz hit mellett mint kell bajt vívni,
 Krisztusban bízni.
/2
#94BC83B4
 Mert nem hiába ezt az ó törvénybe,
 Próféták írták Biblia könyvébe;
 Szép tanulság ez most az új törvénybe',
 Mi eleinkbe'.
/3
#BB16CDC1
 Jól tudja földön ezt minden keresztyén:
 Nem csak fegyverrel oltalmaz az Isten.
 Ezt minden népnek tudására adom:
 Istenünk vagyon!
/4
#DD7B8A7E
 Fejedelemség vagyon csak Istenben,
 Minden hatalom vagyon ő kezében;
 Kiket ő akar, föld kerekségében:
 Emeli égben.
/5
#326D2642
 Ne ess kétségbe ő nagy jóvoltában,
 Az igaz hitben erős légy magadban,
 Mint Dávid, úgy jársz párviadalodban,
 Hitvallásodban.
/6
#EE0D11B2
 Dávidot Isten hagyá királyságban,
 Ő ellenségit veté gyalázatban.
 Dicsérjük Istent nagy hálaadásban,
 Énekmondásban.

>405. Mire bánkódol, ó, te, én szívem
/1
#F7C29EC4
 Mire bánkódol, ó, te, én szívem,
 Mit töröd magad ilyen igen?
 Ím, a testi jókért Bízzál csak az
 Úr Istenben,
 Ki uralkodik menynyekben.
/2
#13B55625
 A nagy Úr Isten nem hágy tégedet,
 Tudja ő minden szükségedet,
 Mind e világ övé;
 Azért minden szükségedben
 Bízzál csak az Úr Istenben.
/3
#0D9ACFB0
 Én Uram és én Istenem te vagy:
 Szegény szolgádat, engem ne hagyj,
 Ó, én édes Atyám!
 Én te szegény fiad vagyok,
 Senkiben bízni nem tudok.
/4
#4279691D
 A gazdag bízik az ő kincsében,
 De én bízom csak az Istenben,
 Noha szegény vagyok,
 Mert aki őbenne bízik,
 Soha meg nem csalatkozik.
/5
#F0BA3B51
 Nem kérek, Uram, én gazdagságot,
 Ami szükségem, te jól tudod,
 Mert mindeneket látsz,
 De főképpen arra kérlek,
 Uram, hogy teveled légyek.
/6
#7BCD9283
 Minden, ami e világon vagyon,
 Ezüst vagy arany s egyéb vagyon,
 Akármi légyen az,
 Csak kevés ideig tarthat,
 És idvességet nem adhat.
/7
#2BF9143B
 Nagy hálát adok, Uram, teneked,
 Értenem hogy adtad ezeket
 A te szent igédből.
 Adj mindvégig megmaradást.
 És idvességes kimúlást!

>406. Tebenned bízni, Ó, áldott Istenség!
/1
#7D66F2D3
 Tebenned bízni, Ó, áldott Istenség!
 Be drágább kincs ez minden földi jónál!
 Hová lennénk, mi lenne a mindenség,
 Ha menynyen, földön nem te országolnál?
 Homály körülem, bennem félelem,
 Magam ha nálad nélkül képzelem.
/2
#6FAE6846
 De nappal virrad, és remény kél bennem,
 Világ kormányán jobbodat ha látom;
 És bár nehéz úton kell néha mennem:
 Bátran megyek, ha a hit lett sajátom;
 Ragadd, ragadd, szerencse, mindenem!
 Gazdag maradtam, mert van Istenem!
/3
#041F9C5C
 Van Istenem, hatalmasabb mindennél,
 Ki Úr, ameddig hat világnak tére;
 Villámok szárnyán bár szüntelen mennél,
 Nagy országának nem jutnál végére;
 S akarni e királynak már elég:
 Szavának együtt hódol föld és ég.
/4
#CDB11D5C
 Van Istenem, ki hat mindenbe mélyen,
 S egyszerre látja mind, mi volt s jövend el.
 Ő tudja: hol, mikor, mivel segéljen,
 És csak jó célra, csak bölcs eszközt rendel;
 Nincs, nem volt, nem lesz egyszer is hiba, Legkisebb is dicső munkáiba'.
/5
#A8BA9FA9
 Van Istenem, kit jó Atyának hirdet
 E szép világon minden, amit látok.
 Nem kell rettegve hajtnom neki térdet,
 Ő mindig áld, nem jöhet tőle átok;
 E földön engem égnek ő nevel,
 S fokonkint feljebb, mind feljebb emel.
/6
#01097EAE
 Van Istenem, ki meg nem szűnik lenni:
 Határi közt ő nincsen az időnek.
 Ezred meg ezred múljék el bármennyi,
 Marad jövő s múlt nála egyenlőnek;
 Világon minden más erő avul:
 Ő jó s hatalmas változatlanul.
/7
#0F0AB644
 Ki lenne hát, kibe reményt vethetnél,
 Oly ingatlant, ó, szívem, mint Istenbe?
 Mily bátran nyughatol ily őrizetnél,
 S mehetsz akármilyen erővel szembe!
 Bár emberektől elfelejtve vagy:
 Van néki gondja rád, ő el nem hagy.
/8
#1D89FFFC
 Ő el nem hagy, ne félj, igazság híve,
 Csak bátran menj, nehéz is bár a pálya;
 Hiába vonva rád gonoszság íve,
 Tied lesz mégis győzelem pálmája;
 És semmi nyíl halálra nem talál,
 Mert közbe Istened paizzsal áll.
/9
#EC229C26
 Ő el nem hagy, ne félj, erény barátja,
 Ha lábad jónak útjáról nem tér el;
 Ő hív küzdésed helybenhagyva látja,
 S nemes célodra bölcsen elvezérel.
 Ne csüggedezz a szenvedés alatt,
 Ő akkor is jó Istened marad.
/10
#53A87C07
 Ő el nem hagy, ha szemfedél kárpitja
 Végórád kondultával rád lehull is;
 Sőt ekkor utad szebb hazába nyitja,
 S Atyád marad mindig a síron túl is.
 Most hidd, majd tisztán általlátod ott, \.
 Hogy ő neked mindenkor jót adott!

>407. Térj magadhoz, drága Sion
/1
#4ABF677B
 Térj magadhoz, drága Sion,
 Van még néked Istened,
 Ki atyádként felkaroljon,
 Szívét oszsza meg veled!
 Azt bünteti, kit szeret,
 Másképp ő nem is tehet:
 Sion, ezt hát jól gondold meg,
 Szabj határt bús gyötrelmednek.
/2
#0F6D1979
 Hullámok ha rémítenek
 Mérhetetlen víz felett,
 S a habok közt szíved remeg,
 Hogy sírod is ott leled;
 Ha aludni látod őt,
 Ki reményed és erőd:
 Sion, soha ne feledd el:
 Ő megvívhat tengerekkel!
/3
#5CB14F56
 Bár hegy, halmok rengenének,
 Miket égi kéz emelt,
 S indulása a nagy égnek
 Végromlásra adna jelt:
 Ezt látva is el ne hidd,
 Hogy ez a perc elveszít;
 Sion, addig meg nem dőlhetsz,
 Míg oltalmad Istentől lesz!
/4
#78CBE604
 Bár könnyűid omlanának
 Gyöngyökül a tengerbe,
 És elhalván hangja szádnak,
 Csak pihegnél, mint gerle,
 Bár vér volna bíborod
 S kő megszánná nyomorod:
 Sion, ne félj a gonosztól,
 Baj nem ér, míg Benne bízol!
/5
#D597046C
 Bár hordozzad zsarnok láncát,
 Érjen kínos rabhalál,
 Ha hitedet el nem játszád,
 Utad égbe nyitva áll.
 Örvendj mindig és vigadj,
 Emlékezz, ki népe vagy!
 Sion, nincs több Isten egynél,
 Benne hát ne kételkedjél!
/6
#B159B226
 Ó, ne csüggedj, ím, az estnek
 Már leszállnak árnyai,
 Kihez ajkid oly hőn esdnek,
 Halld: Atyádnak hangja hí.
 Ő gyalázat, kín helyett
 Néked jobbján ad helyet;
 Sion, a menny lesz te részed,
 Föld gyötrelmét hát ne nézzed!
/7
#30DC253E
 Végső áldást mondj hazádra,
 Mely távolról int feléd,
 Égi honnak a határa
 Van már hozzád közelébb.
 Édes érzés mért fog el,
 Melytől olvad szív, kebel?
 Sion, minden másképp lesz ott,
 El fog tűnni nagy sírásod.
/8
#DC1E5EAE
 Angyalok, ti fényes lelkek,
 Zengjetek víg éneket,
 Mert már biztos révbe tért meg,
 Kit bús szélvész hányt-vetett!
 Már meggyőzte a halált,
 Istenéhez égbe szállt:
 Sion, onnnan számkivetni
 Nem fog téged soha senki!

>408. Hazádnak rendületlenül
/1
#D62E6888
 Hazádnak rendületlenül
 Légy híve, ó, magyar!
 Bölcsőd az, s majdan sírod is,
 Mely ápol s eltakar.
 A nagy világon e kívül
 Nincsen számodra hely.
 Áldjon vagy verjen sors keze:
 Itt élned, élned, halnod kell!

>Hálaadás

>409. Adjunk hálákat az Atya Istennek
/1
#A74A992A
 Adjunk hálákat az Atya Istennek,
 Mennynek és földnek szent teremtőjének,
 És embereknek gondviselőjének, Éltetőjének.
/2
#DC8D9FD1
 Mert ő mihozzánk atyai szerelmét,
 Kijelentette drága szent igéjét,
 Mellyel táplálja híveinek lelkét,
 Nyújtja kegyelmét.
/3
#958C116A
 És ő megáldja benne reménylőket,
 Erősít minden erőtelenséget,
 Világosítja homályos szívünket,
 Setét elménket.
/4
#59B00FBB
 Atya Istennek mindezekért légyen
 Dicséret és nagy dicsőség mennyégben,
 Fiával s Szentlelkével egyetemben,
 Örökké! Ámen.

>410. Adjunk hálát mindnyájan
/1
#279FB3B2
 Adjunk hálát mindnyájan
 Az Atya Úr Istennek,
 És mondjunk dicséretet
 Mi teremtő Istenünknek,
 Ki egybegyűjte most minket,
 Hogy ünnepet szenteljünk,
 És szent Igéjével éljünk.
/2
#395838A5
 Ó, kegyes Atya Isten,
 Te vagy Úr mindenekben,
 Ki megjelentéd magad
 Szent igédben itt e földön,
 És sok csudatételidben,
 A te áldott Fiadban,
 Mi kegyes Idvezítőnkben.
/3
#CEE8F81A
 Légy kegyelmes minékünk
 A te áldott Fiadért,
 Az Úr Jézus Krisztusért,
 Mi szentséges Megváltónkért,
 És ne állj bosszút mirajtunk
 A nagy hitetlenségért,
 Fertelmes sok bűneinkért!
/4
#CE6B9C62
 De szentelj meg bennünket,
 Bírj és segíts meg minket,
 Gerjeszd fel mi lelkünket
 És a mi gyarló szívünket,
 Hogy téged megismerhessünk,
 Segítségül hívhassunk,
 Néked hálákat adhassunk.
/5
#F6EE20C7
 Világosíts meg minket
 A Szentlélek Istennel,
 Hogy szépen tündököljék
 Bennünk az evangyéliom,
 És erősíts meg, Úr Isten,
 A te áldott Igéddel
 Minden tévelygések ellen!
/6
#AE819EB5
 Te igazgasd elméjét
 És minden tanúságát,
 Vezéreld útát, nyelvét
 A mi lelkipásztorinknak,
 És oktassad elméjüket \.
 A te szent beszédedet
 Figyelmesen hallgatóknak.
/7
#DEB102E0
 Tarts meg minden időben
 Minket az igaz hitben,
 És igaz értelmében
 A szent evangyéliomnak;
 Légyen foganatos köztünk
 A te kegyes beszéded,
 És minden jóra intésed.
/8
#E0780C5A
 Hogy sok népek tehozzád
 Megtérjenek bűnükből
 És téged szolgáljanak
 Az ő szívükből-lelkükből;
 Hogy mindenek imádjanak
 S csak téged tiszteljenek,
 Hogy örökké élhessenek.

>411. Adjunk hálát az Úrnak, mert érdemli
/1
#03902854
 Adjunk hálát az Úrnak, mert érdemli,
 Mert minden gazdagságát velünk közli.
/2
#8FDAC1B7
 Ő, mint bő irgalmú kegyelmes Atya,
 Fiait testben, lélekben megáldja.
/3
#57903159
 Énekeljünk néki egy akarattal:
 Dicsőség, Atya Isten, szent Fiaddal!
/4
#8134C008
 Ki ételt adtál alkalmas időben,
 S felruháztál mezítelenségünkben.
/5
#2C8371C4
 Adjad, hogy tégedet megismerhessünk
 És szent Fiad által üdvöt nyerhessünk!
/6
#288EA2C5
 Szent Igédet hirdettessed közöttünk,
 Hogy éhen meg ne haljon a mi lelkünk!
/7
#CA1750EC
 Hálát adunk ezekért mi Atyánknak,
 Jézus Krisztusnak, mi Közbenjárónknak.
/8
#480F532F
 Szentlélek Istennel egyenlőképpen,
 Ki minket vigasztaljon, mondjuk: Ámen.

>412. Jöjj, mondjunk hálaszót
/1
#6A8916B0
 Jöjj, mondjunk hálaszót
 Hűszájjal és hű szívvel,
 Mert rajtunk itt az Úr
 Nagy csoda dolgot mível.
 Már anyaölben is
 Volt mindig gondja ránk.
 A sok jót, melylyel áld,
 Ki sem mondhatja szánk.
/2
#5AA484E2
 Dús kincséből az Úr
 Jó békességet adjon,
 Hogy szívünkben a kedv
 Víg és derűs maradjon.
 Ne hagyja híveit
 Bú-bajban sohasem;
 A rossztól óvja meg
 Itt s túl ez életen.
/3
#268E4D08
 Az Atyát és Fiút És a
 Szentlelket áldom;
 A menny Urát, kiben
 Szent egybe forrt a három;
 Aki úgy szól ma is,
 Ahogy régente szólt,
 Nem változik: Az Ő,
 És az lesz, aki volt.

>413. Mily jó, ha bűntől már szabad
/1
#9B1F657C
 Mily jó, ha bűntől már szabad,
 Az Úr szolgája vagy;
 A bűn szolgája gyáva rab,
 A Krisztusé szabad.
/2
#64BEF598
 A bűn sötétben tévelyeg
 És bajba dönt vakon;
 De Krisztus kézen fog s vezet
 Világos utakon.
/3
#7D5BF647
 A bűnben kín van s gyűlölet,
 Mi mást, más minket öl;
 Öröm köt egybe s szeretet
 Az Úr szívén belöl.
/4
#94BE5644
 Már szolgád lettem, Jézusom,
 Ki értem áldozál;
 Más uram nincsen, jól tudom,
 Mert bűnből kihozál.
/5
#1A418C8B
 Légy áldott, Krisztusom, te nagy!
 Hadd adjam át szívem:
 Vedd szívesen, hogy hol te vagy,
 E szív is ott legyen.

>414. Nagy hálát adjunk az Atya Istennek
/1
#C46EF86E
 Nagy hálát adjunk az Atya Istennek,
 Mennynek és földnek szent teremtőjének,
 Oltalmazónknak, kegyes éltetőnknek,
 Gondviselőnknek.
/2
#7B56A60C
 Hála tenéked, mennybéli nagy Isten,
 Hogy szent igédet adtad mi elménkben,
 És hogy ezáltal véssz ismeretedben,
 Te kegyelmedben.
/3
#9202949C
 A romlás után nem hagyál bűnünkben,
 Sőt te Fiadat ígéréd igédben,
 Hogy elbocsátod őtet miközénkben,
 Emberi testben.
/4
#EF57EEA1
 Felséges Isten, tenéked könyörgünk,
 Hogy mutassad meg szent
 Fiadat nékünk, Hogy őtet látván,
 benne remélhessünk, Idvezülhessünk.
/5
#ADD16476
 Adjad, hogy lássuk a világosságot,
 Te szent igédet, az egy igazságot,
 A Krisztus Jézust: örök vigasságot,
 És boldogságot.
/6
#82F98661
 És ne ismerjünk többet a Krisztusnál,
 Ne szerethessünk egyebet Jézusnál;
 Maradhassunk meg a te szent Fiaddal,
 Krisztus Urunkkal.
/7
#704AE04C
 Adj igaz hitet a te szent Fiadban,
 És jó életet minden útainkban;
 És Szentlelkeddel vigy be hajlékodba,
 A boldogságba.
/8
#40536984
 Dicsőség néked mennyben,
 Örök Isten, Ki dicsértetel a te szent igédben,
 Krisztus Jézusban, mi Idvezítőnkben,
 És Szentlélekben.

>Megtérés

>415. Hadd menjek, Istenem, Mindig feléd
/1
#F031FB8A
 Hadd menjek, Istenem, Mindig feléd,
 Fájdalmak útjain Mindig feléd.
 Ó, sok keresztje van,
 De ez az én utam,
 Mert hozzád visz,
 Uram, Mindig feléd.
/2
#DD7CE638
 Ha este száll reám
 S csöndes helyen
 Álomra hajtanám
 Fáradt fejem:
 Nem lesz hol nyughatom,
 Kő lesz a vánkosom,
 De álomszárnyakon
 Szállok feléd.
/3
#10A8F8F0
 Szívemtől trónodig -
 Mily szent csoda -
 Mennyei grádicsok
 Fényes sora,
 A szent angyalsereg
 Mind nékem integet;
 Ó, Uram, hadd megyek
 Én is feléd!
/4
#56AF12AC
 Álomlátás után
 Hajnal ha kél,
 Kínos kővánkosom
 Megáldom én.
 Templommá szentelem,
 Hogy fájdalmas szívem,
 Uram, hozzád vigyem,
 Mindig feléd!
/5
#7F1BDA8E
 Csillagvilágokat
 Elhagyva már,
 Elfáradt lelkem is
 Hazatalál. Hozzád ha eljutok,
 Lábadhoz roskadok:
 Ottan megnyugodhatok \.
 Örökre én!

>416. Ébredj, alvó, hív a szózat!
/1
#7004BC8D
 Ébredj, alvó, hív a szózat!
 Az őrök hangja meszsze elhat:
 Fel, Sion, ébredj, itt az Úr!
 Éjféltájra jár az óra,
 Mondd, kész vagy -é a hívó szóra,
 Vagy szíved mécse meg se gyúl?
 Ím, vőlegényed jő,
 Már kész a menyegző, Halleluja!
 Ó, halld szavát, Kelj föl tehát,
 Vedd menyegzői díszruhád!
/2
#B379DC5E
 Halljuk jól az őrök hangját,
 És örvendeznek, akik hallják,
 Hisz itt a várva várt idő.
 Íme, üdvünk napja virrad,
 mert fölkelt már a Hajnalcsillag,
 S az igazságos bíró jő.
 Szent trónusod fényben áll!
 Jöjj, Jézus, néped vár! Halleluja!
 Mi elmegyünk, Mindegyikünk,
 Hogy menyegződön ott legyünk.
/3
#762F5B7C
 Akkor trónod köré állunk,
 S az angyalokkal téged áldunk,
 Zeng cimbalom és hárfahúr.
 Földi lényt ott ami várja,
 Fül nem hallotta, szem nem látta,
 A szívünk háladalra gyúl.
 Oly szép az égi hon!
 Sok gyöngykapuja van,
 Halleluja! Zeng énekünk,
 Az Úr velünk:
 Légy áldott érte, Istenünk!

>417. Ébredj, bizonyságtévő Lélek!
/1
#2339CE3C
 Ébredj, bizonyságtévő Lélek!
 A várfalakra őrök álljanak,
 Kik bátran szólnak harcra készek,
 Ha éj borul le, vagy ha kél a nap.
 Hívásuk zengjen meszsze szerteszét,
 Az Úrhoz gyűjtve népek seregét!
/2
#040FC424
 Ó, bárha lángod fellobogna
 S ébredne föl sok nemzet fényinél
 Ó, bár sok szolga, sarlót fogva,
 Aratna, mígnem leborul az éj!
 Urunk, e roppant, ért vetésre nézz:
 A munka sok, a munkás oly kevés!
/3
#D5F6B75C
 Küldd útra hírnökid csapatját,
 És adj erőt onnan felül nekik,
 Hogy veszni a pogányt se hagyják,
 És szerteűzzék Sátán seregit.
 Országod jöjjön el minél elébb,
 Hirdetve szent neved dicséretét!

>418. Halld meg, bűnös ember: Jézus szeret!
/1
#C2D58F0E
 Halld meg, bűnös ember: Jézus szeret!
 Bűnöd bár nagy tenger, Jézus szeret.
 Csak jöjj hozzá hittel,
 Töredelmes szívvel,
 Nem vet ő el, hidd el,
 Jézus szeret.
/2
#0412A121
 Légy hát, szívem, bátor: Jézus szeret!
 Megvéd a hű pásztor: Jézus szeret.
 Szent igéje mondja,
 Hogy van reám gondja;
 Hitem is suttogja:
 Jézus szeret.
/3
#E1A47F93
 Ez a boldogságom: Jézus szeret!
 Nincs más orvosságom: Jézus szeret.
 Bár a kereszt rajtam,
 E világi zajban
 Egyre zengi ajkam:
 Jézus szeret.
/4
#0EE1B2B2
 Nem rémít a halál: Jézus szeret!
 Ott fenn ő maga vár: Jézus szeret.
 Mennyben élek újra, \.
 Angyalikar zúgja:
 Béke, halleluja!
 Jézus szeret!
/5
#C9EF26A9
 Zengjük hát, testvérek:
 Jézus szeret!
 Hallják minden népek:
 Jézus szeret!
 Hozzá hívek légyünk,
 Tőle el ne térjünk,
 Zengjük, amíg élünk:
 Jézus szeret!

>419. Ó, Sion, ébredj, töltsd be küldetésed
/1
#4287F487
 Ó, Sion, ébredj, töltsd be küldetésed,
 Mondd a világnak: hajnalod közel!
 Mert nem hagy az, ki népeket teremtett,
 Senkit sem éjben, bűnben veszni el.
 Légy örömmondó békekövet,
 Hirdesd: a Szabadító elközelgetett!
/2
#AFD6F1C1
 Lásd: millióknak lelke megkötözve,
 Rabláncként hordoz sötét bűnöket;
 Nincs kitől hallja: Megváltónk keresztje
 Mily gazdag élet kútja lett neked.
 Légy örömmondó békekövet,
 Hirdesd: a szabadító elközelgetett!
/3
#A2B8635C
 Mondd minden népnek: elveszett juháért
 Mit tett a Pásztor - csuda szerelem -
 Földig hajolt a kárhozott világért
 S meghalt alant, hogy élhess odafenn.
 Légy örömmondó békekövet,
 Hirdesd: a Szabadító elközelgetett!
/4
#8B4904C7
 Küldj fiaidból, akik nemhiába
 Élvezik kincsed: Hirdessék szavad;
 Öntsd lelked értük győzelmes imába:
 Mindent, mit adtál, Krisztus visszaad.
 Légy örömmondó békekövet,
 Hirdesd: a Szabadító elközelgetett!
/5
#3E9AB743
 Ő visszajön, Sion, előbb, mint véled,
 Felfedi titkát minden szív előtt.
 Egy lélekért se érjen vádja téged,
 Hogy temiattad nem látta meg Őt.
 Légy örömmondó békekövet,
 Hirdesd: a Szabadító elközelgetett!

>420. Jobban tiéd, Uram, Jobban tiéd!
/1
#022E798E
 Jobban tiéd, Uram, Jobban tiéd!
 Bár súlyos a kereszt, érzem sebét.
 Bár könynye, kínja van,
 Ez légyen jelszavam:
 Jobban tiéd, Uram,
 Jobban tiéd! S
/2
#54A82D16
 Bár mint Jákóbra ott, Éj száll reám,
 S nem lesz, hol nyughatom,
 Egy kő csupán; Ott is álmomban még
 Szívem egy vágytól ég;
 Jobban tiéd, Uram, Jobban tiéd!
/3
#DFB8719D
 Az álom s éj után Kél újra fény,
 S új hévvel a követ Megáldom én.
 Így lészek bajban még, Uram, jobban tiéd!
 Jobban tiéd, Uram, Jobban tiéd!
/4
#2C3B0B04
 És bár a keskeny út Meredek is,
 De mégis mennybe jut, Dicsfénybe visz.
 Dicső angyalsereg Int ottan, hogy legyek
 Jobban tiéd, Uram, Jobban tiéd!
/5
#1CAE4A2E
 S ha majdan szárnyain Lelkem repül,
 S túl csillag százain Hazakerül,
 Meglesz főkincse még, Hogy lészen a tiéd,
 Jobban tiéd, Uram, Jobban tiéd!

>421. Az egyháznak a Jézus a fundámentoma
/1
#E37FCF77
 Az egyháznak a Jézus a fundámentoma,
 A szent Igére épült fel lelki temploma.
 Leszállt a mennyből hívni és eljegyezni őt,
 Megváltva drága vérén a váltságban hivőt.
/2
#502AEBB7
 Kihívott minden népből egy lelki népet itt,
 Kit egy Úr, egy keresztség és egy hit egyesít.
 Csak egy nevet magasztal, csak egy cél vonja őt,
 És egy terített asztal ád néki új erőt.
/3
#E28E1AD9
 A világ fejedelme feltámad ellene,
 Vagy hamis tudománytól gyaláztatik neve,
 S míg egykor felderül majd az Úrnak hajnala,
 Csak virrasztói kérdik: „Meddig az éjszaka?”
/4
#A744B5BD
 Sok bajban, küzdelemben meghajszolt, megvetett,
 De szent megújulásért És békéért eped,
 Míg látomása egykor dicsőn teljesül
 S a győzelmes egyház Urával egyesül.
/5
#CC53A56B
 A három-egy Istennel már itt a földön egy
 S az üdvözült sereggel egy nép és egy sereg.
 Ó, mily áldott reménység: ha itt időnk lejár,
 Te boldog szenteiddel fenn Nálad béke vár!

>422. Boldogok, akik lelki szegények
/1
#82FE4732
 Boldogok, akik lelki szegények,
 Szívből könyörögnek,
 Mert a mennyország adatik nékiek.
/2
#0C377B56
 Boldogok, akik sírván bánkódnak,
 És károkat vallnak,
 Mert Istentől ők megvigasztaltatnak.
/3
#5CC820F9
 Boldogok azok, akik szelídek
 És nem háborognak,
 Mert ők e földön örökséget bírnak.
/4
#8839A9C9
 Boldogok, akik az igazságot
 Éhezik, szomjazzák,
 Mert Istentől ők megelégíttetnek.
/5
#66DFBF09
 Boldogok azok, kik irgalmasok,
 Máson könyörülnek, Irgalmasságot mert
 Istentől nyernek.
/6
#3861A245
 Boldogok, akik tiszta szívűek,
 Hitből megtisztultak,
 Mert megismervén az Istent, meglátják.
/7
#F64CDCE3
 Boldogok, akik békességszerzők,
 Mert ők hívattatnak
 Az igaz hitben Isten fiainak.
/8
#3278D0ED
 Boldogok, akik az igazságért
 Üldözést szenvednek,
 Mert a mennyország adatik ezeknek.
/9
#71959EBA
 Boldogok lesztek, mikor
 Krisztusért Emberek gyűlölnek,
 Megszidalmaznak s gonoszul üldöznek.
/10
#CD7F216E
 Akkor azért hát örvendezzetek
 és vigadozzatok,
 Mert mennyben lészen
 ti nagy jutalmatok.

>423. Csak vezess, Uram, végig, és fogd kezem
/1
#68012EAB
 Csak vezess, Uram, végig, és fogd kezem,
 Míg boldogan a célhoz elérkezem,
 Mert nélküled az én erőm oly kevés,
 De hol te jársz előttem, nincs rettegés.
/2
#03828769
 Szent irgalmaddal szívemet födjed bé,
 Tedd örömben és bánatban csöndessé,
 Hogy hadd pihenjen lábadnál gyermeked,
 Ki szemlehunyva téged híven követ.
/3
#BBFC065E
 Ha gyarlóságom meg nem is érzené:
 A vak homályból te mutatsz ég felé;
 Csak vezess, Uram, végig, és fogd kezem,
 Míg boldogan a célhoz elérkezem.

>424. Ez a világ csak baj halma, Nincs itt
/1
#8F6E637E
 Ez a világ csak baj halma,
 Nincs itt senkinek nyugalma,
 Minden naphoz új küzdelmet,
 Vállainkra rak sok terhet.
/2
#B16FFEDE
 Mint őz, szarvas a vizekre,
 Vágyom azért oly helyekre,
 Ahol lelkem nyugtot talál,
 Jézusommal egy úton jár.
/3
#A1702B7B
 Legyen bár tövissel rakva,
 Nyugodt szívvel járok rajta,
 Jézus terhe könnyű s édes,
 Igája gyönyörűséges.

>425. Hinni taníts, Uram, kérni taníts!
/1
#AB4693BC
 Hinni taníts, Uram, kérni taníts!
 Gyermeki, nagy hitet kérni taníts!
 Indítsd fel szívemet,
 Buzduljon fel, neked
 Gyűjteni lelkeket! Kérni taníts!
/2
#2AAB18FA
 Hinni taníts, Uram, kérni taníts!
 Lélekből, lelkesen kérni taníts!
 Üdvözítőm te vagy,
 Észt, erőt, szívet adj.
 Lelkeddel el ne hagyj! Kérni taníts!
/3
#7178293E
 Hinni taníts, Uram, kérni taníts!
 Gyorsan elszáll a perc: kérni taníts!
 Lásd gyengeségemet, Erősíts engemet,
 Míg diadalt nyerek: Kérni taníts!
/4
#6D337545
 Hinni taníts, Uram, kérni taníts!
 Jézus, te visszajössz: várni taníts!
 Majd ha kegyelmesen Nézed az életem:
 Állhassak csendesen. Hinni taníts!

>426. Hiszek egy Istenben, Hatalmas Atyában
/1
#5D8B823A
 Hiszek egy Istenben, Hatalmas Atyában,
 Mindent teremtőben, És Jézus Krisztusban,
 Atya egy Fiában, Mi Urunk nevében,
 Aki fogantaték Szentlélek Istentől,
 Máriától születék, Szenvede kínzatott
 Poncius Pilátus Tiszttartósága alatt.
/2
#88E799C0
 Meg is feszítteték, Azután meghala
 És eltemetteték, Szálla a poklokra,
 Harmadnap támada, Mennybe emelteték,
 Ül a mindenható Atya Isten jobbján,
 Onnan lész eljövendő Az eleveneket
 Ítélni s holtakat. Hiszek a Szentlélekben.
/3
#281DA042
 Hiszek egy keresztyén Anyaszentegyházat,
 Szentek egyességét, Bűnünk bocsánatját,
 Test feltámadását És örök életet.
 A szent prófétákkal És apostolokkal
 Mi mindezt hisszük s valljuk,
 Ez hittel, vallással És szent tudománnyal
 Magunkat bátorítjuk.

>427. Hogyha éltünk vándorútja
/1
#97B112AA
 Hogyha éltünk vándorútja
 Néha sötét völgybe tér,
 Jézus ott se hagy magunkra,
 Szava biztat: jer, ne félj!
 Megterheltek, jöjjetek,
 Újuljon meg lelketek,
 Jöjjetek, ti megfáradtak,
 Találjatok nyugodalmat!
/2
#36BE7067
 Ti, kiket az élet terhel,
 Aggasztanak a bajok,
 Kebletekből este, reggel
 Szállnak nehéz sóhajok,
 Sorsotok bár mostoha,
 Isten nem hagy el soha,
 Jöjjetek, ti megfáradtak,
 Találjatok nyugodalmat!
/3
#E66AF3BC
 Kiket nyom a vétek súlya,
 A bűn miatt betegek,
 Azokat is Jézus hívja:
 Megterheltek, jöjjetek!
 A bűnösnek kegyelem
 Az én elégtételem,
 Jöjjetek, ti megfáradtak,
 Találjatok nyugodalmat!
/4
#63EEBDC4
 Ó, ti, kiknek kezetekben
 Könnytől ázik a kenyér,
 Egyik felhő tovalebben,
 Már a másik visszatér;
 Csüggedt szívek, higgyetek,
 Gondol Isten veletek,
 Jöjjetek, ti megfáradtak,
 Találjatok nyugodalmat!
/5
#06BE8EDC
 Megfáradtak, megterheltek,
 Csak énhozzám jöjjetek;
 Élő hittel bízva jertek,
 És megújul Lelketek!
 Kik az Istent szeretik,
 Javukra lesz az nekik,
 Jöjjetek, ti megfáradtak,
 Találjatok nyugodalmat!

>428. Istennel járni, lakozni
/1
#231268B2
 Istennel járni, lakozni,
 Szent élettel illatozni,
 Igaz hitben nem habozni:
 Jézus Krisztus, taníts,
 Taníts imádkozni!

>429. Jer, Krisztus népe, nagy vígan
/1
#DAD62984
 Jer, Krisztus népe, nagy vígan
 Mind egy örömre keljünk,
 És egybeforrva, boldogan
 Csak arról énekeljünk,
 Hogy könyörülő Istenünk
 Mily áldott csodát tett velünk,
 Mit drágán szerzett nékünk.
/2
#AFFDB92F
 A Sátán tett rám rabigát
 És már halálba vesztem,
 A bűn gyötört éj- s napon át,
 Mert benne gyökereztem.
 Mind jobban elsüllyedtem én,
 Nem volt számomra már remény,
 Megült a bűnnek átka.
/3
#FDB0F163
 Nem használt, sőt káromra lett,
 Ha bármi jót is tettem,
 Mert az égi ítéletet
 Önkényből megvetettem.
 Mégis kínzott a félelem,
 Hogy csak a halál van velem
 És a poklokra hullok.
/4
#5274C6FC
 Ám az örök szent irgalom
 Nagy ínségem megszánta,
 És könyörülvén ily bajon,
 Megenyhítni kívánta.
 Mint jó Atya, szívébe vett,
 Nem játék volt, amit megtett
 Legfőbb kincsével értem.
/5
#3E8902EC
 Így szólott Egyszülöttjéhez:
 Jött irgalomnak éve,
 Én diadalmam, menj, siess,
 Légy népem üdvössége,
 Bűn átkából segítsd ki hát,
 Fojtsd meg a ráleső halált:
 Az embert térítsd hozzád.
/6
#C5137ECE
 S ím, az Atyának engedett
 A Fiú, hozzám jöve,
 Egy tiszta szűztől született
 Testvéremül a földre.
 Járt mint nagy titkos hatalom,
 Felölté önnön alakom,
 Hogy a Sátánt lebírja.
/7
#77497F91
 Szólt hozzám: tarts ki már velem,
 Most célod el kell érned,
 Immár enyém a küzdelem,
 Kiállok készen érted.
 Te az enyém, én tied,
 S hol én vagyok, ott lesz helyed:
 Szét nem választ az ellen.
/8
#929D78D2
 És bár kioltja életem,
 És bár kiontja vérem:
 Mindezt javadra szenvedem,
 Hű légy e hitben vélem.
 A halált éltem megveszi
 És szentségem jóváteszi
 A bűnt, hogy üdvözülhess.
/9
#5BF558B1
 Én az Atyához felmegyek,
 Ha végeztem a földön,
 Hogy aztán Mestered legyek,
 A Lelket rád kitöltöm,
 Ki félelmedben bátorít,
 S hogy engem ismerj,
 megtanít És igazságban járat.
/10
#E7C8BC5B
 Mit tettem és hirdettem én,
 Azt kövesd szóban, tettben,
 Az Úr országát építvén
 És dicsőségét egyben.
 Ne hagyd, hogy hívság s emberek
 Megrontsák lelki kincsedet:
 Ez légyen örök részed!

>430. "Imádkozzatok és buzgón kérjetek!"
/1
#3308FADC
 "Imádkozzatok és buzgón kérjetek!"
 Bűnös voltunkért, Uram, ó, ne vess meg!
 Tiszta szívet és Szentlelket adj nékünk,
 Hallgass meg Fiad nevébe', ha kérünk.
/2
#4EB7DB67
 „Keressetek buzgón és megtaláltok!” -
 Téged keresünk, Uram: hogy bűn s átok
 Erőt ne vegyen mirajtunk, légy nékünk
 Égi utunk, igazságunk, életünk!
/3
#F783E3AD
 „Zörgessetek buzgón Isten ajtaján!” -
 Elfáradtunk, Uram, e világ zaján;
 Ó, nyisd meg az égi béke szép honát,
 Add, hogy zenghessünk örök halleluját!

>431. Lelkünk hozzád kívánkozik, Óhajtozik
/1
#1D833574
 Lelkünk hozzád kívánkozik,
 Óhajtozik, Ó, életnek kútfeje!
 Mert nálad talál oltalmat,
 Nyugodalmat, Idvességünk ereje!
/2
#204D17B3
 Nem bízhatunk érdemünkhöz,
 Mert testünkhöz Köttetett a gyarlóság;
 Éltünket, ha megvizsgáljuk,
 Úgy találjuk:
 Nincs bennünk teljes jóság.
/3
#AC755643
 Az Úrnak parancsolatját, Akaratját,
 Ó, hányszor hágtuk által!
 Hányszor volt munkánk éretlen
 És rendetlen Erőtlenségünk által!
/4
#DBC73CCA
 Hogyha lelkünk jóra gerjed,
 Hamar csügged, Mivel a halandó test
 Szüntelenül ostromolja, Meggátolja,
 S így lesz az a jóban rest.
/5
#7A718282
 Ily bűnös gyarlóságunkkal Mi Urunkkal
 Szemben meg nem állhatunk;
 Tettetett igazságunkért Jutalmat s bért
 Magunknak nem várhatunk.
/6
#08366785
 Tudjuk, hogy a kegyességnek
 És hűségnek Vagyon drága jutalma;
 De ezt adja kegyelemből
 S nem érdemből
 Az Istennek irgalma.
/7
#F87D16C7
 Mi tehát ez oltalomhoz Mint kőfalhoz
 Egyedül támaszkodunk;
 Jézusunknak kegyelméhez,
 Érdeméhez
 Hit által ragaszkodunk.

>432. Mit használ keresztyénségem
/1
#1131421F
 Mit használ keresztyénségem,
 Ha nem aszerint élek,
 Ha nincs igaz kegyességem,
 És vétkezni nem félek?
 Ha az Úrnak ösvényét
 Tudván, rontom törvényét,
 azt, aki meghalt érettem,
 Csak szám áldja nem életem?
/2
#DC2226BA
 Mit használ nekem a jó hit,
 Ha azt csak nyelvvel vallom,
 De nem termem gyümölcseit,
 Inkább a bűnt javallom?
 Ha rajtam erőt vesznek
 És rabszolgává tesznek
 A megveszett indulatok,
 Úgy, hogy velük nem bírhatok?
/3
#E6F184C3
 S vajon nyugalmát hol leli
 Szívem, mikor szenvedek?
 Ha lelkemet bú terheli,
 Amely szinte eltemet?
 Mikor a halál rettent,
 S az, aki igaz és szent,
 Megítéli életemet,
 S felfedi minden vétkemet?
/4
#78BCCA39
 Segíts, Uram, hogy éltemet
 Csak tenéked szenteljem,
 S minden cselekedetemet
 Kedved szerint rendeljem!
 Nyújts erőt és értelmet,
 Minden lelki sérelmet
 Hogy gondosan elkerüljek,
 S végre mennyben  örvendezzek.

>433. "Szólj, szólj  hozzám, Uram, mert szolgád hallja szódat!"
/1
#3FBCA68D
 "Szólj, szólj  hozzám, Uram,
 mert szolgád hallja szódat!"
 Így mondom, mert magam
 rég annak érezem.
 Hadd járjak utadon,
 hadd várjam égi jódat
 Hű szívvel szüntelen,
 hű szívvel szüntelen.
/2
#1B9B47B6
 Adj lelkedből erőt,
 hogy értsem és szeressem
 Elrendelt utamat s
 minden parancsodat.
 Egy vágyat hagyj nekem:
 hogy halljam és kövessem
 Szent igazságodat,
 szent igazságodat.
/3
#267F53AF
 Nincs oly tudós sehol,
 ki megtanít utadra,
 A bölcs nem fejti meg
 törvényedet sosem;
 Te fejted meg nekünk,
 te, hű szíveknek Atyja,
 Kinek szavát lesem,
 kinek szavát lesem.
/4
#B306A5E3
 Te nagy csodáidról bár
 fennszóval beszélnek
 És fennen hirdetik
 felséges rendedet,
 Ha nem te szólsz,
 Uram, a szó fülig ha érhet,
 De szívig nem mehet,
 de szívig nem mehet.
/5
#9ECFE328
 Szólj, szólj, én Istenem! -
 szól hangodból a jóság,
 A lelkem megfeszül
 s a hallásban segít,
 És szódban meglelem
 az örökkévalóság
 Jó édességeit, jó édességeit.
/6
#F5565345
 Szólj és csitítsd a bút,
 mert bú és kín gyötörnek,
 Szólj, hogy legyen szavad
 ír s gyógyító erő;
 Szólj, dicsőséged úgy
 még szebben tündökölhet,
 És mindörökre nő,
 és mindörökre nő.

>434. Szólsz hozzám, Istenem, s én választ adni készen
/1
#C0979BCC
 Szólsz hozzám, Istenem,
 s én választ adni készen
 Mármár megindulok,
 hogy rád bízzam magam,
 De látod, köt s lehúz
 még régi csüggedésem,
 Áldd meg ma lelkemet
 több hittel, ó, Uram.
/2
#5F49EE55
 Sok szép ígéretem, ó,
 hányszor megtagadtam,
 A nagy fogadkozást,
 hogy csak tiéd szívem.
 A bűnös gyengeség bús
 rabjának maradtam,
 És törvényed szerint
 nem éltem semmiben.
/3
#3966CE61
 Ha jót tettél velem,
 ha áldva látogattál:
 Én nem dicsértelek s
 nem hirdettem neved;
 Nem értettem, mikor
 szenvedni, sírni hagytál,
 Hogy ha szeretsz, miért
 sújt vessződ engemet?
/4
#86277E29
 Köt még a földi jó,
 a bűn, a földi örvény,
 S tehozzád bűnömért,
 lásd, el nem juthatok.
 A béklyó súlya nyom,
 levetném, összetörném,
 De lelkem gyenge még
 s jaj, összeroskadok.
/5
#3AF5A7AA
 Más nem tanít meg rá,
 csak égi bölcsességed,
 Hogy bölcsen bízzak és
 szolgáljak úgy neked.
 Mit érek nélküled?
 Add, hogy imádva téged,
 Bús, gyarló bűnös én,
 hadd légyek gyermeked.
/6
#40979749
 Nagy lelked élt, Uram,
 a prófétás időkben,
 Az fénylett át a szent
 s apostol életén;
 Áldj meg s kegyelmedet
 reám is töltsd ki bőven,
 Hogy Jézust nézzem és
 ővéle győzzek én.

>435. Uram, a te igéd nekem
/1
#8E0399EC
 Uram, a te igéd nekem
 A sötétben szövétnekem;
 Mind igazak és ámenek,
 Amik szádból kijöttenek,
 Azért amit nem látok szemmel,
 Béveszem szavadra hitemmel.
/2
#4D120D11
 Bízom hozzád erős hittel,
 Hogy te mindent megcselekszel,
 Amit szent igédben ígérsz:
 Hogy kegyelmesen hozzám térsz,
 És megbocsátván bűneimet,
 Megadod örök életemet.
/3
#0317C7A0
 E nagy jót neked köszönöm,
 Mely nekem arra ösztönöm,
 Hogy a Jézust, kiért velem
 Közöltetik a kegyelem,
 Tartsam lelkem megtartójának,
 Szeressem, engedvén szavának.
/4
#AB6EFD3A
 Igazgass, Uram, engemet,
 Hogy megőrizzem hitemet;
 Ha von magához e világ,
 Én mint Krisztusba oltott ág,
 Tőle vegyem tápláltatásom,
 Míg az élők közt lesz lakásom.

>436. Vezess, Jézusunk, S véled indulunk
/1
#DCF3163F
 Vezess, Jézusunk,
 S véled indulunk.
 Küzdelemre hív az élet,
 Hadd kövessünk benne téged;
 Fogjad a kezünk,
 Míg megérkezünk.
/2
#94E8886D
 Adj erős szívet,
 Hogy legyünk hívek.
 És ha terhet kell viselnünk,
 Panaszt mégsem ejt a nyelvünk;
 Rögös bár utunk,
 Hozzád így jutunk.
/3
#56CD6CB7
 Sebzett szívünk majd
 Mikor felsóhajt,
 Vagy ha másért bánat éget,
 Adj türelmet, békességet,
 Reménnyel teli
 Rád tekinteni.
/4
#B537B899
 Kísérd lépteink
 Éltünk végéig,
 És ha roskadozva járunk,
 Benned támaszt hadd találunk,
 Míg elfogy az út
 S mennyben nyitsz kaput.

>Énekek bibliaórákra, vasárnapi iskolai és családi alkalmakra

>Karácsony

>437. Csendes éj, szentséges éj!
/1
#F34E1F25
 Csendes éj, szentséges éj!
 Mindenek nyugta mély,
 Nincs fenn más, csak a szent szülepár,
 Drága kisdedük álmainál,
 Szent Fiú, aludjál,-
 Szent Fiú, aludjál!
/2
#E057F7D8
 Csendes éj, szentséges éj!
 Angyalok hangja kél;
 Halld a mennyei halleluját,
 Szerte zengi e drága szavát:
 Krisztus megszabadít,
 Krisztus megszabadít!
/3
#E8A90FF2
 Csendes éj, szentséges éj!
 Szív, örülj, higgy, remélj!
 Isten szent Fia hinti reád
 Ajka vigaszadó mosolyát:
 Krisztus megszületett,
 Krisztus megszületett!

>438. Halld, mint zeng az egész ég
/1
#B777324E
 Halld, mint zeng az egész ég:
 "A Királynak dicsőség!
 Békesség a földön lenn,
 Istentől jő kegyelem."
 Népek, örvendezzetek!
 Visszhangozzák a Menynyek!
 Hirdesse a természet:
 Krisztusunk megszületett!
 Halld, mint zeng az egész ég:
 "A Királynak dicsőség!"
/2
#B7D2A9B5
 Krisztus, kit imád a Menny,
 Úr a nagy világ előtt, köztünk,
 íme, megjelent,
 Szent szűz az, ki szülte őt.
 Dicsősége eget áthat, Isten
 Ő, ki porruhánkat
 Földön élni vette fel. Jézus
 Ő, Immánuel!
 Halld, mint zeng az egész ég:
 „A Királynak dicsőség!”
/3
#15C0C8E2
 Dicsőség! Ő a Király, Békesség és Igazság,
 Éltető világos nap, Bajainkra gyógyírt ád.
 Kis gyermek lett miérettünk,
 Született, hogy mi élhessünk,
 Porból hogy feltámasszon,
 S újjászületést adjon,
 Halld, mint zeng az egész ég:
 „A Királynak dicsőség!”

>439. Kicsiny Betlehemben felzendült az ének
/1
#25240546
 Kicsiny Betlehemben felzendült az ének:
 "Dicséret, dicsőség Az ég Istenének!"
 Kit oly régtől fogva szívszakadva vártak,
 Megszületett Megváltója a bűnös világnak.
/2
#C3B379EB
 Zengjen ajkunkról is hálaadó ének
 Estéjén karácsony áldott ünnepének;
 És ki bűneinkért földre alászálla,
 Szívesen várt vendég legyen lelkünk
 Messiása!

>440. Csillagfényes éjszakán angyalszózat hangzik
/1
#A7673073
 Csillagfényes éjszakán angyalszózat hangzik,
 Betlehemnek pusztáján, hol pásztornép tanyázik:
 Pásztorok, ne féljetek, vígan örvendezzetek,
 Mert földre jött, kit elküldött az Úr, az Isten néktek:
 A ti üdvösségtek!
/2
#6930634A
 Hideg jászol szalmáján égi gyermek alszik,
 Körülötte szállásán ének szava hangzik:
 Imádják a pásztorok, tiszta szívű jámborok.
 Örvendeznek, hálát zengnek az Úr nagy jóvoltáért,
 Egyszülött Fiáért.

>Nagypéntek

>441. Szomorúan sóhajt szívünk
/1
#DDC3C064
 Szomorúan sóhajt szívünk
 Fel tehozzád, jó Istenünk:
 Ó, tekints ránk, míg elődbe
 Leborulunk könyörögve!
 Nézd sebét a könynyezőnek,
 A bűn miatt kesergőnek,
 S irgalmazz a megtérőnek!

>442. A keresztfához megyek
/1
#8C098085
 A keresztfához megyek,
 Mert máshol nem lelhetek
 Nyugodalmat lelkemnek.
 Ott könnyárban leborulok,
 Bűnterhemtől szabadulok,
 Hol a bűntelen szenved.
/2
#4D664FCD
 Isten ama Báránya
 Hogy jut a keresztfára,
 Mért foly alá szent vére?
 Értem tette ezt Jézusom,
 Fájó szívvel hiszem, tudom,
 Lelkem örök üdvére.
/3
#36043C10
 Én voltam az elveszett,
 Aki halált érdemelt,
 Kire gyászos éj szakadt.
 Isten Fia, a jó Pásztor
 Megváltott a bűn átkától,
 Üdvöm, napom feltámadt.
/4
#E1733F52
 Csodás égi szeretet!
 Szívem mindent elfeled,
 Ami engem földre von.
 Csak rád nézek, én Megváltóm,
 Szereteted sírig áldom,
 Győzedelmes Krisztusom!
/5
#12EBD988
 Tied vagyok, Jézusom,
 Megyek keresztutadon,
 Nyugtot, békét így lelek.
 A szeretet hozzád kapcsol,
 Boldog, aki híven harcol,
 Veled mindig győzhetek!

>443. Én nem tudom, mért szeretett úgy minket
/1
#AF30144F
 Én nem tudom, mért szeretett úgy minket,
 Kit hódoló angyalsereg imád;
 S miért akart ő bűnös tévelygőket
 Keresni úgy, mint elveszett juhát.
 Csak azt tudom: lejött közénk a földre,
 Hol jászolágy volt első otthona,
 S hogy kicsiny Názáret ácsműhelyén át
 Eljött Megváltónk, Isten egyszülött Fia.
/2
#E119DCB8
 Én nem tudom, mint hódol majd előtte
 Egész világ, ha minden vész elül.
 Milyen dicső öröm lesz, hogyha végre
 Szeretettel a szív beteljesül.
 Csak azt tudom: Az ég is harsonázik,
 És karba' zeng a lelkes emberár,
 Az ég, a föld is ujjongja egymásnak,
 hogy a világ üdvözítője a Király.

>444. Ó, Úr Jézus, menynyi bánat
/1
#895AFE8F
 Ó, Úr Jézus, menynyi bánat,
 Menynyi sóhaj száll utánad
 Az örök gyász e nagy napján,
 Míg ott függesz a keresztfán!
 Keresztedre búsan nézünk,
 Érted, veled mi is vérzünk,
 Mert miattunk omlott véred,
 Bűneinkért lett ily véged.
/2
#27D97713
 Sírva nézzük gyötrődésed,
 Sírva halljuk könyörgésed,
 Mely Atyádhoz értünk szállott,
 Fájt a lelked, mégis áldott!
 Átvert tested még vonaglott,
 Ajkadon bocsánat hangzott,
 És a népre átok helyett
 Áldást esdett nagy szerelmed.
/3
#788C5590
 Te, ki Isten Fia voltál,
 Életeddel így áldoztál;
 Hirdetted a békességet,
 A halálra vitt az téged! \.
 Ó, Úr Jézus, kik itt állunk,
 És biztató szódra várunk,
 Nézd bánatos szíveinket,
 Keresztednél áldj meg minket!
/4
#67942600
 Szent hőse a Golgotának,
 Add, hogy menjünk teutánad!
 Kövessük azt a szent utat,
 Mely Istenhez mennybe juttat;
 Vegyük fel mi is kereszted,
 Bár alatta szívünk reszket,
 Mert szenvedés visz az égbe,
 Föl az örök dicsőségbe.

>Húsvét

>445. Örvendezz, örvendezz, minden nép
/1
#B2C9A3D0
 Örvendezz, örvendezz, minden nép,
 Vidd a hírt, vidd a hírt szerte szét:
 Jézus legyőzte a sírok éjjelét.
/2
#1D6B688A
 Sziklakő, sziklakő nincsen ott,
 Szólanak, szólanak angyalok;
 Húsvét hajnalán Jézus feltámadott!
/3
#EDC5BD82
 Él az Úr, él az Úr, mit se félj,
 Benne higgy, benne higgy, jót remélj:
 Jézust meglátja mind, aki hitben él!
/4
#C44E0DFA
 Mondjunk hát, mondjunk hát éneket,
 Hirdessék, hirdessék mindenek:
 Jézus által nyerünk örök életet.

>Énekek Jézusról

>446. Az áldott orvos közeleg
/1
#0CD73325
 Az áldott orvos közeleg,
 A drága főpap: Jézus;
 Szava szívünk enyhíti meg,
 Egyetlen üdvünk: Jézus.
 Halld, mint zeng az égi kar
 Édes visszhangjaival!
 Szívemben is zeng e dal:
 Jézus, Jézus, Jézus!
/2
#E843FC1D
 Ki minden vétket megbocsát
 S bűnünk eltörli: Jézus.
 Megnyitja a menny kapuját
 S vezérel minket Jézus.
 Halld, mint zeng az égi kar
 Édes visszhangjaival!
 Szívemben is zeng e dal:
 Jézus, Jézus, Jézus!
/3
#1A5259A0
 Kiszenvedt Bárány, tisztelet
 S dicséret néked, Jézus!
 Te vagy a legfőbb szeretet,
 Csak érted égek, Jézus.
 Halld, mint zeng az égi kar
 Édes visszhangjaival!
 Szívemben is zeng e dal:
 Jézus, Jézus, Jézus!
/4
#688F6514
 Elmúlik minden fájdalom
 E drága névtől: Jézus!
 Édes örömmel hallgatom
 A te nevedet, Jézus!
 Halld, mint zeng az égi kar
 Édes visszhangjaival!
 Szívemben is zeng e dal:
 Jézus, Jézus, Jézus!
/5
#C11B01C3
 Testvéreim, ó, jöjjetek,
 Áldjuk e nevet: Jézus!
 Dicsérve énekeljetek:
 Megváltó, drága Jézus!
 Halld, mint zeng az égi kar
 Édes visszhangjaival!
 Szívemben is zeng e dal:
 Jézus, Jézus, Jézus!
/6
#1EACD637
 Ti gyermekek, kicsik, nagyok,
 A tietek is Jézus.
 Csak az ő útján járjatok,
 És veletek lesz Jézus.
 Halld, mint zeng az égi kar
 Édes visszhangjaival!
 Szívemben is zeng e dal:
 Jézus, Jézus, Jézus!
/7
#F32F1027
 S ha egykor égbe térhetünk,
 Előttünk lesz majd Jézus.
 Trónjánál zendül énekünk
 E drága névtől: Jézus!
 Halld, mint zeng az égi kar
 Édes visszhangjaival!
 Szívemben is zeng e dal:
 Jézus, Jézus, Jézus!

>447. Jézus, nyájas és szelíd
/1
#CA55B17D
 Jézus, nyájas és szelíd,
 Láss meg engemet,
 Hallgassad meg, hű
 Megváltóm, gyermekedet!
/2
#A23E72F1
 Bűnöm láncát oldja fel
 Kegyelmed s a hit;
 Törjed össze balga
 szívem bálványait!
/3
#40DDFFAB
 Szabadságot adj nekem
 És tiszta szívet,
 Vonj magadhoz, Jézusom,
 hogy járjak veled!
/4
#5E8242D6
 Vezess engem utadon:
 Magad légy az út,
 Melyen lelkem a
 halálból életre jut.
/5
#A1E15473
 Jézus, nyájas és szelíd,
 Láss meg engemet:
 El ne engedd, hű
 Megváltóm, már kezemet!

>448. Isten testbe szállt szerelme
/1
#4856BD0E
 Isten testbe szállt szerelme
 Mennyből földre jött öröm,
 Téged hívunk esdekelve,
 És ujjongunk jöttödön.
 Jézus, teljes jóság vagy Te,
 Mert megszántál és szeretsz;
 Szállást venni jer szívünkbe,
 Megtartónk csak úgy lehetsz.
/2
#D62BC38C
 Ó, leheld rám áldott lelked,
 Nyughatatlan, lásd, szívem.
 Mindaddig, míg csendességet
 Nyervén Benned nem pihen.
 Oltsd el bennem a bűn vágyát,
 Kezdet légy Te és a vég;
 Add meg lelkem szabadságát,
 Melyben hit van s békesség.
/3
#7DE174AA
 Ó, jöjj vissza, Szabadító,
 Tőled nyerjünk életet;
 Jöjj sietve s immár többé
 El ne hagyjad gyermeked!
 Téged áldunk minden órán,
 És szolgálunk, szent Urunk,
 Mindig Hozzád óhajtozván,
 Áldunk és magasztalunk.
/4
#A449C109
 Végezd új teremtő munkád,
 Tiszták, szentek hadd legyünk.
 Add, hogy vágyva vágyjunk
 Hozzád, Míg tart földi életünk.
 Vigy a Mennybe, hol elődbe
 Mind lerakjuk koronánk,
 Áldva zengi nagy szerelmed
 Mindörökké szívünk, szánk.

>449. Jézus, te égi szép, tündöklő fényű név
/1
#FC7A4473
 Jézus, te égi szép, tündöklő fényű név,
 Legszentebb énnekem e föld ölén,
 Benned van irgalom, erőd magasztalom,
 Terólad zeng dalom, ragyogj felém!
/2
#DD7F59BF
 Az élet száz veszély.
 Én lelkem, mégse félj,
 Míg ő hord karjain, hű Mestered!
 Elhagynak emberek.
 Mit árt, ha ő veled?
 Töröld le könnyedet:
 Jézus szeret.
/3
#03423DC9
 Akaratod nekem mutasd
 meg szüntelen,
 Ne rejtsd el, Mesterem, tetszésedet!
 Vezérelj így magad, mutasd meg utadat!
 Idegen tájakon jár gyermeked.
/4
#29F28E96
 Tisztíts meg teljesen,
 szentelj meg, hadd legyen
 Fényedből fénysugár az életem,
 Míg a homályon át a lelkem otthonát,
 Világosságodat elérhetem!

>450. Krisztus, virágunk, Szép termőágunk
/1
#AD7455B4
 Krisztus, virágunk,
 Szép termőágunk,
 Feltámadt Krisztus, vigadjunk,
 Bűnünkből már feltámadjunk!
/2
#1A9AEB95
 Ki feküdt sírban,
 Felkele vígan.
 Feltámadt Krisztus, vigadjunk,
 Bűnünkből már feltámadjunk!
/3
#ED7E7A8D
 Felkele fényünk,
 Krisztus, reményünk.
 Feltámadt Krisztus, vigadjunk,
 Bűnünkből már feltámadjunk!
/4
#89F0B82D
 Jézus hogy felkelt,
 Szentírás bételt.
 Feltámadt Krisztus, vigadjunk,
 Bűnünkből már feltámadjunk!
/5
#8DCC1F49
 A pokol elhűlt, halál elrémült.
 Feltámadt Krisztus, vigadjunk,
 Bűnünkből már feltámadjunk!
/6
#0FCAE1EA
 Él Jézus nyilván,
 A bűnt lebírván.
 Feltámadt Krisztus, vigadjunk,
 Bűnünkből már feltámadjunk!
/7
#AB2CE034
 Jézusnak éljünk,
 Semmit ne féljünk!
 Feltámadt Krisztus, vigadjunk,
 Bűnünkből már feltámadjunk!
/8
#94B4E128
 Úr Jézus, nékünk
 Nagy fényességünk!
 Feltámadt Krisztus, vigadjunk,
 Bűnünkből már feltámadjunk!
/9
#21746A14
 Boldogság véled
 Földön az élet!
 Feltámadt Krisztus, vigadjunk,
 Bűnünkből már feltámadjunk!
/10
#42864C95
 Ha véled járunk,
 Jó halált várunk.
 Feltámadt Krisztus, vigadjunk,
 Bűnünkből már feltámadjunk!
/11
#E3AD198D
 Dicsőség légyen
 Istennek égben!
 Feltámadt Krisztus, vigadjunk,
 Bűnünkből már feltámadjunk!

>451. Te vagy napvilágom, te vagy énekem
/1
#E08F5965
 Te vagy napvilágom, te vagy énekem,
 Téged várlak, Jézus, kínos éjeken.
 Oszlasd el a homályt, űzd el a bánatot,
 Ragyogjon fel rajtam szép fényes napod.
/2
#63FF44C6
 Égő várakozás él minden szívben,
 Bárha nem is tudja, téged vár minden.
 Titkon epedőknek jelenjél meg nyilván,
 Találj meg mindenkit, ki élni kíván!
/3
#36C8BEF3
 Találj meg mindenkit, aki elveszett,
 S vigye dicsőségre hatalmas kezed;
 Hogy amikor eljössz az ég felhőiben,
 Örömmel lásson meg minden földi szem!

>Isten dicsőítése

>452. Mennyben lakó én Istenem
/1
#F4E1C175
 Mennyben lakó én Istenem,
 Vedd füledbe dicséretem!
 Téged dicsér egész világ,
 Néked köszön a kis virág.
 Erős vihar, kis gyenge szél,
 Tenger, patak rólad beszél;
 Az ég s a föld telve veled.
 Legyen áldott a te neved!
/2
#C456F945
 Mennyben lakó én Istenem,
 Könyörgök, légy mindig velem!
 Ha kél a fény napkeleten,
 A te szemed rajtam legyen;
 És ha leszáll napnyugaton,
 Te légy, Atyám, az oltalom!
 Terjeszd ki rám áldó kezed,
 Legyen áldott a te neved!

>453. Örök Isten, merre, merre vagy?
/1
#3ED9E01F
 Örök Isten, merre, merre vagy?
 Vajon hol van a te országod?
 Vágyik hozzád szívünk, ó, te nagy!
 Mert érzi a te nagy jóságod.
 Híveid imája, Mint galamb gyors szárnya,
 Száll utánad, hozzád repesve,
 Égen-földön téged keresve.
/2
#99A5715B
 A te csodás örök létednek
 Dicsőségét minden mutatja,
 A tengerek téged hirdetnek,
 Téged tükröz az égnek boltja;
 Az egész világon, Fent, alant s akárhol,
 Látszik a te dicső mivoltod,
 Mert az eget s földet te tartod.
/3
#4567E4B8
 De az Ige, a te hatalmad,
 Hirdet téged, Uram, legjobban;
 A mi szívünk ha arra hallgat,
 Úgy látunk meg igaz valódban;
 Látjuk nagy szerelmed,
 Mely ránk is kiterjed,
 És körülvesz, mint a fénylő nap,
 Mert mindenütt jelenvaló vagy!
/4
#0F6AA671
 Ó, Úr Isten, a te szent orcád
 Tündököljék mindig előttünk!
 Ki elvezet minket tehozzád:
 Szent Fiadat taníts követnünk,
 Hogy vele s utána Lelkünk megtalálja,
 Ami után annyit sóvárgott:
 A földön is a te országod!

>454. Nagy Isten, téged imád
/1
#F265223E
 Nagy Isten, téged imád
 Menny minden lakója,
 Hasson földről is hozzád
 Gyermekid hálája.
 Atya vagy menynyen, földön,
 Minden a te míved;
 Atyánk, megszentelődjön
 A te áldott neved!
/2
#CB0E624E
 Egész éltünk folytával
 Téged magasztalunk,
 Nap költe- vagy hunytával
 Jóságodért áldunk;
 Keresve igaz hazánk,
 Fényljen igazságod,
 Tegyük a jót, s így hozzánk
 Jöjjön el országod!
/3
#B4A66020
 Legjobb, amit te akarsz,
 Szent Isten, mindenhol;
 S bár sokat bölcsen takarsz,
 Szívünk neked hódol.
 Sorsunk te igazgatod,
 Bölcsen mívelsz mindent,
 Legyen meg akaratod
 E földön, mint ott fent!
/4
#20C6AB2E
 A kincs epesztő szomja
 És a nagy szegénység
 Lelkünk egyaránt nyomja,
 S kísértő ellenség;
 Nem kérünk túl a renden:
 Enyhítsd szükségünket,
 Add meg, Atyánk, a mindennapi
 kenyerünket.
/5
#DEBF3C1B
 Mi jó s szép, látja lelkünk,
 De terhel bűn súlya,
 S mennyet óhajtó keblünk
 Nyugodalmát dúlja;
 Atyánk, bocsásd meg bűnünk,
 Szíves megtérőknek,
 Mint mi is megengedünk
 Ellenünk vétőknek.
/6
#2790AC13
 A test, világ megrendít
 Hitünk- s erényünkbe';
 Világítsd utunk, s minket
 Ne vigy kísértésbe;
 De ha szédítő veszély
 Fenyeget s ostromol,
 Atyánk, te magad segélj
 S őrizz a gonosztól!
/7
#8548735B
 Hozzád, áldás forrása,
 Bízva esedezünk;
 Jézus közbenjárása
 Kezes miérettünk.
 Tiéd a föld és az ég,
 Országolsz mindenen,
 Tiéd erő s dicsőség,
 Mindörökké, ámen!

>455. Nagy vagy, te Isten, nagy a te hatalmad
/1
#F651EE84
 Nagy vagy, te Isten, nagy a te hatalmad,
 Világteremtő a te szózatod.
 Mondád: "Legyen!" s a puszta semmiségből
 Világosság s mindenség támadott.
 A csillagezrek, a nap fényessége,
 Ég, föld követte szent parancsszavad:
 Remeg szívem, s megdöbbenvén csudálja
 Mindenható, dicső hatalmadat.
/2
#F8FA82D7
 Nagy vagy, te Isten, nagy a bölcsességed,
 Mindent mi szépen s bölcsen alkotál!
 A földi embert tetted gyermekeddé,
 Lelkedből lelket őnéki adál.
 Kicsiny fűszálban, óriási tölgyben
 Dicső kezed nyomát szemlélteted;
 Szívem kitárva, hódolással áldom
 Csudálatos nagy bölcsességedet.
/3
#E164C94F
 Nagy vagy, Uram, és mily nagy a szerelmed,
 Fiadban mit velem éreztetél!
 A pislogó kis mécsest el nem oltád,
 S a megtört nádnak megkegyelmezél.
 Mikor hevertem bűnben, megkötözve,
 Felém kinyújtád irgalmas kezed;
 Örömkönnyekkel, térdre hullva áldom
 Te idvezítő nagy szerelmedet!

>456. Szent vagy, szent vagy, szent vagy, Mindenható Isten
/1
#F09EDBBD
 Szent vagy, szent vagy, szent vagy,
 Mindenható Isten,
 Énekünk jó reggel száll hozzád szívesen.
 Szent vagy, szent vagy, szent vagy,
 Végtelen kegyelem,
 Három személyben áldott egy Isten!
/2
#EF02B1B0
 Szent vagy, szent vagy, szent vagy,
 Kit a szentek áldnak,
 Koronájukat letészik teelőtted;
 Angyali seregek térdelve imádnak,
 Ki voltál, vagy s nem érsz soha véget.
/3
#C72DDA6C
 Szent vagy, szent vagy, szent vagy,
 Földi köd bár elfed,
 És bűnös szem nem látja dicsőségedet,
 Csak te vagy szent, Isten, és senki kívüled:
 Teljes hatalmú szentség, szeretet!
/4
#237CFEDD
 Szent vagy, szent vagy, szent vagy,
 Nagy és erős Isten;
 Minden műved dicsér az ég-, föld- s tengeren!
 Szent vagy, szent vagy, szent vagy, áldott, véghetetlen:
 Három személyben egy áldott Isten!

>Bizodalom Istenben

>457. Tudom,  hogy Jézus él
/1
#9B56904B
 Tudom,  hogy Jézus él,
 Tudom,  hogy ő segél,
 Nem bánthat gond és félelem,
 Ő van mindig velem.
/2
#C0FD2C34
 Jó pásztor ő nagyon,
 Szeme juhán vagyon,
 Legeltet szép zöld pázsiton,
 Ég harmatát iszom.
/3
#B0B4715F
 Mikor leszáll az est,
 Sötétje bánt, ijeszt,
 Az éjszakában is tudom,
 Hogy megvéd Jézusom.
/4
#0A161E04
 Ha lábam tévedez,
 Bánt a világ, sebez,
 Jézus szívén a menhelyem,
 Ő gyógyírt ad nekem.
/5
#167FD38D
 Ha rája néz szemem,
 Úgy megvan mindenem,
 Boldog vagyok, hogy tudhatom,
 Jézus szeret nagyon.

>458. Az Úr csodásan működik
/1
#7D68A47B
 Az Úr csodásan működik,
 De útja rejtve van.
 Tenger takarja lábnyomát,
 Szelek szárnyán suhan.
 Mint titkos bánya mélyiben
 Formálja terveit,
 De biztos kézzel hozza föl,
 Mi most még rejtve itt.
/2
#D9F4AFBC
 Ne félj tehát, kicsiny csapat,
 Ha rád felleg borul.
 Kegyelmet rejt s belőle majd
 Áldás esője hull.
 Bízzál az Úrban, rólad ő
 Meg nem feledkezik,
 Sorsod sötétlő árnya közt
 Szent arca rejtezik.
/3
#805B52CC
 Bölcs terveit megérleli,
 Rügyet fakaszt az ág,
 S bár mit sem ígér bimbaja,
 Pompás lesz a virág.
 Ki kétkedőn boncolja őt,
 Annak választ nem ád,
 De a hívő előtt az
 Úr Megfejti önmagát.

>459. Áldó hatalmak oltalmába rejtve
/1
#041F6FC0
 Áldó hatalmak oltalmába rejtve
 Csak várjuk békén mindazt, ami jő.
 Mert Isten ő-riz híven reggel, este,
 Ő hű lesz, bármit hozzon a jövő.
/2
#DAEC6813
 Ha gyötri, bántja szívünket a régi,
 És múlt napoknak terhe ránk szakad,
 Megrettent lelkünk vigaszodat kéri,
 Mit nékünk szerzett, Atyánk, szent Fiad.
/3
#C351965B
 S ha szenvedések kelyhét adod inni,
 Mely színig töltött, keserű s nehéz,
 Te segíts békén, hálával elvenni,
 Hisz áldva nyújtja hű atyai kéz!
/4
#4B5D4E4F
 És ha az úton örömöt adsz nékünk,
 Ha szép napod ragyogva ránk nevet,
 Biztasson, intsen sok nehéz emlékünk,
 Hogy életünket szenteljük neked!
/5
#C3683B0E
 A csend köröttünk mélyen szerteárad.
 Hadd halljuk azt a tiszta éneket,
 Amely betölti rejtett, szép világod,
 Hol téged dicsér minden gyermeked!

>460. Ne csüggedj el, kicsiny sereg
/1
#D2240083
 Ne csüggedj el, kicsiny sereg,
 Bár egész föld tör ellened
 Gyűlölség fegyverével.
 Ne félj, dühe el nem tipor,
 A Lélek ellen földi por
 Harcolhat, de mit ér el?
 Harcolhat, de mit ér el?
/2
#9A4EE6FD
 Bízzál, az Úr veled leszen:
 Se vad bosszú, se félelem
 Nem árt igaz ügyednek.
 Ő áll melléd, ha jön a vész,
 Megtart, a harc bármily nehéz;
 Pokol sem győz feletted,
 Pokol sem győz feletted.
/3
#5F03EBD0
 Az Istené a hatalom
 A mennyen, földön, poklokon:
 Karja legyőzhetetlen!
 Ki ővele tusára száll,
 Mit ér a kard, sok kopjaszál
 Az ő hatalma ellen!
 Az ő hatalma ellen!
/4
#33132ABF
 Ne hagyd el árva nyájadat,
 Jézus, harcolj velünk magad,
 Reményünk napja, kelj fel!
 Benned bíznak te híveid,
 Karod bizonnyal megsegít
 S áldunk vidám lélekkel!
 S áldunk vidám lélekkel!

>461. Mindig velem, Uram, mindig velem
/1
#1266F497
 Mindig velem, Uram, mindig velem,
 Még ha nem láthat is gyarló szemem!
 Azért ez énekem: Velem van Istenem,
 Velem van Istenem, mindig velem.
/2
#A7FD628E
 Nem mondtad-é, Uram, híveidnek,
 Hogy Szentlelked fog majd lakni bennek?
 Templomoddá engem tégy hát, és légy velem,
 Tégy hát, és légy velem, mindig velem!
/3
#EA33F4D6
 Veled megfeszített új életben
 Élek most nem, nem én, te élsz bennem.
 Hit által élek már, mert Jézus velem jár,
 Mert Jézus velem jár, énvelem jár.
/4
#B6989533
 Kétségem-, félelmem- s bánatimban
 Tudom, bizton vagyok karjaidban.
 Nappal s bús éjjelen az Úr mindig velem,
 Az Úr mindig velem, mindig velem.
/5
#EA3C73C7
 Járjak bár a halál sötét völgyén,
 Velem vagy ott is, hát mit féljek én?
 Vessződ megvéd engem.
 Velem vagy, Istenem,
 Velem vagy, Istenem, mindig velem.
/6
#D10C7145
 Majd ha otthon egyszer megláthatlak,
 Bűn, bú, halál hol már nem árthatnak,
 Ezt zengem szüntelen: velem az Úr, velem,
 Velem az Úr, velem, mindig velem!

>Megtérés

>462. Itt van szívem, neked adom, Uram
/1
#7E355B1D
 Itt van szívem, neked adom,
 Uram, Neked, ki alkotád!
 "Rossz a világ, énnékem add, fiam!"
 Szád ily parancsot ád.
 Itt van szerelmem áldozatja,
 Hűségem hű kezedbe adja,
 Itt van szívem, itt van szívem!
/2
#4EE719A0
 Itt van szívem: fogadd kegyelmesen,
 Bár sok hibája van;
 Amint vagyon, kezedbe úgy teszem,
 Ne vesd meg, jó Uram!
 Sok bűnös vággyal van betelve,
 Száz bűnnek nyomja régi terhe,
 Bűnös szívem, bűnös szívem.
/3
#D6FE2B23
 Itt van szívem! Üdve Krisztusban van,
 Keresztednél pihen,
 S így szól: Uram, te vagy minden javam,
 Halálod életem!
 A Megváltó sebébe mélyed
 És ott lel vigaszt, békességet
 Hívő szívem, hívő szívem!

>463. Aki értem megnyíltál
/1
#037941BA
 Aki értem megnyíltál,
 Rejts el, ó, örök kőszál!
 Az a víz s a drága vér,
 Melyet ontál a bűnér',
 Gyógyír légyen lelkemnek,
 Bűntől s vádtól mentsen meg!
/2
#900FC6AE
 Törvényednek eleget
 Bűnös ember nem tehet;
 Buzgóságom égne bár,
 S folyna könnyem, mint az ár:
 Elégtételt az nem ad,
 Csak te válthatsz meg magad.
/3
#6CF94225
 Jövök, semmit nem hozva,
 Keresztedbe fogózva,
 Meztelen, hogy felruházz,
 Árván, bízva, hogy megszánsz;
 Nem hagy a bűn pihenést:
 Mosd le, ó, mert megemészt!
/4
#B25191D1
 Ha bevégzem életem, és lezárul már szemem,
 Ismeretlen bár az út,
 Hozzád lelkem mennybe jut:
 Aki értem megnyíltál,
 Rejts el, ó, örök kőszál!

>464. Hű Jézusom kezébe Teszem kezem belé
/1
#D6C5FFFA
 Hű Jézusom kezébe
 Teszem kezem belé,
 Ő életem vezére,
 Vezet hazám felé,
 A keskeny úton járok,
 De ő közel van ám,
 Magasból vet világot
 Az ő keresztje rám.
/2
#683FDBEE
 Ha nem tudom azonnal,
 Hogy Jézus mit kíván,
 Várok, szemébe nézek,
 Szívem nyugodt, vidám.
 Mily boldog, édes érzet:
 Úr ő éltem fölött!
 Szívem nem ismer kényszert:
 Önként szolgálom őt.
/3
#5D5AD138
 Ha szenvedés, baj érne,
 S az utam is sötét,
 Az Úr keresztje fénye
 Annál dicsőbben ég.
 Tudom, hogy még dicsőbb fény
 Ragyog majd énreám,
 Ha majd a mennybe érvén
 Belépek ajtaján.

>465. Jézus hív, bár zúg, morajlik
/1
#67576223
 Jézus hív, bár zúg, morajlik
 Életünk vad tengere;
 Halk hívása tisztán hallik:
 "Jer, kövess, ó, jöjj ide!"
/2
#B1D8A28F
 Vedd a példát Andrástól, ki
 Hallva hívó szózatot,
 hálóját se vonszolá ki:
 Érte mindent elhagyott.
/3
#CC349D91
 Jézus hív, hogy Őt imádjad,
 Megragad, hogy el ne ess,
 Mert kísért öntelt világod:
 „Jöjj, engem jobban szeress!”
/4
#2B93643B
 Ha nehéz az élet terhe,
 Roskadozva hordom azt:
 Bús orcám Hozzá emelve,
 Jézusban lelek vigaszt.
/5
#B06F9959
 Uram, hozzám légy kegyelmes,
 Tedd Tieddé szívemet,
 Hadd lehessek engedelmes,
 Néked élő gyermeked!

>466. Jézus ölébe bizton Hajtom fejem le én
/1
#ED084F1D
 Jézus ölébe bizton
 Hajtom fejem le én;
 Abban találom üdvöm,
 Ha nyugszom kebelén.
 Áldó szelíd szavával
 Megváltóm hirdeti:
 Jer, hajtsd fejed szí vemre,
 Harcod pihenve ki.
/2
#F6067B80
 Bizton a Jézus keblén
 Elszáll a bú, a baj,
 Bizton a Sátán ellen
 Sebe véd, betakar.
 Enyhül a bánat terhe,
 A kétely megszünik,
 Szívedbe száll enyhület
 S apadnak könnyüid.
 Jézus ölébe bizton
 Hajtom fejem le én;
 Abban találom üdvöm,
 Ha nyugszom kebelén.
/3
#10A0FD1C
 Jézus, ó, szívem vára,
 Te értem vérezél,
 Bizton vagyok e sziklán,
 Te égi jó vezér!
 Itt csöndesen bevárom,
 Az éj míg elhalad,
 Míg túl az aranypartnál
 Fénylőn felkél a nap.
 Jézus ölébe bizton
 Hajtom fejem le én;
 Abban találom üdvöm,
 Ha nyugszom kebelén.

>467. Jöjj, királyom, Jézusom!
/1
#1E5411F7
 Jöjj, királyom, Jézusom!
 Szívem, íme, megnyitom.
 A gonosztól óvj te meg,
 Meg ne rontson engemet.
/2
#A53A55D9
 Véreddel, mely el-kifolyt,
 Mosd le rólam, ami folt;
 Élet útját megmutasd,
 Én meg nem találom azt.
/3
#8397C5BD
 Gyógyítsd meg sok nyavalyám,
 Enyhíts szívem bánatán;
 Kétség, gond ha gyötrenek,
 Biztasd nádszál hitemet.
/4
#FA1C03FA
 Van hatalmad rá, tudom,
 Míveld, édes Jézusom:
 Hit, remény és szeretet
 Töltse be a szívemet.
/5
#A9FDE343
 A keresztet te adod,
 Adj hozzá alázatot:
 Hordjam olyan csendesen,
 Mint egykor te, Mesterem.
/6
#0176326D
 Majd ha véget ér a harc
 S megpihentetni akarsz:
 megragadom jobbodat,
 S mennyországod béfogad.

>468. Kövesd a Jézust, kövesd még ma
/1
#319017D1
 Kövesd a Jézust, kövesd még ma,
 Halld, miként hangzik hívó szava!
 Közel van ő, hogy megáldana;
 Édesen mondja: "Jöjj!"
 Ó, mily szép s dicső lesz egykoron,
 Ha majd bűntől tisztán, szabadon
 Fogad ölébe az égi hon
 Jézus szavára: "Jöjj!"
/2
#043F2284
 Gyermekem, kövesd szelíd szavát,
 Szívednek nyújtván áldozatát,
 Gyógyítja sebét, minden baját:
 Azért hát siess, jöjj!
 Ó, mily szép s dicső lesz egykoron,
 Ha majd bűntől tisztán, szabadon
 Fogad ölébe az égi hon
 Jézus szavára: „Jöjj!”
/3
#94C1F288
 „Ne félj, csak higgy!” Az Úr elfogad,
 Keblére zárva nyugalmat ad;
 Irgalma végig veled marad,
 Jöjj csak, ó, bűnös, jöjj!
 Ó, mily szép s dicső lesz egykoron,
 Ha majd bűntől tisztán, szabadon
 Fogad ölébe az égi hon
 Jézus szavára: „Jöjj!”

>469. Lábaidhoz hullok tört reménynyel
/1
#56A662E3
 Lábaidhoz hullok tört reménynyel,
 Mert por és hamu vagyok csupán;
 Ámde lelkem, mit fényed vezérel,
 Bízva fordul kereszted után.
/2
#9ECC383E
 Bús szívem jobban epedve vár rád,
 Mint esőért szomjú föld eped.
 Ó, jövel, zúdítsd rá élet árját,
 Jöjj és egyesíts engem veled!
/3
#1825AAB3
 Istenem, méltó én nem vagyok rád,
 Ám magam tenéked átadom.
 Itt a szívem, hozza hódolatját
 Nagy kegyelmedért, Szabadítóm!
/4
#E24CF22B
 Ó, örök szeretet mély csodája!
 Mindenek feje, Isten Fia;
 Egyesül velem az ég Királya,
 Bár vagyok a semmiség maga.
/5
#A54A2F5E
 Megváltóm irgalma s szenvedési
 Bűneim örökre elvevék;
 S Lelke lelkembe szeretve vési
 Örök béke mennyei jelét!

>470. Már keresztem vállra vettem
/1
#1109F759
 Már keresztem vállra vettem
 S érted mindent elhagyok.
 Mindenem vagy, árva lettem,
 Honjavesztett szív vagyok.
 Vágyat, célt a múltnak adtam,
 Nincs már bennem vak remény,
 Mégis gazdag úr maradtam:
 Isten és a menny enyém.
/2
#DABC03C6
 Ember bánthat és zavarhat:
 Szíved áldott menedék;
 Sorsom próbál és sanyargat:
 Édes csenddel vár az ég.
 Nincsen búm, mely könnyet adjon,
 Míg szerelmed van velem,
 Nincs öröm, mely elragadjon,
 Hogyha nem benned lelem.
/3
#EF3829A2
 Lelkem, teljes üdv a részed,
 Hagyd a bút s a gondot el;
 Légy vidám, ha meg-megérzed:
 Tenni kell még s tűrni kell.
 Gondold el: ki Lelke éltet,
 Milyen Atya mosolya;
 Megváltód meghalt teérted:
 Mit bánkódnál, menny-fia?
/4
#19EE03A4
 Kegyelemből dicsőségbe
 Szállj, hited majd szárnyat ad,
 S az örök menny fénykörébe
 Bévezet majd szent Urad.
 Véget ér itt küldetésed,
 Elszáll vándoréleted,
 Üdvösséggé lesz reményed,
 Égi látássá hited.

>471. Rád tekint már hitem, Megváltóm, Istenem
/1
#F41BB6C4
 Rád tekint már hitem,
 Megváltóm, Istenem,
 A Golgotán: Halld könyörgésemet,
 És vedd el vétkemet;
 Mostantól hadd legyek
 Tied csupán.
/2
#5BB82F9F
 Szívemet töltse be Kegyelmed ereje
 Buzgósággal! Meghaltál érettem;
 Add: szívem s életem
 Teérted éghessen
 Forró lánggal!
/3
#75D0BF64
 Ha elfog utamon Félelem s fájdalom:
 Fogd kezemet! Derítsd fel éjemet,
 Szárítsd fel könnyemet:
 Tévelygésben ne hagyd
 Én lelkemet!
/4
#92F07899
 Éltem ha fogyva fogy, És a halál ahogy
 Jön már felém: Megváltóm, ments te meg
 Kétségtől engemet, Nálad hogy üdvömet
 Meglássam én.

>472. Most, most, még ifjúkorodban
/1
#E090ED41
 Most, most, még ifjúkorodban
 Add át Jézusnak szíved,
 Míg öröm, remény, vidám kedv,
 Ifjúság, erő tied.
 Fenn ragyogva élted napja,
 Jöjj az Úrhoz szaporán,
 A Megváltó szent ügyéért
 Munkálkodj későn, korán!
/2
#0FC714DE
 Hogy Jézus szíved lakója,
 Ne csak szád, de életed
 Prédikálja, s hogy tanácsát
 Te is híven követed.
 Fenn ragyogva élted napja,
 Jöjj az Úrhoz szaporán,
 A Megváltó szent ügyéért
 Munkálkodj későn, korán!
/3
#B8519ECE
 Búban úgy, mint jó napokban,
 Add időd, erőd neki,
 S hogyha küld, siess örömmel
 Országát terjeszteni!
 Fenn ragyogva élted napja,
 Jöjj az Úrhoz szaporán,
 A Megváltó szent ügyéért
 Munkálkodj későn, korán!

>473. Ó, Jézus, árva csendben az ajtón kívül állsz
/1
#B8C4765A
 Ó, Jézus, árva csendben az ajtón kívül állsz,
 Bejönnél már, de némán kulcsfordulásra vársz.
 Mi mondjuk, hogy miénk vagy, te vagy a név, a jel:
 Ó, szégyen, hogy te légy az, akinek várni kell.
/2
#70B03883
 Ó, Jézus, most kopogtatsz, sebhelyes még a kéz;
 Könnymarta kedves arcod oly búsan intve néz.
 Ó, áldott, drága jóság, mely ennyit tűrve vár!
 Ó, bűnök szörnyű bűne, mely téged így kizár!
/3
#279AD5FA
 Ó, Jézus, szólsz, s a szívhez a szó szelíden ér:
 „Így bánsz velem? - teérted hullt testemből a vér!”
 Bús szégyennel behívunk, az ajtónk nyitva már.
 Jöjj, Jézus, jöjj, ne hagyj el, a szívünk várva vár.

>474. Siessetek, hamar lejár
/1
#D13934DD
 Siessetek, hamar lejár,
 Kegyelme már régóta vár;
 Ma még lehet, ma még szabad:
 Borulj le a kereszt alatt!
/2
#873A9214
 Ha elkésel, mi lesz veled?
 Hogy mented meg a lelkedet?
 Lezárul a kegyelmi út,
 Lelked örök halálba jut.
/3
#94266FC2
 Elszáll a perc, az életed:
 Ma még, ha jössz, elérheted.
 Ne késs tovább, ne várj tovább:
 Ma kérd Atyád bocsánatát!

>475. Új szívet adj, Uram, énnekem
/1
#EDE4C1B9
 Új szívet adj, Uram, énnekem,
 Új szívet adj, én Istenem.
 Amely csupáncsak teérted ég,
 S véled jár szüntelen,
 Csak véled szüntelen.
/2
#CB8F63AE
 Nyájas, vidám, szelíd, jó szívet,
 Mely, Jézusom, te lakhelyed,
 Hol egyedül a te hangod szól,
 Mely véled van tele,
 Csak véled van tele.
/3
#C67F6D92
 Jézus, a te gyógyító kezed
 Megfogta már a szívemet,
 S én is tudom: bűntelen leszek
 Majd nálad odafenn,
 A mennyben odafenn.

>Keresztyén élet

>476. Fel, barátim, drága Jézus zászlaja alatt
/1
#5F36D10D
 Fel, barátim, drága Jézus zászlaja alatt,
 Rajta, bátran! megsegít és győzedelmet ad.
 Bízzatok, mert Jézus eljön, ő a fővezér,
 Zengje ajkunk: hozzád esdünk győzedelemér'!
/2
#F7FC1B10
 Lám, a Sátán serge talpon, szembetörni kész,
 A legbátrabb harcosoknak bátorsága vész.
 Bízzatok, mert Jézus eljön, ő a fővezér,
 Zengje ajkunk: hozzád esdünk győzedelemér'!
/3
#A9A5597A
 Szóljon a kürt, fenn lobogjon győzedelmi jel,
 Így előre Jézusunkkal: néki győzni kell!
 Bízzatok, mert Jézus eljön, ő a fővezér,
 Zengje ajkunk: hozzád esdünk győzedelemér'!
/4
#55FDAB3D
 Harci zajban, küzdelemben oldalunkon áll,
 Benne higgyünk, ő segít meg szívünk harcinál.
 Bízzatok, mert Jézus eljön, ő a fővezér,
 Zengje ajkunk: hozzád esdünk győzedelemér'!

>477. Isten szívén megpihenve
/1
#157C3004
 Isten szívén megpihenve
 Forrjon szívünk egybe hát,
 Hitünk karja úgy ölelje
 Édes Megváltónkat át!
 Ő fejünk, mi néki tagja,
 Ő a fény, mi színei;
 Mi cselédek, ő a gazda,
 Ő miénk, övéi mi.
/2
#BD4F1876
 Szeretetben összeforrva,
 Egy közös test tagjai,
 Tudjuk egymásért harcolva,
 Ha kell, vérünk ontani.
 Úgy szerette földi nyáját
 S halt meg értünk jó Urunk;
 Fájna néki, látva minket,
 Hogy szeretni nem tudunk.
/3
#BAFC478B
 Nevelj minket egyességre,
 Mint Atyáddal egy te vagy,
 Míg eggyé lesz benned végre
 Minden szív az ég alatt;
 Míg Szentlelked tiszta fénye
 Lesz csak fényünk és napunk,
 S a világ meglátja végre,
 Hogy tanítványid vagyunk.

>478. Feljebb emeljetek, feljebb
/1
#8B30DC61
 Feljebb emeljetek, feljebb
 A bűn gyászos éjiből!
 Ott a helyem Jézus mellett,
 Éltét értem adta föl.
 Szálljon, szálljon angyal szárnya,
 Elragadva vi-gyen el,
 Vigyen fel a Golgotára,
 Hol megváltott vérével!
/2
#B8DE1F7A
 Feljebb emeljetek, feljebb
 Fájdalmaim árjából!
 Egyre éget, sajog a seb
 Szenvedésem lángjától.
 Szálljon, szálljon angyal szárnya!
 Fönn, a Tábor ormain,
 Hol fénylett áldott orcája,
 Könnyen gyógyul ott a kín.
/3
#5302506C
 Feljebb emeljetek, feljebb,
 Ez csak gyötrelem hona,
 Mind közelebb és közelebb,
 Hol a mennynek sátora!
 Szálljon, szálljon angyal szárnya,
 Lássam már az ég Urát!
 Föl, Sionnak szent halmára,
 Nyissátok szent kapuját!

>479. Fogjad kezem, oly gyenge vagyok, érzem
/1
#E15103FC
 Fogjad kezem, oly gyenge vagyok, érzem,
 Hogy nélküled én járni nem tudok.
 Fogjad kezem, és akkor, jó Megváltóm,
 A félelemre többé nincsen ok.
/2
#C33CCD24
 Fogjad kezem, és vonj közelb magadhoz,
 Úgy szíveden én megpihenhetek!
 Fogjad kezem, különben tévedezvén,
 Az igaz útról is letérhetek.
/3
#5FB27AF3
 Fogjad kezem! Az út sötét előttem,
 Ha fényeddel meg nem világítod;
 De így, Uram, e mennyből jövő fénynél
 Reád tekintve bizton indulok.

>480. Testvérek, menjünk bátran
/1
#2EC7AFBC
 Testvérek, menjünk bátran,
 hamar leszáll az éj,
 E földi pusztaságban
 Megállni nagy veszély.
 Hát merítsünk erőt
 A menny felé sietni,
 Nem állva megpihenni
 A boldog cél előtt.
/2
#D1352444
 A keskeny útra térünk,
 Ne rettentsen meg az;
 Ki elhívott, vezérünk,
 Tudjuk, hogy hű s igaz.
 Mint egykor Ő tevé,
 Most véle s benne bízva,
 Arcát ki-ki fordítsa
 A szent város felé.
/3
#0D77035D
 Óemberünk ha szenved,
 Az jó nekünk, tudom;
 Ki vérnek, testnek enged,
 Az nem jár jó úton.
 A láthatót ne bánd,
 Csak rázd le, mi kötözne:
 Hadd törjön éned össze,
 Menvén halálon át.
/4
#510BF6CE
 Zarándok módra járva,
 Legyen kezünk üres;
 Csak terhet vesz magára,
 Ki pénzt, vagyont keres.
 Hadd gyűjtsön a világ,
 Mi tőle el se kérjük,
 Kevéssel is beérjük,
 Bennünket gond se bánt.
/5
#4E649879
 Az út el van hagyatva,
 Borítja sok tövis;
 nehéz emelni rajta
 Még a keresztet is.
 De egy út van csupán,
 Így hát előre bátran,
 Keresztül minden gáton,
 Hű Mesterünk után.
/6
#ACA61C0B
 Úgy járunk itt, lenézve,
 Mint ismeretlenek;
 Sokan nem vesznek észre,
 Hangunk se hallva meg.
 De aki ránk figyel,
 Víg énekünket hallja.
 Szent reménység sugallja,
 Mit ajkunk énekel.
/7
#6E4DE8BD
 Ha botlanak a gyöngék,
 Segítsen az erős;
 Hordjuk, emeljük önként,
 Kin gyöngesége győz.
 Tartsunk jól össze hát,
 Tudjunk utolsók lenni,
 A bajt vállunkra venni
 E földi élten át.
/8
#EF3C6697
 Menjünk vígan sietve,
 Hisz utunk egyre fogy;
 Nap megy napot követve,
 S a test majd sírba rogy.
 Csak még egy kis tűrés!
 Ha Őt híven követjük:
 A láncot mind levetjük
 S vár ránk az égi rész.
/9
#887129FD
 Elmúlik nemsokára
 a földi vándorút,
 És az örök hazába,
 ki hű volt, mind bejut.
 Ott vár angyalsereg,
 Ott várnak mind a szentek,
 S az Atyánál pihentek,
 Megfáradt gyermekek.

>481. Tégy, Uram, engem áldássá! Lelkedet úgy várom
/1
#BD35BFF0
 Tégy, Uram, engem áldássá!
 Lelkedet úgy várom.
 Tedd te a szívem hálássá,
 hogy neked szolgáljon!
 Bárhova küldesz, ajkamról
 zengjen az új ének!
 Tégy, Uram, engem áldássá!
 Oly sok a bús élet.
/2
#7E5710BE
 Tégy, Uram, engem testvérré,
 Lelkedet úgy kérem,
 Tedd Te a szívem, tedd késszé,
 bízni a testvérben.
 Sejteni engedd: mit érez,
 boldog-e, szenved-é?
 Vigy közelebb a testvérhez,
 népedet tedd eggyé!

>482. Velem vándorol utamon Jézus
/1
#4E7E3309
 Velem vándorol utamon Jézus,
 Gond és félelem el nem ér.
 Elvisz, elsegít engem a célhoz
 Ő, a győzelmes, hű vezér,
 Ő, a győzelmes, hű vezér.
/2
#C885325E
 Velem vándorol utamon Jézus,
 Ott az oltalom hű szívén.
 Ha a szép napot fellegek rejtik,
 Ő az éltető, tiszta fény,
 Ő az éltető, tiszta fény.
/3
#2099C88A
 Velem vándorol utamon Jézus.
 Bár az út néha oly sötét,
 Soha nincs okom félni a bajtól,
 Amíg irgalmas karja véd,
 Amíg irgalmas karja véd.
/4
#5F44A697
 Velem vándorol utamon Jézus,
 Ez a vigaszom, baj ha jő;
 Bármi súlyosak rajtam a terhek,
 Segít hordani, ott van ő,
 Segít hordani, ott van ő.
/5
#C938E877
 Velem vándorol utamon Jézus.
 Túl a sír sötét éjjelén,
 Fenn a mennyei, angyali karban
 Nevét végtelen áldom én,
 Nevét végtelen áldom én.

>Örök élet

>483. Fönn a csillagok felett, Halleluja, ámen!
/1
#C796DA81
 Fönn a csillagok felett,
 Halleluja, ámen!
 Boldog lelkek zengenek,
 Halleluja, ámen!
 Aki itt elfáradott, Aki hordoz bánatot,
 Azt a béke várja ott. Halleluja, ámen!
/2
#D67F5426
 Ott az üdvözült sereg,
 Halleluja, ámen!
 Istennek zeng éneket,
 Halleluja, ámen!
 Onnét száll a földre le
 Isten Lelke, hogy vele
 A szívünk legyen tele.
 Halleluja, ámen!
/3
#05DAD383
 Én is zengek éneket,
 Halleluja, ámen!
 Kegyelméért Istennek,
 Halleluja, ámen!
 Dicsértessék Jézusom,
 Ezt kiáltom, ezt mondom
 Itt s előtted egykoron.
 Halleluja, ámen!

>484. Fenn a mennyben az Úr minden győztesnek ád
/1
#C553F5D5
 Fenn a mennyben az Úr minden győztesnek ád,
 Aki Jézussal járt, s benne hitt,
 Aranyból koronát, fehér égi ruhát,
 Hárfa húrjait pengethetik.
 R.: Igen, ott minden kész,
 Igen ott minden kész.
 Minden győztesnek jár örök rész.
 Boldog véghetetlen öröm vár odafenn,
 Jézusért én is elnyerhetem.
/2
#DC51F68D
 Vígan lépkedünk színarany utcákon át,
 Halljuk angyalok dicséretét,
 Szemünk láthatja Jézus szent ábrázatát,
 Aki vért ontott bűneinkért.
 Igen, ott minden kész,
 Igen, ott minden kész.
 Minden győztesnek jár örök rész.
 Boldog, véghetetlen öröm vár odafenn,
 Jézusért én is elnyerhetem.
/3
#17BECB21
 Élet forrása kínálja élő vizét,
 Tiszta, hűs vizét bőségesen,
 Fényt az Úr maga áraszt és hint szerteszét,
 Halál nem lesz ott, éjszaka sem.
 Igen, ott minden kész,
 Igen, ott minden kész.
 Minden győztesnek jár örök rész.
 Boldog, véghetetlen öröm vár odafenn,
 Jézusért én is elnyerhetem.
/4
#131436FD
 Minden gyötrelmet, bánatot elfeledünk,
 Ha ő bennünket keblére von,
 Jézus szívére hajthatjuk fáradt fejünk,
 Nála béke vár és nyugalom.
 Igen, ott minden kész,
 Igen, ott minden kész.
 Minden győztesnek jár örök rész.
 Boldog, véghetetlen öröm vár odafenn,
 Jézusért én is elnyerhetem.

>485. Íme, lészen a kései korban
/1
#5C9A70E1
 Íme, lészen a kései korban,
 Hogy az Úr helye áll hegyek orma felett.
 Sion szent helye, temploma ott van,
 Csoda fénye betölti a földet, eget;
 Örök otthona lesz ez a népnek:
 Aki vágyik az Úrra, betérhet.
/2
#C1DA56E8
 Nosza, jöjjetek el, jövevények,
 Akik messzire jártok: az Úr hazavár!
 Neki zengjen a szép örömének:
 Sosem árt a halál, hisz az Úr a Király!
 Keze el nem ereszt soha többé.
 Legyen áldott az Isten örökké!
/3
#07F144D8
 Ha az Úr közel, hallgat a fegyver,
 Ha az Úr szava szól, belereng a világ,
 Ha az Úr ítél, semmi az ember,
 De ő megkönyörült, ideadta Fiát.
 Vele győztek a mennyei fények,
 Vele itt az ígért örök élet!

>Esti énekek

>486. Fölkelt immár a szép hold, A csillagezres égbolt
/1
#5DFD6F24
 Fölkelt immár a szép hold,
 A csillagezres égbolt
 Oly tisztán tündököl;
 Az erdő áll sötéten,
 S fehér köd künn a réten
 Csodásan száll a légbe föl.
/2
#EBC04F98
 Mély csend borult a földre,
 S mit alkony leple föd be,
 Meghitten integet,
 Mint nyájas, tiszta hajlék,
 Hol nappalodnak terhét
 Kialhatod, feledheted.
/3
#E9006160
 Nem látod-é a holdat?
 Fél arca int, mosolyg csak,
 Pedig kerek, egész.
 Van sok, mit itt az ember
 nem lát jól földi szemmel,
 És oktalan nevetni kész.
/4
#FE979FC0
 Mint gyenge, földi férgek,
 Kik bűn útjára tértek,
 Mi nem sokat tudunk,
 Sok csalfa képet űzünk,
 Sok furfangot kifőzünk,
 S a céltól csak messzebb jutunk.
/5
#6E0F155C
 Ó, add, üdvöd keressük,
 Ne a mulandót lessük,
 Ne kössön fénye meg;
 Hagyj egyszerűvé válni
 S előtted élni, járni,
 Mint vidám, boldog gyermekek.
/6
#CCB895AC
 Ha jő a végső óra,
 Fordítsd a kínt is jóra.
 És adj szelíd halált;
 Ha innen elvezetsz te,
 Ó, hadd jutunk egedbe,
 Úr Istenünk és jó Atyánk!

>487. Ó, terjeszd ki, Jézusom
/1
#D8C17438
 Ó, terjeszd ki, Jézusom,
 Oltalmazó szárnyad,
 És csitíts el szívemben
 Bút, örömöt, vágyat!
 Légy mindenem, légy fényem:
 Sötéten jő az éj,
 Nagy irgalmadból élnem
 Szüntelen te segélj!
/2
#5A70FFA5
 Ó, mosson meg az értem
 Bőven hullt drága vér!
 Új lélekért könyörgök,
 Újult akaratér'.
 Kicsik, nagyok, mind kérünk:
 Őrködj vigyázva ránk!
 Békességedbe térünk,
 Te áldd meg éjszakánk!

>Záró ének

>488. Áldásoddal megyünk, Megyünk innen el
/1
#C6C65104
 Áldásoddal megyünk,
 Megyünk innen el,
 Boldog már a szívünk,
 Néked énekel.
 Maradj mindig velünk,
 Ha útra kelünk,
 Őrizd életünk mindenkor!
/2
#0865624B
 Menjünk mindig feléd,
 Tekintsünk terád,
 Szentlelked erejét
 Add nékünk tovább!
 Maradj mindig velünk,
 Ha útra kelünk,
 Őrizd életünk mindenkor!
 Maradj mindig velünk,
 Ha útra kelünk,
 Őrizd életünk mindenkor!
/3
#8C481DBB
 Őrizz, hogy a kezünk
 Legyen tiszta kéz,
 Szólj ránk, hogy a szívünk
 Legyen mindig kész!
 Maradj mindig velünk,
 Ha útra kelünk,
 Őrizd életünk mindenkor!
 Maradj mindig velünk,
 Ha útra kelünk,
 Őrizd életünk mindenkor!
/4
#0A083B89
 Várunk, áldott Király,
 Bizony jöjj hamar!
 Várunk, égi Király,
 Bizony jöjj hamar!
 Maradj mindig velünk,
 Ha útra kelünk,
 Őrizd életünk mindenkor!
 Maradj mindig velünk,
 Ha útra kelünk,
 Őrizd életünk mindenkor!

>Kánonok

>489. Szívem csendben az Úrra figyel, ki segít
/1
#B7E8B010
 Szívem csendben az Úrra figyel, ki segít,
 Szívem csendben az Úrra figyel, ki segít,
 Szívem csendben az Úrra figyel.

>490. A szívem úgy vár, ó, jöjj, ne késs már, Uram
/1
#66AA975A
 A szívem úgy vár, ó, jöjj, ne késs már,
 Uram Jézus, térj be, ó, térj be hozzám!

>491. Karácsony ünnepén, Kegyes keresztyének
/1
#E25ECF68
 Karácsony ünnepén,
 Kegyes keresztyének,
 Zengjen, zengedezzen
 Örvendező ének!
 Krisztus született,
 Halleluja!

>492. Glória szálljon a mennybe fel
/1
#5DCB22A4
 Glória szálljon a mennybe fel,
 Jöjjön a földre a béke,
 És az emberi szívbe a jóakarat!
 Á men, á men!

>493. Tegnap harangoztak, Holnap harangoznak
/1
#1CC4858C
 Tegnap harangoztak,
 Holnap harangoznak,
 Holnapután az angyalok
 Gyémánthavat hoznak.
/2
#CD389515
 Szeretném az Istent
 Nagyosan dicsérni,
 De én még kisfiú vagyok,
 Csak most kezdek élni.
/3
#067212EF
 Isten-dicséretre
 Mégiscsak kiállok,
 De boldogok a pásztorok
 S a három királyok!
/4
#5A4A1805
 Én is mennék, mennék,
 Énekelni mennék,
 Nagyok között kis Jézusért
 Minden szépet tennék.
/5
#FF65C8F6
 Új csizmám a sárban
 Százszor bepiszkolnám,
 Csak az Úrnak szerelmemet
 Szépen igazolnám.

>494. Jézushoz jöjjetek, megfáradtak
/1
#82016A7B
 Jézushoz jöjjetek, megfáradtak:
 Igája néktek is nyugalmat ad.
 Terhe könynyű, irgalma nagy.
/2
#62A6A6BA
 Zengjetek éneket az Istennek,
 Dicsérjék szent nevét minden népek,
 Szívvel, szájjal dicsérjétek!

>495. Meszsziről száll dalunk, halld meg, ó, Jézusunk
/1
#3CE7A254
 Meszsziről száll dalunk,
 halld meg, ó, Jézusunk,
 Csak feléd indulunk vágyakozva;
 Életünk töltse meg
 Lelked szent hatalma!
 Így fogadj el bűneinkkel,
 Krisztusunk, vigy magasra!

>496. Istenünk, itt hozzuk néked mindenünk
/1
#7DBECF0A
 Istenünk, itt hozzuk néked mindenünk,
 Itt állunk, várunk rád újra;
 Kérünk, hogy hallgasd meg hálaénekünk,
 Halleluja, halleluja!
 Halleluja, halle -luja, halleluja, halleluja!
/2
#C612FA72
 Te csak az Istennek országát keresed,
 És az ő szent igazságát;
 Minden egyéb megadatik majd neked,
 Békéje, áldása vár rád!
 Halleluja, halleluja! Halleluja, halleluja,
 halleluja, halleluja!

>497. Az Úr a Pásztorom, jó Pásztor ő
/1
#8C8CAF0C
 Az Úr a Pásztorom, jó Pásztor ő.
 Nyájával arra jár, amerre várva vár
 A forrás hűs vize, a zöld mező.
/2
#F052CCEB
 Ha ő a Pásztorom, semmi sem árt.
 Törődik énvelem, vigyáz rám szüntelen,
 Követem lábnyomát mindenen át.

>498. Atyám, én imádlak
/1
#4B5342C8
 Atyám, én imádlak,
 Leborulva áldlak,
 Úgy szeretlek!
/2
#97F5CA17
 Jézus, én imádlak,
 Leborulva áldlak,
 Úgy szeretlek!
/3
#BCC783FC
 Szentlélek, imádlak,
 Leborulva áldlak,
 Úgy szeretlek!

>499. Ne aggodalmaskodjál, nézz Istenedre fel!
/1
#A37A58B4
 Ne aggodalmaskodjál, nézz Istenedre fel!
 Ő felruház és táplál, rád gondot ő visel.
 Dicső Király, ég és a föld Ura!
 Szívünk tiéd, légy annak is Ura!

>500. Ó, jöjj, élő Lélek, hadd támadjon élet!
/1
#8C8E2F81
 Ó, jöjj, élő Lélek, hadd támadjon élet!
 Sok holt elevenen álljon, az Úr seregei ők!
 Már sorakozik is a tábor a Seregek Ura előtt.

>501. Vége van már a szolgaságnak, Sion
/1
#D9E2915B
 Vége van már a szolgaságnak,
 Sion foglyait várja kész örökség.
 Vigadoz a szívünk, szánk vidám
 éneket mond azért, mit a csoda hű
 Ura tett. Könnyhullajtással szórtuk
 a magot, de most boldog aratás
 a jutalom Istenünktől.

>502. Ó, adj nékünk, Jézus, békét
/1
#5220353C
 Ó, adj nékünk,
 Jézus, békét,
 ó, adj nékünk békét!
 Ó, adj nékünk békét,
 ó, adj nékünk békét!
 Ó, adj nékünk békét,
 ó, adj nékünk békét!

>503. Gloria, gloria
/1
#0D17903F
 Gloria, gloria In excelsis Deo,
 Gloria, gloria, Alleluja, alleluja!

>504. Áldjon meg téged, áldjon az Úr
/1
#57E34AA6
 Áldjon meg téged, áldjon az Úr,
 Őrizzen téged, őrizzen ő,
 Orcáját menynyekből fordítsa le rád Atyád,
 Könyörüljön rajtad, könyörüljön rajtad,
 Adjon békességet!
