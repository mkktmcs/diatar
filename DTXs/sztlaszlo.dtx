;HIMNIKUS ÉS ÉLETRAJZI KÖLTEMÉNYEK
;ÉS KÖNYÖRGŐ NÉPÉNEKEK
;SZENT LÁSZLÓ KIRÁLY TISZTELETÉRE
;http://www.ujvarosiportal.hu/uploads/cdocs/1508916282szent_laszlo_enekek.pdf
;(A forrást nem sikerült beazonosítani, valószínűleg a katped.hu segédanyaga)
;
;2019/08/10 Bemásolta Rieth Kati
NSzent László énekek
RSztLászló
CNépénekes könyvek

>1
/1
#389A860C
 Áve, égi király híve, királyoknak gyöngye, éke
 László, mennynek sorsosa!
 Ég királyát hűn követed, országunkat védelmezed, légy hazánknak bajnoka!
/2
#A9454FEF
 \BM\benedéke magyaroknak,
 Örök társa angyaloknak,
 Égi kegynek eszköze.
 Üdvözlégy, ó, kiváltságos,
 Híres, neves, igazságos,
 Jó ítélet védnöke.
/3
#7320C970
 \BV\bigadozzál, magyar nemzet,
 Harsonáljad, énekeljed
 Új királynak új dalod!
 Boldog Várad, szálljon híred,
 Növekedjék dicsőséged,
 Visszhangozzák századok!
/4
#160DB3E3
 \BK\brisztus, földről himnusz szárnyal,
 Hozzád, aki keresztfáddal
 Magadhoz vonsz népeket.
 Szentek útja Te vagy égbe,
 Az örökös dicsőségbe,
 Néked áldás, tisztelet! Ámen

>1b
/1
#175F5E22
 Regis regum civis, ave, regum gemma,
 Ladislae, regni consors gloriae.
 Regem regum es aggressus, sis defensor indefessus
 et athleta patriae.
 Amen.
/2
#B95F6967
 Salve, salus Hungarorum,
 Rex, coaeres angelorum,
 Vas caelestis gratiae,
 Ab aeterno vas electum,
 Vas insigne, vas effectum,
 Vendicans justitiae.
/3
#4D85D902
 Hungarorum gens, congaude,
 Nova novi regis laude
 Pulsans tintinnabula.
 Felix, ave, Varadinum,
 Cujus augens fama signum
 Resonet per saecula.
/4
#BF9C3E7D
 Tibi, Christe, concors hymnum
 Canit orbis, qui per lignum
 Ad te trahis omnia;
 Scala factus ascensorum,
 Et corona confessorum
 Tibi laus et gloria.
 Amen.

>2
/1
#28B46CEB
 Üdvözlégy, kegyes szent László király,
 Magyarországnak édes oltalma!
 Szent királyok közt drágalátos gyöngy,
 Csillagok között fényességes csillag!
/2
#7CC32DF9
 \BS\bzentháromságnak igaz szolgája,
 Jézus Krisztusnak jó tanítványa,
 Te Szentléleknek tiszta edénye,
 Szűz Máriának választott vitéze.
/3
#09A1F0FD
 \BT\bestedben ékes, karodban erős,
 Lelkedben fényes, szívedben bátor,
 Igaz ügyeknek gondját viselted,
 A szegényeknek te voltál oltalma.
/4
#5362AC8A
 \BM\bert kiválasztott a Szűz Mária,
 És felmagasztalt sok ajándékkal,
 Hogy Te őrizzed, s megoltalmazzad,
 Néki ajánlott jó Magyarországot!
/5
#5641D597
 \BD\bicsérnek Téged szent zsolozsmával,
 Papok, diákok és városnépek,
 Dicsérnek téged magyarok, mondván:
 Üdvözlégy mennyben, Szent László királyunk!

>2b
/1
#D93C135A
 \BI\bdvezlégy, kegyelmes Szent László kerály,
 Magyarországnak édes oltalma,
 Szent kerályok közt drágalátos gyöngy,
 Csillagok között fényességes csillag.
/2
#5D293D5F
 \BS\bzentháromságnak te vagy szolgája,
 Jézus Krisztusnak nyomdoka követi,
 Te Szentléleknek tiszta edénye,
 Szíz Máriának választott vitéze.
/3
#5443EB15
 \BM\bagyarországnak vagy kerályi magzatja,
 Szent kerályoknak fényes tüköre,
 Tenéked atyád kegyes Béla kerály,
 Hogy hozzá képest kegyes kerály lennél.
/4
#F7C90997
 \BN\bekönk sziletí Lengyelországban,
 Mennyből adatál nagy csuda képpen,
 Másszor sziletíl szent kereszt víztől,
 Ősödnek nevén László lőn neved.
/5
#06529F8F
 \BM\bikoron méglen gyermekded volnál,
 Kihoza Béla kerály jó Magyarországba,
 Hogy dicsekednél te két országban:
 Magyarországban és mennyországban.
/6
#C0485D44
 \BL\betelepedél Bihar-Váradon,
 Az várusnak lől édes oltalma,
 Templomot rakatál Szíz Máriának,
 Kiben most nyugoszol minden tisztességvel.
/7
#7DC846A2
 \BK\börnyölfekszenek téged császárok,
 Püspökök, kerályok és jobbágy urak,
 Olaj származik szent koporsódból,
 Tetemed foglalták az szép sár aranyból.
/8
#8C963814
 \BT\béged dicsérnek szent zsolozsmával
 Papok, diákok és várusnépek,
 Téged földnek kereksége,
 Mert téged dicsérnek Istennek angyali.
/9
#E007D8E3
 \BT\be dicsekedel kerályi székedben,
 Képed feltötték az magas kőszálra,
 Félnlik, mint nap, salyog, mint arany,
 Nem elégszik senki terád nézni.
/10
#D8D5983E
 \BT\be arcul teljes, szép piros valál,
 Tekéntetedben embereknél kedvesb,
 Beszédedben ékes, karodban erős,
 Lám, mendent te ejtesz, ki tevéled küzdik.
/11
#B2E497DC
 \BT\bestedben tiszta, lelkedben fényes,
 Szívedben bátor, miként vad oroszlán,
 Azért neveztek Bátor Lászlónak,
 Mikoron méglen ifjúdad volnál.
/12
#B666066B
 \BT\bagodban ékes, termetedben díszes,
 Válladtul fogva mendeneknél magasb,
 Csak szépséged császárságra méltó,
 Hogy szent korona téged méltán illet.
/13
#36628CE4
 \BM\bert választa az Szíz Mária,
 Megdicsőíte sok jó ajándékkal,
 Hogy te őriznéd és oltalmaznád
 Néki ajánlád jó Magyarországot.
/14
#2EAC276A
 \BF\bejedben kele az Szent Korona,
 Megbátorejta téged az Szentlélek,
 Kezdéd követni atyádnak életét,
 Rózsákot szaggatál, koronádban fizéd.
/15
#AE2EDF26
 \BT\be tatároknak vagy megtörője,
 Magokat szaggatád az havasokban,
 Te pogányoknak vagy rettenetük,
 Terek mondottak feled félelmének.
/16
#740C14A1
 \BT\be kivagdalád az eretnekeket,
 Elszaggatád, mind kigyomláltad,
 Nem volt idédben gonoszól tevé,
 Mert csak hírneved mindenek rettegék.
/17
#3ADF3CE0
 \BA\bzért igazságnak valál bírója,
 Az szép szízségnek valál koronája,
 Te tisztaságnak tiszta oltalma,
 Irgalmasságnak teljes követője.
/18
#E8A1CC70
 \BD\bicsérjük, magyarok, Szent László kerályt,
 Bizony érdemli mi dicséretönköt,
 Dicsérik őtet angyalok, mondván:
 Üdvözlégy, kegyelmes Szent László kerály!

>3
/1
#7C788805
 Szent László király, Istennek szolgája,
 Krisztus hitének igaz plántálója,
 Légy magyarságnak mennyben szószólója, és orvoslója.
/2
#2E7C01A1
 Kegyes voltából irgalmas Istennek,
 Szent László király adattál e földnek,
 Hogy szent híreddel útja igaz hitnek, lennél e népnek.
/3
#A9EAAD41
 Ki Géjza után királlyá téteték,
 A magyaroktól holott választaték,
 Mert szent életét mindnyájan szerették, és megkedvelték.
/4
#7ABF2583
 Magyar Koronát hogy fejére tevé,
 Sok szent dolgokra mindjárt szívét vévé,
 Római hitnek lőn nagy őrzőjévé, Szent László király.
/5
#5A7A43C5
 Azért őbenne Istennek telt kedve,
 Mindenütt járván Isten előtt félve,
 Szent Koronával azért lőn tisztelve, Szent László király.
/6
#22BC7FAE
 Krisztusnak Anyját, áldott Szűz Máriát
 Híven tisztelé, mint édes Asszonyát,
 Ennek érezte országostul hasznát, Szent László király.
/7
#380A29EE
 A Körös mellett egy várost építe,
 Melyet Váradnak Szent László neveze,
 És szép kolostort Mária nevére  mellé építe.
/8
#A00916A8
 Ott püspökséget és káptalant szerze,
 A tiszta élet virágzott őbenne,
 Temetését is ő oda rendelte, s ott nyugszik teste.
/9
#45900BB4
 Koporsójánál betegek gyógyultak,
 Vakok és némák hozzá folyamodtak,
 És több ilyen nagy, sok csodákat láttak, bizonnyal voltak.
/10
#B3F9747C
 Légyen örökké áldott Atyaisten,
 És egyetemben a Fiú Úristen,
 A magas mennyben Szentlélek Úristen, örökké, ámen.

>4
/1
#624B818A
 Gyönyörköhetünk nyilván, magyar népek,
 Kik eleinknek útján vagyunk épek,
 Kiknek nyomdoki nékünk igen szépek,
 Mint leírott képek.
/2
#398040C3
 Légyen örökké áldott Atyaisten,
 És egyetemben a Fiú Úristen,
 A magas mennyben Szentlélek Úristen, örökké, ámen.
/3
#F6AFDE4C
 \BT\biszta, szűz élet tündöklött őbenne,
 Amely leginkább tetszett az Istennek,
 Kiért szent dolgok néki megjelentek, általa értettek.
/4
#420BCE5F
 \BK\bik közül egyik volt Szent László király,
 Ki szüzességben az Istennek szolgált,
 Melyért csudákkal Isten dicsőítte, holtában s éltében.
/5
#9F1914FD
 Mert Géjza herceg a váci mezőben
 Salamon királyt meggyőzte vitézen,
 A Szent Korona tétetett fejében  a Géjza hercegnek.
/6
#757B993F
 Ki tette oda, azt senki nem tudta,
 De a Szent László testi szemmel látta,
 Hogy angyal volt az, aki ráhelyezte,  hogy lenne királlyá.
/7
#6284F109
 Azután azon a mezőn vadásztok,
 Géjza és László azon tanácskoztak,
 Egy szép templomot a Boldogasszonynak, hogy ott rakassanak.
/8
#7EFDB66B
 Nagy véletlenül egy szarvas elöl jött,
 Egy templom helyet nyomával megjegyzett,
 Vadászok előtt sűrű erdőben ment,  Dunába enyészett.
/9
#9923EC4A
 Ez a nagy szarvas sokaktól láttatott,
 De csak Szent László tudta, hogy angyal volt,
 Kit az Úr Isten a végre bocsátott,  templomról jelt adott.
/10
#BC135B17
 Azon a helyen Géjza Szent Lászlóval,
 Egy nagy templomot alapít klastrommal,
 Mely váci templom nagy erős munkával,  ma is fönn álljon.
/11
#CF401130
 Idő jártában Géjza hogy meghala,
 Magyar Királyság Szent Lászlóra szálla,
 Kiben szerencsés győzedelmes vala,  szent életű vala.
/12
#5B8D1144
 Istennek hozzá való sok jó tétin,
 Mint Géjza király, ő is elmélkedvén,
 Hogy egy templomot mi helyre építsen  királyi költségen.
/13
#BA0A1C35
 Isten angyala néki megjelenti,
 Egy templomhelyet nékie megjegyze,
 Kit fölépítvén meg is ékesítte,  Váradnak neveze.
/14
#087F7871
 Nagy káptalant is amellett rakatott,
 Jövedelemmel megajándékozott,
 Püspökséget is jó móddal elosztott,  és rendesen hagyott.
/15
#F534221A
 Istenes maga viselése után,
 Fáradtságinak végére eljutván,
 Meghala 1095 tájban,  juta mennyországba.
/16
#10B5199D
 Kinek holttestét szekér csak egyedül,
 Váradig vitte minden vonás nélkül.
 Szent életével azt nyerte Istentül,  hogy most mennyben örül.
/17
#53318333
 Dicséret légyen mennyben az Istennek,
 Tisztesség légyen minden ő szentinek,
 Kik már ez földről mennybe érkeztenek,  értünk esedeznek. Ámen.

>5
/1
#E23C750E
 \BJ\bertek, örvendjünk kegyes Istenünkben,
 Az, ki Szent Lászlót, magyarok királyát
 Megdicsőítvén fölemelte lekét
 Az magas égre.
/2
#95EFF70C
 \BV\bének és ifjak, valamennyi hívek
 Szent királyunknak jeles érdemérül
 Éneket mondjunk az egész Hazánkban,  ünnepi napján.
/3
#7851F56C
 \BÉ\bkes erkölcsű vala teste, lelke.
 Tűzzel és vassal mire mások érnek,
 Őtet országunk akarata ellen
 Fogta királynak.
/4
#1BD85256
 \BS\bőt, az éltében Salamon királynak
 Nem vevé az Szent Koronát fejére,
 Nyújt szegényeknek keze bő segédet  nyájas urakhoz.
/5
#7F1BA4D6
 \BD\béli országok veszekednek, hívják
 Közbírájuknak: kegye csillapítja,
 Nincs mi vérontás: magokat megadták
 Szent Koronánknak.
/6
#00445EA9
 \BG\byőzik erkölcsök fejedelmi székét,
 Püspököt, számos papi rendet állít,
 Nagy monostrokkal gyarapítja szerte  széjjel hitünket.
/7
#98BD5886
 \BT\bemplomot Várad nyere Szűz Anyánknak,
 Tőle, kit tisztelt, kinek olvasóját
 Kardra fűzvén ment haragos pogánynak  ellene, s győzött.
/8
#BDF9D470
 \BS\bzent imádságban gyakran úgy találták,
 Égben hogy földről ragadták az őrző
 Angyalok: látták ragyogóan olykor  mennyei fénnyel.
/9
#DF58376B
 \BA\bz tatár egykor szaladásnak indult,
 Hányta sok pénzét, hogy hamar szaladjon.
 Az könyörgésed csudaképen az pénzt  váltja kövéccsé.
/10
#A906BF50
 \BL\best vet a Kunság, veszedelmes helyre
 Jut vitéz néped, nyomorítja éhség,
 Szarvasok, vastag bivalyok sietnek  életet adni.
/11
#FF75052A
 \BM\básodik Mózest ugyan is csudálunk
 Benned, hogy szomjas katonáid akkor
 Kősziklából friss vizet untig ittak,  általad éltek.
/12
#E808BBA4
 \BG\bondolád és től fogadást, hogy indulsz
 Szép magyar néppel Jeruzsálem-földre.
 Kérnek, hogy császár-hatalommal indulj,  róla föl adtál.
/13
#AA1BE313
 \BM\bég is elkészülsz fejedelmi társul,
 Vad, pogány néptől, szerecseny töröktől
 Szent örökséget szabadítni, más az  isteni végzés.
/14
#E54C695B
 \BÚ\brban elnyugszol, szomorú magyarnak
 Szent halálod, mert feketés ruhában
 Három esztendőt-szaka gyászol, eltilt  muzsika szót is.
/15
#F970C333
 \BA\bz midőn vinnék tetemét temetni,
 Váradon rendelt maga már koporsót.
 Úri kísérők lefeküdnek útban,
 S mélyen alusznak.
/16
#988B54A6
 \BA\bz szekér, melyben vitetett a holt test,
 Hat lovak nélkül sebesen csörögvén
 Várad útjának megyen helyre: senki  útba sem érte.
/17
#6E84804D
 \BM\béglen eltart a temetési pompa,
 Egy gonosz szájú: „büdös, ím, ezen test",
 Monda, száját az sokaság csudálta  hátra tekertnek.
/18
#86EFE97D
 \BB\bűne bánatján segedelme légyen
 Néki Lászlótól, könyörög, s tapasztal
 Hirtelen mennyből segedelmét, íme:  szája helyén van.
/19
#56C19ACC
 \BS\bánta, vak, poklos, süketes, mirigyes,
 Láb, szem, ín, fül, tag nyavalyáiból ment,
 Tart azért téged maga szenti közt az  római szentszék.
/20
#470EACA0
 \BA\bz szerencsétlen török háborúban
 Győr szerencsés volt, Koponyádat ingyen
 Nyerte Váradról, örömünket ez nap  hirdeti nyelvünk.
/21
#CDAED7FD
 \BD\böghalált űzött be gyakorta tőlünk
 Közbenjárásod magas Istenünknél:
 Nyerj esőt száraz napokon s ha szükség,  tiszta időt is.
/22
#D4750B9C
 \BS\bzent Király! Szálljon mireánk egekből,
 Amidőn téged szoros ügyeinkben
 Szent ereklyédhez folyamodva kérünk,  mennyei áldás. Ámen.

>6
/1
#904EC51D
 Pannóniának, Magyarországnak ragyogó Csillaga,
 Béla királynak szerelmes rajzatja.
/2
#659DB6F2
 \BE\bgész magyarság Szent László királyt mely nagyra becsülte,
 Mert nemzetünknek volt kegyes vezére.
/3
#2F0096A6
 \BS\bzent László vala romlott hazánknak második királya,
 Ki Gejza után választatik vala.
/4
#77F55A27
 \BA \bSzűz Máriát, Krisztusnak Anyját, Szent László tisztelte,
 Szüntelen őtet segítségül kérte.
/5
#FD881D66
 \BP\bogány kunokkal és tatárokkal mikoron harcolna,
 Szent Rozáliom volt nagy erőssége.
/6
#CD8A7D90
 \BÍ\bgy segíttetik s bátoríttatik, ki Szent Szüzet kéri,
 Minden ügyében szent erejét érzi.
/7
#BBC5DDC3
 \BD\bicséret mennyben és széles földön a Szentháromságnak,
 És nagy tisztesség Szent László királynak. Ámen.

>7
/1
#6C0B97D7
 Isten, első királyunkkal, aki frigyet kötöttél,
 Máriával, Asszonyunkkal pártfogást pecsételtél.
 A mi édes atyáinkkal szerelmünkhöz esküdtél,
 Szent életű királyokkal minket ékesítettél.
/2
#3B442818
 \BM\bint Izraelt Egyiptomból Kánán földjére vitted,
 Azonképpen Szkítiából magyar népet vezetted.
 Pannóniának földében megörökösítetted,
 És vélük az igaz hitben magad megismertetted.
/3
#44E70BBE
 \BA\bpostoli királyunknak Szent Istvánt választottad,
 És mások közt Szent Lászlónak nagy kegyelmedet adtad,
 Szomjúságát magyaroknak átala megoltottad,
 Gyomrából a kősziklának forrásvizedet adtad.
/4
#E359A8E6
 \BA\bz ő élte szentségének oly nagy volt méltósága,
 Hogy engedett kérésének szarvasok sokasága,
 Kikkel táplálta életét az ő magyar népének,
 Ki viszont hálaadást tett irgalmas Istenének.
/5
#E3C45E8A
 \BM\bidőn Szent László királyunk a rút tatárt kergette,
 Hogy ő véle szembe szálljunk, azt meg nem szenvedhette,
 Az ellenség csalárdsággal sok kincsét elhintette,
 De Szent László imádsággal kőváltozásban tette.
/6
#D1EAFFA7
 \BS\bzent László temetésére az Isten azt rendelte,
 Körös vizén a szekere, hogy szárazon mehetne,
 Mely szentséges koporsóját Nagyváradra vezette,
 Vévén élte koronáját testét eltemettette.
/7
#93640D55
 \BO\bh, Szent László, édesatyánk, nyújtsad szent oltalmadat,
 Hogy ellenség jöjjön reánk, ne engedd fiaidat,
 Szentséged ne felejtkezzen, légyen a mi részünkre,
 Hogy az Isten emlékezzen feltett szövetségünkre!

>8
/1
#13537006
 Áldott légyen Isten az ő kedves sok szentiben,
 Kiket felmagasztal mennybéli dicsőségében,
 És részeltetett itt e földön,
 A Krisztusnak Szent Keresztében,
 Üldözések szenvedésében.
/2
#ECF30EFE
 \BB\boldog ily szentek közül Szent László király vala,
 Ki a szent Istennek éltében kedvet talála,
 Az angyali Szent Koronának,
 Magyar nemzetnek és országnak
 Második szentséges királya.
/3
#8D7E17C2
 \BT\biszta szép életről fogadást Istennek téve,
 Ki szeretetéből házastársat nem is véve,
 Sőt élete veszedelmére,
 Egy fogoly szűz leány kértére
 A pogány ellen harcra méne.
/4
#D201751C
 \BB\bátor és szerencsés vitézsége olyan nagy volt,
 Hogy ő idejében senki sem látott hasonlót.
 Mert az ő tiszta szent élete,
 Érdemes volt, hogy angyalok is
 Még megharcolnának mellette.
/5
#D3B806FF
 \BH\badakat viselvén mindenkor bízék Istenben,
 És azért szerencsés dolga vala mindenekben,
 Mert maga javát nem keresé,
 Hanem mi lenne nép javára,
 És csak mi volna Istenessé.
/6
#CEE09025
 \BI\bnnen az Úr Isten tábora szorultságában,
 Gyakran megsegíté őtet többféle csudákban,
 Száraz kősziklából vizekkel
 Hadait tartá pusztaságban,
 Szarvas, bivaly húsételekkel.
/7
#EDD027C1
 \BK\begyetlen tatárok midőn rabolván országot,
 Magokkal vinnének prédabeli sok jószágot,
 Szent László király hadaival
 Érkezvén reájuk rohana,
 És bátran vélük megharcola.
/8
#BBAEE357
 \BS\bzaladván sok szép pénzt magok után elhintének,
 Kapózó kergetőktől, hogy megmenekednének.
 De Szent László imádságára,
 A pénz mind mindjárt kővé vála,
 Pogányok nagyobb romlására.
/9
#74EE8A3E
 \BI\bsten intéséből Körös folyó vize mellett,
 Boldog Szűznek templomot építtetett és emelt
 Váradon, s másutt több egyházat
 Fundála és püspökségeket,
 Adván hozzájuk nagy jószágot.
/10
#C28FA547
 \BO\bly lelki örömmel és szívbéli buzgósággal,
 Istenhez emelkedik ő nagy szeretete által,
 Hogy az égre is emelkedék,
 És mennyei világossággal,
 Dicsőült teste fényeskedik.
/11
#884F20AE
 \BD\be hogy már állandó, örök boldogságba jusson,
 Dicsőült Karok közt üljön mennyei trónuson,
 Megvetvén világ hiúságát,
 A lelke testétől elválván
 Megnyeré boldog kívánságát.
/12
#0B9319C5
 \BS \bmár hogy Várad felé teste tetemét vinnének,
 És egy helyt útjukban tovább nyugovást tennének,
 Szent teste szekér s lovak nélkül
 Elragadtaték, s helyre juta,
 Hol több csuda is lőn az égből.
/13
#18FB9FF5
 \BM\bint szerette légyen egész ország a Szent Lászlót,
 Megtetszik, mert felvőn három esztendeig gyászot,
 És addig muzsikát nem hallott,
 Mert kinek-kinek édes kedvét,
 Szent László maga után elvont.
/14
#CAB43620
 \BÉ\bdes Királyunknak drága, nemes, szent erkölcsét,
 Kövessük, magyarok, látván élte ily gyümölcsét:
 Hogy hozzá szívünk kívánságát,
 Istent dicsérvén megnyerhessük,
 Mi is mennyország boldogságát. Ámen.

>9
/1
#C1D88D08
 Éjfél után egy óra már,
 Szent László napja van már.
 
 \BÖ\brüllenek szíveikbe,
 Vigadjanak lelkeikbe!
 
 \BH\bogy Szent László lett pátrónánk,
 Szűz Mária édesanyánk.
 
 \BT\bávoztassa a döghalált,
 Eeressze ránk szent áldását!
 
 \BÉ\bjfél után egy óra már,
 Szent László napja van már.

>10
/1
#DBA26F9D
 Szent László, Isten szolgája,
 Magyarok fényes Lámpása
 Esedezzél mi érettünk,
 Hogy mennyben véled lehessünk!
/2
#E6F6C96D
 \BÓ\b, Szent László, magyar király,
 Mi nemzetünk óhajtva vár,
 Imádkozzál miérettünk,
 Hogy Isten kedvében legyünk!
/3
#ACE59679
 \BS\bzent László, verd ellenségünk,
 Ördögöt űzd el mitőlünk,
 Ó, Szent László, légy oltalmunk,
 Ne pusztuljon mi országunk!
/4
#2D403CE6
 \BK\biért dicséret Atyának,
 Légyen az ő Szent Fiának,
 Mi országunk oltalmának,
 Boldog Szent László királynak. Ámen.

>11
/1
#DF6C5F13
 Serény szolgádat Úr kedveli,
 Minden javára rendeli,
 \ISzent László, Isten szolgája, magyar .\i
/2
#B77D837A
 \BV\bigasztalj Urad javára,
 Reád bízott országodra!
 \ISzent László, Isten szolgája, magyar .
/3
#0EAB2467
 \BH\bűségedet megmutattad,
 Néki magad meg tartottad,
 \ISzent László, Isten szolgája, magyar .
/4
#A82B3507
 \BK\brisztus hitét gyarapítván,
 Romlott voltát gazdagítván,
 \ISzent László, Isten szolgája, magyar .
/5
#D5F0720C
 \BÓ\b, te dicső serény szolga,
 Máriának fő hadnagya,
 \ISzent László, Isten szolgája, magyar .
/6
#70E96CCD
 \BT\be vagy hitünk erős pajzsa,
 Egyházunknak bizodalma,
 \ISzent László, Isten szolgája, magyar .
/7
#4DAA098E
 \BH\bogy Máriát Pátrónánknak,
 Te hagytad, hívjuk Anyánknak!
 \ISzent László, Isten szolgája, magyar .
/8
#585C1D8B
 \BT\be vagy nemzetünk doktora,
 Szent Istvánnak tanítványa,
 \ISzent László, Isten szolgája, magyar .
/9
#FFEC105F
 \BL\bégy tovább nékünk szószólónk,
 Isten előtt oltalmazónk!
 \ISzent László, Isten szolgája, magyar .

>12
/1
#BD0B0DB8
 Dicsőséges Szent László, istennek szolgája,
 Népeidért légy szóló, nemzetünk istápja,
 Mai nap nagy örömben üljük Ünnepedet,
 Gyászba borult hazánkra fordítsd szemeidet!
/2
#D1710771
 \BÉ\bdes árva hazánknak kívánt fő bajnoka,
 Dicsőséges Szent László, ki voltál bástyája,
 Török, tatárok ellen fegyverrel rontója,
 Más sok ellenségeknek vagy rettentő karja.
/3
#01B68247
 \BS\boha nagy szentségedet el nem felejthetjük,
 Sok csudatételedet szívből emlegetjük,
 Hozzánk való jó voltod szüntelen köszönjük,
 Mindörökké az Istent tebenned dicsérjük.
/4
#CAFCBE99
 \BK\bérünk téged, Szent László, nyerj irgalmat nékünk,
 A mennyei Krisztusnál könyörögj érettünk,
 Hogy légyen reménységben ez kis magyar népünk,
 Oltalmadban vigadjon idelent, míg élünk.
/5
#84BBD12A
 \BH\bozd régi szabadságra megromlott hazánkat,
 Tejjel és mézzel folyó drága országunkat,
 Add meg még egyszer érnünk szép szabadságunkat,
 Országunkban regnáló magyar királyokat.
/6
#C5B302BA
 \BK\brisztus, Istennek Fia, nekünk is engedje,
 Midőn mi életünknek elközelget vége,
 Mi is Szent László után vitessünk mennyekbe,
 Angyalok, kerubimok rendes seregébe! Ámen.

>13
/1
#A735D82C
 Zengd, magyar nemzetünk, Szent László szentségét,
 Kiből áldja lelkünk Isten ő felségét.
 Áldjuk e nagy királyt, e nap tőlünk elvált,
 De az égből néz reánk, ha hozzá kiált hazánk.
/2
#C0A9A966
 \BF\bejedelemsége csupa irgalmasság,
 Minden vitézsége keresztény jámborság,
 Így tatár fortéját  s széjjelhányt aranyát,
 Szemét égre fordítja,  kövekké változtatja.
/3
#6FD7C1F1
 \BK\bőszál forrásokat önt ki italára,
 Erdő szarvasokat küld népe számára,
 Nem hagyja el az ég,  mert égő reménység
 Volt Szent László szívében  s mint benne, úgy népében.
/4
#BCFC2412
 \BI\blyen szent életet szent halál követte,
 S midőn népe testét Várad felé vitte,
 Alvók oldalától, szereke magától
 Elsietett előre  a rendelt temetőre.
/5
#99154248
 \BD\bicső fejedelmünk, ki ilyen csudákkal
 Ragyogsz, add, hogy szívünk égjen oly szikrákkal,
 Melyek gerjesztenek  s véled emeljenek
 Mennyei királyságra  az örök boldogságra.
/6
#4FC5A039
 \BA\bdd, Nagyasszonyunkat, Máriát szeressük,
 Ebben királyunkat, s példádat kövessük,
 Ha te s e Nagyanyánk  gondjában lesz Hazánk,
 Ki boldogabb nálunknál?  S hanyatló országunknál! Ámen.

>14
/1
#12D7AD28
 Dicsérjük az Istent nagy szentjében,
 Kit Árpád háza szült vala régen,
 Hogy legyen a magyar pártfogója,
 Ki minden ínségből őt megóvja.
/2
#84A366DE
 \BM\bennybéli áldás volt királyságod,
 Mert törvényt tettél és igazságot.
 Szegletkő szívedben a jámborság,
 Erősen áll azon minden ország.
/3
#8A4E1AAA
 \BV\bitéz karjaidat, ah, mint félte
 A kereszt és a hon ellensége,
 Mert az Isten angyala járt véled,
 S csodával halmozta minden lépted.
/4
#99F1A002
 \BN\byertél az Istentől csodafüvet,
 Mellyel a döghalált messze űzted,
 Éh-szomjas népedért imádkozol,
 S csorda jő sziklából, forrás omol.
/5
#84DF63A1
 \BT\béged a keresztény fejedelmek
 Mint hősök csillagát kiszemeltek,
 Hogy vezesd keletre a szent hadat,
 Mely Krisztus sírjának oltalmat ad.
/6
#B85503DF
 \BT\be, ki gyönyörködtél mondhatatlan
 Általad épített templomokban,
 Könyörögj, engedje Isten nékünk,
 Egyház és hon díszét építenünk. Ámen.

>15
/1
#055B12C8
 Szent László, csuda királyunk,
 E földről hozzád kiáltunk!
 Nagy ínségünkben sírván fohászkodunk.
 Ó, nézd, mint futunk, rémül városunk,
 Mely szörnyű rajtunk földindulásunk,
 Oltalmazz, kérünk, mi édes Gyámolunk!
/2
#845D4D8E
 Ím, lásd, falaink omlanak,
 Szent helyeink szakadoznak,
 Talán utolsó veszedelmet hoznak?
 Ha nem segítesz, Pártfogónk ki lesz?
 Kérünk, Istenhez járulj s esedezz,
 Mert szószónknak Nagy-Győr téged nevez.
/3
#AE99990C
 Tekintsd Szent Fejed váltságát,
 Pogány kéztől megtartását,
 Nyerjed földünknek csendes nyugovását!
 Ó, mi nagy Szentünk, tégedet kérünk,
 Könyörögj értünk, ó, édes Vérünk,
 Üdvözítőnket engeszteld meg nékünk!
/4
#33F4FC4A
 Siratjuk bűnünk rútságát,
 Istenünk súlyos bántását,
 Ígérjük életünknek jobbítását.
 Könyörülj rajtunk, ó, Szent Királyunk,
 Nagy mi ostorunk, vedd el azt rólunk,
 Úgy lesz tenéked víg hálaadásunk! Ámen.
