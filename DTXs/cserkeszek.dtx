;**************************************
; "Cserkészek daloskönyve"
;	összeállította: Ivasivka Mátyás
;
; Forrás: Márton Áron Kiadó;  9., javított és bővített kiadás
; 2008/06/01 begépelte, átformázta RJ
;**************************************
NCserkészek daloskönyve
RCserkész
CIfjúsági

>Himnusz
#F228552E
 Isten áldd meg a magyart, jókedvvel, bőséggel,
 Nyújts feléje védő kart, ha küzd ellenséggel.
 Balsors akit régen tép, Hozz rá víg esztendőt.
 Megbűnhődte már e nép a múltat s jövendőt.

>Szózat
#00B75DAD
 Hazádnak rendületlenül légy híve, ó magyar;
 Bölcsőd az, s majdan sírod is, mely ápol s eltakar.
 A nagy világon e kívül nincsen számodra hely;
 Áldjon vagy verjen sors keze: itt élned, élned, s halnod kell!

>Boldogasszony
/1
#E38543C5
 Boldogasszony Anyánk, régi nagy Pátrónánk!
 Nagy ínségben lévén, így szólít meg hazánk:
 Magyarországról, édes hazánkról,
 Ne felejtkezzél el szegény magyarokról!
/2
#E3D56FF4
 Ó, Atyaistennek kedves, szép leánya,
 Krisztus Jézus Anyja, szentlélek mátkája!
 Magyarországról, édes hazánkról,
 Ne felejtkezzél el szegény magyarokról!
/3
#AE0AD9D0
 Nyisd fel az egeket sok kiáltásunkra,
 Anyai palástod fordítsd oltalmunkra.
 Magyarországról, édes hazánkról,
 Ne felejtkezzél el szegény magyarokról!
/4
#4068F0A3
 Sírnak és zokognak árváknak szívei,
 Hazánk pusztulásán özvegyek lelkei.
 Magyarországról, édes hazánkról,
 Ne felejtkezzél el szegény magyarokról!

>Szt.Istvánhoz
/1
#01A8B0C2
 Ah, hol vagy magyarok tündöklő csillaga,
 Ki voltál valaha országunk istápja?
 Hol vagy István király? Téged magyar kíván.
 Gyászos öltözetben teelőtted sírván.
/2
#C98B8AEE
 Rólad emlékezvén csordulnak könnyei,
 Búval harmatoznak szomorú mezei,
 Lankadnak szüntelen vitézlő karjai,
 Nem szűnnek iszonyú sírástól szemei.
/3
#BD9CA52E
 Előtted könyörgünk, bús magyar fiaid,
 Hozzád fohászkodunk, árva maradékid.
 Tekints, István király, szomorú hazádra,
 Fordítsd szemeidet régi országodra.

>46.zsoltár
/1
#20EE0523
 Erős vár a mi Istenünk, jó fegyverünk és pajzsunk.
 Ha ő velünk, ki ellenünk, az Úr a mi oltalmunk.
 Az ősellenség most is üldöz még,
 Nagy a serege, csalárdság fegyvere,
 Nincs ilyen több a földön.
/2
#8751C22C
 Erőnk magában mit sem ér, mi csakhamar elesnénk;
 De küzd értünk a hős vezér, kit Isten rendelt mellénk.
 Kérdezed: ki az? Jézus Krisztus az,
 Isten szent Fia, az ég és föld Ura,
 Ő a mi diadalmunk.
/3
#032E1AF5
 E világ minden ördöge ha elnyelni akarna,
 Minket meg nem rémítene, mirajtunk nincs hatalma.
 E világ Ura gyúljon bosszúra:
 Nincs ereje már, reá ítélet vár,
 az ige porba dönti.
/4
#5B74D438
 Az ige kőszálként megáll, megszégyenül, ki bántja.
 Velünk az Úr táborba száll, Szentlelkét ránk bocsátja.
 Kincset, életet, hitvest, gyermeket:
 Mind elvehetik, mit ér ez őnekik!
 Mienk a menny örökre!

>Fel, barátim!
/1
#B0CF1815
 Fel, barátim, drága Jézus zászlaja alatt,
 Bátran, bátran, segedelme diadalmat ad.
 Bízzatok, mert Jézus eljön, ő a fővezér,
 Zengje ajunk, hozzád esdünk győzedelemért.
/2
#E5F20886
 Lám, a sátán ser'ge talpon, szembetörni kész;
 A legbátrabb harcosoknak bátorsága vész.
 Bízzatok, mert Jézus eljön, ő a fővezér,
 Zengje ajunk, hozzád esdünk győzedelemért.
/3
#B949A577
 Ránk borítja fátyolát a csendes alkonyat,
 Elpihent a kis madár a fáradt lomb alatt.
 Illatsóhajt küld az alvó, néma messzeség,
 Ránk mosolyg a csillagfényes nyári esti ég.
/4
#B7A9D038
 Édes hangú kis madárka csattogó dala,
 Tiszta szívű víg cserkészek gondtalan hada
 Fennen-vígan hirdeti: az élet, ó, mi szép,
 Istenadta drága kincs, dicsérjük őt ezért!

>Térj magadhoz…
/1
#00901ABF
 Térj magadhoz, drága Sion, van még neked Istened!
 Ki atyádként fölkaroljon, s szívét ossza meg veled!
 Azt bünteti, kit szeret, másképp ő nem is tehet!
 Sion, ezt hát gondoljad meg, szabj határt bús gyötrelmednek.
/2
#CD9DFFDA
 Hullámok, ha rémítenek mérhetetlen víz felett,
 S a habok közt szíved remeg, hogy sírod is ott leled;
 Ha aludni látod őt, ki reményed és erőd:
 Sion, soha ne feledd el, ő megvívhat tengerekkel!

>Magyar szentekről
/1
#7FC0AAFE
 Isten, hazánkért térdelünk elődbe.
 Rút bűneinket jóságoddal född be.
 Szent magyaroknak tiszta lelkét nézzed,
 Érdemét idézzed.
/2
#B5999EF0
 István királynak szíve gazdagságát,
 Szent Imre herceg kemény tisztaságát,
 László királynak vitéz lovagságát,
 Ó, ha csak ezt látnád!
/3
#753D7475
 Boldog Özsébnek lángjai lobognak,
 Bús magyar éjben árva magyaroknak
 Vigasztalásul fehér pálosoknak
 Imái ragyognak.
/4
#00EDD8A2
 Szent Erzsébetből hős szeretet árad,
 Margit imái vezekelve szállnak,
 S minket hiába, Uram, ne sirasson
 Áldott Boldogasszony.
/5
#F6DF9A6F
 Ránk bűnösökre minden verés ránk fér,
 De könyörögnek ők tépett hazánkért,
 Hadd legyünk mink is tiszták, hősök, szentek,
 Hazánkat így mentsd meg!

>Naphimnusz
/1
#41D18D1F
 Mérhetetlen Úr, ki vagy a kéklő mennyekben:
 Téren innen s túl igazak ajka hirdessen;
 Testvérünk, a Nap, nevedet zengje szárnyalva nagy fennen!
/2
#20548F02
 Dicsérjen a Hold, halaványarcú húgocskánk,
 S mind a termő Ég, akiben annyi csillag szánt;
 Dicsérjen a Föld, patakot, rétet ringatva, bölcs dajkánk!
/3
#2C5C5329
 Friss nénénk, a Szél, mikoron zúgva gátat tép;
 Öcskösünk a tűz, egeket járva táncosképp;
 Áldjon minden szív, örökkön ifjú, mennybéli Fényesség!

>Magnificat
/1
#FD5F15C1
 Magasztalja lelkem az én Uramat,
 Örvendezve élek szárnyai alatt.
 Szolgálója lettem, lepillantott rám,
 Alázattal zengi dicséretét szám.
/2
#17DB6B27
 Ami velem történt, nem volt soha még,
 Boldognak mond engem minden nemzedék.
 Nagyot művelt rajtam hatalmas keze,
 Jóságában százszor szent az Ő neve.
/3
#7EFABBA5
 Irgalma leárad, népünkre borul,
 Félelmünkre válasz és szívemre hull.
 Mindenható karja csak egyet suhint;
 S a kevélykedőket széjjelszórja mind.
/4
#9246B66B
 Letöri a gőgöst képzelt magasán,
 Alázatos hívét áldja trónusán.
 Éhezőknek étket tele kézzel ád,
 De a gazdagoktól zárja kapuját.
/5
#45D1A46E
 Igazak dolgára örök gondja van,
 Szívén hordja népét nagy irgalmasan.
 Atyáinkhoz szólott évezreken át,
 Benne bízó népe ismerte szavát.
/6
#3DC07C0D
 Dicsőség az Úrnak, áldjuk az Atyát,
 Fiút és a Lelket századokon át.
 Tudom, hogy mit adtál: töviskoronát;
 Áldott legyél érte minden koron át.

>Zsoltár gyermekhangra
/1
#9AA87B02
 A Jóisten őriz engem,
 Mert az Ő zászlóját zengem;
 Ő az áldás, ő a béke,
 Nem a harcok istensége.
/2
#7046EFBA
 A Jóisten örök áldás,
 Csíra, élet és virágzás.
 Kard ha csörren, vér ha csobban,
 Csak az ember vétkes abban.
/3
#09E0347D
 A Jóisten őriz engem,
 Mert az Ő országát zengem:
 Kell, hogy Isten áldja, védje,
 Aki azt énekli: Béke!

>Reggeli himnusz
/1
#80741AAF
 Már kél a fénynek csillaga,
 Esengve kérjük az Urat,
 Járjon ma mindenütt velünk,
 ne rontsa ártás életünk.
/2
#4BA416C1
 Nyelvünket fogja fékre ma,
 Ne szóljon rút perek szava.
 Szemünket védőn óvja meg,
 A hívságot ne lássa meg.
/3
#B09F8031
 Lakjék szívünkben tisztaság,
 Távozzék minden dőreség.
 A testnek dölyfét törje meg
 Étel- s italban hősi fék.
/4
#6AB723D3
 Hogy majd a nap ha távozott,
 S az óra újra éjt hozott,
 Lemondásunk szent éneke
 Legyen az Úr dicsérete.
/5
#636C19A2
 Dicsérjük az örök Atyát,
 dicsérjük egyszülött Fiát,
 S a Lelket, a vigasztalót,
 Most és örök időkön át. Amen

>Esti himnusz
/1
#781F1478
 A napvilág leáldozott,
 Kérünk, Teremtőnk, tégedet,
 Maradj velünk kegyelmesen,
 Őrizzed, óvjad népedet!
/2
#32D59DE9
 A rossz álmok távozzanak,
 És minden éji képzelet.
 Ellenségünket űzzed el,
 Hogy testünket ne rontsa meg.
/3
#FFE5D16C
 Add meg, kegyelmes jó Atyánk,
 És Egyszülöttje, add nekünk,
 És Szentlélek, Vígasztalónk,
 Örök fölséges Istenünk. Amen

>Étkezés előtt
#18C8B4CA
 Jöjj el Jézus, légy vendégünk,
 Áldd meg, amit adtál nékünk.
 Adjad, Uram, hogy jól essék,
 Jézus neve dicsértessék.

>Kalevala 45.ének
#358B4056
 Tőlem semmi ki nem telne,
 Ha nincs Alkotóm kegyelme.
 Adj, ó Alkotóm, kegyelmet,
 Hozz, óh Isten, segedelmet.

>Étkezés előtt
#443D8A45
 Aki ételt, italt adott,
 Annak neve legyen áldott.
 Mi jóllaktunk, hála Isten!
 Annak is adj, kinek nincsen.

>Esti dal
/1
#B6026DDF
 Erdő mellett estvéledtem,
 Subám fejem alá tettem.
 Összetettem két kezemet,
 Úgy kértem jó Istenemet:
/2
#960FE2B9
 Én Istenem, adjál szállást,
 Már meguntam a járkálást,
 A járkálást, a bújdosást,
 Az idegen földön lakást.
/3
#481826DF
 Adjon Isten jó éjszakát,
 Küldje hozzánk szent angyalát,
 Bátorítsa szívünk álmát,
 Adjon Isten jó éjszakát.
/R
#E1A77A2F
 Bátorítsa szívünk álmát,
 Köszöntsük a Szűz Máriát.

>Shalom haverim
#FA5E3325
 Shalom haverim, shalom haverim, shalom, shalom,
 Le hitra ot, le hitra ot, shalom, shalom!

>Az Úr veletek
#137CC1E6
 Az Úr veletek, az Úr veletek, ó áldás, béke,
 Az Úr veletek, az Úr veletek, ó áldás, béke!

>Esti búcsú
#830F9500
 Adj, Uram, csöndet, békességet a földnek.
 Jó éjszakát.

>Ardere…
#30ADE1A8
 Ardere et lucere,
 ardere et lucere,
 ardere et lucere.

>Christus vincit
#2F3E8B36
 Christus vincit,
 Christus regnat,
 Christus, Christus imperat!

>Szt.Imre
/1
#FDB1DDDC
 Szent Imre herceg, Magyarország éke,
 Szűztiszta élet legszebb példaképe.
 serdülő ifjak pajzsa, menedéke:
 Kérd Istent értünk!
/2
#B8E6F6B9
 Jó magyar népünk alighogy megtére,
 Már jeles szentet termett Árpád vére,
 Gyors választ adván Isten kegyelmére.
 Kérd Istent értünk!
/3
#20B05823
 Jól kiművelve ifjúkorba értél,
 Égi hívásra a szűk útra léptél,
 Emberi testben angyal-módra éltél.
 Kérd Istent értünk!
/4
#6625B006
 Szent Imre herceg, magyar ifjak pajzsa,
 A veszélyt tőlünk karod távol tartsa,
 Ne engedj jutnunk kísértésbe, bajba.
 Kérd Istent értünk!

>Magyar cserkészinduló
/1
#2D3CF5D5
 Fiúk, fel a fejjel, a harsona zeng,
 Álljunk csatasorba vidáman,
 Ránk vár a világ, ez a harc a mienk,
 Katonái vagyunk valahányan.
 Jó fegyverünk izmos karunk,
 Égő szemünk, vidám dalunk.
 Amerre nézünk, megterem
 A győzelem, a győzelem.
/2
#0AF1DFAF
 Szabadba, fiúk! A nap arca nevet,
 Ott pezsdül a friss, tüzes élet.
 Járjuk be a mezőt, meg a rengeteget,
 Szabad ott a szabadban a lélek!
 Ott szemben, szívben tiszta láng,
 S az Isten arca néz le ránk.
 Leheletén ott megterem
 Erő, szabadság, győzelem!

>Elindult a tutajflotta
/1
#C2163921
 Elindult a tutajflotta Kralovánból,
 Díszlövések surrogtak a mozsarakból.
 Parton állt a falu népe,
 A sasok meg a tutajok tetejébe'.
/2
#5CF4592E
 Künn a parton sorakozott Janó törzse,
 Fedélzetén álldogált a Sasok őrse.
 Éljent adtak, éljent kaptak,
 Míg végre az ebédlőbe beszakadtak.

>Mócsy Jánostól
#2EB68717
 Megy a cserkész, megy a cserkész hosszú bottal.
 A hátán egy sportüzlettel, bádogokkal.
 Van ott fejsze, sátorponyva,
 Az egész egy ruhatár és mozgókonyha.

>Fel, fel, cserkészpajtás!
#0E63D586
 Fel, fel, cserkészpajtás, fel, fel, cserkészpajtás, fel a magas hegyre!
 Szedjünk kék ibolyát, szedjünk kék ibolyát a zöld nyakkendőre!
 Jó a cserkész dolga, nincsen semmi gondja.
 Hogyha megéhezik, hogyha megszomjazik, hátzsákját kibontja.

>Nézd!
#F7B369DE
 Nézd, de szépen lángol a tábortűz!

>Daltól hangos
/1
#714DDB3B
 Daltól hangos erdő, mező, berek, merre járunk mi, cserkészek.
 Dalol a szív, muzsikál a lélek, vele dalol a természet.
 Aki dalol, sohse fárad el, aki felnéz, sohse csügged el.
 Mert a dalos kedvet a Jóisten adja, ő ad erőt a nagy harcra.
/2
#166DEE00
 Dobog a föld lépéseink alatt, visszhangoznak a kőfalak.
 Öreg házak közt ifjú legények dacosan, keményen lépnek.
 Nyisd ki, anyám, az ablakodat, csókold meg a cserkészfiadat!
 Szép Magyarországnak kicsi katonája, büszke szemmel nézhetsz rája.

>Forrjon össze
/1
#D852C759
 Forrjon össze dalban ajkunk, úgy, amint szívünk is egy.
 Hogyha vígan összetartunk, munka, harc sikerre megy.
 Váll a vállhoz, kéz a kézbe', ez a cserkész jóba', vészbe'!
 Váll a vállhoz, kéz a kézbe', ez a cserkész jóba', vészbe'!
/2
#EC0C8DBE
 Liliomnak hordjuk ékét tiszta, bátor szív felett,
 Mely legyőzi szenvedélyét, és segít, ahol lehet.
 Váll a vállhoz, kéz a kézbe', ez a cserkész jóba', vészbe'!
 Váll a vállhoz, kéz a kézbe', ez a cserkész jóba', vészbe'!

>Megjött…
#A34AA76F
 Megjött már a fecskemadár, fészket rakott nálunk.
 Hívogat már a napsugár, nagytáborba vágyunk.
 Virág nyílik a hegyoldalon, nincs szebb annál semmi.
 Harsog a kürtszó, cserkészpajtás, táborba kell menni.

>A madár…
#E1C21377
 A madár fütyüli már az ágon,
 Magyar cserkés a legelső a világon.
 Hogyha megy, hogyha indul táborába,
 A jó Isten mosolyogva néz le rája.
 Kipakkol és felveri sátorfáját,
 Dalos könyvvel, vígan végzi munkáját.
 Ha dolgozik, énekel, táborozik lélekkel,
 Akadályt ő nem ismer a világon.

>Nagytáborba indulnék
#FC717173
 Nagytáborba indulnék, nem bírja a hátam,
 Mindent összepakoltam, amit csak találtam.
 Hej, édes, kedves zsákocskám, én csak attól félek,
 Hogy összenyomsz, és kiszáll belőlem a lélek.

>Egyszer esik…
/1
#D7366938
 Egyszer esik esztendőben nagy tábor,
 Egszer eszünk szárazbabot csajkából!
 Te szakács, te szakács, a babra vigyázz,
 Mert ha jő Kozma úr, az lesz csak a gyász!
/2
#A611C3E3
 Hogyne volna, hogyne volna jó világ,
 Ez a tábor csupa derék mákvirág!
 csebogár, csebogár, szentjánosbogár!
 Táborban, sátorban, jaj, de jó a nyár!
/3
#5EFF40A1
 Egyszer esik esztendőben karácsony,
 Egyszer főztem száraz babot bográcson!
 Csebogár, csebogár, csöpög a kanál,
 Tinéktek, cserkészek, főzzön a tatár!

>Ici-pici…
#ED849250
 Ici-pici csapatunk, a cserkészek mi vagyunk,
 Télen-nyáron ballagunk, mégis vidámak vagyunk.
 Trararara, trararara bumm!

>Valami bűzlik…
/1
#96145307
 Valami bűzlik, valami bűzlik Dániában, Dániában:
 Halszag, halszag! És nem bírjuk orral!
/2
#3FE25370
 Valami bűzlik, valami bűzlik a táborban, a táborban:
 Kozma, kozma! És nem bírjuk orral!

>Már elpihent…
#03B49087
 Már elpihent a táborunk, csak bent az erdőn zizeg a levél,
 Lent szétterült a ködmező, elült a pajkos szél.
 Pajtás, a tűzre tégy kicsinyt, hadd, hogy lobogjon jól a láng,
 Ködben megfürödve kél a hold, az Isten is néz ránk.

>Hűvös az éj…
#075C7270
 Hűvös az éj, vége van a nyárnak,
 Már soka az iskola sem várhat.
 Ó, be szép itt minden este,
 Mért is megyünk vissza Budapestre?

>Regős I.Bevezető
#E8E3F381
 Cserfakéreg bocskorunk,
 Nyírfakéreg nadrágunk,
 Ha szabad becsoszognunk,
 Haj, regő rajta, azt is megadhatja az a nagy Úristen.

>Regős II.Induló
/1
#AA166E19
 Ne fuss, ne fuss, ne fuss, Szent István királyunk,
 Mi nem vagyunk ördögök, hanem te szolgáid.
 Haj, regő rajta, azt is megadhatja az a nagy Úristen.
/2
#D150D2A8
 Kelj fel, gazda, kelj fel, szállott Isten házadra,
 Sereg angyalával, vetett asztalával, tele poharával.
 Haj, regő rajta, azt is megadhatja az a nagy Úristen.
/3
#F143F1EC
 Adjon az Úristen ennek a gazdának
 Két kis ólat, meg egy koszos malacot.
 Az egyikből kifusson, a másikba befusson.
 Haj, regő rajta, azt is megadhatja az a nagy Úristen.
/4
#91E47ACF
 Adjon az Úristen ennek a gazdának
 Négy kis ökröt, két kis bérest,
 Csengős-pengős szekeret, arany ostornyelet,
 Hadd vigye továbbra szegény regölőket.
 Haj, regő rajta, azt is megadhatja az a nagy Úristen.

>Tó partján
#E859B36C
 Tó partján ott ring egy imbolygó csónak,
 hajlong az árboc, mint a nád.
 Tavi tündér ingatja, puha szellő ringatja,
 leng, ring, hajlong, mint a nád.

>Tájékozódás
#4B5C79B8
 Ha előttem van észak, hátam mögött dél,
 Balra a Nap nyugszik, jobbra a Nap kél.
 Tanuljuk meg!

>Viva la musica
#8D03AFCE
 Énekelj, magyar ifjúság,
 Míg a nemzetek meghallják,
 S jobb hajnal virrad ránk!

>Bizakodó
#37F7067C
 Sej, haj, nincsen baj, süt még ránk az Isten napja.

>Szellő zúg…
/1
#E3AA2D35
 Szellő zúg távol, alszik a tábor.
 Alszik a tábor, csak a tűz lángol.
 Rakd meg, rakd meg, cserkészpajtás, azt a tüzet,
 Isten tudja, mikor látunk megint ilyet.
/2
#DBB720B4
 Szellőzúgásnak fárad a hangja,
 Kis falucskának szól a harangja.
 Hallga, hallga, szól a harang: bim-bam, bim-bam,
 Lelkünk mélyén kél rá visszhang: bim-bam, bim-bam.

>Erdő mellett…
#590DACF2
 Erdő mellett kanyarog el a patak, magasra van dagadva.
 Nagytáborba készülődik a csapat, beszállt már a vonatba.
 Bakonybéli pataknál holnap ott is leszünk már.
 Nagytáborba készülődik a csapat, beszállt már a vonatba.

>Fiúk, fiúk…
/1
#B4AFA89D
 Fiúk, fiúk, cserkészfiúk, rajta!
 Vár a Bakony rengetegje, vadja.
 Bejárjuk a Bakony minden táját,
 Ott ütjük fel a sátorunk tanyáját.
/2
#F0BA0A55
 Már ezután úgy éljük világunk,
 Erdők ölén, hegyoldalán sátrunk.
 Kék ég alatt, kristály forrás mellett
 Dalos kedvvel éljük életünket!
/3
#CEDBD43C
 Tábortűznél nótázgatni kezdünk,
 Hadd viduljon ifjú cserkész-lelkünk!
 Hadd lobogjon a láng fel az égig:
 Lelkesítsen minket életünk végéig.

>Nézem az eget
#D18C2CA6
 Nézem az eget, nézem az eget, ragyognak a csillagok,
 De ott köztük, de ott köztük legjobban csak egy ragyog.
 Az a csillag a magyar cserkészcsapatoknak a csillaga;
 Azért csillog a legjobban a többi közt egymaga.

>Erre van…
#CE4B1DC1
 Erre van a gyalogút, erre megy a kocsiút!
 Táborban, sátorban elfelejtek minden bút.

>Esik eső…
#F279BE14
 Esik eső, zúg a gát, adjon az Isten jójszakát!
 Adjon Isten az embernek oly szokást,
 Sej, hogy szeresse mind egymást!

>Elvittek egy cserkészt
/1
#9C2D9E60
 Elvittek egy cserkészt sátor alá lakni,
 Nincs szebb dolog, mint táborban sátor alatt hálni.
/2
#71BD4140
 Sátor alatt hálni, tűz mellett őrt állni,
 Erdő-zúgást elhallgatni, csillagot számlálni.
/3
#9BD049C4
 Sátor alatt hálni, tűz mellett őrt állni,
 Ég fordultát, csillag hunytát, hajnal jöttét várni.

>Főszakács úr…
#64F4D0DE
 Főszakács úr, sejehaj, főszakács úr előveszi bicskáját.
 Kiszámítja, sejehaj, kiszámítja szegény cserkész adagját.
 Nincsen olyan kis kenyér e széles világon,
 Bicskájával, sejehaj, bicskájával mit kisebbre ne vágjon.

>Nyáron virágzik…
#022790AA
 Nyáron virágzik a rózsa,
 Nyáron megyünk nagytáborba.
 Hej, csak már az évnek vége volna,
 Hej, csak mehetnénk már el a nagytáborba!

>Rásüt a nap
#40331F63
 Rásüt a nap, a nyári nap, sej-haj, az erdőre.
 Megy a cserkész hegyen-völgyön előre,
 Egész úton vidám nótaszó járja,
 Szép az élet, cserkészélet, sej-haj, nincs párja.

>Azért, hogy én…
#8F458DA6
 Azért, hogy én cserkész vagyok, sárgarigó galambom,
 Télen, nyáron kirándulok, sárgarigó galambom.
 Botot veszek a kezembe, kalapomhoz árvalányhajat;
 De sokszor jutsz az eszembe, árnyas erdő, kis patak!

>A budai cserkészek
#A8972813
 A budai cserkészek tábort vertek szegények,
 Addig verték, agyalták, agyalták, agyalták,
 Míg szétmentek a balták.

>Azért csónak
#EF52933A
 Azért csónak a csónak,
 Hogy ártson a kacsónak.
 Hogyha csónak nem volna,
 Kézen hólyag sem volna.

>Tábortüzi Credo
#4E718AED
 Hiszek a kevesek egybetett kezében,
 Hiszek a Krisztustest nagy közösségében,
 Hiszek, hiszek a megváltó kereszt erejében,
 Hiszek a szeretet végső győzelmében.

>Hajtsd a csónakot magad!
/1
#B5D52FDB
 A férfiú nem bamba juh, mit birkanyáj lök, húz odább.
 Küzd és akar, feszül a kar, és maga hajtja csónakát.
/2
#AF079FA6
 Meg nem remeg, ha közeleg Bor asszony sziklazátonya:
 El nem merül, odább kerül vidáman hajtott csónaka.
/3
#373BB256
 E szép világot hát szeresd; szeresd hűn embertársadat:
 Ne jajgass, és ne szitkozódj! De hajtsd a csónakod magad!

>Ha összefog…
#B56DC140
 Ha összefog karunk és egybecseng dalunk, mi sziklaerősek vagyunk,
 Ha egy ütemre ver hű szerető szívünk, legyőzhetetlenek leszünk.
 Fel cserkésztáborba, eljött a nyár, erdőknek mélye vár,
 Ha körülüljük tábortüzünk s föllobog a láng, a nóta hangja körbejár,
 Ha földerül a hajnalpír és imánk égbe száll, az Isten arca néz mireánk.

>Röppen a nyíl
#E96C519A
 Napfényes égen mint fürge sólyom,
 Zizzenve, zúgva röppen a nyíl,
 Mint víg madárdal, messzire szárnyal,
 Ég fele törni versenyre hív.
 Nyíl röpte nékünk, víg dakotáknak
 Halld, mit tanít, testvér,
 Ki élte útján, mint nyíl a légben,
 Egyenesen jár, célhoz ér!

>Vár a tábor
#50CC4CE4
 Vár a tábor, az erdő, a mező,
 Vár ránk testvér, a sok kismadár.
 Patakcsobogás, madárdalolás;
 Az Úristen minket templomába vár.

>Télitábori dal
#693C1039
 Csak a szél fújdogál, elpihent már a táj,
 s az erdőn a dalosajkú víg kismadár.
 A fák közt a hold fénye rettegve jár,
 mert fél, hogy nem jön el az új, tarka nyár.
 De ifjú szívünk dala csendülve száll,
 és tőle visszhangzik az alvó határ.
 Az égből, Teremtőnk, vigyázz, ó reánk,
 a földre és reád, te édes hazánk!

>Búcsút int a tűz
/1
#FA471890
 Búcsút int a tűz, halkan jő az éj,
 Útnak indulunk, testvér jöjj, ne félj!
 Új harc vár miránk, zengjen víg dalunk,
 Jézus van velünk, véle indulunk,
 Bármint sújt a sors, jöjjön bármi vész,
 Ajkunk csak dalol, szívünk tettrekész.
/2
#5949D9BB
 Szálljon hő imánk, hozzád, Istenünk.
 Engedj háladalt néked zengenünk.
 Szívünk mint a tűz, izmunk friss acél,
 Égre néz szemünk, tudjuk, ott a cél.
 Bármint sújt a sors, jöjjön bármi vész,
 Ajkunk csak dalol, szívünk tettrekész.

>Erdők királynője
/1
#7DFBC539
 Erdők királynője, szép csillag, Szűz Mária,
 Köszönt a magyar föld hódoló imája.
 Harangvirág kelyhe tenéked harangoz,
 Erdő, mező érted madárdaltól hangos.
/2
#480535EF
 A szél Ave Mariát orgonál a fákon,
 Harmatgyöngyöt hullat elébed az alkony.
 Téged köszönt, Úrnő,az erdő temploma,
 Téged dicsér, Anyánk, fiaid \I(lányaid)\i halk dala.

>Jöjj, testvér…
/1
#F4166C48
 Jöjj, testvér, énekelj,
 Hisz összetartozunk,
 A tó fölött s a hegytetőn
 Hadd szálljon víg dalunk!
/2
#14725664
 Izmunkban friss erő,
 Szemünkben fény ragyog,
 Kis lángjaink - mert zord az éj -
 Mind fénylő csillagok.
/3
#16949A68
 Harctér a nagyvilág.
 Az élet: küzdelem;
 Velünk van, él, kit mind dicsér
 Parány s a végtelen.
/4
#8955A716
 Testvér, az Ég velünk,
 Testvér, a Föld velünk:
 S ha két karunkban ily erő,
 Ki győzhet ellenünk?!

>Miénk a nagyvilág
/1
#D33C6A5D
 Miénk a nagyvilág,
 Miénk a fa, virág,
 Magasság és mélység,
 Napsugár és kék ég;
 Mi pedig a tüzesszívű
 Krisztusé vagyunk!
/2
#754D58E6
 Miénk a technika,
 Gép és energia,
 Sebesség és szépség.
 Munka és reménység;
 Mi pedig a tüzesszívű
 Krisztusé vagyunk!
/3
#CD515395
 Miénk az ifjúság,
 Jövő s az új világ,
 A magyar föld tája,
 S rajt' a kereszt fája;
 Mi pedig a tüzesszívű
 Krisztusé vagyunk!

>Láb nem járta…
/1
#7807CE11
 Láb nem járta sziklabércek orma minket vár.
 Napsugár s a tett hevít, nem számít tél vagy nyár.
 Az Úristen lovagjai, cserkészek vagyunk;
 Erdők ölén, fák alatt van árnyas otthonunk.
/2
#846DC8FB
 Nóta, móka jó barátunk együtt jár velünk,
 Tó fölött s a fák alatt, ha száll az énekünk.
 Hegyre hágva, völgybe szállva hét határon át,
 Ifjú lányok víg danája zengi: szép a nyár.

>Hajnal támad
/1
#9C3E497D
 Már vígan zeng ez a csengő dal, üde ünnepi hajnal támad,
 Ifjú dalunkat messze sodorja a szálló szél a világnak,
 Mert fénylő cél ragyog égből ránk, s a szívünkben is új tüzek égnek,
 Hisz Krisztus az Úr, s a miénk lett Ő, a miénk ez az élet.
/2
#7169C3F8
 És száz nyelven tör az égnek fel ez a tűzszavú lelkes nóta,
 Zúgja a bércek rengetege s ez a fáradt vén Európa.
 Mert fénylő cél ragyog égből ránk, s a szívünkben is új tüzek égnek,
 Hisz Krisztus az Úr, s a miénk lett Ő, a miénk ez az élet.
/3
#C1915C0A
 Míg vérben fürdik a bűnös föld, beborulva kísért a holnap,
 Köznapok útján az égre tekintve csak szőjük szűz lobogónkat.
 Mert fénylő cél ragyog égből ránk, s a szívünkben is új tüzek égnek,
 Hisz Krisztus az Úr, s a miénk lett Ő, a miénk ez az élet.

>Pirkad a hajnal
/1
#BF3917FC
 Pirkad a hajnal és zengnek a madarak, hő ima száll fel az égre.
 Ébred a tábor a réten, a hegy alatt, odasüt a napnak fénye.
 Ajkunk víg dala csendül és üde tisztán zakatol a szív;
 Jöjjön a munka, jöjjön a játék, s győzzön a krisztusi béke!
/2
#3C835447
 Rőtszínű alkony a távoli hegyeken, csend van a völgyben, a réten.
 Csillagok tábora felragyog odafenn, kel a telihold az égen.
 Ajkunk víg dala csendül és üde tisztán zakatol a szív;
 Hála-imádság röppen az Úrhoz s alszik a tábor az éjben.

>Sárkányölő
#C5943280
 Sárkányölő Szent György vitéz,
 Éled a föld, bármerre mégy!
 Sárkánnyal küzdő hős ifjúság,
 Számodra példát Szent György lovag ád.

>Jöjj már…
#2926AE03
 Jöjj már, itt a nyár, csodavizű tiszta tó.
 Csöndje mind idevár, nézd, milyen bíztató,
 Tollabda messzire száll, mosolyog a szívünk.

>Bércen túl…
#2F5C4F33
 Bércen túl már nyugszik a nap, esteli szél, hozz jó álmot!
 Békés éjjel jő a földre, hint a tájra harmatot.
 S majd ha az új nap ébred, éled, nótás ajkunk dalba fog.

>Leánycserkész induló
/1
#E2E1B2DA
 Liliomos cserkészzászlónk fenn lobog az égen,
 Jertek alá magyar lányok, jó munkára készen!
 Kéz a kézbe, egyetértve szívünk összedobban,
 A jó Isten néz le reánk minden napsugárban.
/2
#033976D6
 Nékünk dalol minden madár, miénk a természet,
 Sátraink közt pajkos szellő magyar zászlót lenget.
 Vadvirágos erdő-mező víg dalunktól zendül,
 Hegyek ormán, tenger partján magyar népdal csendül.
/3
#70F3DE93
 Tiszta testben tiszta szív és áldozatos lélek,
 Előre hát, magyar lányok, ez a cserkészélet!
 Csillagokkal versenyt ragyog tábortűzünk lángja,
 A világot átöleli szeretetünk lánca.

>Leánycserkész tábortűznyitó
#A75B2754
 Lobbanj lángra tábortűzünk, nyugszik már az erdő.
 Ragyog fent a csillagvilág, álmot hoz a szellő.
 Tűz mellől a lányok dala száll az esti csöndben,
 Tiszta szívünk csengő hangja Urunk elé röppen.

>Láb, mely…
#6A1BAC35
 Láb, mely hangtalanul lépked,
 Szem, mely mindig mindent lát,
 Fül, mely hallja, szél ha rezdül,
 Szív, mely megérzi más baját;
 Üdv néked, cserkésztestvér,
 Erről ismerek rád!

>Szent ez a jelvény
/1
#3C4B0AEA
 Feméliliommal járd a nagyvilágot, társad a jó remény.
 Szent ez a jelvény, áldja sok barátod, általa terjed a fény.
 Cél csak egy lehet, a szép jövő, égjen érte a láng,
 Légy a tettre készen, új erő, hogy egy új haza várjon ránk!
/2
#7A899883
 Állni a vártán hívd az ifjú népet, lásd, mire lenne kész!
 Védd, s mire felnő, várja boldog élet, őrzi a szív meg az ész!
 Cél csak egy lehet, a szép jövő, égjen érte a láng,
 Légy a tettre készen, új erő, hogy egy új haza várjon ránk!
/3
#0F4F3113
 Várjon az erdőn fáknak lombja, zöldje, hívjon az esti szél,
 Menj föl a bércre, nézz a messzi völgybe, lásd, ez a nép hogyan él!
 Cél csak egy lehet, a szép jövő, égjen érte a láng,
 Légy a tettre készen, új erő, hogy egy új haza várjon ránk!

>Az igaz cserkész
#F88A58FD
 Az igaz cserkész tudja jól, közössége nem juhakol,
 De szabad lelkek szövetsége, vállalt rendben termékeny béke.
 Célja nem hazug szolgaság, de emberséges, jobb világ.

>Tartsd fel…
/1
#81305BBD
 Tartsd fel a zászlót, mondd ki a jelszót,
 Mondd ki merészen: "Légy résen!"
 Szent akarattal vívj nemes harcot,
 Jót tenni mindig állj készen!
 Légy résen, légy résen!
 Jót tenni mindig állj készen!
/2
#11C4CC36
 Áldd Teremtődet, tiszteld a földet,
 Otthonod erdő, vízpart, rét.
 Minden erőddel védjed a szépet,
 Krisztus vitéze, bátor légy!
 Bátor légy, bátor légy!
 Krisztus vitéze, bátor légy!
/3
#5F2B071B
 Isten Igéje, néped reménye
 Indít a jóra, ifjúság.
 Nézz fel a hitre, nézz az erényre,
 Mindig a jobbról végy példát!
 Jó munkát! Jó munkát!
 Mindig a jobbról végy példát!
/4
#E1D3DDFD
 Tartsd fel a zászlót, mondd ki a jelszót,
 Mondd ki merészen: "Légy résen!"
 Szent akarattal vívj nemes harcot,
 Jót tenni mindig állj készen!
 Légy résen, légy résen!
 Jót tenni mindig állj készen!

>Bi-Pi-kánon
#BA058694
 Bi-Pi, Bi-Pi, Bi-Pi, Bi-Pi;
 Scouting for me, Scouting for you,
 Scouting for us, Scouting for all!

>Gyújtsál tüzet
#CBF3D144
 Gyújtsál tüzet, hadd égjen tábortüzünk az éjben,
 Erdőzsongás elült már, csillagsátor borul ránk.
 Munkánk nyomán derűsebb s boldog lesz az élet.

>Dán cserkészdal
/1
#41EC8E63
 Te hősi nemzedék, te bátor, jó csapat,
 A léptedet ma védi, óvja Isten és hazád,
 Hogy meg ne tántorodj, hogy győzd le önmagad,
 S a küzdelembe' célhoz érve: vár a pálmaág!
 Ha tenni kell, a szépre, jóra nyisd a két szemed,
 A lelkiismeret szavára másokért teszed...
 Mi hősi nemzedék, mi bátor, jó csapat,
 Mi cserkészek vagyunk!
/2
#63B4842F
 Mi cserkészek vagyunk, ha tartotok velünk,
 A napsugár is énekel, mi véle ébredünk.
 Ha rádhajolt az est, a csillagot keresd,
 A holdvilág fölött szemed csak Alkotódra vesd!
 Az esti szél, a tűz, a víz, a régi jóbarát,
 A bérc, a völgy, a gerle párja zengi víg dalát.
 Mi cserkészek vagyunk, no tartsatok velünk,
 Mi cserkészek vagyunk!

>Útrakelek…
#6683890A
 Útrakelek, elmegyek, várnak rám a kék hegyek, vár reám új, ismeretlen táj.
 Vándorbotot fog kezem, hív a titkos végtelen. Utam végét, jaj, nem ismerem.
 Csak fel, csak fel, csak fel, hosszú útra, messze-messze el!
 Édesanyám, ég veled, búcsúcsókra nyújtsd kezed, messze járva szívem nem feled.

>Cserkészek májusa
#03D23AB5
 Vígan tekint le ránk a kék tavaszi ég,
 Ifjú szívünk dalát a szél röpíti szét.
 Velünk fütyül a madár lombos ágú fán: \I(füttyszó)\i
 Szép május Úrnője áldva légy! Alleluja! Zeng föld s az ég.

>Minden út
/1
#812F3351
 Halld, hogy zúg a rét és hogy zeng az ég, ilyen dal nem volt még,
 Most dalaloni kezdett a föld s az ég, dalolni kezd a mindenség.
 Minden út az erdőn, réten indulás az Úr felé,
 Visz a szellő hűs kezében, visz az élő Úr elé.
/2
#7054C597
 Halld, hogy zúg a rét és hogy zeng az ég, ilyen dal nem volt még,
 Most dalaloni kezdett a föld s az ég, dalolni kezd a mindenség.
 Minden út a messzi rónán, mely a láthatárig ér,
 Odajut egy titkos órán, hol a lélek Hozzá tér.
/3
#C3927F8E
 Halld, hogy zúg a rét és hogy zeng az ég, ilyen dal nem volt még,
 Most dalaloni kezdett a föld s az ég, dalolni kezd a mindenség.
 Mert az élet minden útja oda visz, hol Isten vár.
 Ha bejárod, Ő azt nyújtja, mit a célról álmodtál.

>Cikázó lángok
#86F4B865
 A nap leszállott, cikázó lángok magasba törnek és száll a dal,
 Vidáman, testvér, az élet így szép, köszöntse nótánk az ég Urát.
 Trallala...

>Ne törj le, cserkészpajtás!
/1
#7ACC8BA2
 Rá se ránts, ne törj le, cserkészpajtás, csak dalolva járj, nevess, ha száz gond bánt!
 Félni mindig nem lehet, kívül hordd a szívedet, megér ennyit a világ.
 Ne terheld a hátizsákod, hagyd a holmit, légy szabad!
 Mezítláb is célba érhetsz, sose áruld önmagad!
/2
#BFA6FC21
 Rá se ránts, ne törj le, cserkészpajtás, csak dalolva járj, nevess, ha száz gond bánt!
 Félni mindig nem lehet, kívül hordd a szívedet, megér ennyit a világ.
 Ne izgasson, hogyha másnak négy kerékkel több jutott;
 Talán épp a "Jaguárban" ülnek lelki koldusok.
/3
#9ADB2A67
 Rá se ránts, ne törj le, cserkészpajtás, csak dalolva járj, nevess, ha száz gond bánt!
 Félni mindig nem lehet, kívül hordd a szívedet, megér ennyit a világ.
 Nevess, hogyha aggályosan biztonságra intenek;
 Biztos kézben van a sorsod, bátran járd az életet.
/4
#1C56874B
 Rá se ránts, ne törj le, cserkészpajtás, csak dalolva járj, nevess, ha száz gond bánt!
 Félni mindig nem lehet, kívül hordd a szívedet, megér ennyit a világ.
 Ha majd egyszer mások járják utadat a Nap alatt,
 Ősz-szakállú nagyapóként (\Ivagy:\i Fehér hajú nagyanyóként) fütyüld, mint a madarak:

>Géza fejedelemről
#08FB17DC
 Erat quidem dux Hungariae,
 nomine Geysa: ritu paganismi obvolutus
 coepit attente tractare,
 quomodo orthodoxae fidei semen
 pectore suo delectaretur emittere.

>Szt.István király ünnepére
#600BBFA0
 Adest festum venerandum Sancti regis Stephani,
 multa laude celebrandum omni die saeculi.
 Gaudeant qui sunt heredes: per hunc facti Domini
 et sanctorum coheredes meruerunt fieri.

>Ómagyar Mária-siralom
/1
#4C0C9082
 Volék sirolmtudotlon,
 Sirolmol sepedek,
 Búol oszuk, epedek.
/2
#A3F39C6D
 Választ világomtúl,
 Zsidóu fiodomtúl,
 Ézes ürümemtűl.
/3
#4E5D7241
 Ó, én ézes urodum,
 Egyen így fiodum,
 Síróu anyát teküntsed
 Búabeleül kinyuhhad.
/4
#BD0E2CA6
 Szemem könnyüel árad,
 Én jonhum búol fárad.
 Te vérüd hullotja,
 Én jonhum olélotja.
/5
#7941539B
 Világ világa,
 Virágnak virága.
 Keserűen kínzatul,
 Vosszegekkel veretül.

>Üdvözlégy…
/1
#F6EF0608
 Üdvözlégy, kegyes Szent László király,
 Magyarországnak édes oltalma,
 szent királyok közt drágalátos gyöngy,
 csillagok között fényességes csillag.
/2
#50BA5B51
 Testedben tiszta, lelkedben fényes,
 szívedben bátor, miként oroszlán,
 azért neveztek Bátor Lászlónak,
 mikoron még csak ifjú férfi voltál.
/3
#E10B3256
 Az igazságnak valál bírója,
 nem volt idődben gonoszul tévő,
 a szegényeknek erős oltalma,
 irgalmasságnak jeles követője.
/4
#A5B495F3
 Te dicsekedtél király székedben,
 képed föltették magas kőszálra,
 fénylik mint a nap, salyog mint arany,
 nem elégszik senki terád nézni.
/5
#95CDB9F0
 Dicsérjük, magyarok, Szent László királyt,
 bizony érdemli dicséretünket,
 dicsérik őtet angyalok, mondván:
 Üdvözlégy, kegyes Szent László király!

>Hunyadi Jánosról
#CF950DB0
 Néktek emlékezem, ha meghallgatjátok, jó Hunyadi Jánosról,
 nagy jámborságáról, hív szolgálatjáról, erős viadaljáról,
 az ő idejében két László királyról, Amurátes császárról,
 Nándorfehérvárról, Jankula vajdának utólszor haláláról.

>Cantio Optima
/1
#F2381D70
 Siralmas énnékem tetőled megválnom,
 áldott Magyarország, tőled eltávoznom,
 vajon s mikor lészen jó Budában lakásom?!
/2
#CD60FC64
 Az Felföldet bírják az kevély németek,
 Szerémséget bírják az fene törökök,
 vajon s mikor lészen jó Budában lakásom?
/3
#F104CE02
 Engemet kergetnek az kevély németek,
 engem körülvettek az pogány törökök,
 vajon s mikor lészen jó Budában lakásom?
/4
#1736133B
 Engem eluntattak az magyari urak,
 kiűzték közülük az egy igaz Istent,
 vajon s mikor lészen jó Budában lakásom?
/5
#84203702
 Legyen Isten hozzád, áldott Magyarország,
 mert nincsen tebenned semmi nagy uraság,
 vajon s mikor lészen jó Budában lakásom?
/6
#255BAFDD
 Ez éneket szerzék jó Husztnak várában,
 Bornemissza Péter az ő víg kedvében,
 vajon s mikor lészen jó Budában lakásom?

>Egri históriája
/1
#68B4932A
 Summáját írom Eger várának,
 megszállásának, viadalának,
 szégyenvallását császár hadának,
 nagy vigasságát Ferdinánd királynak.
/2
#404DBE19
 Megírtam bűvön históriáját,
 Eger várának ő nagy romlását,
 mest rövidedön annak summáját,
 megérthetitök ő nagy sok csudáját.
/3
#909A82C3
 Kárt az törökök nagyot vallának,
 az ó-kapura is ostromlának,
 ám szégyenükre eltágulának,
 dobjuk fakadva elbúsulának.
/4
#7D33A051
 Amhát táborát hamar elindítja,
 minden hadával onnat kiszálla,
 negy dicséretöt egriekről szólla,
 Ferdinánd király nagy örömet halla.
/5
#4AC9D33E
 Már meghallátok rövid summáját
 egri szállásnak históriáját,
 azon imádjuk mennyei atyát,
 tőlünk se vonja nagy irgalmasságát.
/6
#67D1275F
 Egert minékünk éltig megtartsa,
 több végházakkal megszabadítsa,
 az pogány kéztől megoltalmazza,
 körösztyén népet igaz hitben tartsa.
/7
#F2B82B79
 Nagy históriáét az ki szerzette,
 ez summáját is azon jegyzötte,
 nevét versfőben megjelentötte,
 ezerhatodfélszázháromban szerzötte.

>Egervár viadaláról
/1
#5961F8E2
 Ti magyarok, már Istent imádjátok,
 És őnéki nagy hálákat adjatok,
 Jelösben Tiszán innét, kik lakoztok,
 Egri vitézeknek sok jót mondjatok!
/2
#E87D6F1E
 Ím egy krónikát mondok, meghalljátok,
 Talán mássát soha nem hallottátok,
 Magyarok végházban sem szolgáltatok,
 Eger várát mint most oltalmazátok.
/3
#CCE055FF
 Nem emberi hatalom ezt művelé;
 Mert csak Úristen hatalmát jelentené.
 E világ hatalmát semmivé tevé,
 Török császár erejét szégyeníté.

>Rákóczi Ferenc imádsága
/1
#FD6C41D1
 Győzhetetlen én kőszálom,
 Védelmezőm és kővárom!
 A keresztfán drága árom,
 Oltalmamat tőled várom.
/2
#F3F9689F
 Sebeidnek nagy voltáért,
 Engedj kedves áldozatért,
 Drága szép, piros véredért,
 Kit kiöntél e világért.
/3
#6E2E3FDF
 Mutass, Jézus, kies földet,
 Lakásomra adj jó helyet,
 Ez életben csendességet,
 Jövendőben üdvösséget.

>Csínom Palkó
/1
#1551B02F
 Csínom Palkó, Csínom Jankó, csontos karabélyom,
 Szép selymes lódingom, dali pár pisztolyom.
 Nosza, rajta, jó katonák, igyunk egészséggel.
 Menjen táncba ki-ki köztünk az ő jegyesével.
/2
#AD33926F
 Ne bánkódjék senki köztünk, menjünk az Alföldre;
 Megrontatik kezünk által az labanc ereje.
 Szabad nekünk, jó katonák, Tisza-Duna közi,
 Labancságnak, mert nincs sehult ottan semmi közi.
/3
#C66D976A
 Darulábú, szarkaorrú, nyomorult ellenség,
 Fut előttünk, retteg tőlünk nyomorult nemzetség.
 Görbe hátú, mert lenyomta füstös muskétálya,
 Elfárasztott, elbágyasztott, dióverő pózna.
/4
#07B3C8D6
 A jó vitéz kurucnak van szabott dolománya,
 Sarkantyús csizmája, futó paripája;
 És a gyalog jó kurucnak van frissen járása,
 Mint a szárnyas Pelágusnak, van sebes futása.

>Kuruc tábori dal
/1
#FDF6F515
 Te vagy a legény, Tyukodi pajtás!
 Nem olyan, mint más, mint Kuczug Balázs.
 Teremjen hát országunkban jó bor, áldomás.
 Nem egy fillér, de két tallér kell ide, pajtás!
/2
#7AF77121
 Szegény legénynek olcsó a vére:
 Két-három fillér egy napra bére;
 Azt sem tudja elkölteni, mégis végtére,
 Két pogány közt egy hazáért omlik ki vére.
/3
#6EDEDCE3
 Bort a kupámba! Embert a gátra!
 Tyukodi pajtás! Induljunk rája!
 Verjük által a labancot a másvilágra,
 Úgy ád Isten békességet szegény Hazánkra.

>Török bársony süvegem
/1
#5227DFF8
 Török bársony süvegem,
 most élem gyöngy életem,
 Balogh Ádám a nevem,
 ha vitéz vagy, jer velem.
/2
#5994DF17
 Fakó lovam a Murza,
 Lajta vizét átússza,
 császárt hogyha megkapom,
 Bécs várába ugratom.
/3
#39994DBB
 Sándor Ferkó sógorom,
 adsza ide a borom,
 harc után ha szomjazom,
 az áldomást megiszom.
/4
#D3079C56
 Zsindelyesi eszterház,
 ég a város, ég a ház,
 nem is egy ház, háromszáz,
 mert a kuruc ott tanyáz.
/5
#9442CFF0
 Vörös bársony süvegem,
 most élem gyöngy életem,
 bokréta van mellette,
 barna babám kötötte.
/6
#0A32E026
 Ha kötötte, jól tette,
 csókot kapott érette,
 kössél, babám, máskor is,
 megcsókollak százszor is.

>Rákóczi búcsúja
/1
#D3A2A86A
 Hallgassátok meg, magyarok, amit beszélek,
 tanácsoljatok, vitézek, mitévő legyek?
 jön a német, dúl-fúl, pusztít, rabol, kerget, mindent éget,
 jaj, már mit tegyek?
/2
#E4C97E1F
 Tanácsoljuk felségednek, meg se is vesse,
 aranyait, ezüstjeit rúdba veresse,
 rakassa fel, vitesse el, hordassa el és menjen el
 merre szeme lát!
/3
#56724AFB
 Sok pénzembe, költségembe került váraim,
 ahhoz készült pompásaim, szép palotáim,
 itt hagylak már, pataki vár, nem szánlak már, munkácsi vár,
 Isten megáldjon!
/4
#2131242E
 Országomból, az hazámból már ki kell menni,
 mordságimért, hibáimért, engedj Thököly!
 Megölellek, megcsókollak, amíg élek, mindig szánlak,
 kedves Bercsényi!
/5
#39467922
 Megengedjen a magyarság, csak azt kiáltom:
 édes hazám, mire jutál, csak azt sajnálom,
 idegen lesz a vezéred, de tenéked az nem véred,
 jaj, ki nem szánna!

>Zöld erdő harmatát
/1
#53FCFFEB
 Zöld erdő harmatát,
 Piros csizmám nyomát
 Hóval lepi be a tél.
/2
#753A3F6C
 A magas hegyeken,
 Kietlen bérceken
 Fújdogál a hideg szél.
/3
#DD0D5998
 Jobb hát a darvakkal
 Vagy más maradakkal
 Elbujdosnom messzire.
/4
#33BFDF2F
 Ha minden elmarad,
 Isten el nem marad,
 Reá bízom magamat.

>Nem lesz…
#AB9ED018
 Nem lesz mindenkor így, szívedben bús ne légy;
 El ne hagyd magadat, várjad jobb sorsodat;
 A szerencse változik, Naptúl felleg távozik,
 Szívedben bús ne légy, nem lesz mindenkor így.

>Katonaének
/1
#CE6A8EB7
 A szép fényes katonának gyöngyarany élete,
 Csillog, villog mindenfelől jó vitéz fegyvere.
 Szép élet, víg élet, Soha jobb, soha jobb nem lehet.
 Hopp, hát jöjjön katonának, ilyet ki szeret.
/2
#4B4B6BCC
 Szikrát üt a paripája nagyvároson nézik,
 Hogy jó lovas és vitéz is, mindenütt dicsérik.
 Szép élet, víg élet, Soha jobb, soha jobb nem lehet.
 Hopp, hát jöjjön katonának, ilyet ki szeret.
/3
#EBA9C13E
 Közemberből káplár, hadnagy, kapitány is lészen,
 Óbesterség, generálság várja ottan készen.
 Szép élet, víg élet, Soha jobb, soha jobb nem lehet.
 Hopp, hát jöjjön katonának, ilyet ki szeret.
/4
#71E4B4CC
 Menjünk azért seregesen, tartsuk meg hazánkat,
 Vérrel, bérrel oltalmazzuk szent, szent koronánkat.
 Szép élet, víg élet, Soha jobb, soha jobb nem lehet.
 Hopp, hát jöjjön katonának, ilyet ki szeret.

>Gábor Áron
/1
#57FB52C8
 Gábor Áron rézágyúja fel van virágozva.
 Indulnak már a tüzérek messze a határba.
 Nehéz a rézágyú: fölszántja a hegyet, völgyet.
 Édes rózsám, Isten véled, el kell válnom tőled.
/2
#42F9DA4F
 Véres a föld, magyar tüzér vére folyik rajta,
 Csak még egyszer gondolj vissza szép magyar hazádra.
 Anyám, te jó lélek! Találkozom-e még véled?
 Holnapután messze földre, hosszú útra kélek.

>Kossuth Lajos táborában
/1
#8A94F7E8
 Kossuth Lajos táborában két szál majoránna,
 egy szép barna magyar huszár sej, lovát karélyozza.
/2
#DF93CEEF
 Ne karélyozz, magyar huszár, mert leesel róla,
 nincsen itt a te édesanyád, sej, aki megsiratna.
/3
#2245E730
 Ne sirasson engem senki, jól vagyok tanítva,
 sem lépésbe', de sem vágtába' sej, le nem esek róla.
/4
#5A4A88AA
 Mert a huszár a nyeregbe bele van teremtve,
 mint a rozmaring a jó földbe, sej, belegyökerezve.

>Klapka-induló
#2EEFFEBB
 Fel, fel, vitézek, a csatára,
 A szent szabadság oltalmára,
 Mennydörög az ágyú, csáttog a kard,
 Ez lelkesíti csatára a magyart.

>Zúg az erdő
/1
#EAC7F69F
 Zúg az erdő, zeng a mező, harsog a trombita,
 Keljetek föl vitézeim, mer' eljött a muszka!
/2
#8CF75F20
 Muszka ide, muszka oda, vitéz gyerök vagyok,
 Angyalomér', szép hazámér' möghalni kész vagyok.
/3
#7BBEA385
 Megy honvédnak, megy huszárnak, aki arra való,
 Ez a nyalka magyar verbung de kedvemre való.
/4
#225824C2
 Félre innen német fattyú, nem vagy idevaló,
 Csak a szegény magyar legény katonának való.

>A nagy bécsi kaszárnyára
/1
#506B7025
 A nagy bécsi kaszárnyára rászállott egy gólya,
 Vizet hozott a szájába' regruták számára.
 Mosdjatok regruták, mert porosak vagytok,
 Azt csak a jó Isten tudja, mikor szabadultok.
/2
#DFEE0A4F
 Gyere pajtás abfürolni, fegyver lesz az első,
 Hull a könnye a bundásnak, mint a záporeső.
 Ne sírj bundás, ne sírj, kitelik az idő,
 Nékem is volt, de már elmúlt a három esztendő.
/3
#B0275010
 Fiumei kikötőben megállt egy gőzhajó,
 Közepébe', négy sarkába' nemzetiszín zászló,
 Fújja a szél, fújja, hazafelé fújja,
 Szegény, rongyos magyar bakák jönnek szabadságra.

>Kimegyek…
/1
#A0D4F2D1
 Kimegyek a doberdói harctérre,
 Feltekintek a csillagos nagy égre.
 Csillagos ég, merre van a magyar hazám,
 Merre sirat engem az édesanyám?
/2
#5AC8F429
 Jaj, Istenem, hol fogok én meghalni,
 Hol fog az én piros vérem elfolyni?
 Olaszország közepében lesz a sírom:
 Édesanyám, arra kérem, ne sírjon.

>Nemzeti hitvallás
#41FA855D
 Hiszek egy Istenben, hiszek egy hazában,
 Hiszek egy isteni örök igazságban.
 Hiszek, hiszek Magyarország feltámadásában,
 Magyarország, Magyarország feltámadásában.

>Székely dal
/1
#F07FE7B7
 Én Istenem, jó Istenem,
 Oltalmazó segedelmem.
 Vándorlásban reménységem,
 Ínségemben lágy kenyerem.
/2
#B47FAC04
 Vándorfecske sebes szárnyát,
 Vándorlegény vándorbotját,
 Vándor székely reménységét,
 Jézus, áldd meg Erdély földjét!
/3
#1260B922
 Vándorfecske hazatalál,
 Édesanyja fészkére száll;
 Hazamegyünk, megsegít a
 Csíksomlyói Szűz Mária.

>Székely himnusz
#070500F0
 Ki tudja, merre, merre visz a végzet, göröngyös úton, sötét éjjelen.
 Segítsd még egyszer győzelemre néped, Csaba király a csillagösvényen.
 Maroknyi székely porlik, mint a szikla, népek harcától zajló tengeren.
 Fejünk az ár ezerszer elborítja; ne hagyd el Erdélyt, Erdélyt, Istenem!

>Hunyadi, bízzál!
/1
#7DF0B806
 Hunyadi, bízzál, a nép melletted áll,
 Elszántan védjük e meggyötört hazát.
 Zászlódat követjük tűzön-vízen át,
 Elűztük végre az ozmánok hadát.
/2
#15B5A15E
 Vezesd a népet, csak őbenne higgy,
 Szívére tűzd a kereszted jelét.
 S a pogányok ellen győzelemre vidd
 A puszta karddal Jézus szent nevét!

>Napfényes utakon
/1
#128F25DA
 Szívünk mélye új lánggal ég, új dalra hív most az élet!
 Tűző fényben megtört a jég, minden, mi szunnyadt, fölébred.
 Víg szél ágat lenget a légben, tél volt, most a Nap tüze égjen!
 Víg szél ágat lenget a légben, tél volt, most a Nap tüze égjen!
/2
#639D788F
 Mondd csak, pajtás, indulsz-e már erdőbe, völgyekre, bércre?!
 Hogyha új út szépsége vár, mondd, tudsz-e fáradni érte?!
 Mért félsz, csak menj, csak indulj, szálljon az ének, nézz szét: híre sincsen a télnek!
 Mért félsz, csak menj, csak indulj, szálljon az ének, nézz szét: híre sincsen a télnek!

>Ha én…
/1
#699A43C0
 Ha én rózsa volnék, nem csak egyszer nyílnék;
 Minden évben négyszer virágba borulnék.
 Nyílnék a fiúnak, nyílnék én a lánynak,
 Az igaz szerelemnek, és az elmúlásnak.
/2
#92326130
 Ha én kapu volnék, mindig nyitva állnék,
 Akárhonnan jönnek, bárkit beengednék.
 Nem kérdezném tőle: Hát téged ki küldött?
 Akkor lennék boldog, ha mindenki eljött.
/3
#4F774918
 Ha én ablak volnék, akkora nagy lennék,
 Hogy az egész világ láthatóvá váljék.
 Megértő szemekkel átnéznének rajtam,
 Akkor lennék boldog, ha mindent megmutattam.
/4
#7DA7718C
 Ha én utca volnék, mindig tiszta lennék,
 Minden áldott este fényben megfürödnék.
 És ha engem egyszer lánckerék taposna,
 Alattam a föld is sírva beomolna.
/5
#C3CCB0BF
 Ha én zászló volnék, sohasem lobognék;
 Mindenféle szélnek haragosa volnék.
 Akkor lennék boldog, ha kifeszítenének;
 S nem lennék játéka mindenféle szélnek.

>Októberi induló
/1
#36268EB3
 Szent István keze, jó oltalmunk voltál ezer bajban.
 Halld mg most is e hős nép jaját, ebben a pogány zivatarban!
 Hajnal fénye vonja be az eget, elveszünk, ha egyedül állunk.
 Ám, ha szent kezed segít nékünk, holnapra új világ lesz nálunk.
/2
#213F541D
 Sok vért, sok üde friss életet kívánt a sors tőlünk,
 És tengersok a kínunk, bajunk, mégse hal a hit ki belőlünk,
 Bár az ellen ördögi ereje ránk tört, hogy eltiporja néped;
 Szent István keze, hogyha segítsz, nemzeted új életre ébred.

>István, a király
#9314279D
 Felkelt a napunk, István a mi urunk,
 Árad a kegyelem fénye ránk.
 Hálás a szívünk, zengjen az örömünk,
 Szép Magyarország, édes hazánk.

>Erdők, völgyek
/1
#7E68C419
 Erdők, völgyek, szűk ligetek,
 Sokat bujdostam bennetek.
 Bujdostam én a vadakkal,
 Sírtam a kis madarakkal.
/2
#FF7F61DC
 Édesanyám sok szép szava,
 Kire nem hajlottam soha,
 Hajlanék én, de már késő,
 Hullik fölöttem az eső.
/3
#740312DE
 Édesanyám édes teje,
 Keserű a más kenyere.
 Keserű is, savanyú is,
 Néha-néha panaszos is.

>Ködözik a Mátra
/1
#06ED3D9C
 Ködözik a Mátra, eső akar lenni, eső akar lenni,
 Morog a manga nyáj, haza akar menni, haza akar menni.
/2
#A99EA0DE
 Ne morogj, manga nyáj, nem megyünk még haza, nem megyünk még haza.
 Messze van Szent György nap, akkor megyünk haza, akkor megyünk haza.

>Piros alma
/1
#B423D169
 Piros alma mosolyog a dombtetőn,
 Sárga kendős kislány sétál a mezőn.
 Szép a mező, megszépül a virágtól,
 Vagy attól a sárga kendős kislánytól.
/2
#A197BE16
 Ha a Tisza tinta volna, jó volna,
 Minden legény írnya tudna, jó volna.
 Azt az egyet mégsem tudná leírnya,
 Kit szerettem nagy Menyhébe' legjobban.

>Erdő, erdő, de magas
/1
#C044FC36
 Erdő, erdő, de magas a teteje.
 Jaj, de régen lehullott a levele,
 Jaj, de régen lehullott a levele,
 Árva madár párját keresi benne.
/2
#B093045C
 Búza közé szállt a dalos pacsirta,
 Mert odafönt a szemeit kisírta,
 Búzavirág, búzakalász árnyába'
 Rágondol a régi első párjára.

>Hegyek, völgyek
/1
#102D69C7
 Hegyek, völgyek között állok,
 Szívemet öli a bánat.
 Hegyek, völgyek, álljatok meg,
 Hogy panaszkodjam meg nektek.
/2
#FBDC295F
 Engem anyám megátkozott,
 Mikor a világra hozott.
 Azt az átkot mondta reám:
 Ország-világ legyen hazám!
/3
#571770C5
 Ott se legyen megállásom,
 Csipkebokor a szállásom.
 Ott se legyen megállásom,
 Csipkebokor a szállásom.

>Azhol én elmegyek
/1
#4BDEA542
 Azhol én elmegyek, még az fák is sírnak,
 Gyenge ágairól levelek lehullnak.
/2
#652C69BA
 Hulljatok levelek, rejtsetek el engem,
 Mert az én édesem sírva keres engem.
/3
#FE9C6AF5
 Ne keress, édesem, mert el vagyok rejtve,
 Bánattal levélbe el vagyok temetve.

>A tavaszi…
/1
#E6412A0A
 A tavaszi szép időnek
 Lám, hogy mindenek örülnek,
 Erdők, mezők megzöldülnek,
 A madarak zengedeznek.
/2
#215BD7F3
 Kisütött a nap sugára
 Az én rózsám ablakára.
 Tündöklik gyémánt orcája,
 Ragyog szép csillag módjára.
/3
#BA5D9E29
 Repülj, fecském, ablakára,
 Kérjed, nyissa meg szavadra.
 Mondd, ezüstös lapot vevék,
 Rá arannyal írom nevét.

>A csitári…
/1
#F3F05EE3
 A csitári hegyek alatt régen leesett a hó.
 Azt hallottam, kisangyalom, véled esett el a ló.
 Kitörted a kezedet, mivel ölelsz engemet,
 Így hát, kedves kisangyalom, nem lehetek a tied.
/2
#A095700D
 Amoda le van egy erdő, jaj, de nagyon messze van,
 Közepibe, közepibe két rozmaring bokor van.
 Egyik hajlik vállamra, másik a babáméra,
 Így hát, kedves kisangyalom, tied leszek valaha.

>Fúj, süvölt…
/1
#C3796F65
 Fúj, süvölt a Mátra szele,
 Üngöm, gatyám lobog bele:
 Kalapom is elkapta már,
 Tiszába vitte a tatár.
/2
#953E27A2
 Istenadta falu nyája,
 Ez is csak utamat állja;
 Talán itt van hét vármegye
 Juha, borja és tehene.
/3
#3602DAA3
 Habos sárga csikóm nyaka,
 Ujjnyi vastag a por rajta;
 Meg ne utálj, kedves rózsám,
 Hogy oly sötét poros orcám.
/4
#4D35CC94
 Kalapom a Tiszán úszkál,
 Subám zálog a bírónál:
 De a szívem itten dobog,
 Forró lángja feléd lobog.

>Zöld erdőben
/1
#06DBEADE
 Zöld erdőben, zöld mezőben sétálott egy páva,
 Kék a lába, zöld a szárnya, aranyos a tolla.
 Hítt engem is útitársnak, nem mentem el véle,
 Vásárhelyen nincsen olyan leány, se, aki nekem kéne.
/2
#893C6AFE
 Anyám, anyám, édesanyám, elmék az erdőbe,
 Fölkeresem azt a pávát, megkérdezem tőle:
 Nem láttad-e a kedvesem abban az erdőben,
 Kiért fáj a, fáj a gyenge szívem, sej, meg is hasad érte.

>Tizenhárom fodor…
/1
#E3AACFC4
 Tizenhárom fodor van a szoknyámon.
 Azt gondoltam, férjhez megyek a nyáron.
 De már látom, semmi se lesz belőle.
 Tizenkettőt vágatok le belőle.
/2
#1976CE92
 Tizenhárom veréb ugrál a jégen,
 Azt gondoltam, hogy elvesznek a télen,
 De már látom, semmi se lesz belőle,
 Tizenkettőt elhessentek belőle.

>Tizenhárom ezüst…
#21BA3CA1
 Tizenhárom ezüst pityke fityeg a mentémen,
 A legelső hajdú vagyok Hajdú vármegyében.
 Vikszos bajszom úgy áll, mint az öreg bika szarva,
 Nincsen nálam különb legény Hajdú meg Birharba'.

>Jaj, de szépen…
/1
#E7E39EA7
 Jaj, de szépen muzsikálnak
 Zöld erdőben a huszárnak!
 Fekete gyűrűfalevél,
 Jaj, de szép a huszárlegény!
/2
#AF7FF5D8
 Jaj, de szépen harangoznak
 Az én kedves galambombnak.
 Most viszik az esküvőre,
 El se búcsúzhattam tőle.

>Huszárgyerek
/1
#95355310
 Huszárgyerek, huszárgyerek szereti a táncot,
 Az oldalán, az oldalán csörgeti a kardot.
 Ha csörgeti, hadd csörgesse, pengjen sarkantyúja:
 Kossuth Lajos verbunkja a muzsikáltatója.
/2
#F63C2049
 Falu végén, falu végén szépen muzsikálnak,
 Oda hívnak engemet is magyar katonának,
 Be is állok a verbunkba, ha már verbuválnak,
 Elmegyek a pajtásimmal jó lovas huszárnak.
/3
#328EF203
 Szép a huszár, szép a huszár, felül a lovára,
 Aranymente a hátára, kard az oldalára,
 Virágcsokor a csákóján, úgy megy a csatába.
 Ne sírj, rózsám, megtérek még a szabad hazába!

>Kukorica
/1
#2312E57A
 Kukorica, kukorica, pattogatott kukorica.
 A mennyasszony pattogtatta, hej, vőlegénye ropogtatta.
/2
#F9FC5F24
 Megfogtuk már a galambot, verjük neki a dorongot.
 Igen is kedvünkre való, méltó érte a foglaló.

>Elmegyek
/1
#01DFCC2D
 Elmegyek, elmegyek, el is van vágyásom,
 Ebbe' rongyos kis tanyába' nincsen maradásom.
/2
#4AA9E2BC
 Ha elmégy, ha elmégy, csak hozzám igaz légy,
 Igaz szereteted, babám, hamisra ne fordítsd!

>Hej, rozmaring
/1
#24CA75E0
 Hej, rozmaring, rozmaring, leszakadt rólam az ing,
 Van már nékem kedvesem, ki megvarrja az ingem.
/2
#F1851AEF
 Három icce kendermag, jaj, de büszke legény vagy,
 Mit ér a büszkeséged, nem szép a feleséged!
/3
#EBB940A7
 Három bokor saláta, három kislány kapálta,
 Nem kell nékem saláta, csak az, aki kapálta.
/4
#42CE997B
 Hej, kis katlan, kis katlan, benne a sok mosatlan,
 Mosd el kislány az edényt, úgy öleld meg a legényt!
/5
#187F7E5D
 Hej, kendermag, kendermag, barna kislány, kié vagy?
 Nem vagyok én senkié, csak a kedves babámé.
/6
#0E864A9B
 Hej, kikirics, kikirics, már énnekem ne viríts,
 Virítottál eleget, mégsem lettem a tied.
/7
#9BE88EAB
 Kölcse, Csepel, Teleki, barna kislány gyere ki,
 Nem mehetek, angyalom, mert a hajam most fonom.
/8
#7AD61D0B
 Hej, piros bor, piros bor, mingyár' ránk kerül a sor,
 Egy icce bor nem árt meg, régi babám, ölelj meg!
/9
#CB007CB1
 Ha ittatok, ettetek, innen bizony menjetek,
 Félig van már a hordó, nem éri el a lopó.

>A Vargáék ablakja
/1
#C93C0EEE
 A Vargáék ablakja
 Rózsából van kirakva, kirakva,
 Kirakva, kirakva, de kirakva.
/2
#495593A7
 Azért van az kirakva:
 Garzó Péter jár ide, jár oda,
 Jár ide, jár oda, de jár oda.
/3
#F6DBF608
 Hej, galambom, gyere ki,
 Vár már Péter ideki, ideki,
 Ideki, ideki, de ideki.
/4
#E0B4B1D7
 Tudnám én azt, ha várna,
 Mert a szívem dibegne, dobogna,
 Dibegne, dobogna, de dobogna.

>A jó lovas…
/1
#925F3315
 A jó lovas katonának de jól vagyon dolga,
 Eszik, iszik a sátorban, semmire sincs gondja.
 Hej, élet, be gyöngy élet, ennél szebb sem lehet,
 Csak az jöjjön katonának, aki ilyet szeret.
/2
#A45B1187
 Paripáját megforgatja, úgy megyen dolgára,
 Csillog-villog a mezőben virágszál módjára.
 Hej, élet, be gyöngy élet, ennél szebb sem lehet,
 Csak az jöjjön katonának, aki ilyet szeret.
/3
#FCE31AC8
 Isten hozzád, apám, anyám és édes szerelmem,
 Húgom, bátyám, sógor, komám, avagy jertek vélem!
 Hej, élet, be gyöngy élet, ennél szebb sem lehet,
 Csak az jöjjön katonának, aki ilyet szeret.

>A gyulai…
/1
#513ED433
 A gyulai kert alatt, kert alatt
 Barna legény rozmaringot arat.
 Én vagyok a rozmaring kévekötője,
 Barna legény igaz szeretője.
/2
#7F776435
 Benedeki kert alatt, kert alatt
 Rézsarkantyúm ott maradt, ott maradt.
 Eredj, babám, keresd meg, sej-haj, keresd meg,
 Ha megleled, pengesd meg, pengesd meg!

>Megrakják a tüzet
/1
#3BCD7067
 Megrakják a tüzet, mégis elaluszik,
 Nincs az a szerelem, ami el nem múlik.
/2
#351D57FC
 Rakd meg, babám, rakd meg lobogó tüzedet,
 Hadd melegítsem meg gyönge kezeimet.
/3
#DD79930F
 Gyönge lábam fázik, köpönyegem ázik,
 Piros pej paripám kert alatt bánkódik.
/4
#84862451
 Meggyújtják a tüzet, jaj, ha elaluszik!
 Jézus szeme rajtam, a lángomban bízik.
/5
#C761B4E1
 Rakd meg, Uram, rakd meg lángoló tüzedet,
 Hadd melegítse meg a dermedt lelkeket.

>Télen-nyáron
#F56ABF18
 Télen-nyáron jókedvem van, az ördög fog búsulni.
 Nincs is jobb, mint tábortűznél a hajdútáncot járni.
 Leszakadt, haj, a csizmám szára, jókedvem van, az ördög bánja,
 majd hoz újat a Mózsi.

>Elindultam…
/1
#312D6A6E
 Elindultam szép hazámbúl,
 Híres kis Magyarországbúl,
 Visszanéztem félutambúl,
 Szememből a könny kicsordult.
/2
#8D20DA5A
 Bú ebédem, bú vacsorám,
 Boldogtalan minden órám.
 Nézem a csillagos eget,
 Sírok alatta eleget.
/3
#CA474BD7
 Én Istenem, rendelj szállást,
 Mert meguntam a bujdosást,
 Idegen földön a lakást,
 Éjjel-nappal a sok sírást.

>Este van már
/1
#5EF60140
 Este van már, csillag van az égen,
 Varga Julcsa mezítláb a réten:
 Sajnálja a cipőjét felhúzni:
 Garzó Péter nem vesz többet néki.
/2
#7C570834
 Garzó Péter elment katonának,
 Acélfegyvert csináltat magának;
 Acélfegyvert, rózsafa a nyele,
 Rá van írva Varga Julcsa neve.

>Hej, halászok
/1
#EE8FC527
 Hej, halászok, halászok,
 Merre mén a hajótok?
 Törökkanizsa felé,
 Viszi a víz lefelé.
/2
#86E4BCB7
 Hej, halászok, halászok,
 Mit fogott a hálótok?
 Nem fogott az egyebet,
 Vörös szárnyú keszeget.
/3
#2D9C3F13
 Hát a keszeg mit eszik,
 Ha a bárkába teszik?
 Nem eszik az egyebet,
 Petrezselyemgyökeret.

>Ablakomba
/1
#53D5CB9A
 Ablakomba, ablakomba besütött a holdvilág,
 Aki kettőt, hármat szeret, sosincs arra jó világ.
 Lám, én csak egyet szeretek, mégis de sokat szenvedek,
 Az az álnok béreslegény csalta meg a szívemet.
/2
#045157EF
 Szeged felől, Szeged felől jön egy fekete felhő,
 Szaladj kislány, szaladj kislány, mert megver a nagy eső.
 Nem szaladok olyan nagyon, fáj a szívem, sajog nagyon,
 Most tudom már, miért sajog: elhagyott a galambom.

>Nem jó csillag
#309879B4
 Nem jó csillag lett volna énbelőlem,
 Éjfél előtt nem tűnnék fel az égen.
 Éjféltájban körüljárnám az eget,
 Akkor tudnám meg, hogy a babám kit szeret.

>Lement a nap
/1
#B8112D35
 Lement a nap a maga járásán,
 Sárgarigó szól a Tisza partján.
 Sárgarigó meg a fülemile,
 Szép a rózsám, hogy váljak meg tőle.
/2
#87E0C54F
 Ha meghalok, temetőbe visznek,
 A síromra fakeresztet tesznek.
 Jöjj ki hozzám holdvilágos este,
 Úgy borulj le a sírkeresztemre.

>Megyen már…
/1
#E2059E03
 Megyen már a hajnalcsillag lefelé,
 Az én kedves galambom most megyen hazafelé,
 Lábán van a csizmája, lakkosszárú kis csizma,
 Rásütött a hajnalcsillag sugára.
/2
#A2ECE079
 Amerre mégy, édes rózsám, kívánom,
 Hogy előtted a rét is tiszta rózsává váljon.
 A zöld fű is előtted édes almát teremjen,
 A te szíved holtig el ne felejtsen.

>Este van már
/1
#CF714E9F
 Este van már, késő este,
 Pásztortüzek égnek messze.
 Messze, messze, más határon,
 Az alföldi rónaságon.
/2
#BD7DD05D
 A faluban minden csendes,
 Még az éjmadár sem repdes,
 Nyugodalom lakik benne,
 Mintha temetőkert lenne.

>Szól a húros madár
/1
#00274EDE
 Szól a húros madár,
 Talán megvirrad már,
 Fordulj felém, babám,
 Magad maradsz mindjár'.
/2
#19D758C9
 Lám, megmondtam, rózsám,
 Ne szeress engemet,
 Mert Somogy vármegye
 Hajszoltat engemet.
/3
#2FFFE7F1
 Az sem hajt egyébért,
 csak egy pár csikóért,
 Ahhoz tartozandó
 kocsiért, szerszámért.

>Szólal a kincsem
/1
#7A726186
 Szólal a kincsem, messzire szólal,
 Jajgat a Saimai partszélen.
 Csónak a parton egy sincsen,
 Nem evezhet által az én kincsem.
/2
#203EAA36
 Kukkuu, kukkuu, kaukana kukkuu,
 Saimann rannalla ruikuttaa.
 Ei ole ruuhta rannalla,
 Joka minun kultani kannattaa.

>Vejnemöjnen muzsikál
/1
#D61CD923
 Jászott Vejnemöjnen ujja,
 Harsogott a hárfa húrja.
 Hegy, völgy rengett, szikla zengett,
 Mind a szirtek mennydörögtek.
/2
#7C06B917
 Még a fák is vigadoztak,
 Mezőn tuskók táncot roptak.
 Még a fák is vigadoztak,
 Mezőn tuskók táncot roptak.
/3
#75B2DD12
 Légben szárnyon kik repülnek,
 Lábok ujjára leülnek,
 Hallgatni a szép muzsikát,
 A hárfának zengő szavát.
/4
#976D1881
 Vízi hal, temérdek fajta
 Kitelepszik mind a partra.
 Föld alól a férgek, nyűvek
 Mind a rög fölébe gyűlnek.
/5
#2150A7A2
 Ott forognak, ott figyelnek,
 Hogy az édes nótát hallják,
 Ős örömnek hárfahangját,
 Vejnemöjnen vígadalmát.

>Sűrű a berek
/1
#15D73204
 Sűrű a berek, dong benne a szél,
 Asszony lettem, nem akartam,
 Tatár uraság szolgája vagyok,
 Cseremisz menyecske nem lehettem.
/2
#2D2446E9
 Tatár uraság, méz a szava is,
 Mézes-mázos ő maga is,
 Cipőt, sok ruhát vesz ő nekem,
 Bárcsak sohasem vehetne már!
/3
#C0F7A7CC
 Gyöngy és aranylánc semmit sem tehet!
 Aki hazavágy, sohasem mehet,
 Fonnyad, mint a mák őszidőn,
 Szíve fáj, haza száll mindegyre csak.

>Fut a nyúl
#44F69496
 Fut a nyúl, fut a nyúl, a kutya utána,
 Beszökik, beszalad sűrű vadonába.
 Ne szaladj, te kutya, ügyes az a kis nyúl,
 Kikerül, hazaér a hegyeken is túl.

>Édes almát…
/1
#5A42D57F
 Édes almát hoztam én,
 gömbölyű, tűzszínű, mint a fény.
 Enné hét falu, kapd el kisfiú,
 üljünk fa tövibe kettecskén.
/2
#AB04432F
 Mérges pulykát hoztam én,
 szárnyai, tollai, mint a szén.
 Enné hét falu, fogd el, kisfiú,
 üljünk fa tövibe kettecskén.

>Jöjj ide, pajtás
/1
#C023D79D
 Jöjj ide, pajtás, lekaszáljuk a füvet,
 a fiatal, üde füvet egykettő.
 Minekünk kenyerünk, hogy a rögön aratunk,
 sok a dolog, letelik az esztendő.
/2
#C7889F9D
 Fogjuk a sarlót, ide-oda aratunk,
 sose lehet a gyökeret elvágnunk,
 ha kikél, megin' él a fű meg a falevél,
 a gabona közepibe sétálunk.

>Gyere hát
/1
#AD6B0B2F
 Gyere hát, a danát sose bánd, ajjaja,
 ideát de lobog a kukoricaszár.
 Gyere hát, ez a dana kukorica kusza haja,
 hegyen át, ugye koma, messzire száll.
/2
#D2607CAE
 Kel a tánc, megy a tánc, fut a tánc, ajjaja,
 aki lejt ide-oda, kutyabaja, ládd!
 Sose bánd, ha ropog a zenebona kusza zaja,
 hegyen át, ugye Kati, messzire szállt!

>Lassú folyóvíznek
#D9535670
 Lassú folyóvíznek alacsony partja
 zöld fűvel ékes, virágtól tarka.

>Kaukázusi szűrmellény
#C904CC7D
 Kaukázusi szűrmellény, Dzsingisz Kán, gyere, ülj mellém!
 Egy icce bor nem páros, kettőt hozzon a csapláros!
 Sárga dolmány, selyemsujtás, fújjad a nótát, pajtás!
 Dzsinnajja, dzsin, dzsin, fújjad a nótát, pajtás!

>Tölgyes bucka…
/1
#B93786A7
 Tölgyes bucka tövibe,
 Gyere, pajtás, izibe.
 Tölgyes buckán iszalag,
 Táncolj, amíg legény vagy.
/2
#324E8484
 Három kecske, meg egy bak,
 Szökj fel, pajtás, szökj fel csak!
 Szöknék biz' én jó nagyot,
 De a lábam megbotlott.
/3
#DFA4CECA
 Hopsza pajtás, pattanj fel!
 A kecskét ki hajtja el?
 Én bizony elhajtanám,
 Csak ne lesne farkas rám.

>Szép nyári reggelen
#3CDEB800
 Szép nyári reggelen kertemben csendesen bujkál a szél.
 Bokrok közt jár a fény, halk szellő táncra kél, rólad mesél.
 Csak érted, csak érted fáj a szívem.
 Kék nyári ég alól hozzád száll kis dalom, szép kedvesem.

>Zöld mezőnek közepében
#A6A9B790
 Zöld mezőnek közepében van egy öreg nyír,
 Az én kedves kisangyalom, jaj, de nagyon sír;
 Kökény szeme csupa könny, kezében egy virágszál,
 Rozmarint tűz csákómhoz a legszebb virágszál.

>Zöld levél
/1
#F85D2478
 Zöld levél, zöld levél, illatot hoz már a szél.
 Illatot hoz már a szél, traj-laj-laj-laj lalala.
/2
#55BEEE08
 Kék virág, kék virág, bokrok alján ibolyák.
 Bokrok alján ibolyák, traj-laj-laj-laj lalala.

>Balalajka
/1
#C6E6A457
 Áll egy ifjú nyírfa a réten,
 virágfürtös nyírfácska a réten,
 Dúli, dúli a réten,
 dúli, dúli a réten.
/2
#3BDEAE9F
 Nyírfa ága, hej, kifaragva,
 abból lesz a jó balalajka.
 Dúli, dúl balalajka,
 dúli, dúl balalajka.

>Aranyosom
/1
#5F58239F
 Aranyosom, hogy vegyelek feleségül én,
 Hisz apádnak házeleje patyolatfehér?
 Sejehaj, kisvirágszál, patyolatfehér!
/2
#94DB6F27
 Hogy repüljek én veled, hisz nem vagyok madár?
 Nincsen szárnyam, kék egekbe nem vihetlek ám!
 Sejehaj, kis virágszál, nem vihetlek ám!
/3
#35FDAD1B
 Mint a rétnek jó illatát, elfújnálak én.
 De szellőcske nem lehetek, nem lehetek én!
 Sejehaj, kis virágszál, nem lehetek én!

>Csalogány
#0D3D8DB3
 Erdő mélyén esti csendben
 Hallod a csalogány édes dalát? Szívedbe zengi bánatát.
 Hallod a csalogány édes dalát? Szívedbe zengi bánatát.

>A nap nyugodni tér
/1
#E5907EE2
 A nap nyugodni tér.
 Leszáll a csendes éj,
 Már búcsút mondunk.
 A tölgy, a bérc alatt,
 Oly búsan bólogat,
 Mert el kell válnunk.
/2
#8AFE5625
 Míg jő a pirkadat,
 És újra kél a nap,
 Én ébren várlak.
 Ha zeng majd száz madár,
 S a selymes pille száll,
 Hát újra látlak!

>Már nyugosznak…
/1
#DFFEC796
 Már nyugosznak a völgyek, az erdők s minden földek, már alszik a világ.
 Elszunnyad szívünk gondja, egy titkos kéz kibontja az éjnek csillagsátorát.
/2
#B0261EF3
 Ha jő az alkonyóra, térj testem nyugovóra, a munka véget ért.
 De te, ó lelkem, este a földi terhet vesd le, s légy könnyű, mint a lenge szél.

>Vén erdők alól
#BE41D93A
 Vén erdők alól mi szép, ha szól a kürt,
 ím halld szavát, a kört, ím halld szavát!
 A visszhang kél a bérc alatt
 és hallkan zeng tovább, és halkan zeng tovább.

>Kinn a holdsütésen
#61B17C58
 Kinn a holdsütésen, barátom, Pierrot
 Kölcsönözd a tollad, nyelvemen egy szó.
 Elaludt a gyertyám, nincsen több tüzem,
 Nyisd ki már az ajtót, régen döngetem…

>A jóbarátság
/1
#973653C7
 Az ifjúkor szép hajnalán, az élet reggelén
 Egy jóbarát int énfelém, és visszaintek én.
 Tárd szívedet a láng felé, és járjon át a tűz,
 Mely milliókat lelkesít és mindig összefűz.
/2
#8FA87D35
 Mint százezer kis fénysugár, úgy törj az éjen át,
 És énekeld e régi dalt, te kedves jóbarát!
 Tárd szívedet a láng felé, és járjon át a tűz,
 Mely milliókat lelkesít és mindig összefűz.

>Nyár-kánon
/magyar
#2F6E4727
 Künn a fákon újra szól a víg kakukkmadár,
 Napsugárban úszik minden, száll az illatár:
 Nyár van, nyár!
 Röpke lepke száll virágra, zümmög száz bogár,
 Lombos ágon csókot vált az ifjú gerlepár:
 Nyár van, nyár van, kakukk szól már a fák közt:
 „Kakukk, be szép a nyár!”
/angol
#D6BFEE41
 Summer is icumen in Lhude sing cuccu,
 Groweth sed and bloweth med, and springth the wdenu;
 Sing cuccu;
 Awe bletheth after lamb, lhouth after calwe cu;
 Bulluc stertheth, bucke vertheth, mu rie sing cuccu,
 Cuccu, cuccu, wel singes thu cuccu, Ne swik thu navernu.

>Tipperary
/magyar
#9F8E18D7
 Víg a kedvem ma Tipperary, nem riaszt el az út.
 Még elérünk ma Tipperary, hol egy szép lány lángra gyújt;
 Good by, Piccadily, Ködből festett táj;
 Bármi messze van még Tipperary, de szívünk ott már.
/angol
#184F59DD
 It's a long way to Tipperary, It's a long way to go.
 It's a long way to Tipperary, to the sweetest girl I know;
 Good by, Piccadily, Farewell Leicester square;
 It's a long, long way to Tipperary, but my heart's right there.

>Klementina balladája
/1
#E036B392
 Egy bányában, egy tárnában élt egy bányász egymaga,
 Vele lakott magányában leánya, Klementina.
/refr
#09807EF3
 Ó, a kedves, ó, a drága, jaj, de hamar elmúla,
 Elpusztultál, megfulladtál, jaj nekem, Klementina!
/2
#5BD22B5C
 Lenge volt ő, mint a szellő, a cipője oly parány;
 Konzervdoboz üres alja szolgált néki, mint szandál.
/refr
#09807EF4
 Ó, a kedves, ó, a drága, jaj, de hamar elmúla,
 Elpusztultál, megfulladtál, jaj nekem, Klementina!
/3
#A515832E
 Egy nap reggel, pont kilenckor kiment a tó partjára,
 Megcsúszott és beleesett, így lett szörnyű halála.
/refr
#09807EF5
 Ó, a kedves, ó, a drága, jaj, de hamar elmúla,
 Elpusztultál, megfulladtál, jaj nekem, Klementina!
/4
#1E3F163B
 Láttam én őt elmerülni s a habokkal küzdeni,
 Mivel úszni nem tanultam, nem tudtam kimenteni.
/refr
#09807EF6
 Ó, a kedves, ó, a drága, jaj, de hamar elmúla,
 Elpusztultál, megfulladtál, jaj nekem, Klementina!
/5
#B3F15C3C
 Édesapja, a jó bányász belehalt bánatába,
 S utána ment leányának százhúsz éves korába'.
/refr
#09807EF7
 Ó, a kedves, ó, a drága, jaj, de hamar elmúla,
 Elpusztultál, megfulladtál, jaj nekem, Klementina!
/6
#571C9E92
 Klementinát nem feledtem, de megláttam kis húgát,
 Egyszeriben boldog lettem, s feledtem Klementinát.
/refr
#09807EF8
 Ó, a kedves, ó, a drága, jaj, de hamar elmúla,
 Elpusztultál, megfulladtál, jaj nekem, Klementina!
/7
#FEBCF9D5
 Klementina ma is élne – hiszitek-e hát fiúk?
 Ha cserkészhez hasonlóan minden ember úszni tud.
/refr
#09807EF9
 Ó, a kedves, ó, a drága, jaj, de hamar elmúla,
 Elpusztultál, megfulladtál, jaj nekem, Klementina!

>Tréfás köszöntő
#852EDCCF
 Kösd fel magad-, kösd fel magad-, kösd fel magadra vitézi kardodat!
 Hogy meggebedj-, hogy meggebedj-, hogy meggebedjenek ellenségeid.
 Fejed borítsa kosz-, borítsa kosz-, fejed borítsa koszorúfüzér!
 Szakadjon rád az ég-, szakadjon rád az ég-, szakadjon rád az égnek minden áldása!

>Pigi-ja-baba
/1
#5416B2FC
 Pigi-ja-baba laszi-daka tóra.
 Laszi-daka tórababa, laszi-daka tóra.
/2
#AD238746
 Pici a baba, leszakadt az orra.
 Leszakadt az orra, baba, leszakadt az orra.
/3
#71694ECA
 Viszi a papa Lacikát a tóra.
 Lacikát a tóra, papa, Lacikát a tóra.

>Hazaszeretet
/1
#4793FDF1
 Napkeleti vizeken hömpölyög az ár,
 Nézd a hajnalt az egen, mennyi napsugár!
 Víztarajon fut a hal, ráragyog a fény,
 Nyolc szigeten süt a nap, táncol a remény.
 Ó hogyha virradásunk szép ködje száll,
 Égbe repül az ima, feltöri magát,
 Tűzhegyünk, a Fudzsima zengi a szavát.
 Száll a tűz, ég a láng, ne félj, szent hazánk!
/2
#D778F777
 Nippon arca hova néz, felfelé tekint;
 Mert a császár, a vitéz, hogyha egyet int...
 Tiszta szeme csupa fény, áldja meg az ég!
 Benne örök az erény: hősi ivadék.
 Ó fel, fel, hav a császár, menjünk tehát!
 Négy a tenger, egy a szív, rajta, figyelem!
 Harcba hát! Ha kardja hív, s kész a győzelem.
 Merre jár s kardja vág, nyit majd száz virág!
/3
#05E0F2D2
 Védte karunk a hazát száz viharon át,
 Diadalmat akarunk, élet igazát!
 Mert a mienk ez a föld s végtelen egünk,
 Őseinktől örökölt drága tengerünk!
 Ó mindig élt e föld és élt Istenünk...
 Napkeleti vizeken hömpölyög az ár,
 Nézd a hajnalt az egen, mennyi napsugár!
 Nippon él, Nippon él! Egy nép, egy a cél!

>Pálos-mindenszentek
/1
#5B397EAC
 Pálos-mindenszenteket köszönti hazánk,
 Életüket követi magyar ifjúság.
 Miérettünk éltek ők: pálos hivatás,
 Boldog Özséb nyomdokán fellobog a láng.
/refr
#24B00946
 Ó fel, fel, hív az Isten, számít Ő ránk!
 Él a lelke mibennünk, istenfiúság,
 Magyar népünk jövője, boldog hivatás.
 Isten él, Krisztus él! E nagy Cél miénk!
/2
#5522A996
 Égbetörő szent ösvény, rajta ifjúság!
 Az Igazat keresi, élet igazát.
 Mert a miénk a jövő, s minden emberi,
 S mi keresztből születik, örök krisztusi.
/refr
#24B00947
 Ó fel, fel, hív az Isten, számít Ő ránk!
 Él a lelke mibennünk, istenfiúság,
 Magyar népünk jövője, boldog hivatás.
 Isten él, Krisztus él! E nagy Cél miénk!

>Ökumenikus esti dal
/1
#4AC2A046
 Ó, terjeszd ki, Jézusom, oltalmazó szárnyad,
 És csitíts el szívemben bút, örömet, vágyat!
 Légy mindenem, légy fényem: sötéten jő az éj,
 Nagy irgalmadból élnem szüntelen Te segélj!
/2
#7AF5321F
 Ó, mosson meg az értem bőven hullt drága vér!
 Új lélekért könyörgök, újult akaratért.
 Kicsik, nagyok mind kérünk: őrködj vigyázva ránk!
 Békességedbe térünk, Te áldd meg éjszakánk!

>Invábú
/1
#918D1565
 Ingojáma, gonjáma, Invábú!
 Jábú! Jábú! Invábú!
/2
#1694F9C0
 Ő egy hős, egy oroszlán!
 Sőt ennél is több: víziló!
/3
#A2EE66C9
 Hős vad tigris, vad tigris, több annál:
 Király, király, víziló!

>Mondd el…
/refr
#EDF3F48C
 Mondd el a hegyek ormán, a boldog hírtől zengjen a vidék,
 mondd el a hegyek ormán, hogy földre szállt az ég.
/1
#FEA35D76
 Kis nyája mellett őrt áll a jámbor pásztornép,
 az égen, földön fényár, nem volt éj még ily szép.
/refr
#EDF3F48D
 Mondd el a hegyek ormán, a boldog hírtől zengjen a vidék,
 mondd el a hegyek ormán, hogy földre szállt az ég.
/2
#D103D2A8
 Száz angyal hirdet jó hírt, mind erre vártunk rég,
 ma Betlehemben megnyílt és mézet ont az ég.
/refr
#EDF3F48E
 Mondd el a hegyek ormán, a boldog hírtől zengjen a vidék,
 mondd el a hegyek ormán, hogy földre szállt az ég.

>Hallelu'
#C485156E
 Hallelu', hallelu', halleluja, áldott az Úr!
 Áldott az Úr, halleluja, áldott az Úr, halleluja!
 Áldott az Úr, halleluja, áldott az Úr!

>Ne félj, ne aggódj
/1
#14C145B3
 Ne félj, ne aggódj, ne sírj, ne bánkódj:
 ha tiéd Isten, tiéd már minden.
 Ne félj, ne aggódj, ne sírj, ne bánkódj:
 elég ő néked.
/2
#ABA202C1
 Nada te turbe, nada te'espante:
 quiena Dios tiene, nada le falta.
 Nada te turbe, nada te'espante:
 sólo Dios basta.

>Ne féljetek
/1
#1053F24B
 Ne féljetek, örüljetek!
 Krisztus győzelmesen föltámadt.
/2
#331CA6A1
 Niebójcie sie, radujcie sie!
 Chrystus rzecy wiscie zgrobuwstal.

>Gyújts…
/1
#57C1DAD9
 Gyújts éjszakánkba fényt,
 hadd égjen a soha ki nem alvó tűz, a ki nem alvó tűz.
 Gyújts éjszakánkba fényt,
 hadd égjen a soha ki nem alvó tűz, a ki nem alvó tűz.
/2
#51C3E620
 Dans nos obscurités,
 allume le feu qui ne s'eteint jamais qui ne s'eteint jamais.
 Dans nos obscurités,
 allume le feu qui ne s'eteint jamais qui ne s'eteint jamais.

>Bizakodjatok
/1
#23E35F7B
 Bizakodjatok, jó az Úr,
 jósága éltet.
 Bizakodjatok, jó az Úr.
 Alleluja!
/2
#4D173EED
 Confitemini Domino
 quoniam bonus.
 Confitemini Domino.
 Alleluja!

>Csak vándorlunk
/1
#9B181529
 Csak vándorlunk az éjben,
 mert forrás vizére vágyunk,
 szomjunk a fény a sötétben,
 szomjunk a fény a sötétben.
/2
#7C974150
 De nochéi remos, de noche,
 que paraen contrar la fuente,
 sólo la sed nos a lumbra,
 sólo la sed nos a lumbra.

>Jó az Úrban…
/1
#B9206DB5
 Jó az Úrban bizakodni, jó az Úr.
 Remélj és bízz benne, jó az Úr.
/2
#07EF60EF
 Bonum est confidere in Domino.
 Bonum sperare in Domino.

>Reggeli hála
/1
#6A68A944
 Hálát adok, hogy itt a reggel,
 hálát adok az új napon.
 Hálát adok, hogy minden percem
 Néked adhatom.
/2
#8D5F7292
 Hálát, nemcsak a jótestvérért,
 hálát mindenkiért adok.
 Hálát adok, hogy minden sértést
 megbocsáthatok.
/3
#1393C5EE
 Hálát adok, hogy munkát küldesz,
 hálát adok, hogy fény ragyog.
 Hálát adok, hogy lelkem fényes
 és boldog vagyok.
/4
#99B51A08
 Hálát adok a vidám percért,
 hálát, ha szomorú leszek.
 Hálát adok, hogy adsz elém célt,
 és kezed vezet.
/5
#285A3BA3
 Hálát adok, hogy hallom hangod,
 hálát adok a jó hírért.
 Hálát adok, hogy Tested adtad
 minden emberért.
/6
#6CEA75B0
 Hálát adok az üdvösségért,
 hálát adok, hogy várni fogsz.
 Hálát adok, hogy ma reggel még
 hálát adhatok.

>Esti hála
/1
#A3D4B7CC
 Hálát adok az esti órán,
 hálát adok, hogy itt az éj.
 Hálát adok, hogy szívem mélyén
 hála dala kél.
/2
#030E9727
 Hálát adok a csillagfényért,
 hálát a sűrű éjjelért.
 Hálát adok, hogy nem vagy messze,
 most is küldesz fényt.
/3
#F17B8BA1
 Hálát adok, hogy küldtél testvért,
 társként aki ma mellém állt.
 Hálát adok, hogy megengedted
 azt is, ami fájt.
/4
#BAA7A24B
 Hálát adok, hogy tested táplál,
 hálát, hogy ma is szólt szavad.
 Hálát adok, hogy lelkem biztos
 cél felé halad.
/5
#B91D5A78
 Hálát adok, hogy mellém álltál,
 hálát, hogy szép az otthonom.
 Hálát, hogy munkám elvégeztem
 s békén alhatom.
/6
#2AF2EBD0
 Hálát adok a sok-sok jóért,
 hálát adok, hogy megbocsátsz.
 Hálát adok, hogy el nem hagytál,
 újabb útra vársz.
/7
#438C6600
 Hálát adok, hogy szemed rajtam,
 hálát adok, hogy reményt adsz.
 Hálát adok, mert biztos eljössz,
 mint a virradat.
